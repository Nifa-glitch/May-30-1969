\setvariables[article][shortauthor={Prater}, date={December 2020}, issue={5}, DOI={https://doi.org/10.7916/archipelagos-m2fg-gv81}]

\setupinteraction[title={Always Together: A Digital Diasporic Elegy},author={Tzarina T. Prater}, date={December 2020}, subtitle={Always Together}, state=start, color=black, style=\tf]
\environment env_journal


\starttext


\startchapter[title={Always Together: A Digital Diasporic Elegy}
, marking={Always Together}
, bookmark={Always Together: A Digital Diasporic Elegy}]


\startlines
{\bf
Tzarina T. Prater
}
\stoplines


{\startnarrower\it This essay takes up the question of how diasporic Caribbean subjects deal with death, dying, and grieving in online spaces. With a focus on the genre of {\em digital diasporic elegy}, the author argues that by analyzing the digital life of a Sino-Caribbean diasporic cultural text, a digital diasporic elegiac practice emerges. This practice allows for consideration of the very code that enables our traversal between digital nodes, thereby allowing for a more nuanced understanding of rituals of mourning in our contemporary moment.

 \stopnarrower}

\blank[2*line]
\blackrule[width=\textwidth,height=.01pt]
\blank[2*line]

{\em For Cheryl Wall (1948-2020)}

\thinrule

The night of 19 June 2020 marked the conclusion of 40 Nights of the Voice, a digital wake initiated by Ronald Cummings, Nalini Mohabir, and Kaie Kellough to commemorate the life and works of historian, poet, critic, and theorist Edward Kamau Brathwaite, who had transitioned on 4 February 2020. The wake was launched on 11 May, on what would have been Brathwaite's ninetieth birthday, and continued for the next forty nights. In the inaugural upload, Kellough informed viewers that participants in the wake---creative writers, like himself, and scholars---would be reading from Brathwaite's oeuvre, with a different upload announced each night on Twitter @Kamauremix, along with links to full videos on YouTube at the Kamau Brathwaite Remix Engine account. The hashtags \#KamauBrathwaite and \#40NightsofTheVoice were used to aggregate and connect posts. The project brought together contributors from the Caribbean, Canada, the United Kingdom, Africa, and the United States in what Kelly Baker Josephs, building on Brathwaite's work, has theorized as the \quotation{digital yard.}\footnote{See Kelly Baker Josephs, \quotation{Digital Yards: Caribbean Narrative on Social Media and Other Digital Platforms,} in {\em Caribbean Literature in Transition}, vol.~3, {\em 1970--2020}, ed. Ronald Cummings and Alison Donnell (Cambridge: Cambridge University Press, forthcoming), 219.} All the participants were united by the fact that they had been \quotation{shaped and guided by Kamau's thinking, his writing and his presence,}\footnote{Kaie Kellough (video), @KamauRemix, \quotation{Launching 40 Nights of the Voice. A celebration of Kamau's works,} Twitter, 11 May 2020, \useURL[url1][https://twitter.com/kamauremix/status/1259864754778226690]\from[url1].} and they produced what I am calling an explicitly Caribbean practice of \quotation{digital diasporic elegy.}

This essay is divided into three sections. The first defines and unpacks my key term, {\em digital diasporic elegy}, and establishes my focus on texts generated in, by, and about Caribbean diasporic subjects. The second drops the reader into a moment---more specifically, into a moment in which I trace the digital crumbs of what I read as a digital elegy involving a Sino-Caribbean diasporic cultural text. The final section returns to the multiplatform project, 40 Nights of the Voice, a project readily recognizable as digital, diasporic, and Caribbean, in the hope that the meditation on digital elegiac practice will enable a more nuanced understanding of the expressions of mourning and grief and of the possibilities of consolation they contain.

\subsection[title={Defining the Term: Digital Diasporic Elegy},reference={defining-the-term-digital-diasporic-elegy}]

\placefigure[here]{The Zea Mexican Diary}{\externalfigure[issue05/prater/fig1.jpg]}


With roots in classical literature, the term {\em elegy} denotes \quotation{a short poem, usually formal or ceremonious in tone and diction, occasioned by the death of {[}a{]} person.}\footnote{{\em The New Princeton Encyclopedia of Poetry and Poetics}, 3rd ed. (Princeton, NJ: Princeton University Press, 1993), s.v.~\quotation{elegy.}} Elegiac texts can be textual or photographic; can take the form of plastic arts, monuments, or memorials; and can be literal and symbolic. They are at once public and intensely private, collective and personal, and they are often tied to the operations of nationalism. They are performative, at times signaling narcissistic, self-interested, or at the very least self-regarding plays for immortality. The elegy is a genre of affective expression that pushes at the limits of language, and strains and exposes its competencies and inefficiencies.\footnote{See Diana Fuss, {\em Dying Modern: A Meditation on Elegy} (Durham, NC: Duke University Press, 2013), in which Fuss explicitly poses the rhetorical questions, \quotation{Why are poets repeatedly drawn to the precise moment beyond which language is no longer possible? Pushing voice to its furthest limit, what exactly do poets hope to learn by imagining, and reimagining, the dying hour?} (9).} In the strictest sense, elegies are texts that memorialize the dead, frame loss for the living, and, according to Karen Weisman in her introduction to {\em The Oxford Handbook of the Elegy}, exist \quotation{between the extremes of life and death, joy and sorrow, the receding past and the swiftly moving present.}\footnote{Karen Weisman, introduction to {\em The Oxford Handbook of the Elegy}, ed.~Karen Weisman (Oxford: Oxford University Press, 2010), 1.} Most important for my analysis, the elegiac genre is defined by the exigencies and technologies of the historical moment in which the elegy is produced. I am interested in what constitutes the elegy in our contemporary moment---a moment wherein a global pandemic has pushed so many of us to live much of our lives online, a time also of upheaval and social unrest, in which the daily news, filled with obscene images of black death, comes to us via our digital news feeds and timelines. How do we deal with death, dying, mourning, loss, and grief, both personal and collective, as we are subject to a palpable fatigue resulting from the \quotation{always now-ness} of our \quotation{swiftly moving present}?

Literary theorist Jahan Ramazani claims that the modern elegy is interested in not so much healing, transcendence, or redemption but rather immersion. It is the immersive quality of social media that is relevant for my analysis.\footnote{See Jahan Ramazani, {\em Poetry of Mourning: The Modern Elegy from Hardy to Heaney} (Chicago: University of Chicago Press, 1994), 4.} According to Ramazani, there are two \quotation{distinct elements that fuse in the imagery of national elegies,} notably, \quotation{the collective mourning that is often fundamental in the formation of group identity} and the \quotation{recursive or echoic quality of poetic language.}\footnote{Jahan Ramazani, \quotation{Nationalism, Transnationalism, and the Poetry of Mourning,} in Weisman, {\em Oxford Handbook of the Elegy}, 605.} It is this pattern, repetition, consonance, and assonance---the beats and rhythms of the poetic---that creates interpenetrative effects among participants, community members, creators, and readers/audiences. Nadia Ellis, however, contests Ramazani's claim that modern elegies do not \quotation{heal the living.}\footnote{Ramazani, {\em Poetry of Mourning}, 4.} In her \quotation{Elegies of Diaspora,} Ellis links loss and pain, two threads inextricably interwoven into processes of diaspora and migration, to elegiac texts produced in diaspora, arguing that diasporic elegies explore the failures and \quotation{duplicities of the neoliberal state} and the pain and grief necessarily attendant to citizenship.\footnote{Nadia Ellis, \quotation{Elegies of Diaspora,} {\em Small Axe}, no. 43 (March 2014): 169.}

I use the formulation \quotation{digital diasporic elegy} to describe a set of contemporary texts, practices, and forms of participation that take up death, dying, loss, mourning, and lamentation as their subject matter and circulate among diasporic subjects on social media platforms and other digital spaces. I am particularly interested in how Caribbean diasporic subjects struggle with grief and feelings of dislocation as they navigate violently nationalistic terrains. Social media platforms are nodal points where subjects in diasporic formations connect and commune. During this time of global pandemic, people's lives have been constrained to domestic spaces; to venture \quotation{outside} has, for many, been to go online. Social media engulf users in ever-present grievances. Databases like YouTube and Instagram use an algorithmic code to bring \quotation{recommended} and \quotation{of interest} posts into individual feeds. Producing a constant stream of imagery, these platforms create a perpetual, painful present. The mediated present of the news feed is undeniably relentless. It is shaped by user's likes, follows, searches, and susceptibility to \quotation{click bait,} all of which comes together to create the contours of individual digital personae and the communities to which they belong. Kelly Baker Josephs offers insightful critical reflections on a host of online creative activities in which Caribbean writers engage: tweeting and retweeting, posting and reposting, and blogging and microblogging on social media sites like Facebook, YouTube, and Twitter.\footnote{See Josephs, \quotation{Digital Yard} 219; see also her \quotation{Me, Myself, and Unno: Writing the Queer Caribbean Self into Digital Community} in this issue of {\em archipelagos journal}, \hyphenatedurl{http://archipelagosjournal.org/issue05/josephs-blogging.html.}} Leaning on Josephs's work, I analyze YouTube as a digital yard in which Caribbean subjects engage in and create digital diasporic elegiac practices and texts.

Unlike brick-and-mortar sites that have historically had an elegiac function as material anchors for nationalistic fantasies, social media sites like Twitter, Instagram, YouTube, and Facebook can memorialize and destabilize the privileged relationship between \quotation{history} and temporality. Whereas museums and archives create and preserve a version of the past, social media sites create living and evolving archives of the now. Further, they allow users to function and envision themselves as walking archives, impulsively curating often carefully edited and pixelated, uploadable versions of themselves and their lives.

Social media timelines invoke \quotation{linearity} but are not always faithful to it. A destabilization of temporality is possible on these platforms insofar as current and past events can readily merge. While much is made of the brevity of the \quotation{news cycle,} timelines offer challenges to cultural and social memory by asynchronously interpolation images and videos of the immediate or recent past alongside images and events from long ago. As members of particular social circles discover and share cultural texts anew, the \quotation{grievable} can seem to extend out of the digital ether to take hold of and traumatize \quotation{in real time,} so to speak. Most recently, the names of black victims of police violence have been regularly \quotation{popping into} timelines from multiple news media sources and personal connections. Circulating along with those names are visual components, elegiac images that can, according to Josh Ellenbogen, refuse to do the work of conventional elegy. They risk \quotation{{[}taking{]} us out of a narrative process of grief,} bringing pain into the present and inhibiting the \quotation{\quote{normal} work of mourning,} which is to move emotionally toward consolation.\footnote{Josh Ellenbogen, \quotation{On Photographic Elegy,} in Weisman, {\em Oxford Handbook of the Elegy}, 676. Also helpful when thinking about the temporality of the photographic image and its relationship to mourning and loss is Susette Min, \quotation{Remains to Be Seen: Reading the Works of Dean Sameshima and Khanh Vo,} in {\em Loss: The Politics of Mourning}, ed.~David L. Eng and David Kazanjian (Berkeley: University of California Press, 2003), 229--50. \quotation{Photography's privileged relation to melancholia,} Min claims, \quotation{is in part one of temporality\ldots{}. The freezing of time creates a dimension in which the future perfect of the photographic image---this will-have-been---may be suspended, manipulated, and reworked to become the past perfected} (241). And, I would add, the present.} Deep ontological wounds are renewed and reopened as the historical past bleeds into the present through racialized articulations of power in images of torture, pain, and death. With no clear beginning, middle, or end, the constant accessibility of these images and narratives---that is, their \quotation{reach}---can, according to Elizabeth Helsinger, prolong reflection and extend the duration of grieving in ways that literary texts, because of specific work done by visual texts, do not.\footnote{Elizabeth Helsinger, \quotation{Grieving Images: Elegy and the Visual Arts,} in Weisman, {\em Oxford Handbook of the Elegy}, 662.} The proliferation and dominance of American popular culture, through global conduits of capital, have done much to establish a shared conceptual and cultural map, so much so that obscene images of black death, colored by nationalistic violence and its intimacies, have provided an arguably global vernacular.

Before 2020, the notion of a virtual wake or funeral was not something many had considered outside online virtual and gaming communities. Few had heard of Cemetery.org, the oldest online cemetery and memorial site, but many had participated in some form of online memorialization, whether for a loved one, acquaintance, pet, respected colleague, hero, or even adversary.\footnote{Cemetery.org, the \quotation{World Wide Cemetery,} was created by Canadian engineer and internet pioneer Mike Kibbee two years before his own death from Hodgkin's Lymphoma in 1995. For a one-time fee of \$90, a \quotation{personally customized, permanent memorial} can be linked to relatives, merged with partners, and linked via a QR-Code to a physical site of interment (\quotation{How It Works,} \useURL[url2][https://cemetery.org/create-a-memorial/]\from[url2]). One can leave (vetted) messages and photos and even virtual flowers (from a limited selection). While I do not preclude Cemetery.org as a possible digital node in digital diasporic elegiac practice, this essay makes clear the differences between my foci and the elegiac work in which the site engages. Cemetery.org formalizes grieving in ways that are analogous to IRL (in real life) practices, from its business model to its execution. To put it bluntly, it is not \quotation{alive} but functions more like a digital ossuary. At the time of this writing, it is not a social media site with millions of active daily users and high degrees of interactivity, nor has it been accessed by or included memorials created for Caribbean subjects. It is not part of the \quotation{digital yard.}} My own Facebook timeline contains shared, coauthored, and self-authored posts dedicated to Michelle Cliff, V. S. Naipaul, Toni Morrison, Derek Walcott, Amiri Baraka, Nadine Gordimer, Gabriel Garcia Marquez, Winston Chung Fah, Cheryl A. Wall, Kamau Brathwaite, and, of course, Prince, among several other literary and cultural figures. These posts are a part of larger processes of collective grieving. They become digital nodal points that link (to) others and work toward the traditional goals of elegiac texts, that of healing and consolation. In one sense, these texts can be viewed as twenty-first-century Ozymandius-like digital monuments erected to appease our age-old preoccupation with railing against impermanence.\footnote{\quotation{Ozymandias,} a sonnet written in 1817 by Percy Bysshe Shelley (1792--1822), condemns the hubris of those who build monuments in a bid for immortality only to see them ravaged by time. It is believed Shelley took his inspiration from a statue of Rameses II, and the poem's power lies in the image of a statue in ruins bearing the words, \quotation{Look on my Works, ye Mighty, and despair!} Percy Bysshe Shelley, \quotation{Ozymandias,} {\em Examiner}, 11 January 1818 (\useURL[url3][https://www.poetryfoundation.org/poems/46565/Ozymandias]\from[url3]).} Yet they are also lamentations connecting diasporic subjects, and diasporic digital elegy does more than memorialize. It acknowledges the fluidity of technologies that make these utterances possible in ways that are linked to the technologies that made diaspora possible.

\subsubsection[title={\quotation{Always Together}: Following the Digital Crumbs down the Rabbit Hole},reference={always-together-following-the-digital-crumbs-down-the-rabbit-hole}]

My fall down this particular rabbit hole began around the anniversary of my mother's death, which occurred exactly one year before two planes hit the twin towers in New York City and arguably changed the literal, national, and political landscape of the United States. With each anniversary of this national day of mourning, I am brought back to my own personal loss. In the process of conducting research for a project on diasporic Sino-Caribbean subjects and their participation on social media platforms, I \quotation{happened upon} an upload of \quotation{Always Together,} a rocksteady reggae song performed by Byron Lee and the Dragonaires and featuring vocals by late Taiwanese singer Stephen Cheng. I use the phrase \quotation{happened upon} somewhat facetiously, knowing that what drops into my \quotation{recommended for you} feed has been shaped by hundreds of prior decisions I have made: by my \quotation{likes,} visits, and \quotation{cookies,} all the subterranean code that contours the digital worlds I visit and inhabit. Encountering this song in my own time of remembrance became an occasion to think through what constitutes the digital diasporic elegy and how a participant's engagement with a concatenation of visual and aural signifiers constitutes a digital diasporic elegiac practice.

In the comments section of the upload, I noted an appeal for information regarding the track posted by an individual with the username Pascal Cheng. In his post, Cheng claims to be the son of Stephen Cheng and mentions that he had never known about the track or about his father's experience in Jamaica, nor does he think his father ever knew of the track's classic status.\footnote{See the comments under \quotation{Stephen Cheng---Always Together,} posted by italrel, 17 December 2010, \useURL[url4][https://www.youtube.com/watch?v=R30SYupoQo8]\from[url4].}

\placefigure[here]{Pascal Cheng's Search}{\externalfigure[issue05/prater/fig2.png]}


I had dropped, {\em en media res}, onto one of many nodal points Pascal Cheng created in his search for information about a part of his father's life that had been unknown to him. In following the younger Cheng's digital crumbs, I became witness not just to his interaction with other users but to an instance of digital elegiac practice. Pascal Cheng's search, with the aid of other YouTube users, took him to official news media resources like the Jamaica {\em Gleaner}, as well as to music aficionado, crate digger, and commercial sites such as Dub Store Records.\footnote{According to Crate Digging Cooperative Records, the term {\em crate digger} originates in hip-hop culture, specifically with the DJ practice of utilizing literal milk crates to transport and store vinyl records. \quotation{Crate digging} is the active, physical, often voracious scouring of record stores in search for vinyl records \quotation{to sample.} More recently, the term has loosened to include internet searches, digital downloads, video watching, and any activity that allows you to engage in a \quotation{self discovery of music.} rchecka, \quotation{Crate Digging (Re)Defined,} Crate Digging Cooperative, \useURL[url5][http://cratedigging.co/cratedigging.htm]\from[url5].} Users also directed Pascal Cheng to other media and cultural texts, including {\em The Man with the Iron Fist}, a 2012 film starring and produced by RZA, legendary member and producer of the hip-hop music group Wu Tang Clan; RZA had included Frances Yip's popular 1974 version, \quotation{Green Is the Mountain,} on the film's soundtrack. Pascal Cheng was also directed to Jeannette Kong's {\em Finding Samuel Lowe: From Harlem to China}, a 2014 documentary film about Paula Williams Madison, a woman of black and Chinese Jamaican descent engaged in an analogous search for her Chinese grandfather and family.

Rocksteady, the precursor to contemporary reggae and dancehall, with origins in mento, rhythm and blues, jazz, and ska, had been taken up in the mid-1960s by the first wave of British skinheads, a youth subculture that emerged from working-class enclaves of London and whose members rejected both political conservatism and the bourgeois cathexis to American-identified \quotation{hippie}/pop culture. Several YouTube uploads of \quotation{Always Together} use cover art from Skinheads on the Dancefloor albums (fig.~3) or use photographic stills of skinheads in England.\footnote{See, for example, uploads of Stephen Cheng's \quotation{Always Together} at \useURL[url6][https://www.youtube.com/watch?v=PF6vPx8SSmA]\from[url6] and \useURL[url7][https://www.youtube.com/watch?v=KXr04RVuNl4]\from[url7].}

\placefigure[here]{Skinheads on the Dancefloor}{\externalfigure[issue05/prater/fig3.jpg]}


In terms of composition, several users have uploaded the track with static pictures of either a 1967 Sunshine Records or 2018 Federal Records 45rpm, a symbolic gesture toward the analog (figs. 4 and 5). A growing number have used cover art from the album {\em BMN Ska and Rocksteady: Always Together, 1964--1968}, a collection released in May 2019 by Dub Store Records, a Japanese record company out of Tokyo (fig. 6).

\placefigure[here]{Sunshine Records 45rmp}{\externalfigure[issue05/prater/fig4.jpg]}


\placefigure[here]{Images of 45rpm records sometimes used over the track of Stephen Cheng's "Always Together" in YouTube uploads.}{\externalfigure[issue05/prater/fig5.jpg]}


\placefigure[here]{BMN Collection}{\externalfigure[issue05/prater/fig6.jpg]}


The BMN collection cover art is a cluster of signifiers. It depicts a young black man in what looks like a Panama hat and wearing dark sunglasses, with the shadow of the photographer discernable in the lenses. He is the embodiment of \quotation{cool.} Added to this visual signifier of \quotation{authenticity} is the inclusion of Ronnie Nasralla, named on the cover as producer. Nasralla, a Jamaican-born record producer of Lebanese and Jamaican descent, had gone to school with Byron Lee and was part of an early incarnation of the Dragonaires. The collection had been recorded in Kingston and distributed by BMN out of Tokyo, Japan.

The upload I had happened upon was not the first under which Pascal Cheng had made his appeal. By the time he posted his request in the comments under the video I found, it was clear from his exchange with a user with the name 1970Rudeboy that he had acquired and was responding with some knowledge---a fuller understanding of the multiple national and music histories of the original Taiwanese track and the one produced in Jamaica.

\placefigure[here]{\goto{Exceprt from an exchange between Pascal Cheng and 1970Rudeboy (among others) in the comments below an upload of Stephen Cheng's "Always Together" on YouTube. (https://www.youtube.com/watch?v=R30SYupoQo8)}[url(https://www.youtube.com/watch?v=R30SYupoQo8)]}{\externalfigure[issue05/prater/fig7.png]}


Pascal Cheng's process of following the digital breadcrumbs is a dramatic instance of a digital elegiac practice. His public search for traces of his father can be followed across dozens of YouTube uploads, in their comments sections and through the hyperlinks users shared to sites for music aficionados and crate diggers, as well as to Stephen Cheng's 2012 {\em New York Times} obituary. Anyone who chooses to can follow Pascal Cheng's digital footprints and fall into these various narratives, one of which provides an understanding of the historical, cultural, and political contexts in which Stephen Cheng and Byron Lee created a remarkable recording that brings together multiple national threads and histories. For example, on 27 June 2019, YouTube user Mike Aylward entered the conversation thread begun by Pascal Cheng and 1970RudeBoy to inform Cheng of the existence of Krish Raghav's longform comic \quotation{Redemption Songs,} a graphic historical essay published online in {\em Topic Magazine}.\footnote{See Krish Raghav, \quotation{Redemption Songs,} {\em Topic Magazine}, no. 24, June 2019, \useURL[url8][https://www.topic.com/redemption-songs]\from[url8].}

\placefigure[here]{Redemption Songs}{\externalfigure[issue05/prater/fig8.png]}


\quotation{Redemption Songs} begins with the image of an animated record bearing the title \quotation{Beijing Babylon,} and on the record player arm appears \quotation{Reggae's 60-year Journey to China} (fig.~8). The essay-comic then informs the reader of Stephen Cheng and Byron Lee's meeting and recording of \quotation{Always Together} and provides a Soundcloud audio file link. The rest of the essay-comic provides a brief history of the Chinese in Jamaica and explores the little-known role of Chinese Jamaicans in Jamaican music production. It goes on to inform the reader about the dissemination of reggae music in Mainland China through {\em dakou} culture, the black-market sale of surplus unsold Western music stock. Reggae found an audience in Mainland China through its appeal to ethnic minorities chafing under oppression and those invested in political resistance.\footnote{For more, see Marvin Sterling, {\em Babylon East: Performing Dancehall, Roots Reggae, and Rastafari in Japan} (Durham, NC: Duke University Press, 2010); Claire Huot, {\em China's New Cultural Scene: A Handbook of Changes} (Durham, NC: Duke University Press, 2000); Beth E. Notar, \quotation{Off and On the Road to Reform,} chap.~6 in {\em Displacing Desire: Travel and Popular Culture in China} (Honolulu: University of Hawai'i Press, 2006); and Ying Xiao, \quotation{\quote{Hip Hop Is My Knife, Rap Is My Sword}: Hip Hop Network and the Changing Landscape of Image and Sound Making,} chap.~6 in {\em China in the Mix: Cinema, Sound, and Popular Culture in the Age of Globalization} (Jackson: University Press of Mississippi, 2017).} Raghav's three-part digital essay-comic ends with an animated image of Lee and Cheng and these words of lamentation: \quotation{Starting from Cheng and Byron Lee in 1967\ldots{} / A yearning. / A moment of freedom. / Of Zion glimpsed.} Within twenty-four hours of Aylward's YouTube post, Pascal Cheng confirmed having read the essay-comic.

One way to read this graphic historical essay is as a reclamation, a text that addresses the collective trauma created when the Hakka were forced from their homeland over 1650 years ago. They were made to endure four intercontinental migrations before being pushed, in a fifth major migration, into the Pacific, the Americas, and the Atlantic in search of economic prosperity, political and religious asylum, and freedom.\footnote{Historically, the terms {\em Hakka} and {\em Hakga}, which bring together {\em hak}, meaning \quotation{guest,} and {\em ga} or {\em ka}, meaning \quotation{family,} have been used interchangeably. For my purposes here, I use {\em Hakka} throughout.} \quotation{Redemption Songs} points to a long labor history linking African, Caribbean, and Chinese diasporas to networks of (neo)colonial capital, and it suggests, visually and aurally, collaboration between descendants of diasporized peoples. Narratively speaking, the essay-comic is more than a simple digression. I include it to mirror the meandering process of \quotation{falling down the rabbit hole.} \quotation{Redemption Songs} is a nodal point, a digital crumb that, in this agglutinative process, provides a more nuanced understanding of the complex historical forces colluding to produce the song \quotation{Always Together} and its makers.

In 1967 Stephen Cheng, a singer known for mixing Western and Eastern styles (Cheng was born and raised in China and educated in the United States), visited Jamaica to perform traditional Chinese folk music.\footnote{See the 2012 {\em New York Times} obituary for Stephen Chun-Tao Cheng, \useURL[url9][https://www.legacy.com/obituaries/nytimes/obituary.aspx?n=stephen-chun-tao-cheng&pid=159981536]\from[url9].} During his short stay, he met Byron Lee, of black and Chinese Jamaican descent, and they recorded the now classic track with Lee's backing band, the Dragonaires, at Lee's Dynamic Sounds Recording Studio.\footnote{For more on the history of Chinese Jamaicans and Jamaican cultural and music production, see Kevin O'Brien Chang and Wayne Chen, {\em Reggae Routes: The Story of Jamaican Music} (Philadelphia: Temple University Press, 1998); Lloyd Bradley, {\em This Is Reggae Music: The Story of Jamaica's Music} (New York: Grove, 2000); the invaluable Timothy Chin, \quotation{Notes on Reggae Music, Diaspora Aesthetics, and Chinese Jamaican Transmigrancy: The Case of VP Records,} {\em Social and Economic Studies} 55, nos. 1--2 (2006): 93--114; Jeff Chang, {\em Can't Stop Won't Stop: A History of the Hip-Hop Generation} (New York: Picador, 2005); and Alexandra Chang, {\em Circles and Circuits: Chinese Caribbean Art} (Durham, NC: Duke University Press, 2018). See also Yanique Hume and Aaron Kamugisha, eds., {\em Caribbean Popular Culture: Power, Politics, and Performance} (Kingston: Ian Randle, 2016), for larger contexts of the aesthetic contributions of the Chinese throughout the Caribbean.} This meeting and recording took place five years after the 1962 declaration of Jamaica's independence and three years after Edward Seaga, in his capacity as the head of Social Welfare and Economic Development, had handpicked Lee and the Dragonaires to back the black performers Jimmy Cliff, Millie Smalls, and Prince Buster as they represented Jamaica and Jamaican culture at the 1964 New York World's Fair.\footnote{Edward Seaga (28 May 1930--28 May 2019), born in Boston to Philip George Seaga of Lebanese Jamaican descent and Erna (née Maxwell) of Jamaican African, Scottish, and Indian descent, was the prime minister of Jamaica from 1980 to 1989 and the leader of the Jamaican Labour Party from 1974 to 2005. He was educated in the social sciences, and his academic interests led him to research and collect the history of reggae music. He founded West Indies Records Limited, which he sold to Byron Lee in 1962, after being elected to Parliament.} It was just two years after the upheavals of 1965, first when anti-Chinese riots in Jamaica created a domestic disturbance that escalated to a week of rioting, violence, and destruction of Chinese property, and then when the Jamaican government declared a state of emergency in West Kingston because of political unrest in the form of proxy battles between the Jamaican Labour Party and the People's National Party. It is against this backdrop of larger debates about citizenship and national belonging occurring in Jamaica in the 1960s that the composition and recording of \quotation{Always Together} should be understood.

The recording, \quotation{curiously s{[}u{]}ng in Chinese,} brought together two representatives of Chinese identity with historically vexed relationships to dominant constructions of Chinese national identity.\footnote{Item description for Stephen Cheng's \quotation{Always Together,} ReggaeRecord.com, \useURL[url10][https://www.reggaerecord.com/en/catalog/description.php?code=101779]\from[url10].} Stephen Cheng, born to a wealthy family in Shanghai in 1923, had been part of an exodus of people who emigrated to the United States in 1948, shortly before the Nationalist KMT government led by Chiang Kai-Shek officially came to power. Byron Lee, born in 1935 in Jamaica, was of black and Hakka Chinese Jamaican descent. The name {\em Hakka}, which literally means \quotation{guest people,} had been assigned to a subset of the Han people when genealogical data was being codified for imperial purposes fifteen hundred years prior to their migration to the Caribbean basin in the mid-1800s as indentured workers brought in to replace recently emancipated Afro-descended laborers. The Hakka are a fundamentally diasporic people with a history of outsiderness, political rebellion, and resistance. They were the driving force of one of the bloodiest uprisings in recorded history, the Taiping Rebellion (1850--64), which sought to end dynastic rule, institute land and education reform, and topple the Manchurian Qing dynasty.\footnote{Hong Xiuquan, leader of the Taiping Rebellion, was of Hakka descent. This war between the Taipings and the Manchu-led Qing Dynasty, which sought to dismantle the dynastic social and political order, resulted in the deaths of 20--70 million and the displacement of millions more. The uprising sought to expel foreign political and economic interests, suppress private trade, institute socialized land reform, ensure that education included women, and stop foot binding. It is seen as the ideological forerunner to the Chinese civil war.} People of Hakka descent were also at the ideological core of the Chinese Revolution: Chiang Kai-Shek, director-general of the Kuomintang and ruler of Taiwan as the president of the Republic of China, and Trinidadian-born Eugene Acham Chen were architects of the Chinese Revolution. They were pupils of Dr.~Sun Yat Sen, along with Mao Zedong, who became the chairman of the Communist Party of China and leader of the Peoples Republic of China.

When Lee and his band were chosen by Seaga to perform at the World's Fair, there was public outcry. Lee and the Dragonaires were considered too \quotation{uptown} and not Jamaican enough (i.e., not black enough) to represent the people.\footnote{The terms {\em uptown} and {\em downtown} refer to both geographical regions of Kingston as well as class markers, with {\em uptown} signifying upper or bourgeoisie socioeconomic class status and the inverse for {\em downtown}.} At the same time, the choice of Lee and his racially mixed band signified both Jamaica's national motto, \quotation{Out of Many, One People,} and the long history of Sino-Caribbean participation in cultural production. Sino-Caribbeans, because of their particular diasporic history, have participated in Jamaican cultural production as musicians, producers, sound engineers, and record shop and studio owners. The history of the Chinese in Jamaica and their relationship to music production illustrates the historical, political, and economic forces that made \quotation{Always Together} possible.

\quotation{Always Together} is supposedly adapted from a Taiwanese folk song that emerged from an indigenous ethnic minority in Taiwan, and it was speculated to have originally been sung, at least partially, in the Alit dialect. But this \quotation{truth} of the song's composition is tied to fantasies of indigeneity and appropriation. The song was actually composed by film director Chang Cheh (Zhang Che) for his 1949 {\em Storm Cloud over Alishan / Alishan Fengyun}, the first Mandarin-language film produced in Taiwan.\footnote{See \quotation{Chang, Che (Zhang Che) (1923--2002),} in Rachel Braaten and Lisa Odham Stokes, {\em Historical Dictionary of Hong Kong Cinema}, 2nd ed.~(Lanham, MD: Rowman and Littlefield, 2020), 61.} Originally titled \quotation{Green Is the High Mountain} or \quotation{Girl from Alishan,} it later became a folk standard, appropriated by the very same ethnic group to whom the song was originally attributed. The lyrics of the song incorporate familiar conventional signifiers:

\startblockquote
Green is the high mountain Blue is the mountain stream The girl of Ali Mountain is pretty as the water The youth of Ali Mountain is strong as the mountain The mountain forever green, the water forever blue The girl and that youth will never be parted---like the blue water flows around the green mountain\footnote{The reference to Ali Mountain in the song is derived from {\em alit}, meaning \quotation{ancestor.} This translation of \quotation{Green Is the Mountain} is from \useURL[url11][https://www.musixmatch.com/lyrics/Frances-Yip-3/Green-Is-the-Mountain]\from[url11]. Cheng's Mandarin lyrics can be found at \useURL[url12][https://www.letras.com/stephen-cheng/always-together-ali-shan-de-gui-niang/]\from[url12].}
\stopblockquote

The lyrics deploy traditional romantic metaphors of virile young men and fecund young women so familiar they do not bear comment, but in the version produced by Lee and Cheng, these lyrics are sung over a rocksteady beat. In terms of soundscape, Cheng's \quotation{plaintive warbling} laid over this rocksteady beat creates \quotation{something entirely different: one of the most unlikely fusions ever committed to wax.} According to James Baron, \quotation{There is no way it should work, but somehow it does, and quite brilliantly.}\footnote{James Baron, \quotation{The Tao of Stephen Cheng,} {\em Taipei Times}, 28 November 2019, \useURL[url13][https://www.taipeitimes.com/News/feat/archives/2019/11/28/2003726596]\from[url13].} Cheng does not sing in the Hakka dialect, which would have been the dialect of the Chinese ethnic group that made up more than 80 percent of the Chinese immigrants to Jamaica, nor in the Alit dialect of the Tsou people, the indigenous Taiwanese group of the song's supposed origin. Instead, he sings in Mandarin, the dialect that would be understood in Taiwan and Beijing, the seats of opposing powers of the Revolution. The composition appropriates and romanticizes indigeneity and highlights ethnic cultural specificity at a time when violent debates were being waged by the opponents in the Chinese civil war (1927--49) about what it meant to be Chinese.\footnote{Benedict Anderson discusses the unifying qualities of anthems, claiming that however \quotation{banal the words and mediocre the tunes,} the collective experience of singing with others unifies disparate members of a community, enabling them to imagine themselves as a people. Benedict Anderson, {\em Imagined Communities} (London: Verso, 2016), 145.}

The colonial, imperial, and national forces that created Byron Lee and the Dragonaires and Stephen Cheng are not the same. Lee was part of an \quotation{always already} diasporic people able to move into microbusinesses as the result of generations-old established conduits of skilled trades and industry. Cheng, whose family \quotation{lost everything} in the Chinese civil war, immigrated to the United States, was educated at Julliard and Columbia University, and eventually moved to Canada with his French Canadian wife. Both Lee and Cheng were known for blending musical genres, and both performers represent diasporic groups who struggled with national belonging and historical (in)visibility. The song's title ironically evokes a coherent Chinese identity that is destabilized by the fact of diaspora. In doing so, \quotation{Always Together,} enacts griefs attendant to the diasporic subject's struggle with inherently violent nationalisms and that subject's vexed relationships to citizenship. It also speaks to the difficulties of belonging consistent with Nadia Ellis's formulation of the diasporic elegy's ability to articulate the \quotation{simultaneous apprehension of long-past and all-too-fresh violence.}\footnote{Ellis, \quotation{Elegies of Diaspora,} 170.} In the case of Lee, it is his not being Jamaican enough to fully belong to and represent the nation. For Cheng, it is the loss of home during a time of civil war, the subsequent remove to America, and feelings of dislocation compounded by his attending two of the arguably top institutions in America for music and education but still being relegated to \quotation{ethnic performances} because of North American racism and stereotypes.\footnote{In \quotation{Stephen Cheng Released One Single That Sounded Like Nothing Else. But Who Was He?,} Hua Hsu notes that the Juiliard trained Cheng was cast in roles in what are now deemed problematic Broadway shows \quotation{The World of Suzie Wong} and \quotation{Flower Drum Song.} (The New Yorker, August 23, 2019) \useURL[url14][https://www.newyorker.com/culture/culture-desk/stephen-cheng-released-one-single-that-sounded-like-nothing-else-but-who-was-he]\from[url14].} Lee and Cheng are \quotation{together,} in terms of both being Chinese diasporic subjects, and the \quotation{always} in the song's title can be read as suggesting a perpetuity to their diasporic condition.

The various uploads of \quotation{Always Together} are not, for the most part, cultural texts created explicitly for the platform.\footnote{See, for example, the video of Mike Long dancing to \quotation{Always Together,} \useURL[url15][https://www.youtube.com/watch?v=00794lK01Ws]\from[url15].} In and of themselves, these texts are not elegiac. However, Pascal Cheng's process of searching, learning, and interacting with other users to gain knowledge of his father very much is. The activities in the digital yard surrounding this Caribbean cultural text are made possible by the underlying code of the platform that enables creative interaction and data/file storage.

When cultural studies folks talk about the digital, they do not necessarily talk about code, that mysterious \quotation{stuff} that operationalizes language---that \quotation{stuff} that ultimately is akin to the deep structure, grammar and syntax, and symbolism and signifiers that make all forms of linguistic communication possible.\footnote{I use the term {\em operationalize} to convey and retain for the reader the relationship between a given word and its function in the deep structure of both poetics and code. Code is a language understood by computers. It is not a \quotation{natural language} per se, despite human language being the symbolic system used to create it. Human language has to be converted into a set of \quotation{words} that the computer will understand. The keywords are akin to verbs. Verbs describe or indicate action, and keywords initiate action. An arrangement of keywords for successful execution of a desired computation (or computational task) is referred to as {\em syntax}, and when a set of keywords is put together with syntax, the result is a programming language.} The relationship of poetry to code is not necessarily obvious. Sandy Baldwin, reading inversely from code to poetry, claims that \quotation{poetry works like the doubleness of code (as something to be read and something to be performed).}\footnote{Sandy Baldwin, \quotation{A Poem Is a Machine to Think With: Digital Poetry and the Paradox of Innovation,} {\em Postmodern Culture} 13, no. 2 (2003), \useURL[url16][http://pmc.iath.virginia.edu/issue.103/13.2baldwin.html]\from[url16], para. 13. Loss Pequeño Glazier's {\em Digital Poetics: The Making of E-Poetries} (Tuscaloosa: University of Alabama Press, 2002) was helpful in my distinguishing that I am not referring to e-poetry, or solely literary texts that have been migrated to digital environments.} It is the deep structure, and the messiness within that structure, that creates connection and produces meaning, and messy code is not necessarily code that does not work.\footnote{In fielding a question about the difference between clean and messy code, Daniel Super claims that the key difference is \quotation{comprehensibility}: \quotation{You can understand clean code when you read it. It has meaningful variable and function names~ The responsibilities are separated out into discrete pieces. The methods are organized in a manner that makes sense, with related functionality grouped together. It has comments that actually help you understand what's happening. You can make sense of it. You don't have to read through it ten times and struggle to make sense of what the person who wrote it was trying to do. Clean code is code that someone else, other than the person who wrote it{[},{]} could maintain without wanting to strangle you and feed your corpse into a wood chipper.} Daniel Super, \quotation{What Is the Biggest Difference between \quote{Clean Code} and \quote{Messy Code}?,} {\em Quora}, 12 August 2015, \useURL[url17][https://www.quora.com/What-is-the-biggest-difference-between-clean-code-and-messy-code]\from[url17].} Similar to poetic form, what looks like an anomaly or deviation in form may in fact be an alternative way of thinking about the symbolic connections created when language is operationalized. The underlying code makes traversing digital space possible, allows users to move from one nodal point to another, and brings together material and immaterial, digital and analog technologies. It is this deep structural code that links together uploads associated with \quotation{Always Together,} as well as linking them to clips such as \quotation{Chinese Rocksteady,} a video captured on a cell phone in which musician and producer Clive Chin---eldest son of Vincent \quotation{Randy} and Pat Chin, founders of VP Records and Randy's Studio 17---is seen and heard playing \quotation{Always Together} in his capacity as \quotation{selector} (MC {[}master of ceremonies{]} and arbiter of musical selections) at the Shelter, an underground club in Shanghai.\footnote{See \quotation{Chinese Rocksteady,} YouTube, \useURL[url18][https://youtu.be/3xum1CIiSxI][][https://www.youtube.com/watch?v=3xum1CIiSxI]\from[url18]. Clive Chin, one of the founders of \quotation{dub,} has worked as a musician and producer with the Wailers, Dennis Brown, Lee \quotation{Scratch} Perry, Black Uhuru, and Gregory Isaacs, among others. The Shelter, one of Shanghai's most popular clubs, was established by local DJ Gary Wang and expat Gareth Williams and operated in Shanghai's French Concession from 2007 to its closing in late 2016. The club was housed in a renovated bomb shelter, one of over two thousand built during the 1950s by the Chinese government in response to the threat of Soviet encroachment, and was known for playing roots reggae, dancehall, and dubstep and for promoting local music. See Gareth Williams, \quotation{Cutting Edge Music Finds a Home in Shanghai at the Shelter,} {\em Electronic Beats}, 24 June 2016, \useURL[url19][https://www.electronicbeats.net/club-feature-shelter-shanghai/]\from[url19].}

In the clip, uploaded 30 November 2010, the dub music pioneer Chin tells the enthusiastic crowd that in playing this \quotation{beautiful chune} he is fulfilling a promise made a year prior, and that the crowd may be able to understand the song because of the dialect in which it is sung. The Shelter, located in the French Concession, is in the part of Shanghai where Stephen Cheng was born. Eight years after the initial upload, a user by the name of JujuBean claimed to be the grandchild of Stephen Cheng and asked for information about the upload; in fact, Pascal Cheng is JujuBean's father. The search for knowledge, history, connection, and, potentially, community and consolation has now been taken up by the next generation. I read Pascal Cheng's efforts to unravel diasporic threads as an instance of digital diasporic elegiac practice.

There is an element of pleasure in the dive down the rabbit hole and in discovering and exploring the affective ligaments/code connecting me to Pascal Cheng's search for his father. I fell into this process, allowed myself to meander, to feel frustration and disappointment when links failed or were broken, and to be happy when I realized that JujuBean had taken up the search for their ancestor. I continue to question what it means to be bereft in this historical moment, what it means to explore in this manner, courtesy of the algorithmic code that links us on and to digital platforms, and I read the digital yard as providing a space for creation, invention, and expression---a space in which grief, as Judith Butler argues, is not privatized but \quotation{exposes the constitutive sociality of the self, a basis for thinking a political community of a complex order.}\footnote{Judith Butler, \quotation{Beside Oneself: On the Limits of Sexual Autonomy,} in {\em Undoing Gender} (New York: Routledge, 2004), 17.} I have defined digital diasporic elegiac process as an agglutinative and interactive one, linking texts, practices, and participation of diasporic subjects across social media platforms in the \quotation{digital yard} to address themes of death, dying, loss, mourning, and lamentation. I would like to now turn to a cultural text that is explicitly a digital diasporic elegy.

\subsection[title={\#40NightsofTheVoice},reference={nightsofthevoice}]

I conclude with a return to a cultural text created for the digital yard that is immediately identifiable as a digital diasporic elegy, 40 Nights of the Voice. I make this turn to distinguish between elegiac process and elegiac objects. The organizers' elegiac mission was made explicit when they announced, \quotation{We hope that this collective celebration will serve both as commemoration of {[}Brathwaite's{]} legacy and a way to commune together during this time of quarantine.}\footnote{\quotation{\#40NightsoftheVoice---Tribute to Kamau Brathwaite,} Griffin Poetry Prize Events, \useURL[url20][https://www.griffinpoetryprize.com/event/40nightsofthevoice-tribute-to-kamau-brathwaite/]\from[url20].} It began with thirteen poets and scholars reading Brathwaite's poem \quotation{Wake.}\footnote{In partnership with NGC Bocas Lit Fest, this collaborative reading was posted on bocaslitfest.com and uploaded to YouTube. Subsequent uploads were announced on Twitter @Kamauremix, with links provided to the Kamau Brathwaite Remix Engine account on YouTube.}

That a born-digital elegy would be created in honor of Brathwaite seems inevitable, given Brathwaite's techne in creating his \quotation{Sycorax Video Style.} Brathwaite's personal digital elegiac practice coheres in his \quotation{time of salt,} a culmination of three traumatic events: the death of Doris, his first wife and bibliographer; the destruction of his home, including his Library of Alexander, by Hurricane Gilbert in 1988; and a home invasion in which a gun was held to his head, the trigger pulled, and the empty chamber lodged a symbolic bullet in his head. Unable to access computer files on his recently deceased wife's Apple Mac SE30, Brathwaite experimented with the technology available to him as he struggled to express his grief. Through this tussle with language that seemed insufficient to the purpose, Brathwaite found, in the tension between analog and digital, an alternative aesthetic language that took the {\em para} out of {\em paratextual}. He experimented with alternating fonts, sizes, leading, and images achievable through the Mac and Stylewriter printer. Once we consider the navigational code, the computational infrastructure that links keystroke to lighted image(s) on the screen, the underlying keywords and syntax that make up the programming code that links desktop computer to printer, and the (im)material magic he perceived therein, then Brathwaite becomes the architect of digital diasporic elegy in its inchoate form.

Ultimately, and unsurprisingly, the 40 Nights of the Voice project could not be contained by the spatiotemporal parameters the organizers originally set. It had to respond to the exigencies of the moment, which meant in some cases the doubling up of posts; missing a day because of a social-media blackout protest; and unexpected participant choices, decisions, and technologies that effected the length, composition, style, and so on of the contributions. The mise-en-scène of many uploads strike familiar notes in terms of the \quotation{talking head.} There is the positioning of the reader: most centered, some off center. Many are framed by stacks of books. Some are more theatrical; some are more musical. Others still are more \quotation{academic,} offering prefatory remarks, situating the reading in the context of Brathwaite's oeuvre, situating themselves in relation to the reading, situating themselves in relation to the man. Several chose to read from Brathwaite's considerable body of elegiac poems. Some offered something more personal---a memory of their first encounter with the man, their first encounter with his work, and so on. Each upload is now a point along a digital archipelago that invokes structures of mediation between the analog and the digital, codes and subcodes.

The digital yard is a site of creation and community. It is a restorative space in which the traditional rites of mourning we have been denied in our contemporary moment can take shape anew in the context of a complex system---one that does not have simple discursive explanations, that cannot be explained by simple or explicit equations.\footnote{In her prologue to {\em My Mother Was a Computer}, N. Katherine Hayles claims, \quotation{Because complex systems exhibit nonlinear behaviors that typically cannot be described by equations having explicit solutions, the kind of mathematics that give us classical mechanics and other triumphs of modern science has little traction in the case of complex systems, which leaves us with simulations and discursive explanations.} N. Katherine Hayles, \quotation{Computing Kin,} prologue to {\em My Mother Was a Computer: Digital Subjects and Literary Texts} (Chicago: University of Chicago Press, 2005), 5--6.} What links the 40 Nights of the Voice project and the uploads of \quotation{Always Together} is the platform, the yard, and the computational infrastructure that makes expression, movement, and connection possible between digital objects and participants. I propose here that there is a Caribbean digital diasporic elegiac practice that embraces the subterranean codes and networks that transform the platform---the yard---into a site of generative mourning and healing.

\thinrule

\subsection[title={Bibliography},reference={bibliography}]

Anderson, Benedict. {\em Imagined Communities}. London: Verso, 2016.

Baldwin, Sandy. \quotation{A Poem Is a Machine to Think With: Digital Poetry and the Paradox of Innovation.} {\em Postmodern Culture} 13, no. 2 (2003). \useURL[url21][http://pmc.iath.virginia.edu/issue.103/13.2baldwin.html]\from[url21].

Baron, James. \quotation{The Tao of Stephen Cheng.} {\em Taipei Times}, 28 November 2019. \useURL[url22][https://www.taipeitimes.com/News/feat/archives/2019/11/28/2003726596]\from[url22].

Braaten, Rachel, and Lisa Odham Stokes, eds.~{\em Historical Dictionary of Hong Kong Cinema}, 2nd edition. Lanham, MD: Rowman and Littlefield, 2020.

Bradley, Lloyd. {\em Reggae: The Story of Jamaican Music}. London: BBC Worldwide, 2002.

Bradley, Lloyd. {\em This Is Reggae Music: The Story of Jamaica's Music}. New York: Grove, 2000.

Brathwaite, Kamau. {\em The Zea Mexican Diary:} {\em 7 Sept 1926--7 Sept 1986.} Madison: University of Wisconsin Press, 1993.

Butler, Judith. \quotation{Beside Oneself: On the Limits of Sexual Autonomy.} In {\em Undoing Gender}, 17--39. New York: Routledge, 2004.

Chang, Alexandra. {\em Circles and Circuits: Chinese Caribbean Art}. Durham, NC: Duke University Press, 2018.

Chang, Jeff. {\em Can't Stop Won't Stop: A History of the Hip-Hop Generation}. New York: Picador, 2005.

Chang, Kevin O'Brien, and Wayne Chen. {\em Reggae Routes: The Story of Jamaican Music}. Philadelphia: Temple University Press, 1998.

Chin, Timothy. \quotation{Notes on Reggae Music, Diaspora Aesthetics, and Chinese Jamaican Transmigrancy: The Case of VP Records.} {\em Social and Economic Studies} 55, nos. 1--2 (2006): 93--114.

Ellenbogen, Josh. \quotation{On Photographic Elegy.} In Weisman, {\em Oxford Handbook of the Elegy}, 681--99.

Ellis, Nadia. \quotation{Elegies of Diaspora.} {\em Small Axe}, no. 43 (March 2014): 164--72.

Fuss, Diana. {\em Dying Modern: A Meditation on Elegy}. Durham, NC: Duke University Press, 2013.

Glazier, Loss Pequeño. {\em Digital Poetics: The Making of E-Poetries}. Tuscaloosa: University of Alabama Press, 2002. Hayles, N. Katherine. \quotation{Computing Kin.} Prologue to {\em My Mother Was a Computer: Digital Subjects and Literary Texts}, 1--11. Chicago: University of Chicago Press, 2005.

Helsinger, Elizabeth. \quotation{Grieving Images: Elegy and the Visual Arts.} In Weisman, {\em Oxford Handbook of the Elegy}, 658--80.

Hume, Yanique, and Aaron Kamugisha, eds.~{\em Caribbean Popular Culture: Power, Politics, and Performance}. Kingston: Ian Randle, 2016.

Huot, Claire. {\em China's New Cultural Scene: A Handbook of Changes}. Durham, NC: Duke University Press, 2000.

Josephs, Kelly Baker. \quotation{Digital Yards: Caribbean Narrative on Social Media and Other Digital Platforms.} In {\em Caribbean Literature in Transition}, vol.~3, {\em 1970--2020}, edited by Ronald Cummings and Alison Donnell. Cambridge: Cambridge University Press, forthcoming.

Josephs, Kelly Baker. \quotation{Me, Myself, and Unno: Writing the Queer Caribbean Self into Digital Community.} {\em archipelagos journal} 5, November 2020. \hyphenatedurl{http://archipelagosjournal.org/issue05/josephs-blogging.html.}

Min, Susette. \quotation{Remains to Be Seen: Reading the Works of Dean Sameshima and Khanh Vo.} In {\em Loss: The Politics of Mourning}, edited by David L. Eng and David Kazanjian, 229--50. Berkeley: University of California Press, 2003.

Notar, Beth E. \quotation{Off and On the Road to Reform,} chap.~6 in {\em Displacing Desire: Travel and Popular Culture in China}. Honolulu: University of Hawai'i Press, 2006.

Raghav, Krish. \quotation{Redemption Songs.} {\em Topic Magazine}, no. 24, June 2019. \useURL[url23][https://www.topic.com/redemption-songs]\from[url23].

Ramazani, Jahan. \quotation{Nationalism, Transnationalism, and the Poetry of Mourning.} In Weisman, {\em Oxford Handbook of the Elegy}, 601--19.

Ramazani, Jahan. {\em Poetry of Mourning: The Modern Elegy from Hardy to Heaney}. Chicago: University of Chicago Press, 1994.

Shelley, Percy Bysshe. \quotation{Ozymandias.} {\em Examiner}, 11 January 1818.

Sterling, Marvin. {\em Babylon East: Performing Dancehall, Roots Reggae, and Rastafari in Japan}. Durham, NC: Duke University Press, 2010.

Weisman, Karen. Introduction to Weisman, {\em The Oxford Handbook of the Elegy}, 1--9.

Weisman, Karen, ed.~{\em The Oxford Handbook of the Elegy}. Oxford: Oxford University Press, 2010.

Williams, Gareth. \quotation{Cutting Edge Music Finds a Home in Shanghai at the Shelter.} {\em Electronic Beats}, 24 June 2016. \useURL[url24][https://www.electronicbeats.net/club-feature-shelter-shanghai/]\from[url24].

Xiao, Ying. \quotation{\quote{Hip Hop Is My Knife, Rap Is My Sword}: Hip Hop Network and the Changing Landscape of Image and Sound Making.} Chapter 6 in {\em China in the Mix: Cinema, Sound, and Popular Culture in the Age of Globalization}. Jackson: University Press of Mississippi, 2017.

\thinrule

\page
\subsection{Tzarina T. Prater}

Tzarina T. Prater is an associate professor in the Department of English and Media Studies at Bentley University. She has published articles on Sino-Caribbean literature and culture as well as US spectatorship of Hong Kong action cinema, digital platforms, and science fiction. Her current book project on Chinese Jamaican literary and cultural production, \quotation{Labrish and Mooncakes: National Belonging in Chinese Jamaican Cultural Production,} is in process.

\stopchapter
\stoptext