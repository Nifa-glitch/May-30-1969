\setvariables[article][shortauthor={Josephs}, date={December 2020}, issue={5}, DOI={https://doi.org/10.7916/archipelagos-mg7h-nf52}]

\setupinteraction[title={Me, Myself, and Unno: Writing the Queer Caribbean Self into Digital Community},author={Kelly Baker Josephs}, date={December 2020}, subtitle={Me, Myself, and Unno}]
\environment env_journal


\starttext


\startchapter[title={Me, Myself, and Unno: Writing the Queer Caribbean Self into Digital Community}
, marking={Me, Myself, and Unno}
, bookmark={Me, Myself, and Unno: Writing the Queer Caribbean Self into Digital Community}]


\startlines
{\bf
Kelly Baker Josephs
}
\stoplines


{\startnarrower\it Caribbean autobiography in the mid-twentieth century grew as a method of combatting aporias in literary production and epistemological models, with novelists in particular writing themselves into visibility. But this tradition was predominantly male and decidedly heterosexual. Even as the Caribbean literary canon slowly diversified toward the end of the twentieth century with the inclusion of various life stories, the voices of queer writers remained at low volume, almost muted. This essay reads how those voices were enabled at the turn of the millennium by digital technologies. Reading digital versions of the personal narrative within the historical context of Caribbean autobiography, it examines the self-fashioning twenty-first-century Caribbean writers performed and published via the blogosphere. With particular focus on blogs by two Jamaican writers---Staceyann Chin and Marlon James---it argues that the form and technical features of blogging offered a safe space for queer Caribbean writers to \quotation{speak out} in ways that continued the tradition of Caribbean autobiography. \stopnarrower}

\blank[2*line]
\blackrule[width=\textwidth,height=.01pt]
\blank[2*line]

In two {\em New York Times} articles, published ten years apart, Staceyann Chin and Marlon James tell deeply personal stories about their experiences as queer Jamaican writers now living in the United States. New York City, the star of both pieces, offered to both \quotation{sexiles} the refuge and freedom intrinsic to the anonymity and individuality that was potentially available far away from their homophobic and claustrophobic homeland.\footnote{Manuel Guzmán is credited with coining {\em sexile} to describe exclusion, or exile, from a home community because of an individual's sexual orientation. The term has been further explored and expanded by Yolanda Martínez-San Miguel, and Jeannine Murray-Román has applied it directly to Staceyann Chin. See Manuel Guzmán, \quotation{{\em Pa' la escuelita con mucho cuida'o y por la orillita}: A Journey through the Contested Terrains of the Nation and Sexual Orientation,} {\em Puerto Rican Jam: Rethinking Colonialism and Nationalism}, ed.~Frances Negrón-Muntaner and Ramón Grosfoguel (Minneapolis: University of Minnesota Press, 1997), 209--28; Yolanda Martínez-San Miguel, \quotation{Female Sexiles? Toward an Archeology of Displacement of Sexual Minorities in the Caribbean,} {\em Signs} 36, no. 4 (2011): 813--36; and Jeannine Murray-Román, \quotation{Staceyann Chin and Zoe Valdés: Sexilic Politics in the Blogosphere,} chap.~4 in {\em Performance and Personhood in Caribbean Literature: From Alexis to the Digital Age} (Charlottesville: University of Virginia Press, 2016).} Chin's 2004 piece declares, \quotation{No one cared if I kissed girls,} and James ends his 2015 piece with this observation: \quotation{Nobody cared that my jeans had a nine-inch rise. I no longer looked over my shoulder in the dark.}\footnote{See Staceyann Chin, \quotation{No One Cared if I Kissed Girls,} {\em New York Times}, 21 November 2004, \useURL[url2][https://www.nytimes.com/2004/11/21/nyregion/thecity/no-one-cared-if-i-kissed-girls.html]\from[url2]; and Marlon James, \quotation{From Jamaica to Minnesota to Myself,} {\em New} {\em York Times Magazine}, 10 March 2015, \useURL[url3][https://www.nytimes.com/2015/03/15/magazine/from-jamaica-to-minnesota-to-myself.html]\from[url3].} The freedom these declarations imply is based less on the isolation of the individual on a quest for independence than on Chin's and James's (re)making of the self via the (re)making of community. It is not that no one cared but that a curated constellation of people cared differently. Chin's writing offered her a global stage on which she could gain emotional and financial support by \quotation{teaching the poetics of writing the self,} and James similarly supported his sexile by writing his \quotation{way out} of Jamaica. Though the narrative of these published essays seems to be one of writing oneself away from the rejecting homespace, Chin and James were also actively writing themselves into a remembered Jamaica and a diasporic Caribbean. In these essays, and in much of their other \quotation{writings of the self,} they both write their way into a Caribbeanness that their autobiographical writings work to redefine as inclusive of nonheteronormative choices.

Writings of the self are not a new method of reshaping the contours of what might be called \quotation{Caribbean.} As Sandra Pouchet Paquet argues in her pivotal 2002 study {\em Caribbean Autobiography: Cultural Identity and Self-Representation}, there is an extensive history of life-writing in the Caribbean literary canon that attempt such interventions via print publications, beginning with early-nineteenth-century narratives such as those by the Hart sisters of Antigua (Elizabeth Hart Twaites and Anne Hart Gilbert) and {\em The History of Mary Prince, a West Indian Slave, Related by Herself}. Paquet's argument elsewhere for the \quotation{centrality of the personal narrative to the unfolding creative project that was Caribbean literature} positions life-writing as foundational to the building of the modern Caribbean canon.\footnote{Sandra Pouchet Paquet, \quotation{On Caribbean Autobiographies,} {\em Anthurium} 10, no. 2 (2013): 2, \useURL[url4][http://doi.org/10.33596/anth.231]\from[url4].} With this essay I hope to expand the range of such narratives to include not only individual, institutionally vetted essays like those published by Chin and James in the {\em New York Times} but also digital writing, specifically posts from personal blogs maintained by Caribbean writers. The autobiographical narratives shared digitally by a number of Caribbean writers in the early twenty-first century were a new, yet familiar, form of life-writing.

Though blogs necessarily present and respond to forces different from those we find in more formally crafted autobiographical narratives, they nonetheless can be read as recent contributions to what Paquet names a \quotation{contested literary space {[}that forms{]} the foundations or diverse conceptions of individual and group identity} for Caribbean peoples.\footnote{Sandra Pouchet Paquet, {\em Caribbean Autobiography:} {\em Cultural Identity and Self-Representation} (Madison; University of Wisconsin Press, 2002), 261.} In the mid-twentieth century, Caribbean autobiography grew as a method of claiming ground in this \quotation{contested literary space.} From George Lamming's {\em In the Castle of My Skin} to Derek Walcott's {\em Another Life}, Caribbean writers struggled to combat aporias in literary production and epistemological models and write themselves into visibility from the margins of colonial empires. But this tradition was predominantly male and heterosexual. Even as questions of female sexual agency became more central in autobiographical texts of the late twentieth century, these questions were most often about what women did with cisgender men or sometimes about what they did by themselves, but rarely were the questions about what women did with other women or with noncisgender men. Therefore, while the twentieth-century Caribbean literary canon slowly became diversified with various life stories, the voices of queer writers remained at low volume. In this essay, I examine how two Jamaican writers in particular further diversified the Caribbean autobiographical tradition at the turn of the century by centering questions of queerness while experimenting with the formal and communal affordances (and limitations) of blogging platforms.

\subsection[title={\quotation{Positions of Enunciation}},reference={positions-of-enunciation}]

\startblockquote
some magic boy named Davidfixed the pageso nowI might write you love notesmore frequent than the drip drip dripof mute helloshi worldhere we go again
\stopblockquote 

\startalignment[flushright]
\tfx{Staceyann Chin. "Long Time\ldots{}," Cyberjournal, 15 February 2006}
\stopalignment
\blank[2*line]


The form of blogging---with its imagined audience(s), networked sites, and nested comments---makes evident the tensions and interdependence inherent in navigating individual and group identity. Traditional print autobiographical works convey these tensions secondhand, but the interactive nature of blogs lays such conflict and connectedness bare, particularly when a blogger positions themself as speaking from or to a specific community. In his 1990 essay \quotation{Cultural Identity and Diaspora,} Stuart Hall writes,

\startblockquote
Practices of representation always implicate the positions from which we speak or write---the positions of {\em enunciation}. What recent theories of enunciation suggest is that, though we speak, so to say \quotation{in our own name,} of ourselves and from our own experience, nevertheless who speaks, and the subject who is spoken of, are never identical, never exactly in the same place.\footnote{Stuart Hall, \quotation{Cultural Identity and Diaspora,} in {\em Identity: Community, Culture, Difference}, ed.~Jonathan Rutherford (London: Lawrence and Wishart, 1990), 222. Hall published an earlier version of his essay in 1989 as \quotation{Cultural Identity and Cinematic Representation,} based on a talk he gave in 1988. As the revised title indicates, in the version I quote here, Hall has shifted his focus a bit more toward diasporic identity formation.}
\stopblockquote

Hall provides language to think about the ways Caribbean bloggers a decade later were involved in \quotation{practices of representation} that were dependent on, and deeply dedicated to, the positions/places/people from which they \quotation{enunciated} themselves into the World Wide Web. Because their blog posts were self-consciously instituted practices of self-representation, wherein the distance between the enunciated \quotation{I} and the enunciating \quotation{I} was less mediated by editors and publishing processes (and the lengths of time inherent in such processes) than in traditionally published autobiographical narratives, the different \quotation{places} of the subject and of the speaker were more closely aligned. And the interactive format of blogging promised that the different places of the speaker and the subject, {\em and} of their audiences, could also be (or be made) more closely aligned.

This promise of unfixed positioning offered to the self-speaking subject by early blogging platforms is a key difference between print and digital autobiography, a key difference between most, if not all, print and digital enunciations of the \quotation{I.} From 2000 to 2010, particularly toward the latter half of the decade, Caribbean writers of all kinds began to turn to blogging to speak themselves into the digital universe and to connect to established and potential audiences for their nondigital writings. The work that blogs could (and still can) do for such Caribbean \quotation{litbloggers} is varied.\footnote{I take the term {\em litbloggers} from Nicholas Laughlin's interview series with Caribbean bloggers on {\em Global Voices}. The series begins with \quotation{Talking to Jamaican Litblogger Geoffrey Philp} on 14 May 2007, \useURL[url5][https://globalvoices.org/2007/05/14/talking-to-jamaican-litblogger-geoffrey-philp]\from[url5]. Several of the interviewed writers discuss why they turn to blogging and what it offers separate from, and in connection to, their nondigital writings.} Blogs might serve the function of \quotation{reaching out} to readers of one's published work to simply make a connection; they might function to publicize work and appearances; they might serve to grow create/grow an audience; they might offer insight into an author's process. In the Caribbean blogosphere, writers such as Tobias Buckell, Nalo Hopkinson, Nicholas Laughlin, and Geoffrey Philp were relatively early adopters of the form, embracing the advantages the internet offered for communication with various audiences (including among themselves). Although their blogs were arguably just as autobiographical as Chin's and James's, I narrowed my corpus for this essay because I am particularly interested in how queer Caribbean writers navigated the shifting terrain of digital representation, of self-enunciation and audience. Additionally, the timing of Chin's and James's blogs offers diverse perspectives of digital autobiographical writing across the peak decade of Caribbean blogging. Chin began her cyberjournal in 2001, which was earlier than most of her Caribbean contemporaries, and ended in 2006; James began blogging in 2006 and ended in 2009, when blogging as a whole became less autobiographical (as more writers turned to Facebook and other social media platforms designed specifically for self-representation).

While by the end of the 2000s writers might have questioned if anyone was reading {\em their} blog, they did not question if anyone read blogs at all or the legitimacy of blogging as a genre.\footnote{When asked in an interview with Nicholas Laughlin if he felt his blog had \quotation{widened the audience for {[}his{]} fiction,} James responded, \quotation{I'm not even sure that people read what I write.} But shortly before that he also reveals, \quotation{My friend Don Lee is convinced that my blog persona is a completely different person. A creative writing teacher told me once that my literary persona is a far better writer. At one point there were people who knew me as a blogger, and nothing else.} Nicholas Laughlin, \quotation{Talking to Jamaican Writer and Blogger Marlon James,} {\em Global Voices}, 8 May 2009, 2009, \useURL[url6][https://globalvoices.org/2009/05/08/talking-to-jamaican-writer-and-blogger-marlon-james]\from[url6].} Chin's early posts indicate the pervasive uncertainty surrounding her entrance into what was then a fairly new field of digital public self-disclosure. Her choice of a cyberjournal instead of a titled blog is owed in large part to her relatively early adoption of this form of communication. According to Jill Walker Rettberg, while \quotation{the first online diaries appeared around 1994,} they were not yet called {\em blogs}:

\startblockquote
{\em Web log} at that time was used to refer to the statistics available to website administrators showing the number of visitors to a website. In 1997, Jorn Barger proposed that the term {\em weblog} be used to refer to websites that post links to interesting material with commentary\ldots{}. Posts in these early weblogs were brief, and although the comments usually had a clear individual voice and offered personal opinions, the content was usually comments on news stories and online finds rather than stories from the writer's life. Weblogs tended to be seen in opposition to online diaries, which were more confessional\ldots{}. {[}But{]} within a few years, the once quite separate genres of online diaries and blogs merged. Blog posts became longer and more essayistic, often using a more personal voice, and online diaries came to include more essayistic material and commentary in addition to the autobiographical content.\footnote{Jill Walker Rettberg, \quotation{Online Diaries and Blogs,} in {\em The Diary: The Epic of Everyday Life}, ed.~Batsheva Ben-Amos and Dan Ben-Amos (Bloomington: Indiana University Press, 2020), 412--13.}
\stopblockquote

Chin's cyberjournal appeared amid the merging of online diaries and political and cultural commentaries. Even as her posts slowed in frequency after the first two years, she maintained her unique mix of uncompromisingly mundane chronicle, large-scale polemic, and snippets of draft poetry until the middle of the decade. This type of life-writing---this public, almost immediate, seemingly unvetted by others self-representation---allows for a slide into a perceived over-sharing. However, insofar as Chin sometimes refers to multiple revisions of her entries before posting (until mid-2002 she did not actually have the ability to post the entries herself and had to send drafts to someone with more technical knowledge for posting), we must consider her intensely personal style purposeful.\footnote{In her 31 March 2002 entry, Chin refers to a spoof of her site that parodies her intimate, diaristic style of blog writing and her centering of her sexuality in those writings. Textual clues on Chin's and others' blogs indicate this parody was done by Peter Dean Rickards, a Jamaican photographer who ran the popular website {\em Afflicted Yard} (I have not, however, been able to find the parody itself online).}

Chin's entries chronicle her performance work, her travel and days off, her time with lovers/family/friends, and her random interactions with people she meets at colleges and the doctor's office. Threaded through these seemingly inconsequential revelations are the quotidian experiences of micro and macro aggressions against Caribbean queerness and Chin's insistent claiming of her right to Jamaicanness, to Brooklyn, and to something approaching \quotation{home.} As with more polished, published autobiographical narratives, Chin's cyberjournal imagines that the details of her life represent a larger import for her readers. Sometimes, however, the technicalities of web-writing, which were new to her and had to be filtered through her webmaster's mysterious magic, distanced her from her imagined audience. When shifting her posts to a new site in early 2002, she temporarily lost the comments section of the cyberjournal, and in several of her posts she questions whether anyone is listening. On 3 March 2002, in her first post to the cyberjournal in its new home, she writes, \quotation{I'm rambling because it feels like I'm talking to a blank screen. I was more connected when people were responding to stuff I was writing. Hopefully people will respond again and then it won't be such a one-sided unorgasmic experience.}\footnote{Staceyann Chin, \quotation{March 03, 2002,} {\em Cyberjournal}; printout in author's possession.} The comments section for her posts soon began to fill again, however, and the pleasures of self-enunciation resumed.

\subsection[title={Strategic Enunciations: What to Tell, and How},reference={strategic-enunciations-what-to-tell-and-how}]

\startblockquote
But so many of us have things that need to be said, and the Internet is the only place where those voices can be heard.
\stopblockquote 

\startalignment[flushright]
\tfx{Marlon James quoted in Laughlin, "Talking to Jamaican Writer and Blogger Marlon James."}
\stopalignment
\blank[2*line]


\startblockquote
Such is the temperament of the technological crapshoot that frames our small lives.
\stopblockquote 

\startalignment[flushright]
\tfx{Staceyann Chin. "May 21, 2002," Cyberjournal; printout in author's possession.}
\stopalignment
\blank[2*line]


In general, I have organized this essay around three questions of digital self-enunciation: To whom do these writers speak? Of what do they choose to (not) speak? And how do---how can---they speak? These questions are about content as much as technology. Print capitalism continues to shape digital publishing via the forms of creative expression that can be recognized as such. Of course, capitalism in general also affects digital writing through the limitations of commercial digital spaces, which are often the spaces most available to Caribbean writers in the region for digital expression. Discourse on digital publishing often promotes the practice as freeing and enabling for creative enterprise, but, as with the technologies of language and print, digital communication also structures what can be said. In this section, I turn to the what and the how of queer Jamaican self-enunciation through the digital technologies and platforms popular in the early twenty-first century.

What could be written of queer Caribbean selfhood on blogs of this time? The answer is elusive, since it lies as much in what was not written as in what was and could be written, both via the available technologies and in a public space. For the \quotation{not written} we can look to a post's comments, which might reveal silences and silencings by the blogger, or to unmentioned events signaled as contemporary by the time/date stamps built into most blog technologies. But even nondigital autobiographies by queer Caribbean writers have had to contend with necessary silences and the constraints of form in order to tell truths larger than an individual life. Michelle Cliff's first book, {\em Claiming an Identity They Taught Me to Despise}, is a collection of autobiographical writings published in 1980 that weaves together myth, literature, and personal history with a blend of prose and poetry. Cliff begins by declaring, \quotation{I must make myself visible against my habitat,} then immediately acknowledges that \quotation{there exists a certain danger in peeling back.}\footnote{Michelle Cliff, {\em Claiming an Identity They Taught Me to Despise} (Watertown, MA: Persephone, 1980), 3.} Two years later Audre Lorde published {\em Zami: A New Spelling of My Name}, naming her autobiographical {\em Künstlerroman} composed of \quotation{dreams/myths/histories} a \quotation{biomythography.}\footnote{Audre Lorde, acknowledgments to {\em Zami: A New Spelling of My Name} (Berkeley, CA: Crossing, 1982), n.p. Lorde's \quotation{biomythography} had originally been published in 1982 by Persephone Press, which had also published Cliff's {\em Claiming an Identity}. I note this as potential evidence of the difficulties queer writers of color may have had in gaining, and maintaining, access to publishing their stories. Persephone was a small independent press specializing in feminist and lesbian books. The press was short-lived (1976--83) and, after publishing fourteen titles, folded because of financial troubles. In a 1983 interview, Lorde spoke of having to pay for the rights to republish {\em Zami} and criticized Persephone for their treatment of women writers of color. See Karla Jay, \quotation{Speaking the Unspeakable: Poet Audre Lorde,} in Audre Lorde, {\em Conversations with Audre Lorde}, ed.~Joan Wylie Hall (Jackson: University Press of Mississippi, 2004), 111; originally published in {\em Gay News} (Philadelphia), 15 March 1984.} As lesbians of color, even with the relative privilege of diasporic residency, both Cliff and Lorde convey that \quotation{writing the self} requires formal experimentation and generic promiscuity to safely tell and not-tell their stories, to speak (to echo Stuart Hall) in their own name, of themselves and from their own experience.

The act of writing---of writing about oneself especially---implies a level of freedom enjoyed by the writer, even when that narrative might focus on being unfree.\footnote{I am thinking here of Lisa Lowe's categorization of autobiography as a \quotation{genre of liberty.} In her reading of {\em The Interesting Narrative of the Life of Olaudah Equiano, or Gustavus Vassa, the African}, Lowe argues, \quotation{The intense social value accorded the autobiographical genre illustrates how liberal emancipation required a literary narrative of the self-authoring autonomous individual to be distilled out of the heteronomous collective subjectivity of colonial slavery.} Autobiography, then, raises the narrative subject (and therefore the writer) above the very conditions that make for the content and potential import of the narrative. Lisa Lowe, {\em The Intimacies of Four Continents} (Durham, NC: Duke University Press, 2015), 2, 47.} Both Cliff and Lorde allude to this paradoxical freedom to name their oppressions as shadowed by people and circumstances they \quotation{can not yet afford to name.}\footnote{Lorde, acknowledgments.} In digital spaces, this freedom for Caribbean queer writers to tell is still noticeably constrained by the need to not-tell. Marlon James began his blog, first named {\em Croaking Marley}, in 2006, during his time as a graduate student at Wilkes University in Pennsylvania. Beyond a September 2006 post referring to his being on a student visa and in Pennsylvania, however, he provides little information about his time at Wilkes. In his early posts, he writes \quotation{our} and \quotation{we} primarily for Jamaicans and refers to himself as a nonresident in America. While there are a few personal details to be gleaned from his first posts---such as his turn to comics as a kid (\quotation{I learned more about accepting my outsider status from X-men than both the Bible and Notes From the Underground}\footnote{Marlon James, \quotation{On X-Men 3,} {\em Marlon James: Among Other Things\ldots{}} (blog), 10 June 2006, \useURL[url7][http://marlon-james.blogspot.com/2006/06/on-x-men-3.html]\from[url7].})---the autobiographical nature of the blog is most visible in his discussions of his work, his process of writing, and his work to become a writer. Readers learn more about James the capital-W Writer than James the person. That is, until about five months into the blog, in November 2006, when he suddenly gifts readers with the post \quotation{Where I'm Coming From,} which details two life-shaping moments for the young Marlon, in primary and secondary school in Jamaica, and ends with the revelation of a drastic decision:

\startblockquote
There was just no way I could continue like this, hated because I like art and lit and history, didn't like football (soccer) and walked and talked funny. I realized that more than everything, it must have been my screwed up way of speaking that made people hate me.

So in the summer of 1987 I stopped speaking.\footnote{Marlon James, \quotation{Where I'm Coming From,} {\em Marlon James: Among Other Things\ldots{}}, 27 November 2006, \useURL[url8][http://marlon-james.blogspot.com/2006/11/where-im-coming-from.html]\from[url8].}
\stopblockquote

James's next blog post, published a week later, discusses writing sex scenes in literary fiction: \quotation{When it comes to good explicit action the gays guys seem to have it locked even if the straight audiences may not want to read it.}\footnote{Marlon James, \quotation{Spacebreak Sex,} {\em Marlon James: Among Other Things\ldots{}}, 4 December 2006, \useURL[url9][http://marlon-james.blogspot.com/2006/12/spacebreak-sex.html]\from[url9].} There's no mention of the revelation in the prior post, making its last sentence about his refusal to speak a cliffhanger of monstruous proportions. Perhaps the full stop after \quotation{stopped speaking} is meant to reinforce the decision of the young Marlon. Yet the adult Marlon does keep speaking, albeit about something else, in the very next post. And that is one of the liberating affordances of the nonlinear, post-based format of blogs. Continuity of narrative is unnecessary, and positions of enunciation are unfixed. One can tell and tell, and then not tell. One can simply \quotation{stop speaking,} then begin anew, elsewhere, in the next post.

In 2013--14, James returned to some of the material in \quotation{Where I'm Coming From} for a vaguely fictional, second-person-perspective, thirteen-part series of vignettes for the online literary publication {\em Revolver},\footnote{The project begins with \quotation{Part 1: Cop a Feel for All Your Sins,} on 13 May 2013, and continues through to \quotation{Part 13: You're Now about to Witness the Strength of Street Knowledge,} on 7 August 2014. {\em Revolver} was a digital publication produced by a literary arts organization based in Lowertown, St.~Paul. See \useURL[url10][http://www.around-around.com/the-regulars/marlon-james/]\from[url10] for all thirteen pieces.} and then again for his more-well-known 2015 essay \quotation{From Jamaica to Minnesota to Myself} for the {\em New York Times Magazine}. The latter piece is generally considered his official \quotation{coming out} essay, but the earlier pieces evidence his experiments with strategic \quotation{positions of enunciation} in individual blog posts as well as across the posts for the four-year duration of the blog. Despite the backward chronology of the blog format (especially for noncontemporary readers), our tendency toward narrative often leads readers to perceive the individual posts as part of a whole, stitching together congruent elements to cloak irrelevant or even contradictory ones. In autobiography, we read for what Lisa Lowe calls \quotation{a fluid story of a unitary {[}author{]}.}\footnote{Lowe, {\em Intimacies of Four Continents}, 50.} The technology that enables blogs makes it nearly impossible to read them as fluid narratives; but readers nevertheless need to work against the impulse to suture posts if they seek the stories told by the gaps, paradoxes, and silences. The writers themselves often aim to present a unitary subject in their blogs, repeating anecdotes and referring to previous posts to weave together a composite but coherent identity. James's reworking of material from his blog, first to the {\em Revolver} series and then to the {\em New York Times} {\em Magazine} essay, forces us to pay closer attention to how he leveraged the blog format to tell (and to not-tell) as he worked out how he wanted to form this autobiographical narrative into a fluid story.\footnote{In the {\em Global Voices} interview, Nicholas Laughlin asks James, \quotation{How is writing for your blog different to, say, writing a short essay for a print publication?} James responds, \quotation{My blog posts are very much searches, me writing towards finding what I really want to say. For some reason that feels more natural in a blog format, the act of getting my s**t together. In print, my post about \quote{The Bigots On My Bookshelf} would seem unformed and slightly inconclusive, but online it seems like what it is, my going through a process.} Laughlin, \quotation{Talking to Jamaican Writer and Blogger Marlon James.}}

Chin also reworks material from her cyberjournal entries for later performances and publications, including her 2009 memoir, {\em The Other Side of Paradise}, which crafts into a linear narrative several of the stories she had previously told via her blog, her poetry, and her one-woman shows.\footnote{See Staceyann Chin, {\em The Other Side of Paradise: A Memoir} (New York: Scribner, 2009).} Her navigation of various positions of enunciation becomes obvious as the same stories are retold and refashioned. More supple than James's reworking of his stories of silence and subterfuge for {\em Revolver} and later the {\em New York Times}, Chin's autobiographical narratives are revised across genres and reordered several times. In the (cyber)space of her cyberjournal, which allowed for polemic, poetry, dramatic monologue, and asynchronous dialogue with readers, Chin could endlessly \quotation{remix} her writing choices and, as Jeannine Murray-Román argues, was able to \quotation{amplify} the stories of self she wished to tell.\footnote{See Murray-Román, \quotation{Staceyann Chin and Zoe Valdés.}} Within the reach and technological limits of their chosen platforms, queer Caribbean bloggers like Chin and James can manipulate circulation of their writings of the self---revealing, remixing, amplifying, and suppressing---in ways unavailable to earlier writers such as Michelle Cliff and Audre Lorde.

There remains, however, a drive toward print publication evident in the blogs of this period. Although they were less constrained by the printed page and the order it entails, bloggers like Chin and James, who imagined themselves in a literary lineage with canonical (Caribbean) writers, had to make for themselves the prestige that the printed page (and selection by the hands of print capitalism) grants to published writers. Consequently, in their blog posts, Chin and James position not only their personal experiences but the very labor of writing as what individualizes them, makes them special, and makes their lives representative of worthy (Caribbean and/or queer) lives. For both Chin and James, the labor of writing, both on and off their blogs, became one of the themes that threaded the varied posts together. Whether in celebration or vulnerable self-questioning, whether as an aside or as the focus of a full post, it is the {\em work} of writing that they most often reveal as they express their struggles with their chosen art forms and with the industries they must contend with to become capital-W Writers. This focus on the labor of writing also becomes one of the threads that connect them with their audience and with other writers of the time experimenting with the affordances of digital platforms.

\subsection[title={Navigating Digital Networks},reference={navigating-digital-networks}]

Nicholas Laughlin's 2006 {\em Global Voices} post titled \quotation{11 Key Moments in {[}Anglo-{]} Caribbean Blog History} provides an overview of how Caribbean bloggers in the 2000s took themselves seriously as a network and took blogging seriously as a genre.\footnote{Nicholas Laughlin, \quotation{11 Key Moments in {[}Anglo-{]}Caribbean Blog History,} {\em Global Voices}, 13 January 2006, \useURL[url11][https://globalvoices.org/2006/01/13/11-key-moments-in-anglo-caribbean-blog-history]\from[url11].} Being a part of this networked crew meant linking to other blogs. It meant celebrating when you or your colleagues were mentioned elsewhere on the net or in print; it meant watching for \quotation{spikes} in traffic and perhaps pandering to these spikes. Overall, it was the Caribbean reading the Caribbean writing in a technologically new way.\footnote{I present some of these ideas about a network of bloggers in a more detailed discussion of the Caribbean blogosphere in my forthcoming \quotation{Digital Yards: Caribbean Narrative on Social Media and Other Digital Platforms,} in {\em Caribbean Literature in Transition}, vol.~3, {\em 1970--2020}, ed.~Ronald Cummings and Alison Donnell (Cambridge: Cambridge University Press, forthcoming).} Much of that is lost now, for multiple reasons, including, as mentioned above, writers revising their digital selves. The rise of social media platforms such as Facebook, Twitter, and Instagram has contributed to the decline of Caribbean blogging. In different ways, each of these platforms has become a site for \quotation{microblogging,} where one can present open-ended (and endlessly open to comment) snippets of one's life. And where the more one writes, the faster one's past posts disappear down the timeline, a timeport of \quotation{supermodernity.}\footnote{\quotation{The term supermodernity ({\em surmodernité}),} writes A. González-Ruibal, \quotation{was coined by French Anthropologist Marc Augé in 1992 to define the period commonly known as late modernity. The usefulness of the concept of supermodernity, as compared to other definitions of the later part of modernity, rests on two things: the idea of exaggeration and the retention of the concept of modernity. Regarding the first point, Augé talks of three excesses: factual overabundance (which is associated with an acceleration of historical time), spatial overabundance (the abolishing of distance by electronic media and transportation), and an excess of self-reflexive individuality.} A. González-Ruibal, \quotation{Supermodernity and Archaeology,} in {\em Encyclopaedia of Global Archaeology}, ed.~C. Smith (New York: Springer, 2014), 7125.} The ease of posting to such social media spaces, versus crafting and publishing a blog post, and the near-instant rewards of social media likes and comments (themselves also often easier to enact than blog commentary) have, understandably, made the work of blogging less attractive. Many of the bloggers Laughlin mentions in his {\em Global Voices} overview no longer maintain their sites or have slowed their posting activity considerably.

The blog posts and corresponding comments that remain available to us today are evidence of the active network of Caribbean bloggers that existed before the decline in the 2010s. Some of James's posts reference Laughlin, Geoffrey Philp, or Annie Paul as inspirations (or instigators) for the content, particularly for lists or challenges such as in \quotation{Five Note Mojo} and \quotation{Five Songs I Must Have on My i-Pod.}\footnote{See Marlon James, \quotation{Five Note Mojo,} {\em Marlon James: Among Other Things\ldots{}}, 11 February 2007, \useURL[url12][http://marlon-james.blogspot.com/2007/02/five-note-mojo.html]\from[url12]; and \quotation{Five Songs I Must Have on My i-Pod,} {\em Marlon James: Among Other Things\ldots{}}, 13 March 2008, \useURL[url13][http://marlon-james.blogspot.com/2008/03/five-songs-i-must-have-on-my-i-pod.html]\from[url13]. There are currently no comments visible on James's blog, but the content of some posts indicates that they were open to comments at the time, and some of these comments are available on Internet Archive captures of the site.} Although Chin, blogging as she did earlier in the decade, was somewhat outside this particular network, the comments on her cyberjournal and her own use of hyperlinks indicate that she was similarly navigating an online community.\footnote{See Murray-Román, \quotation{Staceyann Chin and Zoe Valdés,} for more on Chin's usage of hyperlinks in her blog.} James had been very much entrenched in a more Caribbean digital network. Even the beginning of his personal blog seems to have stemmed from a playful taunt by Laughlin in his 2006 {\em Global Voices} post \quotation{West Indian Literature Online.} When listing current bloggers in the \quotation{small, vibrant, and growing literary sector in the Caribbean blogosphere,} Laughlin prodded: \quotation{And maybe Marlon James---whose first novel, {\em John Crow's Devil}, was nominated for both a Commonwealth Writers' Prize and a Los Angeles Times Book Prize---will start a \quote{real} blog one of these days---meanwhile, he's been keeping a so­called \quote{plog} over at Amazon.com, where he's been writing about, among other things, Jean Rhys and literacy in Jamaica.}\footnote{See Nicholas Laughlin, \quotation{West Indian Literature Online,} {\em Global Voices}, 28 April 2006, 2006, \useURL[url14][https://globalvoices.org/2006/04/28/west-indian-literature-online]\from[url14].} On 29 May, James commented on Laughlin's post: \quotation{Alright! Alright! Damn it! I will set up my blog and stop with this \quote{plogging} business right away.} And the first post on his personal blog is dated 30 May. From posts and comments like these---on sites such as {\em Global Voices} as well as on popular personal blogs such as Annie Paul's {\em Active Voice} and Geoffrey Philp's blog---we can trace the ways this Caribbean digital network shaped many of the blog posts published.\footnote{See Annie Paul, {\em Active Voice} (blog), \useURL[url15][https://anniepaul.net]\from[url15]; and Geoffrey Philp, \useURL[url16][https://geoffreyphilp.blogspot.com]\from[url16].} Reading across blog posts and comments, we become privy to some of the workings of these networked influences: the conversations that were public but hidden in the comments section and those that were conducted privately via emails and face-to-face dialogues but then hinted at in comment threads. These exchanges are articulated into the blog posts that are eventually crafted for the public, representing individuals navigating a community of digital and nondigital encounters as they write themselves into subjectivity. Even as the size and very existence of an audience may be questioned, this blogging constitutes a writing against invisibility and into community.

Paradoxically, it is this community that also works against strategic invisibility. Deletion, whether deliberate or accidental, is never guaranteed to be complete when there are so many potential avenues of reference to the deleted material. As it is, what remains available from the blogs of this time indicates that even in focusing on only Chin and James, I am working with not just a partial but a palimpsestic and changeable archive of revised and in-progress autobiography, an archive constructed of simultaneously too much and too little material. Much of the material is no longer available because older portions of blogs, or in some cases entire sites for them, have been neglected or removed by the authors as they progressed to other platforms and to other selves. And the material we do have some access to can be overwhelming in scope (neither Chin nor James are known for their reticence or brevity). Still, less than fifteen years after Chin closed her cyberjournal, and ten years after James stopped posting on his blog, there is significant and traceable evidence of what might be considered archival loss. Although I have a sizeable portion of Chin's early cyberjournal entries saved in hard copy, and the Internet Archive provides additional access through 2006, the blog is no longer available on her website. When I began this project, I felt an urgency to capture James's 2006--9 blog before he decided it was not in keeping with his post--Booker Prize self, but I later learned that sometime between 2007 and 2010 he had already significantly revamped the blog to reflect his post--{\em Night Women} self. Such archival gaps are often strategic, and in many ways they are built into each author's initial technological choices: James used BlogSpot, which allows for (his) deletion of all comments; and Chin's site was built on Adobe Flash technology that is no longer supported online. In a 2003 interview I conducted with Chin, she revealed, \quotation{The kind of half-poems that exist in my journal never go anywhere else. There are no copies of them anywhere else. If my website crashes\ldots{}that's it.}\footnote{Kelly Baker Josephs, \quotation{Ventriloquising the Caribbean: Interview with Staceyann Chin,} {\em Jamaica Journal} 30, no. 3 (2007): 31.} Of course, that never really is \quotation{it} when one participates in digital community, but the threat and allure of deletion remain integral parts of writing the self online. Revision and deletion, the opportunity to take back information---inasmuch as one can take back anything from the internet---offers a blogging safety net. This is particularly important for writers like Chin, who occupy vulnerable social and professional positions: black, female, migrant, queer, un(der)published, un(der)employed.

Now, almost two decades after Chin began her cyberjournal, internet users are more familiar with this paradox of the permanent-but-ephemeral archive of digital writing.\footnote{I am thinking here of Sonya Donaldson's forthcoming essay \quotation{The Ephemeral Archive.} But with autobiographical blogs, particularly queer Caribbean blogs, as I note later in the paragraph I am uneasy with the \quotation{degree of vigilance and labor} that Donaldson recommends as archival practice, in the form of \quotation{careful annotation, screenshotting, downloading, storage and backup.} See Sonya Donaldson, \quotation{The Ephemeral Archive: Unstable Terrain in Times and Sites of Discord,} in {\em The Digital Black Atlantic}, ed.~Roopika Risam and Kelly Baker Josephs (Minneapolis: University of Minnesota Press, forthcoming).} But what did digital permanence mean for bloggers when the genre was relatively new and search engines were limited? For this reason, I have been selective in my use of materials from Chin's and James's digital writings. I recognize that my own archiving of posts that were perhaps created with ephemerality in mind may be counter to the safety net that I argue blogging provided in those early years. At the same time, I want to convey the spirit of this moment and the positions and possibilities of enunciation blogging offered queer Caribbean writers with access to digital platforms. When I asked Chin in 2003 why she chose to start sharing her life online, she responded,

\startblockquote
A lot of white straight history is allowed into the canon. Even the idea of the canon is very white and straight and I feel like the web gives us the place where we can write our own histories. Whether they're relevant or irrelevant. I don't send out emails to people with my cyberjournal entries in them; I chose not to do that for a reason. So it is only the ones who want to come. And that's what I like about it.

The net gives you a way to grow maybe longer and wider fingers.\footnote{Josephs, \quotation{Ventriloquising the Caribbean,} 31.}
\stopblockquote

Reaching with these longer and wider fingers for what audience may come, Chin and James equivocally thrust their stories of self into the public domain. Frequently expressing doubt about the venture, they nevertheless continued to write themselves, pushing against canonical expectations while creating digital space for others in a collective of Caribbean voices. As Sandra Pouchet Paquet writes, \quotation{Self-focus enhances rather than diminishes authority; it serves to establish the author's credibility in a cultural space of his own creation.}\footnote{Paquet, {\em Caribbean Autobiography}, 137.} With their self-focused writings, Chin and James propelled themselves into a literary tradition while also wrestling with the myopia of that tradition. Taking advantage of turn-of-the-twenty-first-century technologies that promised freedom even as they withheld ultimate control, Chin and James insisted on the relevance and credibility of queer and Caribbean and queer Caribbean lives. In this place where they could write their own histories, they created and deleted, reproduced and revised, refusing to let their self-narratives be fixed.

\subsection[title={Acknowledgments},reference={acknowledgments}]

Endless gratitude to Nicholas Laughlin, without whose generosity, then and now, I could not have written this essay. I am also indebted to the following people for their critical readings of various early drafts: Matthew Chin, Ronald Cummings, Tzarina T. Prater, and the editors and anonymous peer reviewers for {\em archipelagos}.

\thinrule

\page
\subsection{Kelly Baker Josephs}

Kelly Baker Josephs~is a professor of English at York College, City University of New York, and a professor of English and digital humanities at the CUNY Graduate Center. She is the author of~{\em Disturbers of the Peace: Representations of Insanity in Anglophone Caribbean Literature}~(2013), the founder and former editor of~{\em sx salon: a small axe literary platform}, and the manager of~\useURL[url1][http://caribbean.commons.gc.cuny.edu/][][{\em The Caribbean Commons}]\from[url1]~website.

\stopchapter
\stoptext