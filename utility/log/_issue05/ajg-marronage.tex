\setvariables[article][shortauthor={Joseph-Gabriel}, date={December 2020}, issue={5}, DOI={https://doi.org/10.7916/archipelagos-ar70-sf84}]

\setupinteraction[title={Mapping Marronage},author={Annette Joseph-Gabriel}, date={December 2020}, subtitle={Mapping Marronage}]
\environment env_journal


\starttext


\startchapter[title={Mapping Marronage}
, marking={Mapping Marronage}
, bookmark={Mapping Marronage}]


\startlines
{\bf
Annette Joseph-Gabriel
}
\stoplines


{\em Mapping Marronage} is a fundamentally engaging idea for arranging archival materials that will certainly frame new conversations in Caribbean studies.\footnote{See {\em Mapping Marronage}, \useURL[url1][http://mapping-marronage.rll.lsa.umich.edu/]\from[url1].} With some additional attention to interface, primarily in service of enhancing the methodological innovations the project hints at, certain of the platform's stated intentions might be more robustly in evidence. The comments that follow offer suggestions for ways the site might better fulfill its promise as an archival resource that, in its use, produces new kinds of knowledge.

\subsection[title={Contribution},reference={contribution}]

Insofar as the site ultimately grounds its archival claims in materials from outside the site, the project's specific contribution is necessarily grounded in how it asks users to navigate archival materials differently. The site points to the code base of Alex Gil and Kaiama L. Glover's mapping visualization \useURL[url2][http://sameboats.org/][][{\em In the Same Boats}]\from[url2] as structural blueprint.\footnote{See {\em In the Same Boats}, \useURL[url3][http://sameboats.org/]\from[url3].} While the {\em Same Boats} ({\em SB}) interface is a great one, in the case of {\em Mapping Marronage} ({\em MM}), claims regarding multiple kinds of flight need more elaboration, since the claims (as stated) by {\em MM} are not quite the same as those made by {\em SB}. The {\em MM} site seems to offer layers of experiences, propose engagement with evidential documentation of movements, and call for attention to ephemera. To more fully actualize these elements would make for a significant contribution.

It is always crucial to think about people's movement as an index of their lived experience. Much like {\em SB}, {\em MM} helps elicit a sense of discovery in how rich a single life can be, especially when put in relation to/network with the lives of others. But, again, there is great potential here to complicate more dramatically the very idea of mapping itself. What, for instance, might this project present for a person who never leaves a very tight geographic radius but who fits one of this site's multiple definitions of {\em marronage}?

\subsection[title={Design},reference={design}]

Is the site's ultimate intent to map people's movements, or to map the life/transaction of things they produce? Is it to map individuals or their traces, or to conceive of one as the index of the other? These queries, engendered by {\em MM}, have so much potential, but the intent needs to be more consistently framed across the site---on the level of the sentence and, at the same time, in design terms. The site's creator (rightfully) wants {\em marronage} to encompass phenomena on multiple registers. But because this is a hands-on experience, more might be done to align those various goals with visually differentiated design cues---for instance, color or icon. As it stands, every kind of history represented on the site looks and feels the same, which undermines the central tenet that marronage encompasses multiple kinds of experiential states (e.g., physical or psychological fleeing, ephemeral feeling), traces (written correspondence versus other ephemeral knowings), and so on.

One suggestion might be to articulate more fully what the site can actually show and to offer examples of what a user should be able to determine or intuit from that data, thus demonstrating the stakes of the data presented in the map interface, for instance. Also, more explicit language might be included so as to add nuance to the specific uses of {\em marronage} throughout the site.

An additional design suggestion would be to consider opening links in pop-up windows, with a frame that includes the full citation for the given resource. Though not always the prettiest solution, an i-frame would help produce a more seamless user experience, while maintaining the integrity of the site. Lastly, the maps need a zoom function for political map capacity. One wonders, further, whether other kinds of data visualization might better represent some of networks involved. There are ways the idea of the world map risks discursively limiting the kinds of connection and overlap the site means to illustrate.

\subsection[title={Credit},reference={credit}]

It remains unclear---unstated in any explicit way---how to understand the roles of contributors to the project or the labor conditions under which the project was produced, though one can glean that this is a single-authored project developed by a professional design firm. A bit more information concerning conceptual provenance and funding would be welcome on the \quotation{About} page, for example. Although all documents in the projects are at external links, these hyperlinks should not replace citations for those sources. A \quotation{Works Cited} page, organized by who is featured on the site, with all sources for each person listed beneath her or his name, might be an elegant solution. Finally, the site author might consider including more information on the main page about each linked document; it is currently rather difficult to use because everything looks the same (multiple items are named \quotation{correspondence,} for example).

\subsection[title={Preservation},reference={preservation}]

Currently, no information is provided regarding the life cycle and or preservation of the project.

\subsection[title={Conclusion},reference={conclusion}]

On the whole, {\em Mapping Marronage} is an exciting project that has the potential to offer both substantive and methodological innovations to the study of the colonial black Atlantic. A valuable resource for both research and pedagogy, the site's curated content approaches the lives of enslaved peoples in the eighteenth and nineteenth centuries from the perspective of movement, network, and exchange. In so doing, it usefully refigures the intersecting landscapes of the colonial Americas.

\subsection[title={{\em Mapping Marronage}: Response},reference={mapping-marronage-response}]

Annette Joseph-Gabriel

Given the debates of our present moment about the exclusionary and often revisionist stories that statues tell about the past, the irony is not lost on me that {\em Mapping Marronage} began at the base of a monument. The imposing equestrian statue of Henry Christophe in Port-au-Prince memorializes Christophe as he was in his later years: a revolutionary leader in the Haitian fight for liberty. But the narrative told by tour guides at the site begins with his early life in slavery in Grenada. Standing beneath the monument to Christophe on a hot afternoon in June 2016, reminded of his Grenadian origins and of the enslavement of Boukman Dutty (another early leader of the Haitian Revolution) in Jamaica, I was struck by the geography of remembrance. The monument to Christophe in Haiti serves as a reminder that the fixed location of seemingly immovable statues of marble or granite or bronze belies the fluid, mobile, entangled stories of personhood and self-determination in the Atlantic world.

{\em Mapping Marronage} is neither monument nor commemoration, though it does seek to retrace and remember the movement of enslaved people within and beyond the Atlantic world. The primary goal of the digital map is to visualize enslaved people's flights from captivity as processes, as lived realities, and as relationships to space and power in slaveholding societies. Because this vision of mobility is so expansive, it requires different kinds of maps. The site therefore offers two kinds of visualizations---flights and networks. The \quotation{\useURL[url4][http://mapping-marronage.rll.lsa.umich.edu/flight][][Flight]\from[url4]} section is a visual rendering of movement through space, movement that was deliberate and haphazard, that doubled back on itself and created networks and communities that were defined by enslavement but also imagined life otherwise. A focus on travel, however, precludes the experiences of those who lived and died on a single plantation but whose influences, stories, ideas, and myths extended beyond the tightly circumscribed sphere of their captivity. Indeed, as the project's reviewer asks, \quotation{What, for instance, might this project present for a person who never leaves a very tight geographic radius but who fits one of this site's multiple definitions of {\em marronage}?} \quotation{\useURL[url5][http://mapping-marronage.rll.lsa.umich.edu/networks][][Networks]\from[url5]} attempts to account for these varied forms of mobility. Pierre Toussaint, for example, moved only once, from Haiti to New York. Yet his philanthropic works, economic transactions, correspondence, and family networks extended far beyond New York City. His story, like that of Charles Pierre Lambert, whose network shows what my students have termed \quotation{intergenerational mobility,} is a story about Atlantic bonds that often included fraught intimacies between enslaved people and their enslavers.

The {\em Mapping Marronage} digital map was built by the Agile Humanities Agency, whose members have been wonderfully thoughtful and communicative collaborators. Much of the data on the map was generated by undergraduate research assistants and students in my course Mapping the French Atlantic. I am still working through best practices for attribution that is attentive to questions of student data privacy and that takes into account the ephemeral nature of undergraduate researchers' involvement with the project. The Colored Conventions Project offers a useful model for crediting collaborators that I intend to adapt for this purpose. Because {\em Mapping Marronage} tries to account for varied forms of enslaved people's mobility whose meanings are not always readily decipherable, the data points on the map are accompanied by text that briefly summarizes the nature and circumstances of movement. In my mapping course, I encouraged my students to incorporate direct quotes from the narratives they were working with, as one way to think about which voices could be amplified or become more audible through this map. Each point on the map also opens out further onto a digitized version of the relevant archival source. As the reviewer aptly points out, it is perhaps best to think about {\em Mapping Marronage} as layers: layered accounts of layered lives. The project's envisioned users are students, teachers, and researchers who want to work with visual and spatial renderings of enslaved people's narratives. The site's future iterations have these audiences in mind, and the reviewer's helpful and generative suggestions provide clarity on the way forward. I particularly appreciate the nudge to consider more robust citation, and so the updated site will include a list of works cited, with hyperlinks to the archival sources.

The reviewer also notes, more broadly, that the site does not yet show all that it promises to show. The user interface currently does not show differences in the kinds of movement that are encompassed in the project's expansive understanding of {\em marronage}. To rectify this, the reviewer suggests reworking the framing text on the site to be more transparent about its limits, to state what the site can actually show {\em for now}. I am struck by this deceptively simple but really profound remedy, because I have always approached digital work from the opposite direction, by laying out my aspirations, the hopes and dreams that underlie the project, as it were, and then allowing those goals to unfold over time as I add to the project. But there is something to be said, here, for the stakes of this particular work and for the need to more accurately align the framing narrative of {\em Mapping Marronage} with its functionality as part of a practice of being accountable to the story the site seeks to tell. For this reason, I am deeply grateful to the reviewer for this opportunity to think otherwise, to begin not from the end of the project and feeling my way toward a goal but from the beginning. To implement this suggestion, I plan to rewrite the site's framing narrative and to divide the \quotation{\useURL[url6][http://mapping-marronage.rll.lsa.umich.edu/about][][About]\from[url6]} section into two parts. The first part will give an overview of the project's overarching aims. The second will walk the user through which of these aims are attainable in the moment, therefore functioning as part tutorial, part framing essay. It is my hope that providing users with a clearer vision of what is possible and what is not (yet) will open up avenues for user feedback on ways the project might evolve to live up to its promise of a multilayered visualization of marronage.

I hope that as users navigate the map, it becomes clear that this project does not see the lives of enslaved people as mere data points. However, spatial renderings, as the reviewer rightly notes, can flatten complexity. The material conditions of doing digital work largely account for this: more complex sites are more expensive to build. In an ideal world where funds are plentiful and equitably distributed, the trajectories on {\em Mapping Marronage} would be color coded by category, allowing users to easily distinguish between travel, correspondence, and legal testimony. In my ongoing conversations with Agile Humanities, we also envision adding texture, as the reviewer recommends, particularly to show archival uncertainty---solid lines will show movement whose details, such as start and end dates, are fairly certain; dashes will represent the archival breaks, the silences and ruptures in narratives that tell of movement that was concealed and subversive.

While the expense associated with building a more complex site explains some of the map's flattening, it does not account for all of it. Telling stories about slavery requires us to grapple with the ethics of speaking the unspeakable and of recounting often unimaginable horrors. How do we visualize the journeys of those African captives on slave ships who were jettisoned overboard during the Middle Passage? Would a dot on the map in the middle of the Atlantic Ocean mark loss, or would it replicate violence? {\em Mapping Marronage} began at a site of commemoration, Henry Christophe's statue in Port-au-Prince, and the ethics of commemoration continue to haunt this project as it evolves. I extend my sincerest gratitude to the reviewer for these productive new directions.

\thinrule

\page
\subsection{Annette Joseph-Gabriel}

Annette Joseph-Gabriel is a scholar of\ldots{}

\stopchapter
\stoptext