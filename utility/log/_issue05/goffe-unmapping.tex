\setvariables[article][shortauthor={Goffe}, date={December 2020}, issue={5}, DOI={https://doi.org/10.7916/archipelagos-72th-0z19}]

\setupinteraction[title={Unmapping the Caribbean: Toward a Digital Praxis of Archipelagic Sounding},author={Tao Leigh Goffe}, date={December 2020}, subtitle={Unmapping the Caribbean}]
\environment env_journal


\starttext


\startchapter[title={Unmapping the Caribbean: Toward a Digital Praxis of Archipelagic Sounding}
, marking={Unmapping the Caribbean}
, bookmark={Unmapping the Caribbean: Toward a Digital Praxis of Archipelagic Sounding}]


\startlines
{\bf
Tao Leigh Goffe
}
\stoplines


{\startnarrower\it Tackling the conceptual grounds of how maps have been deployed as tools of imperial capitalist extraction, this essay critiques how the two-dimensional visualization of land has traditionally flattened the racial entanglement of the Caribbean archipelago. It explores how born-digital cartography can be used to open up a new sensory possibility for understanding space amplified by sonic and video technologies. The author embarked on the digital project {\em Unmapping the Caribbean} with her students and a team of technologists, employing Esri's ArcGIS Story Maps platform to examine the contours of {\em marronage} and indigeneity in five geographies: New York City, Suriname, Hispaniola, Cuba, and Jamaica. The project anchors the relationality of sedimented racial histories in the archipelago through the concept of \quotation{unmapping}---that is, by creating digital audiovisual story maps of the five Caribbean spaces. Centering the way sound orients the human body in space, the project examines the politics of the opacity of spaces of black and indigenous refuge in Caribbean and Caribbean diasporic communities. The essay considers the pedagogical stakes of collaborative digital assignments that include mapmaking and producing visual soundtracks. The five geographic narratives that {\em Unmapping the Caribbean}~has woven together are part of an ongoing project that aims to articulate archipelagic being. \stopnarrower}

\blank[2*line]
\blackrule[width=\textwidth,height=.01pt]
\blank[2*line]

\startblockquote
You will find a way.
\stopblockquote 

\startalignment[flushright]
\tfx{Jamaican police officer on motorbike giving directions in Mandeville, 2014}
\stopalignment
\blank[2*line]


The literary and sonic traditions of the archipelago are equally critical to Caribbean study. The musical soundtrack of island life is inseparable from studying the radical political tradition of the region, and I center this approach in a course I have been teaching since 2015 titled Caribbean Writing, Reggae, and Routes. Vernacular oral traditions trace the roots and routes of the Caribbean and its diaspora to Africa, Asia, and Europe. Each year that I teach the course, music and the sounds of natural and built environments become more essential to the argument of Caribbean critical theory and much more than an accompaniment to novels and poetry featured on the syllabus. In the course, which is both born-digital and born-musical, students annotated literature and songs using online platforms in the first iteration in 2015.\footnote{Students use Genius (formerly RapGenius) and SoundCloud. See \quotation{About Genius,} Genius, \useURL[url2][https://genius.com/Genius-about-genius-annotated]\from[url2]; and \quotation{About SoundCloud,} SoundCloud, \useURL[url3][https://soundcloud.com/pages/contact]\from[url3].} The course has become more digitally and musically engaged as I have encouraged students to produce their own soundtracks, embedding them in maps in a practice I call \quotation{sonic un/mapping.} This multisensorial method of reading the Caribbean archipelago attunes a spatial awareness. Sonic analysis requires attention to the spatial because sound is quite literally how our ears perceive the vibrations that bounce off our environment. As such, my sonic praxis has become amplified by digital geographic technologies.

As a professor and a DJ, a \quotation{PhDJ,} I see the role of an educator as one intimately tied to that of a DJ. The \useURL[url4][https://caribbeanliterature.princeton.edu/][][syllabus]\from[url4] is a mixtape---I narrate the progression of texts on a syllabus, almost toasting, to introduce and string together a line of thought about Caribbean artistic and political expression.\footnote{For the course syllabus, see \useURL[url5][https://caribbeanliterature.princeton.edu/]\from[url5].} Reggae songs are integral, and each week I ask students to choose a song connected to the readings on the syllabus to form a collective weekly playlist, housed on our course blog. Students are required to write a few sentences explaining the reasoning for their choice, be it thematic or affective.

Recognizing the limitations of close reading, in how it assumes a text has an inherent transparency, we instead perform a practice of deep listening. This involves a multisensorial receptivity through the practice of what Audre Lorde calls \quotation{deep participation.}\footnote{Audre Lorde, \quotation{Uses of the Erotic: The Erotic as Power,} in {\em Sister Outsider: Essays and Speeches} (Berkeley, CA: Crossing, 1984), 59.} It also involves the receptivity that Julian Henriques describes as a spatial way of knowing, foundational, for instance, to dancehall culture in Jamaica.\footnote{See Julian Henriques, \quotation{Preamble: Thinking Through Sound,} introduction to {\em Sonic Bodies: Reggae Sound Systems, Performance Techniques, and Ways of Knowing} (New York: Continuum, 2011), xv--xxxvi, \useURL[url6][http://research.gold.ac.uk/id/eprint/4257/1/HenriquesSonicBodiesIntro.pdf]\from[url6].} Attention to sound and where it places the body in space is an intimate part of what I am calling \quotation{unmapping.} For example, Little Roy's sonic exegesis in the song \quotation{Christopher Columbus} poses a challenge to coloniality and Western historiography. Little Roy describes the multiple voyages of Columbus to the Caribbean---\quotation{Him come again and go away}---and says that Natty Rasta, the black man, was metaphysically in the Caribbean long before Columbus.\footnote{Little Roy and Ian Rock, \quotation{Christopher Columbus,} 45rpm single, Tafari, 1975.} Politicizing Afro-Jamaican identity in ways that engage with what Jamaican theorist Sylvia Wynter has conceptualized as {\em indigenization}, Little Roy \quotation{unmaps} the Caribbean, challenging the fixity of Columbus's colonial map through a sonic and lyrical reordering of time and space.\footnote{On {\em indigenization}, see Sylvia Wynter, \quotation{Jonkonnu in Jamaica: Towards the Interpretation of Folk Dance as a Cultural Process,} {\em Jamaica Journal} 4, no. 2 (1970): 34--48.}

In our sonic un/mapping in my course, I ask students to think about how they would map Babylon as a geographic, historic, and imagined place in the context of reggae lyrics. Similarly, I ask them to think about the intertextuality of reggae and Rastafari to the Hebrew Bible and the Christian Testament. Where is the Zion that Kiddus I sings of in \quotation{Graduation in Zion}? We also explore urban spatial politics through the more literal and lyrical mapping of Kingston that Bob Marley performs in \quotation{Natty Dread} by walking through First Street and Second Street, and so on. Then we look to his son Damian Marley's survey of the island in \quotation{Welcome to Jamrock,} and how he sings a refrain that hails each county of Jamaica---Cornwall, Middlesex, and Surrey---delineating a geography that the island postcolony reclaims from the English counties it mirrors.\footnote{Kiddus I, \quotation{Graduation in Zion,} Shepherd, 1977; Bob Marley and the Wailers, \quotation{Natty Dread,} on {\em Natty Dread}, Island Records, 1974; Damian \quotation{Jr.~Gong} Marley, \quotation{Welcome to Jamrock,} on {\em Welcome to Jamrock}, Universal Records, 2005.}

Reggae and dub are as much about unfinished spatial mapping as about cosmic awareness and cosmological mapping. Music is the argument of unmapping. Sound, like light, is energy comprised of waves, a series of perceived reflections and refractions that orient us in time and space. Sound reflects our relation to other bodies and objects. This sonic data is central to a poetic strategy that I call \quotation{decolonial echolocation}---a method of bodily knowing, a somatosensory epistemology of embodied navigation of coloniality for those who were not mean to survive. It is the diasporic physiological process of sensing distant or invisible objects by sound waves emitted back to the emitter, to draw on the dictionary definition of {\em echolocation}.

The map is engraved with the echo of runaway desire, of refusal, in place names like the District of Look Behind, Don't Come Back, and Quick Step. A decolonial register would imply a strategy of detecting the colonizer and finding spaces of refuge. This refuge is encoded in both reggae music and Rastafarian ontology. Rastafari, which Stuart Hall describes as an \quotation{imagined community} formed with an Africa that has moved on but is nevertheless symbolically reconstructed in music, is a form of worlding and navigation beyond coloniality, never quite intended as a literal return to Africa. Echolocation resonates with music as integral to the archipelagic being of Jamaican reggae artist Chronixx when he sings of \quotation{capture land.} It is the replantation of the provision grounds that Wynter describes at the periphery of the colonial plantation.\footnote{See Stuart Hall, \quotation{Negotiating Caribbean Identities,} {\em New Left Review}, no. 209 (January--February 1995): 11; Chronixx, \quotation{Capture Land,} on {\em Dread and Terrible}, Soul and Circle Music, 2014; and Sylvia Wynter, \quotation{Novel and History, Plot and Plantation,} {\em Savacou}, no. 5 (June 1971): 95--102.} The echoes that reverberate through the karstic mountainous terrain of Jamaica's Cockpit Country and Blue Mountains are an intimate part of Caribbean geography that narrates the meaning of refuge for black and native peoples on the island.

To geolocate Caribbean space through literature and music is to chart archipelagic being. It decenters ocularcentric forms of knowledge production and the written word. Archipelagic being attunes one to the power of {\em other} senses to orient oneself in space. Sound---in particular, how we receive certain frequencies---engages the sensorium in a way that is central to a Caribbean vestibular sensibility.

Unmapping as a digital praxis embraces the opacity of sensory disorientation. If mapping is about knowing and surveying, then unmapping is about unknowing and uncharting the territory. To welcome not knowing allows one to be receptive to other forms of knowledge. The West Indies are, after all, a misnaming, so why continue to depend on a Columbian geographic misrecognition for reading the archipelago? The colonial map obscures this dialectic between Wynter's indigenization and Kamau Brathwaite's creolization by enforcing European geography on the archipelago.\footnote{See Edward Kamau Brathwaite, {\em Contradictory Omens: Cultural Diversity and Integration in the Caribbean} (Mona: Savacou, 1974).} Toward unmapping, I challenge my students to consider a spatial politics centering spaces of Caribbean refuge carved out by indigenous peoples and Maroons across the archipelago. Some clues are embedded in place names, but again, music becomes central, especially for examining the hidden African presence indigenized in the landscape. Unmapping therefore challenges the myth of extinction in favor of intentional Caribbean opacity, in the Glissantian register.\footnote{Édouard Glissant argues for \quotation{the right to opacity}; see his {\em Poetics of Relation}, trans. Betsy Wing (Ann Arbor: University of Michigan Press, 1997).}

The Caribbean archipelago was formed through a process of layering that I have elsewhere described as \quotation{racial sedimentation.}\footnote{See Tao Leigh Goffe, \quotation{\quote{Guano in Their Destiny}: Race, Geology, and a Philosophy of Indenture,} {\em Amerasia Journal} 45, no. 1 (2019): 27--49{\em .} See also Tiffany Lethabo King, {\em The Black Shoals: Offshore Formations of Black and Native Studies} (Durham, NC: Duke University Press, 2019); Manu Karuka, {\em Empire's Tracks: Indigenous Nations, Chinese Workers, and the Transcontinental Railroad} (Berkeley: University of California Press, 2019); and Iyko Day, {\em Alien Capital: Asian Racialization and the Logic of Settler Colonial Capitalism} (Durham, NC: Duke University Press, 2016), all of which rigorously tackle the relationality of settler colonialism in forms that decenter whiteness.} I argue that the sedimented presences settle and become transformed in metamorphic rock in accordance with the geological metaphor of the rock cycle.\footnote{Trinidad and Tobagoan-born Canadian author Dionne Brand, for instance, takes up these questions in {\em A Map to the Door of No Return}, and Canadian scholar of black geographies Katherine McKittrick writes of {\em demonic} grounds, invoking Wynter's provision grounds. See Dionne Brand, {\em A Map to the Door of No Return: Notes to Belonging} (Toronto: Vintage Canada, 2001); and Katherine McKittrick, {\em Demonic Grounds: Black Women and the Cartographies of Struggle} (Minneapolis: University of Minnesota Press, 2006).} The materiality of bodies, bones, and decomposing biomatter have become part of the geology of the archipelago over the centuries since Columbus's invasion. Mutilation has made the Caribbean the site of, according to Hortense Spillers, a \quotation{human sequence written in blood.}\footnote{Hortense J. Spillers, \quotation{Mama's Baby, Papa's Maybe: An American Grammar Book,} {\em Diacritics} 17, no. 2 (1987): 67.} Maroon strategies of warfare, especially guerilla warfare, against the British in Jamaica depended on the cover the rainforest provided, and locations in Cockpit Country are accordingly named to reflect these battles. I task my students with examining how the topography of the Caribbean---its crags, its valleys, its mountains, its caves, its ravines---has provided refuge. I wanted to design a digital platform that could represent distinct racialized copresences and coproductions enmeshed through settlement. If, as historian Vincent Brown writes, we ought to consider {\em marronage} as a species of West African warfare, then what are the political stakes of a map that speaks or sings in Maroon vernaculars?\footnote{See Vincent Brown, {\em Tacky's Revolt: The Story of an Atlantic Slave War} (Cambridge, MA: Bellknap, 2020).}

\subsection[title={Committing Treason with the Master's Tools},reference={committing-treason-with-the-masters-tools}]

In 2018, I embarked on the digital project \useURL[url7][https://nyuds.maps.arcgis.com/apps/MapSeries/index.html?appid=489f1aee6b324a75b709d5d37f0cea2a][][{\em Unmapping the Caribbean: Sanctuary and Sound}]\from[url7] with my students and a team of technologists.\footnote{See {\em Unmapping the Caribbean: Sanctuary and Sound}, \useURL[url8][https://nyuds.maps.arcgis.com/apps/MapSeries/index.html?appid=489f1aee6b324a75b709d5d37f0cea2a]\from[url8].} As a method, unmapping challenges the fixity of the map as a technology of colonial organization. As a medium, I used \useURL[url9][https://storymaps.arcgis.com/][][ArcGIS Story Maps]\from[url9],\footnote{See \quotation{ArcGIS Story Maps,} Esri, \useURL[url10][https://storymaps.arcgis.com/]\from[url10].} a cloud-based platform by international geographic information system (GIS) software supplier Esri (Environmental Systems Research Institute), and, in light of our decolonial mission, there were necessarily many limitations and ethical concerns with which to contend. Founded in 1969 as a land-use consulting firm, US-based Esri evolved into a GIS software vendor embedded in the history of US military defense digital technology, holding contracts with the US Department of Defense. As a global tech company, Esri perhaps typifies, to reference Audre Lorde's famous provocation, the \quotation{master's tools}---that is, the structure of the ongoing colonial present.\footnote{See Audre Lorde, \quotation{The Master's Tools Will Never Dismantle the Master's House,} in {\em Sister Outsider}, 110--13.} But I knew that the technologists helping with the project had the most proficiency with this platform because of the contracts Esri had with the university where I worked.

\useURL[url11][https://storymaps-classic.arcgis.com/en/][][Esri Story Maps]\from[url11] describes its educational mission as a strategy of digital storytelling by and through graphic organization.\footnote{See \quotation{Classic Story Maps,} Esri, \useURL[url12][https://storymaps-classic.arcgis.com/en/]\from[url12].} The platform hosts maps that users can design themselves using the Esri database. But a project need not actually be a map---the emphasis is on narrative, which is why I find it so appealing. Esri is essentially a platform for presenting graphic, multimedia information, into which ArcGIS maps can be easily integrated, if you so choose. Another reason I used Esri Story Maps is its accessibility---both in the ease of not needing to download or install a program, since it runs online, and in being able to reach it through mobile devices. Since starting the {\em Unmapping} project, I began a new job at a university, and while it also has a contract with Esri, the platform is not for use by large numbers of students. As such, I am in the process of devising new forms of unmapping that will engage XR (extended reality), which includes mixed reality, open-source AR (augmented reality), and VR (virtual reality) tools such as Unity. The digital amplifies the prospect for a cartographic synesthesia, which is why the potential of embedding digitized sound and video offers so many narrative and epistemological possibilities.

The transparency that military technologies prioritize for the purpose of warfare and seizing dominion is the antithesis of my goal for reading the archipelago. My digital praxis instead employs a cartographic approach to Glissantian opacities---that is, to what remains unknown and unknowable about the Caribbean. My hope is that modes of storytelling can be enhanced experientially through a digital sensorium, a range of audiovisual features that lead to an immersive experience. The body is a geography of inheritances, knowledges, and intuitions that military technologies will never comprehend. With its arteries, veins, and capillaries; its curves; its wrinkles---the body is a terrain, unmapped and unmappable. Thus as someone of Caribbean descent, I began with my body, literally using my hand to anchor the project. I sought to locate the affective mutual coordinates of the \quotation{intimacies} of Africa, Asia, and Europe in the Caribbean.\footnote{See Lisa Lowe, \quotation{The Intimacies of Four Continents,} in Ann Laura Stoler, ed., {\em Haunted by Empire: Geographies of Intimacy in North American History} (Durham, NC: Duke University Press, 2006).}

The first European map of the Caribbean, a 1492--93 drawing by Christopher Columbus, amounts to a few squiggles and splotches of ink (fig.~1). It approximates the curve of the coast of what he would name the Insula hispana, which Bartolomé de las Casas renamed Española. The island is the epicenter, the first place of disembarking, the cradle. I find Columbus's map inscribed with playful desire and possibility---as much a fantasy as a tool of navigation for further colonial conquest. The ways Haiti and the Dominican Republic are peripheral to the map of \quotation{America,} even though Columbus never stepped foot on the mainland, then become a conundrum of hemispheric mapmaking origins. In my process of unmapping, I was led to a later map of Hispaniola, dating to the mid-sixteenth century, that had been drafted by Italian geographer Giovani. Columbus's squiggles had become solidified in a cartographic imaginary practice that sought to \quotation{know} as much as it sought to claim the island for European powers.

\placefigure{West Indies, Christopher Columbus, 1492--93}{\externalfigure[images/goffe/fig1.jpg]}
For the {\em Unmapping} project, I challenged my students to unmap five Caribbean locations---Hispaniola, New York City, Suriname, Cuba, and Jamaica---through sound and attention to the body. I began the semester by juxtaposing analog and digital processes of mapping, asking each student to draw a freehand map of the Caribbean from memory. Most were exasperated by the challenge, realizing that they were far more familiar with the geography of Western Europe than with that of the Caribbean, despite the latter's more relative geographic proximity to the United States. Students who were of Caribbean origin expressed shame at having only \quotation{a sense} of the geography of their homelands.

\placefigure{Maps from Memory}{\externalfigure[images/goffe/fig2.jpg]}
Students began to question whether Guyana, Venezuela, and Bermuda are part of the Caribbean (I did not answer these questions until the activity was over and we had considered the multiplicity of definitions of archipelagic space as well as the imperial positionality implicit in our studying the region from New York). Some of the hand-drawn maps featured only Jamaica or Cuba; others depicted Haiti and the Dominican Republic not on Hispaniola but as separate islands (fig.~2). While many showed south Florida, none included New York City, where, as of 2013, roughly a quarter of the residents were of Caribbean origin.\footnote{Arun Peter Lobo and Joseph J. Salvo, \quotation{Growth and Composition of the Immigrant Population,} in {\em The Newest New Yorkers: Characteristics of the City's Foreign-Born Population} (New York: City of New York, Department of City Planning, and Office of Immigrant Affairs, 2013), 13, \useURL[url13][https://www1.nyc.gov/assets/planning/download/pdf/data-maps/nyc-population/nny2013/nny_2013.pdf]\from[url13].} No maps featured Suriname, the former Dutch colony that is part of the South American landmass; none of my students had even heard of the country, much less the language, Sranan Tongo, or the music, {\em kaseko}. For this reason, I chose to use my hand to visually represent our five disparate but joined geographies---five separate \quotation{fingers,} part of one \quotation{hand.} I projected Battista Ramusio's mid-sixteenth-century map onto my hand to juxtapose the materiality of the body and the epistemic violence the map inscribes on flesh. My palm anchored the inquiry into the archipelagic and served as a starting point for unmapping, a palmistry of wayfinding (fig.~3).

\placefigure{You Will Find a Way}{\externalfigure[images/goffe/fig3.jpg]}
I split the nineteen students into five groups and assigned each a location---I then challenged them to consider how each Caribbean space troubles the concept of the archipelago. The students exploring Suriname, for example, reckoned with the Dutch Empire, which often gets overlooked in the anglophone context. They contrasted the transparency of the paved roadways of the port city Paramaribo with the geographic opacity of the rainforest interior, parts of which are accessible only by small private planes. I taught them about how the Netherlands traded North New Amsterdam, now New York, for the South American New Amsterdam of Suriname because it was a sugar colony. New York, arguably a Caribbean city of the global North, is its own archipelago of forty-odd islands. On Hispaniola, the linguistic, cultural, and racial fracturings between Haiti and the Dominican Republic are as much a barrier as they are spaces of fluid musical exchange. Cuba has existed under a state of economic embargo and technological isolation that forms an opacity of exclusion by the United States, the communist nation's punishment for its refusal to acquiesce to the US geopolitical domination in the hemisphere. The students exploring Jamaica centered the global flows of reggae and how these are entangled with tourism and ecological degradation. Based on the students' research, I used Esri's \quotation{Map Series} template to stitch together five story maps to form one cohesive project with five tabs. Attention to the sensorium is amplified by the digital, insofar as multiple registers can be experienced at once on a story maps. It was important that the students understand how these disparate geographies connected to each other to form one archipelago, a chain of Caribbean/diasporic thought.

By anchoring these disparate geographies on my hand, I wanted students to consider the significance of the body in colonizing and mapping. I turned the image of the map superimposed on my hand into a postcard and wrote a message from the diaspora to an imagined addressee named X---a love letter of longing---which I then submitted to \useURL[url14][https://mina-loy.com/endehorsgarde/][][Post(card)s from the "en dehors garde]\from[url14]," a digital project on the feminist website {\em Mina Loy: Navigating the Avant-Garde} (fig.~4). The project collaborators had invited artists, writers, and scholars to write postcards expressing ideas about the {\em en dehors garde}, a term they had coined to describe the women, people of color, and others marginalized in or excluded from histories and theories of the avant-garde. They arranged seventy postcards on the website, curated together as a \useURL[url15][https://mina-loy.com/chapters/avant-garde-theory-2/digital-flash-mob/][][digital flash mob]\from[url15].\footnote{See \quotation{Post(card)s from the \quote{en dehors garde,}} MinnaLoy.com, https://mina-loy.com/endehorsgarde/. The digital flash mob is here: \useURL[url16][https://mina-loy.com/chapters/avant-garde-theory-2/digital-flash-mob/]\from[url16].} I later adapted my postcard as the landing page for {\em Unmapping the Caribbean}. Thus we began our joint unmapping quest---one professor, nineteen students, and three technologists.\footnote{The students whose assistance has been invaluable to {\em Unmapping the Caribbean} are Moriah Dowd, Zainab Floyd, Sophia Gumbs, Awura Gyimah, Harmony Hemmings-Pallay, Elliot Levy, Efosi Litombe, Naomi Malcolm, Matthew Martinez, Michelle Mawere, Kevwe Okumakube, Massiel Perez, Amodhya Samarakoon, Jesse Sgambati, Tatyana Tandanpolie, Mahalet Tegenu, Eva Toscanos, Hawau Touray, and Xiaolong Woods. Throughout the narrative I link them to specific parts of the project.}

\placefigure{Dear X}{\externalfigure[images/goffe/fig4.png]}
Once the base maps were created, students then meditated on global positioning system (GPS) technologies to consider how a Caribbean GPS might position our five locations. Geographic information systems (GIS) technologies such as Google Maps center individuals: for the first time in human history the center of the map is always the self. I asked students to grapple with what that could mean for a region like the Caribbean, which has been rendered peripheral for so long, despite its being the epicenter of Columbus's destruction. I also asked students to reckon with the power of opacity, of being off the bureaucratic map, of being undetectable, unmappable.

Western maps are often developed and explicitly deployed as tools of capitalist extraction invested in rendering colonial knowledge transparent, and also as ways of marketing, surveying, and making the land knowable and thus safe for tourists to consume. It is not incidental that gas station maps are a genre of petro-capitalist extractivism, considering the petrochemical infrastructure of Esso, Shell, and British Petroleum. These maps occlude the irrevocable ecological damage caused by Big Oil as well as the political violence of extraction and refinement of fossil fuels in Trinidad, Venezuela, and now Guyana and the ways these processes disproportionately threaten Maroon and indigenous communities. But there are other types of maps, even maps containing secret codes. There are Native Australian and Maori maps, for instance, made from tree bark and in the shape of animals, or maps that are guides to navigating waves at certain times of the day. Black and indigenous navigation knowledge production of this sort is precisely what I wanted my students to model in their design.

\subsection[title={Me-no-Sen-You-no-Come: On Opacity and the Impossibility of Mapping},reference={me-no-sen-you-no-come-on-opacity-and-the-impossibility-of-mapping}]

There are Caribbean spaces of refuge that evade the colonial order. I read this refusal---this Glissantian declaration of the right to opacity---in the place name Me-no-Sen-You-no-Come.\footnote{See Kei Miller, {\em The Cartographer Tries to Map a Way to Zion} (Manchester, UK: Carcanet, 2014). Miller's series of poems relay a conversation between a colonial mapmaker and a Rastafarian that includes musings on toponymy and on the narratives inscribed in the maps of the Caribbean. The attempt signaled by the word {\em tries} in the title is key: Miller's cartographer will never be able to map or reach Zion, a place of refuge and salvation as much for the Israelites as for Rastafarians.} This now-extinct village settlement was established and named by runaways fleeing enslavement on plantations in the parish of Trelawny in 1812.\footnote{See B. W. Higman and B. J. Hudson, {\em Jamaican Place Names} (Kingston: University of the West Indies Press, 2009).} Located in the District of Look Behind, the town thrived until emancipation in the mid-1830s. Me-no-Sen-You-no-Come came under siege by the British in 1824 and then by the Maroons of Accompong Town, yet the settlement survived with up to as many as sixty resident runaway peoples into the 1820s. It might at first seem that Maroons would have been natural allies for other runaways, but because of Maroon leader Cudjoe's 1736 land treaty with the British, Jamaican Maroons were supposed to return non-Maroon runaways to British plantations. To find Me-no-Sen-You-no-Come would require an archipelagic wayfaring and wandering, a kind of intellectual Glissantian errantry.

In many ways, just as the Caribbean refuses to be known to the colonial authority, it refuses to be mapped. It is an aqueous and geological geography of mangroves, caves, and coral reefs that cannot be surveyed and conquered entirely. I asked my students to embrace this impossibility.\footnote{During this time, I discovered my colleague Mimi Onuoha's course Impossible Maps at New York University, New York, New York, in which she is asking similar questions. See \useURL[url17][https://github.com/MimiOnuoha/Impossible-Maps]\from[url17].} I assigned excerpts of feminist scholar Macarena Gomez-Barris's {\em The Extractive Zone}, encouraging students to consider what should {\em not} be mapped. Gomez-Barris describes her work as proceeding from a decolonial femme methodology, \quotation{a mode of porous and undisciplined analysis shaped by the perspective and critical genealogies that emerge within these spaces as a mode of doing research{\em .}}\footnote{Macarena Gomez-Barris, {\em The Extractive Zone: Social Ecologies and Decolonial Perspectives} (Durham, NC: Duke University Press, 2017), xvi.} Gomez-Barris's meditation on opacity as a shield, a form of resistance to extractivist geospatial technologies, illuminated for students how mapping was a thorny endeavor, especially for indigenous communities of the Amazon. These technologies can never be truly disentangled from the visuality and transparency of coloniality.

Ethical mapping is a question of how to show the way to those with good intentions. Me-no-Sen-You-no-Come is unequivocal in this regard: If I do not send for you, you will never find your way here. The place name has its own opacity. Black feminist historian Tina Campt poignantly defines {\em opacity} as a type of refusal, different from resistance in that it can involve at-times-subtle tactics of performing illegibility and inaction as a form of defiance.\footnote{See Tina Campt, \quotation{Black Visuality and the Practice of Refusal,} {\em Women and Performance} 29, no. 1 (2019): 79--87, \useURL[url18][https://www.womenandperformance.org/ampersand/29-1/campt]\from[url18].} The place name Me-no-Sen-You-no-Come signifies a fleeting moment of freedom before freedom was formalized in 1834. The runaways who settled there mattered enough to be inscribed on the colonial map. In the patwa inflection of dashes in the village's name, I hear the shorthand rhythm of the lilt of \quotation{nation language,} of creole Caribbean tongues.\footnote{On \quotation{nation language,} see Kamau Brathwaite, \quotation{History of the Voice,} in {\em Roots} (Ann Arbor: University of Michigan Press, 1993), 259--304.} The land refuses in defiance. The toponymy commands. The symmetry and pace of the patwa phrase says something altogether mystical, distinct from what would be the literal English (mis)translation, \quotation{If I do not send for you, you do not come.} The Jamaican phrase is clear and concise. It declares how {\em you} can act toward {\em me}. It declares the right to not be touched, the right to be left alone.\footnote{See Hortense Spillers, \quotation{To the Bone: Some Speculations on Touch} (presented at the Gerrit Rietveld Academie conference \quotation{Hold Me Now---Feel and Touch in an Unreal World,} Stedelijk Museum Amsterdam, 23 March 2018), \useURL[url19][https://www.youtube.com/watch?v=AvL4wUKIfpo]\from[url19].}

My unmapping project was inspired by travels to the Caribbean. It is a place I will always feel disjointed from, having been born in the metropole, London. In 2017 I visited the Windsor Caves in Cockpit Country, not far from Me-No-Sen-You-No-Come, where I experienced the chilling majesty of thousands of bats cascading from the cave at dusk.\footnote{In 2017, I was hosted by Susan Koenig and Sugar Belly of the Windsor Research Centre, where I was able to lodge and from where I was led on a tour of bat caves in search of guano. Several years prior, I had researched guano in relation to Ian Fleming's 1958 James Bond novel {\em Dr.~No}, which centers on a guano island called Crab Key. See Tao Leigh Goffe, \quotation{007 versus the Darker Races: The Black and Yellow Peril in {\em Dr.~No},} {\em Anthurium} 12, no. 1 (2015), article 5, \useURL[url20][http://doi.org/10.33596/anth.280]\from[url20].} They flew past my face, dodging and diving elegantly, screeching, swiftly flying out of a cold cave, their space of refuge, a cathedral of shelter from the rays of sunlight during the day. Bats have an intimate way of knowing, of hearing, of navigating by ultrasound beyond the range of the human ear. Sylvia Wynter considers this fact in \quotation{Towards the Sociogenic Principle,} describing Thomas Nagel's description of the consciousness and being of bat sonar, the way the mammals form three-dimensional forward perception beyond human vocabulary. Bats are fundamentally alien to us, she says, and there is no reason to imagine they perceive the world as we do.\footnote{See Sylvia Wynter, \quotation{Towards the Sociogenic Principle: Fanon, Identity, the Puzzle of Conscious Experience, and What It Is Like to be \quote{Black,}} in Mercedes F. Durán-Cogan and Antonio Gómez-Moriana, eds., {\em National Identities and Socio-political Change in Latin America} (London: Routledge, 2001), 32.} The Maroons and the runaways had inhabited these caves alongside Caribbean wildlife. Through sound, in the dark, they had learned to navigate Cockpit Country in an intimate echolocation and engagement with nature.

The neurophysiology of bats is emblematic of an alternative way of knowing space beyond the ocularcentric. Once the binary between human and nonhuman animals is challenged, other sensorial epistemologies become possible. While I was at the Windsor Caves it occurred to me that during the time the Maroons and the runaways coexisted in interspecies spatial intimacy in these very caves, the bats would have signaled the time of day, the shifting of dawn and dusk; their flight, their movements would have signaled a diurnal fugitivity and anticircadian rhythm, another way of measuring time and navigating space through the nonhuman world. Sensory deprivation or perceptual isolation can heighten other senses. What did the Maroons and the runaways have to tune into for survival? What frequencies had they been attuned to that the British soldiers and planters would never be able to hear?

\subsection[title={{\em Unmapping the Caribbean}: The Website},reference={unmapping-the-caribbean-the-website}]

These questions of freedom, sovereignty, and futurity were what the students in the Jamaica group considered for their story map (fig. 5).\footnote{The students who assisted with the Jamaica story map include Moriah Dowd, Efosi Litombe, Tatyana Tandanpolie, and Mahalet Tegenu.} They titled it \quotation{Xaymaca: Wi Likkle but Wi Tallawah,} highlighting the island's original Arawak name Xaymaca, which means \quotation{the land of wood and water.} This choice of title honors the location's indigenous origin and juxtaposes it with the patwa of a Jamaican folk saying. Jamaican poet laureate Miss Lou riffed on the saying's significance and how it means that we might be a small place, but we are mighty in impact. This small-island sensibility is integral to archipelagic being. No map can accurately represent the size and impact of the Caribbean on the world, flattening, as maps do, the islands into a series of tiny dots.

The Jamaica group centered vernacular and folk knowledge with indigenous knowledge, taking a cue from University of the West Indies professor Carolyn Cooper's essay \quotation{Professing Slackness: Language, Authority, and Power Within the Academy and Without.} When they first read this essay, my students, even those who understood patwa, were disoriented, but they soon realized the metatheatrical stakes of writing a formal academic article in patwa. Cooper defied traditional academic norms to profess in what she describes as \quotation{barefoot language.} Again, the body and sexuality are invoked here in Caribbean orality against colonial power as Cooper explains there is a slackness that \quotation{isn't only about sexual morality} but rather about a \quotation{naked savagery} and a defiance of colonial norms, including wearing \quotation{proper} clothes.\footnote{See Carolyn Cooper, \quotation{Professing Slackness: Language, Authority, and Power Within the Academy and Without,} {\em E-misférica: Caribbean Rasanblaj} 12, no. 1 (2015), \useURL[url21][http://archive.hemisphericinstitute.org/hemi/en/emisferica-121-caribbean-rasanblaj/cooper]\from[url21].} To unmap through the accent of patwa offered the students a way to theorize sonic modes of declaring independence through the theatricality of Jamaicanisms.

\placefigure{Xaymaca: Wi Likkle but Wi Tallawah}{\externalfigure[images/goffe/fig5.png]}
Engaging further with Jamaican sound, the students contrasted sonic recordings (found on YouTube) of tourism in popular locations such as Dunn's River Falls with the sounds of nature---bird, bats, and frogs---in Cockpit Country. The accents of Americans and Western Europeans form a sonic colonialism that disrupts the local soundscape. The students found recordings of the Jamaican Maroon language Coromantee, derived from archaic Ghanaian languages, and embedded these into maps depicting sites of marronage in Accompong. Drawing on what they had learned about all-inclusive resorts and capitalist extraction from the pairing on the syllabus of Jamaica Kincaid's {\em A Small Place} and Stephanie Black's documentary {\em Life and Debt} (with voice-over of Kincaid's text adapted from {\em A Small Place}), the students contended with how the development of tourism irrevocably altered the natural soundscape of the island. The students remixed reggae songs using echoing and reverb DJ effects to consider how Bob Marley forms a global sonic imaginary, putting Jamaica and the broader Caribbean on the world map, so to speak.

The New York group was named Mannahatta, the Lenape word meaning \quotation{place where the wood is gathered.} New York City, where roughly a quarter of the recorded population is Caribbean (as of 2013), allowed an opportunity to meditate on how many more Caribbean people call New York home than were counted, considering those who are not documented, especially from Guyana and the Dominican Republic. The students in this group had the advantage of being able to include interviews and other recorded material from a local context. They chose the title \quotation{New York as Caribbean City: Gentrification as Colonization} to highlight the layers of occupation of Lenape land.\footnote{The students who assisted with the New York story map include Awura Gyimah, Matthew Martinez, Massiel Perez, and Xiaolong Woods.} The only group to directly tackle the issue of opacity directly, the students created a landing page that queries the intentions of those accessing the map: after the title \quotation{Manahatta,} a warning reads, \quotation{The following is only for individuals who have roots in the Caribbean and their allies. Do not proceed if you do not meet the criteria stated above.} They used a flick FBI warning against copyright infringement as the backdrop, which was a curious choice, considering the fraught questions of jurisdiction situating Manhattan as Lenape land. If the user does not fit the criteria of allyship, she has the option to click \quotation{Leave} in the upper right corner, which leads to a broken \useURL[url22][http://leave.leave.leave./][][hyperlink]\from[url22]---a creative strategy for addressing the challenges of retaining opacity and deflecting uninvited/unwanted audiences. The key to the {\em Unmapping the Caribbean} website is a decolonial practice of coalition building; these students did not want the Manahatta map to fall into the \quotation{wrong} hands.

\placefigure{New York as Caribbean City: Gentrification as Colonization}{\externalfigure[images/goffe/fig6.png]}
Listening is the key to opening the map. To welcome the viewers who proceed, the students produced a forty-second soundtrack, layering the rumbling sound of a train on subway tracks with the crashing of waves and Caribbean salsa music and drums. This is followed by four recordings: \quotation{Ven, este mapa es para ti}; \quotation{Cum, this map a fi yuh}; \quotation{Vini non, map sa a se pou ou}; and \quotation{Come, this map is for you.} In each phrase, the word {\em map} or {\em mapa} is hyperlinked and leads to an expandable side accordion story map featuring ten layers of gentrified spaces in New York City (fig.~6). This was the perfect type of story map for their narrative because of its capacity to enact a relational aesthetic of layering. For example, one page features historically Caribbean neighborhoods such as Williamsburg, home to Puerto Rican communities like Los Sures. Similar to how the Xaymaca group aurally focused attention on tourism, the New York group showed gentrification by contrasting the sounds of bodegas and people playing dominos on the street with markers of sonic colonialism---notably, the idle chatter of gentrifying hipsters.

The students in the Suriname group, named Surinen, for the indigenous people of that area, titled their project \quotation{Mapping the Interior: Navigating Black and Brown Queer Intimacy.}\footnote{The students who assisted with the Suriname story map include Zainab Floyd, Kevwe Okumakube, Amodhya Samarakoon, and Hawau Touray.} Rather than focusing on tensions among black and Asian communities, which we had studied in the Eastern Caribbean, they chose to center queer intimacy between black and Asian people in Suriname; one part of the story map is titled \quotation{Navigating the Interior: Black and Brown Queerness.} \quotation{Mati work,} what Dutch-Surinamese feminist theorist Gloria Wekker defines as \quotation{an old practice among Afro-Surinamese working-class women in which marriage is rejected in favor of male and female sexual partners,} and music were central to their story map.\footnote{See the synopsis for Gloria Wekker, {\em The Politics of Passion: Women's Sexual Culture in the Afro-Surinamese Diaspora} (New York: Columbia University Press, 2006) at \useURL[url23][http://cup.columbia.edu/book/the-politics-of-passion/9780231131629]\from[url23].} The students stitched and edited together footage from Surinamese dancehall and {\em kaseko} performances with footage from pride parades and celebrations of LGBTQ life in the country. They also wrote original poems in conversation with the concept of the rainforest that is commonly described as \quotation{the interior} in Suriname.

\placefigure{Surinen: Mapping the Interior}{\externalfigure[images/goffe/fig7.png]}
The Surinen story map also showcases rituals of the African-derived religion Winti and how Amerindian and Maroon communities retain their cultural practices. The students presented time-lapse nature videos of flowers blooming from YouTube alongside the words of Wekker (fig.~7). Though in the beginning of the term my students had not heard of Suriname, by the end they had an intimate relationship with the country. Having conducted research in Paramaribo, I personally found the sonic rendering of the Suriname story map so powerful that it felt as though the students had gone there to carry out this research.\footnote{After the 2016 US presidential election, through my desire to organize against new fascism I found community with the Sanctuary Coalition of New York University. I would like to thank Paula Chakravartty for her generosity and her leadership of the Sanctuary Coalition. I contributed to the Sanctuary syllabus designed by graduate students and published by {\em Public Books}. See \quotation{Sanctuary Syllabus,} {\em Public Books}, 5 December 2017, \useURL[url24][http://www.publicbooks.org/sanctuary-syllabus/]\from[url24].}

The Hispaniola group, named Kiskeya/Quisqueya after the island's original Taíno name, focused on the contours of Afrolatinidad in the Caribbean. These students considered how music has been a space of Afro-Caribbean refuge and expression, and they subtitled their story map both \quotation{A Journey through the Border} and \quotation{A Journey along the Border} (fig.~8).\footnote{The students who assisted with the Hispaniola story map include Sophia Gumbs, Harmony Hemmings-Pallay, Naomi Malcolm, and Jesse Sgambati.} Keeping in mind that several students had rendered Haiti and the Dominican Republic as separate islands in their freehand analog maps at the beginning of the semester (see fig.~3), they wanted to center the connections between the two countries along their linguistic and national border using the \quotation{Cascade} Esri Story Map, which creates an immersive scrolling experience. The students presented how Haiti and the Dominican Republic are joined by music and how the narratives of Taino history inflect both nations, filling the screen with maps, 3D scenes, images, and video.

As one of the most prominent and poetic voices on immigration and the status of black refuge and refugees, Haitian American novelist Edwidge Danticat served as the students' literary guide. In her essay \quotation{Black Bodies in Motion and in Pain,} Danticat evocatively draws connections between black death and flight in the Mediterranean, the Great Migration in the United States, and the deportations at the Haitian-Dominican border. I encouraged the students to also consider the twin-island imaginary of historic Saint-Domingue that Danticat presents in {\em Krik? Krak!}\footnote{See Edwidge Danticat, \quotation{Black Bodies in Motion and in Pain,} {\em New Yorker}, 22 June 2015 (online), \useURL[url25][https://www.newyorker.com/culture/cultural-comment/black-bodies-in-motion-and-in-pain]\from[url25]; and {\em Krik? Krak!} (New York: Soho, 2015).} They historicized their map with information about the border, especially the 1937 \quotation{parsley massacre} and the more recent deportations of Afro-Dominicans.\footnote{In 1937, the Dominican dictator Rafael Trujillo ordered the mass killing of Haitians living in the Dominican Republic.}

\placefigure{Kiskeya/Quisqueya: A Journey along/through the Border}{\externalfigure[images/goffe/fig8.png]}
The students discovered that there are certain instruments that connect Haitian music ({\em kadans}) to Dominican music ({\em bachata}). They featured music videos of two global superstars---Dominican bacahata singer Raulín Rodriguez and Puerto Rican merengue singer Elvis Crespo---to illustrate how genres transcend national and geographic borders. The students also highlighted the Haitian guitar-based Latin genre {\em twoubadou}, which developed through migration across the border with the Dominican Republic. One student, Jesse Sgambati (Jesediah), who was pursuing a degree in recorded music, produced \useURL[url26][http://archipelagosjournal.org/assets/extras/issue05-goffe-recording.mp3][][an original composition]\from[url26], an instrumental track that blends Haitian polyrhythms and features shakers, skin drums, and wood blocks alongside electronic elements, snares, 808 kicks, and synth sounds. The string elements of the song gesture to merengue but are played on a {\em charango}, an Andean folk instrument, which was paired with the {\em banza}, the banjo of Haiti. The song is a beautiful example of sonic copresence and decolonial rhythm, and the classroom presentation of this story map was vibrant---students began to dance and gave the group a standing ovation.

\placefigure{Cubao: Santería from Obscurity to Hypervisiblity}{\externalfigure[images/goffe/fig9.png]}
Continuing on the theme of negotiating and defining Afrolatiniad and Latinx identity in the Caribbean, students in the Cuba group, named Cubao (meaning \quotation{abundant,} derived from the country's original indigenous name, Cubanascnan) titled their story map \quotation{Cuban Santería: Religious Transformation from Obscurity to Hypervisiblity} (fig. 9).\footnote{The students who assisted with the Cuba story map include Elliott Levy, Michelle Mawere, and Eva Toscanos.} The students, one of whom had filmed a documentary in Cuba the year before, mapped the cosmology of Cuba through Santería's global circulation. They were interested in the retention of West African language, rituals, and Yoruba beliefs. Dash Harris, the then Cuba-based founder of the company Afro-Latinx Travel and a guest lecturer to my course, described Catholicism as a cover for practicing original and secret religion, distinct from the ways academics often flatten Santería as a simple expression of syncretism. Though the various religions in the Caribbean certainly retain elements from African, European, and Asian cosmologies, it is significant to consider the mapping of saints and orishas as a strategy as opposed to a multiculturalist celebration. The students were fascinated by religion's role in representing a hidden world of performing West African rituals. I encouraged them to explore Amerindian imagery and Chinese influences present in Cuba as well. They organized their narrative thread around the circulation and hypervisibility of Santería through Afro-diasporic popular musical artists with global audiences, such as Beyoncé, who has invoked Yemaya and Ochun, and Afro-French Cuban singers Ibeyi, the twins Lisa-Kaindé Diaz and Naomi Diaz, who sing in multiple languages, including Yoruba. The students compiled news footage from YouTube of the fraudulent Santería grey economy of false practitioners, a market unsanctioned by the Cuban government, that has arisen alongside tourism in Cuba.

\subsection[title={Future Grounds of Sonic Unmapping},reference={future-grounds-of-sonic-unmapping}]

Questions regarding labor, credit, and archiving are critical for this decolonial cartography. In the interest of interrogating my and my students' US-based perspective, I arranged video conference sessions with experts based in the Caribbean. In addition to our class with Dash Harris in Cuba, we had a session with Trinidad-based Felicia Chang and Zaake De Coninck, cofounders of the media publishing company \useURL[url27][https://www.plantain.me/][][Plantain]\from[url27]. Based on their expertise in oral history, these entrepreneurs provided their digital storytelling experience in consultation with all the groups.\footnote{See Plantain, \useURL[url28][https://www.plantain.me/]\from[url28]. Additional consultants included LGBTQ rights activist Robert Taylor Jr., of Queeribbean (\useURL[url29][https://www.queeribbean.com/]\from[url29]), who consulted with the Jamaica group; and performance artist Alicia Grullon, of Dominican and Haitian heritage and raised in New York, who helped the New York City group. Omar Daujahre, the deputy director of NYU's Center for Latin American and Caribbean Studies, and Melissa Fuster Rivera, an assistant professor of food studies at Brooklyn College, both from Puerto Rico, also helped the New York group.} I also received assistance over the years from technologist Armanda Lewis, who helped me design templates for assessment of collaborative work and the for logistics of the assignment.\footnote{Armanda Lewis brainstormed with me about an approach to grading and provided templates for peer grading.} In addition, Armanda helped me design a rubric clarifying my expectations in terms of creativity, originality, research, citational practice, and collaboration. For the on-boarding process and technological troubleshooting with Esri Story Maps, technologists Michelle Thompson and Himanshu Mistry, of New York University Libraries Data Services, were invaluable.\footnote{See \quotation{Data Services: Home,} New York University Libraries, \useURL[url30][https://guides.nyu.edu/dataservices]\from[url30].}

In the vein of Stefano Harney and Fred Moten's {\em The Undercommons: Fugitive Planning and Black Study}, I encouraged my students to consider treason, in a metaphorical sense. If, as Moten suggests, the university is an engine of settler colonialism---in that weapons are designed there as well as technologies that devastate the natural environment and financial instruments that exploit the most vulnerable with subprime mortgages, for instance (these links were especially clear in the Manahatta group's story map as it paralleled gentrification and imperialism in New York City)---then it is our duty to deploy a university's resources for purposes other than continuing a settler colonial regime.\footnote{See Stefano Harney and Fred Moten, {\em The Undercommons:} {\em Fugitive Planning and Black Study} (Wivenhoe, UK: Minor Compositions, 2013).} Universities vary in access to resources, and I foregrounded awareness of all the technological resources available to students in the university, emphasizing that they are paying tuition, or tuition is being paid on their behalf, for access to these technologies. I made sure to investigate open-source resources as well as any software and hardware available through university subscriptions. And if it is \quotation{treason} to use technology for something other than its intended purpose, then, indeed, we also committed just such an act in our decolonial Caribbean unmapping by using technology from Esri, a company invested in US defense contracts.

It was important for the students to see early in the semester what was possible and that it would not necessarily require advanced computer coding skills. They were able to experience what digital maps---data visualizations centered on geography---could illuminate and trouble. I was fortunate to have been introduced to \useURL[url31][https://newyorkscapes.org/project/unmapping-the-caribbean/][][{\em New York Scapes}]\from[url31] by Nicholas Wolf and Tom Augst, organizers in a research community for people engaged in cultural mapping.\footnote{See New York Scapes, \useURL[url32][https://newyorkscapes.org/]\from[url32]. {\em Unmapping the Caribbean} is included under \quotation{Projects.}} With financial assistance from New York Scapes, I was able to hire Ryn Stafford as project manager for {\em Unmapping the Caribbean}. A former student of mine, Ryn was familiar with the sorts of sonic creative deejaying projects I have assigned in the past, and so they were the perfect guide to translate the project and provide additional help to the five groups of students.\footnote{Familiar with the rhythms of the semester, Ryn encouraged students to seek help at the right moments. Another former student, Lu Biltucci, who has an interest in cartography, provided guidance to my students during the initial stages of the project. In addition, English PhD student and technologist Grace Afsari-Mamagani provided a demo of the various digital decolonial mapping projects that are currently being made. For example, {\em Torn Apart / Separados}, a two-volume digital project designed by Manan Ahmed and Alex Gil that visualizes the territory and infrastructure of US Immigration and Customs Enforcement, was a major inspiration. See {\em Torn Apart /Separados}, vol.~1, \quotation{a rapidly deployed critical data & visualization intervention,} at \useURL[url33][http://xpmethod.columbia.edu/torn-apart/volume/1/]\from[url33]; and vol.~2, \quotation{a deep and radically new look,} at \useURL[url34][http://xpmethod.columbia.edu/torn-apart/volume/2/]\from[url34].}

Beyond what students were able to learn about Caribbean culture through the process of independent research for this project, they also implicitly learned about copyright, the politics of fair use, and proper citation practice. They meditated on the politics and ethics of circulation as well. Who, beyond me, the professor assigning a grade, was the intended audience? The right to copy and the concept of {\em the commons} was deeply embedded in the intellectual exercise of sonic unmapping we engaged in. Citation is politics. Citation is ethics, and yet it is rarely done ethically. Students contended with the differences in copyright between various jurisdictions and territories in the Caribbean and asymmetrical imperial legal legacies. The assignment also gave them a chance to consider the currency of cultural production and how it does and does not travel across an archipelago of many tongues and varying digital accessibility.

{\em Unmapping the Caribbean} is necessarily unfinished, much in the way that musicologist Michael Veal describes dub music as having an unfinished sonic quality.\footnote{See Michael E. Veal, {\em Dub: Songscape, Studio Craft, Science Fiction, and the Shattering of Song Form in Jamaican Reggae} (Middletown, CT: Wesleyan University Press, 2007), 90.} My project of unmapping continues as long as my course does. It is critical that the unmapping is anchored with me and does not belong to any institution. There will at some point be a physical, material installation to complement and extend the digital life of the map as an immersive process using projection mapping. Already, moving between institutional infrastructures has required me to be intentional about disentangling and methodically archiving students' work: I am the guardian of the collected work. Like the image of my hand anchors the map on the landing page, the maps accumulate as part of my pedagogy in ways that continue to inform the layers of my research on archipelagic being.

I must also consider, for future iterations of the project, {\em where} to task my new cohort of students with unmapping. Do I ask them to map five new locations? Perhaps Puerto Rico, St.~Thomas, Miami, St.~Barthes, Toronto, Zion, Port of Spain? Or do I ask them to add layers to the maps made by previous students? In any case, the digital space as a repository of knowledge production becomes a growing archive of unmapping. My students complete the course having gained new technological or research skills that they can highlight on their resumes and in their portfolios.

\subsection[title={Trust Your Inner Map: \#JamaicaIsNotARealPlace},reference={trust-your-inner-map-jamaicaisnotarealplace}]

Part of the reason the unmapping is unfinished is that I did not seek anything definitive from the assignment, and this was vital to it being a process of critique through undoing and unlearning colonial myths. We were not simply remapping, and thus adding new layers of epistemic violence and division to the land. Unmapping is a decolonial digital praxis that embraces sensorial knowledges of black and indigenous cosmologies. In this way it is in conversation with the hashtag \#JamaicaIsNotARealPlace, which has been popular on Jamaican social media. Just as Jamaica \quotation{is not a real place,} the editors of the new Caribbean arts magazine {\em PREE} consider that maybe the Caribbean is not a real place, choosing this as a theme for one of their issues.\footnote{See {\em PREE}, no. 3, 2019, \useURL[url35][https://preelit.com/issue-three-thecaribbeanisnotarealplace/]\from[url35].} They broaden the concept and claim about dysfunction, frustration, and the stalled futurity of the region. In a similar spirit, I posit that the Caribbean defies mapping because it is not a real place. The poetics of Caribbean refusal involve exclusion as much as they can be coded as a set of refusals. The Caribbean seeks legibility but not by everyone. These are pertinent questions for institutions such as the University of the West Indies Mona Geoinformatics Institute, a hub for instruction and contract work for both the private and public sectors. There are generations of Caribbean-trained cartographers whose genealogy is not entirely imperial. For example, GIS specialist Michelle Thompson, of Data Services, who was most instrumental to {\em Unmapping the Caribbean}, is of Guyanese, Trinidadian, and Jamaican background and is the daughter of a rural physical planner with the Ministry of Agriculture in Jamaica who studied at UWI in the 1980s.

It was of special significance to be able to debut the collective {\em Unmapping the Caribbean} project at the Caribbean Digital V conference at UWI, St.~Augustine, in December 2018, with my Trinidad-based collaborators Felicia Chang and Zaake De Coninck (of Plantain), to center Caribbean digital and artistic production. It helped to see the project as part of an ecosystem of Caribbean diasporic digital engagements facing similar ethical concerns, much as we had, about the complicity of using Esri as a tool. Representation puts populations on the map, so to speak, but certain populations may prefer to be unmapped. If you cannot see me, you cannot colonize me. The echo of Me-no-Sen-You-no-Come across the map resists the order of the coloniality of mapping, where trains run on time, where people are where they say they will be at a given time. Predictive analytics work only if people are predictable or their habits can be easily traced, tracked, surveilled, and learned by algorithms.

A moment that typifies \#JamaicaIsNotARealPlace~is the story of my first meeting of a distant cousin in 2014. The epigraph to this essay is drawn from the utterance of a police officer whom my family and I had asked for directions to my cousin's house in Mandeville as we were driving. Just after our query, the traffic light turned green and the policeman zoomed away, leaving us with, \quotation{You will find a way.} His response, which has stayed with me, was flippant, prophetic, mystical, matter of fact, and spoken with a Jamaican affect that cannot quite be described in English---not quite cynical but dismissive, with an air of nonchalance. So we continued on our way, navigating by asking people on the street if they knew where Mr.~Goffe lived. We eventually encountered a few schoolboys who replied that we could find our cousin's house all the way down the road and around the corner.

As frustrating as it was, in the end the policeman was correct. We found a way. We found our cousin. His house, it turns out, did not have an exact address. And we found him in a time capsule of sorts. Our cousin is a collector, a hoarder of things, knickknacks mostly but also the materials of the Afro-indigenous and colonial history of Jamaica. His treasures were strewn across his house---Arawak arrowheads, for example, were scattered next to wedding albums full of colonial-era images of Jamaica's \quotation{mulatto aristocracy.}

The disorientation of being in and journeying through Jamaica made me realize the need to trust our inner maps, to listen to our bodies, to crowdsource people on the roadside in order to tune into our embodied and inherited histories. This is the type of sensorial listening that involves not only the ear but the entire body. It is a detour through the past in order to fashion a future (to recall the way Stuart Hall describes negotiating Caribbean identities as a continuous process). Unmapping is turning attention to an inner vestibular frequency, to an inner balance of finding a way just as so many Caribbean people have had to do to survive.

\subsection[title={Map Key},reference={map-key}]

The key to unmapping was relational. I would very much like to thank Kaiama Glover and Alex Gil for their support and inspiration in creative technology and collaboration. Seeing their early projects at the first Caribbean Digital conferences prompted me to dream and engage in a vulnerable pedagogy of learning with my students assisted by technologists. Thanks also to the anonymous readers and Kelly Martin for attention to the shape of my narrative. None of this would be possible without my students trusting the process of my teaching through digital-born coauthored collaboration. A special thanks to Ryn for their generosity, intellectual curiosity, and work as project manager.

\thinrule

\page
\subsection{Tao Leigh Goffe}

Tao Leigh Goffe is an assistant professor of literary theory and cultural history at Cornell University. She is also a sound artist, specializing in the narratives that emerge from histories of imperialism, migration, and globalization. Her interdisciplinary research and practice examine the unfolding relationship between ecology, infrastructure, and the sensorium. Deejaying, film production, digital cartography, and oral history are also integral to her praxis and pedagogy. She is the cofounder of the \useURL[url1][https://www.darklaboratory.com/][][Dark Laboratory]\from[url1] (see https://www.darklaboratory.com/), a humanities collective of theorist and creative technologists that examines black and indigenous crossroads in ecology and technology.

\stopchapter
\stoptext