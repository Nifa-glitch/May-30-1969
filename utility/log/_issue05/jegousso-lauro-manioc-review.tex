\setvariables[article][shortauthor={Jégousso, Lauro}, date={December 2020}, issue={5}, DOI={https://doi.org/10.7916/archipelagos-fvgc-xs31}]

\setupinteraction[title={*Manioc*: A Pioneer Digital Library in the Francophone Caribbean},author={Jeanne Jégousso, Raphaël Lauro}, date={December 2020}, subtitle={Manioc Review}]
\environment env_journal


\starttext


\startchapter[title={{\em Manioc}: A Pioneer Digital Library in the Francophone Caribbean}
, marking={Manioc Review}
, bookmark={*Manioc*: A Pioneer Digital Library in the Francophone Caribbean}]


\startlines
{\bf
Jeanne Jégousso
Raphaël Lauro
}
\stoplines


Harvested by pre-Columbian societies all over the Americas, manioc (also known as cassava) has become over the centuries one of the staple foods of this vast territory. It is associated with several myths and spirits, and its production and transformation are grounded in a collective effort to sustain entire communities. It comes as no surprise, then, that the digital library \useURL[url1][http://www.manioc.org][][{\em Manioc: Bibliothèque numérique Caraïbe, Amazonie, Plateau des Guyanes}]\from[url1], which celebrated its ten-year anniversary in 2019, continues to nurture knowledge, reflection, and research on history, cultures, rituals, and people of the Caribbean and South and Central America.\footnote{See \hyphenatedurl{http://www.manioc.org.} The site is in French, with links for Google translation to English and Spanish.} Indeed, this extensive digital library of the Caribbean, Amazonia, and the Guiana Shield has allowed for a better \quotation{understanding of the complex relationship that these regions have maintained with the rest of the world} by making visible a long-ignored regional heritage and by enhancing productions from and on the Caribbean and the surrounding territories.\footnote{\quotation{Compréhension des relations complexes que ces régions ont entretenu avec le reste du monde}; Anne Pajard, \quotation{Manioc.org: Une bibliothèque numérique aux racines de la Caraïbe-Amazonie,} {\em Ar(abes)ques}, no. 86 (July--September 2017): 20.} It also provides researchers with authentic and original documents, amounting to tens of thousands of texts and images now accessible to users around the world.

\placefigure[here]{{\em Manioc} Homepage Banner}{\externalfigure[issue05/jegousso-lauro/fig1.png]}


Started in 2006 and launched in 2009, {\em Manioc} initially aimed to collect disseminated texts and historical sources from different time periods, before diversifying to include contemporary materials and promoting its abundant catalogue to aid new academic research and general enquiries. Once a preliminary phase of digitization provided a first wave of documents, the digital library broadened its area of expertise and activity. Going beyond its original mission to preserve and share archives, {\em Manioc} established a series of partnerships with private collections, research departments, regional libraries, and documentation services to conserve documents, augment the site's content, increase its exposure, and answer a growing demand from international scholars.

Co-led and co-managed by the library services of the University of the Antilles and the University of French Guiana,\footnote{At the time of {\em Manioc}'s creation, these universities were a single entity, the University of the French West Indies and Guiana.} the collaborative project benefits from the support of private \useURL[url2][http://www.manioc.org/partenaires.html][][partners]\from[url2], such as the Clément Foundation, in Martinique, and of public institutions both in the Caribbean and France, including the Guyanese association Grappe ORkidé; the {\em Digital Library of the Caribbean}; the National Library of France and its digital platform, {\em Gallica}; the Culture, Education, and Environment Council of Martinique; and the French Department of Education.\footnote{For a complete list of partners, see \quotation{Partenaires,} {\em Manioc}, \hyphenatedurl{http://www.manioc.org/partenaires.html.}}~Today, {\em Manioc} has secured strong ties with more than fifteen institutions on both sides of the Atlantic, and, by also bringing together several databases, has accomplished the centralization of thousands of collections, cementing its position as a leader in Caribbean digital initiatives.

According to Anne Pajard, a trained engineer who leads and manages {\em Manioc} from the Schœlcher campus in Martinique, it makes sense that such a challenging project found its rhizomatic roots within the Caribbean and South American regions: \quotation{In the 1970s, university libraries of the Caribbean took on a specific role given the deficiencies of public libraries, the vulnerability of patrimonial holdings in a tropical climate, and the influence of intellectual movements engaged in the postcolonial reconstruction of society.}\footnote{\quotation{Les bibliothèques universitaires de la Caraïbe ont pris dans les années 1970 une place spécifique liée aux carences de la lecture publique, à la vulnérabilité particulière du patrimoine en climat tropical et à l'influence des mouvements intellectuels engagés dans la reconstruction postcoloniale de la société,} Pajard, \quotation{Manioc.org,} 21.} In fact, {\em Manioc}'s~team illustrates this idea by creating and maintaining a network of librarians, scholars, and institutions across the Caribbean Sea and its surrounding coasts, keeping alive the \quotation{manioc civilization.}\footnote{See Patrice Bidou, \quotation{Une civilisation du manioc amer,} in {\em Le mythe de Tapir Chamane: Essai d'anthropologie psychanalytique} (Paris: Jacob, 2001), 56--57.} Alongside a scientific committee led by Gerry L'Etang (professor of anthropology at the University of the Antilles) and Sylvain Houdebert (director of the Documentation Services in Martinique), {\em Manioc} is run by \useURL[url3][http://www.manioc.org/presentation.html][][a team]\from[url3] of ten people, scattered throughout Martinique, Guadeloupe, and French Guiana, tasked with cataloging, documentation services, and other technical duties.\footnote{See \quotation{Présentation,} {\em Manioc}, \hyphenatedurl{http://www.manioc.org/presentation.html.}}

\placefigure[here]{{\em Manioc} Homepage Content}{\externalfigure[issue05/jegousso-lauro/fig2.png]}


The {\em Manioc} digital library is mainly organized into five categories, displayed on the \useURL[url4][http://www.manioc.org/][][homepage]\from[url4]: \quotation{Patrimoine} (\quotation{Heritage}); \quotation{Thématiques} (\quotation{Themes}); \quotation{Recherche} (\quotation{Research}); \quotation{Création} (\quotation{Creation}), a collection of new initiatives; and \quotation{Revues universitaires} (\quotation{Scholarly Journals}). Each section is then divided into anywhere from two to five subsections, with some headings indicating how many documents are viewable. The section \quotation{Patrimoine,} for example, offers \quotation{Livres anciens} (\quotation{Rare Books}), with 2,698 documents; \quotation{Images,} a collection of more than 15,000 items; and \quotation{Catalogue collectif des périodiques,} a valuable collection of hundreds of periodical titles, essentially from the Caribbean cultural production. The section also offers two slavery databases: the still-expanding \quotation{Esclaves de Guyane} (\quotation{Slaves of French Guiana}) and the \quotation{Esclavage en Martinique,} (\quotation{Slavery in Martinique}) which concerns more than 14,700 enslaved people and indexes more than 3,200 notarial reports. The \quotation{Thématiques} section is subdivided into the topics \quotation{Or} (\quotation{Gold}) and \quotation{Esclavage et résistances} (\quotation{Slavery and Resistance}), which might have presented a more substantial thematical category about slavery in the Americas had it been combined with the two preceding subsections. From this organization of the site structure emerges the project's main goal---to go beyond the role of a digital library and to render \quotation{visible the invisible,} creating a space where indigenous people of the Caribbean, anticolonial struggles, and Creole and minor languages studies can take the forefront. As a digital initiative, {\em Manioc} gathers materials from several fields, making it an interdisciplinary platform aimed at uniting all useful information pertaining to this cultural area.

The \quotation{Création} section demonstrates more explicitly the diversity of materials available on {\em Manioc}. The first subsection, \quotation{Écritures contemporaines Caraïbe---Amazonie} (\quotation{Contemporary Caribbean and Amazonian Writings})---a project begun in January 2019---aims to gather archives and unpublished and rare documents related to Caribbean authors. The first collection (as of March 2020) is dedicated to Guadeloupean writer Maryse Condé, who donated her archives to the University of the Antilles. This section notably includes her 1976 doctoral thesis; an unpublished novel, {\em Les Pharisiens}; several interviews and photographs; recent scholarship on her work; and a forum through which users are invited to help complete the digital bibliography of work by and on Condé. The second subsection, \quotation{Mémoires et creations} (\quotation{Memory and Creation}), focuses on the work of photographers whose themes relate to the intangibility of cultural heritage, including the Italian photographer Nicola Lo Calzo, who has been developing the \quotation{Cham Project} for the past ten years, distilling in photographic form the memories of slaves, resistance, and abolitionist struggles from all across the Atlantic.

\placefigure{Écriture contemporaine Caraïbe---Amazonie}{\externalfigure[images/jegousso-lauro/fig3.png]}
In practical terms, the library hosts more than twenty-one thousand documents and provides access to more than thirty thousand additional items from other web portals. The historical documents (numbering around two thousand, including essays, novels, travel notebooks, pamphlets, poetry, and administrative notes, as well as nine thousand images dating from the seventeenth to the twentieth century) come primarily from private and public collections that were digitized specifically for the project in its early stages. Contemporary documents were gathered mostly from faculty, students, and staff during colloquia and symposiums organized at the University of the Antilles. More than two thousand videos have been collected during these events. Having this temporal duality between historical and contemporary materials on a single platform allows for an ongoing dialog between the past and the present. This makes for a dynamic space and highlights {\em Manioc}'s intention to be emblematic of a \quotation{Caribbean library,} which does not meet the criteria of a linear history, as explained by Martinican author Édouard Glissant in {\em Caribbean Discourse}: \quotation{Our quest for the dimension of time will therefore be neither harmonious nor linear.}\footnote{See Édouard Glissant, {\em Caribbean Discourse: Selected Essays}, trans. J. Michael Dash (Charlottesville: University of Virginia Press, 1992), 106.}

Based on the ascertainment that a gap has existed between old and contemporary collections, revealing a problem of circulation between productions, {\em Manioc}'s team also developed projects in four categories: archival, scientific, editorial, and creative. The initiative \quotation{Catalogue collectif des périodiques Caraïbes-Amazonie} was launched to preserve, archive, and index mostly contemporary local periodicals (see the \quotation{Patrimoine} section). As of today, these abundant and diverse bibliographical references are linked to physical collections across sixteen libraries. We hope a digitization project will soon be initiated by {\em Manioc}'s team to make those precious and often confidential documents accessible online. Sciencewise, {\em Manioc} supports the \useURL[url5][http://www.tramil.net][][Program of Applied Research to Popular Medicine in the Caribbean]\from[url5] (TRAMIL), which includes the work of more than two hundred experts in the fields of ethnobotany, chemistry, pharmacology, medicine, and social work (see the \quotation{Recherche} section).\footnote{See also TRAMIL, \hyphenatedurl{http://www.tramil.net.}} As for the editorial category, the digital library promotes three publications led by faculty members at the University of the Antilles: \useURL[url6][http://etudescaribeennes.revues.org][][{\em Études Caribéennes}]\from[url6], \useURL[url7][https://www.archipelies.org][][{\em Archipélies}]\from[url7], and \useURL[url8][https://journals.openedition.org/ced/][][{\em Contextes et Didactiques}]\from[url8] (see the \quotation{Revues universitaires} section).\footnote{See also {\em Études Caribéennes}, \hyphenatedurl{http://etudescaribeennes.revues.org;} {\em Archipélies}, https://www.archipelies.org; and {\em Contextes et Didactiques}, https://journals.openedition.org/ced/.} Lastly, creatively speaking, the subsection \quotation{Mémoires et creations} includes work by contemporary artists focusing on traditional activities and rituals shared widely across Caribbean region (see the \quotation{Création} section).

Technically, {\em Manioc} is an open-access library, meaning all documents can be accessed, consulted, and downloaded for free from any location{\em .} The platform's efficient search engine (OAI-PMH protocol) is connected not only to {\em Manioc}'s own database but to those of its partners as well, such as {\em Gallica} at the National Library of France. In addition to traditional parameters available in advanced searches (e.g., languages, time period, author, topic, publisher, title, etc.), users may also choose which additional portals they would like to include in their search ({\em Digital Library of the Caribbean}, University of the West Indies, {\em Gallica}, HAL-UA, and {\em Thèses en ligne}){\em .} Because of this interconnectivity and the ongoing addition of ancient and recent documents, {\em Manioc} has become over the years more than a research space for scholars in humanities and social sciences; it is also a documentary space for society, contributing to the legitimization of cultures by promoting and highlighting Caribbean and Amazonian heritage while granting a relatively unfettered continuous access to knowledge. According to data shared by Anne Pajard at ACURIL 2014, the forty-fourth annual meeting of the Association of Caribbean University, Research, and Institutional Libraries, more than nine hundred thousand articles were downloaded from the website between 2009 and 2014.

A pioneer in the field of digital Caribbean studies, {\em Manioc} has evolved significantly over the past eleven~years and is now pursuing technological improvements. The project's team is working on completely revamping the {\em Manioc} site with updated design and software in anticipation of inevitable technological obsolescence and to correct certain operational issues in an effort to perfect the platform's organization, visibility, management of materials, and accessibility. Currently designed with DRUPAL, the site will soon be redesigned with Omeka S, which will allow, for example, an easier integration of \useURL[url9][https://www.geonames.org][][GeoNames]\from[url9] interfacing and dynamic maps.\footnote{See https://www.geonames.org.} Moreover, the team has collectively decided to revise the website's languages, which might bring further international collaborators, notably in the anglophone and hispanophone Caribbean. As of today, the site is in French and can be automatically translated by Google into English and Spanish, which leaves a lot to be desired. Other modernizations comprise creating more persistent uniform resource locators, increasing the variety of collections, and outsourcing some of its content. These long-term efforts and updates will certainly improve a digital tool already effective in helping researchers and inspiring important collaborations.

{\em Manioc} also hosts a blog and maintains an active presence on social networks. The team shares detailed studies and special features that highlight library documents and collections, while prompting questions about historical figures, Caribbean cultural traditions, and past and current events. Through Facebook and Instagram, they relay happenings and bring attention to items that would otherwise go unnoticed. In fact, the site allows a ceaseless circulation between collections, offering the possibility to discover numerous documents one might not necessarily have searched for (much like falling into an archival box or a YouTube rabbit hole). This asset might also be an inconvenience of this digital library. Nonetheless, we are sure that the improvements now being made will make for a more focused research experience in the future.

A precious and invaluable tool for scholars, students, and all curious minds, {\em Manioc: Bibliothèque numérique Caraïbe, Amazonie, Plateau des Guyanes} has facilitated and will continue to facilitate research on history, culture, and knowledge production in the Americas, especially in the Antilles and French Guiana. This archipelagic digital library links together disseminated collections and puts in relation knowledge sources of various origins, time periods, and mediums, thus creating a rhizomatic atemporal platform---a mirror of the Caribbean. Thanks to its plurality of horizons, {\em Manioc} assembles, without reducing or summarizing, all this diversity into a rich, imminently plurilingual catalogue, reflecting the heterogeneity and multifariousness of the Caribbean archipelago.

\thinrule

Since 2017, Jégousso and Lauro have managed the \useURL[url10][http://wwww.glissantstudies.com][][{\em Library of Glissant Studies}]\from[url10] project, a collaborative digital bibliography collecting the scattered works of Édouard Glissant and the critical texts related to his oeuvre.

\thinrule

\page
\subsection{Jeanne Jégousso}

Jeanne Jégousso is an assistant professor at Hollins University, where she specializes in francophone literatures written in the Caribbean and the Indian Ocean. She is a co-editor of {\em Teaching, Reading, and Theorizing Caribbean Texts} (Lexington Books, 2020).~

\subsection{Raphaël Lauro}

Raphaël Lauro is an assistant professor of francophone Caribbean literature at the University of Montreal. After his doctoral dissertation, {\em Édouard Glissant: Thinker of the World, Poet of the Earth}, he published several articles on Glissant in peer-reviewed journals. Lauro, who was Glissant's personal secretary from 2008 to 2011, also classified and completed the first inventory of the author's personal archives.~

\stopchapter
\stoptext