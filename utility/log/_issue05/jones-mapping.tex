\setvariables[article][shortauthor={Curci, Jones}, date={December 2020}, issue={5}, DOI={https://doi.org/10.7916/archipelagos-mx57-xz29}]

\setupinteraction[title={Mapping the Haitian Revolution},author={Stephanie Curci, Christopher Jones}, date={December 2020}, subtitle={Mapping the Haitian Revolution}]
\environment env_journal


\starttext


\startchapter[title={Mapping the Haitian Revolution}
, marking={Mapping the Haitian Revolution}
, bookmark={Mapping the Haitian Revolution}]


\startlines
{\bf
Stephanie Curci
Christopher Jones
}
\stoplines


How many of us, as university professors, have found ourselves teaching college students in the fields of, for example, Africana studies, American studies, Latin American studies, and history, among others, whose knowledge of Haiti's revolution is strikingly limited if not entirely nonexistent? How many of us have had to teach into the void of disinformation that has long excluded Haiti from the so-called Age of Revolutions? The extent to which Haiti's struggle for independence has been effectively silenced from US-American \quotation{common knowledge} is remarkable. Moreover, it is a failing we can attribute at least in part to the omission of Haiti from secondary education curricula across the country. The \useURL[url1][https://www.mappinghaitianrevolution.com/][][{\em Mapping the Haitian Revolution}]\from[url1] aims to push back against that particular lacuna by providing to high school educators a rich and dynamic online pedagogical resource, featuring compellingly presented historical information as well as sample syllabi and bibliographical resources.\footnote{{[}Eds. note:{]} The site was originally called {\em Haitian Revolution Axis Map} by our reviewer. As a result of this review, the site was renamed {\em Mapping the Haitian Revolution} (https://www.mappinghaitianrevolution.com/); To learn more, see the site authors' response, which follows the review.}

\subsection[title={{\em archipelagos} review},reference={archipelagos-review}]

\subsubsection[title={Contribution},reference={contribution}]

­Teaching the Haitian Revolution, with its many stages, actors, and battles, can be an overwhelming task for educators. Stephanie Curci and Christopher Jones, instructors at Phillips Academy Andover (of English and history, respectively), have taken up the {\em Haitian Revolution Axis Map} project to bring clarity to the complicated narrative of the Haitian Revolution in ways that will most certainly help high school teachers integrate the history of the Haitian Revolution and of enslavement in the Americas into their classrooms. \footnote{Stephanie Curci, a scholar of Haitian studies, is also the creator of {\em Mapping Haitian History}, an online visual records site for Haitian historical and cultural sites that contains images of forts throughout Haiti, many of which date back to the revolutionary period. See {\em Mapping Haitian History}, \useURL[url2][http://www.mappinghaitianhistory.com/]\from[url2].}

The site authors have included a brief bibliography and multilingual resources in the \quotation{About} section to provide increased context to educators wanting to learn more about the Haitian Revolution. This section also includes syllabi for tenth- and twelfth-grade units, featuring readings by scholars such as Michel-Rolph Trouillot, Laurent Dubois, and John D. Garrigus. Charts on Saint Domingue's population growth, Léger-Félicité Sonthonax's emancipation proclamation, and links to the Institut de Sauvegarde du Patrimoine National's historical bulletins are also provided. It should be noted that the bibliography currently omits some of the websites, charts, and bulletins that appear in the map's narrative and events

High school teachers would benefit from a quick guide and webinar on how to use this resource, as well as a document on terminology, especially for educators not familiar with the Haitian Revolution in such detail and who perhaps have less time to cover the topic. Additionally, having an option to download PDFs of the site's content would increase accessibility. To facilitate the project's integration into curricula, the creators might also recommend varied approaches to the topic for different learners and provide more information about how the site fits curriculum standards.

\subsubsection[title={Design},reference={design}]

The {\em Haitian Revolution Axis Map} is quite easy to navigate, offering several ways of interacting with the timeline. One can, for example, press the large \quotation{Play} button on the bottom left of the map so that the timeline automatically chronicles the project's narrative in its entirety. It should be noted that while the narrative moves quickly, the site allows the user to pause and focus if additional time is needed to look at the fascinating details. An events timeline located just below the map allows the user to manually move through the history in respect to particular sections of the map and to then click on the circles and square boxes that highlight particular bits of information, including battles, notable figures, and colonial powers. Alternatively, the user can choose instead to jump to different designated time periods (for example, \quotation{1791--1798: Louverture's Rise to Power}) by selecting from a dropdown menu at the top left or by selecting individual points on the timeline below the map.

The different time periods determined by the site authors provide varying degrees of detail and context for actors, events, and battles. Each section begins with a brief description and a list of events. This list provides the user with additional information, including references to archival materials, secondary sources, and other teaching materials. The top of the map provides a list of event types, which are designated by evocative icons: multiple people for colonial settlements, swords to reference battles, and a fire symbol to indicate revolts. The map also provides color-coded indicators of the actors involved in each time period and includes an option to focus solely on the map without any thumbnails appearing. The home button feature allows the user to zoom in closer to view the details of the events and actors mentioned in each time period. All these design elements make clear the thoughtfulness and care the creators put into the user interface. The map is intuitive and engaging in every respect.

\subsubsection[title={Credit},reference={credit}]

The central feature of the website is the collection of maps used to narrate the island's shifting borders. The archival sources for these maps are carefully cited and include the Bibliothèque Nationale de France; the John Carter Brown Library, at Brown University; the Florida Museum of Natural History, at the University of Florida; and Yale University, among others. I commend the creators for the multilingual references, which include work by scholars such as Marlene Daut, Nathan H. Dize, Carolyn E. Fick, P. Gabrielle Foreman, and Julia Gaffield, as well as resources from the Digital Library of the Caribbean and the Slave Voyages Database. Many examples of existing digital scholarship projects are listed as well. Further, the project complements digital humanities work such as Nicole Willson's \useURL[url3][https://www.fanmrebel.com/][][{\em Fanm Rebèl}]\from[url3], which highlights stories of women insurgents in the Haitian Revolution.\footnote{See {\em Fanm Rebèl}, https://www.fanmrebel.com/.} The current resources list might be expanded by citing the \useURL[url4][https://haitianhistory.tumblr.com/readinglist][][{\em Haitian History Blog}]\from[url4] on Tumblr, which currently has an extensive list of books on Haiti divided by time period.\footnote{See \quotation{Reading Suggestions,} {\em Haitian History Blog}, https://haitianhistory.tumblr.com/readinglist.}

Support for the site comes from the Tang Institute at Phillips Academy Andover, a private university preparatory school located in Massachusetts, and Axis Maps built the custom website. This to say that the project relies on resources that might not be readily available to other high school teachers and was not built on open-source technology. I would recommend that the site authors make all documentation for the creation of the map openly accessible.

\subsubsection[title={Preservation},reference={preservation}]

No information is currently provided regarding site maintenance. In addition, given that digital scholarship projects about Haiti continue to emerge, the creators would do well to decide---and to indicate---whether and how often they will update the project site and its many resources.

Overall, the \quotation{\useURL[url5][http://haiti.axismaps.io/][][Haitian Revolution Axis Map]\from[url5]} is an ambitious and beautifully realized project. Though its stated target audience is primarily high school teachers and students, the platform might also serve as a resource for college students in entry-level courses. There is no question that it will be incredibly generative for educators seeking to push back against the persistent silence that surrounds the Haitian Revolution in US pedagogical contexts.

\subsection[title={Response by site authors Stephanie Curci and Christopher Jones},reference={response-by-site-authors-stephanie-curci-and-christopher-jones}]

We would like to begin with a sincere thank you to the editors of {\em archipelagos} and the outside reviewer of our digital project.

The \useURL[url6][https://www.mappinghaitianrevolution.com/][][{\em Mapping the Haitian Revolution}]\from[url6] project works as a companion piece to our efforts to teach the Haitian Revolution in a high school setting. We have long been struck by the lack of good map sequences of the Revolution. While there are a small number of existing maps in books on the Revolution, these are usually minimal because of their publishing cost (these maps were nonetheless immensely helpful and are listed in our \quotation{Works Cited}). Although there has been a significant growth in digital Haitian studies since Stephanie Curci first embarked on her visual records project {\em Mapping Haitian History} in 2007, nothing exists for the Haitian Revolution like the map and timeline materials readily available for other historical events of similar consequence. This was something we believed could and should be remedied.

Our strengths are more pedagogical than digital, but we initially tried to make the site with the open-source format Omeka.~It soon became clear that what we hoped to show on our map and timeline was more complicated than what we could achieve with any of the tools we were capable of managing on our own. Although a tool such as \useURL[url7][https://www.thinglink.com/][][ThingLink]\from[url7], which we have used for student projects, might have accomplished some of the spatial work, and a site such as \useURL[url8][https://timeline.knightlab.com/][][Timeline]\from[url8] from Knightlab could have created some of the timeline, we needed to represent the unique topographies, swiftly unfolding events, and dynamic changes in areas of control that marked the Revolution.\footnote{See ThingLink, https://www.thinglink.com/; and Timeline, https://timeline.knightlab.com/} Haiti is challenging to visit for scholars and students, and we felt a particular investment in not only the concrete history but also the Revolution's key physical locations. For example, it was in the process of creating the map that we noted how often mountain ridges helped to delineate the limits of regional control. Inspired by Vincent Brown's site \useURL[url9][http://revolt.axismaps.com/][][{\em Slave Revolt in Jamaica}]\from[url9] and with generous support from the \useURL[url10][https://tanginstitute.andover.edu/][][Tang Institute at Andover]\from[url10], we pivoted from attempting an open-source map to an animated map project constructed by \useURL[url11][https://www.axismaps.com/][][Axis Maps]\from[url11].\footnote{See {\em Slave Revolt in Jamaica}, \hyphenatedurl{http://revolt.axismaps.com/;} the Tang Institute at Andover, https://tanginstitute.andover.edu/; and Axis Maps, https://www.axismaps.com/.} We chose for our base image a French map from 1803 (the year France lost its battle for the colony) that had been built on earlier maps and that highlighted well the island's mountainous terrain, its rivers, and its coastline.~We hand-drew a seventeen-map progression, each aspect of which was then entered into a spreadsheet (including GPS points, events, and text) and plotted and animated by Axis Maps. This approach met almost all our needs.~We would have liked, for example, to have included an animated illustration of forts across the country to illustrate the scale and significance of Jean-Jacques Dessalines's defense project. Also, each map's events are highlighted separately, even if they were happening simultaneously; showing discrete events, as opposed to showing them all at once, perhaps weakens their collective power. For the most common question about the project---When do we believe the Revolution started and when did it end?---we provide a partial answer through our choice to start the progression of maps with Taino kingdoms (\quotation{0--1492---Taino Ayiti}) and end with \quotation{current} borders (\quotation{1844--2019---Current Haitian Borders}).~

In terms of site preservation, we are able to edit the spreadsheet that populates the seventeen maps in the progression, if the map content needs to be updated or revised, and we will continue to update resources, educational materials, and links on the site. While editorial access allows us to modify most things---from narrative to map titles to events to locations on the maps---we cannot alter the map base or the \quotation{areas of control} themselves (either would require more work from Axis Maps and, though not out of the question, would be beyond the scope of our original project).~The static HTML (hypertext markup language) portions of the site are not updatable by us, which is why our beta version used Google Docs for any narrative information that required frequent updates, such as our working syllabi and resources. These limitations also meant that our beta site did not have an updated \quotation{Works Cited} page for each link in the \quotation{Events} section, for instance.~There are many opportunities for linking to other Haitian studies digital sites, and~now that the bulk of the main project has been completed, those links will continue to be developed. Much of this work has already been added to our revised site.

As a result of this review, we have we have built a landing page with clearer resources for teachers and students. We have also renamed the project more clearly as {\em Mapping the Haitian Revolution}. We believe this transition will be a more effective starting point for map users and will better integrate the supplementary elements of our site. This change has allowed us to turn our bibliography into an active \quotation{Works Cited} page, which will be important as we incorporate and link to more digital humanities projects in Haitian studies as they emerge. The reviewer's suggestion of adding sites such as {\em Fanm Rebèl} and the {\em Haitian History Blog} on Tumblr was a helpful one, and those can now be found among our \useURL[url12][https://www.mappinghaitianrevolution.com/resources][][resources]\from[url12].\footnote{See \quotation{Resources,} {\em Mapping the Haitian Revolution}, https://www.mappinghaitianrevolution.com/resources.}

A focus of our coursework and of any legitimate study of the Haitian Revolution is to understand how power works both in history and through the construction of historical and literary narratives. We are thus interested in thinking about how we might highlight other scholars' work in this area. For example, in addition to all the forts shown on {\em Mapping Haitian History}, there are other important sites of historical memory production, such as those at Merotte and in Arcahaie center that celebrate Dessalines ripping the white band from the French tricolor and its being sewn back together by Catherine Flon, or Pont Rouge, where a statue by Moun Studio's Daniel Elie was recently placed at the site of Dessalines's assassination. A significant challenge we face is in acknowledging and integrating the growing wealth of digital work on the history and memory of the Revolution without sacrificing the focus and cohesiveness of our map project.~

We were particularly excited by some of the reviewer's specific pedagogical suggestions. As a result, by 2021, we plan to feature on our landing page a webinar on how to use the map site, and we plan to create short videos for each map in the progression as part of its already extant and separate short guide. In our classes, we have found terminology to be particularly difficult for high school students, especially around race, class, and legal status; thus we have added a document on terminology and all documents are available as downloadable PDFs.

We found particularly interesting the idea of offering suggestions on varied approaches to both the topic and the site for different learners and across disciplines. Currently, our senior interdisciplinary course can count as an English or a history elective for our twelfth-grade students, but this map is also intended to support tenth-grade students in their one-term history elective on revolutions. We have added supplementary handouts and assignments for each grade level. The map has further application possibilities in, for example, language courses or political science.~{\em Mapping the Haitian Revolution} coursework currently meets standards in two disciplines at our school, but we plan to link it to Common Core standards so that it can be better integrated in US public schools.~If this map project is helpful beyond English-speaking students, we would like to explore funding for translation work, which would also expand the scope of possible linked resources.~We welcome feedback and suggestions for further improvement and would especially appreciate examples of how the map is helpful in courses of any kind.

\thinrule

\page
\subsection{Stephanie Curci}

Stephanie Curci is an instructor and chair of the English Department at Phillips Academy in Andover, Massachusetts. In addition to her work on this project, she has maintained Mapping Haitian History (a visual records project) since 2007, a project she described in the Journal of Haitian Studies in 2008.

\subsection{Christopher Jones}

Christopher Jones is a scholar of the American Civil War, 19th Century America, and African American history. His research explores race, memory, and masculinity in post-Civil War America. He is an instructor of history at Phillips Academy, Andover.

\stopchapter
\stoptext