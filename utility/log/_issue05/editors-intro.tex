\setvariables[article][shortauthor={Glover, Gil}, date={December 2020}, issue={5}, DOI={https://doi.org/10.7916/archipelagos-39cd-mn05}]

\setupinteraction[title={},author={Kaiama L. Glover, Alex Gil}, date={December 2020}, subtitle={}, state=start, color=black, style=\tf]
\environment env_journal


\starttext


\startchapter[title={}
, marking={}
, bookmark={}]


\startlines
{\bf
Kaiama L. Glover
Alex Gil
}
\stoplines


We have had a hard time settling on the \quotation{right} moment to launch this fifth issue of {\em archipelagos} {\em journal}. This is very likely because so much about This Long Year 2020 just hasn't felt, well, right. With dangerous shenanigans ever-emanating from the executive branch, fires burning relentlessly on the West Coast, and state violence being perpetrated against black women and men across the United States and beyond, who were we to interrupt the global doom-scroll with missives from our digital Caribbean? How silly, though, to wait. The tides of history churn and we bob along, roiled certainly but neither silenced nor undone. We very much have things to say that amount, yes, to interventions in these times.

It is around the concept---the necessity---of intervention that the contributions to {\em archipelagos}(5) coalesce. Each essay and platform presented here considers the ways Caribbean subjects in the digital realm have refashioned the terms, the tools, and the terrain of representation of the self in the world. Engaging explicitly popular and accessible forms, from memes to music to blogs to maps, of common concern is the matter of crafting alternative authority within the structures of domination that condition our contemporary online and offline experiences.

Tzarina Prater and Jonathan J. Felix, respectively, direct our attention to the literal and metaphorical comments section with their powerful reminders of the critical pathways that emerge when we attend to the possibilities of online call-and-response. Both contributors explore the oft ignored digital paratext that writes back to and within the more sedimented textual offerings of corporate platforms such as Facebook and YouTube. Kelly Baker Josephs also considers informal spaces of online network- and community-building by Caribbean subjects. Her essay examines how self-identified queer Jamaican writers Staceyann Chin and Marlon James have made use of the blogosphere to cultivate an autobiographical presence outside and independent of traditional publishing venues.

This issue also takes seriously the classroom as a space of real intervention. Both Jacob Edmond and Tao Leigh Goffe lay out ways that seemingly neutral practices like seeing and mapping bear deconstructing when it comes to the matter of representational authority in the Caribbean. Edmond makes the case for privileging the audio over the visual in pedagogical engagements with Kamau Brathwaite's poetry. In his sound-rich essay Edmond argues that digital audio archives provide platforms through which to experience Brathwaite's work in ways that disorder our conventional reading and teaching practices. Goffe similarly proposes shifting our intellectual investments from the visual to the sonic in a determined praxis of what she calls \quotation{unmapping.} Placing black and indigenous Caribbean bodies in contestatory relation to the signposts of imperial modernity---to maps in particular---allows Goffe to bring students into Maroon spaces of both collaboration and, where necessary, refusal.

This issue's second valence also takes up and takes on cartographies of empire and modes of pedagogical refusal. Stephanie Curci and Christopher Jones's visualization, an interactive map and timeline of the Haitian Revolution, is a rich resource for secondary school teachers and students seeking to robustly situate Haiti in the American age of independence. A deliberate pushback to the silences that surround the Haitian past, Curci and Jones's project makes a crucial scholarly intervention into the world beyond the academy---a world that ultimately nourishes the academy in a very real way. Annette Joseph-Gabriel's visualization of {\em marronage} as literal and metaphorical black Atlantic practice of refusal also proposes a radical intervention, pedagogical in both its construction and its intention. The map not only contributes to a burgeoning practice of counter-mapping in our Caribbean digital scholarly communities, it does so in conscious intertextual relationship with that growing corpus. This kind of work perfectly embodies the connective aspirations and trajectory of our journal in this time---at this moment---of foreboding discontinuities.

Read. Explore. Enjoy. We hope {\em archipelagos}(5) will have been worth the wait.

Onward, Kaiama and Alex

\page
\subsection{Kaiama L. Glover}

\useURL[url1][https://barnard.edu/profiles/kaiama-l-glover][][Kaiama L. Glover]\from[url1] is Associate Professor of French and Africana Studies at Barnard College, Columbia University. She is the author of \useURL[url2][http://liverpooluniversitypress.co.uk/products/61903][][Haiti Unbound: A Spiralist Challenge to the Postcolonial Canon]\from[url2] (Liverpool UP 2010), first editor of \useURL[url3][http://yalebooks.com/book/9780300214192/yale-french-studies-number-128][][Marie Vieux Chauvet: Paradoxes of the Postcolonial Feminine]\from[url3] (Yale French Studies 2016), and translator of Frankétienne's Ready to Burst (Archipelago Books 2014). She has received awards and fellowships from the National Endowment for the Humanities, the Mellon Foundation, and the Fulbright Foundation. Current projects include forthcoming translations of Marie Vieux Chauvet's {\em Dance on the Volcano} (Archipelago Books) and René Depestre's {\em Hadriana in All My Dreams} (Akashic Books), and the multimedia platform {\em In the Same Boats: Toward an Afro-Atlantic Visual Cartography}.

\subsection{Alex Gil}

\useURL[url4][http://www.elotroalex.com/][][Alex Gil]\from[url4] is the Digital Scholarship Librarian at Columbia University Libraries. His research and practice focuses on digital humanities, epistemic design, minimal computing, and Caribbean literature. He is co-founder and moderator of \useURL[url5][http://xpmethod.plaintext.in/][][Columbia's Group for Experimental Methods in Humanistic Research]\from[url5], and coordinator of the Butler Studio at Columbia University Libraries.

\stopchapter
\stoptext