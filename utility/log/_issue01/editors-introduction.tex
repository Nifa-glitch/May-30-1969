\setvariables[article][shortauthor={Glover, Gil, Josephs}, date={July 2017}, issue={1}, DOI={Upcoming}]

\setupinteraction[title={},author={Kaiama L. Glover, Alex Gil, Kelly Baker Josephs}, date={July 2017}, subtitle={}]
\environment env_journal


\starttext


\startchapter[title={}
, marking={}
, bookmark={}]


\startlines
{\bf
Kaiama L. Glover
Alex Gil
Kelly Baker Josephs
}
\stoplines


The essays in this inaugural issue of {\em sx archipelagos} attend to the myriad opportunities and concerns that inform the Caribbean digital---its spaces, methodologies, and imaginations---as praxis and as historicized societal phenomenon. Contributors propose critical perspectives on the challenges and opportunities presented by media technologies that evermore intensely reconfigure the social and geographic contours of the Caribbean. The majority of the essays were generated from a broad public conversation initiated in 2014 in \useURL[url11][https://wayback.archive-it.org/1914/20151224034027/http://caribbeandigital.cdrs.columbia.edu/][][The Caribbean Digital I]\from[url11], a conversation to which we returned in 2015 with \useURL[url12][http://caribbeandigitalnyc.net/2015/][][The Caribbean Digital II]\from[url12] and which we intend to continue far into the future. Over the course of these two events, participants engaged with the so-called digital humanities in the so-called global South as an ethical, political, social, and creative phenomenon. The sustained reflections offered by our contributors very much illustrate the premises we laid out in these public forums. They are attentive to both the possibilities and the constraints of the technologies they evoke, and they consider the ways the Internet does and does not provide structural means for facilitating broad engagement, communication, and accessibility in the Caribbean.

In \quotation{Digitizing the \quote{Sound Explosions} of Anglophone Caribbean Performance Poetry,} Janet Neigh engages questions of archival best practices, audience, and representation in an insightful discussion of Trinidad's youth-led artist collective the 2 Cents Movement. What, Neigh asks, are the costs and the benefits of relying on the digital for the crucial functions of preserving and circulating Caribbean poetry and other forms of oral and performance art? And how does the relatively new performance platform that is the Internet shed light on the politics of preservation and circulation? Similarly, in \quotation{Twitter and @douenislands's Ambiguous Paths,} Jeannine Murray-Romàn considers how Caribbean artistic movements---in this instance, the Trinidadian-based Douen Islands Project---embrace digital technology and social media in their efforts to connect with and implicate a wide and diverse audience. Murray-Romàn's \quotation{Twitter-focused analysis} explores figurings of the folk---of the {\em douen} in particular---in both the substance and the practice of reading (in) the Caribbean. Both Neigh and Murray-Romàn affirm the digital as space wherein, despite myriad and significant challenges, artists and/as social actors are able to connect and collaborate transregionally. Also concerned with technology and the arts in the increasingly digitized Caribbean space, Martin Munro interrogates the relationship between music and digital media in \quotation{Who Stole the Soul? Rhythm and Race in the Digital Age.} Munro considers whether the affordances of technology risk endangering the originary radicalism of black music. Might the digital chance making rhythm accessible to global beat-makers in a troublingly race-and-history-free vacuum?

Tonya Haynes's essay, \quotation{Mapping Caribbean Cyberfeminisms,} also thinks about expanded opportunities for connection and community building throughout the (cyber)space of the Caribbean and its diasporas. Focused specifically on the phenomenon of online feminist activism, Haynes situates and critiques the vast network of feminist bloggers whose media presence at once forges new political paths and shores up the work of existing on-the-ground activist communities. Finally, mapping is at the heart of Yarimar Bonilla and Max Hantel's \quotation{Visualizing Sovereignty: Cartographic Queries for the Digital Age.} Reflecting on the ways formal representation is deeply connected to and even determinant of philosophical and political phenomena, Bonilla and Hantel ask how sovereignty in the Caribbean might be generatively reimagined through attention to visualization. From infographics to animation, the authors experiment with cartography as dynamic scholarly praxis through which to challenge existing hierarchies of nationhood.

\quotation{The Caribbean Digital} is, of course, a complex and unbounded space---one that we have only begun to explore in this special section. It is our hope that this wide-ranging and interdisciplinary collage of interventions will provide a rich point of departure for ongoing discussion of the political, ethical, and theoretical stakes of the digital in the Americas.

\page
\subsection{Kaiama L. Glover}

\useURL[url1][https://barnard.edu/profiles/kaiama-l-glover][][Kaiama L. Glover]\from[url1] is Associate Professor of French and Africana Studies at Barnard College, Columbia University. She is the author of \useURL[url2][http://liverpooluniversitypress.co.uk/products/61903][][Haiti Unbound: A Spiralist Challenge to the Postcolonial Canon]\from[url2] (Liverpool UP 2010), first editor of \useURL[url3][http://yalebooks.com/book/9780300214192/yale-french-studies-number-128][][Marie Vieux Chauvet: Paradoxes of the Postcolonial Feminine]\from[url3] (Yale French Studies 2016), and translator of Frankétienne's Ready to Burst (Archipelago Books 2014). She has received awards and fellowships from the National Endowment for the Humanities, the Mellon Foundation, and the Fulbright Foundation. Current projects include forthcoming translations of Marie Vieux Chauvet's {\em Dance on the Volcano} (Archipelago Books) and René Depestre's {\em Hadriana in All My Dreams} (Akashic Books), and the multimedia platform {\em In the Same Boats: Toward an Afro-Atlantic Visual Cartography}.

\subsection{Alex Gil}

\useURL[url4][http://www.elotroalex.com/][][Alex Gil]\from[url4] is Digital Scholarship Coordinator for the Humanities and History at Columbia University Libraries. He collaborates with faculty, students and the library on the use of technologies on humanities research, pedagogy and scholarly communications. His research is focused on textual scholarship, digital humanities and Caribbean studies. Current projects include \useURL[url5][http://elotroalex.github.io/ed/][][Ed]\from[url5], a foundation for {\em sx archipelagos}; the Open Syllabus Project; a geo-bibliography of Aimé Césaire; the Translation Toolkit; and, In The Same Boats, a visualization of trans-Atlantic intersections of black intellectuals in the 20th century. He is co-founder and active member of the Global Outlook::Digital Humanities initiative, \useURL[url6][http://xpmethod.plaintext.in/][][Columbia's Group for Experimental Methods in the Humanities]\from[url6], and the Studio@Butler at Columbia University.

\subsection{Kelly Baker Josephs}

\useURL[url7][https://www.york.cuny.edu/portal_college/kjosephs][][Kelly Baker Josephs]\from[url7] is Associate Professor of English at York College, CUNY. She is the author of \useURL[url8][http://www.upress.virginia.edu/title/4572][][{\em Disturbers of the Peace: Representations of Insanity in Anglophone Caribbean Literature}]\from[url8] (2013), editor of \useURL[url9][http://smallaxe.net/sxsalon][][{\em sx salon: a small axe literary platform}]\from[url9], and manager of \useURL[url10][http://caribbean.commons.gc.cuny.edu][][{\em The Caribbean Commons}]\from[url10] website. Her current project, "Caribbean Articulations: Storytelling in a Digital Age, explores the intersections between new technologies and Caribbean cultural production.

\stopchapter
\stoptext