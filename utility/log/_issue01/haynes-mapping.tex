\setvariables[article][shortauthor={Haynes}, date={May 2016}, issue={1}, DOI={10.7916/D8HM58J8}]

\setupinteraction[title={Mapping Caribbean Cyberfeminisms},author={Tonya Haynes}, date={May 2016}, subtitle={Mapping Caribbean Cyberfeminisms}]
\environment env_journal


\starttext


\startchapter[title={Mapping Caribbean Cyberfeminisms}
, marking={Mapping Caribbean Cyberfeminisms}
, bookmark={Mapping Caribbean Cyberfeminisms}]


\startlines
{\bf
Tonya Haynes
}
\stoplines


{\startnarrower\it Caribbean cyberfeminisms are diverse, heterogeneous, and polyvocal. Networks may be simultaneously national, regional, global, transnational, and diasporic. Through practices of media creation, curating, (re)blogging, (re)tweeting, sharing, and commenting across multiple social media platforms, Caribbean feminist knit together an online community that is often linked to on-the-ground organizing and action. Online feminist practices are therefore rich archives for the study of Caribbean feminisms. To date, scholarly work on women's and feminist movements in the region has failed to document and analyze these practices and sites of activism. Similarly, Caribbean feminist critiques of technology and new media are not well developed. The essay attends to this gap by offering a partial and preliminary mapping Caribbean cyberfeminisms primarily through documentation and analysis of Caribbean feminist blogs. \stopnarrower}

\blank[2*line]
\blackrule[width=\textwidth,height=.01pt]
\blank[2*line]

The murder of Japanese pannist Asami Nagakiya at Trinidad and Tobago's carnival generated international headlines. The mayor of Port of Spain, Raymond Tim Kee, seemed to blame Nagakiya for her own death, couching the violence against her as a result of \quotation{vulgarity and lewdness in conduct,} underscoring that \quotation{women have a responsibility to ensure they are not abused.}\footnote{Sean Douglas, \quotation{Mayor: Beware Dangerous Sub-cultures,} {\em Trinidad and Tobago Newsday}, 11 February 2016, \useURL[url1][http://www.newsday.co.tt/news/0,223841.html][][http://www.newsday.co.tt/news/0,223841.html.]\from[url1]} Trinidadian women (and men) assembled outside City Hall to demand the mayor's resignation. They also shifted the public conversation on gender-based violence away from what they deemed \quotation{victim-blaming} and \quotation{slut-shaming} toward one of state accountability and respect for women's autonomy and bodily integrity. The leaders of the feminist organization womantra\footnote{(Correction issued 04/31/2019) While WOMANTRA was a key space for mobilisation of women, the protest was not organised by WOMANTRA's leadership but rather by activists Angelique Nixon and Atillah Springer.} were among those identified as coordinating the protest action and media engagement. Womantra is one of the most vibrant, feminist Pan-Caribbean online spaces linked to on-the-ground action in Trinidad and Tobago. Such spaces serve as a key ground for mobilizing Caribbean feminist constituencies, as evidenced by their street protest and online petition with more than 10,500 signatures.

A Jamaican government minister dismissed his country's Twitter users as an \quotation{articulate minority} that does not express the views of \quotation{ordinary Jamaicans,}\footnote{\quotation{Ordinary Jamaicans Know Nothing about Twitter, Says Pickersgill,} {\em Jamaica Observer}, 26 November 2014, \useURL[url2][http://www.jamaicaobserver.com/news/Ordinary-Jamaicans-know-nothing-about-Twitter--says-Pickersgill][][http://www.jamaicaobserver.com/news/Ordinary-Jamaicans-know-nothing-about-Twitter--says-Pickersgill.]\from[url2]} and the noted Caribbean feminist activist Hazel Brown, in response to her perception of a lack of civic activism, is quoted as lamenting, \quotation{People are writing blogs, but what are we doing?}\footnote{\quotation{Hazel Brown: Making a Difference,} {\em Newsday}, 21 April 2013, \useURL[url3][http://www.newsday.co.tt/features/0,176606.html][][http://www.newsday.co.tt/features/0,176606.html.]\from[url3]} While the perception that Caribbean people are not using online technologies for significant civic participation is one shared by scholars, I seek to challenge such assessments.\footnote{Pearson Broome and Emmanuel Adugu, \quotation{Whither Social Media for Digital Activism: The Case of the Caribbean,} {\em British Journal of Education, Society, and Behavioural Science} 10, no. 3 (2015): 1--21; Kristina Hinds Harrison, \quotation{Virtual Shop Fronts: The Internet, Social Media, and Caribbean Civil Society Organisations,} {\em Globalizations} 11, no. 6 (2014): 751--66.} Studies of Caribbean culture and literature are finally acknowledging the significance of our online-media-saturated worlds, as evidenced by recent research on Afro-Dominican lesbian feminist organizing, Caribbean rhetoric, online cultures, and literatures.\footnote{Marcia A. Forbes, {\em Streaming}, vol.~1, {\em \#Social Media, Mobile Lifestyles} (Kingston: Phase Three, 2012); Daniel Miller and Don Slater, {\em The Internet: An Ethnographic Approach} (Oxford: Berg, 2000); Daniel Miller, {\em Tales from Facebook} (Cambridge: Polity, 2011); Kevin Adonis Browne, {\em Tropic Tendencies: Rhetoric, Popular Culture, and the Anglophone Caribbean} (Pittsburgh: University of Pittsburgh Press, 2013); Bernard Jankee, \quotation{The Word in Cyberspace: Constructing Jamaican Identity on the Internet,} in Annie Paul, ed., {\em Caribbean Culture: Soundings on Kamau Brathwaite} (Kingston: University of the West Indies Press, 2007); Annie Paul, \quotation{Log On: Towards Social and Digital Islands,} in Michael Bucknor and Alison Donnell, eds., {\em The Routledge Companion to Anglophone Caribbean Literature} (Abingdon, UK: Routledge, 2011); Rachel Afi Quinn, \quotation{This Bridge Called the Internet: Black Lesbian Feminist Organizing in Santo Domingo,} in Cheryl R. Rodriguez, Dzodzi Tsikata, and Akosua Adomako Ampofo, eds., {\em Transatlantic Feminisms Women and Gender Studies in Africa and the Diaspora} (Lanham: Lexington Books, 2015).} Building on this move in Caribbean cultural studies, I center Caribbean cyberfeminisms as knowledge-producing spaces of political thought and action that may at times either bridge or reinforce a digital divide but that nonetheless deserve documentation.\footnote{For a discussion on the digital divide in Trinidad and Tobago, see Bheshem Ramlal and Patrick Watson, \quotation{The Digital Divide in Trinidad and Tobago,} {\em Social and Economic Studies} 63, no. 1 (2014): 1--23.} Such documentation, of feminist blogging in particular, is the focus of my attempt to offer a necessarily preliminary and partial mapping of Caribbean cyberfeminisms. Through an analysis of feminist blogs, I make a claim for the existence of Caribbean cyberfeminism as, in Susanna Paasonen's words, \quotation{a critical feminist position for interrogating and intervening in specific technological forms and practices,} visible in the ways bloggers challenge normative and hegemonic configurations of both feminism and digital technologies.\footnote{Susanna Paasonen, \quotation{Rethinking Cyberfeminisn,} {\em Communications} 36, no. 3 (2011): 340.}

\subsection[title={Introducing Caribbean Feminist Cyberspace},reference={introducing-caribbean-feminist-cyberspace}]

Nicholas Laughlin (and the bloggers who left comments on his article) traces key moments in Caribbean blogging, going back to at least 2000 with Taran Rampersad blogging from Trinidad and Tobago and the first blog post from Mad Bull (a Jamaican living the Cayman Islands) in 2001. \useURL[url4][http://sapodilla.blogspot.com][][Guyana Gyal]\from[url4] (Neena Maiya), described by Laughlin as \quotation{the first Caribbean blogger to make a point of writing consistently in a \quote{non-standard local dialect,}} began blogging in 2005.\footnote{Nicholas Laughlin, \quotation{Eleven Key Moments in {[}Anglo-{]}Caribbean Blog History,} {\em Global Voices}, 13 January 2006, \useURL[url5][http://globalvoicesonline.org/2006/01/13/11-key-moments-in-anglo-caribbean-blog-history][][http://globalvoicesonline.org/2006/01/13/11-key-moments-in-anglo-caribbean-blog-history.]\from[url5]} As one of the strongest female Caribbean voices who has been blogging consistently, Guyana Gyal is clearly a pioneer of Caribbean feminist blogging. Another pioneer, the Barbadian woman behind {\em Titilayo}, a blog no longer active or available online, also espoused feminist perspectives beginning in 2001. Thus Caribbean women as bloggers and Caribbean feminist blogs specifically are integral to a history of Caribbean blogging.

Online Caribbean feminisms are extremely diverse, heterogeneous, and polyvocal. Networks may be simultaneously regional, national, and global, or transnational and diasporic. Through practices of media creation, curating, reblogging, retweeting, sharing, and commenting across multiple social media platforms, Caribbean feminists knit together online communities that are often linked to on-the-ground organizing and action. The online feminist activism of Caribbean people has caused major manufacturers to pull advertisements deemed offensive by the community, forced dialogue about and ultimately police investigations into the practices of journalists, and shut down websites with sexist content. Below are key signposts of the significance, emergence, and diversity of Caribbean cyberfeminisms:

\startitemize
\item
  Queen Macoomeh initiated an \useURL[url6][http://www.change.org/p/we-did-it-angostura-has-said-they-will-pull-the-ad-thank-you-all][][online petition]\from[url6] against an Angostura advertisement with the tag line \quotation{Avoid the friendzone, offer her a real drink.} Angostura subsequently removed the advertisement.\footnote{See \useURL[url7][http://www.change.org/p/we-did-it-angostura-has-said-they-will-pull-the-ad-thank-you-all][][http://www.change.org/p/we-did-it-angostura-has-said-they-will-pull-the-ad-thank-you-all.]\from[url7]} The public protest against the advertisement, especially via the online petition, was reported by international media outlets such as Buzzfeed.\footnote{See \useURL[url8][http://www.buzzfeed.com/copyranter/another-liquor-company-creates-a-rapey-ad]\from[url8]}
\item
  Malaika Brooks-Smith-Lowe of \useURL[url9][http://groundationgrenada.com][][Groundation Grenada]\from[url9] and colleagues at the Goat Dairy Project have used crowdfunding to raise US\$63,160 to support a community agriculture project.\footnote{Malaika Brooks-Smith-Lowe, \quotation{Kickstarting the Goat Dairy in Grenada,} {\em Guardian}, 30 August 2013, \useURL[url10][http://www.theguardian.com/global-development-professionals-network/2013/aug/30/kickstarter-goat-dairy-grenada][][http://www.theguardian.com/global-development-professionals-network/2013/aug/30/kickstarter-goat-dairy-grenada.]\from[url10]}
\item
  \useURL[url11][http://womantratt.wix.com/home][][WOMANTRA]\from[url11] won a 2013 grant from the FRIDA Young Feminist Fund to organize a summer camp for girls transitioning to secondary school in Trinidad and Tobago. This demonstrates that communities that have their genesis online may incubate others forms of political organizing and intervention.\footnote{See \useURL[url12][http://womantratt.wix.com/home]\from[url12].}
\item
  Patrice Daniel, a twenty-nine-year-old feminist from Barbados and frequent contributor to International Planned Parenthood Federation's blog, generated what I consider to be a key moment in Caribbean feminist blogging history. Her \quotation{\useURL[url13][http://rhrealitycheck.org/article/2013/03/12/an-open-letter-to-caribbean-men-from-caribbean-women][][An Open Letter to Caribbean Men, from Caribbean Women]\from[url13]} went viral, with over 10,000 shares on Facebook, demonstrating the extent to which her words resonated not only with Caribbean women but with women across the globe.\footnote{Patrice Daniel, \quotation{An Open Letter to Caribbean Men, from Caribbean Women,} {\em RH Reality Check}, 12 March 2013, \useURL[url14][http://rhrealitycheck.org/article/2013/03/12/an-open-letter-to-caribbean-men-from-caribbean-women][][http://rhrealitycheck.org/article/2013/03/12/an-open-letter-to-caribbean-men-from-caribbean-women.]\from[url14]}
\item
  Caribbean women, many of whom identify as feminists and are active in feminist and lesbian, gay, bisexual, and transgender (LGBT) organizing have created online networks for \quotation{women who love women.} LGBT organizations such as Guyana's \useURL[url15][http://www.sasod.org.gy][][Society Against Sexual Orientation Discrimination]\from[url15], Trinidad and Tobago's \useURL[url16][http://www.facebook.com/caiso][][Coalition Advocating for Inclusion of Sexual Orientation]\from[url16], and St.~Lucia's \useURL[url17][http://unitedandstrongstlucia.wordpress.com][][United and Strong]\from[url17] maintain a significant online presence.\footnote{See \useURL[url18][http://www.sasod.org.gy]\from[url18]; \useURL[url19][http://www.facebook.com/caiso]\from[url19]; and \useURL[url20][http://unitedandstrongstlucia.wordpress.com][][http://unitedandstrongstlucia.wordpress.com.]\from[url20]}
\item
  \useURL[url21][http://www.facebook.com/AirMeNow][][{\em Air Me Now}]\from[url21] is a \useURL[url22][http://www.youtube.com/playlist?list=PLLNiArTOn_azPDdIR4FvkyepgRQrb9K8Y][][YouTube show]\from[url22] hosted by women from Belize, the Bahamas, and Jamaica that highlights \quotation{Caribbean women's voices, Caribbean women's issues, Caribbean {\sc oomanism}.}\footnote{See \useURL[url23][http://www.facebook.com/AirMeNow]\from[url23] and \useURL[url24][http://www.youtube.com/playlist?list=PLLNiArTOn_azPDdIR4FvkyepgRQrb9K8Y][][http://www.youtube.com/playlist?list=PLLNiArTOn_azPDdIR4FvkyepgRQrb9K8Y.]\from[url24]}
\item
  Gender consciousness has also meant broader public engagement with questions of gender and sexuality, like that offered by the YouTube video created by a group of young Jamaican men in response to the rape of five women and girls. The video, titled \useURL[url25][http://www.youtube.com/watch?v=IS8uJ9dol5s][][{\em Jamaican Anti-Rape Campaign (Please Share)}]\from[url25], has received almost 40,000 views.\footnote{Danar Royal, dir., {\em Jamaica Anti-Rape Campaign (Please Share)}, 2:15, posted 8 October 2012, \useURL[url26][https://youtu.be/IS8uJ9dol5s]\from[url26]. For a broader discussion on gender consiousness, see Patricia Mohammed, \quotation{Like Sugar in Coffee: Third Wave Feminism and the Caribbean,} {\em Social and Economic Studies} 52, no. 3 (2003): 5--30.}
\item
  The social media efforts of Ashlee Hinds, a twenty-three-year-old Barbadian student, to create positive images of fat black women resulted in a \useURL[url27][http://www.bigbeautifulblackgirls.com][][transnational fashion blog]\from[url27] and \useURL[url28][http://www.facebook.com/BigBeautifulBlackGirls][][Facebook page]\from[url28] titled \useURL[url29][http://bigbeautifulblackgirls.tumblr.com][][{\em Big Beautiful Black Girls}]\from[url29], with 217,434 followers.\footnote{See \useURL[url30][http://www.bigbeautifulblackgirls.com]\from[url30], \useURL[url31][http://bigbeautifulblackgirls.tumblr.com]\from[url31], and \useURL[url32][http://www.facebook.com/BigBeautifulBlackGirls][][http://www.facebook.com/BigBeautifulBlackGirls.]\from[url32]}
\stopitemize

These are but a few examples of the significance of Caribbean cyberfeminist practices.

\subsection[title={Methods},reference={methods}]

In 2014 I hosted a Caribbean blog carnival titled \useURL[url33][file:///C:\Users\Kelly\Documents\sx\%20archipelagos\%20May\%202016\returned\redforgender.wordpress.com\e-mas-caribbean-blog-carnival][][{\em e-Mas: To the Caribbean, with Love}]\from[url33]. Using the hashtag \#dearCaribbean, contributors from Venezuela, Curacao, and St.~Kitts and Nevis, as well as from the wider Caribbean and its diaspora, submitted entries that were hosted on their own blogs as well as on \useURL[url34][https://redforgender.wordpress.com/][][{\em Feminist Conversations on Caribbean Life}]\from[url34].\footnote{See \useURL[url35][http://redforgender.wordpress.com/e-mas-caribbean-blog-carnival][][http://redforgender.wordpress.com/e-mas-caribbean-blog-carnival.]\from[url35]} {\em Global Voices} reviewed the series, as did Trinidadian writer Shivanee Ramlochan, whose review was published in the {\em Trinidad} {\em Guardian} newspaper. US-based feminist media site Feministing.com also featured the series.\footnote{Matthew Hunte, \quotation{Blog Carnival Shows the Caribbean Some Love,} {\em Global Voices}, 4 February 2014, \useURL[url36][http://globalvoices.org/2014/02/04/blog-carnival-shows-the-caribbean-some-love][][http://globalvoices.org/2014/02/04/blog-carnival-shows-the-caribbean-some-love;]\from[url36] Shivanee Ramlochen, \quotation{Bloggers' Paradise,} {\em Trinidad} {\em Guardian}, 16 February 2014, B29, \useURL[url37][http://digital.guardian.co.tt/default.aspx?iid=87563&startpage=page0000101\#folio=100][][http://digital.guardian.co.tt/default.aspx?iid=87563&startpage=page0000101\#folio=100;]\from[url37] Juliana Britto Schwartz, \quotation{This Week in Feminism South of the Border,} Feministing.com, 31 January 2014, \useURL[url38][http://feministing.com/2014/01/31/this-week-in-feminism-south-of-the-border-2][][http://feministing.com/2014/01/31/this-week-in-feminism-south-of-the-border-2.]\from[url38]} {\em E-Mas} highlighted Amina Doherty's photography; Jermaine Ostiana's poem \quotation{Trujillonomics}; Carla Moore's video {\em On Loving Yaad, Leaving Yaad, and Why Some ah We Waan Run Come Back}; Stephanie Leitch's reflections on her queer Baby-Doll mas'; and Gabrielle Hosein's essay \quotation{A Risky Location: What It Means to Be an Indian Feminist in Our Region.} Ramlochan, herself a prolific digital griot, said of the collection: \quotation{These are some of the best and brightest in Caribbean letters, writing letters back home. They may often put forth theories that blast old, time honoured paradigms to shred, indeed, this is one of their most vital advantages. Digital space makes such collaborations not just possible but easily navigable. This blog carnival is proof that room can be constructed, in safe space, for every voice to interrogate its way toward truth.}\footnote{Ramlochen, \quotation{Bloggers' Paradise.}} Ramlochan's analysis suggests that these \quotation{digital griots} must be taken on their own terms.\footnote{Tzarina T. Prater, \quotation{\quote{Look Pon Likkle Chiney Gal}: Tessanne Chin, the Voice, and Digital Caribbean Subjects,} {\em Anthurium: A Caribbean Studies Journal} 12, no. 1 (2015): article 11.} The experience of hosting the \quotation{blog carnival} underscored for me the importance of documenting and analyzing online Caribbean feminisms.

This partial and tentative attempt to map the diverse networks, spaces, and actors that constitute Caribbean online feminisms draws on my own experiences of maintaining a personal blog since 2006 and managing {\em Feminist Conversations on Caribbean Life} since 2010. It is therefore limited by my own networks. It is also linguistically limited, since it draws largely on media and actors published in English and English-based Creoles and dialects, though that does not preclude the multilingual abilities of some of the bloggers nor suggest that English is the official language of the countries from which these bloggers hail. There is reference to Cuban bloggers, but a more robust discussion on Cuban cyberfeminisms is outside of the scope of this essay and can be found elsewhere.\footnote{See Sandra Abd'Allah-Alvarez Ramírez, \quotation{¿Ciberfeminismo en Cuba?,} paper presented at \quotation{Towards a New Social Contract?,} 31st International Congress of the Latin American Studies Association, 30 May--1 June 2013, Washington, DC.} Despite these obvious limitations, this mapping is still useful for understanding Caribbean feminisms in the twenty-first century and for understanding cyberfeminisms as a global and globalized phenomenon.

I selected thirty-six key Caribbean feminist blogs for inclusion in the analysis. The majority of these blogs either expressly identify as feminist or reveal a feminist, gender-conscious politics in the nature of their content. The list is also cross-referenced with the twenty-eight bloggers who chose to register their blogs with the CatchAFyah Caribbean Feminist Network, though not all of those blogs were included in the analysis. The ones included represent almost exclusively blogs that belong to Caribbean women and men. Caribbean bloggers, however, also publish on blogs belonging to regional or international organizations. The blogs highlighted in the analysis are largely owned and maintained by the bloggers themselves.

As part of this documentation of Caribbean bloggers, I sought to establish the national identities of these bloggers. Despite the claim that representing the nation is a key part of Caribbean people's online activities and identities, determining the nationality of a specific blog or blogger is not always straightforward.\footnote{See Miller and Slater, {\em The Internet}.} While \quotation{representing the nation} is important to the online personality and content of some bloggers (e.g., Jamaica's \useURL[url39][http://mooretalkja.wordpress.com][][Carla Moore]\from[url39]) and for others their nationality is signaled in their blog title (e.g., \useURL[url40][http://negracubanateniaqueser.wordpress.com/][][{\em Negra Cubana tenia que ser}]\from[url40]), some blogs are best understood as Pan-Caribbean in scope (e.g., \useURL[url41][http://rootsandrights.wordpress.com][][{\em Roots and Rights}]\from[url41] or \useURL[url42][http://www.antillean.org][][{\em Antillean Media Group}]\from[url42]), as diasporic (e.g., \useURL[url43][http://battymamzelle.blogspot.com][][{\em BattyMamzelle}]\from[url43]), or as multicountry projects (e.g., \useURL[url44][http://addfyahandstir.wordpress.com/][][{\em Add Fyah and Stir}]\from[url44]).

In seeking to discover the issues most important to Caribbean cyberfeminists, I undertook a content analysis of the blogs. I decided to focus exclusively on the text-based blogs that were still active and had updated at least in July 2015, excluding the blog to which I regularly contribute and those not written in English. This gave a total of nineteen blogs, of which fifteen are owned by women and three by men, and one has multiple contributors. For convenience, I selected the five most recent posts on each blog published in 2015 and included them in the thematic analysis. This yielded some ninety-five blog posts that were read closely for recurring topics.

Taking seriously Ramlochan's assertion that the Caribbean's digital subjects must be understood on their own terms, I sought to categorize the thirty-six blogs in the sample. The categories I propose as useful for analyzing Caribbean feminist bloggers are, as much as possible, drawn from the bloggers' own descriptions of their online practices.

\subsection[title={Categorizing Caribbean Cyberfeminisms},reference={categorizing-caribbean-cyberfeminisms}]

{\em Personal-is-political} is used to refer to personal blogs that draw on personal experience to illustrate or analyze relations of power in wider society. For example, in outlining the reason she decided to start blogging, \useURL[url45][http://grrlscene.wordpress.com][][Gabrielle Hosein]\from[url45] argues that the \quotation{\quote{small p} politics that reflects just living a woman's life} while rendered trivial in sexist imagination is, in fact, a source of women's theorizing:

\startblockquote
i'm writing to live. to stand still instead of running from one end of the seesaw to the other, constantly trying to keep the whole in balance. the words do not express the \quote{big P} politics of public revolutions and realisations, but a consciousness of \quote{small p} politics that reflects just living a woman's life. . . .

so much of women's lives, time, work, concerns are considered trivial, yet they are momentous . . . if only to our own self. in this spirit, i start this diary, charting the daily negotiation of managing work and family, for as long as i can and perhaps as long as i need.\footnote{Gabrielle Hosein, \quotation{Momentous Trivialities,} {\em Diary of a Mothering Worker}, \useURL[url46][http://grrlscene.wordpress.com/momentous-trivialities-diary-of-a-mothering-worker][][http://grrlscene.wordpress.com/momentous-trivialities-diary-of-a-mothering-worker.]\from[url46]}
\stopblockquote

Personal-is-political blogs therefore represent a continuation of women's and feminist uses of diarying and journaling as a way of making sense of the world and of insisting that women's experiences are significant sources of knowledge.

I use the term {\em culture critic} to categorize blogs focused on cultural and sociological analyses of relations of power such as gender, race, class, and heterosexism and inequalities broadly. These blogs provide social, political, and economic analyses of everyday life and (popular) culture. Soyini Ayanna's description of her blog \useURL[url47][http://soyluv.wordpress.com][][{\em Creative Commess}]\from[url47] makes explicit her role as culture critic: \quotation{I frequently think about and critique popular West Indian & Trinbagonian cultural trends, soca trends and what I see taking place.}\footnote{Soyini Ayanna, \quotation{About Me,} {\em Creative Commess}, \useURL[url48][http://soyluv.wordpress.com/about][][http://soyluv.wordpress.com/about.]\from[url48]} Likewise, the tag line to Annie Paul's \useURL[url49][http://anniepaul.net][][{\em Active Voice}]\from[url49]---\quotation{sharp, pointed, often witty commentary on current events in Jamaica, the Caribbean, India and the world}---also suggests her role as culture critic.\footnote{Annie Paul, {\em Active Voice}, \useURL[url50][http://anniepaul.net][][http://anniepaul.net.]\from[url50]}

Personal-is-political and culture critic emerged as the most popular classifications and are used to categorize half the blogs. This reflects the extent to which Caribbean feminists are using blogging as a means of analyzing and creating knowledge and dialogue around topical issues in their countries and region. Most personal-is-political blogs are also culture critic blogs; the distinction is that the personal-is-political blogs use personal experience to illustrate, analyze, and comment on broader sociocultural relations of gender, race, class, and sexuality.

{\em +Feminism} is perhaps the most difficult category to conceptualize, though it is the second-most popular. It refers to blogs that do not have an exclusive or primary focus on feminism, feminist perspectives, gender, or sexuality but that do include either writing from a feminist perspective or posts about women's, feminist, and LGBT issues. For example, the \useURL[url51][http://www.antillean.org][][Antillean Media Group]\from[url51] is \quotation{a caucus of journalists, researchers and policy analysts dedicated to the stories that impact the Caribbean.}\footnote{\quotation{Building the New Caribbean Media,} {\em Antillean Media Group}, \useURL[url52][https://web.archive.org/web/20141220170855/http://www.antillean.org:80/about/][][http://www.antillean.org/about.]\from[url52]} Their coverage of Caribbean stories includes issues of gender relations, covered from a profeminist perspective. However, they also address a broad range of issues from multiple perspectives.

{\em Media crossover} is used to refer to blogs that contain content published both in mainstream and online media and private blogs. It includes blogs by journalists and columnists as well as bloggers whose online content is republished in national newspapers or bloggers who have attracted significant mainstream media attention. This categorization is important because it demonstrates new media's imbrications with older forms of media. For example, the blogger Gabrielle Hosein saw her personal-is-political blog attract mainstream attention and subsequently be reprinted as a weekly column. Meanwhile, in the face of uncertainty over whether his weekly column would be renewed by the Trinidad {\em Guardian}, \useURL[url53][http://onenationmanybodies.wordpress.com][][Colin Robinson]\from[url53] continued to make articles available online via his blog. Thus the crossing implied by the category of media crossover suggests multidirectional flows. It also suggests the increasing salience of gender consciousness and feminist thought in Caribbean public space. When the media crossover blogs are added to this list of +feminism blogs, it reflects an important and growing infiltration of feminist ideas into the mainstream, demonstrating the extent to which feminist values, viewpoints, perspectives, and issues form part of a gender-conscious public conversation.

Likewise, while the feminist academics who blog do not necessarily self-identify primarily as academics, I feel that the category of {\em feminist-academic} is important, since it reveals the overlap between digital and academic spaces. Seven of the blogs are categorized as feminist-academic blogs. This demonstrates the extent to which feminists in the academy have been using online media to produce and share knowledge on a variety of issues. This number doubles when I include the blogs of those of current or former students for whom courses in women's/gender/feminist studies at universities in the Caribbean and overseas have been crucial to their own identities and consciousness as feminists. This demonstrates the importance of feminist academia to Caribbean cyberfeminisms. It also suggests that understanding privilege, class, and elitism is critical to unlocking Caribbean cyberfeminisms. To the extent that Caribbean cyberfeminisms serve to reinforce rather than challenge asymmetries of knowledge and power, their subversive potential is undermined.

{\em Curator} describes blogs that gather and reproduce content published elsewhere or curate original content from a variety of contributors (e.g., WomenSpeak Project's \useURL[url54][http://womenspeak.tumblr.com][][{\em Women Speak}]\from[url54], which solicits submissions).\footnote{WomenSpeak Project, {\em Women Speak}, \useURL[url55][http://womenspeak.tumblr.com][][http://womenspeak.tumblr.com.]\from[url55]} The blog \useURL[url56][http://repeatingislands.com][][{\em Repeating Islands}]\from[url56] describes itself as \quotation{a project intended to bring the broader Caribbean community closer through the sharing of news and information that transcends the linguistic divide in the region.}\footnote{Ivette Romero-Cesareo and Lisa Paravisini-Gebert, {\em Repeating Islands}, \useURL[url57][http://repeatingislands.com/our-blog][][http://repeatingislands.com/our-blog.]\from[url57]} This information sharing is done through aggregating and curating news published on multiple Caribbean news sites and serves the political mission of working across the divides of imperial languages. Included in this work of curating are also practices of reblogging, sharing, and citing the work of other Caribbean cyberfeminists. For example, included among the links, images, stories, and responses to reader questions that \useURL[url58][http://bad-dominicana.tumblr.com][][Bad Dominicana]\from[url58] shares on her blog is a quote from a blog post by \useURL[url59][http://battymamzelle.blogspot.com][][BattyMamzelle]\from[url59] titled, \quotation{This Is What I Mean When I Say \quote{White Feminism.}}\footnote{Zahira Kelly (Bad Dominicana) quotes Cate Young (BattyMamzelle), \quotation{This Is What I Mean When I Say 'White Feminism,} imploring, \quotation{Please everybody: read this article!} \useURL[url60][http://bad-dominicana.tumblr.com/post/138419448358/white-feminism-does-not-mean-every-white-woman][][http://bad-dominicana.tumblr.com/post/138419448358/white-feminism-does-not-mean-every-white-woman;]\from[url60] {\em BattyMamzelle}, 10 January 2014, \useURL[url61][http://battymamzelle.blogspot.com/2014/01/This-Is-What-I-Mean-When-I-Say-White-Feminism.html\#.VwxPjXqYuCf][][http://battymamzelle.blogspot.com/2014/01/This-Is-What-I-Mean-When-I-Say-White-Feminism.html\#.VwxPjXqYuCf.]\from[url61]} Here, Caribbean feminists build consciousness-raising communities online, share knowledge, and think collectively through these practices of curating, linking, and reblogging each other's work.

{\em Witty} refers to blogs from writers for whom wit, wordplay, humor, and sarcasm function as key rhetorical devices and serve to make their media both more accessible and appealing to a wide readership/viewership. Annie Paul's tagline promises \quotation{witty commentary,} and \useURL[url62][http://mongoosechronicles.blogspot.com][][Mar the Mongoose]\from[url62]'s \quotation{tales of a fierce, agile West Indian mongoose who moves quickly and sees much} also suggests a quick wit.\footnote{Annie Paul, {\em Active Voice}, \useURL[url63][http://anniepaul.net]\from[url63]; Mar the Mongoose, {\em The Mongoose Chronicles}, \useURL[url64][http://mongoosechronicles.blogspot.com][][http://mongoosechronicles.blogspot.com.]\from[url64]} While only five of the thirty-six blogs examined have been characterized as witty, it remains an important category for understanding feminist rhetorical strategies.

\startplacetable[title={Caribbean Feminist Blogs and Bloggers}]
\startxtable
\startxtablehead[head]
\startxrow
\startxcell[width={0.26\textwidth}] Blog Name \stopxcell
\startxcell[width={0.18\textwidth}] Blogger \stopxcell
\startxcell[width={0.23\textwidth}] Country \stopxcell
\startxcell[width={0.33\textwidth}] Categorization \stopxcell
\stopxrow
\stopxtablehead
\startxtablebody[body]
\startxrow
\startxcell[width={0.26\textwidth}] \useURL[url65][http://amedjafifa.wordpress.com][][1981: A Record of What Happened]\from[url65] \stopxcell
\startxcell[width={0.18\textwidth}] Some of It/DJ Afifa Aza \stopxcell
\startxcell[width={0.23\textwidth}] Jamaica \stopxcell
\startxcell[width={0.33\textwidth}] Personal-is-political, culture critic \stopxcell
\stopxrow
\startxrow
\startxcell[width={0.26\textwidth}] \useURL[url66][http://anniepaul.net][][Active Voice]\from[url66] \stopxcell
\startxcell[width={0.18\textwidth}] Annie Paul \stopxcell
\startxcell[width={0.23\textwidth}] Jamaica \stopxcell
\startxcell[width={0.33\textwidth}] +Feminism, culture critic, witty \stopxcell
\stopxrow
\startxrow
\startxcell[width={0.26\textwidth}] \useURL[url67][http://addfyahandstir.wordpress.com][][Add Fyah and Stir]\from[url67] \stopxcell
\startxcell[width={0.18\textwidth}]  \stopxcell
\startxcell[width={0.23\textwidth}] Antigua & Barbuda/St.~Kitts--Nevis/T&T \stopxcell
\startxcell[width={0.33\textwidth}] Personal-is-political \stopxcell
\stopxrow
\startxrow
\startxcell[width={0.26\textwidth}] \useURL[url68][http://www.amilcarsanatan.com/blog][][Amílcar Sanatan]\from[url68] \stopxcell
\startxcell[width={0.18\textwidth}]  \stopxcell
\startxcell[width={0.23\textwidth}] Trinidad & Tobago \stopxcell
\startxcell[width={0.33\textwidth}] +Feminism \stopxcell
\stopxrow
\startxrow
\startxcell[width={0.26\textwidth}] \useURL[url69][http://www.antillean.org][][Antillean Media Group]\from[url69] \stopxcell
\startxcell[width={0.18\textwidth}]  \stopxcell
\startxcell[width={0.23\textwidth}] Caribbean \stopxcell
\startxcell[width={0.33\textwidth}] +Feminism \stopxcell
\stopxrow
\startxrow
\startxcell[width={0.26\textwidth}] \useURL[url70][http://bad-dominicana.tumblr.com][][The Bad Dominicana]\from[url70] \stopxcell
\startxcell[width={0.18\textwidth}] Zahira Kelly/Bad Dominicana \stopxcell
\startxcell[width={0.23\textwidth}] Dominican Republic \stopxcell
\startxcell[width={0.33\textwidth}] curator \stopxcell
\stopxrow
\startxrow
\startxcell[width={0.26\textwidth}] \useURL[url71][http://battymamzelle.blogspot.com][][BattyMamzelle]\from[url71] \stopxcell
\startxcell[width={0.18\textwidth}] Cate Young/BattyMamzelle \stopxcell
\startxcell[width={0.23\textwidth}] Trinidad & Tobago/US \stopxcell
\startxcell[width={0.33\textwidth}] Culture critic, personal-is-political \stopxcell
\stopxrow
\startxrow
\startxcell[width={0.26\textwidth}] \useURL[url72][http://thecontessawearsnoshoes.blogspot.com][][The Contessa Wears No Shoes]\from[url72] \stopxcell
\startxcell[width={0.18\textwidth}]  \stopxcell
\startxcell[width={0.23\textwidth}] Trinidad & Tobago \stopxcell
\startxcell[width={0.33\textwidth}] Personal-is-political \stopxcell
\stopxrow
\startxrow
\startxcell[width={0.26\textwidth}] \useURL[url73][http://soyluv.wordpress.com][][Creative Commess]\from[url73] \stopxcell
\startxcell[width={0.18\textwidth}] Soyini Ayanna \stopxcell
\startxcell[width={0.23\textwidth}] Trinidad & Tobago/US \stopxcell
\startxcell[width={0.33\textwidth}] Culture critic, personal-is-political \stopxcell
\stopxrow
\startxrow
\startxcell[width={0.26\textwidth}] \useURL[url74][http://grrlscene.wordpress.com][][Diary of a Mothering Worker]\from[url74] \stopxcell
\startxcell[width={0.18\textwidth}] Gabrielle Hosein \stopxcell
\startxcell[width={0.23\textwidth}] Trinidad & Tobago \stopxcell
\startxcell[width={0.33\textwidth}] Feminist-academic, media crossover, personal-is-political, lture critic \stopxcell
\stopxrow
\startxrow
\startxcell[width={0.26\textwidth}] Feminist Aliens \stopxcell
\startxcell[width={0.18\textwidth}]  \stopxcell
\startxcell[width={0.23\textwidth}] Barbados/South Africa/US \stopxcell
\startxcell[width={0.33\textwidth}] Culture critic, feminist-academic, curator, personal-is-political \stopxcell
\stopxrow
\startxrow
\startxcell[width={0.26\textwidth}] \useURL[url75][http://redforgender.wordpress.com][][Feminist Conversations on Caribbean Life]\from[url75] \stopxcell
\startxcell[width={0.18\textwidth}]  \stopxcell
\startxcell[width={0.23\textwidth}] Barbados/Caribbean \stopxcell
\startxcell[width={0.33\textwidth}] Feminist-academic, culture critic \stopxcell
\stopxrow
\startxrow
\startxcell[width={0.26\textwidth}] \useURL[url76][http://freedombyanymeans.wordpress.com][][Freedom by Any Means]\from[url76] \stopxcell
\startxcell[width={0.18\textwidth}] Sherlina Nageer \stopxcell
\startxcell[width={0.23\textwidth}] Guyana \stopxcell
\startxcell[width={0.33\textwidth}] Media crossover, culture critic, personal-is-political \stopxcell
\stopxrow
\startxrow
\startxcell[width={0.26\textwidth}] \useURL[url77][http://generacionyen.wordpress.com][][Generación Y]\from[url77] \stopxcell
\startxcell[width={0.18\textwidth}] Yoani Sánchez \stopxcell
\startxcell[width={0.23\textwidth}] Cuba \stopxcell
\startxcell[width={0.33\textwidth}] +Feminism, personal-is-political, culture critic, media crossover \stopxcell
\stopxrow
\startxrow
\startxcell[width={0.26\textwidth}] \useURL[url78][http://groundationgrenada.com][][Groundation Grenada]\from[url78] \stopxcell
\startxcell[width={0.18\textwidth}]  \stopxcell
\startxcell[width={0.23\textwidth}] Grenada \stopxcell
\startxcell[width={0.33\textwidth}] +Feminism, culture critic \stopxcell
\stopxrow
\startxrow
\startxcell[width={0.26\textwidth}] \useURL[url79][http://sapodilla.blogspot.com][][Guyana Gyal]\from[url79] \stopxcell
\startxcell[width={0.18\textwidth}] Neena Maiya \stopxcell
\startxcell[width={0.23\textwidth}] Guyana \stopxcell
\startxcell[width={0.33\textwidth}] Personal-is-political \stopxcell
\stopxrow
\startxrow
\startxcell[width={0.26\textwidth}] \useURL[url80][http://sarabharrat.wordpress.com][][The Guyanese Experience]\from[url80] \stopxcell
\startxcell[width={0.18\textwidth}] Sara Bharrat \stopxcell
\startxcell[width={0.23\textwidth}] Guyana \stopxcell
\startxcell[width={0.33\textwidth}] Culture critic, personal is political, +Feminism \stopxcell
\stopxrow
\startxrow
\startxcell[width={0.26\textwidth}] \useURL[url81][http://carolynjoycooper.wordpress.com][][Jamaica Woman Tongue]\from[url81] \stopxcell
\startxcell[width={0.18\textwidth}] Carolyn Cooper \stopxcell
\startxcell[width={0.23\textwidth}] Jamaica \stopxcell
\startxcell[width={0.33\textwidth}] Feminist-academic, media crossover, culture critic, witty \stopxcell
\stopxrow
\startxrow
\startxcell[width={0.26\textwidth}] \useURL[url82][http://mongoosechronicles.blogspot.com][][The Mongoose Chronicles]\from[url82] \stopxcell
\startxcell[width={0.18\textwidth}] Mar/Mar the Mongoose \stopxcell
\startxcell[width={0.23\textwidth}] Barbados \stopxcell
\startxcell[width={0.33\textwidth}] Culture critic, witty \stopxcell
\stopxrow
\startxrow
\startxcell[width={0.26\textwidth}] \useURL[url83][http://mooretalkja.wordpress.com][][MooreTalkJa]\from[url83] \stopxcell
\startxcell[width={0.18\textwidth}] Carla Moore \stopxcell
\startxcell[width={0.23\textwidth}] Jamaica \stopxcell
\startxcell[width={0.33\textwidth}] Media crossover, personal-is-political, witty, culture critic, feminist-academic \stopxcell
\stopxrow
\startxrow
\startxcell[width={0.26\textwidth}] \useURL[url84][http://negracubanateniaqueser.wordpress.com][][Negra Cubana tenia que ser]\from[url84] \stopxcell
\startxcell[width={0.18\textwidth}] Sandra Abd'Allah-Alvarez Ramírez/Negra Cubana \stopxcell
\startxcell[width={0.23\textwidth}] Cuba \stopxcell
\startxcell[width={0.33\textwidth}] Feminist-academic, culture critic \stopxcell
\stopxrow
\startxrow
\startxcell[width={0.26\textwidth}] \useURL[url85][http://onenationmanybodies.wordpress.com][][One Nation\ldots{}Many Bodies\ldots{}Boundless Faith]\from[url85] \stopxcell
\startxcell[width={0.18\textwidth}] Colin Robinson \stopxcell
\startxcell[width={0.23\textwidth}] Trinidad & Tobago \stopxcell
\startxcell[width={0.33\textwidth}] Media crossover, LGBT , +Feminism \stopxcell
\stopxrow
\startxrow
\startxcell[width={0.26\textwidth}] \useURL[url86][http://paulalindo.wordpress.com][][Paula Lindo]\from[url86] \stopxcell
\startxcell[width={0.18\textwidth}]  \stopxcell
\startxcell[width={0.23\textwidth}] Trinidad & Tobago \stopxcell
\startxcell[width={0.33\textwidth}] Personal-is-political \stopxcell
\stopxrow
\startxrow
\startxcell[width={0.26\textwidth}] \useURL[url87][http://petchary.wordpress.com][][Petchary]\from[url87] \stopxcell
\startxcell[width={0.18\textwidth}] Emma Lewis \stopxcell
\startxcell[width={0.23\textwidth}] Jamaica \stopxcell
\startxcell[width={0.33\textwidth}] +Feminism \stopxcell
\stopxrow
\startxrow
\startxcell[width={0.26\textwidth}] \useURL[url88][http://under-deconstruction.tumblr.com][][Red Ants]\from[url88] \stopxcell
\startxcell[width={0.18\textwidth}]  \stopxcell
\startxcell[width={0.23\textwidth}] Trinidad & Tobago \stopxcell
\startxcell[width={0.33\textwidth}] Curator \stopxcell
\stopxrow
\startxrow
\startxcell[width={0.26\textwidth}] \useURL[url89][http://repeatingislands.com][][Repeating Islands]\from[url89] \stopxcell
\startxcell[width={0.18\textwidth}] Ivette Romero-Cesareo and Lisa Paravisini-Gebert \stopxcell
\startxcell[width={0.23\textwidth}] Pan-Caribbean \stopxcell
\startxcell[width={0.33\textwidth}] +Feminism, feminist-academic, curator \stopxcell
\stopxrow
\startxrow
\startxcell[width={0.26\textwidth}] \useURL[url90][http://rewindandcomeagain.com][][Rewind and Come Again]\from[url90] \stopxcell
\startxcell[width={0.18\textwidth}]  \stopxcell
\startxcell[width={0.23\textwidth}] US/Guyana \stopxcell
\startxcell[width={0.33\textwidth}] +Feminism \stopxcell
\stopxrow
\startxrow
\startxcell[width={0.26\textwidth}] \useURL[url91][http://rootsandrights.wordpress.com][][Roots and Rights]\from[url91] \stopxcell
\startxcell[width={0.18\textwidth}] Roberta Clarke \stopxcell
\startxcell[width={0.23\textwidth}] Caribbean \stopxcell
\startxcell[width={0.33\textwidth}] Personal-is-political, culture critic \stopxcell
\stopxrow
\startxrow
\startxcell[width={0.26\textwidth}] Sheroxlox \stopxcell
\startxcell[width={0.18\textwidth}] Amina Doherty \stopxcell
\startxcell[width={0.23\textwidth}] Antigua & Barbuda/Nigeria \stopxcell
\startxcell[width={0.33\textwidth}] Personal-is-political, curator \stopxcell
\stopxrow
\startxrow
\startxcell[width={0.26\textwidth}] \useURL[url92][http://hairoun.blogspot.com][][Still I Rise]\from[url92] \stopxcell
\startxcell[width={0.18\textwidth}]  \stopxcell
\startxcell[width={0.23\textwidth}] St.~Vincent & the Grenadines \stopxcell
\startxcell[width={0.33\textwidth}] Personal-is-political, +Feminism \stopxcell
\stopxrow
\startxrow
\startxcell[width={0.26\textwidth}] \useURL[url93][http://churchroadman.blogspot.com][][Thoughts of a Minibus Traveller]\from[url93] \stopxcell
\startxcell[width={0.18\textwidth}] Vidyaratha Kissoon \stopxcell
\startxcell[width={0.23\textwidth}] Guyana \stopxcell
\startxcell[width={0.33\textwidth}] Personal-is-political \stopxcell
\stopxrow
\startxrow
\startxcell[width={0.26\textwidth}] \useURL[url94][http://tillahwillah.wordpress.com][][TillahWillah]\from[url94] \stopxcell
\startxcell[width={0.18\textwidth}] Atillah Springer \stopxcell
\startxcell[width={0.23\textwidth}] Trinidad & Tobago \stopxcell
\startxcell[width={0.33\textwidth}] +Feminism, culture critic, media crossover \stopxcell
\stopxrow
\startxrow
\startxcell[width={0.26\textwidth}] \useURL[url95][http://underthesaltireflag.com][][Under the Satire Flag]\from[url95] \stopxcell
\startxcell[width={0.18\textwidth}] Kei Miller \stopxcell
\startxcell[width={0.23\textwidth}] Jamaica \stopxcell
\startxcell[width={0.33\textwidth}] Culture critic, +feminism, witty \stopxcell
\stopxrow
\startxrow
\startxcell[width={0.26\textwidth}] \useURL[url96][http://womanishwords.blogspot.com][][Womanish Words]\from[url96] \stopxcell
\startxcell[width={0.18\textwidth}]  \stopxcell
\startxcell[width={0.23\textwidth}] The Bahamas \stopxcell
\startxcell[width={0.33\textwidth}] Poetry blog, personal-is-political \stopxcell
\stopxrow
\startxrow
\startxcell[width={0.26\textwidth}] \useURL[url97][http://womenspeak.tumblr.com][][Women Speak]\from[url97] \stopxcell
\startxcell[width={0.18\textwidth}] WomenSpeak Project \stopxcell
\startxcell[width={0.23\textwidth}] Trinidad & Tobago \stopxcell
\startxcell[width={0.33\textwidth}] Curator \stopxcell
\stopxrow
\stopxtablebody
\startxtablefoot[foot]
\startxrow
\startxcell[width={0.26\textwidth}] \useURL[url98][http://debraprovidence.wordpress.com][][Writing \quotation{D}]\from[url98] \stopxcell
\startxcell[width={0.18\textwidth}] Debra Providence \stopxcell
\startxcell[width={0.23\textwidth}] St.~Vincent and the Grenadines \stopxcell
\startxcell[width={0.33\textwidth}] +Feminism, poetry blog \stopxcell
\stopxrow
\stopxtablefoot
\stopxtable
\stopplacetable

\subsection[title={Feminist Positionality},reference={feminist-positionality}]

The ways Caribbean digital griots identify online reveal a cyberfeminist practice that intervenes in dominant configurations of both feminism and digital space as Northern/Western, white, and bourgeois and that render Caribbean, feminist, and female voices as marginal. In their \quotation{about me} pages, many of the bloggers expressly define themselves as feminist:

\startblockquote
This is me. Caribbean. Feminist, activist, poet, academic, educator, T-shirt graffiti artist.\footnote{Gabrielle Hosein, {\em Diary of a Mothering Worker}, \useURL[url99][http://grrlscene.wordpress.com/souldeya][][http://grrlscene.wordpress.com/souldeya.]\from[url99]}

I am a black feminist* and womanist and I prioritize black women & girls always. I often find myself revisiting areas pertaining to race, blackness, anti-blackness, culture and identity to name just a few.\footnote{Soyini Ayanna, {\em Creative Commess}, \useURL[url100][http://soyluv.wordpress.com/about][][http://soyluv.wordpress.com/about.]\from[url100]}

Three fabulous Caribbean feminist cousin-sisters.\footnote{derevolushunwidin, trendsettah, and pieces2peace, {\em Add Fyah and Stir}, \useURL[url101][http://addfyahandstir.wordpress.com/about][][http://addfyahandstir.wordpress.com/about.]\from[url101]}

A feminist pop culture blog focused on film, television, music and critical commentary on media representation.\footnote{Cate Young (BattyMamzelle), {\em Battymamzelle}, \useURL[url102][http://battymamzelle.blogspot.com/]\from[url102]}
\stopblockquote

These feminist bloggers also articulate the critical and political importance of women's voices and spaces, and by affirming \quotation{me nun tak back nuh chat,} they defiantly lay claim to their opinions, analyses, and theorizing without apology:\footnote{Carla Moore, {\em MooreTalkJa}, \useURL[url103][http://mooretalkja.wordpress.com][][http://mooretalkja.wordpress.com.]\from[url103]}

\startblockquote
Lover and defender of my womanness, Africanness, my Caribbean heritage, my Barbados, my right to take up my space and protect our space.\footnote{Mar the Mongoose, \quotation{About Me,} {\em The Mongoose Chronicles}, \useURL[url104][http://www.blogger.com/profile/06241127953404268513][][http://www.blogger.com/profile/06241127953404268513.]\from[url104]}

\quotation{Woman tongue, \quote{was-was} and tamarind tree, the three worse things.} ---Jamaican proverb

Translation: The woman's tongue, the wasp and the tamarind tree sting the most.

This proverb suggests the potency of the female voice as an expression of incisive social critique.\footnote{Carolyn Joy Cooper, \quotation{About,} {\em Jamaica Woman Tongue}, \useURL[url105][http://carolynjoycooper.wordpress.com/about][][http://carolynjoycooper.wordpress.com/about.]\from[url105]}
\stopblockquote

Men also claim and participate in these cyberfeminist spaces, asserting, \quotation{Yes, I am a Rasta, Socialist, Feminist Caribbean Man. Above all, I am an Artist.}\footnote{Amílcar Sanatan, \quotation{About I,} \useURL[url106][http://www.amilcarsanatan.com/about-us][][http://www.amilcarsanatan.com/about-us.]\from[url106]}

The explicit use of the term {\em feminist} signals its acceptance and assumption of shared meaning. Feminism is a key part of the identity of some of the bloggers whose work is examined here, such as Bad Dominicana, who describes herself as a \quotation{raging mujercista,} with cyberfeminism also a key part of how some bloggers self-identify.\footnote{Zahira Kelly (Bad Dominicana), \quotation{About Me,} {\em The Bad Dominicana}, \useURL[url107][http://bad-dominicana.tumblr.com/about][][http://bad-dominicana.tumblr.com/about.]\from[url107]} For example, \useURL[url108][http://negracubanateniaqueser.wordpress.com][][Negra Cubana]\from[url108] describes herself as \quotation{ciberfeminista negra.}\footnote{Sandra Abd'Allah-Alvarez Ramírez (Negra Cubana), \quotation{¿Quién es Negracubana?,} {\em Negra Cubana tenia que ser}, \useURL[url109][http://negracubanateniaqueser.com/quien-es-negracubana][][http://negracubanateniaqueser.com/quien-es-negracubana.]\from[url109]} These bloggers tend to qualify the term {\em feminist} with {\em Caribbean} or {\em black}, suggesting an awareness of the racially and geopolitically exclusionary way normative feminism is often framed and of the antagonism between feminisms and nationalism in postcolonial settings, or as a means of articulating the specificity of their feminist thought and practice. I argue that the insistence of geographical, national, racial/ethnic, and political qualifiers is a key Caribbean practice of cyberfeminists that reveals an understanding of themselves as speaking across a divide that contains differently identified feminists and a sensitivity to communication as key to emancipatory practices. Early 1990s representations of the technological future visualized racialized embodiment as oppositional to its liberatory promise, thus folding excitement about the promise of new technologies into old racialized, gendered, and geopolitical exclusions.\footnote{See Alondra Nelson, \quotation{Introduction: Future Texts,} {\em Social Text} 20, no. 2 (2002): 1--15.} The Net is therefore not a virtual place; it is very much real for identities and economies. These bloggers treat Internet media as extensions of the social spaces in which their bodies are already engaged.

Commercial, consumptive, militaristic, and entertainment usages of the Internet predominate, reinforcing the notion of an individual, apolitical, consuming self as the modal way of being online.\footnote{Paul C. Adams, \quotation{Cyberspace and Virtual Places,} {\em Geographical Review} 87, no. 2 (1997): 260.} As Marcia Forbes has noted, Jamaican youth view a lack of access to the Internet and social media as a stigmatizing disadvantage.\footnote{Forbes, {\em \#Social Media, Mobile Lifestyles}.} Furthermore, even when Internet users are engaged in expressly \quotation{political} projects, as are Caribbean feminists online, their dependence on commercially driven platforms implicates them in these hegemonic, militaristic, and commercial orientations of the Internet, as well as in asymmetrical flows of information, media reach, and capital. On the other hand, the Internet provides a space for a plurality of voices, including the counterhegemonic. Via their explicit articulations of their positionality, Caribbean feminists intervene in these domination configurations of digital space.

The format of the online forum or the blog post with a comments section allows for a multiplicity of voices. It therefore opens up spaces for dialogue and negotiation. However, some Caribbean cyberfeminists warn that you enter the media spaces they have created \quotation{at your own motherfuckin risk,} insisting, \quotation{i want you to feel alienated by me. i want you to have to tiptoe around me like im a delicate creature who will snap your head off if you so even think of messing w my humanity.}\footnote{Kelly (Bad Dominicana), \quotation{About Me.}}

By this they reveal heightened awareness to the ways being online folds into hegemonic notions of what it means to be human and into the gendered and racialized hierarchies of cyber/space. Their discursive, rhetorical, and media-making practices reveal their \quotation{critical feminist position for interrogating and intervening in specific technological forms and practices.}\footnote{Paasonen, \quotation{Rethinking Cyberfeminisn,} 340.} In this sense, it is important to view national or ethnic markers not merely as a part of \quotation{representing the nation online} but as a cyberfeminist practice that directly addresses geopolitical and racial hierarchies inside and outside feminisms.\footnote{See Miller and Slater, {\em The Internet}.}

\subsection[title={Caribbean Feminist Concerns},reference={caribbean-feminist-concerns}]

Cyberfeminists bloggers cover a range of concerns in dialogue with local, regional, national, and international current affairs. They examine diverse issues, including drought, climate change, and environmental degradation, as well as the manifestation of unequal relations of gender in a range of areas. They explore embodiment, transgender identities, skin color, and racisms and offer up sophisticated analyses of heterosexuality and sexual politics that are often muted in academic scholarship. Caribbean cyberfeminists train their critical lenses to Caribbean and US popular culture, using these as modes of engaging with feminisms. Caribbean cyberfeminists challenge heteronormativity through their documenting of local, national, and regional LGBT organizing and articulating a vision of citizenship and nationhood that is inclusive. \quotation{I've always understood the project and vision of nation to include me---that it was meant to include everyone,} writes Colin Robinson. \quotation{I firmly believed there was progress in wrapping ourselves in its symbols, which would make a difference where threatened boycotts, shaming human rights reviews and bans of our culture didn't quite appear to be working.}\footnote{Colin Robinson, \quotation{Mrs.~Joyce Pierre's Daughter,} {\em One Nation . . . Many Bodies . . . Boundless Faith}, 28 June 2015, \useURL[url110][http://onenationmanybodies.wordpress.com/2015/06/28/mrs-joyce-pierres-daugther][][http://onenationmanybodies.wordpress.com/2015/06/28/mrs-joyce-pierres-daugther.]\from[url110]} Across multiple blogs, Caribbean cyberfeminists express concern about governance, the masculinism of state power, corruption, and holding state managers accountable. \useURL[url111][http://generacionyen.wordpress.com][][Yaoni Sanchez]\from[url111]'s words are illustrative of the views expressed by many: \quotation{With so many problems facing the country, which affect millions of people, how could a day of \quote{the official organ of State power} be squandered to sing the praises of a single man? Situations like yesterday are proof that the pernicious cult of personality remains intact among us, fostered by those who idolize a few and those who swell with vanity at the flattery.}\footnote{Yaoni Sánchez, \quotation{Cult of Personality in Cuban Parliament,} {\em Generación Y}, 15 July 2015, \useURL[url112][http://generacionyen.wordpress.com/2015/07/15/cult-of-personality-in-cuban-parliament][][http://generacionyen.wordpress.com/2015/07/15/cult-of-personality-in-cuban-parliament.]\from[url112]}

Bloggers also use social and electronic media to speak out against sexual harassment and assault at the hands of men whose official power would otherwise render these women silent. The excerpt below is illustrative of the ways some Caribbean women have used personal blogs to break the silence on sexual violence:

\startblockquote
\quotation{I went to him expecting some professionalism, you know, and all I get was him trying to bus' a hustle in me. Telling me about how he could do things to my body and how he want fuck me. I mean what shit is that? Can you imagine how I felt? How was I suppose to receive a service from a man like duh eh? And imagine this nonsense now, he get big government wuk.}

I related my own encounter with the same gentleman to her. It had happened more than 7 years ago at the Georgetown Magistrates' Court. Since then, I've heard many similar stories from women about him. The standard response to this particular official is \quotation{O, he? Everybody know how he stay.}\footnote{Sara Bharrat, \quotation{The Senior Government Official Guyanese Women Dislike,} {\em Guyanese Experience}, July 2015, \useURL[url113][http://sarabharrat.wordpress.com/2015/07][][http://sarabharrat.wordpress.com/2015/07.]\from[url113]}
\stopblockquote

Here, Guyanese blogger \useURL[url114][http://sarabharrat.wordpress.com][][Sara Bharrat]\from[url114] relates a conversation she had with a woman taxi driver about a state manager infamous for his sexual exploitation of women. Their narrative exchange, which took place in the taxi, is an act of feminist solidarity that is then extended to digital space.

Bloggers also analyze and expose racial/ethnic tensions in their communities and countries, exposing antiblack and anti-Indian racism and meditating on racialized identities in the region:

\startblockquote
When some Trinidadians make fun of the blackness of Tobagonians, it's not unlike the maligning of Haitians, overtly or subconsciously, as \quotation{{\em that} kind of Black}---too black.\footnote{Soyini Ayanna, \quotation{The Language of Blackness,} {\em Creative Commess}, 11 July 2015, \useURL[url115][http://soyluv.wordpress.com/2015/07/11/the-language-of-blackness][][http://soyluv.wordpress.com/2015/07/11/the-language-of-blackness.]\from[url115]}

{\em \quotation{I do not know if I am East Indian, Trinidadian or West Indian.} Sam Selvon, Opening Address to East Indians in the Caribbean Conference, University of the West Indies, Trinidad, 1979}

While in the beginning I have not been overly concerned with being West Indian, there have been many days when I was not sure whether I was an East Indian or a Guyanese. In fact, I was afraid to be either of these things because I did not know how to make them live in harmony inside of me.

There was a time when I could not fully nor comfortably embrace my East Indian heritage because I felt guilty; I felt as if I were somehow betraying my Guyaneseness. But then, how could I be Guyanese without my Indianness? It took me a while to realize why it was so hard to be Guyanese; I simply did not understand what it meant to be one of us.\footnote{Sara Bharatt, \quotation{Three into One Definitely Can't Go, or Can It?,} {\em Guyanese Experience}, 15 May 2015, \useURL[url116][http://sarabharrat.wordpress.com/2015/05/15/three-into-one-definitely-cant-go-or-can-it][][http://sarabharrat.wordpress.com/2015/05/15/three-into-one-definitely-cant-go-or-can-it.]\from[url116]}
\stopblockquote

Taken together, these diverse areas of concern reveal Caribbean cyberfeminist participation in local, regional, national, and global communities and demonstrate the subversive possibilities of digital public spheres.

\subsection[title={Strategic Space, Strategic Action},reference={strategic-space-strategic-action}]

While I have focused largely on individual bloggers, it is important to recognize that blog posts are often cocreated, open-ended, community projects. They may be reblogged on multiple blogs and shared via Facebook, e-mail, and other media. Comments on the blog post itself, on Facebook, and in lunch rooms co-construct knowledge, open up dialogue, act as a form of consciousness raising, and may also challenge received knowledge on inequalities, gender, and sexuality and serve to build community. The process of commenting can shift the dialogue, help flesh out the analysis initiated by the author, or make space for multiple and dissenting voices.

Caribbean feminists have also published their work on the blogs of others and on more outernational platforms. Jamaican-born, New York-based lesbian poet, author, and performer Staceyann Chin is published regularly in multiple US-based online media platforms and is a very successful producer and user of social media, with 17,000 follows on Twitter and 27,000+ on Facebook. Guyanese activist Sherlina Nageer of Red Thread, Society Against Sexual Orientation Discrimination, and Occupy Guyana writes frequently on her own Facebook page and has shared those writings with a wider audience by publishing on the {\em Huffington Post}, on {\em Black Looks} (a website owned by Nigerian journalist Sokari Ekine, who is currently living in Haiti), and on CODE RED's blog {\em Feminist Conversations on Caribbean Life}. Nageer used these multiple platforms to denounce police killings in Linden{\em ,} to share with a wider audience her submission to the Committee on the Elimination of Discrimination Against Women on the rights of lesbian, bisexual, and transgender women in Guyana, and to report on the efforts of Guyanese activists from women's, LGBT, and disability rights organizations to mobilize against transphobic, homophobic, sexist, and ableist violence in Guyana. Sherlina later started her own blog, \useURL[url117][http://freedombyanymeans.wordpress.com][][Freedom by Any Means]\from[url117], on which she republishes articles from her {\em Stabroek News} column.\footnote{Sherlina Nageer, \quotation{No Women Died, This Is Not a Women's Issue,} {\em Black Looks}, 15 August 2012, \useURL[url118][http://www.blacklooks.org/2012/08/no-women-died-this-is-not-a-womens-issue][][http://www.blacklooks.org/2012/08/no-women-died-this-is-not-a-womens-issue;]\from[url118] \quotation{End Sexual-Orientation and Gender-Identity Discrimination in Guyana,} {\em Huffpost Gay Voices}, 10 July 2012 (updated 2 February 2016), \useURL[url119][http://www.huffingtonpost.com/sherlina-nageer/]\from[url119]; \quotation{In Praise of Bad-john Lil Girls,} {\em Feminist Conversations on Caribbean Life}, 21 August 2013, \useURL[url120][http://redforgender.wordpress.com/2013/08/21/in-praise-of-bad-john-lil-girls][][http://redforgender.wordpress.com/2013/08/21/in-praise-of-bad-john-lil-girls.]\from[url120] See \useURL[url121][http://freedombyanymeans.wordpress.com]\from[url121].}

BattyMamzelle, a US-educated twenty-something from Trinidad and Tobago, was a regular contributor to US-based commercial pop feminism site, {\em Jezebel}, with some of her most popular pieces gaining more than one million unique views. In 2013, she wrote a \useURL[url122][http://battymamzelle.blogspot.com/2013/10/Why-Im-Quitting-Jezebel.html\#.UlqP_FCsim4][][post]\from[url122] outlining her decision to no longer contribute to the site:

\startblockquote
I'm 23. I'm black. I'm West Indian. I didn't have a lot of exposure to feminism growing up. I cringe when I recall the things that I said about other women when I was a freshman in college, and the attitudes that I had towards sexuality and womanhood. I went to a Catholic school in a country that is still largely misogynistic. It was pretty much a given that I'd grown up to be an anti-woman little shit.

But then I found Jezebel. I found Jezebel and I started reading. I'm the kind of person who just likes to know things, so perusing the site pre-Kinja was like a revelation to me. All of a sudden I had this entire vocabulary to explain the little microagressions that I'd faced all my life, and a community of women who were engaged in parsing those issues. I could finally vocalize why I felt an inconsolable rage when I was tone policed. I knew how to defend myself against slut shaming. I could explain in detail why rape culture was so insidious and why restrictions on reproductive freedom were a devastating step backwards for women.

Jezebel taught me how to be a woman.

And then it taught me that it didn't care about the kind of woman that I am.\footnote{Cate Young (BattyMamzelle), \quotation{The Hardest Word to Say Is Goodbye: Why I'm Quitting {\em Jezebel},} {\em BattyMamzelle}, 8 October 2013, \useURL[url123][http://battymamzelle.blogspot.com/2013/10/Why-Im-Quitting-Jezebel.html\#.UlqP_FCsim4][][http://battymamzelle.blogspot.com/2013/10/Why-Im-Quitting-Jezebel.html\#.UlqP_FCsim4.]\from[url123]}
\stopblockquote

BattyMamzelle's experience of the racism of online feminist communities highlights the continuities and articulations with historically sedimented relations of power that remerge in the very spaces that claim to be feminist, polyvocal, rhizomatic, and queer. Her experiences validate my earlier analyses about the strategic use of ethnic, racial, national, and regional markers in bloggers' \quotation{about me} pages as both recognition of and counter to prevailing exclusions in cyberspace. Cyberfeminist spaces are becoming increasingly important to the consciousness raising of a new generation of women and men, even when these spaces themselves may be exclusionary.

Aside from blogging, Caribbean cyberfeminist communities are created through the use of groups, pages, and campaigns, many of which are hosted on Facebook. As Facebook moves to monetize every aspect of the Facebook experience, Caribbean feminists who made strategic use of the site for consciousness raising and community building are faced with challenges resulting from the predominance of the commercial nature of cyberspace. Prior to Facebook's changes to its policy for pages, once someone \quotation{liked} your page, any content you posted there showed up in their personal feed. One of the earliest attempts at building online Caribbean feminist community, \useURL[url124][http://www.facebook.com/redforgender/timeline?ref=page_internal][][CODE RED for Gender Justice]\from[url124]'s Facebook page, which was inaugurated in 2010, capitalized on this capacity to build readers, contributors, and commenters. Now Facebook requires that you pay to promote each individual post to your page, or else it only reaches a tiny percentage of your network. This no doubt severely reduces the reach of Facebook-based campaigns, such as \useURL[url125][http://www.facebook.com/WalkingIntoWalls/info/?tab=page_info][][Walking into Walls]\from[url125], whose page aggregates daily news stories of violence against women, intimate partner violence, rape and sexual violence, and child abuse from a wide range of Caribbean countries: \quotation{Walking into Walls is a social media campaign that found its roots in a 2012 regional meeting organized by the Caribbean Institute for Women in Leadership (CIWiL). This campaign was developed by four passionate, committed Caribbean women who are tired of the walls that are routinely hit in the struggle to end violence against women.}\footnote{\useURL[url126][http://www.facebook.com/WalkingIntoWalls/info/?tab=page_info][][http://www.facebook.com/WalkingIntoWalls/info/?tab=page_info.]\from[url126]} Its Facebook page is a digital media resource of tremendous potential and importance that is intended to contribute to awareness raising and to keep reports of violence against women in the forefront of Caribbean people's minds. Such Caribbean feminist media campaigns are important as independent media sites that seek to compensate for the elisions in mainstream media, though, of course, without the benefit of the same kind of reach. CODE RED for Gender Justice on Facebook similarly covers a wide range of topics related to gender inequality, violence, sexuality, disability, and the environment.\footnote{See \useURL[url127][http://www.facebook.com/redforgender/timeline?ref=page_internal][][http://www.facebook.com/redforgender/timeline?ref=page_internal.]\from[url127]} The dependence of Caribbean feminists on privately owned corporate platforms that claim ownership of their content while modifying their media delivery online with revenue generation strategies undermines the subversive potential of Web 2.0. In \quotation{Rethinking Cyberfeminism,} Susanna Paasonen notes,

\startblockquote
Possibilities of lay users to interact with and shape the medium are more limited than they were fifteen or even ten years ago. While it was entirely possible to set up a catchy web site in the late 1990s with a basic mastery of HTML---and, perhaps, a few touches of the cutting-edge such as JavaScript---this is no longer the case. Code has grown increasingly complex and necessitates rather specialized skills. Rather than building their sites (such as personal home pages and online journals) from a scratch, users make use of customizable templates and social media applications (from Blogger to YouTube, Facebook, PhotoBucket and Flickr) when publishing their images, videos, texts and music. Riot Grrrl 'zines, with their DIY feel, have largely disappeared.\footnote{Paasonen, \quotation{Rethinking Cyberfeminism,} 347.}
\stopblockquote

Feminist and gender-conscious journalists located within mainstream media are nonetheless part of Caribbean cyberfeminism. Caribbean feminist scholar Alissa Trotz's \useURL[url128][http://www.stabroeknews.com/category/features/in-the-diaspora][][{\em In the Diaspora}]\from[url128] column in {\em Stabroek News} brings a mainstream reach to feminist and social justice advocates and writers. Online magazine \useURL[url129][http://www.outlish.com][][{\em Outlish}]\from[url129], though not a feminist magazine per se, has created a space for discussion of issues of gender, feminism, and masculinity, with articles such as \quotation{Why Is There So Much Pressure on Indian Men to Get Married?,} \quotation{This Is Why I'm Feminist,} and \quotation{Don't Laugh but T&T Needs a Men's Movement.} Feminists from the Caribbean have also created their own online lifestyle magazines: for example, \useURL[url130][http://blackgirlinthering.com][][{\em Black Girl in the Ring}]\from[url130], from Antigua and Barbuda/Guyana, and \useURL[url131][http://www.complexdwoman.co.uk][][{\em CompleXD Woman}]\from[url131], from Grenada/UK.\footnote{See \useURL[url132][http://www.stabroeknews.com/category/features/in-the-diaspora][][http://www.stabroeknews.com/category/features/in-the-diaspora;]\from[url132]; \useURL[url133][http://www.outlish.com][][http://www.outlish.com;]\from[url133] \useURL[url134][http://blackgirlinthering.com][][http://blackgirlinthering.com;]\from[url134] and \useURL[url135][http://www.complexdwoman.co.uk][][http://www.complexdwoman.co.uk.]\from[url135]} Online activist practices include petitions, mobilizing to shut down offensive websites, and promotion of offline events such as popular actions, protests, and stands. Many longstanding women's and feminist organizations in the region also maintain an online presence.

Scholars, however, have drawn attention to the strategic space that global South peoples make of the Net, arguing that rather than see marginalized peoples as on the losing end of the digital divide, focus should be on the creative and strategic uses they make of these technologies.\footnote{See Curwen Best, {\em The Politics of Caribbean Cyber Culture} (New York: Palgrave Macmillan, 2008).} Stephanie Leitch of WOMANTRA shares that she regularly uses the WOMANTRA community to mobilize women and men in order to shutdown offensive websites and Facebook pages. She recalls a popular online magazine that published an article titled \quotation{No Matter How Hot She Is} that warned single men not to have sex with women more than twice lest they become \quotation{disillusioned and depressed.} Leitch left a comment below the article, \quotation{basically telling the guy he was a sexist pig} and encouraged others to do the same. The sheer volume of complaints the magazine received caused the editorial staff to pull the article and shutdown the website for a day. One of the managers then contacted Leitch and ironically told her that she should be writing something rather than \quotation{just talking.} Leitch later wrote an article about sexism and media responsibility that was published on the site. However, she recalls of the blogger who wrote the original post: \quotation{{[}He was{]} was very abusive, said i must be ugly or a lesbian and called another guy {[}who also complained about the article{]} a faggit.}\footnote{Stephanie Leitch, personal communication with the author.} The examples cited here reflect diverse cyberfeminist strategies aimed at subverting asymmetrical relations of gender. They intervene into the normalizing of misogynist discourses and images by challenging mainstream media normalization of unequal relations of gender and sexualized images of women and girls.

\subsection[title={Future Directions},reference={future-directions}]

Online feminist spaces are important consciousness-raising and pedagogical sites for Caribbean feminists. These sites, however, are ones of unequal access and privilege. The necessity and challenges of working across and through difference, race/ethnicity, color, class, and privilege that were identified two decades ago by Caribbean feminists organized in Sistren (Jamaica), Red Thread (Guyana), and the Caribbean Association for Feminist Research and Action persist as key issues for Caribbean feminist organizing.\footnote{See Honor Ford-Smith, \quotation{Ring Ding in a Tight Corner: Sistren, Collective Democracy, and the Organization of Cultural Production,} in M. Jacqui Alexander and Chandra Talpade Mohanty, eds., {\em Feminist Genealogies, Colonial Legacies, Democratic Futures} (New York: Routledge, 1997); Andaiye, {\em The Angle You Look from Determines What You See: Towards A Critique of Feminist Politics in the Caribbean}, Lucille Mathurin Mair Lecture series (Mona: Centre for Gender and Development Studies, University of the West Indies, 2002); Rawwida Baksh-Soodeen, \quotation{Issues of Difference in Contemporary Caribbean Feminism,} {\em Feminist Review} 59 (June 1998): 74--85.} Caribbean historians have also called attention to the ways Indo- and Afro-creole nationalisms have marginalized indigenous people in the region.\footnote{See Shona N. Jackson, {\em Creole Indigeneity: Between Myth and Nation in the Caribbean} (Minneapolis: University of Minnesota Press, 2012); and Melanie J. Newton, \quotation{Returns to a Native Land: Indigeneity and Decolonization in the Anglophone Caribbean,} {\em Small Axe}, no. 41 (July 2013): 108--22.} Investigations into the extent to which online feminisms reproduce and reframe these long-standing inequities and oppressions remain a key Caribbean cyberfeminist project.

Online Caribbean feminisms are multigenerational, multiethnic, transnational, and Pan-Caribbean. Caribbean feminists have engaged in cyberactivist practices and used the Internet to build community, organize, and mobilize. Nonetheless, issues of media reach as well as dependence on commercial platforms whose very structures may be inimical to feminist principles limit the subversive potential of these strategies. Strategies such as the use of video, humor, personal stories, and writing in nation languages serve to make online feminism more accessible. However, as participation in digital cultures brings with it notions of progress, modernity, and a supposedly apolitical globalization, it is important for Caribbean cyberfeminism as a critical approach to attend to the ways digital cultures may be complicit in reinscribing inequalities.

\thinrule

\subsection[title={Appendix},reference={appendix}]

{\em {\bf Blog Category:} Description (number of times used)}

{\bf Culture critic:} Blogs focused on cultural and sociological analyses of relations of power such as gender, race, class, and heterosexism and inequalities broadly (18)

{\bf Personal-is-political:} Blogs that draw on personal experience to illustrate or analyze relations of power in wider society (18)

{\bf +Feminism:} Blogs that do not have an exclusive focus on feminism or gender and sexuality but that do include either writing from a feminist perspective or posts about women's, feminist, gender, and LGBT issues (14)

{\bf Feminist-academic:} Blogs by feminist academics (7)

{\bf Media crossover:} Refers to blogs that contain content published both in mainstream and online media; includes both blogs by journalists and columnists and those by bloggers whose online content is republished in national newspapers or who have attracted significant mainstream media attention (7)

{\bf Curator:} Blogs that curate and reproduce content published elsewhere (6)

{\bf Witty:} Blogs by writers for whom wit, humor, and wordplay are key rhetorical devices (5)

\thinrule

\page
\subsection{Tonya Haynes}

Tonya Haynes lectures at the Institute for Gender and Development Studies: Nita Barrow Unit, blogs on redforgender.wordpress.com, and is a founding member of CODE RED for gender justice! and CatchAFyah Caribbean Feminist Network. She holds a PhD from the University of the West Indies and researches in the area of Caribbean feminisms and Caribbean Feminist Thought. Her work has been published in {\em Global Public Health} and {\em Small Axe: A Journal of Caribbean Criticism}.

\stopchapter
\stoptext