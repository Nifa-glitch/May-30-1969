\setvariables[article][shortauthor={Munro}, date={May 2016}, issue={1}, DOI={10.7916/D8NC619F}]

\setupinteraction[title={\quote{Who Stole the Soul?} Rhythm and Race in the Digital Age},author={Martin Munro}, date={May 2016}, subtitle={Who Stole the Soul?}, state=start, color=black, style=\tf]
\environment env_journal


\starttext


\startchapter[title={\quote{Who Stole the Soul?} Rhythm and Race in the Digital Age}
, marking={Who Stole the Soul?}
, bookmark={\quote{Who Stole the Soul?} Rhythm and Race in the Digital Age}]


\startlines
{\bf
Martin Munro
}
\stoplines


{\startnarrower\it This essay asks the following basic questions: What happens to the relationship between rhythm and race in the digital age? What happens when mastery of rhythm is no longer necessarily tied to ritual, to manual drumming, and to the physical, bodily re-creation of rhythm? When electronic and digital media give virtually anyone the ability to \quote{drum} and to create rhythmic music, what happens to the longstanding association between blackness and rhythm? Referring to David Scott's recent arguments on a stalled, tragic time in the Caribbean in particular, the author draws connections between the apparent redundancy of revolutionary, anticolonial thinking in the present and the perhaps less apparent decoupling of rhythm and race in contemporary musical styles. If, as Scott says, the teleologies of anticolonial politics no longer hold true, has rhythm as a marker of time, and as an integral element in the poetics of resistance, lost its association with radical blackness and become a deracialized, dehistoricized commodity?

 \stopnarrower}

\blank[2*line]
\blackrule[width=\textwidth,height=.01pt]
\blank[2*line]

This essay asks the following basic questions: What happens to the relationship between rhythm and race in the digital age? What happens when the production of rhythmic sounds is no longer necessarily tied to ritual, to manual drumming, and to the physical, bodily re-creation of rhythm? When electronic and digital media allow virtually anyone the ability to \quotation{drum} and to create rhythmic music, what happens to the longstanding association between blackness and rhythm? Referring ultimately to David Scott's recent arguments on a stalled, tragic time in the Caribbean in particular, I draw connections between the apparent redundancy of revolutionary, anticolonial thinking in the present and the perhaps less apparent decoupling of rhythm and race in contemporary musical styles. If, as Scott says, the teleologies of anticolonial politics no longer hold true, has rhythm, as a marker of time and an integral element in the poetics of resistance, lost its association with radical blackness and become a deracialized, dehistoricized commodity?

The title of this essay refers, of course, to the 1989 Public Enemy track that rails against the musical industry's exploitation of black artists, from Wilson Pickett to James Brown and Otis Redding. Stealing the soul, in this case, relates to both financial misdeeds and the appropriation of \quotation{soul} itself by the system, which plays the role of the \quotation{pimp.} Although there appears to be no direct connection between the two, I had always related this track to Nelson George's {\em The Death of Rhythm and Blues}, which was published one year previously and which similarly critiques the ways the music business has progressively eroded and appropriated the concept of soul.\footnote{Nelson George, {\em The Death of Rhythm and Blues} (New York: Penguin, 1988).} It is George's text that I will draw on first in order to lay out some of the rhythm-related issues that interest me here.

Specifically, I refer most directly to George's sixth chapter, titled \quotation{Crossover: The Death of Rhythm and Blues (1975--79).} {\em Crossover} is the key term here, since it denotes a process of cultural and social transference, a movement for some across the \quotation{barricades} around which, George says, rhythm and blues music had been formed.\footnote{Ibid., 147.} Socially, by the mid-1970s, a degree of integration and upward mobility had become possible through civil-service jobs, low- and mid-level management positions, and increased college attendance. These factors, George says, \quotation{affected the pocketbooks and tastes of black America,} and he laments this process, arguing that \quotation{much of what had made the R&B world work was lost, perhaps some of it forever}: \quotation{Sadly, R&B as a metaphor and a music, looked like a terminal case; the world that had supported it no longer existed. The connections between blacks of all classes were loosening, as various forms of material success (or the illusion of such success) seduced many into the crossover mentality.}\footnote{Ibid., 156, 147, 156.}

Musically, it was disco and what George terms its \quotation{rigid rhythms} that characterized crossover and announced the apparent death of R&B. These rhythms came not from the R&B tradition that George venerates but from Eurodisco, whose \quotation{metronomelike beat} he derides as being \quotation{perfect for folks with no sense of rhythm.}\footnote{Ibid., 154.} Crossover artists such as Billy Paul and the Stylistics made what George terms \quotation{glossy, upwardly mobile corporate black music,} which tended to underplay the rootsy, rhythmic elements of R&B and \quotation{to be sweet, highly melodic, and unthreatening.}\footnote{Ibid., 155.} This is crucial, in that George relates the advent of Eurodisco's machine-produced rhythms to the coming to prominence of melody in crossover music and the loss of the apparently unshakable bond between blackness and rhythm. In effect, the history of Eurodisco is intimately connected to that of the drum machine and to the automated beats and mechanical rhythms that George sees as the major causes of the death of R&B.\footnote{See Peter Shapiro, {\em Turn the Beat Around: The Secret History of Disco} (Basingstoke, UK: MacMillan, 2006), 98.} My point at this stage is not so much to endorse or refute George's views but to use them as an instance in which the rhythms of electronic music seem to both instantiate and echo a crisis in black identity, understood at least in part in terms of its long relationship to rhythm.

The musicologist Anne Danielsen raises a question similar to those posed by this essay when she asks, \quotation{What happened to the sound and rhythm of African-American derived, groove-directed popular music styles when these grooves began to be produced and played by machines?} She challenges the assumptions that \quotation{groove depends on human performativity to be aesthetically satisfying} and that \quotation{machine-generated music must be necessarily devoid of groove qualities, because it typically lacks the microtemporal variation added by people in performance.} She cites two trends in \quotation{computer-based rhythmic music} that challenge these assumptions: first, the increasing experimentation and manipulation of the microtiming of \quotation{rhythmic events} in digital music, a development she sees particularly in \quotation{African-American dominated genres such as rap, neo-soul, and contemporary r&b, where the use of digital equipment for music production was quickly accepted and cleverly applied.}\footnote{Ann Danielsen, \quotation{Introduction: Rhythm in the Age of Digital Reproduction,} in Anne Danielsen, ed., {\em Musical Rhythm in the Age of Digital Reproduction} (Farnham: Ashgate, 2010), 1.} The second trend constitutes a challenge to the idea that a successful \quotation{groove} requires microtemporal deviations in order to succeed. In apparently rigid digital electronica-related styles, she finds still that the music is danceable and has \quotation{unmistakable groove qualities.}\footnote{Ibid., 2.} Digital music, she says, poses a challenge to the rhythm researcher in explaining the rhythmic qualities of music \quotation{whose rhythmic events lie rigidly on a metric grid.} Importantly, she argues that this challenge applies not only to \quotation{the hyperquantized grooves of our digital age} but also to earlier forms that are characterized by a \quotation{strictly metronomic organization of rhythmic events, such as disco.}\footnote{Ibid., 3.} Her analyses are related more to the technical qualities of music than the discourse around rhythm and race that is this article's primary interest, but her conclusions do echo those that the article will ultimately arrive at, in that, as she argues, implicitly critiquing George's position, the common assumptions regarding the diminished \quotation{soul} of black music in the digital era do not hold firm. As she puts it, \quotation{The grooviness and expressivity of African-American-derived musical styles did not die with the new technology. Rather, they were reproduced and transformed.}\footnote{Ibid., 4.}

In the Caribbean, the relationship between blackness and rhythm dates to the early plantation period, when, with the arrival of African slaves, the colonies became places of encounter between what Jacqueline Rosemain calls the ancient \quotation{cosmogonic rhythms} of Europe and Africa---between the pagan rhythms that were appropriated by European Christianity and the animist rhythms of the Africans. On the plantation there was something of a \quotation{crusade against rhythm.}\footnote{Jacqueline Rosemain, {\em La Danse aux Antilles: Des rythmes sacrés au zouk} (Paris: L'Harmattan, 1990), 34, 37 (translation mine).} All rhythms of European or African origin were subject to antirhythmic legislation: because they were seen as savage and dangerous and because they accompanied the dances of cults not recognized by the official religion, they were prohibited. These prohibitions applied equally to the pagan rhythms of the European folk as to the rhythms of African rituals.

It was with the arrival en masse of millions of Africans (speaking many different languages and with their own diverse cultural traditions) that the notion of distinct black and white cultures began to solidify into the dualistic models that we are still living with today. This process of racial and cultural differentiation, shaped and supported by the racial thinking of European philosophers such as Hegel and de Gobineau, came to attribute rhythmic music and dance exclusively to Africans and to the state of savage, premodern humanity that the Africans were supposed to exemplify. The Africans had brought with them a particular affinity to rhythm, which was never biologically inherited but was an acquired aspect of music and everyday life, social interaction, speech, recreation, and work.\footnote{Gérard Barthélemy agrees that \quotation{there is nothing in rhythm that is due to a particular morphology, and even, less to race; genetics clearly have nothing to do with it. Everything is on the contrary the result of automatisms, themselves due to a long apprenticeship in the course of growing up.} Gérard Barthélemy, {\em Créoles-Bossales: Conflit en Haïti} (Petit-Bourg, Guadeloupe: Ibis Rouge, 2000), 173 (translation mine). For example, in his seminal study, John Miller Chernoff talks of the functional importance of rhythm at all levels and all ages in African existence: \quotation{African children play games and sing songs displaying a rhythmic character \ldots{} learn to speak languages in which proper rhythmic accentuation and phrasing is essential to meaning\ldots{} . Facility with rhythms is something people learn as they grow up in an African culture, one of the many cultural acquisitions that make someone seem familiar to people who have also learned the same things. Rhythms are built into the way people relate to each other.} John Miller Chernoff, {\em African Rhythm and African Sensibility} (Chicago: University of Chicago Press, 1979), 94. Miller's project examines the sociocultural significance of African rhythm and music and their deep \quotation{integration into the various patterns of social, economic, and political life} (35).} These rhythms clashed with the industrial rhythms of the plantation system, the unnatural, dehumanizing, profit-driven cycles of work that radically interrupted natural rhythms of life, birth, and death. African rhythms adapted themselves to this alienating environment, in the antiphonal work songs that accompanied field labor and in the slave dances that were sometimes sacred, sometimes secular gatherings that purged to some extent the suffering endured in working to the machine-like rhythms of the plantation. Across the Caribbean, too, vernacular culture---dances, music, religion, storytelling, language---acted as a repository of rhythm; and as the three-tiered social structure typical of Creole societies developed, the brown-skinned middle class, often as much as the white elite, distanced itself from rhythm and \quotation{black culture} in general.

Rhythm therefore became associated not only with a certain \quotation{race} of people but also with the lower class and with notions of cultural backwardness. As such, especially in the Caribbean, middle class intellectuals of color often aligned themselves with the putatively superior, arrhythmic literary and musical culture of Europe. In the nineteenth century, with a few exceptions, rhythm remained for many mulatto intellectuals a marker of Africa, and thus of social and cultural inferiority. The poetry and art music of this period rarely evoked rhythmic, vernacular culture in anything but derogatory, disdainful terms. At the same time, rhythm has been a primary means through which lower class Caribbean and New World black peoples have asserted and enacted their historical and political agency.

As the twentieth century began, and as various Pan-African movements gathered momentum, rhythm was slowly recuperated into the emerging middle-class, nationalist consciousness as a potent sign of racial difference and a primary element of the lost, devalorized parts of national culture. The terms in which rhythm was now evoked---as something irrevocably premodern, primitive, and antirational---were virtually the same as before, but now these qualities were seen positively as means of uniting separated diasporic peoples and of further solidifying the notion of black culture as radically, incontrovertibly different from white culture. Rhythm now flooded into all areas of middle-class, mulatto culture and became for the first time a prominent, distinguishing feature of poetry, art, music, and choreography. \quotation{Primitive sources} were now raided in the quest to recast national culture in primordial, anti-European (and often anticolonial) terms. The putatively authentic culture of the previously neglected masses was now seen as the single most important marker of national difference, a great, supposedly uncorrupted resource to be plundered and appropriated as the middle class slowly shed its mimetic impulses and embraced cultural nationalism. In the great political, cultural, and social maelstroms of the New World in the twentieth century, rhythm was thrown center stage and played a dynamic role in, for instance, Haitian indigenism, Negritude, and Black Power---movements that harnessed rhythm to their notions of black culture and black difference. In a 1961 interview, Aimé Césaire was asked about the importance of language in his work, and he replied that words were less important to him than rhythm; rhythm, he said, was \quotation{an essential element of the black man.}\footnote{Quoted in Jacqueline Sieger, \quotation{Entretien avec Aimé Césaire,} {\em Afrique} 5 (October 1961): 65.} Much later, Antonio Benítez-Rojo wrote of the rhythmic culture of the Caribbean, albeit in less essentialist terms, citing the importance of African rhythms to the region's cultures: \quotation{Black Africa has an even greater dependence on rhythm {[}than India{]}\ldots{} . In reality, it can be said that African culture's genes have been codified according to the possibilities of percussion. African culture is, above all, a thicket of systems of percussive signifiers, to whose rhythms and registers one lives both socially and inwardly.}\footnote{Antonio Benítez-Rojo, {\em The Repeating Island: The Caribbean and the Postmodern Perspective}, trans James E. Maraniss (Durham, NC: Duke University Press, 1992), 169.}

The Negritude movement allowed for---indeed, largely relied on---the elaboration of dualistic, racialized notions of culture, in which rhythm was recuperated as a marker of Pan-African blackness. Negritude did not, however, lead to political independence for the French Caribbean territories, and, as such, rhythm was not subsequently incorporated into a triumphalist idea of national culture and blackness. Instead, the notion that there existed a natural bond between black peoples and rhythm was quickly rejected and subjected to the piercing post-Negritude critiques of Frantz Fanon and René Ménil, which effectively cut short the essentialist discourse on blackness and rhythm and cleared the intellectual ground for a less racially oriented, more phenomenologically based engagement with rhythm. Edouard Glissant radically undermines the idea that the rhythm of the drum carries comforting memories of Africa and uses repetitive, rhythmic structures as means of working around and toward history, of sounding the past and trying to decode its meanings through listening to its continued reverberations in the present. The novelist Daniel Maximin in turn uses Glissantian techniques in his investigation of circum-Caribbean history. Rhythms and repetitions are at once integral elements of narrative structure and also cultural tropes that unite the Caribbean with the broader New World and in particular the great musical traditions of North America.\footnote{On rhythm in Césaire, Fanon, Ménil, and Glissant, see Martin Munro, {\em Different Drummers Rhythm and Race in the Americas} (Berkeley, University of California Press, 2010), chap.~3.}

Perhaps Maximin's interest in and affinity with African American experience is in part due to the fact that in the United States, as in the French Caribbean, black nationalism did not lead to political independence, and that US black culture has had to negotiate its terms and its place vis-à-vis a politically and economically dominant white establishment. At the same time, however, since the early twentieth century, African American music has benefited from worldwide exposure and corporate backing in ways that music from other New World sites has not. When James Brown turned decisively to rhythm in the mid-1960s, the repercussions of that move were always going to be more significant than those of the similar rhythmic turns in the other places. Although black nationalists such as Amiri Baraka sought to racialize Brown's rhythms and limit their possible meanings, the rhythms of seminal tracks such as \quotation{Cold Sweat} and \quotation{The Payback} escaped this potential prison and spread across the globe---to Africa, where they were enthusiastically greeted as new creations, and to Europe, where everyday Europeans began to rediscover the exhilaration of finding the beat and getting with \quotation{the One.}\footnote{On rhythm, Black Power, and James Brown, see Munro, {\em Different Drummers}, chap.~4.} This process of rhythmic globalization had actually begun in the 1920s, with the spread of swing, boogie-woogie, and other forms of African American music. The global proliferation of rhythm---so often viewed as a marker of cultural primitivism---has paradoxically coincided with the expansion of modernity, urbanization, and industrial development.\footnote{Henri Lefebvre writes of how modern (Western) music has been shaped by a \quotation{massive irruption of exotic rhythms.} {\em Eléments de rythmanalyse} (Paris: Editions Syllepse, 1992), 58.} In a sense, too, the history of popular music from the 1960s to the present has seen the partial deracialization of rhythm, particularly since the late 1980s and the emergence of hybrid forms of dance music such as Chicago house and its various offshoots.\footnote{See also Paul Gilroy's argument that purpose-built reggae sound systems in English inner cities \quotation{disperse and suspend the temporal order of the dominant culture.} As the sound system wires are strung up and the lights go down,” Gilroy says, \quotation{dancers could be transported anywhere in the diaspora without altering the quality of their pleasures.} {\em There Ain't No Black in the Union Jack: The Cultural Politics of Race and Nation} (1987; repr., London: Routledge, 2002), 284.}

In the classic colonial situation of twentieth-century Trinidad, rhythmic popular culture was under constant attack from the white establishment, particularly the Protestant British. Carnival became the focus of dualistic debates over culture and rhythm, and throughout the nineteenth century the colonial authorities systematically repressed the rhythmic music of the masses. Slowly evolving from a mainly white celebration, Carnival was gradually appropriated by the black lower classes who translated their rhythmic culture and music into the secularized space of the festival, spreading fear among the whites who saw (and heard) constantly in black expressive culture intimations of revolt. The many ordinances banning drum playing only further strengthened the connection between black identity and rhythm. At every point, rhythm did not disappear but returned in new forms, by means of new instruments. Something like in Haiti but in a more muted way, in the early- to mid-twentieth-century, Trinidad's rising class of largely nonwhite intellectuals turned to popular culture as an untainted repository of authentic, rhythmic black culture. As the proindependence movement gathered momentum in Trinidad, rhythmic music was slowly rehabilitated into nationalist discourse. The decisive step in this evolution was the promotion of the steelpan, which had emerged from the long-neglected culture of the urban poor as the national instrument of independent Trinidad.

Ultimately, Trinidad's history, sounded out, shaped, and prophesied by the various changes in rhythmic music and developments in percussive instrumentation, is a history of popular improvisation. The apparently chaotic, unplanned shifts from one style of music, one kind of percussive instrument to another mask a deeper, irresistible purpose. While improvisation is usually understood as an act without foresight, prescriptive vision, or deep motivation, in practice it is often a knowing, future-oriented performance that operates as a \quotation{kind of foreshadowing, if not prophetic, description,} according to Fred Moten, that carries \quotation{the prescription and extemporaneous formation and reformation of rules, rather than the following of them.} In the history of Trinidad, music and other kinds of performative improvisations, are far from simple, naive, nonhistorical acts; instead, they foresee and embody what Moten calls \quotation{the very essence of the visionary, the spirit of the new, an organizational planning of and in free association that transforms the material.}\footnote{Fred Moten, {\em In the Break: The Aesthetics of the Black Radical Tradition} (Minneapolis: University of Minnesota Press, 2003), 63, 64.} Improvisation is always also a matter of time and sight, a glance to the future, and in Trinidad's case, the occasion to see the shape of that future, which has been communicated primarily through sound, rhythm, and the music without which the great transformations in Trinidadian society from the colonial era to today could never have taken place.\footnote{For more on the history of rhythm in Trinidad, see Munro, {\em Different Drummers}, chap.~2.}

The case of Trinidad, and of the history of the steelpan in particular, is important, since it offers an implicit critique of the idea put forward by Nelson George that modern, technological innovation inevitably diminishes the \quotation{blackness} of the music. Fashioned literally from the wreckage of industrial society---bits of old cars, paint pots, dustbins, and cement and oil drums---the steelpan is a quintessentially modern invention, and yet its modernity apparently does not loosen its association with black identity. On the contrary, its history suggests that it is rhythmic innovation and technological invention that are some of the defining aspects of black identity in the Americas.

This is a line of thought expressed by Alexander G. Weheliye in {\em Phonographies: Grooves in Sonic Afro-Modernity}. Weheliye questions the idea that the human and the technological can be definitively separated, and he proposes that \quotation{one is hardly conceivable without the other.} More specifically, he critiques the common placement of black culture and being beyond the realm of the technological, the notion that \quotation{Afro-diasporic populations are inherently Luddite and therefore situated outside the bounds of modernity.}\footnote{Alexander G. Weheliye, {\em Phonographies: Grooves in Sonic Afro-Modernity} (Durham, NC: Duke University Press, 2005), 2.} Similarly, he questions the neglect in most academic work on technology of what he calls the \quotation{sonic topographies of popular music,} which he says has long been one of the most fertile areas for the \quotation{dissemination and enculturation of digital and analog technologies.} Popular music is also the primary domain, he says, in which black subjects have engaged with technology, so that \quotation{any consideration of digital space might do well to include the sonic in order to comprehend different modalities of digitalness, but also not to endlessly circulate and solidify the presumed \quote{digital divide} with all its attendant baggage.}\footnote{Ibid., 2--3, 3.}

{\em Phonographies} proceeds to examine some of the many ways twentieth-century black cultural production intersected and engaged with sound technologies, from the phonograph to the Walkman. Throughout, Weheliye regards the technologization of black music and culture not as moments of inauthenticity but as \quotation{a condition of (im)possibility for modern black cultural production.} In general, he makes a compelling case for the reconsideration of the idea that black cultures are pre- or antitechnological and for the study of sound recording and reproduction in relation to black cultural production as a means of identifying the \quotation{singular mode of (black) modernity.} As he puts it later in the work, there is \quotation{no Western modernity without (sonic) blackness and no blackness in the absence of modernity.}\footnote{Ibid., 12, 3, 174.}

This is an idea Weheliye restates in the recent {\em Small Axe} discussion of his book alongside Julian Henriques's {\em Sonic Bodies}, another work that in its analysis of the sound system considers blackness to be integral to sonic technological innovation and vice versa.\footnote{Julian Henriques, {\em Sonic Bodies: Reggae Sound Systems, Performance Techniques, and Ways of Knowing} (London: Continuum, 2011). For the discussion in {\em Small Axe} of this book and Weheliye's {\em Phonographies}, see Tavia Nyong'o, \quotation{Afro-philo-sonic Fictions: Black Sound Studies after the Millennium,} {\em Small Axe}, no. 44 (July 2014): 173--79; Alexander G. Weheliye, \quotation{Engendering Phonographies: Sonic Technologies of Blackness,} {\em Small Axe}, no. 44 (July 2014): 180--90; and Julian Henriques, \quotation{Dread Bodies: Doubles, Echoes, and the Skins of Sound,} {\em Small Axe}, no. 44 (July 2014): 191--201.} Henriques, in his {\em Small Axe} response, echoes to some degree Weheliye's argument on the importance of blackness to modernity, saying that \quotation{those human beings whom slavery attempted to reduce to machines are precisely the ones who gave the modern world its musical humanity} and that \quotation{there is little disjuncture between African sensibilities and technologies.}\footnote{Henriques, \quotation{Dread Bodies,} 200.} Henriques's {\em Sonic Bodies} shows how dancehall \quotation{riddims} draw on Kumina rhythms and that, as such, \quotation{ancient old-world African traditions \quote{come up to the time} \ldots{} with the latest digital technologies.}\footnote{Henriques, {\em Sonic Bodies}, 12.} His analysis of the figure of the emcee further demonstrates the dynamic fusion of traditional performance motifs and ultra-contemporary technological innovation to create a form that constantly renews itself. Henriques talks of how under the emcee's direction \quotation{rhythmic energies flood the show} and of how these energies range from the \quotation{electromagnetic flows within the set, such as the forces of electrical currents \ldots{} to the audio-cultural flows of the music, the musical flow of the selector's segueing from one track to the next, the corporeal and kinetic flow of the crowd, and the socio-cultural flow of the emcee's lyrics themselves.}\footnote{Ibid., 191--92.} Rhythm in this context is \quotation{energy, force, or flow} and is related to time through the \quotation{ultra-fast turnover} of styles, sounds, and fashions that characterizes dancehall.\footnote{Ibid., 192.} This is rhythm much as James Brown saw it---as a force of change and renewal, the embodiment of the imminent future.

Similarly, in his contribution to the {\em Small Axe} discussion, Weheliye states of his book: \quotation{Instead of imagining Afro-diasporic cultures as disconnected from the heart of modernity's whiteness, I demonstrate how black cultures have contributed to the very creation and imagination of the modern, interrogating the facticity of blackness, that is, how certain groups of humans became black through a multitude of material and discursive powers.} This recognition that blackness is an integral element of technological modernity has for Weheliye a potential political value: if blackness is commonly judged to exist beyond what he calls the \quotation{iron grip of the West and modern technologies,} despite it being a product of these forces, such a recognition offers a way of dismantling \quotation{the coloniality of being in Western modernity} that is the sense of separation, almost of exile from modernity that informs certain understandings of blackness, across the Americas, and since at least the Negritude movement. By insisting on the fundamental importance of black people and cultures to the \quotation{territory,} as he puts it, of Western modernity, its unique association with whiteness and alienation is diminished.\footnote{Weheliye, \quotation{Engendering Phonographies,} 181.} It is telling that he uses metaphors of space in his imagining of the political potential of recognizing blackness as an integral part of modernity, since they suggest a sense of belonging physically and culturally to a place, and a possible way out of the feeling of alienation and exile that are found, again, in classic formulations of blackness from Negritude to indigenism and exist still in residual form in Nelson George's critique of disco. Crucially, too, such a recognition does not imply the loss of the specificity of \quotation{black life} itself; rather, it calls into question the nature of modernity, and blackness is reconsidered as a dynamic, evolving, unpredictable phenomenon, which is not (only) to be recovered from the past but (also) is something that is in constant, irregular, future-oriented creation.

Although Weheliye does not refer extensively to rhythm, it follows from his argument that blackness is an effect of modernity, that the blackness-rhythm equation that George invokes is similarly related to and dependent on the binary construction of whiteness and blackness whereby the former is related to technology and machinery and the latter to earthiness and untainted humanity.

A similar argument is proposed by Peter Shapiro in his 2006 work {\em Turn the Beat Around}, which is a more straightforward history of disco music and which suggests that the \quotation{standardized meter and mechanical beats of disco can be traced back to the very birth of African-American music.} In his third chapter, subtitled \quotation{Automating the Beat,} Shapiro pays particular attention to rhythm and traces the history of regular, repetitive beats to Congo Square and the \quotation{regimentation and rigidity} of marching band music.\footnote{Shapiro, {\em Turn the Beat Around}, 90.} He also hears in the rhythms of Bo Diddley and Kraftwerk echoes of the regular, mechanical sounds of the railroads, and in the latter's machine-produced beats he hears something that he says exists in all music: the \quotation{trance ritual,} the \quotation{eternal rhythm loop that could transport you across its waves of sound.}\footnote{Ibid., 93.} Given what he calls the \quotation{strong presence of automated beats and mechanical rhythms throughout the history of African-American music,} Shapiro argues that critics of disco's \quotation{machinelike} qualities such as George tend to ignore the regular, repetitive rhythms of much black American music of the twentieth century and to rely on a nonexistent contradiction between blackness and technological modernity.\footnote{Ibid., 97.} George is right, to the extent that there is a long and complex relationship between rhythm and notions of blackness in the Americas, but he seems not to recognize fully the historical conditions in which that association was formed and that the relationship was always dynamic and evolving, a product of and not an uncorrupted counter to technological modernity.

In many regards, this rhythm-centered debate echoes Fred Moten's critique of Amiri Baraka's notion of \quotation{black spirit,} an idea that black culture is experienced primarily in terms of deeply-felt, antitechnological instincts. Living in a hypermodern, technologized \quotation{white} world, Baraka, like earlier counterparts in the Caribbean such as Césaire and Jean Price-Mars, and somewhat like George in his critique of disco, evokes a vision of black culture as an organic though static and unchanging phenomenon, a state of being and feeling that existed outside the orbit of white experience and as a counterpoint to technologized modernity. Moten argues that Baraka searches endlessly for \quotation{the unique word, the essence, the meaning, the essential quality of \quote{being.}}\footnote{Moten, {\em In the Break}, 144.} Moten further interprets Baraka's conceptions of race and culture in terms of a Heideggerian confusion of {\em humanitas} with {\em animalitas}, that is the human with the animal, the biological. It is as if for Baraka, human nature is programmed biologically, and that racial differences are fixed and inherited inevitably. Moten's skepticism over Baraka's racialized notion of culture leads him to the question of rhythm and blackness, and he asks, pointedly, \quotation{What, then, is the connection between {\em animalitas} and rhythm?,} a question that recurs in various forms throughout the history of rhythm in the Americas. In the specific example Moten provides---his deconstruction of Baraka's racialized critique of the white jazz player Burton Greene---the contradictions of Baraka's theories are exposed. As Moten says, \quotation{Baraka uses the discourse of animality to dehumanize Greene, a discourse marked by race and rhythm, though he criticizes that discourse as it applies to blacks in the very same essay.}\footnote{Ibid., 145.}

Black performance for Baraka is in this way not just or not primarily the sound; it is, Moten says, \quotation{the unmediated performance of essential blackness (and whiteness) that is made apparent in the difference between sounds.}\footnote{Ibid.} It is in and on this space of racial difference, in which rhythm acts as a primary racial differentiator, that Baraka erects his barrier between earthy, authentic, antitechnologial black culture and alienated, rootless, technologized white culture.

There is also an important sexual element in Baraka's and George's discourses of race, rhythm, and music, in that in Baraka's case blackness is hypermasculine, and it self-consciously opposes itself to what Moten calls an \quotation{aestheticized Euro-cultural effeteness that is alienated, commodified, artifactualized, necessarily homosexual.}\footnote{Ibid., 150.} Similarly, George laments what he calls the \quotation{metallic sexuality} of Eurodisco that matched, he says, \quotation{the high-tech, high-sex, and low passion atmosphere of the glamorous discos that appeared in every American city.} His telling alignment of technology and emasculated black sexuality is related to his subsequent remark that most of the club deejays who were pushing disco music were \quotation{gay men with a singular attitude toward American culture, black as well as white.} While black female singers such as Donna Summer, Grace Jones, and Gloria Gaynor were promoted, black male singers, including the funk artists that he sees as bastions of \quotation{raw} rhythmic, male music, were \quotation{essentially shunned.}\footnote{George, {\em Death of Rhythm and Blues}, 154.}

In a similar sense, Baraka's black, male \quotation{earthiness} functions as a corrective to a feminized white aesthetic by deploying what Moten calls a \quotation{purely and necessarily heterosexual, socially realistic (if not naturalistic), lyric masculinity.} This extreme masculinism, as Moten recognizes, is a product of history, a response to and repudiation (and, he says, repetition) of the historical violation of black maternity. It is he says a masculinist radicalism born of the severance of filial ties, \quotation{the aesthetic and political assertion of motherless children and impossible motherhood.}\footnote{Moten, {\em In the Break}, 151, 215.} It is therefore a defensive, compensatory masculinism that responds to very real historical traumas but which tends to trap itself by reducing and limiting its own potential meaning. As perhaps the single most important aesthetic aspect of this male assertion, rhythm acts like mother and father, or stands in place of them in their absence, and is jealously guarded by Baraka and others who assert their earthy, grounded, authentic cultural attachments even as this assertion implies simultaneously their historical lack of attachment and their ongoing social and political emasculation.

If the ideas of Weheliye, Shapiro, and Moten counter and perhaps negate George's argument that electronic rhythms effectively \quotation{steal the soul} of black music, there remains a further issue that they do not engage with but that is crucial to the history of rhythm in circum-Caribbean societies. At many moments in the history of social and political change in the region---indigenism in Haiti, Negritude in Martinique, {\em negrismo} in the hispanophone Caribbean, the civil rights era in North America---rhythm has been allied to radical political movements as a marker of racial and cultural distinctiveness, and in music and literature rhythm has been an integral part of anticolonial, radical expression. In the contemporary period, the relation between rhythm and radical political ideas is less apparent; indeed, one might say that rhythm is no longer an integral element in any poetics of resistance, has lost its association with radical blackness, and has become something of a deracialized, dehistoricized commodity.

Perhaps this apparent decoupling of radical politics and rhythm may be explained in part by the rather loose, unquestioning definition of what constitutes the \quotation{radical} in the discourses of blackness that I have mentioned thus far: chiefly Negritude, Haitian indigenism, and Black Power. These movements would no doubt be included in Anthony Bogues's definition of the \quotation{black radical tradition} as a \quotation{distinct political and intellectual tradition in which the markers are African and African Diaspora elaborations of ideas, practices, cultural and literary forms, as well as religious formations and political philosophy.}\footnote{Anthony Bogues, \quotation{C. L. R. James, Pan-Africanism, and the Black Radical Tradition,} {\em Critical Arts} 25, no. 4 (2011): 484.} In each case, though in different ways and in different circumstances, rhythm was harnessed to a certain idea of blackness that presented itself as radical but that was also in some senses quite conservative, even reactionary in some regards. Negritude's essentialist idea of blackness has been critiqued by Fanon, Ménil, and Glissant in the ways I indicate above, while Haitian indigenism mutated over time into the {\em noiriste} ideology that underpinned the Duvalier dictatorships, and Black Power's radical edge was perhaps blunted by the masculinism exemplified by Baraka. Likewise, one might say that the idea of rhythm in many of these cases was quite traditional and conservative---the relatively straightforward, repetitive beats of \quotation{African rhythm} that could be incorporated into poetry and music to create a simulacrum of radical aesthetics and thought but that retained a conservative edge that may have limited its political impact and allowed it to be appropriated and diluted in the ways George decries.

The truly radical rhythms, in the sense of formal experimentation, and perhaps also the most searching explorations of modern blackness of the mid-twentieth century, were found in styles such as free jazz. As Ekkehard Jost puts it, \quotation{Without question, free jazz, with its retreat from the laws of functional harmony and tonality, the fundamental rhythm that went throughout, and the traditional form schemes, posed the most radical break in the stylistic development of jazz.}\footnote{Ekkehard Jost, quoted in George A. Lewis, {\em A Power Stronger Than Itself: The AACM and American Experimental Music} (Chicago: University of Chicago Press, 2008), 41.} George Lewis draws a distinction between \quotation{nonrepresentational,} experimental \quotation{white} art of the period and the improvised, formally daring music of contemporary black artists who, he says, insist \quotation{music has to be \quote{saying something.}}\footnote{Lewis, {\em A Power}, 41.} Every effort, Lewis says, was made by black musicians \quotation{to recover rather than disrupt historical consciousness} and \quotation{the new black musicians felt that music could effectuate the recovery of history itself.}\footnote{Ibid., 42.} The rhythms of free jazz expressed, in a sense, an idea of blackness, or of being {\em tout court}, that was always on the edge, unpredictable, opaque, and unknowable in ways that reflect in turn the more radical expressions of postplantation identity that one finds in, for example, Glissant---the sense of a history and an identity that shimmers, as it were, or quivers rhythmically, in a way, rather than ever coming into full, knowable being. It is therefore no surprise that in one of his relatively rare references to jazz music, Glissant states, \quotation{My writing style is the jazz style of Miles Davis.} When asked how the two styles could be linked, Glissant answers, \quotation{Through rhythm.}\footnote{Edouard Glissant, quoted in Jean-Luc Tamby, \quotation{The Sorcerer and the {\em Quimboiseur}: Poetic Intention in the Works of Miles Davis and Edouard Glissant,} in Martin Munro and Celia Britton, eds., {\em American Creoles: The Francophone Caribbean and the American South} (Liverpool: University of Liverpool Press, 2012): 148, 153.} And following a searching comparative analysis of the two artists' work, Jean-Luc Tamby concludes, \quotation{Through breathing, through {\em la mesure} and {\em la démesure}, and in the figure of the spiral, we can see that the work of these two creators, each in its own way, is centred on rhythm.} As Glissant says, rhythm is more than a feature of aesthetic style; it is \quotation{a lever of awareness,} which \quotation{opens up space,} liberating, Tamby says, not just style but the imagination and existence itself, carrying with it \quotation{foreknowledge} of a \quotation{complex and unexpected new region of the world.}\footnote{Tamby, \quotation{Sorcerer and the {\em Quimboiseur},} 161, 162.}

What is ultimately at stake in this discussion of rhythm and race in the \quotation{digital age,} and in what sense might rhythm function still as a \quotation{lever of awareness}? In partial answer to that question, and to consider the issue of rhythm and race from a slightly different, though I think complementary, perspective, I close by referring to David Scott's recent argument that this is a stalled, tragic time in the Caribbean, and I draw connections between the apparent redundancy of revolutionary, anticolonial thinking in the present and the perhaps less apparent decoupling of rhythm and race in contemporary musical styles. This disengagement may be related to Scott's idea of a distinct separation of time and history in the Caribbean. In Scott's terms, the once self-evident notion of the natural convergence between time and history requires radical revision, for history itself has in a sense stalled, in the Caribbean and in the broader world. For Scott, history no longer unfolds in a process of \quotation{discrete but continuous, modular change} or as a \quotation{linear, diachronically stretched-out {\em succession} of cumulative instants.} The future no longer overcomes the past, and the present is not experienced as a \quotation{state of expectation and waiting for the fulfillment of the promise of social and political improvement.}\footnote{David Scott, {\em Omens of Adversity: Tragedy, Time, Memory, Justice} (Durham, NC: Duke University Press, 2014), 5.} The primary causes of the sense of historical blockage are the end of the era, again in the Caribbean and elsewhere, of \quotation{revolutionary socialist possibility} and the coming to prominence of \quotation{the new utopia of liberal democracy, its dogma of human rights, and the disciplining and governmental technologies to urge and enforce its realization.}\footnote{Ibid., 4.} As Scott writes, in the moment in which history has appeared so resistant to change, \quotation{time has suddenly become more discernible, more conspicuous, more at odds, more palpably {\em in question}.}\footnote{Ibid., 12 (emphasis in original).} The apparent end of the prospect of a discernibly different and improved political and social future in the Caribbean had led Scott to write previously that \quotation{we live in tragic times.}\footnote{David Scott, {\em Conscripts of Modernity: The Tragedy of Colonial Enlightenment} (Durham, NC: Duke University Press, 2004), 2.} The narrative of Caribbean history, and therefore of time, has shifted from the Romantic anticolonial model of change and overcoming to one of tragedy, which Scott prefers as a mode of interpreting the stalled present.\footnote{Ibid., 8.} In terms of time, to read Caribbean history as tragedy is to question whether the past can be truly disentangled from the present. In place of what Scott calls the \quotation{seemingly progressive {\em rhythm}} of time, tragedy presents \quotation{a broken series of paradoxes and reversals,} unpredictable, nonlinear movements in time and history that confuse any sense of the distinctiveness of past, present, and future.\footnote{Ibid., 13 (emphasis in original).} It is telling that Scott finally invokes rhythm to refer to the apparently lost sense of moving forward in time and history, for it was rhythm that has historically opened colonial and postcolonial societies in the Americas, however imperfectly and fleetingly, through cultural performances such as carnival and Canboulay, and that, always turned toward the future, was among the most persistent means by which social hierarchies were destabilized and finally overturned. Historically, rhythmic performances in the Caribbean and the broader Americas have been future-oriented means of imagining a different, better time to come, and if we see in the present a decoupling of rhythm and race, it is perhaps at least in part because history itself has lost its own forward momentum, its own rhythm of change and renewal. If, however, we start to look beyond the apparently stalled present and consider the dynamic, often unforeseen, ways rhythm has shaped black (and white) being in the Americas, we are left with a final, tantalizing question: could it be that the decoupling of rhythm and race is temporary, or that rhythm is already, as it has always been, in and of the future, a form of foreknowledge levering awareness, as Glissant says, and that its presence in new digital arenas of black (and nonblack) production and performance constitutes already the pre-echo of an as-yet-unheard or -unseen, reconfigured, newly politicized notion of blackness (and perhaps race more broadly) to come?

\thinrule

\page
\subsection{Martin Munro}

Martin Munro is Winthrop-King Professor of French and Francophone Studies at Florida State University. He previously worked in Scotland, Ireland, and Trinidad. His recent publications include: {\em Writing on the Fault Line: Haitian Literature and the Earthquake of 2010} (Liverpool, 2014); and {\em Tropical Apocalypse: Haiti and the Caribbean End Times} (Virginia, 2015). He is Director of the Winthrop-King Institute for Contemporary French and Francophone Studies at Florida State.

\stopchapter
\stoptext