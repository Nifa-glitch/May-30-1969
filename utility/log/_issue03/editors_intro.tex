\setvariables[article][shortauthor={Glover, Gil}, date={July 9 2019}, issue={3}, DOI={10.7916/archipelagos-6nh2-pj27}]

\setupinteraction[title={},author={Kaiama L. Glover, Alex Gil}, date={July 9 2019}, subtitle={}]
\environment env_journal


\starttext


\startchapter[title={}
, marking={}
, bookmark={}]


\startlines
{\bf
Kaiama L. Glover
Alex Gil
}
\stoplines


Two of our articles in this issue lead with an epigraph by singular Trinidadian thinker C. L. R. James. Nick Nesbitt evokes the James who points to revolution in the control of modes of production, and Marisa Parham conjures the James who calls for revolution in the form of \quotation{speculative thought.} These two conceptual bookends capture well what the overall project of {\em archipelagos} has strived to be. As David Scott wrote in his introduction to our first issue, \quotation{The Caribbean won't stand still.} In our role as editors of this platform, we have tried to embrace that inherent reality---to gather and care for work that honors the ever-changing intellectual shorelines of our Caribbean and its study. In doing so, we have been open to experimentation and innovation, letting go of our material and formal assumptions as editors and producers. That said, innovation, in its mainstream guise, is a notion that calls for a measure of caution, too. The idea that whatever is bright, shiny, and new is somehow better than---or a replacement for---what's come before is unhelpful. Innovation for innovation's sake risks wasting valuable resources and hiding labor; it risks distracting from the substance of the work and its maintenance. And so we've tried also to make sure that what we propose are deeply considered {\em interventions}---thoughtful forays into the digital and machinic new that remain mindful of the material and personal realities from which that new has emerged.

We feel confident that our third issue of {\em archipelagos} strikes that balance, and this in large part is thanks to our collaboration with Jessica Marie Johnson. Well aware of Johnson's commitment to both rigor and inventiveness in her scholarly praxis, we readily left our baby in her hands in putting together this issue. We were certain that her proposed special section, \quotation{Slavery in the Machine,} would bring the inquiries of our prior two issues into provocative new spaces---both in the breadth of the content she envisioned and in the technological fluency that undergirds her imagination. In step with Johnson's vision, we have pushed ourselves to reconsider genre boundaries and to experiment---dare we say, to innovate---in our methods of production. We leapt at the chance to work with her, seeing this as the first of many collaborations we hope to sustain with fellow pioneers and supporters of the Caribbean digital. Indeed, we built this space so that scholars like her would come.

Not only does the first section offer finely considered theoretical explorations of the machine as metaphor along with reflections on the infinitely diverse multimedia spaces of digital storytelling, but Johnson has also brought to {\em archipelagos} our first truly multivalent hybrid of an intervention---a generative enactment of the glitch that explodes our habits of reading online and lays bare the ways these habits are overdetermined by machines all-too-often outside our control. The issue's second section reviews a platform that asks explicitly how careful markup and interface can be deployed to trouble old stories. This section further explores the ways and the stakes of storytelling otherwise via its inclusion of a \quotation{reporting in} from the creators of a digital project we featured in our inaugural issue. Both these platforms illustrate with gratifying clarity how creative use of network technology can serve humanistic inquiry. In this regard, they evince precisely the kind of work and the standards of praxis we have laid out for ourselves in developing {\em archipelagos}.

We hope that, like us, you will see this gathering of digital reflections and praxes as participating in the project of (re)shaping expectations around ways of being and knowing in the academy from the {\em specific} space of our Americas. Welcome, then, to our grounded and innovative---our past-, present-, and future-looking---third issue of {\em sx} {\em archipelagos}.

Onward,
\strut Kaiama & Alex
\strut Editors

\page
\subsection{Kaiama L. Glover}

\useURL[url1][https://barnard.edu/profiles/kaiama-l-glover][][Kaiama L. Glover]\from[url1] is Associate Professor of French and Africana Studies at Barnard College, Columbia University. She is the author of \useURL[url2][http://liverpooluniversitypress.co.uk/products/61903][][Haiti Unbound: A Spiralist Challenge to the Postcolonial Canon]\from[url2] (Liverpool UP 2010), first editor of \useURL[url3][http://yalebooks.com/book/9780300214192/yale-french-studies-number-128][][Marie Vieux Chauvet: Paradoxes of the Postcolonial Feminine]\from[url3] (Yale French Studies 2016), and translator of Frankétienne's Ready to Burst (Archipelago Books 2014). She has received awards and fellowships from the National Endowment for the Humanities, the Mellon Foundation, and the Fulbright Foundation. Current projects include forthcoming translations of Marie Vieux Chauvet's {\em Dance on the Volcano} (Archipelago Books) and René Depestre's {\em Hadriana in All My Dreams} (Akashic Books), and the multimedia platform {\em In the Same Boats: Toward an Afro-Atlantic Visual Cartography}.

\subsection{Alex Gil}

\useURL[url4][http://www.elotroalex.com/][][Alex Gil]\from[url4] is the Digital Scholarship Librarian at Columbia University Libraries. His research and practice focuses on digital humanities, epistemic design, minimal computing, and Caribbean literature. He is co-founder and moderator of \useURL[url5][http://xpmethod.plaintext.in/][][Columbia's Group for Experimental Methods in Humanistic Research]\from[url5], and the Studio@Butler at Columbia University Libraries.

\stopchapter
\stoptext