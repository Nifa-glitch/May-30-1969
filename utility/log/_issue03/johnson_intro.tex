\setvariables[article][shortauthor={Johnson}, date={April 2019}, issue={3}, DOI={10.7916/D8NS1689}]

\setupinteraction[title={},author={Jessica Marie Johnson}, date={April 2019}, subtitle={}]
\environment env_journal


\starttext


\startchapter[title={}
, marking={}
, bookmark={}]


\startlines
{\bf
Jessica Marie Johnson
}
\stoplines


i,

\starttyping
<li>the crosssroads</li>
<li>spirit</li>
<li>the dead</li>    
<li>the deathless</li>  
<li>the living</li>   
<li>humanity</li>
\stoptyping

carried

\starttyping
<li>the fractional</li>
<li>the missing</li>
<li>the destroyed</li>
\stoptyping

(this,

political and deliberate lost-in-translation
masking
testimony
mi isla
)
for years

\starttyping
<li>of collaboration</li>
<li>of accountability</li>
<li>of community</li>
\stoptyping

and years

\starttyping
<li>of creating</li>
<li>of remixing</li>
<li>of redoing</li>
<li>of recovering</li>
\stoptyping

Remix of Ibeyi, \quotation{I Carried This for Years,} Ash (2017), using multiverse.js by Ed Summers.

French-Cuban songstresses Lisa-Kaindé Diaz and Naomi Diaz of Ibeyi wrote \quotation{Deathless} out of a haunting. Describing the inspiration behind the song, Lisa-Kaindé told the story of being harassed by a police officer in Paris at sixteen years old. The harassment only stopped when the police officer wrenched her bag from her hands and a book from it spilled on the floor. She related, \quotation{And I think he thought, \quote{Oh, she might be intelligent and have something in her head.} So he just gave me my empty bag and left.}\footnote{Alyssa Edes, \quotation{Ibeyi on Spirituality and Joy in \quote{Ash,}} 4 October 2017, https://www.kbbi.org/post/ibeyi-spirituality-and-joy-ash.} Against saxophonist Kamasi Washington's lilting playing, two haunting verses narrate that moment of terror, evoking the confusion and fear of a teenaged girl who was, \quotation{Innocent / Sweet sixteen / Frozen with fear.} At the same time, in the breaks between these verses, the chorus of the song tells a different story. Defying the authority and threat embodied in the looming figure of the police officer, Ibeyi sings out in defiance, \quotation{Whatever happens, whatever happened / Oh hey / We are deathless / We are deathless.}

This special issue of {\em sx archipelagos}, \quotation{Slavery in the Machine,} sings out with that same defiance. In the face of environmental catastrophes, \quotation{unpayable debt} levied by a global North created out of plantation profits, police and military violence, and the continued predations of US empire, the Caribbean and its diaspora refuse death. Our women, children, and men rebuke the \quotation{bad press} spoken in their name by outsiders. Our artists refuse to be bound by the permutations of forced migration or nation-centric narratives of being, cleaving instead to affinities, networks, relations that cross islands and mainlands, race and empire. If this story of refusal seems timeless as well as deathless, or if in evoking a dense humanity and attention to spirit, the past, present, and future appear to cycle back through wind currents ridden by conquistadores in 1492 or the combustion of revolt that marked 1791, this, too, is part of the work. Which is to say, refusing linear historical accounts for multilayered and collagic narratives is accountable Caribbeanist scholarly praxis. Refusing material or empirical knowledges for the fractional and fractal, the hidden books, the hauntings at the crossroads, is diasporic and ancestral intellectual labor. If being deathless requires an art, a discipline, an ethic, a way of being, then the Caribbean has death mastered.

And if being deathless requires a code book, a metadata, a markup language, then the contributors to this issue have dared to write it. What does the digital have to say, after all, to a region of the world that has often been framed as natural and primitive, exotic and excessive, but rarely as possessing its own indigenous technologies for unpacking the brutalities of the New World? The contributors to \quotation{Slavery in the Machine} speak across historiographies and archives, literatures and audiences to provide an array of answers. As a body, the contributions gathered here challenge readers to expand their frame in horizontal and vertical directions---to stand at the crossroads---and reconsider questions, digital as well as analog, that humanists have taken for granted. Digital projects build themselves around interactions with users. Liberal humanism roots itself in the power of the individual, the head of household, Man. In contrast, the contributors to this special issue question whether self and subjectivity can exist beyond the kinship of community, the spirit world, the living and the dead, humanity, or the earth itself. Digital humanities has made significant inroads in the social sciences and humanistic study more broadly, particularly in the fields of history and English. These fields privilege empirical analysis, large data sets, broad swaths of collected material. Instead of demanding empiricism, the contributors to this special issue take the chaos of colonialism, the destruction of the hurricane, and the cleansing heat of the slave revolt's fire as evidence of the impossibility of the complete data set. Instead, they embrace disaster as data, gather the fractional, the missing, and the destroyed and hold those null values as sites of analysis that center Caribbean realities and mysteries.

The contributors to this special issue trust in black diasporic technologies that predate the digital age. From masking, to evading state testimony, to being lost in translation, to diasporic longing for home(s/lands) and communal spaces, the Caribbean has long offered its own strategies for decoding antiblackness and encoding anticolonial scripts in its people. These contributors take these indigenous knowledges and apply them to code and tech, creating interfaces that reach across disciplines in collaboration, that shape new kinds of community in the alternate time/space of the online forum, and that outline ethics for practicing a radical accountability to Caribbean people themselves. As a result, the contributors apply the digital to the work of creating, remixing, redoing, and recovering humanity from the machine that slavery beget, while also rebuking any framing of the digital that sets slavery, black life, or the Caribbean outside the analysis.

The contributions to this issue offer a rich range of work to engage. \quotation{Slavery in the Machine} is composed of three essays, one hypertext experience, two project reflections, and one project review. In an essay on the stakes of and methodologies for producing new narratives of Haiti, Marlene Daut challenges scholars to \quotation{ensure that the new or alternative narratives we encourage, craft, and center to oppose bad press will in fact not generate similarly, or perhaps newly, hostile and harmful images of Haiti.} Using the crossroads as her methodologic sounding board and analyzing several digital projects that take up Haiti or the Haitian Revolution as their primary topic, Daut explores how \quotation{using multimodal archival methods to document the Haitian past can help us to better understand the political processes that drove Haitians to create and sustain an independent nation-state in the nineteenth century.} In \quotation{The Slave-Machine: Slavery, Capitalism, and the \quote{Proletariat} in {\em The Black Jacobins} and {\em Capital},} Nick Nesbitt explores the machinery of capitalism. Centering C. L. R. James's claim in the iconic text {\em The Black Jacobins} that the enslaved of eighteenth-century Saint Domingue were \quotation{closer to a modern proletariat than any group of workers in existence at the time,} Nesbitt compares the powerful political rhetoric in James's text with Marx's more narrow definition of {\em proletariat} and {\em capitalism}. In \quotation{Xroads Praxis: Black Diasporic Technologies for Remaking the New World,} I return to the devastation spread by Hurricane María and use the blackout created by the hurricane---a blackness so intense it could be seen from space---as a jumping off point for considering what missing data, black spaces, \#BlackTheory, and histories of slavery can do to humanize digitized data sets. By proposing an \quotation{xroads praxis,} I challenge digital humanists to explore \quotation{what digital and analog landscapes hide and reveal, seeing them in their fullness after the storm of 1441/1492, finding black space on the map that does not conclude in black death.} Marisa Parham's \quotation{Breaking, Dancing, Making in the Machine: Notes on .break .dance} proceeds to do precisely this work. Like Daut and me, Parham uses the crossroads, but she expands her analysis to include movement and ephemerality as codes in and of themselves. Parham surfaces black diasporic knowledge as a constituent technology in the making of the New World. This groundbreaking essay is the highest form of humanistic study, digital or analog. Parham's \quotation{.break .dance} exemplifies the kind of rigorous, creative, and challenging possibilities that unfold when black diasporic intellectual work is taken seriously, and digital tools and coding languages are learned for their theoretical as well as their material function, not as grant-chasing pursuits.

In keeping with the theme of missing data and defying death, Ada Ferrer and Linda Rodríguez explore {\em Digital Aponte}, their project translating the life and paintings of nineteenth-century Cuban artist and insurgent José Antonio Aponte into digital form. Spanish officials arrested Aponte during a series of slave revolts in 1812. Officials used his book of painting---a fantastic collection of original, collage, and remixed images featuring black saints, Toussaint Louverture, Egyptian iconography, and more---as evidence of his guilt. The book of paintings, however, has not surfaced in the years since. Ferrer and Rodriguez explore what it means to relate a life like Aponte's. He was a man whose artistic transgressions occupied Spanish investigators and led to his execution (at the crossroads), and his story would seem to end with his death. But it is in his masking, that is, in the work he hid, destroyed, dissembled, and avoided discussing, that Ferrer and Rodriguez find inspiration. In \quotation{The Caribbean Digital and Open Peer Review: A {\em Musical Passage} Hypothesis,} Laurent Dubois, David Garner, and Mary Caton Lingold use the missing, the ephemeral, and the haunting echo of data as music and musical production to imagine community differently. Describing their process of peer review and the online forum they created to discuss and share work related to {\em Musical Passage}, Dubois, Garner, and Lingold remind us that one of the many contributions technology offers scholars of the Caribbean and the African diaspora is the opportunity to find intellectual, political, and sonic kin by creating virtual spaces that transgress borders. Lizabeth Paravisini-Gebert, in her review of {\em Puerto Rico Syllabus} notes a similar community formed from Columbia University's Unpayable Debt Working Group. In 2016 the Puerto Rico Oversight, Management, and Economic Stability Act (PROMESA), signed by President Barack Obama, placed an impossible burden of debt and austerity on the island of Puerto Rico, leading to massive student and activist protests in Puerto Rico and on the mainland. Using the hashtag syllabus as form, medium, and genre, the {\em Puerto Rico Syllabus} surfaced a conversation---a missing dataset---that challenged these austerity measures with evidence from across the disciplines and in a public format accessible to organizers, teachers, and laypeople on the ground.

This special issue, with its wide range of themes, techniques, styles, methods, and applications, deals in black diasporic knowledge as a technology that has been like Aponte's insurgent book---hidden \quotation{deep inside a trunk full of clothes} in \quotation{a pine box with a sliding top.} Not absent, but hidden. Not lost, but secreted away. Never rejected, simply folded up for secret pleasure alone or with kin. Not accountable or even legible to institutions, but brimming with multiple legibilities and infinite literacies, accountable to black life, joy, and freedom projects around the world. Not dead, but deathless. Or, to remix another song by Ibeyi, we have carried

\starttyping
<li>the crosssroads</li>
<li>spirit</li>
<li>the dead</li>
<li>the deathless</li>
<li>the living</li>
<li>humanity</li>
\stoptyping

for years and years. In this issue, scholars of the Caribbean and the African diaspora offer a glimpse of that joy and burden and what it means to confront slavery in the machine.

 This special issue is dedicated to scholar, organizer, and beautiful artistic soul Linda Rodriguez (1978--2018). There are no words to describe the pain of losing her, a pain everyone who knew her shared. Linda, we miss you. We have more to learn from you. We continue to read, to study, and to listen.

\thinrule

\page
\subsection{Jessica Marie Johnson}



\stopchapter
\stoptext