\setvariables[article][shortauthor={Paravisini-Gebert}, date={July 9 2019}, issue={3}, DOI={10.7916/archipelagos-xnx4-bz51}]

\setupinteraction[title={Review of *Puerto Rico Syllabus*: Essential Tools for Critical Thinking about the Puerto Rican Debt Crisis},author={Lizabeth Paravisini-Gebert}, date={July 9 2019}, subtitle={Review of *Puerto Rico Syllabus*}]
\environment env_journal


\starttext


\startchapter[title={Review of {\em Puerto Rico Syllabus}: Essential Tools for Critical Thinking about the Puerto Rican Debt Crisis}
, marking={Review of {\em Puerto Rico Syllabus}}
, bookmark={Review of *Puerto Rico Syllabus*: Essential Tools for Critical Thinking about the Puerto Rican Debt Crisis}]


\startlines
{\bf
Lizabeth Paravisini-Gebert
}
\stoplines


\useURL[url3][https://puertoricosyllabus.com/][][{\em Puerto Rico Syllabus}]\from[url3] is the website for a public syllabus project led by Yarimar Bonilla, Marisol Lebrón, and Sarah Molinari, stemming from the work of the Unpayable Debt working group at Columbia University convened by Frances Negrón-Muntaner and Sarah Muir. Intended as \quotation{a carefully curated portal with various points of entry,} its goal is that of exploring \quotation{critical questions about the role of debt in contemporary capitalism; the relationship between debt, migration, and violence, and the emergence of new political and cultural identities.}\footnote{\quotation{About,} {\em Puerto Rico Syllabus}, puertoricosyllabus.com.} Begun in 2017, {\em Puerto Rico Syllabus} (\#PRSyllabus) identifies salient themes related to Puerto Rico's economic history, offering a collaboratively curated bibliography (often linked to full-text articles) to help the reader access relevant texts, videos, and websites. Its proposed chronologically organized syllabus takes the 1898 US annexation of the island as a point of departure, exploring the connections between imperial exploitation and the island's 2017 filing for bankruptcy protection, paying particular attention to recent developments, such as the imposition of a fiscal control board in 2016, and to the impact on the island's already fragile economy of Hurricanes Irma and Maria in September 2017. The site underscores its collaborative nature by featuring a clear and prominent invitation to readers to join the \#PRSyllabus community as a subscriber to their list, as a collaborator, or as a Twitter follower. Portions of the website are translated into Spanish, with a declared intention to make the site more bilingual and accessible.

\#PRSyllabus is connected to the emergence of public syllabi as spaces for educators to share materials, questions, and ideas for class discussion of complex or difficult topics or events through a hashtag. The first of these, \#FergusonSyllabus, was developed by Marcia Chatelain, an associate professor of history and African American studies at Georgetown University, to support classroom conversations about the fatal shooting of eighteen-year-old Michael Brown Jr.~by a white police officer in Fergusson, Missouri, in August 2014. Public syllabi on the massacre at Charleston's Emanuel African Methodist Episcopal Church in June 2015; on Donald Trump's election in 2016; on the Black Lives Matter movement; on the Standing Rock Sioux in their assertion of territorial sovereignty; on welfare reform, prison abolition, immigration, Islamophobia, and (even) Colin Kaepernich have followed. The format has proven popular among scholars engaged as a community of like-minded peers seeking to facilitate access to carefully selected resources from which other scholars can glean information, guidance, and resources for teaching and research. This crowdsourcing of ideas and information is well suited to collaborative endeavors that benefit from access to a large network of potential contributors. In their definition of {\em crowdsourcing}, Enrique Estellés-Arolas and Fernando González Ladrón de Guevara underscore the participative nature and online basis of a proposal from an individual or institution to \quotation{a group of individuals of varying knowledge, heterogeneity, and number, via a flexible open call {[}for{]} the voluntary undertaking of a task.}\footnote{Enrique Estellés-Arolas and Fernando González Ladrón de Guevara, \quotation{Towards an Integrated Crowdsourcing Definition,} {\em Journal of Information Science} 38, no. 2 (2012): 189--200; PDF available at \useURL[url4][http://www.researchgate.net/publication/216804524_Towards_an_Integrated_Crowdsourcing_Definition/download][][www.researchgate.net/publication/216804524_Towards_an_Integrated_Crowdsourcing_Definition/download]\from[url4].} \#PRSyllabus is envisioned by its proponents as the first of a series of related public syllabi focused on the destructive impact of debt on Detroit, the Caribbean, Argentina, and Greece.

{\em Puerto Rico Syllabus} pays careful attention to both the history of Puerto Rico's colonial economy and its present crisis, presenting its multidisciplinary materials chronologically. This multidisciplinarity is key to its success, since the syllabus succeeds in providing a broad selection of materials that can be just as useful to the economist aiming to present a broad spectrum of the various impacts of the crisis on Puerto Rico's population as to the literary or art historian seeking to offer students an understanding of the crisis as a springboard to more nuanced cultural analysis. As a collaborative work-in-process, there is a degree of unevenness in the topics selected for inclusion as well as in the amount of materials available under each topic. This is natural in an organically developing collaboration, and the site will find its balance as the project develops. One can see how new sections will soon be needed---for example, a section on Puerto Rico's environmental crisis, now covered primarily in the \quotation{Hurricane Maria} section---but this is quite natural in dynamic projects such as this one.

The materials gathered at \#PRSyllabus consists primarily of recent articles, books, and scholarly essays---most written after 2000---with an understandable preponderance of newspaper and magazine articles in the approach to the present economic crisis. The readings are divided very usefully into primary readings and items for \quotation{further reading.} The materials listed are relevant and informative, and the brief but very helpful summaries of the contents of articles and essays guide the would-be instructor toward a selection of nuanced analyses of the topics offered for discussion. Most of the readings (except for book-length studies) are accessible as full texts through links to other websites. A number of very helpful videos (drawn from video-sharing websites like YouTube and Vimeo) are also made available. The choice and relevance of the materials---predominantly secondary sources---could be enhanced by the inclusion of primary texts and historical documents, film, and photographs through which students could learn additional analytical and interpretative skills. These could also offer a first-hand contemporary understanding of how past communities experienced this ongoing dynamic crisis.

The project leaders and contributors envision a fully bilingual site and have begun translating the contextualizing sections of the site (\quotation{About,} invitations to collaborate, etc.) from the original English into Spanish. The developing site, however, assumes a bilingual reader (instructor as well as student), since the articles and videos supplied appear randomly in either Spanish or English, depending on their source and provenance. This is, once again, a natural result of the project's collaborative practice, whose ideal \quotation{target} audience is clearly a bilingual user. The site includes \quotation{\useURL[url5][https://puertoricosyllabus.com/additional-resources/activists-organizations-and-citizen-initiatives/][][Activist Organizations and Citizen Initiatives]\from[url5],} a very helpful list of groups offering opportunities for users to become involved in efforts to address the crisis more directly and personally.

{\em Puerto Rico Syllabus}, built on a relatively simple WordPress theme, is reasonably easy to navigate following the table of contents that runs vertically down a right-hand sidebar accessible from all the website's pages, beginning with the landing page. The landing page itself introduces the project and its team and invites engagement and collaboration. It is free of ads. The site, nonetheless, could benefit from a redesign featuring horizontal rather than vertical navigation. As the number of topics on the sidebar continues to increase, the sidebar navigation forces the visitor to scroll down vertically beyond the point where the content of the main column ends. A horizontal menu near the top of the page with nesting pages would facilitate navigation while offering more flexibility in the organization of the materials, allowing for the highlighting of new or special topics through the creative use of widgets and footers. The landing page, where the viewer now scrolls down to access the \quotation{About,} \quotation{Goals,} and \quotation{Project Leaders} sections, in addition to a lengthy video titled {\em Exploring the Puerto Rico Syllabus Project}, could benefit from being \quotation{nested} horizontally below a footer image in order to keep navigation functionality simple. As it stands now, there is too much crucial information about the project \quotation{beneath the fold.} A comparison of the site to the other public syllabi sites highlighted on the \quotation{About} page shows the number of missed opportunities at the design level.

Visually, the site's appeal is limited. The WordPress theme used allows for a custom background image and a choice of palettes, but the former lacks definition and the latter---maroon type on a white background---lacks visual interest. The narrow visual impact comes from one photograph on the landing page (not a banner image) and the occasional embedded YouTube or Vimeo video. Given the wealth of visual materials associated with the themes of debt, development, migration, and natural disasters in Puerto Rico (which include the works of artists responding to the 2017 hurricanes), their incorporation into the site would both contribute to its appeal to readers and provide a rich archive that could be easily incorporated into the syllabus itself---not as mere points of visual interest but as a fundamental contribution to the usefulness of the site.

\#PuertoRicoSyllabus is an excellent developing collaborative project whose usefulness will continue to increase as we begin to understand the crucial ramifications of Puerto Rico's debt crisis and of the island's increased vulnerability to stronger and more destructive hurricanes as a result of climate change. As a collaborative, crowdsourced project, it has a great potential for growth. Its organizers should be commended for their understanding of the importance of this moment in history, with Puerto Rico at an economic and political crossroads, and for their recognition of the need for collaboration and multidisciplinarity as the perfect point of departure for teaching and scholarship.

\thinrule

\page
\subsection{Lizabeth Paravisini-Gebert}

Lizabeth Paravisini-Gebert is based in the Hispanic Studies Department at Vassar College, where she holds the Randolph Distinguished Professor Chair; she is also a member of the Programs in Environmental Studies, Latin American Studies, International Studies, and Women's Studies. Her most recent book, {\em Extinctions: Colonialism, Biodiversity, and the Narratives of the Caribbean}, is forthcoming this year from Liverpool University Press. She coauthors, with Ivette Romero-Cesareo, the blog \useURL[url1][https://repeatingislands.com/][][{\em Repeating Islands}]\from[url1], and coedits, with Michael Aronna, the \useURL[url2][http://pages.vassar.edu/oviedo/][][The Oviedo Project]\from[url2].

\stopchapter
\stoptext