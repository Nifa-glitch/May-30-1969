\setvariables[article][shortauthor={Johnson}, date={April 2019}, issue={3}, DOI={Upcoming}]

\setupinteraction[title={Xroads Praxis: Black Diasporic Technologies for Remaking the New World},author={Jessica Marie Johnson}, date={April 2019}, subtitle={Xroads Praxis}]
\environment env_journal


\starttext


\startchapter[title={Xroads Praxis: Black Diasporic Technologies for Remaking the New World}
, marking={Xroads Praxis}
, bookmark={Xroads Praxis: Black Diasporic Technologies for Remaking the New World}]


\startlines
{\bf
Jessica Marie Johnson
}
\stoplines


{\startnarrower\it This essay offers \quotation{xroads praxis} as a black diasporic technology for exploring what digital and analog landscapes hide and reveal. Invoking the crossroads, where deals are made and past meets present, a xroads praxis centers black humanity on the other side of the trick/tragedy---the Middle Passage, after the hurricane, deep in the slave ship. At the crossroads, the things we make (hashtags, tweets, Instagram posts, multimedia exhibits, archives, and blogs) may remain or they may disappear, may break or may be stolen/archived/corrupted, even without our permission. But they haunt, they scream, and they remain accountable to the living and lived. Moving through the black space on the map of Puerto Rico after Hurricane María to the haunting compositions created by jazz saxophonist Matana Roberts, this essay engages, in atmospheric, musical, and visual form, with the ways blackness challenges us to remake the New World. \stopnarrower}

\blank[2*line]
\blackrule[width=\textwidth,height=.01pt]
\blank[2*line]

\startblockquote
Memory \ldots{} dulls the lash/for the master, sharpens it for the slave.
\stopblockquote 

\startalignment[flushright]
\tfx{Natasha Tretheway, \quotation{Native Guard.}}
\stopalignment
\blank[2*line]


A dyad of satellite images of Puerto Rico at night were taken by the National Oceanic and Atmospheric Administration and sent out as a tweet. The top image was taken at 2:00 a.m. on 24 July 2017. The bottom image was taken at 2:00 a.m. on 25 September 2017. Although only months apart, the contrast between the two is stark. The second image was taken five days after Hurricane María made landfall in Puerto Rico.

\placefigure{Suomi Puerto Rico 24 September 2017}{\externalfigure[images/johnson-a.png]}
Pictured here, Puerto Rico is a dark mass, spots of light like stars plotting colonial and indigenous space and time. Ghostly white dots congeal around the cities of San Juan and Poncé, named after noble Spanish bloodlines and conquerors. To a lesser extent, the same pattern outlines the cities of Mayaguëz and Arecibo, whose names, in contrast, index ancient caciques and indigenous genealogies. Once upon a time a slaveholding colony, San Juan was filled with traders who met merchant ships from across the Americas, exchanging human cargo for plantation and farm products harvested elsewhere on the island. Poncé and Mayaguëz were lands of sugar and slavery, but even smaller municipalities like Arecibo were dotted by farms and bore witness to anticolonial resistance waged by peasants and small landowners as well as enslaved and free laborers of African and Native descent.\footnote{On slavery and other forms of labor in Puerto Rico, see Jorge L. Chinea, \quotation{A Quest for Freedom: The Immigration of Maritime Maroons into Puerto Rico, 1656--1800,} {\em Journal of Caribbean History; Kingston, Jamaica} 31, no. 1 (1997): 51--87; Luis A. Figueroa, {\em Sugar, Slavery, and Freedom in Nineteenth-Century Puerto Rico} (Chapel Hill: University of North Carolina Press, 2006); Jay Kinsbruner, {\em Not of Pure Blood: The Free People of Color and Racial Prejudice in Nineteenth-Century Puerto Rico} (Durham, NC: Duke University Press, 1996); Francisco A. Scarano, {\em Sugar and Slavery in Puerto Rico: The Municipality of Ponce, 1815--1849} (Madison: University of Wisconsin Press, 1984); Luis M. Díaz Soler, {\em Historia de la esclavitud negra en Puerto Rico} (1953; repr., San Juan: La Editorial, University of Puerto Rico, 1970); and David M. Stark, {\em Slave Families and the Hato Economy in Puerto Rico} (Gainesville: University Press of Florida, 2015). For the period immediately following emancipation through annexation by the United States, see Eileen J. Suárez Findlay, {\em Imposing Decency: The Politics of Sexuality and Race in Puerto Rico, 1870--1920} (Durham, NC: Duke University Press, 2000).} Before emancipation in 1873, enslaved women, children, and men of African descent would have filled those spaces and the other black, black spaces on the map. Some Africans would have absconded as maroons to those darkened mountains together in search of fugitive {\em palenques}. Others may have arrived in the interior, having been kidnapped from plantation districts and sold to cash-poor farmers who could not afford the prices in specie being demanded by international slave-trading merchants.

Human catastrophe leaves a mark, a black spot in time and space. On 20 September 2017, when Hurricane María made landfall on the island, the site of Christopher Columbus's second so-called discovery and the birthplace of the New World, it swept bridges off their supports, tore leaves off the trees in the rainforest, drove wildlife into migration or into hiding. The electrical grid, already a battered, colonial infrastructure, fell to pieces, as did the water and sewage systems and other public services. San Juan, the capital, {\em el santo}, the herald of Spanish empire, bowed and flailed in Guabancex's onslaught until, finally, darkness reigned.\footnote{Guabancex is the Taino goddess of storms: \quotation{Guabancex stirs the great blue pot of the Atlantic until the bitter saltwater breaks on all its shores.} Aurora Levins Morales, {\em Remedios: Stories of Earth and Iron from the History of Puertorriqueñas} (Boston: South End Press, 2001), 53.}

But blackness in the blackened spaces hit hardest by Hurricane María disappeared in the aftermath. In Carolina and Loíza, towns in the constellation near San Juan, that some one-fifth to three-quarters of the population (respectively) described itself as of African descent went without mention in the aftermath of the storm. In Poncé, which did not restore electricity to all its residents until August 2018---328 days after the storm---between 10 and 20 percent described themselves as black.\footnote{US Census Bureau, 2010 Census of Population and Housing, {\em Summary Population and Housing Characteristics}, CPH-1-53, Puerto Rico (Washington, DC: US Government Printing Office, 2012). See summaries of race and Hispanic or Latin origin here: \useURL[url1][https://factfinder.census.gov/bkmk/table/1.0/en/DEC/10_SF1/GCTP3.ST10/0400000US72]\from[url1].} Indigenous spaces suffered similar disappearances. Utuado, which for generations claimed indigenous patrimony on the island and is the home of Caguana Indigenous Ceremonial Park, barely rated a mention in mainstream news generated by the storm. Nor does Utuado, a region where even before the storm running water and electricity were privileges not rights, appear in the satellite image. \quotation{Memory,} poet Natasha Trethaway writes, \quotation{dulls the lash / for the master, sharpens it for the slave.}\footnote{Natasha Trethewey, \quotation{Native Guard,} in {\em Native Guard: Poems} (Boston: Mariner, 2007), 25.} And few cared to remember that the chronic dispossession faced by black and indigenous spaces on the island would leave them most vulnerable and most in need after the storm.\footnote{Powerful exceptions to this in the days immediately following the hurricane include Defend Puerto Rico (\useURL[url2][http://www.defendpr.com/][][http://www.defendpr.com]\from[url2]), Rosa Clemente's documentary unit PR on the Map (\useURL[url3][http://pronthemap.com]\from[url3]), the Palabras for Puerto Rico initiative launched from Michigan State University (\#ProyectoPalabrasPR), and Afro-diasporic descendants on the island and across the diaspora. See Yomaira Figueroa, \quotation{Borikén's Present Past or the Archive of Disappearances,} {\em Yomaira C. Figuero, PhD} (blog), 1 October 2017, \useURL[url4][https://web.archive.org/web/20190223191402/http:/www.yomairafigueroa.com/blog-1/2017/9/30/present-past-or-the-archive-of-disappearances][][http://www.yomairafigueroa.com/blog-1/2017/9/30/present-past-or-the-archive-of-disappearances]\from[url4].}

At the same time, somatic blackness and nonwhiteness saturated screens and devices as images of Puerto Ricans suffering under palm trees appeared on televisions and computers, spread by mainstream media. Blackness appeared in other ways as well, somewhere underneath the biopolitical and ocular. Tapping into a lexicon of Caribbean women, children, and men as both childlike and bestial, President Donald Trump derisively threw paper towels at relief workers on his visit to the island, treating them in the words of one observer, \quotation{like dogs.}\footnote{Benjamin Kentish, \quotation{Puerto Ricans Accuse Trump of Treating Them \quote{like Dogs} during Disaster Visit,} {\em Independent}, 5 October 2017, \useURL[url5][https://web.archive.org/web/20190223191455/https:/www.independent.co.uk/news/world/americas/trump-puerto-ricans-treat-like-dog-hurricane-maria-visit-accusation-a7985326.html][][https://www.independent.co.uk/news/world/americas/trump-puerto-ricans-treat-like-dog-hurricane-maria-visit-accusation-a7985326.html]\from[url5].} On the island itself, the federal and local police-military-surveillance apparatus that preoccupied itself with the war on drugs justified harassment of black islanders, shifted after the storm. Police wasted no time implementing and enforcing a curfew, including breaking up gathering spaces of dance and play---the kind that inevitably erupt when there is one working generator in a neighborhood and the night is long, hot, and dark. And in breaking news about the storm's impact, social media queries circulated desperately and especially among Puerto Ricans in the United States searching for incommunicado kin; blackness appeared in the loss and mourning of a diaspora for its people---for those lost beneath the sea.

Like Hurricane Katrina, Hurricane María was a natural disaster only to the extent that the Caribbean is no stranger to storms. The indigenous population of Puerto Rico had even named a deity after the hurricane: Guabancex, a femme supernatural entity of such power and importance that she was accompanied by two lesser deities as her heralds and commanders. However, ecological damage wrought by human interference has caused a climate catastrophe so disruptive that 2017 became the first hurricane season on record to witness the formation of three hurricanes at Category 3 or higher. Failure to invest in environmental resources or basic infrastructure exacerbated the impact of the storm upon landfall. And as a result of focused neglect at all levels of governance, including by the Federal Emergency Management Agency, the thirty-four people killed in the storm itself would be joined by over one thousand others as life-support systems failed, refrigerators used to store medical necessities like insulin and baby formula grew warm, and residents resorted to suicide to secure relief from trauma.\footnote{Alexia Fernández Campbell, \quotation{It Took 11 Months to Restore Power to Puerto Rico after Hurricane Maria; a Similar Crisis Could Happen Again,} {\em Vox}, 15 August 2018, \useURL[url6][https://web.archive.org/web/20190223191640/https:/www.vox.com/identities/2018/8/15/17692414/puerto-rico-power-electricity-restored-hurricane-maria][][https://www.vox.com/identities/2018/8/15/17692414/puerto-rico-power-electricity-restored-hurricane-maria]\from[url6].} Guabancex, when she \quotation{grows angry,} writes Fray Ramon Pané, \quotation{moves the wind and water and tears down the houses and uproots the trees.}\footnote{Fray Ramon Pané, An Account of the Antiquities of the Indians: A New Edition (Durham, NC: Duke University Press, 1999), 29.}

The black space on the map of Puerto Rico after the storm offers more than a referendum on an environmental event. It surfaces, in atmospheric and visual form, our failure to contend with slavery's past and empire's present. This failure erupted in full hurricane force across the Caribbean as the archipelago paid the price of generations of extractive violence. And yet blackness as density of life is still there. Illuminated or not, it occupies the dark swathes on the map; it continues to refuse and be refused illumination. It is the null value in all our humanities. At the crossroads somewhere between state-sanctioned death and dances on generators, there is a black diasporic technology that charts a path through the dark. Black diasporic theorists have been building this technology over a decade of fantastic scholarship. Black diasporic artists such as Matana Roberts have embodied the experience of this technology in ethereal music that screams with ghosts. Exploring what digital and analog landscapes hide and reveal, seeing them in their fullness after the storm of 1441/1492, finding black space on the map that does not conclude in black death, requires exploring with a xroads praxis.\footnote{Sylvia Wynter, \quotation{1492: A New World View,} in Vera Lawrence Hyatt and Rex Nettleford, eds., {\em Race, Discourse, and the Origin of the Americas: A New World View} (Washington, DC: Smithsonian Institution, 1995), 5--57.}

\subsection[title={Part 1. <!----Black----> Code},reference={part-1.-black-code}]

Confronting the Suomi image of black space in Puerto Rico brings to mind another image. In 2012, digital humanist Ben Schmidt created a data visualization of all the voyages in the US Maury collection. The Maury collection, eighty-eight microfilmed reels of historical ships' weather logs produced between 1784 and 1863, became part of the International Comprehensive Ocean-Atmosphere Data Set in 2002. Schmidt's image tracks all the voyages in black, plotted on a white background of vacated of land masses.

\placefigure{Schmidt ICOADS US Maury}{\externalfigure[images/johnson-b.png]}
Schmidt sought to visualize the wind currents that swept caravans of wood and flesh back and forth across the ocean. Struck by this image, scholar Marisa Parham troubled this reflection. \quotation{Air is energy,} she writes. \quotation{When away from the sea we seldom think about such things because we seldom connect breeze to wind to storm, until we must remember what happens when it fuels itself: gales, hurricanes, typhoons, tornadoes.}\footnote{Marisa Parham, \quotation{Black Haunts in the Anthropocene,} {\em Notebook} (blog), mp285.com, 26 January 2014, \useURL[url7][https://mp285.com/nb/black-anthropocene]\from[url7].}

\placefigure{Johnson NOAA 20sep17_maria_geocolor hurricane}{\externalfigure[images/johnson-c.png]}
In both the image of a darkened Puerto Rico and that of dark lines of ships crossing the ocean in the nineteenth century, the viewer faces a dilemma. On the one hand, the confrontation is with the blackened space made possible through the use of technology. On the other, the viewer contends with the use of white space that has been emptied of life, even as it was epistemologically called in to do the work of outlining the New World. Obliterated are the bound and blackened bodies that occupy, produce, and reproduce in the wake of colonial intervention. In the words of Christina Sharpe, on wake work, \quotation{In the wake, the past that is not past reappears, always, to rupture the present.}\footnote{Christina Sharpe, {\em In the Wake: On Blackness and Being} (Durham, NC: Duke University Press, 2016), 9.} Keeping the past and present in hand, invoking the wake means visualizing the wave pattern left behind in the wake of ships passing. It means visualizing many crossings and crossroads---ships bearing dark-skinned refugees forced to flee their homes in one direction, ships from the global North bringing aid to a global South made disastrous by globalization and neoliberal devastation in the other. Use of the wake also gestures to the vernacular of \quotation{wokeness,} a common twenty-first-century black diasporic invocation of political awareness of injustice. It means visualizing all of these as true all at once.

Because these bodies and processes do not appear, whether in the data sets or seen from space, it becomes difficult to place these images in relation, but in so many ways they are. Some twenty-seven thousand enslaved people traveled on ships riding those currents, arriving as early as 1520 and as late as 1859. Nearly a century ago, the Jones Act, in true mercantilist-colonialist form, limited shipping to the island of Puerto Rico to US ships.\footnote{See Slave Voyages, \useURL[url8][https://slavevoyages.org/][][https://slavevoyages.org]\from[url8], {\em Trans-Atlantic Slave Trade Database}, Itinerary, Place of Landing, Puerto Rico. The first documented slave ship to land in Puerto Rico was an unnamed vessel in 1520. The final documented slave ship was {\em La Majestad} in 1859. The Jones Act was signed in 1920.} Despite calls by elected officials across the political spectrum to suspend the act and allow aid to reach survivors of Hurricane María by any means necessary, conversations abounded on the act for days before the resident waived it. Even these facts, as residue of slave ship manifests and government decrees, do not evidence the undocumented ships and shipping, lives and communities, linked by wind and water to the island.\footnote{On these oft-remarked on but little documented communication networks, see Julius Scott, {\em A Common Wind: Afro-American Organization in the Revolution Against Slavery} (New York: Verso, 2018). For their extension even on land and for a later century, see Anne Eller, {\em We Dream Together: Dominican Independence, Haiti, and the Fight for Caribbean Freedom} (Durham, NC: Duke University Press, 2016).}

As a result, visibility offers little relief. On the other side of disappearing, for black and blackened subjects, lies the overindexed, overexposed voyeurism that accompanies images of death and destruction in places that look like Puerto Rico (or Haiti, Cuba, Brazil, and beyond). The West's tropic and trope-ic desire for blackened flesh fed European men for years through the writings of Moreau de St.~Mery, Edward Long, Agostino Brunias, Jean-Baptiste Labat, John Stedman, and Michel Adanson, to name a few. Moving at the speed of the digital, mainstream media, non-governmental organizations, police and military units, fiction and travel writers continue to render the islands, towns, and regions of the global South spaces of paradise and tourism, illicit desires and illegal release. In other words, visualizing blackness alone does its own damage to black humanity. It coughed up former president George W. H. Bush's lurid laughter in a September 2005 press conference on Hurricane Katrina, when he described New Orleans as a town where he used \quotation{to enjoy myself, occasionally too much.} It excreted from President Trump's statement that Haitian and African migrants to the United States were coming from \quotation{shithole countries.} These imperial discharges, tinged in white masculine voyeurism, claim to visualize blackness. But these men consume only black bodies, not black life.

In other words, illuminating digital lanterns with more and more data does not provide new answers. As scholar Simone Browne has explored, is a black person with a lantern a cyborg? In 1712, two dozen enslaved Africans armed themselves and gathered in the city of New York. Buildings went up in flames. The mathematics of black resistance during the period of slavery often added up to this---a handful of white lives lost, a morass of black life extinguished, and a series of punitive legal and surveillance measures instituted. New York's slave revolt proved no exception. Ten white residents lost their lives. In retaliation, mobs attacked enslaved and free black residents of the town. Police arrested some seventy, others were assaulted in the deluge. Twenty-three were killed. Of those arrested, twenty-five were sentenced to death. In March 1713, in the wake of the event, officials instituted the Lantern Laws: no people of African descent, free or slave, or those described as \quotation{Indians,} over the age of fourteen could be out at night without a light, a lantern, or torch of some kind, unless in the company of a white person. The punishment was forty lashes.\footnote{On lantern laws as surveillance, see Simone Browne, {\em Dark Matters: On the Surveillance of Blackness} (Durham, NC: Duke University Press, 2015). On the New York slave conspiracy, see Leslie M. Harris, {\em In the Shadow of Slavery: African Americans in New York City, 1626--1863} (Chicago: University of Chicago Press, 2003); and Craig Steven Wilder, {\em In The Company of Black Men: The African Influence on African American Culture in New York City} (New York: New York University Press, 2002).}

Laws like these encoded ocularity, visuality, illumination, and data collection as tools of power over black humanity for generations. The New York Lantern Laws demanded a mechanization of black being. A forced illumination. An exposure and an outing. In the presumed visibility cast by the lantern's glow, slaveowners received a false sense of safety and a skewed biometric of what people of African descent look like in the dark. Blackness indexed by empire loads deviance/death/devastation. Where the code ignores blackness, it allows white space on the map to do the work of analysis. But where the code reveals blackness, it does so at the risk of replicating systems of consumption and restraint.

In \quotation{Black Code Studies,} a special issue of {\em The Black Scholar}, Lauren Cramer and Alessandra Raengo write:

\startblockquote
To parse the issue of \quotation{black codes,} it may be worth distinguishing between the black codes understood as the antiblack assemblages that have already been written and continue to define encounters with blackness, and the careful iterative work of blackness writing its own commands within its existing constraints, which we will call \quotation{black coding.}\footnote{Lauren McLeod Cramer and Alessandra Raengo, \quotation{Freeing Black Codes: Liquid Blackness Plays the Jazz Ensemble,} {\em The Black Scholar} 47, no. 3 (2017): 8--21.}
\stopblockquote

Taking Cramer and Raengo's distinction seriously, Lantern Laws, the Jones Act, Immigration and Customs Enforcement, the prison warden, the plantation overseer, or the barracoon provide containers for code work, but the wake work of attending to black life precedes and proceeds in the thick of these.

Africans, by their very existence and through outright resistance and transgression, defied the commodification they had been put to. Those writing slave codes passed the laws for the purpose of constructing and maintaining slavery. Africans had their own idea of how their lives should matter and to whom they were accountable. Where black life remained in the shadow, where slaves gathered out of sight to plot their next act of life and resistance, black life transgressed slave codes and exceeded the register. Even in plain sight, black people who carried their own lanterns were not simply following the rules. They were masking, and the light was their mask.

\subsection[title={Part 2. \#BlackTheory},reference={part-2.-blacktheory}]

Excavating black life from the crossroads of death and forced illumination is an ethical project. Histories of slavery offer one corrective and a vantage point for a new kind of accountability. Studies of slavery have been suffused with the drive for data. Historians of slavery were some of the first to take up computational and digital forms in an attempt to organize knowledge about bondage, the plantation complex, and the structure of African-descended life in the New World. In 1998, David Eltis and David Richardson's {\em Trans-Atlantic Slave Trade Database}, which launched a database containing almost thirty thousand Atlantic slave trade voyages, was just one of several databases and digital projects that emerged at the beginning of this century. In 2000, Gwendolyn Midlo Hall, partnering with Paul Lachance and Virginia Meacham Gould, among others, launched the {\em Afro-Louisiana History and Genealogy} database. A more recent spate of interest in slavery and the digital humanities led to the creation of new database projects. In 2012, {\em Enslaved: People of the Historic Slave Trade} (formerly the {\em Slave Biographies Database}), a project led by Walter Hawthorne and Dean Rehberger, was launched by MATRIX: The Center for Digital Humanities and Social Sciences at Michigan State University. In 2015, Henry Lovejoy led a team in databasing records of \quotation{Liberated Africans} or Africans captured from slave ships by the British after 1808. In 2016, {\em Freedom on the Move}, a project compiling runaway slave ads and led by Edward Baptist and William C. Block, with team members Vanessa Holden, Mary Niall Mitchell, Joshua Rothman, and more, launched at Cornell University. In 2018, the {\em Trans-Atlantic Slave Voyages Database} project partnered with the {\em Inter-American Slave Trade Database} project, expanding the number of searchable slave trade voyages into the Americas. These projects apply historical methods and digital tools to the explication of black life under bondage. They each offer rich archives of data on black life under slavery and place numbers, graphs, tables, and, in some instances, written texts into a digital realm where users can access them without traveling to an archive or sitting in a university classroom.

Black studies scholars, creatives, and activists have also modeled explicit ways of attending to black humanity in the face of illuminating technology.\footnote{See Abdul Alkhalimat, \useURL[url9][https://web.archive.org/web/20190223192132/http:/www.eblackstudies.org/][][http://www.eblackstudies.org]\from[url9]; Alondra Nelson, \quotation{Afrofuturism: Archive,} 16 June 2011, \useURL[url10][https://web.archive.org/web/20180711072802/https:/alondranelson.wordpress.com/2011/06/16/afrofuturism-archive/][][https://alondranelson.wordpress.com/2011/06/16/afrofuturism-archive]\from[url10]; Moya Bailey and Ayana A. H. Jamieson, \quotation{Guest Editors' Introduction: Palimpsests in the Life and Work of Octavia E. Butler,} {\em Palimpsest: A Journal on Women, Gender, and the Black International} 6, no. 1 (2017): v--xiii; Radical Women of Color Bloggers Blogring, \useURL[url11][https://web.archive.org/web/20190223192331/https:/www.ringsurf.com/ring/idabwells/][][https://www.ringsurf.com/ring/idabwells]\from[url11]; Brittney C. Cooper, Susana M. Morris, and Robin M. Boylorn, eds., {\em The Crunk Feminist Collection} (New York: Feminist Press at CUNY, 2017); the Brown Girl Museums Blog, \useURL[url12][https://web.archive.org/web/20190223192411/http:/www.browngirlsmuseumblog.com/][][http://www.browngirlsmuseumblog.com]\from[url12]; Angel Nieves et al., \quotation{Black Spatial Humanities: Theories, Methods, and Praxis in Digital Humanities (A Follow-up NEH ODH Summer Institute Panel),} in {\em DH} (2017); Kim Gallon, \quotation{Making a Case for the Black Digital Humanities,} in Matthew K. Gold and Lauren Klein, eds., {\em Debates in the Digital Humanities 2016}(Minneapolis: University of Minnesota Press, 2016); Center for Solutions to Online Violence, \useURL[url13][https://web.archive.org/web/20190223192455/http:/femtechnet.org/csov/do-better/][][http://femtechnet.org/csov/do-better]\from[url13]; and \#transformDH, \useURL[url14][https://web.archive.org/web/20190223192540/http:/transformdh.org/][][http://transformdh.org]\from[url14].} In 2000, Abdul Alkhalimat launched eBlackStudies with the goal of bringing black studies, history, and issues of social justice, particularly around the Black Power movement, into the digital realm. In 2005, Howard Dodson and Sylviane Diouf completed work on a massive digital project and presence hosted at the Schomburg Center for Research in Black Culture titled {\em In Motion: The African American Migration Experience}. Scholars such as Alondra Nelson, Moya Bailey, Aleia Brown, and Joshua Crutchfield created digital communities that centered blackness and included creatives and thought producers invested in radical black futures and social justice, including attending to trans* and queer of color livelihoods. Others, such as Kim Gallon, Angel David Nieves, Nishani Frazier, and Christy Hyman, have created productive communities of digital humanities scholars invested in black studies within the academy but attuned to the needs of those beyond the Ivory Tower walls. These projects model the work black digital practice makes possible. Black digital practice exceeds the academy, defying boundaries between institution and community, form and function.

Excavating black life from the null value requires digital humanists to understand the wide range of labor and creativity developed under the heading \quotation{the digital.} This work can be divided into at least four forms:

\startitemize[packed]
\item
  Digital Humanities: digital humanities within academia
\item
  digital humanities: humanities work rendered digitally
\item
  Digital Media: content created online
\item
  digital practice: 💋💩😎🕷💦
\stopitemize

The fourth form, digital practice, embraces the locus of hybrid, playful, fluid, and meta possibilities that the digital creates and problematizes. When hybrid practitioners engage in a {\em black} digital practice, they expose the \quotation{queer, femme, fugitive, and radical} possibilities of exceeding the field, breaking the CSV, and excavating life from null values.\footnote{Jessica Marie Johnson and Mark Anthony Neal, \quotation{Introduction: Wild Seed in the Machine,} {\em The Black Scholar} 47, no. 3 (2017): 1. For more on black digital practice, see Jessica Marie Johnson, \quotation{Markup Bodies: Black {[}Life{]} Studies and Slavery {[}Death{]} Studies at the Digital Crossroads,} {\em Social Text}, no. 137 (December 2018): 57--79; and Jessica Marie Johnson, \quotation{4DH + 1 Black Code / Black Femme Forms of Knowledge and Practice,} {\em American Quarterly} 70, no. 3 (2018): 665--70.} Black digital practice builds on methods, research ethics from theoretical frameworks, research methodologies, and narrative structures (in short, \#BlackTheory) developed in Black studies:

\startitemize[packed]
\item
  blues epistemology \letterbar{} Woods
\item
  wake work \letterbar{} Sharpe
\item
  dark sousveillance \letterbar{} Browne
\item
  disruption poetics \letterbar{} Morgan
\item
  along the bias grain \letterbar{} Fuentes
\item
  hauntings \letterbar{} Parham
\item
  digital ethics \letterbar{} Bailey
\item
  Man1 and Man2 \letterbar{} Wynter
\item
  critical fabulation \letterbar{} Hartman
\item
  undercommons \letterbar{} Harney and Moten
\item
  \ldots{}and more
\stopitemize

What makes this rich and deep intellectual work so powerful is the way it challenges the container of West and Western thought. \#BlackTheory holds space for black life to be seen without the rancor of the voyeur, appreciated without the demand for consumption. \#BlackTheory remakes the archive, the human, the land itself. The intellectual projects outlined above offer only a handful of the reading practices and strategies necessary for intervening in the replication of black death in our texts, on device screens, and in our everyday engagement with academic work (\quotation{Digital Humanities}). They demand a practice of either writing a \quotation{history of the present,} as Marisa Fuentes and Saidiya Hartman have espoused; reorganizing the past, as Jennifer Morgan and Clyde Woods challenge; or creating community and art that transgresses, as Moya Bailey, Stefano Harney, and Fred Moten have done.\footnote{See Marisa J. Fuentes, Dispossessed Lives: Enslaved Women, Violence, and the Archive (Philadelphia: University of Pennsylvania Press, 2016); Saidiya V. Hartman, Scenes of Subjection: Terror, Slavery, and Self-Making in Nineteenth-Century America (New York: Oxford University Press, 1997); Jennifer L. Morgan, \quotation{Accounting for \quote{the Most Excruciating Torment}: Gender, Slavery, and Trans-Atlantic Passages,} History of the Present 6 (2016): 184--207; Clyde Adrian Woods, Development Arrested: The Blues and Plantation Power in the Mississippi Delta (New York: Verso, 1998); Moya Bailey, \quotation{\#transform(ing)DH Writing and Research: An Autoethnography of Digital Humanities and Feminist Ethics,} Digital Humanities Quarterly 9, no. 2 (2015); and Stephen Matthias Harney and Fred Moten, The Undercommons: Fugitive Planning and Black Study (New York: Autonomedia, 2013). See also Sharpe, {\em In the Wake}; Browne, {\em Dark Matters}; and Wynter, \quotation{1492: A New World View.}}

Xroads praxis builds from and with this work. The digital humanities exists in the wake of, and indexes an astronomical devaluation of, life. Historians of slavery have a charge to produce truths about the course of that devaluation. \#BlackTheory provides a language for translating those truths into tools that remake the New World. A xroads praxis does not demand that these goals intersect so much as that they challenge users, creators, researchers, and teachers to move in multidimensional time and space between our disciplines, communities, and institutions. A xroads praxis requires us to learn from the black feminist metaphysicians who, as Alexis Pauline Gumbs has related, \quotation{knew the body would never be enough.}\footnote{Alexis Pauline Gumbs, M Archive: After the End of the World (Durham, NC: Duke University Press, 2018), 6.}

Xroads praxis evokes African diasporic intellectual and cultural production, where the crossroads plays a central role in systems of belief, cosmology, history, and social life. The crossroads is where, according to diasporic Yoruba practices and beliefs, Elegua or Papa Legba grants the humble petitioner-practitioner access to spirit world. The crossroads is linked, at times, to the Bakongo cosmogram, brought with enslaved Africans to the Americas and representing four temporal elements (dawn, noon, sunset, and midnight).\footnote{See Robert Farris Thompson, {\em Flash of the Spirit: African and Afro-American Art and Philosophy} (New York: Vintage, 2010). See also Nettrice R. Gaskins, \quotation{The African Cosmogram Matrix in Contemporary Art and Culture,} {\em Black Theology} 14, no. 1 (2016): 28--42.} It is the symbol found at enslaved burial sites, carved into headstones and stones, or created by enslaved and free blacks who left artifacts marking the points of the cross at the four corners of their homes, as occurred in households from Annapolis, Maryland, to the Bahamas. Throughout the eighteenth century and into the nineteenth, before the institutionalization of prisons, the crossroads was where those convicted of crimes were beaten, flogged, or executed. After execution, the bodies of enslaved Africans; indigenous women, children, and men; and poor whites were often left at crossroads across the Caribbean and the South, displayed as lurid warning against future infraction. At the crossroads, the devil met heroes and fools, those in the wrong and those wronged. It was where Robert Johnson, famous blues musician, made his deal with the devil, giving up his soul for supernatural musical ability.

The crossroads continues to be of contemporary cultural relevance to black people around the world. In hip-hop artist Common's video for \quotation{Black America, Again,} a cosmogram was painted on the ground in front of Gilmor Homes, Freddie Gray's neighborhood. It was sculpted into the floor of the Schomburg Center for Research in Black Culture by artist Houston Conwill. It was where black women wait for justice in Kimberlé Crenshaw's theory of intersectionality. The crossroads is, in short, where worlds come together and break apart, good and evil meet to make deals, utter desperation meets otherworldly possibility, and spirit meets flesh, where a choice must be made but perhaps not just yet. It is the quiet before the beat drops.

By describing the collaboration between histories of slavery, the digital humanities, and \#BlackTheory as requiring a xroads praxis, I am demanding that scholars create data without losing affect, sensation, and kinship as a framing for black life. Xroads praxis, for instance, requires heated participation rooted in queerness, corporeality, and desire. A xroads praxis self-consciously deals in dead things. Slave manifests. Eighteenth-century testaments. Woodcut headers on ads for recalcitrant runaways. Myspace posts. Livejournal graphics. 140 characters that disappear after ten days. In a xroads praxis, black life is no longer invisible, although it may retain a redacted/unknown quantity (X). It does not shy away from ephemerality. Instead of finding limits in the absence, xroads praxis uses silence as Maroon code, allows absconded figures, bodies in freedom, to offer a practicum on what, how, and who our digital selves could be if we dared to embrace the sensorium of hauntings left behind.

\subsection[title={Part 3. {\em screams}},reference={part-3.-screams}]

\placefigure[here]{\goto{Light a candle. Drink a glass of water. Press Play. Listen for two minutes.}[url(https://w.soundcloud.com/player/?url=https\letterpercent{}3A//api.soundcloud.com/tracks/12068585&auto_play=false&show_user=true&visual=true)]}{\externalfigure[issue03/johnson-audio-a.png]}


The second track of Matana Roberts's {\em Coin Coin Chapter One: Gens de couleur libres} is titled \quotation{Pov Piti.} {\em Pov Piti} means \quotation{poor little one} in Creole/Kreyol.\footnote{Matana Roberts, \quotation{pov piti,} {\em Coin Coin Chapter One: Gens de couleur libres}, CD (Constellation Records, 2011).} \quotation{Pov Piti} is the title of a Creole folksong found along the Gulf Coast, most often referring to a woman who has lost the love of her life. Sometimes this woman is Lolotte; more often she is Mamzelle Zizi. One of the most cited versions of Mamzelle Zizi may be found in Louisiana Creole in Lyle Saxon's 1945 {\em Gumbo Ya-Ya: Folk Tales of Louisiana}. Saxon, chronicler of Louisiana and New Orleans folklore and director of the Louisiana branch of the Federal Writers' Project from 1935 to 1943, printed his version as Mamzelle Zizi mourning her lover. Jealous of her rival \quotation{who wears pretty clothes, which at that time consisted of a brilliant madras tignon, imported from the Indies, a gaily embroidered petticoat and earrings,}\footnote{Lyle Saxon, Edward Dreyer, Robert Tallant, and the Louisiana Writers' Project, {\em Gumbo YA-YA: Folk Tales of Louisiana} (Boston: Houghton Mifflin, 1945), 436.} Mamzelle Zizi is inconsolable:

\startblockquote
Pov' piti Mamzelle Zizi!\crlf
Li gagnin bobo dans coeur!\crlf
Pov' piti Mamzelle,\crlf
Li gaignin tristesse dans coeur!

Calalou porté madras,\crlf
Li gagnin jupon brodé,\crlf
Li gagnin des belles allures,\crlf
Boucle d'oreilles en or tout pure.

{[}{\em Saxon's Translation}{]}\crlf
(Poor lil' Mamzelle Zizi!\crlf
She has a pain in her heart!\crlf
Poor lil' Mamzelle Zizi,\crlf
She has sadness in her heart!)

(Calalou wears madras,\crlf
She has an embroidered petticoat,\crlf
She has fine manners,\crlf
Earrings made of pure gold.)\footnote{Ibid., 436--37 (Saxon's translation).}
\stopblockquote

In 1956, Adelaide van Wey sang an updated version, released by Smithsonian Folkways under the title \quotation{Pauv' Piti' Mom'zelle Zizi.} A folk ballad meant to be sung on repeat, Van Wey's intepretation had three parts. Throughout the first two, Van Wey crooned throatily over a piano and zither in a dark lament. Entering the third part, a half step, a motif, quickened the pace, and the song shifted into dance tempo, Van Wey singing lightly over Mamzelle Zizi's pain.\footnote{Adelaide van Wey, \quotation{Pauv' Piti' Mom'zelle Zizi,} {\em Street Cries and Creole Songs of New Orleans}, LP (Folkways Records RF 203 {[}1964{]}, 1956). Many thanks to musicologist Guthrie Ramsey for helping me put this recording in context; any mistakes are my own.} Pov piti has inspired other, even more upbeat versions, from New Orleans jazz icon Dr.~John's 2000 rendition of Mamzelle Zizi to the 2001 Orchestre Bourbon's rendition (led by Georges Fourcade) of Mamzelle Zizi and recorded on the album {\em Le barde créole}.\footnote{Dr.~John and the Donald Harrison Band, \quotation{Mamzelle Zizi,} {\em Funky New Orleans}, CD (Union Square Music, 2000); Georges Fourcade, \quotation{Mamzelle Zizi,} {\em Le barde créole}, CD (Takamba, 2001).}

The pov piti invoked by Matana Roberts is not a Creole dance tune. Matana begins her version of this poor little girl's story with silence, broken gently by melodic piano chords. Behind the piano are choked, coughing sounds, barely comprehensible, almost words. A wave of synth-electronic sound with percussive accents begins to flood in. As it does, the choking becomes a strangled stutter, the stutter transforms into guttural cries, the cries become hoarse screams. The expression is one of either extreme pain or extreme pleasure---or both, at the point where the two coincide, a sonic expression of sensation at its most extreme. Amber Musser, writing on race and masochism, theorizes sensation as an analytic, as \quotation{both individual and impersonal; it occupies a sphere of multiplicity without being tethered to identity.}\footnote{Amber Jamilla Musser, {\em Sensational Flesh: Race, Power, and Masochism} (New York: New York University Press, 2014), 2.} Sensation, then, might be likened to raw data---except sensation requires corporeality.

The body attached to these cries haunts their sound. In these first twenty seconds of the song, it is impossible to ascribe a gender to the body screaming. The cries are genderless and ungendered. Still, these first two minutes of pov piti are marked by excess, the kind of excess that invokes a necessary corporeality, bridging data (digital sound) and sensation, suggestive but not illustrative (yet) of identity, intention, or power. An excess of sound, an excess of desire, though it is unclear whether that desire is to communicate, to cry for help, to continue the torment or pleasure, or to simply be heard. The cries range from choked gasps to garbled screams to inarticulate howls, loud, hoarse, and coarse. The strain on the vocal chords, the throat, the voice, the breath can be heard in the rasp, sigh, screech, and scrape of air expelled from the body. The effort required to produce the screams is tangible, even as words, in fact, {\em language itself}, remains incomprehensible, lost to sensation. By the end of the second minute, it seems every possible {\em human} iteration sounding out excess has been achieved. The screams crescendo before descending into strangled rattles of breath and half-formed croaks. As they do, a jazz saxophone begins to play. A woman begins to speak, her words more clear. The woman is Matana Roberts channeling her ancestress, Marie Thérèze \quotation{Coincoin} Metoyer.

In short, decisive sentences, Coin Coin, an enslaved woman of African descent, offers testimony. \quotation{I am,} Coin Coin, through Roberts, declares over and over in sing-song. Coin Coin reflects. On her birth. On her Catholic baptism. On her owner. On the death of her mother from yellow fever. On being a slave, a \quotation{nigra,} and therefore a \quotation{tool of the earth.} She repeats her name: \quotation{I am Coin Coin Coin Coin Coin Coin Coinnnn Coin,} pronouncing it \quotation{Kwon Kwe.} She muses in and on memory. Her narrative devolves into a refrain: \quotation{I was only 16. I was only 16. There will never be any pictures of me.} As Coin Coin/Roberts speaks, in the spaces left behind by, between, and synchronized with her words, a second/third voice can be heard. Faint and femme, this ghost echoes everything spoken, the same historical narrative, but in French.

Matana Roberts is a sound experimentalist, jazz saxophonist, composer and improviser. Born on Chicago's South Side, Roberts currently resides in New York City. {\em Coin Coin: Gens de Couleurs Libres} is the first chapter of her second project. The work is named after the formerly enslaved woman of African descent named Marie Thérèze Coincoin Metoyer, the founder of the Cane River Creoles community of Natchitoches, Louisiana. Coin Coin was born a slave in 1742. Her owner, Louis Juchereau de St.~Denis, was the commandant of Natchitoches when the rural outpost, some four miles north of New Orleans, was still under French control. In 1767, a yellow fever epidemic decimated the outpost, killing her mother and her owner. With St.~Denis's passing, ownership of Coin Coin passed to his youngest daughter, who leased her to Claude Thomas Pierre Metoyer. Coin Coin bore Metoyer several children, some of whom Metoyer then freed. In 1778, Metoyer finally freed Coin Coin herself. By this time, Louisiana had passed from French ownership to Spanish. As a free woman of color, Coin Coin took advantage of Spanish laws on land settlement to \quotation{improve} and eventually claim a plot of land in the area of Natchitoches, which she farmed with her children and slaves of her own. By 1810, Coin Coin had seven living sons with land, cattle, and fifty-eight slaves among them. The only family in the area to own more slaves than Coin Coin and her children were the white Metoyers and their extended families.\footnote{See H. Sophie Burton, \quotation{Free People of Color in Spanish Colonial Natchitoches: Manumission and Dependency on the Louisiana-Texas Frontier, 1766--1803,} {\em Louisiana History} 45, no. 2 (2004): 173--97; H. Sophie Burton and F. Todd Smith, {\em Colonial Natchitoches: A Creole Community on the Louisiana-Texas Frontier} (College Station: Texas A&M University Press, 2008); Elizabeth Shown Mils, \quotation{Marie Thérèse Coincoin: Cane River Slave, Slave Owner, and Paradox,} in Janet Allured and Judith F. Gentry, eds., {\em Louisiana Women: Their Lives and Times} (Athens: University of Georgia Press, 2009); and Gary B. Mills and Elizabeth Shown Mills, {\em The Forgotten People: Cane River's Creoles of Color} (1997; repr., Baton Rouge: Louisiana State University Press, 2013).}

Roberts, through family lore and history, learned she was descended from one of these children. She speaks in interviews about growing up steeped in references to her ancestress---including a grandfather who used to call Roberts \quotation{Coin Coin} to tease. Constellation Records, Roberts's label, describes {\em Coin Coin} as \quotation{a multi-chapter work that combines conceptual scoring (graphic notation, \quote{chance} strategies), storytelling and historical narrative, performative theatre (personae, costume, multi-media), and a deeply considered channeling of personal ancestry and the \quote{universal} experience of Africans in America.} Peter Margasak, writing for National Public Radio, framed Coin Coin even further: \quotation{She {[}Roberts{]} realized putting together fragments and snippets that wouldn't serve much purpose on their own, could result in something new and meaningful.}\footnote{\quotation{Matana Roberts,} Constellation Records, \useURL[url15][https://web.archive.org/web/20190223192654/http:/cstrecords.com/matana-roberts/][][http://cstrecords.com/matana-roberts]\from[url15]; Peter Margasak, \quotation{Matana Roberts Drops the Stunning Second Chapter of Her Coin Coin Project,} {\em Chicago Reader}, 4 October 2013, \useURL[url16][https://web.archive.org/web/20190223192826/https:/www.chicagoreader.com/Bleader/archives/2013/10/04/matana-roberts-drops-the-stunning-second-chapter-of-her-coin-coin-project][][https://www.chicagoreader.com/Bleader/archives/2013/10/04/matana-roberts-drops-the-stunning-second-chapter-of-her-coin-coin-project]\from[url16]. Graphic notation is representation of music through the use of visual symbols outside the realm of traditional music notation.}

Roberts calls her technique \quotation{panoramic sound quilting,} but her act of quilting moves well beyond the audiovisual experience to encompass and involve her past, herself, her band, live audiences, the jazz community, a digital community of lurkers, followers, rebloggers, and other sharers and casual listener-viewers.\footnote{\quotation{Matana Roberts,} Constellation Records.} When the first chapter, {\em Gens de couleur libres}, was released in 2011, Roberts used archival material, personal recollections, interviews with family members, and digital and social media in a very dynamic fashion. She created the blog {\em In the Midst of Memory} and posted clips from the album. When Roberts eventually uploaded the full album on SoundCloud, she embedded links to it on her blog, alongside historical information about Marie Thérèze Coincoin Metoyer, New Orleans {\em gens de couleur libres}, slavery, Louisiana, St.~Louis, Chicago, and more. The liner notes for the physical album included graphic notation Roberts created while making the album; that blog no longer exists. The album remains available on SoundCloud for free.

\placefigure{Matana Roberts Notation}{\externalfigure[images/johnson-d.jpeg]}
In the process of composing {\em Mississippi Moonchile} and {\em River Run Thee}, {\em Coin Coin} chapters 2 and 3, Roberts began using her Instagram and Tumblr accounts (@steelkiltrose) to document and share her cross-country journey around and throughout the United States.\footnote{Matana Roberts, {\em Coin Coin Chapter Two: Mississippi Moonchile}, CD (Constellation Records, 2013); Matana Roberts, {\em Coin Coin Chapter Three: River Run Thee}, CD (Constellation Records, 2015).} While traveling and researching material for these and subsequent {\em Coin Coin} chapters, Roberts shared her experience in the form of digital photography and commentary, readily engaging with followers who joined her virtually on her travels. {\em Mississippi Moonchile} was released in 2013, {\em River Runs Thee} in 2015. The Instagram and Tumblr accounts also no longer exist. A short film, completed in 2010 in Montreal, accompanied the release of {\em Mississippi Moonchile}. Directed by Radwan Moumneh, it remains available on Vimeo and YouTube. In the grayscale audiovisual romp through sound, text, and spectral movement, Roberts, channeling Coin Coin, stands on empty streets in a white dress. Saxophone in hand, she plays a song that at times the viewer cannot hear over the screams of her archive. In {\em River Run Thee}, Roberts recites from slave ship manifests, British Naval ship registers, and travel narratives. In her accounting, the words of men like Captain G. L. Sullivan, whose crew intercepted and seized slave ships after the formal abolition of the trade in 1808, reverberate against a chanting chorus.\footnote{George Lydiard Sulivan, {\em Dhow Chasing in Zanzibar Waters and on the Eastern Coast of Africa: Narrative of Five Years' Experiences in the Suppression of the Slave Trade} (London: S. Low, Marston, Low, and Searle, 1873).}

Hauntings and crossings---the sound of them, the emotion in them, and the sensation of them---dominate Matana's project. In a 2015 interview, Roberts described her process:

\startblockquote
I have a really big interest in the spirit world: spooks and the things we can't necessarily see but feel. An exploration of ghosts and things of that nature. There was a period of my childhood where I tried to contact people on that plane and I stopped doing that as a teenager because I heard it can induce states of psychosis if you don't have a proper guide. So I left that and I realised that music is my medium, my guide.\footnote{Matana Roberts, quoted in Stewart Smith, \quotation{Traces of People: An Interview with Matana Roberts,} 6 October 2015, \useURL[url17][http://thequietus.com/articles/18937-matana-roberts-interview]\from[url17].}
\stopblockquote

The song \quotation{Pov Piti,} the {\em Coin Coin} project, and Matana Roberts herself form a singularity---an epicenter or node---of black digital practice and its relationship with slavery's archive. Roberts and her ghosts pose critical questions and possibilities for what can and cannot be done with the data found in slavery's archive, with digital and social media, and historical narratives of bondage---particularly narratives of intimacy and sex, violence, and property. Roberts feared hearing voices would induce states of psychosis, then reframed her haunting against encounters with her own ancestry as a descendant of Coin Coin. She created sonic histories where poor little enslaved girls scream, materialize on city streets, and spill themselves across blogs, grams, and tweets. Unafraid of emotion, Roberts rooted the unspeakable affect of the trade in her own mourning, even sounding out the cries heard in pov piti herself, in the wake of her mother's death.\footnote{In an interview with Molly Sheridan of New Music Box in 2013, Roberts explained the context of the screams on the first volume: \quotation{My mother had passed away, maybe ten days before that was recorded. So those screams were therapeutic in a different kind of way. But there's a welcoming to them too---you know? We're here, I'm alive, let's celebrate what we do have.} Matana Roberts, quoted in Molly Sheridan, \quotation{Matana Roberts: Creative Defiance; Interview by Molly Sheridan,} {\em New Music Box}, 1 February 2013, \useURL[url18][https://web.archive.org/web/20190223193226/https:/nmbx.newmusicusa.org/matana-roberts-creative-defiance/][][https://nmbx.newmusicusa.org/matana-roberts-creative-defiance]\from[url18].} She played with quantity, layering femme whispers upon femme whispers, one colonial language over another. Coin Coin is (\quotation{I am}) regardless of whether there is visual proof (\quotation{There will never be any pictures of me}), and in Roberts's creation, she time-travels from the past into a new present to live again.

\placefigure[here]{\goto{Matana Roberts, \quotation{Mississippi Moonchile,} {\em Coin Coin}. Directed by Radwan Moumneh.}[url(https://player.vimeo.com/video/23003301?autoplay=1&loop=1&color=ffffff)]}{\externalfigure[issue03/johnson-vid-a.png]}


Fluid and fugitive, Matana Roberts and Coin Coin do not occupy a single digital project space or site. Coin Coin, through Roberts, does what she could not in life---she absconds. She becomes Maroon code. The {\em Coin Coin} project escapes from written documents, the analog, and the archive. It transforms the use of digital tools as they become available and so transforms the listener-user's relationship to sound, subjects, and subjectivity, the living and the dead. At points, over and over, Coin Coin disappears from sight and reappears on sites in whispers and phantasms. It is possible to find, for instance, images Roberts previously posted on Instagram, where they have been archived by different jazz websites. This mediated existence echoes the archive enslaved men and women created with their testimony, in slave narratives dictated to but edited by white abolitionists. Marisa Parham, theorizing hauntings in African American literature, described them as more than echoes of past people, lives, or experiences. Hauntings, particularly of past trauma, both return from beyond and constitute lifeways in the present. The {\em Coin Coin} project, with its missing and mediated, ephemeral and historically tangible, gives hauntings a power to use and be used in the service of a historical narrative that does not reproduce humans as solely nodes of data.

Matana Roberts's intellectual and creative works illustrate the kind of interventions black digital practice makes in conversations around digital humanities, justice, and history. Using digital tools to excavate slavery's archive generates histories of intimate violence, torment, and dispossession. A digital praxis attuned to black life and not to slavery's penchant for black death cannot ignore loss, heartbreak, absence, and silence. Black digital practice makes space for this praxis, for black diasporic longing, insurgence, hauntings, and community mourning to inform how digital tools might be used to uncover acts of resistance, insurgent politics, and evidence of self-fashioning. Black digital practice incorporates analog pasts; transgresses boundaries between institutions, communities, and individuals; and is unafraid of the ephemeral or incremental in comparison. In the black diasporic subterranean that is slavery's archive, black digital practice has the potential to call into being the fluidity, dispersal, and mobility that has always existed alongside structural, personal, and intimate embodiments of violence. It offers an opportunity to look both into and at the data without losing emotion, sensation, and redress as a framing for black life.

\subsection[title={Part 4. We Keep the Light},reference={part-4.-we-keep-the-light}]

Dark space and illumination represent the twin poles of commodification slavery necessitated. One occasioned the disappeared and dispossessed, the genocidal stakes at the heart of making a New World, the other a rapacious and ravenous consumption only colonialism could beget. A xroads praxis exceeds this binary. It recognizes that the absence is there and then attends to it.

\placefigure{Parham's Inverted Maury Map}{\externalfigure[images/johnson-e.png]}
To return to the spiderweb of Maury voyages, in \quotation{Black Haunts in the Anthropocene} Parham writes:

\startblockquote
I find the images haunting, but the dark lines of the triangle trade are not sooty from centuries of moving Africans into new world slavery, nor from moving the fruits of their labor to fuel others' growth. They are dark because those sea roots were there before the slave trade. This fact gives me a feeling of a timeless before and after, which is difficult to reconcile with any sense of enduring historical impact. Even despite their immense violence, the men pushing and pulling bodies across the Atlantic were only men, hoping for a good wind. A bug on a leaf, torn from the tree.\footnote{Parham, \quotation{Black Haunts in the Anthropocene.}}
\stopblockquote

Striving, desiring, lusting for a xroads practice affirms that there is black humanity somewhere on the other side of the Middle Passage, after the hurricane, deep in the prison cell. At the crossroads, the things we make (hashtags, tweets, Instagram posts, multimedia exhibits, archives, and blogs) may remain or they may disappear, may break or may be stolen/archived/corrupted, even without our permission. But they haunt, they scream, and they remain accountable to the living and lived. In a xroads praxis, the destination is less important than our survival, our machines are less important than the humans inside them, and we never forget that even in the dark we keep the light.

\placefigure[here]{\goto{Lighter Thieves. \quotation{We Keep the Light.}}[url(https://www.youtube.com/embed/mO-PAYLa0Y0)]}{\externalfigure[issue03/johnson-vid-b.png]}


\subsection[title={Acknowledgments},reference={acknowledgments}]

An earlier version of this essay was first presented as a keynote address at the \quotation{Race, Memory, and the Digital Humanities Symposium} at the College of William and Mary, Williamsburg, Virginia, 26--28 October 2017. Thank you to Liz Losh for hosting that historic event. Thank you to Alex Gil, Kaiama Glover, and anonymous reviewers for their constructive critique of this essay; any mistakes are mine and mine alone. This work was incubated with and influenced by Clyde Woods; Katherine McKittrick; Rae Paris; Bettina Judd; \#Scenesat20, a symposium in honor of Saidiya Hartman's {\em Scenes of Subjection} (particularly Fred Moten's \quotation{Blackness Is X}); Yomaira Figueroa; Bianca Laureano of the LatiNegrxs Project; Mark Anthony Neal; students in the Black Code studies course I taught at Johns Hopkins University (Spring 2017); students in the \#femdh course I taught (with Liz Losh) at the 2017 Digital Humanities Summer Institute, University of Victoria, British Columbia; contributors on the \#BlackTheory hashtag; Matana Roberts; Marie Thérèse Coincoin Metoyer; Farris Armand of the Lighter Thieves; New Orleans, Louisiana; Utuado, Puerto Rico; and my grandmother, Mary Nuñez.

\thinrule

~

\page
\subsection{Jessica Marie Johnson}

Jessica Marie Johnson is a writer and historian of slavery at Johns Hopkins University. She is~the~author of~{\em Practicing Freedom:~Black Women, Intimacy, and Kinship in New Orleans Atlantic World}~(University of Pennsylvania Press, forthcoming) and~a co-editor, with Mark Anthony Neal, of~\quotation{Black Code,} a special issue of {\em The Black Scholar~}(2017). Her work has appeared in {\em Slavery and Abolition}; {\em The Black Scholar}; {\em Meridians: Feminism, Race, and Transnationalism}; {\em American Quarterly}; {\em Social Text}; the {\em Journal of African American History}; the {\em William and Mary Quarterly}; {\em Debates in the Digital Humanities}; {\em Forum Journal}; {\em Bitch Magazine}; {\em Black Perspectives} (AAIHS); {\em Somatosphere}; and {\em Post-Colonial Digital Humanities} ({\em DHPoco}). She tweets as @jmjafrx.

\stopchapter
\stoptext