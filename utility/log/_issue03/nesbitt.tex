\setvariables[article][shortauthor={Nesbitt}, date={July 9 2019}, issue={3}, DOI={10.7916/archipelagos-mq09-v798}]

\setupinteraction[title={The Slave-Machine: Slavery, Capitalism, and the “Proletariat” in *The Black Jacobins* and *Capital*},author={Nick Nesbitt}, date={July 9 2019}, subtitle={The Slave-Machine}]
\environment env_journal


\starttext


\startchapter[title={The Slave-Machine: Slavery, Capitalism, and the \quotation{Proletariat} in {\em The Black Jacobins} and {\em Capital}}
, marking={The Slave-Machine}
, bookmark={The Slave-Machine: Slavery, Capitalism, and the “Proletariat” in *The Black Jacobins* and *Capital*}]


\startlines
{\bf
Nick Nesbitt
}
\stoplines


{\startnarrower\it This essay argues that C. L. R. James's Marxist humanism is inherently inadequate for describing the distinction and transition between slavery and capitalism. To do so, the essay interrogates James's famous claim in {\em The} {\em Black Jacobins} (1938) that the slaves of St.~Domingue were \quotation{closer to a modern proletariat than any group of workers in existence at the time,} by comparing James's understanding of the concept of {\em proletariat}---there and in {\em World Revolution} (1937)---with Marx's various developments of the concept across the three volumes of {\em Capital}. This analysis distinguishes James's political and historicist deployment of the term from Marx's analytical usage of the notion in his categorial critique of capitalism. In contrast with James's linear, Marxist-humanist understanding of the passage from slavery to capitalism, Marx himself demarcates a well-defined delineation between these two basic categories, understood in {\em Capital} as analytically (as opposed to historically) distinct modes of production. The essay thus concludes by analyzing Marx's conceptual differentiation of slavery and industrial capitalism in {\em Capital}, drawing on Etienne Balibar's analysis of the concepts of mode of production and transition in {\em Reading Capital} (1965). \stopnarrower}

\blank[2*line]
\blackrule[width=\textwidth,height=.01pt]
\blank[2*line]

\startblockquote
The slaves worked on the land, and, like revolutionary peasants everywhere, they aimed at the extermination of their oppressors. But working and living together in gangs of hundreds on the huge sugar-factories which covered the North Plain, they were closer to a modern proletariat than any group of workers in existence at the time, and the rising was, therefore, a thoroughly prepared and organized mass movement.
\stopblockquote 

\startalignment[flushright]
\tfx{C. L. R. James, The Black Jacobins.}
\stopalignment
\blank[2*line]


In this famous passage that opens the fourth chapter of {\em The Black Jacobins: Toussaint L'Ouverture and the San Domingo Revolution}, to initiate its narration of the Haitian Revolution proper, C. L. R. James, almost in passing, without explicitly developing its theoretical implications but with the incisive originality that marks this famous text from start to finish, places the lived experience of the slaves of St.~Domingue at the incipient point of the world-historical struggle of the global proletariat to overcome the injustices of global capitalism and to initiate a more just, communist form of social organization.\footnote{C. L. R. James, {\em The Black Jacobins: Toussaint L'Ouverture and the San Domingo Revolution} (1938; repr., New York: Vintage, 1989), 85--86.}

Only the year before, James, a militant Marxist-Leninist, had completed {\em World Revolution, 1917--1936: The Rise and Fall of the Communist International}, his incisive history of the Bolshevik Revolution modeled on Leon Trotsky's famous study.\footnote{C. L. R. James, {\em World Revolution, 1917--1936: The Rise and Fall of the Communist International}, ed. Christian Høgsbjerg (1937; repr., Durham, NC: Duke University Press, 2017). See, in particular, Høgsbjerg's meticulous and illuminating introduction to this recent republication of one of James's most important books. On James's inspiration, see Leon Trotsky, {\em History of the Russian Revolution}, trans. Max Eastman, 3 vols. (Ann Arbor: University of Michigan Press, 1932). On James's Marxism, see Scott McLemee and Paul Le Blanc, eds., {\em C. L. R. James and Revolutionary Marxism: Selected Writings of C. L. R. James, 1939--1949} (1994; repr., Chicago: Haymarket, 2018).} James stood at the time as perhaps the leading anglophone voice of global Trotskyism. Pursuing this engagement in even more original terms, {\em The Black Jacobins} is, among its many virtues and characteristics, an early and perhaps the greatest work of what was once called Leninist third-worldism.\footnote{See Robert J. C. Young, {\em Postcolonialism: An Historical Introduction} (London: Blackwell, 2016). Young analyzes the founding role of Leninist internationalism in the rise of anticolonialism after 1917, a phenomenon he terms \quotation{Tricontinentalism} and of which James is among the most eminent representatives.}

While James's Leninist internationalism is evident throughout {\em The Black Jacobins}, to the point of his famous musings on what Vladimir Lenin might have done in Louverture's position, in what follows I would like to probe the meaning and implications of James's reference in the above passage to the slaves of St.~Domingue as a \quotation{modern proletariat,} along with the corresponding, unstated implication in the passage that the system of plantation slavery production they experienced was an incipient form of, or tending toward, modern industrial capitalism.\footnote{This is a question that I raised in a previous essay, without adequately formulating an answer; see Nick Nesbitt, \quotation{Fragments of a Universal History,} in Charles Forsdick and Christian Høgsbjorg, eds., {\em The Black Jacobins Reader} (Durham, NC: Duke University Press, 2017), 141--42.}

To do so, I will limit my analysis to considering whether Karl Marx's understanding of the concepts of {\em proletariat} and {\em slavery} in his mature work correspond to the uses, both literal and implied, that James makes of them in the above passage. While Marx's conceptualizations of the proletariat and slavery by no means constitute the final word on the relation of slavery to capitalism, such a move is well justified, I think, not only by James's lifelong dedication to both political Marxism and Marx's theoretical critique of capitalism itself but also by the penetrating originality and rigor of Marx's theorization of the terms themselves. I focus on this passage from {\em The Black Jacobins}, then, because it is James's single clearest and most direct statement of the proletarian identity of the St.~Domingue slaves. The few other passing mentions of the \quotation{proletariat} in the book lack the specificity of this passage and merely equate the proletarian condition with that of the exploited laboring classes in general, independent of their historically specific subsumption to capital.\footnote{See James, {\em The Black Jacobins}, 283, 284. This is the case as well with the single other extensive reference to the proletariat in {\em The Black Jacobins}, the memorable passage from the 1963 appendix, \quotation{From Toussaint Louverture to Fidel Castro,} that echoes the initial formulation I have cited. In his 1963 revision, James underscores the transformative force of \quotation{the large-scale agriculture of the sugar plantation, which was a modern system}: \quotation{It further required that the slaves live together in a social relation far closer than any proletariat of the time. The cane when reaped had to be rapidly transported to what was factory production\ldots{}.The Negroes, therefore, from the very start lived a life that was in its essence a modern life} (392). In comparison with the earlier passage I am considering, here James reiterates the conclusion he had first drawn in 1938 from this proto-industrial form of the plantation, repeating the original terms of his argument: the defining characteristics of the proletarian condition are again summarized as (1) the \quotation{large-scale} of production processes, (2) the rapid internal distribution of raw materials to \quotation{factory production,} and (3) the intensive proximity of living conditions. In the absence of any more explicit definition of the term, I would argue that {\em proletariat} is for James, writing in 1963, simply to be equated with a transhistorical notion of the exploited laborer and laboring {\em masses} (a term far more common in {\em The Black Jacobins} than {\em proletariat}).}

A literal reading of the passage reveals both James's visionary rhetorical power of invention and a certain theoretical imprecision. James does not claim the slaves of {\em St.~Domingue} actually formed a proletariat, nor that plantation slavery was industrial capitalist in nature, but only that the situation {\em tended} in these directions.\footnote{Selma James interprets this passage differently, writing that C. L. R. James rejected \quotation{rigid definitions of class,} celebrating his precocious and probing examination of the \quotation{question of how to define who is working class.} Selma James, \quotation{The Black Jacobins, Past and Present,} in Forsdick and Høgsbjorg, {\em The Black Jacobins Reader}, 76.} How or in what ways the slaves were \quotation{closer to a modern proletariat than any group of workers in existence at the time,} James does not precisely say. The key factor is stated to be the scale of groups of laborers (\quotation{gangs of hundreds}) as well as that of the sites in which they were grouped (\quotation{huge sugar-factories}). The reference to factories, given the overarching Leninist perspective of the book, clearly implies, in my view, a vision of the sugar plantation as {\em tending toward} the modern industrial factory form that Marx would analyze in {\em Capital}, and, perhaps even more precisely (I will return to this key distinction below), as an instance of the intermediate form of production that he termed \quotation{manufacture.}\footnote{Karl Marx, {\em Capital: A Critique of Political Economy}, vol.~1, {\em The Process of Production of Capital}, trans. Ben Fowkes (New York: Penguin, 1976); originally published as {\em Das Kapital: Kritik der politischen Oekonomie} in 1867. After Marx's death in 1883, volumes 2 and 3 were prepared by Friedrich Engels from Marx's notes and were published in 1885 and 1894, respectively. On \quotation{manufacture,} see vol.~1, chap. 14.}

The conclusion James readily draws from these assertions---both here and above all in the full deployment of this argument across the book's mighty span---is justifiably renowned: the Haitian Revolution was not a chaotic rebellion but \quotation{a thoroughly prepared and organized mass movement,} perhaps the first instance of a consciously conceived and successfully implemented proletarian revolution, one that, like the Bolshevik Revolution a century later, occurred not in the center of the industrial West, as Marx and Friedrich Engels initially hoped, but at its putatively underdeveloped periphery.

One might justifiably point out that with the single exception of this key passage---which points forward in visionary terms to the future developments of this argument by figures such as Eric Williams, Eugene Genovese, and Robin Blackburn---the notion of the slave as proletarian is not a focus of {\em The Black Jacobins}.\footnote{See Eric Williams, {\em Capitalism and Slavery} (1944; repr., Chapel Hill: University of North Carolina Press, 1994); and Robin Blackburn, {\em The Overthrow of Colonial Slavery: 1776--1848} (1988; repr., New York: Verso, 2011). The key intertext to {\em The Black Jacobins} in this respect is undoubtedly W. E. B. Dubois's 1935 {\em Black Reconstruction in America}, in which Dubois writes---with militant, lyrical force fully the equal of James's---of a complex dialectic between white and black \quotation{proletariats,} identifying therein a violently repressed and historically disavowed tendency toward \quotation{a democracy which should by universal suffrage establish a dictatorship of the proletariat ending in industrial democracy.} W. E. B. Dubois, {\em Black Reconstruction in America} (1935; repr., Oxford: Oxford University Press, 2017), ix. James discusses the importance of Dubois's {\em Black Reconstruction} in his \quotation{The Revolution in Theory} (1977), in Forsdick and Høgsbjorg, {\em The Black Jacobins Reader}, 363.} The same cannot be said, however, of James's {\em World Revolution}, in which the terms {\em proletariat} and {\em proletarian} appear many hundreds of times. In {\em World Revolution}, which was written a year before {\em The Black Jacobins}, James unsurprisingly invokes on virtually every one of its five-hundred-some pages the proletarian working class and its struggle for the \quotation{dictatorship of the proletariat.} With a single exception, however, his use of the term is no more enlightening (as to the precise meaning he attributes to it) than in his subsequent history of the St.~Domingue revolution. In every other instance, of which there are some four hundred in the book, \quotation{proletariat} is deployed as a purely {\em political} term, referring to the (working) class engaged in a militant struggle for power against the bourgeois domination of the state and society.\footnote{James's {\em World Revolution} is redolent with references to \quotation{the dictatorship of the proletariat} (100, 103, 105); \quotation{the petty-bourgeoisie and the proletariat} (84); \quotation{the revolutionary proletariat} (106); \quotation{the Russian proletariat} (122); \quotation{the international proletariat} (124); \quotation{the Socialist proletariat} (163); \quotation{the advanced proletariat} (232); \quotation{the great German proletariat} (312); and so on. While the many deployments of the concept in {\em World Revolution} clearly serve to drive home the revolutionary, Trotskyist argument of James's history, the fact that James thought it unnecessary to define the term with any specificity in 1937 is nonetheless undoubtedly symptomatic, in the full Althusserian sense of the term. Symptomatic, that is to say, of what is arguably a traditional, Leninist interpretation of Marx and {\em Capital} more specifically as a critique made from the standpoint of labor rather than that of capitalism as a social totality. On the latter, see Moishe Postone, {\em Time, Labor, and Social Domination: A Reinterpretation of Marx's Critical Theory} (Cambridge: Cambridge University Press, 1993). James's critique of global capitalist imperialism, for all its brilliance, arguably remained a political critique of class-based exploitation that rarely if ever drew on the theoretical insights of Marx's categorial critique of political economy. James will write in this vein, in his 1944 essay \quotation{The American People in \quote{One World,}} that \quotation{the laws of dialectics are to be traced not in metaphysical abstractions\ldots{}but in economic development and the rise, maturity, and decline of different social classes within the expansion and constriction of the capitalist world market} (McLemee and Le Blanc, {\em C. L. R. James and Revolutionary Marxism}, 175). On James's original appropriation of traditional, Leninist-humanist Marxism, which Anthony Bogues terms \quotation{black Marxism,} see Anthony Bogues, {\em Caliban's Freedom: The Early Political Thought of C. L. R. James} (Chicago: Pluto, 1997), 6--9; on the original and precocious brand of Marxist humanism developed by the Johnson-Forrest Tendency, see 69--75: \quotation{We {[}the members of the Johnson-Forrest Tendency{]} were convinced on a re-examination of Marx's capital {[}{\em sic}{]},} observed James in 1947, \quotation{that the solution to the economic ills of capitalism was the human solution} (qtd. on 75).}

The exception to this uniformity occurs early in {\em World Revolution}, in James's initial presentation of Marxism, when he writes of the historical division of society into classes. There, he notes that the \quotation{struggle of economic forces for their full expansion was translated, as always, into a political struggle, the struggle for control of the State-power without which it is quite impossible to transform the organisation of society.}\footnote{James, {\em World Revolution}, 70.} In this context, James continues, there appeared \quotation{a new class,} one that,

\startblockquote
suffering intolerably from the difficulties created by the chaos in production, would be driven to seize the State-power and create the conditions for the new Socialist society\ldots{}. This class Marx and Engels found in the proletariat. And as the bourgeoisie within feudal society had been consolidated and disciplined by the direction and organisation of wealth, in the same way the proletariat, organised in factories by the development of the Capitalist system of production, disciplined by the increasing discipline of large-scale Capitalist organisation, would be forced to combine industrially and ultimately politically by the increasing pressure upon them of the bourgeois system of production.\footnote{Ibid., 78--79.}
\stopblockquote

Here, James gives more extensive description and attention to the notion of the proletariat, as a specific historical development of European social class structures in the wake of the French Revolution. That said, the basic characteristics he ascribes to the concept are virtually identical to those of {\em The Black Jacobins.} The proletariat, in this formulation, is characterized by the large-scale organization of laborers in a factory system of production, by intensive proximity of working conditions (\quotation{forced to combine industrially and ultimately politically}), and by what James calls \quotation{the increasing pressure upon them of the bourgeois system of production,} by which he presumably means the demands for increases in both intensity and duration of production (what Marx called the pressure to realize increases in \quotation{absolute surplus value}).\footnote{On Marx's concept of absolute surplus value, see {\em Capital}, vol.~1, section 3, \quotation{The Production of Absolute Surplus-Value.}}

In light, then, of James's politico-historical, Marxist-Leninist deployment of the term {\em proletariat}, both in {\em World Revolution} and {\em The Black Jacobins}, generally and above all in the key passage from the latter with which I began, it seems to me fair and even compelling to ask whether the terms of James's presentation, both literal (\quotation{proletariat}) and implied (plantation slavery as proto-industrial capitalism), actually correspond to Marx's uses of them. I believe in each case the answer is no.\footnote{Unsurprisingly, James appears to have read the three volumes of {\em Capital} intensively, though not until well after the composition of {\em World Revolution} and {\em The Black Jacobins}. James thus asserted in a 1939 letter to Constance Webb his intention to study in its entirety Marx's magnum opus: \quotation{And now for {\em Das Kapital}. My dear young woman, I have some news for you. One CLR James, reported marxist having thought over his past life and future prospects, decided that what he needed was a severe and laborious study of? guess! The Bible? Wrong. Ferdinand the Bull? Wrong again. Not {\em Das Kapital}? Right. I shall do these three volumes and nothing will stop me but a revolution} (qtd. in Bogues, {\em Caliban's Freedom}, 53). Despite this familiarity, and in contrast to his extensive commentaries on Marx's {\em 1844 Manuscripts} and Hegel's {\em Science of Logic}, texts that he critiqued intensively with Raya Dunayevskaya and the other members of the Johnson-Forrest Tendency, James never seems to have undertaken a similar critical appraisal of Marx's {\em Capital}. The one exception is his 1967 talk \quotation{Marx's {\em Capital}, the Working-day, and Capitalist Production,} in which James discusses not the concept of labor power or the capitalist mode of production (concepts at issue in the present essay) but chapter 10 of {\em Capital}, vol.~1, on the lived experience of class struggle over the duration of the working day: \quotation{{[}Marx{]} is concerned with what is happening to members of the working-class as living human beings in a factory.} C. L. R. James, {\em You Don't Play with Revolution: The Montreal Lectures of C. L. R. James}, ed. David Austen (Edinburgh: AK, 2009), 151. It is in any case certain that a central dimension of the intellectual project of the Johnson-Forrest Tendency in the 1940s involved a close reading of Marx's {\em Capital}, a reading the group brought to bear against their corresponding intensive study of Hegel's {\em Logic}. Grace Lee Boggs observed in 1947, \quotation{As soon as C. L. R. discovered that I could read German and had studied Hegel, he had me translating sections of {\em Das Kapital} and comparing the structure of capital with Hegel's logic} (qtd. in Bogues, {\em Caliban's Freedom}, 60).}

How did Marx define the proletariat in his mature (post-1857) writings? His use of the word {\em proletariat} across the three volumes of {\em Capital} can in fact be readily summarized, and in no case, except for the most metaphorical, impressionistic sense, can it be applied to plantation slavery. Occasionally, and also most imprecisely, Marx uses the term in a general way to indicate the exploited working classes, as, for example, in the following evocation of the origins of the 1830 July Revolution in France: \quotation{The learned dispute between the industrial capitalist and the wealthy landowning idler as to how the booty pumped out of the workers may most advantageously be divided for the purposes of accumulation had to fall silent in the face of the July Revolution. Shortly afterwards, the urban proletariat sounded the tocsin of revolution at Lyons, and the rural proletariat began to set fire to farmyards and hayricks in England.}\footnote{Marx, {\em Capital}, 1:743. With similar vagueness, in {\em Theories of Surplus-Value} (written in 1863 but initially projected as vol.~4 of {\em Capital}), Marx writes, \quotation{Malthus defends the interests of the industrial bourgeoisie only in so far as these are identical with the interests of landed property, of the aristocracy, i.e., against the mass of the people, the proletariat} (chap. 9.1; emphasis mine). See \useURL[url1][https://www.marxists.org/archive/marx/works/download/epub/capital-v4.epub]\from[url1].} This is {\em proletariat} taken in the sense of the working class as such, in a capacious, transhistorical internationalism, one that would include workers of all sorts in their common experience of exploitation, independently of the mode of production in which they labored, whether archaic, feudal, or capitalist.

This sense of the term is universally familiar in Marx and Engels's famous invocation in the {\em Communist Manifesto} to \quotation{Proletarians of all countries!,} in which proletarians are those who \quotation{have nothing to lose but their chains.}\footnote{Karl Marx and Friedrich Engels, {\em Manifesto of the Communist Party}, ed. Terrel Carver (Cambridge: Cambridge University Press, 1996), 30.} In this most famous early text---written before Marx began with the {\em Grundrisse} in 1857, his critical analysis of capitalism as social form---it is the entire worker herself, and not merely, as in the analysis of {\em Capital}, her {\em labor power}, that is the commodity she sells to capital. Arguably, while the purchase and deployment of the slave as commodity falls under the compass of this universal experience of exploitation, to enlarge the concept of proletariat to such an extent renders it rhetorically compelling but analytically meaningless.

That said, this broad, conceptually vague use of the term {\em proletariat} clearly resonates with its traditional, pre-Marxian usage. Famously derived from the Latin {\em proletarius}, meaning any citizen of the lowest class in Roman society, it entered into modern usage from the sixteenth century to refer to those who live off their manual labor, along with the poor and indigent more generally.\footnote{Le Robert, {\em Dictionnaire historique de la langue française} (2018), s.v. \quotation{prolétariat.}} The implication I am suggesting here is that James's use of the term {\em proletariat} in {\em The Black Jacobins} might reasonably be thought to invoke this expansive sense of exploitation---from the ultra-feudal application of direct violence in slavery, symbolized by the master's whip, to the invisible, fetishized extraction of surplus value from the \quotation{free} wage laborer by the capitalist---were it not, that is, for the single adjective the militant Marxist-Leninist places before the term: \quotation{They were closer to a {\em modern} proletariat than any group of workers in existence at the time.}

Marx did not invent the modern notion of {\em proletarian} that he immortalized in the {\em Manifesto} and went on to analyze in scientific detail in {\em Capital}, in the sense, that is, of a wage laborer in the employ of an industrial capitalist. Already in the 1840s, left Ricardians and socialists such as Henri de Saint-Simon, Pierre-Joseph Proudhon, J. C. L. Simonde de Sismondi and their lesser followers used the term freely. Marx knew this literature intimately, and one finds scores of citations and their scathing critique scattered across the thousands of pages of Marx's post-1857 writings.\footnote{Two of Marx's many citations gleaned from Capital, vol.~1: \quotation{One might almost say that modern society lives at the expense of the proletarians, on the portion of the wages of labour which it withdraws from their pockets} (Sismondi, qtd. on 207); \quotation{The proletarian, by selling his labour for a definite quantity of the means of subsistence ({\em approvisionnement}), renounces all claim to a share in the product} (Charbuliez, qtd. on 718). J. C. L. Simonde de Sismondi, {\em Études sur l'économie politique}, vol.~1 (Bruxelles: Société Typographique Belge, 1837), 1:24; Antoine-Elisée Charbuliez, {\em Richesse ou pauvreté} (Paris: A. Le Gallois, 1841), 14.} However, in his systematic analysis and critique of capitalism as a general social form or structure, {\em Capital}, that is to say, Marx, developed two analytically precise and, in fact, mutually exclusive notions of {\em proletariat.}\footnote{That one finds in {\em Capital} differing, even contradictory, uses and definitions of the concept {\em proletariat} is not surprising, given the often rough and unfinished nature of its three volumes. While it is well known that after Marx's death, Engels constructed volumes 2 and 3 for publication from various drafts Marx left in his papers, even volume 1---the publication of which Marx oversaw through its second, revised German edition and first French translation (1872--75)---bears no final form; Marx intended and made notes for a further complete revision of it in 1881 (which he never carried out). What we know as volume 1 is a bricolage Engels made of the second German edition of 1872 and the French edition of 1872--75. While the first draft of volume 1 from 1863 is lost (the {\em Grundrisse} of 1857--58 is not, properly speaking, a draft of {\em Capital} but a mass of preparatory notes that Marx termed an \quotation{inquiry} {[}{\em Forschungsweise}{]}), the first German edition of 1867 as well as Marx's notes for his intended 1881 revision contain many important formulations and developments of his demonstration (in particular, a radically different presentation of the key first chapter on the value-form) absent from Engels's posthumous \quotation{definitive edition.} In fact, attention to the totality of Marx's drafts, notebooks, and letters from 1857 to 1881, gradually becoming available in the definitive MEGA\high{2} edition of Marx and Engels's works, leads to the conclusion, in Michael Heinrich's words, that \quotation{in a strict sense, a three-volume work \quote{Capital} written by Marx does not really exist\ldots{}.There is no clear difference between drafts and the final work---we have only differently developed drafts of a shifting, unfinished and incomplete project.} Michael Heinrich, \quotation{Reconstruction or Deconstruction? Methodological Controversies about Value and Capital, and New Insights from the Critical Edition,} in Riccardo Bellofiore and Roberto Finneschi, eds., {\em Re-reading Marx: New Perspectives after the Critical Edition} (London: Palgrave, 2009), 96. If, as Heinrich shows, it is the case that these various iterations of Marx's project reveal distinct and ambivalent understandings of such key concepts as \quotation{value} and \quotation{labor-power,} it is hardly surprising that this should be the case for concepts of lesser importance to Marx's categorial critique, such as \quotation{proletariat.} Moreover, here I am only considering the standard English edition of {\em Capital}, vols. 1--3. Undoubtedly, there remain to be discovered further variations in Marx's conceptualization of the \quotation{proletariat} in the voluminous editions, drafts, correspondence, and notes of the MEGA\high{2}.} Neither, in point of fact, is in any conceivable fashion applicable to plantation slavery understood as a specific, historical mode of production.

The first is that just mentioned, that of the social class of wage laborers employed by capital. In the final chapters of {\em Capital}, volume 1, Marx takes up and even cites his earlier formulation of the concept in the {\em Manifesto},\footnote{\quotation{The other classes decay and disappear in the face of large-scale industry, the proletariat is its most characteristic product} (Marx, {\em Capital}, 1:930).} but goes on in the final chapters of volume 1 to give the term far more analytical precision than in the earlier, polemical setting of the {\em Manifesto}.\footnote{\quotation{The workers have been turned into proletarians, and their means of labour into capital, as soon as the capitalist mode of production stands on its own feet} (Marx, {\em Capital}, 1:928).} In fact, in a footnote to chapter 25 (\quotation{The General Law of Capitalist Accumulation}), Marx categorically affirms this first sense to be the manner in which he understands the concept within his systematic analysis:

\startblockquote
\quotation{Proletarian} must be understood to mean, economically speaking, nothing other than \quotation{wage-labourer,} the man who produces and valorizes \quotation{capital.}\ldots{}\quotation{The sickly proletarian of the primitive forest} is a pretty Roscherian fancy. The primitive forester is the owner of the primitive forest and uses it as his property, meeting as few obstacles to this as an orang-utang. He is not, therefore, a proletarian. This would only be the case if the primitive forest exploited him, instead of being exploited by him.\footnote{Marx, {\em Capital}, 1:764n1. By \quotation{Roscherian fancy,} Marx is referring to Wilhelm Roscher, author of the 1858 volume {\em Die Grundlagen der Nationalökonomie}.}
\stopblockquote

Marx is here unambiguous, and, regarding James's own assertion, it is clear that by this definition, the plantation slaves were no proletariat whatsoever. There can be no question of tendency or degree by Marx's definition, no \quotation{the closest thing to} earning a wage for their labor. Either one's entire being is an object of possession and forced to labor by the application of direct violence, or one earns a wage, however derisory.

This unambiguous clarity is muddled, in Marx's case, by a second, quite distinct definition of the proletariat that he puts forward in the famous chapter of {\em Capital}, \quotation{The Secret of Primitive Accumulation.} There, Marx identifies the proletarian as the person who formerly, and by secular right, owned certain means of production and, also by right, farmed land for her and her family's sustenance; from these means and rights she assured her existence and reproduction.\footnote{Adding to our uncertainty, Marx at one point even seems to distinguish this \quotation{free} proletarian from wage labor---in time, space, and nature: \quotation{We have considered the forcible creation of a class of free and rightless proletarians, the bloody discipline that turned them into wage-labourers} ({\em Capital}, 1:905).} In the brutal process of primitive accumulation, these means and rights are violently stripped away, and the rightless, possessionless proletarian is cast out, \quotation{free,} expelled from society: \quotation{The proletariat created by the breaking-up of the bands of feudal retainers and by the forcible expropriation of the people from the soil, this free and rightless proletariat could not possibly be absorbed by the nascent manufactures as fast as it was thrown upon the world.}\footnote{Marx, {\em Capital}, 1:896. Marx's famous historical example is that of the Scottish Crofters, the dispossession of whom created a labor \quotation{reserve army,} one willing and indeed forced subsequently to take on wage labor for their very survival. Similarly, he writes of the Italian case: \quotation{In Italy, where capitalist production developed earliest, the dissolution of serfdom also took place earlier than elsewhere. There the serf was emancipated before he had acquired any prescriptive right to the soil. His emancipation at once transformed him into a \quote{free} proletarian, without any legal rights, and he found a master ready and waiting for him in the towns} ({\em Capital}, 1:876). On the centrality of such transformations of property rights for the imposition of capitalism, see Ellen Meiksins Wood, {\em The Origin of Capitalism: A Longer View} (1999; repr., New York: Verso, 2017). Wood argues convincingly that the essential factor in the imposition of capitalism as a social form is the transformation of property rights. More specifically, this transformation made inaccessible to the impoverished classes (such as the Scottish Crofters) the basic amenities essential for survival, such as food and shelter. Cast out by the Scottish Lords from their secular homesteads as rightless, \quotation{free} proletarians (in Marx's example), they were forced to accept the contractual conditions of capitalist wage labor in, for example, the Glasgow shipyards in order to purchase the means of survival and their reproduction on the market with these wages.}

This second definition of the term on Marx's part evokes the original, literal meaning of the term: the {\em pro-letarii}, utterly marginalized from society, those whose only activity is to \quotation{produce offspring.} Marx repeatedly, and with savage irony, describes these rightless proletarians as {\em vogelfrei}, \quotation{free as birds} (as in the phrase, \quotation{Seine Emanzipation verwandelt ihn also sofort in einen vogelfreien Proletarier}); free, that is, as entirely outside human society, rightless and unable to sustain themselves.\footnote{Marx, {\em Capital}, 1:897.} The point here, as with the contrasting definition of the proletarian as wage laborer, is that the plantation slave, in her juridically codified bondage, can in no sense be considered \quotation{free,} even in this most ironic, legalistic sense.

It would instead be far more appropriate to use the term {\em proletariat} in Marx's second sense, to indicate the experience of {\em emancipated} slaves in the French Caribbean, a newly \quotation{enfranchised} community forced to survive by whatever means available. While in the wake of the 1794 Jacobin abolition survival entailed the construction of autonomous communes, such as that in Les Cayes, by the time of France's second, 1848 emancipation, the increasing subsumption of Antillean society to the demands of capitalist valorization meant that this veritable proletarianization was experienced as forcible subjection to the contractual demands of wage labor.\footnote{On the 1790s Les Cayes Maroon communes, see Carolyn Fick, {\em The Making of Haiti: The Saint-Domingue Revolution from Below} (Knoxville: University of Tennessee Press, 1990). In terms perfectly coherent with Marx's analysis of primitive accumulation, Rosamunde Renard describes this historical process in post-1848 Martinique and Guadeloupe, where the commodification of the basic means of survival and reproduction (along with the imposition of various mandatory taxes and employment requirements) effectively forced former slaves (despite various and unrelenting resistance) to accept the new conditions of contractual wage labor on the plantations: \quotation{Those ex-slaves who continued to do plantation work did not do so voluntarily in many instances. The ex-slaves incurred expenses unknown to the slave. Food, medical bills, clothing, rents---all wholly or partly supplied by the plantation before emancipation---were now their personal responsibility. Moreover the plantocracy and the administration increased these expenses even more in the attempt to create an artificial need for money wages so that the blacks would be forced to work on the plantations.} Rosamunde Renard, \quotation{Labor Relations in Martinique and Guadeloupe, 1848--1870,} {\em Journal of Caribbean History} 26, no. 1 (1992): 39.}

The historical shift between plantation slave-based and industrial-capitalist modes of production in the eighteenth and nineteenth centuries---from the categorial perspective of Marx's critique of political economy---neither comprised a gradual, tendential shift from the former to the latter, nor was it a matter of the size of manufactures and concentrations of laborers, as James implies in {\em The Black Jacobins}. Instead, Marx arguably identifies three key factors that distinguish these modes of production, factors that are binary rather than linear.\footnote{To further develop this argument, the distinction I am presenting between James's linear, developmental understanding of the relation of slavery to capitalism and Marx's binary, either/or analysis of the concept of mode of production should be confronted as well with James's appropriation of Trotsky's notion of combined and uneven development. In this view, plantation slavery would constitute not a barbaric remnant but a variegated element in the overdetermined complex of Atlantic modernity. Such a view, however (and this is the point of the present analysis), by simply positing \quotation{slavery} as a differential element within an equally underdefined totality (\quotation{capitalist modernity}), continues to put forth the question of the precise, adequate analytical distinction between slavery and capitalism. On James's appropriation and deployment of Trotsky's concept of combined and uneven development, see Bill Schwarz, \quotation{Haiti and Historical Time,} in Forsdick and Høgsbjorg, \quotation{The Black Jacobins} Reader, 99--102. On uneven development, see Bruno Bosteels, \quotation{Reading Capital from the Margins: Notes on the Logic of Uneven Development,} in Nick Nesbitt, ed., {\em The Concept in Crisis: Reading \quotation{Capital} Today} (Durham, NC: Duke University Press, 2017), 113--65.} I say arguably because Marx nowhere analyzes this question as such, and understandably so, since the object of {\em Capital} is the categorial analysis and critique of capitalism in its ideal form; the historical problems of both of its genesis and supersession are never addressed in more than passing detail.\footnote{The famous chapters on primitive accumulation are the exception that proves the rule, insofar as they are excurses to Marx's object of analysis, placed precisely after volume 1's analysis of the capitalist mode of production, despite the fact that historically the events they discuss occur prior to the development of capitalism in its high-industrial form.} Instead, the determinant factors of these two modes of production must be identified and collated from analyses occurring across the vast expanse of the three volumes of {\em Capital}.

What distinguishes the various economic formations of society, the distinction between, for example, \quotation{a society based on slave-labour and a society based on wage-labour,} Marx writes, \quotation{is the form in which this surplus labour is in each case extorted from the immediate producer, the worker.}\footnote{Marx, {\em Capital}, 1:327.} On this count, the categorial distinction between the plantation slavery and capitalist modes of production is not a matter of degrees and tendencies. Size of manufactures, concentrations of workers, degrees and forms of suffering and violence (the criteria James mentions in both {\em World Revolution} and {\em The Black Jacobins}) are indeed, as James suggests, matters of relative degree and differentiation; for the question at hand, however, they are transhistorical, nonspecific to capitalism, and, as such, cannot serve to constitute the distinction between these two modes of production.

Marx instead identifies various other \quotation{factors,} such as property rights, the source of surplus value, and the dual nature of living labor under capitalism as productive of both use values and surplus value.\footnote{Among the key theoretical distinctions Marx makes in his critique of political economy is that between the production of wealth (in the form of use values), exchange value (manifest in the form of prices), and value itself. This fundamental theoretical insight not only marks Marx's decisive break with Ricardian value theory (Marx was not a Left Ricardian) but remained as well a distinction generally overlooked by the productivist orientation of traditional, Leninist Marxism such as that professed, in singular fashion, to be sure, by James. On this founding theoretical distinction of Marx's critique---developed from the very first paragraph of {\em Capital}---and its invisibility in traditional Marxism, see Postone, {\em Time, Labor, and Social Domination}, 7--15. On the concept of Left Ricardianism---that is, the failure to clearly distinguish wealth from value and the consequent promotion of the redistribution of that wealth rather than the destruction of the capitalist mode of production---see Patrick Murray, \quotation{The Illusion of the Economic: The Trinity Formula and the \quote{Religion of Everyday Life,}} in Martha Campbell and Geert Reuten, eds., {\em The Culmination of Capital: Essays on Volume III of Marx's Capital} (New York: Palgrave 2002), 250--52.} One is either a chattel slave, the material possession of a slave owner, or one is a contractual, \quotation{free} seller of her labor power for its market value. Such a distinction is not a matter of degree, of being, in James's words, \quotation{closer to a modern proletariat,} but is rather absolute. In conclusion, then, I shall briefly review Marx's understanding of the difference between plantation slavery and capitalism, to more clearly grasp the distinction between the two---a distinction James himself invokes only in passing.

Briefly---since to adequately treat the topic of Marx and slavery would require a book unto itself---in plantation slavery Marx reasonably understands the slave's entire person to be the possession or property of the slaveholder. She is, in the infamous words of the 1685 Code Noir, his \quotation{meuble.} This in fact remained legally the case in the French colonies until 1848. The words of the slave merchant Boisset, writing in 1802 to ask a friend to sell six of his slaves, invoke this property right: \quotation{I give them to you, dear Comrade, and authorize you to sell them; they are my property.}\footnote{Quoted in Josette Fallope, {\em Esclaves et citoyens: Les noirs à la Guadeloupe au XIXe siècle} (Basse Terre: Société d'histoire de la Guadeloupe, 1992), 574 (all translations mine).} Correspondingly, as property, slaves were to be legally prevented in turn from possessing property of their own. Section 28 of the code is explicit on this count: \quotation{Slaves can possess nothing which is not their Master's, and all that comes to them by industry or the liberality of other persons\ldots{}is acquired as the full property of their Master.}\footnote{Ibid., 593.}

As such, the slave is to be considered as, analytically---as opposed to experientially or phenomenologically---what Marx calls {\em constant capital.} Like any other machine, animal, or tool for production that the plantation owner purchases as a fixed cost, the entire person of the slave as means of production, as productive machine, is to be used up as profitably (i.e., as productively and cheaply) as possible: \quotation{In the slave system, the money capital laid out on the purchase of labour-power plays the role of fixed capital in the money form, and is only gradually replaced as the active life of the slave comes to an end\ldots{}.The purchase and sale of slaves is also in its form a purchase and sale of commodities.}\footnote{Karl Marx, {\em Capital: A Critique of Political Economy}, vol.~2, {\em The Process of Circulation of Capital}, trans. David Fernbach (New York: Penguin, 1978), 554, 115. In volume 1 of Capital, Marx observes in this vein, \quotation{The slave-owner buys his worker in the same way as he buys his horse. If he loses his slave, he loses a piece of capital, which he must replace by fresh expenditure on the slave-market\ldots{}.The slave is the property of a particular master; the worker {[}in contrast{]} must indeed sell himself to capital, but not to a particular capitalist, and so within certain limitations he may choose to sell himself to whomever he wishes; and he may also change his master} (1:377, 1032).} The bookkeeping practices of the plantation system make this eminently clear (see fig. 1).

\placefigure[here]{\quotation{Tableau des Finances et du Commerce de la partie Françoise de St.~Domingue,} 1792 (detail). Copy in the John Carter Brown Library, Brown University.}{\externalfigure[issue03/nesbitt-1.jpg]}


Marx's assertion of the structural equivalency of slaves, machines, and animals in the plantation mode of production is strikingly confirmed in this 1792 document from St.~Domingue, which summarizes in a single page the wealth of St.~Domingue in the calendar year 1791.\footnote{On the development of qualitative management practices in New World slavery, see Caitlin Rosenthal, {\em Accounting for Slavery: Masters and Management} (Cambridge, MA: Harvard University Press, 2018); and Ian Baucom, {\em Specters of the Atlantic: Finance Capital, Slavery, and the Philosophy of History} (Durham, NC: Duke University Press, 2005).} The document tabulates the enormous slave-based wealth of St.~Domingue during the interim period between 1789 and the 6 February 1794 abolition, when slavery was steadfastly sustained by revolutionary governments nominally committed to creating a state based on the universal rights of man and citizen. Despite the strict {\em structural} equivalency it asserts between slaves (\quotation{nègres}), lime ovens (\quotation{fours à chaux}), horses, mules, and \quotation{horned animals,} the enormous inequality of value between the latter and the 455,000 slaves the document enumerates is immediately striking. Even more ironically, by the time the document would have been communicated to the Metropolitan government---perhaps by François de Barbé-Marbois, the last intendant of St.~Domingue---its content would have been rendered null by the Northern Plain slave uprising of 21 August 1791 that destroyed many hundreds of the sugar, coffee, and indigo plantations this document so methodically enumerates and values.

The essential analytical distinction this document conveys, between slavery and capitalist wage labor, is theoretically grounded in Marx's categorial analysis: \quotation{The capitalist,} Marx observes in the second volume of {\em Capital}, \quotation{cannot sell the worker again as a commodity, for he is not his slave, and the capitalist has bought nothing more than the utilization of his labour-power for a certain time.}\footnote{Marx, {\em Capital}, 2:118.} As such, and this is Marx's key point, in contrast to the wage laborer, no form of constant capital (whose value component, in other words, remains constant), including slaves, is able to produce the essential and defining element of capitalism---incremental increases in surplus value---but instead merely passes on to a commodity preexisting value. Though enabling the {\em capture} of market profit by their owner, they themselves {\em produce} only material use values (sugar, coffee).\footnote{On the crucial but complex and difficult distinction Marx makes throughout the three volumes of {\em Capital} between profits captured by any given capitalist firm---via competition---from a total global mass of surplus value and the actual production of surplus value by variable capital (i.e., living labor), see the writings of Fred Moseley, including {\em Money and Totality: A Macro-monetary Interpretation of Marx's Logic in \quotation{Capital} and the End of the \quotation{Transformation Problem}} (Chicago: Haymarket, 2017).}

Among the key factors Marx identifies in the transition to the capitalist mode of production, the imposition of the contractual wage labor relation is key.\footnote{\quotation{In slave labour, even the part of the working day in which the slave is only replacing the value of his own means of subsistence, in which he therefore actually works for himself alone, appears as labour for his master. All his labour appears as unpaid labour. In wage-labour, on the contrary, even surplus labour, or unpaid labour, appears as paid. In the one case, the property-relation conceals the slave's labour for himself; in the other case the money-relation conceals the uncompensated labour of the wage-labourer. We may therefore understand the decisive importance of the transformation of the value and price of labour-power into the form of wages, or into the value and price of labour itself} (Marx, {\em Capital}, 1:680).} It allows for extraction of surplus value from living labor, a unique feature of the capitalist mode of production. Moreover, one might add, among the key points {\em The Black Jacobins} as a whole makes eminently clear is that the historical transition from slavery to capitalism is precisely not gradual and tendential but revolutionary, the explosive and often violent destruction of one system and its replacement by another: 1791, 1804, 1848.\footnote{\quotation{When the former slave-owner engages his former slaves as paid workers, etc., then we find that what is happening is that production processes of varying social provenance have been transformed into capitalist production\ldots{}.The slave ceases to be an instrument of production at the disposal of his owner.} Marx, Capital, 1:1020. In volume 3 of Capital, Marx writes similarly, \quotation{Among other things, the capitalist mode of production is distinguished from the mode of production founded on slavery by the fact that the value or price of labour-power is expressed as the value or price of labour {[}power{]} itself, i.e.~as wages.} Karl Marx, {\em Capital: A Critique of Political Economy}, vol.~3, {\em The Process of Capitalist Production as a Whole}, trans. David Fernbach (New York: Penguin, 1981), 121.}

In notebook 2 of the {\em Grundrisse}, written in November 1857, Marx makes what in the context of James's assertions is an extremely interesting observation on plantation slavery in the United States and its relation to capitalism: \quotation{Negro slavery--a purely industrial slavery---which is, besides, incompatible with the development of bourgeois society and disappears with it, presupposes wage labour, and if other, free states with wage labour did not exist alongside it, if, instead, the Negro states were isolated, then all social conditions there would immediately turn into pre-civilized forms.}\footnote{Karl Marx, {\em Grundrisse: Foundations of the Critique of Political Economy}, trans. Martin Nicolaus (1973; repr., New York: Penguin, 1993), 223; see also 464. Marx embarks on extensive, if fragmentary, analyses of slavery and capitalism at various points in the {\em Grundrisse} notebooks: see 419--20, 464--69, 471--514, and 547--48. Marx closely observed the fight to abolish US slavery in the 1850s and 1860s and wrote detailed analyses of the North American struggle in many of his articles for the New York Daily Tribune---at the time the most widely distributed paper in the Atlantic world---even writing directly to Lincoln on behalf of the International Working Man's Association to congratulate him on the victory in the Civil War. See Karl Marx, {\em Dispatches for the \quotation{New York Tribune}: Selected Journalism of Karl Marx} (New York: Penguin, 2008). The 28 January 1865 letter to Lincoln is available at \useURL[url2][https://www.marxists.org/archive/marx/iwma/documents/1864/lincoln-letter.htm]\from[url2]. On the {\em Grundrisse}, see Nick Nesbitt, \quotation{The Grundrisse,} in Jeff Diamanti, Andrew Pendakis, and Imre Szeman, eds., {\em The Bloomsbury Companion to Marx} (London: Bloomsbury, 2018).} In this brief but dense comment, Marx observes that while American slavery is \quotation{industrial} in scale and methods of production, this does not mean it is any sense approaching or taking the form of the capitalist {\em mode of production}; this is the case because a mode of production as Marx understands it is a {\em form of social relations} (property, legal, political, familial, reproductive, economic, etc.) and not a mere (\quotation{industrial}) method of producing use values. As such, he states, US (\quotation{negro}) slavery remains \quotation{incompatible with the development of bourgeois society,} an entirely distinct (social) mode of production. Nonetheless, it exists historically within a broader socio-economic American context based on capitalist wage relations (\quotation{free states with wage labour}), and even shares certain superficial similarities with that larger economic context (industrial scale of production, etc.). Left on its own, however, its {\em social form} would revert, Marx asserts, to its own inherent, purely precapitalist, \quotation{pre-civilized} (slavery is, Marx observes, fundamentally barbaric) forms of appearance.\footnote{James in this vein makes a rare analytical commentary on the historical transformation of American slavery and its approximation to capitalism in the 1949 essay \quotation{Stalinism and Negro History}: \quotation{In proportion as the export of cotton became of interest to the United, States, patriarchal slavery was, in the words of Marx, \quote{drawn into the whirlpool of an international market dominated by the capitalist mode of production.} The structure of production relations was thereby altered. By 1860 there were over 2,000 plantations each with over a hundred slaves. Division of labor increased. Slaves began to perform skilled labor, were hired out for wages. Slave production took on more and more the character of social labor.} McLemee and Le Blanc, {\em C. L. R. James and Revolutionary Marxism}, 190. On the historical development of American slavery and capitalism, see Clavin Schermerhorn, {\em The Business of Slavery and the Rise of American Capitalism, 1815--1860} (New Haven, CT: Yale University Press, 2015).}

Analogously, while slavery in St.~Domingue existed in a wider eighteenth-century context of increasing industrialization of production and imposition of wage labor, and despite the fact that it adopted some of the superficial forms of appearance of nascent capitalism (\quotation{working and living together in gangs of hundreds on the huge sugar-factories}), as a mode of production, in Marx's sense of the term, slavery in St.~Domingue remained entirely distinct from early Atlantic capitalism, barbarically premodern, a social form and mode of production in which humans constituted legal, machinic property (see fig. 2).

\placefigure[here]{\quotation{Sucrerie,} from Denis Diderot et al., {\em Encyclopédie, ou, Dictionnaire raisonné des sciences,des arts et des métiers\ldots{}/ par Diderot {[}et{]} d'Alembert} (1751--71) (Parma: F. M. Ricci, 1970--79), vol.~1, plates.}{\externalfigure[issue03/nesbitt-2.jpg]}


The complex, overdetermined historical process of the transition from plantation slavery to the industrial production of sugar in Guadeloupe and Martinique in the 1840s in fact serves precisely to underscore the imperative of analytical clarity regarding the problem of transition. If in the years immediately prior to the 1848 abolition there developed a series of bizarre anomalies in the system, from the renting out of \quotation{slaves} who then received salaried employment in the towns and cities, to the various documented cases of so-called slaves actually employing and paying salaries to their so-called masters, such phenomena testify not to the gradual, linear, becoming-capitalist of the plantation mode of production but instead to the chaotic breakdown and \quotation{decomposition} of a secular social structure, the anarchic disordering of the distinguishing characteristics of the slave-based mode of production as social-form, prior to its industrial restructuration after 1848.\footnote{\quotation{On the plantation\ldots{}near Basse Terre, it is not uncommon to see slaves pay a salary to freemen, and to employ them in the planting of their gardens. On the plantation Saint-Charles, most of the blacks are well-off; some among them live off their rents, paying others to work their land, even by freemen, and receive a regular income {[}redevance{]} from it.} {\em Exposé général des résultats du patronage des esclaves des colonies françaises} (Paris: Impimerie Royale, 1844), 128; quoted in Fallope, {\em Esclaves et citoyens}, 596. On the economic, social, political, and historical decomposition of slavery in 1840s Guadeloupe, see 237--308.}

Such chaotic mutations and anomalies of the transition from slavery to capitalism are the stuff of the historical record. To make sense of them requires---beyond the mere registration and cumulative narration of their phenomenal appearance scattered across the archival records---the clear conceptual distinction between two modes of production that Marx developed across the three volumes of {\em Capital}, such that these incongruous and inconsistent {\em facts} might be subject to an analytical measure: if a well-off \quotation{slave} who employs his \quotation{master} for a wage to make from this labor a profit is perhaps not yet a capitalist, prior to gaining the legal entitlements of abolition, he no longer, beyond any doubt, functions socially and economically as a slave.

While much has been written on the relation of slavery and capitalism, the question has arguably never been addressed in terms adequate to Marx's categorial, structural analysis in {\em Capital.}\footnote{On the methodology of {\em Capital} as a form-based, categorial---as opposed to historical or empiricist---analysis of capitalism (specifically, as what Marx calls \quotation{value-form analysis}), see Postone, {\em Time, Labor, and Social Domination}, 123--44; Juan Iñigo-Carrera, \quotation{Method: From the {\em Grundrisse} to {\em Capital},} in Riccardo Bellofiore, Guido Starosta, and Peter D. Thomas, eds., {\em In Marx's Laboratory: Critical Interpretations of the \quotation{Grundrisse}} (Chicago: Haymarket, 2014), 43--70; and Riccardo Bellofiore, \quotation{The Multiple Meanings of Marx's Value Theory,} {\em Monthly Review}, 1 April 2018, \useURL[url3][https://monthlyreview.org/2018/04/01/the-multiple-meanings-of-marxs-value-theory/]\from[url3].} The place to begin such a value-form analysis of the slavery and capitalist modes of production remains Etienne Balibar's brilliant, foundational essay in which he develops for the concept of {\em mode of production} the structural reading of {\em Capital} initiated by Louis Althusser in their famous 1965 collaboration {\em Reading Capital}.\footnote{Etienne Balibar, \quotation{On the Basic Concepts of Historical Materialism,} in Louis Althusser, Jacques Rancière, Pierre Macherey, Roger Establet, and Etienne Balibar, {\em Reading Capital: The Complete Edition} (1965; repr., New York: Verso, 2017), 357--480. See also Nick Nesbitt, \quotation{Value as Symptom,} in Nesbitt, {\em Concept in Crisis}, 229--79. I would argue that while {\em Reading Capital} invents a whole range of singular concepts central to contemporary critical theory, its various chapters also initiate, in 1965, this now-dominant categorial or value-form reading of {\em Capital}, an approach to be found in theorists from Postone, Robert Kurz, Elena Louisa Lange, and Michael Heinrich to the outstanding series of publications of the members of the International Symposium on Marxist Theory (Bellofiore, Moseley, Tony Smith, et al.), and this despite the silent disavowal or even active hostility toward Althusser's thought among the majority of these thinkers, a disavowal that is, I think, symptomatic, to redeploy Althusser's famous term, rather than analytically coherent.} Though Balibar does not discuss slavery as a mode of production prior to capitalism, the key factors in the transition to capitalism he identifies in {\em Capital} can be readily applied to the preceding analysis of slavery in Marx's conceptual system.

The central analytical assertion of Balibar's complex, book-length chapter is, from this perspective, that the primary condition for the existence of the capitalist mode of production is a two-part legal distinction, \quotation{the basic elements of which are the {\em law of property} and the {\em law of contract.}} Under the capitalist mode of production, this division takes the form of a primary distinction between, firstly, the \quotation{{\em abstract universalistic character}\ldots{}of {\em human persons}}' who are entitled to enter into {\em contractual} relations (i.e., between the capitalist owner of the means of production and wage labor), and, on the other hand, the {\em property} relation that \quotation{is established exclusively between human persons and things (or what are {\em reputed} to be persons and what are {\em reputed} to be things).} \quotation{This system simply distributes,} Balibar observes, \quotation{the concrete beings which can support its functions into two categories within each of which there is no pertinent distinction from the legal point of view.}\footnote{Balibar, \quotation{On the Basic Concepts of Historical Materialism,} 392 (italics in original).}

While Balibar is here discussing only capitalism, it is immediately obvious that this criterion constitutes the founding, enabling distinction between slavery and capitalism as modes of production: if capitalism distinguishes between the \quotation{abstract universalistic character} of human persons as subjects of right (of the universal, inalienable right to the property of one's self-same person, as well as to the corresponding, temporally limited right to sell and purchase the commodity labor power), the legal system subtending slavery knows no such abstract universality. Under slavery, some human persons are rights-bearers legally entitled to enter into contracts with other property owners (to buy and sell slaves and other means of production), while other human persons (slaves) are considered property, disentitled from entering into legal contracts, subject to the purchase and use of their entire person as property. Balibar's analysis thus offers a formal definition of the discrete distinction differentiating these two modes of production, as opposed to any linear criterion of degree or relative extension. It constitutes a clear and distinct criterion (possessing, to be sure, manifold subdistinctions and limitations such as those to be found in the Code Noir), one that serves analytically to contrast in the absolute two distinct modes of production. \footnote{The analytical apprehension of the criteria distinguishing slavery from industrial capitalism such as I am proposing here does not ignore, but in fact necessarily precedes and enables, the obviously more complicated problem of the actual historical transition between these two modes of production (in colonial France, globally, etc.). While such an examination lies beyond the scope of this essay, it might begin, as I suggested above regarding the 1848 abolition process, by addressing historically the implications of the temporal asynchronicity and overdetermined complexity posited in Balibar's analysis of the problem of transition. See Alberto Toscano, \quotation{Transition Deprogrammed,} {\em SAQ} 113, no. 4 (2014): 761--75. For a rich account of the archival material describing the phenomenal unfolding of this transition in 1840s Guadeloupe in its multiple dimensions (economic, political, social, moral, etc.), see Fallope, {\em Esclaves et citoyens}.}

This founding {\em legal} distinction then allows for Balibar's subsequent analysis of the key internal factors composing and distinguishing the capitalist mode of production itself.\footnote{Note that as with Marx's {\em Capital} itself, the order of exposition of Balibar's analysis is analytical rather than empiricist or historicist. To proceed from the enabling legal framework for the capitalist mode of production to the key factors whose particular arrangement determines its nature is to follow the logic of a structural presentation with no isomorphic relation to the historical, empirical supersession of capitalism over various precapitalist modes of production such as feudalism and plantation slavery.} In a penetrating critique of the traditional Marxist notion of the development of the forces of production, Balibar argues that the development of the capitalist mode of production, when understood analytically rather than empirically, is radically discontinuous rather than linear and cumulative.\footnote{\quotation{Marx's concept cannot be made to coincide with the categories of a sociology which, for its part, proceeds by the distribution and adding together of levels---the technological, the economic, the legal, the social, the psychological, the political, etc.---and bases its peculiar historical classifications on these distributions (traditional societies and industrial societies, liberal societies and centralized-totalitarian societies, etc.)} (Balibar, \quotation{On the Basic Concepts of Historical Materialism,} 398).} In fact, he argues, the so-called \quotation{productive forces are not \quote{things} at all, but are instead to be conceived as a {\em relation} constituted by a pattern of arrangement and temporal \quote{rhythm} distinctive to the capitalist mode of production.}\footnote{Ibid. (emphasis mine).}

Balibar derives the nature of this \quotation{arrangement} from Marx's presentation of the concept of relative surplus value in three central chapters of {\em Capital}, volume 1, specifically, those focused on the contrasting modes of production Marx names manufacture and large-scale industry.\footnote{Marx, {\em Capital}, vol.~1, chaps. 13--15.} The essential transition from precapitalist manufacture to capitalist industry, in this view, does not consist in a gradual increase in scale and degree of co-operation, since both manufacture and industrial production require a complex division of labor that distinguishes them together from traditional handicraft (see fig. 3).\footnote{Adam Smith famously analyzed the development of the division of labor in manufacturing based on the example of pin manufacturing. See Adam Smith, {\em The Wealth of Nations} (1776; repr., New York: Penguin, 1999), 4--5.}

\placefigure[here]{\quotation{Aiguillier,} from Diderot et al., {\em Encyclopédie}, vol.~1, plates.}{\externalfigure[issue03/nesbitt-3.jpg]}


Instead, the crucial analytical distinction (and this will prove key as well, I will argue below, to distinguish the often highly developed forms of manufacture in plantation slavery from industrial capitalism) is that all the various \quotation{fractional operations} in the diverse, less- and more-complexly divided processes of manufacture---as with handicraft production---are still performed {\em manually}, through the manual manipulation, that is to say, of various tools by laborers.\footnote{Balibar, \quotation{On the Basic Concepts of Historical Materialism,} 401.} The analytical unity of handicraft and manufacture as modalities of the manual manipulation of tools constitutes, therefore, the element founding their common distinction from the capitalist (industrial) mode of production (fig. 4).

\placefigure[here]{\quotation{Aiguillier (detail),} from Diderot et al., {\em Encyclopédie}, vol.~1, plates.}{\externalfigure[issue03/nesbitt-4.jpg]}


In other words, handicraft and manufacture (including plantation slavery), distinguished only by their various {\em degrees} of division of labor, nonetheless consist in a more basic unity as modes of production in which living labor directly manipulates tools in the production process (see fig. 5).

\placefigure[here]{\quotation{Récolte de l'indigo,} from José Mariano da Conceição Velloso, {\em O fazendeiro do Brazil, cultivador} (Lisbon, 1806), vol.~2, plate 1, foldout following 341. Copy in the John Carter Brown Library, Brown University.}{\externalfigure[issue03/nesbitt-5.jpg]}


The gradual, cumulative increases in productivity observable in the transition from handicraft to manufacture derive from the rationalization process applied to each element of a given production process; rather than a rupture, Balibar writes, the passage from handicraft to manufacture thus occurs as a continuous \quotation{extension of the analytical movement of specialization peculiar to handicrafts, a movement which simultaneously affects both the perfection of technical operations and the psychophysical characteristics of the workers' labour-power. These are merely two aspects, two faces of one and the same development. Indeed, {\em manufacture is merely the extreme radicalization of the distinctive feature of handicrafts: the unity of labour-power and means of labour}.}\footnote{Ibid., 402 (emphasis mine).} The tendencies of scale, concentration, and division of labor observable in the plantation production processes of St.~Domingue, characteristics that James names \quotation{proletarian,} are, I would argue, analogous to the developments in Adam Smith's pin manufactures, inherent, that is to say, to \quotation{manufacture} rather than \quotation{large-scale industry} (in Marx's sense of the terms).

In contrast, Balibar argues, the crucial distinguishing shift from manufacture to fully capitalist industrial production is the suppression of the anthropological function of living labor as \quotation{tool-bearer} (see fig. 6).

\placefigure[here]{\quotation{Intérieur de la Sucrerie de betteraves de Château Freyes, près de Villeneuve-St.-Georges,} {\em L'Ilustration Journal Universel}, 13 May 1843.}{\externalfigure[issue03/nesbitt-6.png]}


The introduction of machinery and mechanized production suppresses the direct contact of labor with the object of labor (the commodity), and an entirely distinct criterion for the development of the production process replaces the (analytically) previous division and specialization of living labor via its manual manipulation of tools.\footnote{This is the process Marx analyzes in the crucial chapter 15 of volume 1, \quotation{Machinery and Large-Scale Industry.}} In its place, development under industrial production occurs as the tendential replacement and {\em elimination} of manual labor by machine production. Not only is production henceforth organized independently from the anthropological characteristics of human labor power and its rationalization, but development furthermore occurs as a separation of living labor from the means of labor (i.e., machines).\footnote{Balibar, \quotation{On the Basic Concepts of Historical Materialism,} 403.}

Instead of the unity of living labor and the means of labor (tools) characteristic of handicrafts and manufacture, an entirely distinct structural unity takes its place: \quotation{The unity of the means of labor and the object of labor.}\footnote{Balibar, \quotation{On the Basic Concepts of Historical Materialism,} 403. \quotation{An organism of production,} Balibar continues, \quotation{is now no longer the union of a certain number of workers, it is a set of fixed machines ready to receive any workers. From now on, \quote{a technique} is a set of certain materials and instruments of labour, linked together by a {[}general, social{]} knowledge of the physical properties of each of them, and of their properties as a system} (403).} Crucially, it is not the {\em historical} introduction of machine tools such as the steam engine that is at issue in Marx's categorial analysis (though of course he notes these empirical phenomena in passing) but the {\em theoretical expression} of this dislocation as the succession of two {\em forms of relation} characteristic of handicrafts/manufacture versus industry, a displacement, that is to say, from (1) the unity of the means of labor (tools) and living labor to (2) an entirely distinct unity of the means of labor (machines) and the object of labor (commodities).\footnote{Ibid., 406. And so, Balibar concludes, \quotation{The movement from one form to the other can be completely analysed: not as the mere dissolution of a structure (the separation of the worker from the means of labour {[}via so-called primitive accumulation{]}), but as the transformation of one structure into another\ldots{}.It is the forms of the labour process which have changed} (407; emphasis mine).}

This displacement, analytically rather than empirically apprehended, entails, Balibar concludes,

\startblockquote
a reorganization of the entire system, of the relation of the real appropriation of nature, of the \quotation{productive forces.}\ldots{}The machine which replaces the ensemble of tools and educated, specialized labor-power is in no way a product of the development of that ensemble. It replaces the previous system by a different system: the continuity is not that of elements or individuals, but of functions\ldots{}.The subject of development is nothing but what is defined by the succession of the forms of organization of labour.“\footnote{Ibid., 407, 412. Balibar has recently offered an autocritique of the structural logic of modes of production developed in his seminal 1965 text: \quotation{My own text in {\em Reading Capital} was---at least this is how I see it today---a perfect example of a blind alley into which a blind use of the concept of structure was directed.} The object of Balibar's self-criticism is what he now sees as an indiscriminate deployment of two contradictory senses of structure, that of the social formation as a \quotation{structured totality} possessing in any historical case a particular and characteristic \quotation{invariant,} combined with a contradictory attempt to think structure as \quotation{the description of the transformation from one figure to the second,} or, in other words, to articulate \quotation{the laws of the transformation} or transition between modes of production. In the present argument, needless to say, I do not fully accept Balibar's autocritique, with the essential proviso that the analytical determination of the actually distinctive elements in such a comparison (legal, economic) do not constitute an end point but only the very beginning, a mere but essential prolegomenon to the examination of the complexly overdetermined and variegated phenomenon of the transition from plantation slavery to industrial capitalism. Etienne Balibar, \quotation{Theory and Politics in the Thought of Louis Althusser: An Interview with Etienne Balibar by Petr Kužel,} 2017; courtesy of Petr Kužel.}
\stopblockquote

The point to be taken from this excursus on the structural logic of the capitalist mode of production is not merely that the displacement of plantation slavery by industrial capitalism is not to be understood, as James argues, as a gradual, linear process of complexification of the division of labor and numerical expansion of its labor force (historically real but irrelevant factors, in this view, in distinguishing these two modes of production).\footnote{That American plantation slavery became increasingly productive of cotton due to various proto-Taylorist, rationalizing transformations in labor practices and their analysis in no way implies, in this light, that it consequently became increasingly \quotation{capitalist} in nature, as Rosenthal claims, since (in light of Rosenthal's explicit refusal to offer a coherent definition of capitalism), in the terms of Marx's analysis, the production of use values such as cotton, unlike surplus value, is a transhistorical anthropological practice in no way specific to capital ({\em Accounting for Slavery}, 5--6).} Moreover, in the terms of Marx's categorial analysis, this distinction can only be adequately apprehended analytically, as a discrete {\em displacement} from one relation of factors to another.\footnote{\quotation{A scientific analysis\ldots{}is possible only if we can grasp the inner nature of capital, just as the apparent motions of the heavenly bodies are intelligible only to someone who is acquainted with their real motions, which are not perceptible to the senses} (Marx, {\em Capital}, 1:433).}

Such an analysis ultimately points, then, to the inherent {\em necessity} of this displacement as industrial capitalism and the value-form became the predominant structural logic of Atlantic societies. One of the principal conclusions Marx drew from his critique of political economy is that capital attains its fully developed form only with the real subsumption of labor and the predominance of the extraction of relative rather than absolute forms of surplus value---a process manifest historically as the supplanting of manual labor by automatic machinery.\footnote{\quotation{The production of relative surplus value completely revolutionizes the production process of labour and the groupings into which society is divided\ldots{}.It then becomes the universal, socially predominant form of the production process} (Marx, {\em Capital}, 1:645--46). See also Postone, {\em Time, Labor, and Social Domination}, 283--84.} From this it is clear that the manual manipulation of tools by slaves, like the use of horses, mules, and wind power---no matter what the degree of division of labor, accumulation of labor power, or intensity and rate of production---must be displaced by machinic, automated production processes subject to continuous revolutions in productivity, in order to realize the inherently infinite demand of capital for ceaseless increases of surplus value. This is to say, that the living labor of horses, mules, and slaves can be pushed only to certain absolute limits in intensity and duration of the working day (what Marx termed the extraction of {\em absolute} surplus value); the elimination of animal labor, and the displacement of living human (slave) labor that Marx and Balibar describe, are in this view essential, necessary moments in the development of capitalism to its fully adequate, industrial form.

More than mere semantics, the debate over the relation of plantation slavery to the rise of capitalism remains in question today.\footnote{See, for example, the excellent review article by Charles Post, \quotation{Slavery and the New History of Capitalism,} {\em Catalyst} 1, no. 1 (2017), \useURL[url4][https://catalyst-journal.com/vol1/no1/slavery-capitalism-post]\from[url4]. See also Seymour Drescher, \quotation{Capitalism and Slavery after Fifty Years,} {\em Slavery and Abolition} 18, no. 3 (1997): 212--27.} In their unparalleled originality and critical force, both Marx's {\em Capital} and James's {\em The Black Jacobins} remain as relevant as ever to this discussion. If James was arguably less engaged in the systematic critique of capitalism than its revolutionary overthrow across the global south, to read these two texts together can provoke a multiplication of their analytical and rhetorical powers.

As such, the preceding analysis implies that the entire debate over the transition from slavery to capitalism since Eric Williams, whether for or against and for all its real historical insights, has been in a certain sense misguided. This is arguably so insofar as the a priori question at stake---whether the wealth produced by plantation slavery enabled the initiation of and transition to capitalism---is incapable of furnishing an adequate response to the debate. The production of various forms of wealth and their social distribution is a transhistorical anthropological constant, as such irrelevant to the distinction between these two social forms. Wealth, moreover, is a mere superficial form of appearance of capital, Marx argues, incapable of accounting in its quantification for the analytical and historical distinction between these two modes of production. To argue otherwise is to reject the most basic insights of Marx's critique---of so-called primitive accumulation and the basic distinction between the dual forms of value (use-value and exchange-value) and of labor (living-labor and abstract)---and to revert to mere left Ricardianism. It is to assert the mere negative flipside of the tired myth of primitive accumulation: in the place of the good, Calvinist protocapitalists saving and reinvesting their profits to jumpstart industrial capitalism, one instead blames or exculpates the bad slave owners. Primitive accumulation, Marx showed, addressed instead the forcible, often violent replacement of one juridical structure and accompanying mode of production with another.~

The slaves of St.~Domingue were no proletariat in the modern senses Marx has given the term. Plantation slavery certainly contributed enormous wealth to the Atlantic North, wealth that allowed for the imposition of the capitalist social form, but in itself it did not constitute a form of primitive accumulation and remained till its destruction analytically distinct from the industrial capitalist mode of production. Slavery in St.~Domingue was modern, all too modern, only in the scale of its dehumanizing brutality and violence.

C. L. R. James was among the most original writers and thinkers of revolutionary Marxism. Decades ahead of his time, he drew the consequences of his penetrating historical analyses of the Haitian and Bolshevik Revolutions to argue that African Americans constituted a revolutionary class; as he wrote in a 1939 resolution to the Socialist Worker's Party, African Americans \quotation{are designated by their whole historical past to be, under adequate leadership, the very vanguard of the proletarian revolution.}\footnote{C. L. R. James, quoted in Paul Le Blanc, introduction to McLemee and Le Blanc, {\em C. L. R. James and Revolutionary Marxism}, 5. This is the primary conclusion James advances in his 1939 text \quotation{Revolution and the Negro,} reproduced in the same text: \quotation{What we as Marxists have to see is the tremendous role played by Negroes in the transformation of Western civilization from feudalism to capitalism} (77). Similarly, in the 1948 essay \quotation{The Revolutionary Answer to the \quote{Negro Problem} in the United States,} also reproduced in {\em C. L. R. James and Revolutionary Marxism}, James writes, \quotation{{[}The{]} independent Negro movement\ldots{}is in itself a constituent part of the struggle for socialism\ldots{}.The Negro people, we say, on the basis of their own experiences, approach the conclusions of Marxism} (180).} While James was a radically inventive and visionary political thinker, he analyzed and presented the problem of the transition from feudalism and plantation slavery to industrial capitalism not via Marx's structural analysis of the forms, factors, and tendencies of the capitalist mode of production, but in vital, politico-historical terms: \quotation{What are the decisive dates in the modern history of Great Britain, France, and America? 1789, the beginning of the French Revolution; 1832, the passing of the Reform Bill in Britain; and 1865, the crushing of the slave-power in America by the Northern states. Each of these dates marks a definitive stage in the transition from feudal to capitalist society.}\footnote{James, \quotation{Revolution and the Negro,} 78.} When James did speak of the \quotation{scientific} analysis of history, as, for example, in his 1940 article \quotation{Trotsky's Place in History,} he did so as an affirmation of an empiricist methodology: \quotation{Trotsky claimed and irrefutably demonstrated that his history was scientific in that it flowed from the objective facts. He challenged anyone to question his documentation.}\footnote{C. L. R. James, \quotation{Trotsky's Place in History,} in McLemee and Le Blanc, {\em C. L. R. James and Revolutionary Marxism}, 123.}

For all the immense accomplishments of Trotsky's and James's famous works, they remain linear, narrative histories of the events they present, entirely distinct, methodologically and analytically, from Marx's categorial presentation of the underlying social forms, relations, and structures of the capitalist mode of production. For the most part, the aims and accomplishments of {\em The Black Jacobins} and {\em Capital} remain entirely distinct. The specific problem of the transition from feudal slavery to capitalism is arguably, however, impossible to adequately treat from the perspective of a linear narrative description of a political history that James adopts, and demands instead Marx's properly scientific mode of analysis and his narrative presentation of a conceptual logic in a discursive form utterly distinct from the linearity of historiographic inquiry.\footnote{\quotation{For all their limits,} writes Alberto Toscano, \quotation{the semi-structuralist concepts forwarded by Balibar to ground Marx are substantial antidotes to a thinking of transition as a homogeneous \quote{expressive} totality in historical development} (\quotation{Transition Deprogrammed,} 765).} What is more, the logic of {\em Capital} itself must be conceived differentially, as a dominant order of logical exposition (of the value-form) in overdetermined tension with other, heterogeneous orders that \quotation{interrupt and cut across the first,} delimiting and marking off the contingent necessity of Marx's analytical focalization on the social logic of valorization.\footnote{On this point, see Althusser's long-neglected but essential reformulation of the problem of Marx's methodology in his 1977 preface to Gérard Duménil's {\em Le concept de loi économique dans \quotation{Le Capital.}} The preface, translated by G. M. Goshgarian, appears in {\em Rethinking Marxism} 30, no. 1 (2018): 4--24. While in this late text Althusser develops a brilliant interpretation of Marx's methodology and the contingency of the latter's analytical focus on valorization in {\em Capital}, this arguably culminates in Althusser's problematic (not to say symptomatic) hypostatization of the categories \quotation{use value, labour productivity, and---class struggle! {[}{\em sic}{]}} as heterogeneous to, rather than constituted by, the logic and social demands of valorization (20). For all its brilliance, in other words, Althusser's remains a critique from the standpoint of labor rather than a critique of (capitalist) labor per se. See also Fabio Bruschi, \quotation{Splitting Science: The Althusserian Interpretation of {\em Capital}'s Multiple Orders of Exposition,} {\em Rethinking Marxism} 30, no. 1 (2018): 25--43.}

The unparalleled brilliance of {\em The Black Jacobins} lies elsewhere.

As James was the first to show, the black Jacobins of the Haitian Revolution were, if anything, even more modern, more radically, uncompromisingly modern than the Robespierreist French Jacobins who initiated the destruction of late feudalism and the world-historical imposition of justice as equality.\footnote{See Nick Nesbitt, {\em Universal Emancipation: The Haitian Revolution and the Radical Enlightenment} (Charlottesville: University of Virginia Press, 2008), and {\em Caribbean Critique: Antillean Critical Theory from Toussaint to Glissant} (Liverpool: Liverpool University Press, 2015), chap. 1.} James's history of their unyielding will to destroy slavery and to impose, by any means necessary, universal emancipation from its misery remains one of the great narrative analyses of leftist revolutionary history, equaled in originality of insight and rhetorical force perhaps only by Marx's own 1871 visionary history of the Commune and the incipient actuality of communism, {\em The Civil War in France.}

To pair James and Marx is not to play one off against the other in a zero-sum critical ploy but rather to initiate, productively, a more complexly wrought interrogation of the transition from slavery to capitalism, one in which the analysis of these two discrete modes of production enables a more encompassing critique of the discrepant multiplicity of any mode of production in transition. Transition, in this view, is not to be grasped as the idealist metamorphosis of the structure (the structure is, as Marx famously said, not the materialist real itself but its {\em reproduction} as a \quotation{thought-concrete} {[}{\em Gedankenkonkretum}{]}).\footnote{Marx, {\em Grundrisse}, 101. On Marx's methodology and his essential distinction between the real object of analysis and its reproduction as a \quotation{thought-concrete} in the analysis of {\em Capital}, see Iñigo Carrera, \quotation{Method.}} Instead, the preceding analysis can hope to initiate consideration of transition as the complex \quotation{recombination of elements} in what Althusser famously named {\em décalage}, or noncorrespondence, the historical passage between the various divergent forms and temporalities of the legal, economic, productive and reproductive, racial, gendered, and political components of the historical transition from slavery to capitalism.

\thinrule

\page
\subsection{Nick Nesbitt}

Nick Nesbitt is a professor of French at Princeton University. Most recently, he is the author of {\em Caribbean Critique: Antillean Critical Theory from Toussaint to Glissant} (Liverpool University Press, 2013) and the editor of {\em The Concept in Crisis: Reading \quotation{Capital} Today} (Duke University Press, 2017).

\stopchapter
\stoptext