\setvariables[article][shortauthor={Daut}, date={July 9 2019}, issue={3}, DOI={10.7916/archipelagos-53xt-3v66}]

\setupinteraction[title={Haiti @ the Digital Crossroads: Archiving Black Sovereignty},author={Marlene L. Daut}, date={July 9 2019}, subtitle={Haiti @ the Digital Crossroads}, state=start, color=black, style=\tf]
\environment env_journal


\starttext


\startchapter[title={Haiti @ the Digital Crossroads: Archiving Black Sovereignty}
, marking={Haiti @ the Digital Crossroads}
, bookmark={Haiti @ the Digital Crossroads: Archiving Black Sovereignty}]


\startlines
{\bf
Marlene L. Daut
}
\stoplines


{\startnarrower\it In the spirit of Papa Legba (a Haitian lwa who is the arbiter of the crossroads between the human and nonhuman worlds), this essay examines the challenges and opportunities presented when using a digital humanities approach to archiving early Haitian sovereignty, a critical but often forgotten part of the story of the making of the modern world-system. Abdul JanMohamed and David Lloyd have written about \quotation{archival work, as a form of counter-memory} that is \quotation{essential to the critical articulation of minority discourse.} However, because archives, like other kinds of texts, reflect the worldview of their creators, the archivist working to articulate \quotation{minority discourse} must be careful not to reproduce patterns of domination or cultural exploitation. For Haiti, this means that we must work against the idea that the abundant historical resources now made readily (and often freely) available by various digitization projects, represent a \quotation{new frontier} for research, an idea that encourages the notion that the country is \quotation{open for business} on a variety of levels. Instead, by using the metaphor of the crossroads, this essay demonstrates how a multimodal approach---involving content, context, collaboration, and access---can allow for alternative ways of (humanely) archiving black sovereignty.

 \stopnarrower}

\blank[2*line]
\blackrule[width=\textwidth,height=.01pt]
\blank[2*line]

{\em Dedicated to the memory of Denise Groce (1958--2018)}

\subsection[title={Introduction},reference={introduction}]

There are two major poles in contemporary discourse about Haiti: what Robert Lawless has called \quotation{Haiti's bad press,} on the one hand, and the resonant, opposing concept of \quotation{new narratives,} called for by scholars such as Gina Athena Ulysse, Kaiama L. Glover, and Laurent Dubois, on the other.\footnote{Robert Lawless, {\em Haiti's Bad Press} (Rochester, NY: Schenkman, 1992). For theorization of the concept of \quotation{new narratives,} see \quotation{New Narratives of Haiti,} a 2013 special edition of the journal {\em Transitions} (no. 111), guest edited by Laurent Dubois and Kaiama L. Glover. See also Gina Athena Ulysse, {\em Why Haiti Needs New Narratives: A Post-Quake Chronicle} (Middletown, CT: Wesleyan University Press, 2015).} Those of us in the field of Haitian studies immediately understand the meaning of \quotation{bad press} for Haiti when we see it, especially when it contains what Joel Dreyfuss has aptly called \quotation{the phrase.}\footnote{I have consciously chosen not to include \quotation{the phrase} itself to avoid even unwittingly reproducing bad press for Haiti. See Joel Dreyfuss, \quotation{A Cage of Words,} {\em Haitian Times}, November 1999; reprinted at {\em Bob Corbett Listserv}, \useURL[url4][http://faculty.webster.edu/corbetre/haiti-archive/msg01471.html]\from[url4].} It is much less clear, however, how we might go about generating \quotation{new narratives} in order to combat what Ulysse has referred to as the persistence of \quotation{Haiti's image problem.}\footnote{Ulysse, {\em Why Haiti Needs New Narratives}, 30.} In other words, how do we ensure that the new or alternative narratives we encourage, craft, and center to oppose bad press will in fact not generate similarly, or perhaps newly, hostile and harmful images of Haiti?

The British Museum's April 2018 exhibit {\em A Revolutionary Legacy: Haiti and Toussaint Louverture} provides a concrete example of how attempts to create new narratives about Haiti could end up unwittingly reinforcing some of the same old narratives commonly found in Haiti's bad press. The long awaited and heavily advertised exhibit included portraits of Toussaint by Jacob Lawrence and Lubaina Himid, among others, and excerpts of writings about the famous revolutionary leader by C. L. R. James. There was also audio of Ulysse's highly celebrated one-woman show, {\em Because When God Is Too Busy}. Charles Forsdick, in a blog post dedicated to the exhibit, described it as seeking to showcase Toussaint as one of the world's most important historical figures: \quotation{Louverture's increasing visibility as a global revolutionary figure suggests that he is achieving a transcultural iconic status rivalled only by Che Guevara. Yet this presents clear challenges: does it imply the final stage in what some see as Louverture's conscription to forms of neo-colonial modernity; or does it contain the residual potential for reignition, in the present, of the Haitian revolutionary's struggle for universal emancipation, not least in Haiti itself?}\footnote{Charles Forsdick, \quotation{Visualising Toussaint Louverture,} {\em British Museum Blog}, 12 March 2018, \useURL[url5][https://blog.britishmuseum.org/visualising-toussaint-louverture/]\from[url5].}

Yet in focusing on representations of Toussaint largely produced outside Haiti, the exhibit, as Tabitha McIntosh observed, placed undue attention to the {\em idea} of Toussaint in the \quotation{European imagination.} Primary real estate in the exhibit room was given over, for instance, not to the way Toussaint represented himself in his extremely influential memoirs but to the British poet William Blake.\footnote{See Tabitha McIntosh, \quotation{A Revolutionary Legacy: Haiti and Toussaint Louverture, Room 3, British Museum, 22 February--22 April 2018,} {\em H-Haiti}, \useURL[url6][https://networks.h-net.org/node/116721/discussions/1686456/revolutionary-legacy-haiti-and-toussaint-louverture-room-3]\from[url6]. For Toussaint's memoirs, see {\em Mémoires du général Toussaint Louverture}, ed.~Daniel Desormeaux (Paris: Classiques Garnier, 2011).} Along with the erasure of Toussaint as one of the authors of Haitian revolutionary discourse itself, nineteenth-century Haitian writers who wrote about the revolutionary leader---namely, Baron de Vastey, Thomas Madiou, Beaubrun Ardouin, and, most importantly, Joseph Saint-Rémy---were not included in the exhibit either. The reliance on foreign contextualization for an exhibit supposedly devoted to rethinking how the world might better understand contemporary Haiti through the legacy of Toussaint left McIntosch wondering, \quotation{Is the display about the revolutionary legacy of Haiti, or Haiti in the American twentieth-century imagination, or Louverture as a revolutionary trope for global black nationalism?}\footnote{McIntosh, \quotation{A Revolutionary Legacy.}}

Questions about the aim of the exhibit only continue to accumulate when we consider the unfortunate way much of the history of Haiti after the Revolution was both occluded and erased from the very parts of the display meant to provide historical context. King Henry Christophe, for instance, along with nearly every Haitian ruler from the years between 1825 and 1915, was not included in the timeline of Haitian history presented at the museum. The timeline, a photo of which is included below (see fig.~1), leaps from Haiti's independence from France in 1804, with Jean-Jacques Dessalines, to the year 1825, which is the moment when then Haitian president Jean-Pierre Boyer signed a treaty agreeing to pay an indemnity to France's King Charles X, to the US occupation of the country in 1915.

\placefigure[here]{Close up of museum timeline showing the gap.}{\externalfigure[issue03/daut-a.jpg]}


This kind of elision, and indeed suppression, has the effect of framing Haitian history almost entirely around its relationship to Europe and the United States. What matters on this timeline is a break from colonial rule, a treaty with France, and occupation by the United States. The image of Haiti presented, therefore, despite the exhibit's lofty aims, is the portrait of a country overdetermined by its past colonial relationship to France and its future/current neocolonial subjugation in the US and European world-systems. Indeed, the rest of the timeline narrates Haiti as a place of authoritarianism and military dictatorship that was simultaneously dominated by the Western powers (see fig.~2).

\placefigure[here]{Long view of the timeline.}{\externalfigure[issue03/daut-b.jpg]}


The problem with forgetting Haiti's long nineteenth century is that this time period was, in many respects, the last time Haiti saw generally good press. By good press I do not mean analyses that refrain from criticism; I mean press that acknowledges a priori that Haiti has a right to exist as a sovereign and independent state.

There is a Haitian proverb that states, {\em Se vye chodye ki kwit bon manje}, or, \quotation{It is the old pot that makes the best food.}

Although nineteenth-century French, British, and US authors often indirectly promoted and sometimes even directly argued for the recognition of Haitian sovereignty after the Revolution,\footnote{Marlene L. Daut, \quotation{The \quote{Alpha and Omega} of Haitian Literature: Baron de Vastey and the US Audience of Haitian Political Writing, 1807--1825,} in Elizabeth Maddock Dillon and Michael J. Drexler, eds., {\em The Haitian Revolution and the Early United States: Histories, Textualities, Geographies} (Philadelphia: University of Pennsylvania Press, 2016), 287--313.} this nineteenth-century Haiti---the one not isolated by the Western world---has remained relatively understudied by the vast majority of North Atlantic scholars, researchers, and students. Until very recently much of the scholarship on Haiti produced in US universities has traditionally concerned itself with US and Western European reactions to and readings of the Haitian Revolution and especially with what Mimi Sheller has called the \quotation{Haytian fear.}\footnote{Mimi Sheller, \quotation{\quote{The Haytian Fear}: Racial Projects and Competing Reactions to the First Black Republic,} {\em Politics and Society} 6 (1999): 286.} Overburdened attention to Atlantic world anxieties about the meaning and consequences of the Haitian Revolution for the colonial powers (England, Spain, Portugal, France, the United States, and the Netherlands) has come most often at the expense of analyzing nineteenth-century Haitian interpretations of the history of the Atlantic world, Haitians' own reactions and responses to US and European policies of nonrecognition of their sovereignty, and the contributions of Haitian historians to the field of Haitian revolutionary historiography itself. For example, even though the {\em Revue de la Société Haïtienne d'Histoire, de Géographie et de Géologie}, which was founded in Port-au-Prince in 1925, is without a doubt the richest source of Haitian-produced historiography about both the nineteenth and twentieth centuries, the more than thirteen hundred articles the journal contains are rarely cited by US scholars.\footnote{See Chelsea Stieber, \quotation{Why the Lab?,} RSHHGG Lab, \useURL[url7][http://rshhgglab.com/why-a-lab/]\from[url7]. Other notable works dealing with the nineteenth century and published by Haitians are Michel-Rolph Trouillot, {\em Haiti: State Against Nation; The Origins and Legacy of Duvalierism} (New York: Monthly Review, 1990); Patrick Bellegarde-Smith, {\em Haiti: The Breached Citadel} (Boulder, CO: Westview, 1990); and Alex Dupuy, {\em Haiti in the World Economy: Class, Race, and Underdevelopment since 1700} (Boulder, CO: Westview, 1989). More recently, see Michel Hector and Laënnec Hurbon, eds., {\em Génèse de l'État haïtien (1804--1859)} (La Rochelle: Éditions de la Maison des Sciences de l'Homme, 2009); and Jean Casimir, {\em Une lecture décoloniale de l'histoire des Haïtiens: Du Traité de Ryswick à l'occupation américaine} (Port-au-Prince: L'Imprimeur SA{\em ,} 2018). See also, from non-Haitian writers, Julia Gaffield, {\em Haitian Connections in the Atlantic World: Recognition after Revolution} (Chapel Hill: University of North Carolina Press, 2015); Laurent Dubois, {\em Haiti: The Aftershocks of History} (New York: Henry Holt, 2012); Kate Ramsey, {\em The Spirits and the Law: Vodou and Power in Haiti} (Chicago: University of Chicago Press, 2011); and Matthew J. Smith, {\em Liberty, Fraternity, Exile: Haiti and Jamaica after Emancipation}. (Chapel Hill: University of North Carolina Press, 2014).}

Without delving into Haitian-produced sources about Haiti, we understand comparatively little about what happened in between the Revolution and the occupation and are therefore incapable at present of fully thinking about the kind of Haiti where Haitians appear as people, individuals, with political dreams and desires all their own. In the words of Robin D. G. Kelley, today's Haiti remains in need of new narratives precisely because much of the contemporary discourse fails to represent the Haitian people as \quotation{subjects and agents, as complex human beings with desires, imaginations, fears, frustrations, and ideas about justice, democracy, family, community, the land, and what it means to live a good life.}\footnote{Robin D. G. Kelley, foreword to Ulysse, {\em Why Haiti Needs New Narratives}, xiii--xiv.}

My goal in this essay, therefore, is to return to the histories, legacies, and interpretations of nineteenth-century Haiti written by Haitian authors, politicians, and journalists. In so doing, I claim less that scholars should now to try to reproduce Haiti's good press of the nineteenth century than to merely document what we can all learn about Haitian sovereignty from consulting it. In order for scholars in the North Atlantic to go beyond simply {\em mentioning} Haiti as a pit stop on the way to extra-national interests and imperatives, we must, as Chelsea Stieber has recently urged, investigate the multiple aspects of the way nineteenth-century Haitian authors and politicians represented themselves to the world after their storied revolution.\footnote{See Chelsea Stieber, \quotation{Beyond Mentions: New Approaches to Comparative Studies of Haiti,} {\em Early American Literature} 53, no. 3 (2018): 974.} While it is true that the mere fact of consulting nineteenth-century Haitian archives---as equally as reading contemporary Haitian scholars---will not necessarily prevent us from contributing to bad press,\footnote{Ibid., 971; see also Michel-Rolph Trouillot, \quotation{The Odd and the Ordinary: Haiti, the Caribbean, and the World,} {\em Cimarrón} 2, no. 3 (1990): 7.} what turning to nineteenth-century Haitian historians and other writers from sovereign Haiti firmly demonstrates is that the annals of Haiti's history do not contain empty pages waiting to be filled by artists and scholars from abroad.

In the following sections of this essay, first evoking the image of the Haitian crossroads, guarded by Papa Legba, and then concentrating on the concepts of access, content, and context, I examine several recent online projects on Saint-Domingue and Haiti. In so doing, I consider how digital scholarship is currently contributing to our ability to tell more capacious and nuanced stories from, of, and about a sovereign Haiti. I subsequently suggest that using multimodal archival methods to document the Haitian past can help us to better understand the political processes that drove Haitians to create and sustain an independent nation-state in the nineteenth century.

In the final section of this essay, devoted to thinking about the concept of digital collaboration as {\em konbit}, I turn directly to Haiti's voluminous nineteenth-century print culture and specifically, the two official newspapers from the administration of Henry Christophe (1807--20), the {\em Gazette Officielle} and {\em Gazette Royale d'Hayti}. By rooting my interpretation of early Haiti's newspapers in Haitian intellectual history, I suggest that postrevolutionary Haitian political and historical writing provides vital clues for understanding how sovereignty in Haiti unfolded (and eventually fell apart) after the Revolution.

Ultimately, in thinking about how to use digital archival methods to develop a Haitian-informed digital praxis---centered on access, content, context, and collaboration---I hope to put forward alternative ways of approaching the Haitian past as a living history, one that has the ability to, in Karen Salt's words, \quotation{highlight\ldots{}black politics alongside and within the structures of power that attempt to deny the presence and political agency of black bodies.}\footnote{Karen Salt, \quotation{The Language of Politics in the Literary Archive of Black Sovereignty,}~{\em J19}~3, no. 2 (2015): 394.}

\subsection[title={{\em Kalfou}},reference={kalfou}]

For practitioners of Vodou, the {\em lwa} Papa Legba (Attibon Legba) is the god of the {\em kalfou}, or \quotation{crossroads} who acts as an arbiter between the human beings of the living world and the gods of the spirit world. Legba is also traditionally known as the \quotation{opener of gates} and therefore of opportunities. Vodou songs routinely implore, \quotation{Papa Legba, louvri barrye pou mwen}---open the gate for me. In the first history of the Haitiian Revolution written entirely in Haitian Kreyòl, {\em Ti dife boule sou istwa Ayiti}, Michel-Rolph Trouillot gave chapter 5 the iconic words of Papa Legba as a title, \quotation{Louvri barye.}\footnote{Michel-Rolph Trouillot, {\em Ti dife boule sou istwa Ayiti}, ed. Lionnel Trouillot (Port-au-Prince: Edisyon KIK, 2012). Originally published in 1977 as {\em Ti difé boulé sou istoua Ayiti} (Brooklyn, NY: Kóleksion Lakensièl); all quotes are from the 2012 edition.} This simultaneously secular and sacred evocation of Vodou practice was used by Trouillot to describe the process by which the French state came to officially abolish slavery in colonial Saint-Domingue in 1793/94. Trouillot asks of Léger-Félicité Sonthonax's 1793 decree abolishing slavery, \quotation{Why did he liberate the slaves? What could have brought him to such a decision?}\footnote{\quotation{Poukisa li lebere esklav yo? Kouman l fè te kapab pran desizyon sila a?}; Trouillot, {\em Ti dife boule}, 51.} For Trouillot, walking through the history of how the enslaved Africans of Haiti, and specifically Toussaint Louverture, had set the conditions for Sonthonax's revolutionary decision was about opening the gates to an entire world of thought, one that would show how the Haitian past has informed, shaped, and created the Haitian present. In the introduction Trouillot explains, \quotation{In order for us to understand {[}the early states of Haiti and{]} our own society, we must come to understand what kind of life faded into the bush\ldots{}. In order for us to understand the disease from which we are suffering, we must know what kind of diseases course through our blood.}\footnote{\quotation{Pou n komprann sosyete sa a, sosyete pa nou an, fò nou rive konnen ki kalite lavi ki disparèt nan raje {[}\ldots{}{]} Pou nou resi konn ki maladi n soufri, fò nou rive konnen ki maladi k pase nan san n}; ibid., 8.}

Thinking about digital historical study as a process of entering Legba's crossroads, the place where the living and the dead, the past and the present must meet, is important not simply because it provides language that draws on Haitian intellectual history (rather than European or US intellectual traditions) but also because, as Papa Legba's {\em vèvè} pictured below demonstrates (see fig.~3), these crossroads are not entirely straightforward. Not every path can be taken without a great leap, and the lines and circles are isolated yet connected, drawn together yet existing apart. The center of the crossroads is not the most important place to arrive at either, or rather it might not be the final destination. If \quotation{there is no center,}\footnote{See George Lang, \quotation{{\em No hay centro}: Postmodernism and Comparative New World Criticism,} {\em Canadian Review of Comparative Literature / Revue Canadienne de Littérature Comparée} 20, nos. 1--2 (1993): 105--24.} this is because every place is the center for someone or something, an idea rooted in Haitian historians' conceptions of what I have elsewhere referred to as \quotation{\useURL[url8][https://s3.amazonaws.com/uploads.knightlab.com/storymapjs/ba15390b4fd1e7d29a79c367f2799f1f/transnational-african-american-text-network/draft.html][][the Haitian Atlantic]\from[url8].}\footnote{See Marlene L. Daut, {\em The Haitian Atlantic: A Literary Geography}, \useURL[url9][http://s3.amazonaws.com/uploads.knightlab.com/storymapjs/ba15390b4fd1e7d29a79c367f2799f1f/transnational-african-american-text-network/draft.html]\from[url9].}

\placefigure[here]{Vèvè of Papa Legba}{\externalfigure[issue03/daut-c.jpg]}


If Legba opens the gates so that the living can have access to the dead, this is not solely for the purposes of remembrance, and it is certainly not for co-optation of their voices. Rather, it is so we might engage the dead in conversation. In {\em The Reaper's Garden}, Vincent Brown examined \quotation{how the dead affected the history of the living.}\footnote{Vincent Brown, {\em The Reaper's Garden: Death and Power in the World of Atlantic Slavery}. (Harvard University Press, 2010), 6.} Now, drawing on nineteenth-century Haitian sources, I ask different but related questions: How do we talk to the dead rather than solely about them? Can digital humanities methods both literally and metaphorically facilitate conversations between the living and the dead and in so doing aid scholars in defining new methods for doing what is very old work---archiving?

In a sense, all historians are talking to the dead when they use archival sources to study the pre-twentieth century.\footnote{See Marc Bloch, {\em The Historian's Craft} (New York: Vintage, 1953), 47.} But it is not always clear what questions we should pose to or about the fragments of these real lives resting in the archives. As Arlette Farge has written, \quotation{The archives {[}are{]} not lacking, but they created a void and emptiness that no amount of academic study can fill. Today, to use the archives is to translate this incompleteness into a question, and this begins by combing through them.}\footnote{Arlette Farge, {\em The Allure of the Archive} (New Haven, CT: Yale University Press, 2015), 55.} Alessandra Benedicty, evoking the works of Ann Stoler, has similarly sought to bring attention to the \quotation{relationship, the codependency even, between on the one hand, the ideas, texts, and objects that we put into our archives; and, on the other hand, the questions that we ask.} Benedicty continues, \quotation{What is important\ldots{}is that we pose questions that are worth asking; and to do so we must put into our archive, objects, books, and ideas, that help us not so much to resolve the uncertainties that lead to the formulation of our questions, but rather that we construct our archives to help us better invite questions that interrogate the urgencies of our present human condition.}\footnote{Alessandra Benedicty, \quotation{Questions We Are Asking: Hegel, Agamben, Dayan, Trouillot, Mbembe, and Haitian Studies,} {\em Journal of Haitian Studies} 19, no. 1 (2013): 7.}

In {\em Une lecture décoloniale de l'histoire des haïtiens}, Jean Casimir uses the concept of the \quotation{counter-plantation,} derived from the readings of Baron de Vastey, Beaubrun Ardouin, Thomas Madiou, and Anténor Firmon, precisely to ask questions about \quotation{how Haitians have survived, subsisted, and lived in the midst of political structures that exclude all participation on their part.}\footnote{\quotation{Je cherche à appréhender comment les Haïtiens arrivent à exister, à subsister et à vivre au sein de structures politiques qui excluent toute participation de leur part}; Casimir, {\em Une lecture décoloniale de l'histoire des Haïtiens}, 38.} His thesis, in effect, is that the majority of the Haitian people, whose sovereignty has become the victim of a world order based on capitalism, have created their own vision of the world in negotiation with dominant governments, including their own. Similarly, what I hope that attending to the myriad forms of documentation left behind by nineteenth-century Haitians, and especially to that found in early-nineteenth-century Haitian newspapers under the government of Henry Christophe, will allow me to do is both to put \quotation{better objects}---Haitian-produced objects---into my archive and to ask \quotation{better questions,} namely, how did the early sovereign period of postindependence affect Haiti's long march toward sovereignty against the Atlantic World powers?

\subsection[title={Access},reference={access}]

We have arguably never found ourselves in a better position to think about Haiti's development and contribution to the history of Atlantic sovereignty, broadly speaking. In 1983, David Geggus published \quotation{Unexploited Sources for the History of the Haitian Revolution,} an article which focused on describing the \quotation{vast quantity of neglected manuscript material {[}in Spain, Great Britain, the Caribbean, and the United States{]} that concerns} what he calls \quotation{this unique and profound event.}\footnote{David Geggus, \quotation{Unexploited Sources for the History of the Haitian Revolution,} {\em Latin American Research Review} 18, no. 1 (1983): 95.} Archival recovery work of the stripe that Geggus promoted so long ago is as equally critical to preserving, learning from, and contributing to the history of Haitian independence as it has been for Haitian revolutionary studies. Nineteenth-century Haitian print culture is extensive, encompassing not just traditional historical writing and the customary political documents found in any state but also novels, poetry, plays, music, artwork, and journalism.\footnote{For a resumé of the vastness and plurality of Haitian intellectual production, see Jacquelin Dolce, Gérard Dorval, and Jean Miotel Casthely, {\em Le romantisme en Haïti: La vie intellectuelle, 1804--1915} (Port-au-Prince: Editions Fardin, 1983).}

Delving into the archives of this robust culture of print is crucial work for understanding Haitian history, but it is also fraught work. \quotation{Archival work,} Abdul JanMohamed and David Lloyd have written, \quotation{{[}is{]} a form of counter-memory that is essential to the critical articulation of minority discourse.}\footnote{Abdul JanMohamed and David Lloyd, \quotation{Introduction: Toward a Theory of Minority Discourse,}~{\em Cultural Critique}, no. 6 (Spring 1987): 5--12.} However, because archives, like the texts found within them, reflect the worldview of both their creators and the institutions that house them, those who write about archives or use them as sources must be careful not to reproduce patterns of domination or cultural exploitation (to use with much irony a variation of a term from Geggus's essay title). For Haiti, this means that we must work against the idea that the abundant historical resources now made readily (and often freely) available by various digitization projects (including HATHI Trust, Gale's Slavery and Anti-Slavery Database, Gallica, and archive.org) indicate that Haiti represents \quotation{history's new frontier} for research.\footnote{Philippe R. Girard, \quotation{The Haitian Revolution, History's New Frontier: State of the Scholarship and Archival Sources,} {\em Journal of Slave and Post-Slave Studies} 34, no. 3 (2013): 485--507.} Seeing Haiti as a \quotation{frontier} not only calls forth the terrible legacy of the conquest of the Americas but encourages the notion that the country is \quotation{open for business.} An open-for-business model of Haitian digital archives is problematic not least because it reflects the economics of disaster capitalism that have so plagued Haiti in the twentieth century and especially since the 2010 earthquake. Despite Haiti's history of colonialism and (recent) foreign occupation, Bloomberg TV, drawing on the words of Haitian president Michel Martelly, aired a special on 12 January 2015 encouraging foreign businesses to consider the country to be monetarily open to them: \useURL[url10][http://www.bloomberg.com/news/videos/2015-01-13/haiti-open-for-business][][{\em Haiti: Open for Business?}]\from[url10]\footnote{{\em Haiti: Open for Business?}, Bloomberg TV, 26:09 min., www.bloomberg.com/news/videos/2015-01-13/haiti-open-for-business.} Nadève Ménard reminded us, nevertheless, at a meeting of the 2015 Haitian Studies Association conference in Montréal, that Haiti has always been open for business where the world powers are concerned.\footnote{\quotation{Haiti is not newly open for business, it has always been open for business.} Nadève Ménard, Annual Meeting of the Haitian Studies Association, 23 October 2015; quoted by Marlene L. Daut, @FictionsofHaiti, \useURL[url11][https://twitter.com/FictionsofHaiti/status/657671004089663488][][twitter.com/FictionsofHaiti/status/657671004089663488]\from[url11].}

Thinking about how to approach the vast archival and intellectual terrain with respect for the peoples and cultures of Haiti has led me to question how the vocabulary we are using to describe this moment in Haitian studies (where the digital collides with renewed interest in Haitian history in the North Atlantic) might be framed differently so as to generate productive ideas about Haiti rather than merely rehashing old ones or proffering similarly damaging new ones in their place. Thinking about archiving as a crossroads, where institutions, scholars, the public, and the spirits of Haiti must meet on uneven, and sometimes conflictual, terrain is only a starting point.

Because most of the archival material used in scholarship about the Revolution and early-nineteenth-century Haiti is not actually housed in Haitian institutions, scholars interested in the revolutionary period and the period immediately after independence find themselves traveling all over the world, including to Jamaica, Puerto Rico, England, France, Germany, Spain, and Denmark, as well as to the United States and Canada. While many documents from Haiti can be found in archives located abroad, most of the archives in England, France, and the United States overwhelmingly contain sources about Haiti that were very often written by natives of those respective countries. I have not mentioned Haiti itself as a destination for traditional archival work, although, of course, it is. Aside from what we would traditionally think of as the archive, that is, written documents, Michel-Rolph Trouillot noted that the archive was as equally in the \quotation{buildings, dead bodies, censuses, {[}and{]} monuments} as it was in \quotation{diaries and political boundaries.}\footnote{Michel-Rolph Trouillot, {\em Silencing the Past: Power and the Production of History} (Boston: Beacon, 1995), 29.} Such an idea is rooted deeply in nineteenth-century Haitian intellectual history. Hérard Dumesle, for instance, wrote in his 1824 {\em Voyage dans le nord d'Hayti;} {\em ou, Révélations des lieux et des monuments historiques}, \quotation{Persuaded that nothing in nature is mute if we know how to read it, I interrogated monuments and places.}\footnote{\quotation{Persuadé que rien n'est muet dans la nature pour qui sait la consulter, j'interrogeai les lieux et les monumens}; Hérard Dumesle, {\em Voyage dans le nord d'Hayti;} {\em ou, Révélations des lieux et des monuments historiques} (Aux Cayes: De L'imprimerie du Gouvernment, 1824), 74.} Throughout the body of the work, Dumesle describes how he traveled to various sites in Haiti, including the Citadel and the Palace at Sans Souci, allowing these buildings, these \quotation{monuments,} as equally as the very land itself, to provide him with sources for telling the story of Haitian history: \quotation{Stones, mountains, and valleys, oh how your echoes resound!}\footnote{\quotation{Rochers, monts et vallons, que vos échos gémissent!}; ibid., 29.}~

While Dumesle had looked for the ruins to speak, and Baron de Vastey had turned to ashes, bones, and the bodies of survivors, today's historians increasingly rely on more traditional archival sources: written documents.\footnote{Ibid., 73; see Marlene Daut, {\em Baron de Vastey and the Origins of Black Atlantic Humanism} (Basingstoke, UK: Palgrave Macmillan, 2017).} In \quotation{Unexploited Sources,} Geggus had warned researchers that there were no written documents about the Haitian Revolution or even colonial Saint-Domingue in Haiti. \quotation{Paradoxically, but understandably enough,} he wrote, \quotation{Haiti itself seems to possess no primary sources concerning its revolution that destroyed so much. According to its director, the Archives Nationales in Port-au-Prince contains no documents from the colonial period.}\footnote{Geggus, \quotation{Unexploited Sources,} 98.} Julia Gaffield, noting that she herself traveled to Jamaica, the Netherlands, and Denmark, along with England, Spain, France, and various places in the United States, has also observed that because the \quotation{collections {[}in Haiti{]} relate mostly to the latter half of the nineteenth century and twentieth century,} her research on the \quotation{early years of Haitian independence could not be based exclusively on research in Haiti.}\footnote{Julia Gaffield, \quotation{Haiti's Declaration of Independence: Digging for Lost Documents in the Archives of the Atlantic World,} {\em Appendix}, 5 February 2014, \useURL[url12][http://theappendix.net/issues/2014/1/haitis-declaration-of-independence-digging-for-lost-documents-in-the-archives-of-the-atlantic-world][][theappendix.net/issues/2014/1/haitis-declaration-of-independence-digging-for-lost-documents-in-the-archives-of-the-atlantic-world]\from[url12].}

Such imbalance in the location of the archive led to Gaffield's unearthing of the first printed copy of the Haitian Declaration of Independence in the United Kingdom's National Archives, rather than in those of Haiti. Just after it was proclaimed, the {\em Acte de l'indépendance} circulated around the Atlantic world, both instantiating black Atlantic humanism and influencing the meaning and shape of political sovereignty after colonialism.\footnote{Deborah Jenson, \quotation{Dessalines's American Proclamations of the Haitian Independence,} {\em Journal of Haitian Studies} 15, nos. 1--2 (2009): 77--83.} Given Haiti's important stature after the Revolution, it is not surprising that a document of this significance would be found in the archives of one of Haiti's greatest trading partners in the nineteenth century.\footnote{See Gaffield, {\em Haitian Connections in the Atlantic World}.} The writers who most ardently advocated for the recognition of Haitian sovereignty in the first years after independence were, in fact, in large part from Great Britain. But how exactly did the only two known government-printed copies of the act end up {\em solely} in the archives of the UK?\footnote{Later, Gaffield located a second government-printed copy of the act in the UK National Archives. See Gaffield, \quotation{Haiti's Declaration of Independence.}} Why are there no copies of such documents in Haiti (can a history of fire and natural disaster really explain it all)? And should the UK return these documents to the country that originally created them?

After Gaffield located the declaration, Patrick Tardieu, head of Haiti's oldest library, the Bibliothèque Haitienne des Pères du Saint Esprit, pleaded to the UK National Archives to support a Haitian-led archival mission. \quotation{I once again appeal to the authorities of the country,} Tardieu wrote in the Haitian newspaper {\em Le Nouvelliste}. \quotation{A cultural mission imposes itself on the banks of the Thames; Her Majesty's Ambassador has he not made a formal invitation to the Haitian government in 2011? Are there other previously unrecognized documents in the archives of London? Only an official mission will tell.}\footnote{Patrick Tardieu, \quotation{Les femmes et l'effort de guerre: Ordonnance de Dessalines du 20 janvier 1804,} {\em Le Nouvelliste} 12 September 2013; translation quoted in Gaffield, \quotation{Haiti's Declaration of Independence.}}

For Haitians, access to the archives is bound by not only the questions of patrimony, ownership, and governance, to which Tardieu's statement above alludes, but the much more material difficulty of finances. While it seems unlikely that the UK will allow either of the Haitian government's printed copies of the Haitian Declaration of Independence to be returned to Haiti, digital archiving does offer a distinct possibility for us to think differently about how to bring home these documents.\footnote{See Laura Wagner, \quotation{{\em Nou toujou la!}~The Digital (After-)Life of Radio Haïti-Inter,} {\em sx archipelagos} 2 \useURL[url13][http://archipelagosjournal.org/issue02/nou-toujou-la.html][][smallaxe.net/sxarchipelagos/issue02/nou-toujou-la.html]\from[url13].} Gaffield, for instance, has already made the document publicly available on her website, and it is also now on the website of the UK National Archives, available for anyone to copy, save, or download.\footnote{Julia Gaffield, \quotation{1 January 1804 (printed pamphlet),} {\em Dessalines Reader}, Haiti and the Atlantic World, 28 October 2015, \useURL[url14][https://haitidoi.com/2015/10/28/dessalines-reader-1-january-1804-printed-pamphlet/]\from[url14]; \quotation{Haitian Declaration of Independence,} UK National Archives, CO 137/111/1, \useURL[url15][http://discovery.nationalarchives.gov.uk/details/r/C12756259][][discovery.nationalarchives.gov.uk/details/r/C12756259]\from[url15].} In this sense, digitization, while not to be confused with democratization, is helping both to bring greater attention to the existence of this document and to provide the possibility of greater access to it, as well as to knowledge of its vital history.

Through the painstaking efforts of digital archivists, archival repositories are aiming toward and achieving increasing decolonization. It is now no longer only the most well-funded academics and institutions who can access, consult, and learn from these rare archival materials. Still, in order to realize the full promise of digital archiving to not only change narratives of Haiti but to include nineteenth-century Haiti and provide access to the archives to Haitians themselves, we must do much more than simply make historical documents available on the web. Specialists must also work together with archivists and librarians in Haiti and beyond to better catalog the holdings, for one thing, and to ask better questions about how to categorize them, for another.

One of the reasons that the location of the first copy of the government-issued version of the {\em Acte de l'indépendance} that Gaffield eventually located in the UK archives was not previously unearthed is because it was in a box of documents that were categorized as pertaining to Jamaica. Similarly, Gaffield describes the second copy that she located during a subsequent trip to the UK National Archives as having \quotation{been removed from the Admiralty records and re-cataloged as a map so that it would not have to be folded in the bound volume.} She concludes, therefore, \quotation{This second discovery emphasized one of the key explanations for why all the previous efforts to find an official copy of the Haitian Declaration of Independence had failed. Both extant versions are located in unexpected archival locations.}\footnote{Gaffield, \quotation{Haiti's Declaration of Independence.}}

The ongoing digitization of the Archives Nationales in Port-au-Prince, principally in collaboration with the Digital Library of the Caribbean (DLOC), has similarly allowed the scholars, archivists, and librarians working on the project to determine that it really is not true that there are no documents from the colonial period in Haiti. For example, one document digitized by DLOC is titled \quotation{\useURL[url16][http://dloc.com/CA00510270/00001?search=revolution][][La désignation des postes pour l'armée de St. Domingue]\from[url16]}; another is a piece of Toussaint's \useURL[url17][http://dloc.com/CA00510263/00001?search=toussaint][][correspondence]\from[url17].\footnote{\quotation{La désignation des postes pour l'armée de St.~Domingue,} \useURL[url18][http://dloc.com/CA00510270/00001?search=revolution]\from[url18]; \quotation{Correspondance du général en chef au général Toussaint 5 prairial,} \useURL[url19][http://dloc.com/CA00510263/00001?search=toussaint]\from[url19]. DLOC is administered by Florida International University, in partnership with the University of the Virgin Islands and the University of Florida. DLOC's technical infrastructure is provided by the University of Florida.} In addition, the Haitian Library of the Pères du Saint-Esprit has also now been partially digitized; it offers a great deal of material from the colonial period, including the only known extant copy of Saint-Domingue's almanac of 1764 (a crucial record of people living in the French colony).\footnote{See the~Bibliothèque Haïtienne des Pères du Saint-Esprit~(est. 1873), \useURL[url20][http://www.dloc.com/ibhpse]\from[url20]. See also, on the~Archives Nationales d'Haïti~(est. 1860),~\quotation{Archives Nationales d'Haiti (National Archives of Haiti) and the DLOC Project: A New Step toward Regional Cooperation and Integration,} \useURL[url21][http://ufdc.ufl.edu/UF00083680/00001/pdf]\from[url21]; on the~Bibliothèque Nationale d'Haïti~(est. 1939), \quotation{Bibliothèque Nationale d'Haiti---May/June 2011,}~Smithsonian Libraries, {\em Unbound} (blog), 17 August 2011, \useURL[url22][https://blog.library.si.edu/blog/2011/08/17/bibliotheque-nationale-dhaiti-mayjune-2011/]\from[url22]; on the~Bibliothèque Haïtienne des Frères de l'Instruction Chrétienne~(est. 1912),~\useURL[url23][http://www.dloc.com/ibhfic]\from[url23]; and on the~Musée du Panthéon National Haïtien (MUPANAH) (est. 1983), \useURL[url24][https://www.facebook.com/Mupanah]\from[url24].} The library has also digitized several additional letters written by Toussaint and the only known existing copy of volume 7 of Moreau de Saint-Mery's works on the constitutions of the French-claimed islands.\footnote{See \quotation{Welcome at la Bibliotheque Haitienne des Peres du Saint-Esprit,} \useURL[url25][http://web1.dloc.com/UF00095578/00001/8j]\from[url25].} These projects highlight how digitization with respect to the Haitian archive must both promote increasing access to Haitian-produced sources in the country itself and work on finding guides that can improve accessibility for everyone.\footnote{As Laura Wagner, of \quotation{The Radio Haiti} digitization project at Duke University, has stressed, the ability for Haitians living in Haiti to access digital archives cannot be taken for granted, particularly because the majority of digital projects on Haiti come from the United States. Laura Wagner, \quotation{Viv Radyo Ayiti! Vive Radio Haïti! Radio Haiti Lives!,} Duke University Libraries, {\em The Devil's Tale} (blog), 30 March 2018, \useURL[url26][https://blogs.library.duke.edu/rubenstein/2018/03/30/viv-radyo-ayiti-vive-radio-haiti-radio-haiti-lives/]\from[url26].}

Chelsea Stieber's project, the RSHHGG Lab, undertaken in collaboration with the US Library of Congress (LOC) and the Société Haïtienne d'Histoire, de Géographie et de Géologie (SHHGG), concretely demonstrates how digital-finding guides rather than digitization alone can improve access while recognizing and respecting the innately uneven power relations involved in projects that originate outside Haiti. While Stieber was in residence at the LOC, she observed that this US library contains one of the only complete collections of the society's important journal, {\em Revue de la Société Haïtienne d'Histoire, de Géographie et de Géologie}, which is the longest-running periodical in Haiti. The {\em RSHHGG} is also the greatest repository of historical research produced on Haiti, from Haiti, and yet its contents are rarely used by scholars outside the country. Stieber's idea was to make the hundreds of articles in this journal more accessible to North Atlantic scholars. To that end, she proposed a digital indexing project, which librarians from the LOC eagerly encouraged and supported. The project, which is already online (\useURL[url27][http://rshhgglab.com/search/]\from[url27]), is designed as a laboratory because it allows users to begin to search the contents of more than thirteen hundred articles and help to index them. It is not, however, a digitization project, per se. At the request of the Société, the journal will not be fully digitized at this time. The creators, contributors, and collaborators of the lab, while undoubtedly helping Haitian scholarship to digitally travel, recognized that their Haitian partners wished them not to encourage the \quotation{Columbusing} of Haitian-produced scholarship, or in the words of Brenda Stiles, the discovery of \quotation{something that is not new.}\footnote{Brenda Stiles, \quotation{\quote{Columbusing}: The Art of Discovering Something That Is Not New,} NPR, 6 July 2014, \useURL[url28][https://www.npr.org/sections/codeswitch/2014/07/06/328466757/columbusing-the-art-of-discovering-something-that-is-not-new]\from[url28].} Thus we might say that the RSHHGG Lab has provided a model for a digital project that has the capacity to ensure that Haitian scholars are neither silenced nor exploited in this moment of resurgent interest in Haitian history.

\subsection[title={Content},reference={content}]

Although many digitization projects are now being accomplished through large grant sponsorship, one way to improve access to the content of libraries and archives that do not have such funding is to think more about what individuals can do to help digitally archive research materials. Scholars of Haiti, over the course of their research, before this era of mass digitization even, amassed great amounts of content. The Alfred Nemours Collection of Haitian history is a good example of an archive linked to specific scholar's library.

Nemours (1883--1955) was a Haitian writer---primarily a military historian---who had been a general in the Haitian army and who also occupied a number of different official posts for the Haitian state in the early twentieth century. He was quite prolific, publishing several books, among them his notable {\em Histoire militaire de la guerre d'indépendance de Saint-Domingue} in 1925. Nemours collected numerous original documents from various archives, too. Once he passed away, this collection of historical documents, articles, and books, as well as his personal library, was acquired from Nemours's widow by the University of Puerto Rico, Río Piedras.\footnote{Jane Toth and William A. Trembley, \quotation{The Alfred Nemours Collection of Haitian History: A Catalogue,}~{\em Caribbean Studies}~2, no. 3 (1962): 61. See also Humberto García Muñiz et al., \quotation{La Collección Alfred Nemours de Historia Haitiana, Une Fuente Olvidada, en el Bicentenario de la independcia de Haití,} {\em Caribbean Studies} 32, no. 2 (2004): 181--241, https://www.redalyc.org/articulo.oa?id=39232206.} The collection was allowed to leave France, despite strict laws about French patrimony, because the French government did not think these materials actually pertained to France.\footnote{García Muñiz et al., \quotation{La Collección Alfred Nemours de Historia Haitiana.}} In other words, the government of France did not think that the history of Haiti concerned their country. Yet in the Nemours Collection are not only multiple signed documents by Napoléon Bonaparte and Generals Leclerc and Rochambeau (whom Bonaparte had sent to the colony in 1802 in order to reinstate slavery) but multiple original documents pertaining to Toussaint's captivity, imprisonment, death, and autopsy, as well as crucial documents relating to the family history of the Louvertures in France after the former revolutionary general's death. Numerous documents related to early-nineteenth-century Haiti also populate this collection, particularly newspapers, treaties, and paintings of the kingdom of Hayti and King Christophe himself, along with sketches and portraits of Toussaint, Boyer, Pétion, and Soulouque (see figs. 4--6). The UPR library is interested in the digitization of these materials, but there is a serious lack of funding, exacerbated by the recent hurricane and untold destruction unleashed by US government neglect in its wake.

One wonders, though: If digitization is not going to be an institutional priority, could scholars with organization and coordination from the UPR librarians help to scan the Nemours Collection on their own, using what we might refer to as historian archivists, a more finely honed variation, perhaps, of the \quotation{citizen historian} model?

\placefigure[here]{Letter from General Donatien Rochambeau to General Jean-Jacques Dessalines.}{\externalfigure[issue03/daut-d.jpg]}


\placefigure[here]{Engraving by Barincou of Boyer in full dress uniform for a royal reception, 17 April 1825.}{\externalfigure[issue03/daut-e.jpg]}


\placefigure[here]{Proclamation from 4 June 1812 signed by Alexandre Pétion.}{\externalfigure[issue03/daut-f.jpg]}


The \quotation{citizen historian} model of digital history projects has been criticized for what some see in the end as having resulted in little more than crowdsourcing. In the words of Mia Ridge of Open University, actual citizen history versus crowdsourcing requires \quotation{build{[}ing{]} a critical mass of discussion and usage} while \quotation{expos{[}ing{]} people to historical materials that are potentially interesting.} However, Ridge is careful to stress that formally trained historians are crucial to the success of such public history projects, and that their expertise should therefore not be hidden away. She calls for \quotation{crowdsourcing and citizen history project organisers to be more careful with the terminology they use}: \quotation{Signing up to a project and doing a bit of transcription work does not make that person a historian, but this can become the end result. Projects need to be clear about what it is they are offering and asking, and what exactly is required to become a citizen historian rather than, perhaps, a citizen transcriber.}\footnote{Mia Ridge, quoted in Matt Phillpott, \quotation{Citizen History and Its Discontents: Postscript,} {\em Digital History Seminar}, Institute of Historical Research, 1 December 2014, updated 3 August 2015, \useURL[url29][https://ihrdighist.blogs.sas.ac.uk/2014/12/citizen-history-and-its-discontents-postscript/]\from[url29].} Without a doubt, neither trained archivists nor traditional historians can be replaced in digital historical scholarship. But transcription and crowdsourcing do each have their role in encouraging conversation around digital humanities projects and promoting the decolonization to which public history aims. The Smithsonian's Digital Volunteers movement, for example, has been highly successful at recruiting members of the general public (11,980 and counting, at the time of the writing of this essay) to \quotation{help\ldots{}make historical documents and biodiversity data more accessible.}\footnote{Smithsonian Institution, \quotation{Smithsonian Digital Volunteers: Transcription Center,} \useURL[url30][https://transcription.si.edu/]\from[url30].} Volunteer transcribers are working to \quotation{decipher everything from handwritten specimen tags to the personal letters of iconic artists to early U.S. currency.}\footnote{Helen Thompson, \quotation{The Smithsonian Wants You! (to Help Transcribe Its Collections),} 12 August 2014, \useURL[url31][https://www.smithsonianmag.com/smithsonian-institution/smithsonian-wants-you-help-transcribe-its-collections-180952333/\#LSv6eyti8Ywhcwsh.99]\from[url31].} Digitizing an extensive collection, like that of Nemours's, could be undertaken, as with the Smithsonian's project, in collaboration with trained archivists and historians. Historians who archive their own findings, in fact, have already been engaged in precisely this kind of work.

A contemporary historian archivist who is amassing a great amount of material and who has made it largely available to the public is John Garrigus. \useURL[url32][https://www.zotero.org/garrigus/items][][Garrigus's personal library]\from[url32], which is available through Zotero, contains copies of both primary and secondary sources (where copyright laws allow), making it a rich and invaluable resource for researchers of colonial Saint-Domingue and revolutionary Haiti.\footnote{John D. Garrigus Library, \useURL[url33][https://www.zotero.org/garrigus/items]\from[url33].} Making archival documents available online in this way, as well as the more contemporary items in our personal libraries, can contribute significantly toward the possibility that others can also achieve a similar level of content for their own projects, even if they do not have the funds to travel around the world to collect it.

Julia Gaffield's born-digital \useURL[url34][http://haitidoi.com/category/dessalines-reader/][][{\em Dessalines Reader}]\from[url34] is another prime example of the possibilities that \quotation{sharing} by historian archivists can open up, both in terms of publishing and promoting greater equality in terms of access to content. In the introduction to the first installment, Gaffield explains her motivation for publishing this reader online: \quotation{Recently, I proposed {[}to a publisher{]} a volume of the texts attributed to Dessalines but the publisher concluded that there was not a market for this material, especially in light of the question of authorship. This response has renewed my conviction,} she writes, \quotation{that there is a need for broader access to Dessalines's texts in order to facilitate a larger conversation about his contributions (both positive and negative) to the Age of Revolutions.}\footnote{Julia Gaffield, introduction to {\em Dessalines Reader}, Haiti and the Atlantic World, \useURL[url35][https://haitidoi.com/dessalines-reader/]\from[url35].} The {\em Dessalines Reader} contains dozens of handwritten documents signed by Dessalines, but because these letters had to be culled from libraries and archives around the world, without Gaffield's intervention they would likely have remained largely unavailable not simply to researchers but to the general Haitian public. One of the greatest impacts that historian archivists can have is to use the internet to \quotation{digitally repatriate} documents that no longer live where they were born.\footnote{Paul Resta et al., \quotation{Digital Repatriation: Virtual Museum Partnerships with Indigenous Peoples,} proceedings of the 2002 International Conference on Computers in Education, \useURL[url36][https://pdfs.semanticscholar.org/12b2/e4bf7892aa39950c26d4fb5bf575c04433a9.pdf]\from[url36]}

\subsection[title={Context},reference={context}]

Working together with archivists to provide access to important content is only a small part of a larger journey that requires that we put nineteenth-century Haitian documents in context through digital curation. The \quotation{Digital Humanities Manifesto} (version 2.0) introduces the importance of curation in the following way:

\startblockquote
Whereas the modern university segregated scholarship from curation, demoting the latter to a secondary, supportive role, and sending curators into exile within museums, archives, and libraries, the Digital Humanities revolution promotes a fundamental reshaping of the research and teaching landscape. It recasts the scholar as curator and the curator as scholar, and, in so doing, sets out both to reinvigorate scholarly practice by means of an expanded set of possibilities and demands, and to renew the scholarly mission of museums, libraries, and archives.\footnote{Humanities Blast, \quotation{The Digital Humanities Manifesto 2.0,} \useURL[url37][http://www.humanitiesblast.com/manifesto/Manifesto_V2.pdf]\from[url37].}
\stopblockquote

An important element of digital archiving is our ability not only to share our content with the general public but to provide the kind of context that we would not get from simply visiting the archive. These physical boxes in archival repositories that we consult in the course of research at various libraries do not have implicit meaning and cannot fully be interpreted outside specific contexts. Digital archiving projects, unlike digital databases, put obscure documents into specialized contexts through curation.

One of the things that distinguishes a digital archiving project from a digital database is the ability for curation, which provides context not only for the project but for the documents themselves. Preexisting knowledge of the context is required to truly profit from the vast collection of documents contained in \useURL[url38][https://www.gale.com/primary-sources/slavery-and-anti-slavery][][Gale's Slavery and Anti-Slavery]\from[url38] online databases, for instance.\footnote{Gale, {\em Slavery and Anti-Slavery: A Transnational Archive}, \useURL[url39][https://www.gale.com/primary-sources/slavery-and-anti-slavery]\from[url39].} One must know beforehand what one is searching for; otherwise, the materials will appear as just a random and likely overwhelming selection of documents about slavery and abolition. Moreover, this database is not free to access; it requires a very expensive library subscription of each separate part of the database, four parts in total. Digital archiving projects, on the contrary, teach the reader/user the significance and importance of a defined set of documents, while also promoting decolonization of knowledge and, in many respects, defetishization of the archive as precious relic that only a few people have the right to behold.

Confronting how we contextualize the archive from the perspective of decolonization rather than democratization allows us to acknowledge both the opportunities and limitations of what is contained in archives and to reveal our own privilege in being able to access and interpret the documents. \useURL[url40][https://colonyincrisis.lib.umd.edu/][][{\em A Colony in Crisis}]\from[url40] provides a salient example of not only how researchers at any stage of their career can become historian archivists but also how we can ensure that the context we create around an archive recognizes the legacy of colonial privilege in which the documents were created and in which scholars now operate.

The {\em Colony in Crisis} project is an online reader about the Saint-Domingue grain crisis of 1789 that has published translations in English and Kreyòl of \quotation{key primary sources} in French held by the University of Maryland Library's Special Collections Department. The project makes public these pamphlets published in the late eighteenth-century \quotation{dealing with the grain shortage faced by the colony of Saint-Domingue in 1789.} Editors Kelsey Corlett Rivera, Abby Broughton, and Nathan Dize note that \quotation{alongside the French original, each translation is presented with a brief historical introduction to situate the reader in the time period and help understand how this particular pamphlet fits into the episode.}\footnote{\quotation{Welcome to {\em A Colony in Crisis},} {\em A Colony in Crisis: The Saint-Domingue Grain Shortage of 1789}, \useURL[url41][https://colonyincrisis.lib.umd.edu/]\from[url41].} The editors have also annotated and otherwise contextualized these materials so that the reader might understand how this historical moment was unfolding.

\useURL[url42][https://colonyincrisis.lib.umd.edu/category/translations/issue-3-0/][][Issue 3.0]\from[url42] of the project can help us to suspect the kind of \quotation{knowledge} that these particular pamphlets reveal about the crisis. The pamphlet called \quotation{Response from the Deputies and Manufacturers\ldots{},} dated 13 September 1789, for example, refutes the idea that there was a \quotation{famine} or \quotation{shortage} of food in the colony at all, and the pamphlet's authors sought even to contest the number of deaths by starvation of enslaved Africans reported to France, implying that these figures had been exaggerated.\footnote{\quotation{Response from the Deputies of Manufacturers and Commerce of France: To the Motions of MM. de Cocherel & de Reynaud, Deputies from the Isle of Saint-Domingue to the National Assembly, September 13, 1789,} {\em A Colony in Crisis}, issue 3.0, \useURL[url43][https://colonyincrisis.lib.umd.edu/1789/09/13/response-from-the-deputies-of-manufacturers-and-commerce-of-france-to-the-motions-of-mm-de-cocherel-de-reynaud-deputies-from-the-isle-of-saint-domingue-to-the-national-assembly-september-13/]\from[url43].} Two governing statements can help us understand how it is the fact of the archive's coloniality that produces its inherent incoherence. Katherine Clay Bassard has urged us not to forget that \quotation{given the violence enacted against black women's histories, bodies, and texts, the archives will remain incomplete and inadequate sources for an empirically complete counter-narrative of black women's literary production,} whereas Barbara Bush avers, \quotation{History has been written for men, by men, and thus records only what men wish to see.}\footnote{Katherine Clay Bassard, {\em Spiritual Interrogations: Culture, Gender, and Community in Early African American Women's Writing} (Princeton, NJ: Princeton University Press, 1999), 5; Barbara Bush, \quotation{Defiance or Submission? The Role of the Slave Woman in Slave Resistance in the British Caribbean,} {\em Immigrants and Minorities} 1, no. 1 (1982): 16.} In turn, colonial documents, as we see with {\em Colony in Crisis}, often tell us only what was useful for colonial and metropolitan officials to know, including information that the government of France did not want to make known and thus hoped to suppress. It is recognition of this very dynamic that led Haiti's Baron de Vastey to open his {\em Essai sur les causes de la révolution et des guerres civiles d'Hayti} (1819) by stating, \quotation{Haiti lacks a general history written by someone native to the country; the majority of historians who have tried it have been Europeans who concerned themselves primarily with the part of our history that involves them; and when they were led, by the subject at hand, to speak of the natives, they did so with that spirit of prejudice and bias that they never seem to be able to abandon\ldots{}.}\footnote{\quotation{Hayti manque d'une histoire générale écrite par un indigène du pays; la plupart des historiens qui en ont donné quelque fragments étaient des européens; ils ne se sont occupés principalement de la partie historique qui les concernent, et quand ils ont été entraîné, par le sujet, à parler des indigenes, ils l'ont fait avec cet esprit de préjugé et de partialité qu'ils ne peuvent abandoner\ldots{}.}; Baron de Vastey, {\em Essai sur les causes de la revolution et des guerres civiles d'Hayti, faisant suite au \quotation{Réflexions politiques sur quelques ouvrages et journaux français, concernant Hayti}} (Sans-Souci, Haiti: L'Imprimerie Royale, 1819), 1.}

Beyond the suspicions that the question of race itself raises about our possibilities to know outside the colonial epistemologies that inform the creation of and the preservation of documents in the archive, through examining the pamphlets themselves we can see a related problem: the way various colonial officials concerned themselves with black lives---the so-called crisis---solely in order to preserve black enslavement. Colonial officials, in other words, wanted to preserve black human life only to serve their own monetary ends; but who says that enslaved peoples wanted to be alive anyway?

In his 1814 {\em Le système colonial dévoilé}, channeling the testimony of the ancestors, Baron de Vastey writes,

\startblockquote
Is it any wonder if we resorted toward suicide, to poisoning? And if our women extinguished in their hearts all the sweet feelings of motherhood, when with cruel pity they caused the deaths of the dear, sad fruits of their lovers? In fact, how do you support life when it has reached the final limits of degradation and misery? When you must die a thousand times in one life by undergoing the most cruel tortures, when you are reduced to this deplorable situation, without any hope of escaping from it; to want to go on living, would that not be the utmost symbol of cowardice? Oh, why give life to such unfortunate beings, whose entire existence would condemn them to lead a pithy existence of torture and opprobrium in a long tissue of death without end; to extinguish such an odious life, was that such a great crime? it was compassion, humanity!!!\footnote{\quotation{Est-il étonnant si nous étions enclins aux suicides, aux empoisonnemens; et si nos femmes éteignaient dans leur cœurs les doux sentimens de la maternité, en faisant périr par une cruelle pité les chers et tristes fruits de leurs amours? En effet, comment supporter la vie quant elle est parvenue au dernier période de la degradation et de la misère? Quant il faut mourir mille fois pour une\ldots{}? Eh pourquoi donner le jour à des infortunes, dont la vie entière était d'être condamné à trainer leur frêle existence dans l'opprobre et les tourmens, dans un long tissu de peine sans fin; éteindre une vie aussi odieuse était-ce donc un si grand crime? c'était compassion, humanité!!!}; Baron de Vastey, {\em Le système colonial dévoilé} (Cap-Henry: P. Roux Imprimerie du Roi, 1814), 71--72.}
\stopblockquote

That living a human life was the irreducible goal of early modern and eighteenth-century people cannot be taken for granted, as Monique Allewaert has argued.\footnote{Monique Allewaert, {\em Ariel's Ecology: Plantations, Personhood, and Colonialism in the American Tropics} (Minneapolis: University of Minnesota Press, 2013), 109--10.} To that end, in 1790 Jacques Pierre Brissot de Warville (secretary of the Société des Amis des Noirs) concluded a long diatribe on the abuses that the Code Noir allowed to be perpetrated against enslaved Africans by stating, \quotation{We are still afraid of being human.}\footnote{{\em Réflexions sur le Code noir, et dénonciation d'un crime affreux commis à Saint-Domingue}, 6 August 1790, {\em A Colony in Crisis}, \useURL[url44][https://colonyincrisis.lib.umd.edu/1790/08/15/reflections-on-the-code-noir-and-denunciation-of-an-atrocious-crime-committed-in-saint-domingue-addressed-to-the-national-assembly-by-the-society-of-the-friends-of-blacks-paris-august-1790/]\from[url44].}

If it is imperative that we suspect the kind of \quotation{knowledge} and assumptions about humanity---and who is a human---offered to us by the colonial archive, what kind of hermeneutic should we offer in its place? One approach is to perhaps let our encounters with the history of the Haitian people be shaped by the interpretations and systems of knowledge of Haitian historians and writers themselves rather than by Haiti's \quotation{visitors,} as Louis Joseph Janvier would later call them.\footnote{Louis Joseph Janvier, {\em La République d'Haïti et ses visiteurs (1840--1882): Réponse à M. Victor Cochinat (de la Petite presse) et à quelques autres écrivains} (Paris: Marpon et Flammarion, 1883).} The methodology employed by early Haitian historians, in fact, provides a precious hermeneutic for approaching Haiti's colonial history with attention to the hopes, desires, and fears of the Haitians themselves rather than that of their European interlocutors

In his 1855 {\em Histoire des Caciques}, Émile Nau wrote that Haitians, as occupiers of the land once inhabited by the Caciques, had a duty to tell the island's indigenous history because the Caciques population could no longer do so. \quotation{The fact of inhabiting today the country where they once lived,} he wrote, \quotation{requires us, more than anyone else, to inquire into our predecessors.}\footnote{\quotation{Mais le fait d'habiter aujourd'hui le pays où ils vécurent, nous oblige, nous plus que personne, à nous enquérir de nos prédécesseurs}; Émile Nau, {\em Histoire des Caciques} (Port-au-Prince: T. Bouchereau, 1855), iv.} Nau was careful, however, to acknowledge that the history he was setting down would be only a fragmentary one, filled with voids, gaps, and a multitude of unanswered questions and profound silences. The Haitian historian described his methodology as a combination of folk history and traditional historical excavation of prior sources:

\startblockquote
It is impossible to go back even a day before the discovery, without giving into vagaries and conjecture. I have only gathered together in one set of works related to Haiti the fragments of this history, sparse in the works of historians and the accounts of voyagers who embraced the discovery of the entire continent of America. I gave, or at least I tried to give, a little more consistency to the facts and actions of the aborigines, and to recount, during a short moment of their existence, according to fables handed down and traces of traditions, their character, their mores, and their life.\footnote{\quotation{Il est impossible de remonter d'un jour au delà de la découverte, à moins de se jeter dans le vague et l'arbitraire des conjectures. J'ai seulement réuni en un corps d'ouvrage relatif à Haïti les fragmens de cette histoire, épars dans les historiens et les récits de voyage, qui embrassent la découverte entiere de l'Amérique. J'ai donné, ou du moins j'ai essayé de donner plus de consistance aux faits et gestes des aborigenès, et de reproduire, durant un court moment de leur existence, d'après des faibles données et des traditions incomplètes, leur caractère, leurs mœurs et leur vie}; ibid., v.}
\stopblockquote

Attention to the history of the indigenous population of Ayiti/Kiskeya would ensure that all of Haiti's \quotation{annals would be conserved and transmitted by its proper citizens} and would underscore the importance of \quotation{preventing {[}Haiti's{]} history from being forgotten.}\footnote{\quotation{Sauver son histoire de l'oubli}; ibid., ii.}

Combating erasure of Haiti's long history---one that also included the pre-Columbian era---by consulting alternative archives, including talking to the dead (which is what we are doing when we implore Papa Legba to open the gates and also what we are doing when we consult archives), textures the way I confront contextualizing the documents I uncover in the course of archival research. Adopting a Haitian epistemology drawn from Haitian sources and Haitian culture reminds us to suspect the knowledge that comes from the colonial archive and to keep metaphorically moving between the two worlds (that of the living and that of the dead) in order to listen to our historical sources and ask them what they want us to do and what they want us to say. That the creators of our \quotation{sources} are no longer living does not absolve us of responsibility to them. It is possible that we have just as much a responsibility to collaborate with our dead subjects as we do with those who are still living. And what early Haitian historians consistently urge is that we never forget the violent history of colonialism and the tortures meted out upon the future people of Haiti. Vastey wrote, to that end, \quotation{Oh, you young Haitians who have had the good fortune to have been born under the reign of laws and of liberty! You, who do not remember these times of horror and barbary; read this writing; never forget the misfortunes of your fathers, and teach yourselves to always defy and hate your enemies.}\footnote{\quotation{O vous jeunes haytiens qui avez le bonheur de naître sous le règne des lois et de la liberté! Vous qui ne connaissez pas ces temps d'horreurs et de barbaries; lisez ces écrits; n'oubliez jamais les infortunes de vos pères, et apprenez à vous défier et à haïr vos tyrants!}; Vastey, {\em Le système}, 90.} A framework of sovereign defiance against the colonial drive for oblivion can help us recognize that all scholarly work necessarily entails collaboration, with both those who came before us and those who will come after.

\subsection[title={Collaboration},reference={collaboration}]

In Jacques Roumain's 1944 novel {\em Gouverneurs de la rosée}, which was translated into English and published as {\em Masters of the Dew} in 1947 by Mercer Cook and Langston Hughes, the main character, Manuel, tries to convince the people of his small rural town to work together to end the drought. He tells them,

\startblockquote
We're poor, that's true. We're out of luck, that's true. We're miserable, that's true. But do you know why, brother? Because of our ignorance. We don't know yet what a force we are, what a single force---all the peasants, all the Negroes of plain and hill, all united. Some day, when we get wise to that, we'll rise up from one end of the country to the other. Then we'll call a General Assembly of the Masters of the Dew, a great big {\em coumbite} of farmers, and we'll clear out poverty and plant a new life.\footnote{Jacques Roumain, {\em Masters of the Dew}, trans. Langston Hughes and Mercer Cook (Oxford: Heinemann, 1947), 75.}
\stopblockquote

The kind of collaboration described in Roumain's novel has been traditionally undervalued in US American academy, and that undervaluation is reflected in today's humanities departments. This is not only in terms of preferring single-author scholarship---over translations, edited collections, and digital work, all of which is necessarily collaborative---but in terms of preferring to exalt what Frances Smith Foster has called \quotation{the lone fugitive,\ldots{}the singular sojourner, or the inspired visionary.}\footnote{Frances Smith Foster, \quotation{Creative Collaboration: As African American as Sweet Potato Pie,} in Harold Bloom, ed., {\em African American Poets}, vol.~1 (New York: Infobase, 2009), 87.} Working collaboratively should be a hallmark of digital Haitian studies, though, precisely because collaboration is a way of being that is deeply tied to Haitian culture and what Ulysse has termed anew {\em rasanblaj}, or gathering together.\footnote{Gina Athena Ulysse, \quotation{Seven Keywords for this Rasanblaj,} {\em Anthropology Now} 8, no. 3 (2016): 122.}

{\em Men anpil, chay pa lou}. This Haitian proverb roughly means, \quotation{With many hands, the burden is not heavy.} We say in English, \quotation{Many hands lighten the load.} One of the most important Haitian digital humanities projects, the trilingual site \useURL[url45][http://islandluminous.fiu.edu/contributors1.htm][][{\em An Island Luminous}]\from[url45], describes the process by which it came into being exactly as \quotation{truly a grand konbit.} Contributors to the site include a variety of US- and Haitian-based librarians and archivists, scholars and activists, as well as graduate students. The site, which \quotation{combines rare books, manuscripts, and photos scanned by archives and libraries in Haiti and the United States,} also contains \quotation{commentary by over one hundred (100) authors from universities around the world.}\footnote{\quotation{Contributors,} {\em An Island Luminous}, \useURL[url46][http://islandluminous.fiu.edu/contributors1.htm]\from[url46]; \quotation{Mission,} {\em An Island Luminous}, \useURL[url47][http://islandluminous.fiu.edu/mission1.htm]\from[url47].} This project has provided a distinct and concrete example of how gathering together across multiple kinds of boundaries can promote global interest in Haitian history.

A great deal of my own success with digital archiving has been accomplished through collaboration, or the {\em konbit}, too, the same kind of collective work that drives the narrative of Roumain's novel. In the course of my \useURL[url48][https://www.palgrave.com/us/book/9781137479693][][research]\from[url48] on the history of early-nineteenth-century Haiti's northern government, I attempted to gather as many issues as possible of both versions of Henry's Christophe's official state newspaper, the~{\em Gazette Officielle de l'Etat d'Hayti} (1807--11) and the {\em Gazette Royale d'Hayti} (1813--20), as well as a complete set of the six different versions of its yearly almanac, {\em L'Almanach Royal d'Hayti}.\footnote{See Marlene L. Daut, {\em Baron de Vastey and the Origins of Black Atlantic Humanism} (New York: Palgrave Macmillan, 2017).} However, I soon learned that no single library contains a complete collection of any of these publications and that the various issues that remain extant are scattered in dozens of archives across Europe, North America, and the Caribbean. Nevertheless, I spent three years searching the web and various library catalogs in order to collect the ninety-five issues and six almanacs currently featured on the site \useURL[url49][http://lagazetteroyale.com/][][{\em La Gazette Royale d'Hayti}]\from[url49].\footnote{\quotation{Explore Haiti's Early Print Culture,} {\em La Gazette Royale d'Hayti}, \useURL[url50][http://lagazetteroyale.com/]\from[url50].} I knew these documents, precious sources of information about the early years of Haitian statehood, would be of great interest to historians, anthropologists, archaeologists, and literary scholars who are studying early Haiti's political and literary actors or the Palace of Haiti at Sans-Souci, which has been designated a UNESCO World Heritage Site and which stood as the ultimate symbol of black sovereignty in the nineteenth-century Atlantic world. Rather than keeping the dozens of issues I had managed to collect confined to brief mentions in my own publications, I came to believe that making the originals of these texts publicly available and more accessible through transcription (and eventually through translation) was the key to combating the kind of colonial rhetoric that usually surges through public discourses about Haiti. One of the most important axioms of this project is that Haiti's early history is not an open book whose pages will remain empty until they are filled in by scholars from abroad. As is demonstrated by the newspapers and almanacs, as well as the myriad writings of nineteenth-century Haitian intellectuals, Haitians have been making and writing their own history---intellectual, literary, political---from the inception of the country until now.

Moreover, these newspapers and almanacs have the capacity to completely change our understanding of early-nineteenth-century Haiti's history. What we can learn by reading through this archive has both local and global significance. We observe at once, for example, the real-time reaction of the Haitian kingdom to Napoléon's excommunication from the Catholic Church and the government of the north's astonishment at both his escape from the island of Elba and the reestablishment of his authority, as well as their satisfaction with his presumed role in the subsequent abolition of the slave trade in France.\footnote{{\em La Gazette Royale}, 23 November 1809, \useURL[url51][http://lagazetteroyale.com/23-novembre-1809/]\from[url51]; {\em La Gazette Royale}, 2 June 1815, \useURL[url52][http://lagazetteroyale.com/2-juin-1815/]\from[url52]; {\em La Gazette Royale}, 13 June 1815, \useURL[url53][http://lagazetteroyale.com/13-juin-1815/]\from[url53].} Also recorded are details of the New England abolitionist Prince Saunders's establishment of a school in northwest Haiti in the city of Port-de-Paix and his arrival to the kingdom with doses of the Small Pox vaccine.\footnote{{\em La Gazette Royale}, 23 November 1809, \useURL[url54][http://lagazetteroyale.com/8-fevrier-1816/]\from[url54].} The pages of these newspapers are populated with precious descriptions of the different phases of the construction of the palace at Sans Souci, including a memorable etching to Christophe on one of the outside columns, \quotation{To the First Monarch Crowned in the New World.}\footnote{{\em La Gazette Royale}, 27 January 1814, \useURL[url55][http://lagazetteroyale.com/27-janvier-1814/]\from[url55]; \quotation{Au Premier Monarque couronné du Nouveau-Monde,} {\em La Gazette Royale}, 19 July 1815, \useURL[url56][http://lagazetteroyale.com/19-juin-1815/]\from[url56].} Painstakingly precise descriptions of lavish festivities, bombastic speeches, trade statistics, and the glorious attire of the royal guard, military, and the king himself are sure to incite a great deal of interest, along with descriptions of two {\em arcs de triomphes} on the road leading to the palace. The {\em Gazette} also importantly records the Kingdom of Hayti's singular construction of the only antislavery state in the Atlantic world. To that end, on 10 October 1817, the {\em Gazette} reported the Haitian military's capture of a slave ship carrying captive Africans and the government's subsequent release of 145 \quotation{unfortunate brothers, victims of greed and the odious traffic in human flesh.}\footnote{{\em La Gazette Royale}, 24 August 1816, \useURL[url57][http://lagazetteroyale.com/24-aout-1816/]\from[url57]; {\em La Gazette Royale}, 21 August 1816, \useURL[url58][http://lagazetteroyale.com/21-aout-1816/]\from[url58]; \quotation{infortunés frères, victimes de la cupidité et de l'odieux trafic de chair humaine,} {\em La Gazette Royale}, 10 October 1817, \useURL[url59][http://lagazetteroyale.com/10-octobre-1817/]\from[url59].} It is perhaps the issue dated 10 September 1819, however, that has the capacity to change our understanding of the events that led up to downfall of the Kingdom of Hayti.

One of the final newspapers in the {\em Gazette} collection reveals that King Henry's son, Prince Victor Henry, had recently returned from a fairly extensive tour of northern Haiti. The piece talks about how beloved the prince is throughout the kingdom, and the writer very much seems to be setting the tone for Victor Henry to assume the throne. The writer tells us that \quotation{satisfaction was the same everywhere the prince traveled; Haitians all being of one heart and mind, in their desire to give the prince testaments of their love.} Christophe is barely mentioned in this issue, and his absence from the tour of the kingdom, which he had done in previous years, seems a bit conspicuous. The article in question finishes by declaring that the prince is a \quotation{worthy inheritor of the name and the throne of Henry.}\footnote{\quotation{La satisfaction a été égale, comme dans tous les lieux où le Prince a passé, tous les haytiens n'avaient qu'un cœur, qu'une âme, pour donner à l'envi au Prince des témoignages d'amour}; \quotation{un digne héritier du nom et du trône d'Henry}; {\em La Gazette Royale}, 10 September 1819, \useURL[url60][https://lagazetteroyale.com/28-decembre-1818-version-2/]\from[url60].} Because Christophe had undertaken these tours in previous years, one wonders if his health was already suffering, leading him to position his son in such a way as to immediately assume the throne upon the king's perhaps anticipated demise.

There is a backstory of collaboration that enabled these newspapers and the vastly complex stories they tell---and the questions they raise---about early Haitian sovereignty to be brought to the public. Seven issues of~the~{\em Gazette Royale} were generously provided to me by historian Julia Gaffield, who shared the images she took with her personal camera at the National Library of Denmark in Copenhagen and the UK National Archives; a single issue of the~{\em Gazette Officielle}~housed at the Institution Saint-Louis de Gonzague in Port-au-Prince was happily shared with me by Paul Clammer; while Tabitha McIntosh alerted me to two issues of the {\em Gazette Royale} held by the Bibliothèque du Port-Royal in Paris, as well as to fifteen additional issues available at the Library of the American Philosophical Society in Philadelphia. Finally, a series of enthusiastic archivists, librarians, and specialists in document conservation at various institutions, but especially David Gary and Aura Díaz López, have also enabled this project to succeed in far more ways than I could enumerate here.

Once I had the issues, the question became once again not only what to do with them beyond my own scholarship but how to bring about further collaboration in relation to them and, more particularly, collaboration with the Haitian people who I felt needed to be able access them above all. One of the most important original goals of this project was to ensure that these vital publications from the earliest years of Haitian statehood would become immediately accessible to people in Haiti. Users will not fail to note that although all these documents have original provenance in Haiti,~\useURL[url61][https://lagazetteroyale.com/19-novembre-1814/][][only a single issue]\from[url61]~of the~{\em Gazette Officielle}, as far as we know at this time, can be found in archives there today.\footnote{{\em La Gazette Officielle}, 21 January 1808, \useURL[url62][https://lagazetteroyale.com/28-decembre-1818-version-2/]\from[url62].} Minimally, this situation required that we commit to our chief goals---preservation and institutionalization---in a way that eliminates paywalls and provides open access for Haitians.

Transcription (and eventually translation) seemed key to this because full text lessens the bandwidth of the presentation as it allows the documents to be read without downloading them and allows for greater searchability. Because OCR technology cannot be reliably employed for accurate document transcription (especially where diacritics are involved), my research assistant, the late Denice Groce, transcribed without the use of this technology the first drafts of fifty of the fifty-one issues of the~{\em Gazette Officielle}~featured on the website, as well as eight issues of the~{\em Gazette Royale}. I transcribed the lone issue of the~{\em Gazette Officielle~}from Port-au-Prince, along with the remaining~thirty-six issues of the~{\em Gazette Royale}.

Collaboration in service of access, translation, and transcription to promote international availability, as well as lowering the barriers of current provenance through the development of a website, are crucial elements in this attempt to create a decolonial archive. \quotation{Decolonizing the university,} Achille Mbembe has written, \quotation{starts with the de-privatization and rehabilitation of the public space---the rearrangement of spatial relations\ldots{}.It starts with a redefinition of what is public, i.e., what pertains to the realm of the common and as such, does not belong to anyone in particular because it must be equally shared between equals.}\footnote{Achille Mbembe, \quotation{Decolonizing Knowledge and the Question of the Archive,} 2015, \useURL[url63][https://wiser.wits.ac.za/system/files/Achille\%20Mbembe\%20-\%20Decolonizing\%20Knowledge\%20and\%20the\%20Question\%20of\%20the\%20Archive.pdf][][https://wiser.wits.ac.za/system/files/Achille\letterpercent{}20Mbembe\letterpercent{}20-\letterpercent{}20Decolonizing\letterpercent{}20Knowledge\letterpercent{}20and\letterpercent{}20the\letterpercent{}20Question\letterpercent{}20of\letterpercent{}20the\letterpercent{}20Archive.pdf]\from[url63], 5.} Though Mbembe was talking quite literally about the buildings on a university campus, his comments function quite well when thinking about the provenance of sources and ownership of historical documents. In a very material way, I think we must operate with a form of \quotation{{\em lakou} consciousness} that moves us away from a logic of individuality (private property) and toward the feeling of togetherness (the commons) that living collectively requires, that is, being a part of a public.\footnote{Myriam J. A. Chancy, \quotation{{\em Lakou} Consciousness: Empathy, or the Other Side(s) of the Fence,} academic convocation, Scripps College, 30 August 2016; see Marlene Daut, \quotation{{\em Lakou} Consciousness: Myriam J. A. Chancy Gives Academic Convocation at Scripps College,} {\em H-Haiti}, 12 September 2016, \useURL[url64][https://networks.h-net.org/node/116721/discussions/143483/lakou-consciousness-myriam-ja-chancy-gives-academic-convocation]\from[url64].} Sustainability over profitability, and decolonization over democratization, requires that institutions and researchers alike work together to remove the many barriers that exist to accessing historical resources. These include not simply problems of language, which translation often appears able to solve, but the much more heady and material problem of financing.

While digital archiving holds much promise to create more equity with respect to the public's ability to access historical documents, the notion that the digital humanities is a super hero capable of redressing the violent pasts of slavery and colonialism remains problematic. Digital projects involving archives are still susceptible to and indeed operating within the same unequal and oppressive processes of colonial extraction and finance capitalism that created the conditions proponents of digital decolonization seek to rectify in the first place. That is to say, it is colonialism that is responsible for so much of Haitian patrimony existing in foreign archives, libraries, and museums, and digital archiving, including digital repatriation, can neither redress this wrong nor provide a form of reparations for it. In fact, putting documents on the internet haphazardly, as if we were all the digital cowboys of the world wide web, could end up mirroring colonial extraction on a wider scale, particularly, if it is done without collaboration. But historians who digitally archive might help us to avoid recreating, through the process of expensive (and thus often inaccessible) scholarly publications, the very systems of domination, exploitation, and ownership that we rail against when writing about the machines of slavery and colonialism.

\subsection[title={Conclusion},reference={conclusion}]

An anecdote by David Nicholls from his influential monograph {\em From Dessalines to Duvalier} (1979) frames my final comments. In his preface to the first edition of the book, Nicholls quite ironically thanked Hénock Trouillot, director at the time of the Archives Nationales in Port-au-Prince and a formidable scholar in his own right, \quotation{for his unintended kindness in refusing to allow {[}Nicholls{]} access to the archives.} \quotation{If he had,} Nicholls writes, \quotation{I would probably have been faced with such daunting practical problems that this book might never have been finished.}\footnote{David Nicholls, {\em From Dessalines to Duvalier: Race, Colour, and National Independence in Haiti} (Cambridge: Cambridge University Press, 1979), xliii.} Trouillot placed an immediate obstacle of access in Nicholls's way. Maybe Nicholls didn't have the right passcode ({\em Papa Legba,} {\em louvri barrye pou mwen}), but he had to make his own peace with this lack of access and move forward with the publication of his book anyway. I think that we, too, must remain unafraid of opacity, those moments when the archive refuses to speak to us, to make sense to us, or even to let us in.\footnote{On opacity, see \quotation{One World in Relation: Édouard Glissant in Conversation with Manthia Diawara,} trans. Christopher Winks, {\em Nka}, no. 28 (Spring 2011): 14--15.} The fact remains that for a variety of complicated reasons, most Haitian archives are not open to the general public.

We must also learn what it is to know in fractions (it is very unlikely, for instance, that we will ever even come close to having a full run of {\em La Gazette Royale}/{\em Officielle}). Understanding, interpreting, and posing questions about those fragments we do manage to obtain will be entirely contingent not only on what the archive holds but also on what it does not. Like Nau, then, we have to open our minds to partially knowing, and perhaps to recognizing our inability to know everything. Most importantly, we must respect the fact that we do not necessarily have a right to this knowledge.

Part of digitally archiving from the perspective of decolonization, therefore, is not only making our work available and accessible to others but also letting go of control---perhaps through collaboration---and letting go of perfection by allowing work to go out into the world in medias res. In so doing, we can take advantage of the ability the internet offers us not only to publish quickly but to revise our work constantly, to seek immediate feedback and even to change the content (and our conclusions) when necessary.

If I began this essay by expressing a sense of trouble about the way we talk about Haiti, I remain troubled. Archiving the sovereignty of nineteenth-century Haiti is always going to be fraught, principally because what we are archiving lies in ruins. Haiti remains, in many ways, under both figurative and literal military occupation. The memories of lost sovereignty hurt and are conjured up again and again when we study the history of early Haiti. \quotation{And the museums,} Aimé Césaire reminds us, \quotation{it would have been better not to have needed them\ldots{}.No, in the scales of knowledge all the museums in the world will never weigh so much as one spark of human sympathy.}\footnote{Aimé Césaire, {\em Discourse on Colonialism}, trans. Joan Pinkham (New York: Monthly Review, 1972), 71, 72.} One way to work through this pain is to take it to the crossroads where the living and the dead, the colonial and the decolonial, have no choice but to interrogate one another. Drawing on the words of George Lamming from {\em The Pleasures of Exile}, it seems imperative, now more than ever, for us all to recognize that we are \quotation{direct descendant{[}s{]} of Prospero worshipping in the same temple of endeavour, using his legacy of language---not to curse our meeting---but to push it further, reminding the descendants of both sides that what's done is done, and can only be seen as a soil from which other gifts, or the same gift endowed with different meanings, may grow towards a future which is colonized by our acts in this moment, but which must always remain open.}\footnote{George Lamming, {\em The Pleasures of Exile} (Ann Arbor: University of Michigan Press, 1991), 15.}

\thinrule

\page
\subsection{Marlene L. Daut}

Marlene L. Daut specializes in early Caribbean, nineteenth-century African American, and early modern French colonial literary and historical studies. Her first book, {\em Tropics of Haiti: Race and the Literary History of the Haitian Revolution in the Atlantic World, 1789--1865} (Liverpool University Press, 2015) was part of the Liverpool Studies in International Slavery. Her second book, {\em Baron de Vastey and the Origins of Black Atlantic Humanism} (Palgrave Macmillan, 2017) was part of the series in the New Urban Atlantic. She is now working on a collaborative project with Grégory Pierrot and Marion Rohrleitner titled {\em An Anthology of Haitian Revolutionary Fictions (Age of Slavery}), which is under contract with the University of Virginia Press. Daut is the cocreator and coeditor of H-Net Commons's digital platform \useURL[url1][https://networks.h-net.org/h-haiti][][H-Haiti]\from[url1]. She also curates a website on early Haitian print culture at \useURL[url2][http://lagazetteroyale.com/][][{\em La Gazette Royale d'Hayti}]\from[url2] and has developed an online bibliography of fictions of the Haitian Revolution from 1787 to 1900 at \useURL[url3][http://haitianrevolutionaryfictions.com]\from[url3].

\stopchapter
\stoptext