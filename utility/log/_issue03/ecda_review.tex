\setvariables[article][shortauthor={Dillon}, date={April 2019}, issue={3}, DOI={Upcoming}]

\setupinteraction[title={The *Early Caribbean Digital Archive*},author={Elizabeth Dillon}, date={April 2019}, subtitle={EACD}]
\environment env_journal


\starttext


\startchapter[title={The {\em Early Caribbean Digital Archive}}
, marking={EACD}
, bookmark={The *Early Caribbean Digital Archive*}]


\startlines
{\bf
Elizabeth Dillon
}
\stoplines


\subsection[reference={the-early-caribbean-digital-archive},
bookmark={The *Early Caribbean Digital Archive*},
title={The *Early Caribbean Digital Archive*}]

\subsection[reference={sx-archipelagos-review},
bookmark={*sx archipelagos* review},
title={*sx archipelagos* review}]

The {\em \useURL[url1][https://ecda.northeastern.edu/][][Early Caribbean Digital Archive]\from[url1]} is a compelling platform that engages with the pitfalls and possibilities of uncovering alternative narratives within the constrained space of the colonial historical record. Clearly structured~and aesthetically appealing, the site explicitly and implicitly raises and grapples with crucial~questions regarding methodology and scholarly responsibility.

Perhaps the most important question the {\em Early Caribbean Digital Archive} brings to the fore is whether---or to what extent---it is in fact possible to \quotation{decolonize the colonial archive} via digital practices of \quotation{remixing.} The site authors acknowledge at the outset that the materials they feature \quotation{are primarily authored and published by Europeans,} but they quickly promise to push against that coloniality by foregrounding the narratives and experiences of nonwhite peoples in the early Americas.\footnote{\quotation{The Early Caribbean Digital Archive,} https://ecda.northeastern.edu/.}~The project's stated goals are to go beyond digitization and cataloging to provide both understanding of the mechanisms of coloniality and opportunities for users to revise the archive so to tell different, lesser-known stories.

Before getting to the matter of whether the site accomplishes these objectives, it would be of interest for the site authors to reflect on a number of grounding methodological questions, perhaps within the context of the site's \quotation{\useURL[url2][https://ecda.northeastern.edu/home/about/][][About]\from[url2]} section. Notably, how do we think about those sources we know are curated (selected, abridged, otherwise manipulated) by those in power---sources whose very presence in the archive reflects choices made by colonial subjects? Even more provocatively, what precisely is \quotation{decolonizing} and (why) do we want to \quotation{decolonize the archive}---might not our goal be instead to replace the archive with anticolonial and antislavery methods? Why, also, as per Marisa Fuentes's citation on the site's landing page, should the goal of such projects be to present an \quotation{unbiased account?} Why should \quotation{coherency} be the goal? Might not the goal of history be to acknowledge our biases---or even to have \quotation{the right} biases (it might be argued, even, that the attempt to be neutral is to operate from the position of the colonizer---or what Mary Louise Pratt has called the \quotation{monarch-of-all-I-survey} position)?\footnote{Mary Louise Pratt, {\em Imperial Eyes: Travel Writing and Transculturation} (New York: Routledge, 1992), 201.}~Such broad questions are worth posing.

Also important,~if \quotation{troubling} to consider, is the matter of how to~determine whether the \quotation{embedded narratives} might in fact have been mere literary devices. In other words, given that the invented interlocutor is a major trope of the Enlightenment, can we accept at face value the idea that these enslaved narrators were real and not characters invented to illustrate something the colonial author wanted to prove? Foregrounding such opacities would do the work of responsibly framing the project---acknowledging not just its potentialities and affordances but also its constraints and open queries.

Given their compelling claim that digitization makes a difference, it would be of great value for the site authors to do the work of extraction for a greater number of the embedded slave narratives discoverable among the documents included in their archive. Clearly/prominently identifying/naming enslaved narrators as the defining offering of the platform would go a long way to doing the work of \quotation{decolonizing.} Indeed, this is an extremely intriguing and very promising dimension of the site's contributions.

The \quotation{About} section under the \quotation{Archive} tab might more accurately be labeled \quotation{Using the ECDA} or \quotation{Suggestion for Use,} or something along those lines.

The \quotation{Classroom} tab is fantastic. How, though, do the site authors determine whether materials uploaded via the \quotation{Submit Materials} button at the bottom of the page ultimately end up on the site? Are these materials vetted by a committee and, if so, what kinds of criteria are used? Clarity around these questions would be useful to include on the site.

On the whole, {\em The Early Caribbean Digital Archive} offers a useful and wide-ranging portal into significant materials from the plantation Americas and, as important, proposes compelling paths for querying and pushing against the constraints of the colonial record.

\subsection[reference={response-from-the-creators-of-the-early-caribbean-digital-archive},
bookmark={Response from the creators of *The Early Caribbean Digital Archive*},
title={Response from the creators of *The Early Caribbean Digital Archive*}]

\thinrule

\page
\subsection{Elizabeth Dillon}



\stopchapter
\stoptext