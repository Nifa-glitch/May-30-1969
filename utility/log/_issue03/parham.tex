\setvariables[article][shortauthor={Parham}, date={April 2019}, issue={3}, DOI={doi:10.7916/D812651T}]

\setupinteraction[title={Break Dance},author={Marisa Parham}, date={April 2019}, subtitle={Break Dance}]
\environment env_journal


\starttext


\startchapter[title={Break Dance}
, marking={Break Dance}
, bookmark={Break Dance}]


\startlines
{\bf
Marisa Parham
}
\stoplines


{\startnarrower\it .break .dance~is a time-based web experience opened in response to xxx and xxx, thinking about~"Slavery in the Machine. In thinking through and against the machineries of commercial interface efficacy, this pocket intentionally shows its material and discursive seams. Rooted in a sense of anarchival play, it~is designed for multiple engagements, changes over time, and assumes no one will take the same path through. \stopnarrower}

\blank[2*line]
\blackrule[width=\textwidth,height=.01pt]
\blank[2*line]

\subsection[title={Opening},reference={opening}]

.break .dance~is an ongoing research pocket opened by~Marisa Parham, with support from the~irlH~team.~

A pocket is a place; it is also a position. A pocket is a place for experimentation in form and content. It is also a kind of space, a thing behind a thing, the thing you hold on to, just in case. It is the ephemeral non-space between thought and its elaboration or instrumentation, a tiny universe of intellectual possibility.~

.break .dance~is a time-based web experience opened in response to xxx and xxx, thinking about~\quotation{Slavery in the Machine.} In thinking through and against the machineries of commercial interface efficacy, this pocket intentionally shows its material and discursive seams. Rooted in a sense of anarchival play, it~is designed for multiple engagements, changes over time, and assumes no one will take the same path through.~This pocket was last updated on 3/8/2019.

In honor of the many ways Black diasporic peoples use memory, performance, and speculative prayer to recode time and space,~.break .dance~begins with a cosmogram of Kongo origin,~tendwa kia nza-n' Kongo, the four movements of the sun. Cosmograms are two-dimensional figures that tie the cardinal directions to the sky's map--- sun, moon, stars. The~tendwa~reminds us of how our simplest navigational moments are always multi-dimensional: one foot in front of the other can also mean face toward the wind, a turn east toward the sun, or following the North Star to freedom.

\subsection[title={Instructions},reference={instructions}]

As a series of interlocking machines moving toward a state of flux,~.break .dance~surfaces its own work, your work as you read. Many of its features are time based, and there are places that cannot be accessed until some time has passed on a page. Because it is hypertextual, branching, and interlocking, it is reasonable to expect that every user will not read the same thing. Indeed, to highlight this there are also pages where it does not matter which direction you read, and others where you can move the argument, or use Twitter to add your own lines. If you are worried that you have waited too long for a page to finish, check the title.~The titles turn to fuchsia when the page is complete.

Sometimes~.break .dance~asks the reader to move forward, from one staging to the next, for instance from theory to illustration.~

There are also arrows to indicate how to move backwards. Please note that if you return to the homepage, its timer is reset to 0.~

Sometimes~.break .dance~tempts the reader to look around, to notice how there might be more than one good example, that the best word is not always clear, even as intention hinges on the word. The play on word and example intimates the multiverse of inquiry. I mark these places with a bit of glitched text, so that the reader can see if something they read on the screen has since changed. Because things change. I allow the screen to glitch a bit on those places.~

Sometimes~.break .dance~invites the reader to look underneath the text, to see what lies underneath thought, the other place a thought comes from and to glimpse what it might mean to launch to elsewhere. I use {\em tendwa} symbols to show where these places currently are. Based in Congolese cosmology to denote porosity between ancestral timespaces and the known world--- dimensionality--- a {\em tendwa} invites you to listen for what is thrumming on the lower frequencies.~

.break .dance~also asks readers to look behind the project, to notice citation and origin. Citation is always in process, because I have been learning how I do not always know where things are coming from. I have switched my citational practice into a meditative practice, so those pages are always growing out by growing back.~You can read a bit about the background for the project here.

\thinrule

To access the project click \useURL[url1][\%7B\%7Bsite.baseurl\%7D\%7D/issue03/break-dance/index.html][][here]\from[url1]\{: target="_blank"\}.

To access the project process piece click \useURL[url2][\%7B\%7Bsite.baseurl\%7D\%7D/issue03/break-dance-process/index.html][][here]\from[url2]\{: target="_blank"\}.

\page
\subsection{Marisa Parham}

Marisa Parham is Professor of English at Amherst College, and directs the Immersive Reality Lab for the Humanities, which is an independent workgroup for digital and experimental humanities (irLh). irLh develops and incubates digital projects for AR, VR, and screen, and supports the work of digital scholars. Parham currently serves as faculty diversity and inclusion officer (FDIO) at Amherst College. Marisa Parham is also the author of {\em Haunting and Displacement in African-American Literature and Culture} (2008), {\em The African-American Student's Guide to College} (1998), and is co-editor of {\em Theorizing Glissant: Sites and Citations} (2015).

\stopchapter
\stoptext