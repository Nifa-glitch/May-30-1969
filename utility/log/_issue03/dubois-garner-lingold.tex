\setvariables[article][shortauthor={Dubois, Garner, Lingold}, date={July 9 2019}, issue={3}, DOI={https://doi.org/10.7916/archipelagos-j09v-vg80}]

\setupinteraction[title={The Caribbean Digital & Peer Review: A Musical Passage Hypothesis},author={Laurent Dubois, David Kirkland Garner, Mary Caton Lingold}, date={July 9 2019}, subtitle={The Caribbean Digital & Peer Review}, state=start, color=black, style=\tf]
\environment env_journal


\starttext


\startchapter[title={The Caribbean Digital & Peer Review: A Musical Passage Hypothesis}
, marking={The Caribbean Digital & Peer Review}
, bookmark={The Caribbean Digital & Peer Review: A Musical Passage Hypothesis}]


\startlines
{\bf
Laurent Dubois
David Kirkland Garner
Mary Caton Lingold
}
\stoplines


{\bf Featuring:} Kenneth Bilby, Marlene Daut, Anne Eller, Marc Fields, Rebecca Geoffrey-Schwinden, Kim Hall, Jessica Krug, Jeffrey Menzies, Patricia Van Leeuwarde Moonsammy, Gregory Pierrot, Pete Ross, and Richard Rath

\thinrule

What forms of peer review can best nurture and sustain the practice and potential of digital work in Caribbean studies? This question has animated {\em sx archipelagos} from its founding and, in the spring of 2017, this journal's editors posed it to us directly. Having \useURL[url2][http://archipelagosjournal.org/issue01/musical-passage.html][][reviewed and presented]\from[url2] in issue 1 of the journal a project we were then developing, the editors invited us one year later to participate in an experiment with our by-then-finished site. They proposed that we use the annotation tool \useURL[url3][https://web.hypothes.is/][][hypothes.is]\from[url3] as a way of inviting comments on the project from an array of specialists. Together we developed a plan to do so by facilitating a week-long opportunity for open commentary using this tool. It was our hope that by staging this period of focused online critique, response, and conversation, we might get a real sense of how others were actually using and experiencing \useURL[url4][http://www.musicalpassage.org/][][{\em Musical Passage: A Voyage to 1688 Jamaica}]\from[url4].

We set the date for this period of commentary for 19--23 June 2017 and sent out invitations to colleagues who had done or were doing research in areas related to the site. We cast a wide net, reaching out to Caribbeanists with a specialty in music, to scholars broadly interested in the black Atlantic, and to others working around the question of sound, music, and the digital, including two contemporary banjo makers. In addition, David Garner was teaching a summer course on music to high schoolers through the talent identification program at Duke University---Duke TIP---and these students also joined in the discussion. The only requirement for all participants was to create an account with hypothes.is in order to use the annotation software. We invited as much or as little commentary as people were able to offer.

In a letter sent to invitees, we described our goal as being \quotation{the creation of a commons space that will offer a community of readers a punctual opportunity~to enter into dialogue with contributors, leaving comments on and posing questions about their work.} We committed to respond to and engage with all comments as they were posted throughout the course of the week-long commons event. We advertised the project widely online and via social media, extending a broad invitation to participate. Ultimately, however, all those who contributed were people we had invited directly.

The event enabled us to create an interface between the site itself and an additional set of digital materials that had been generated by it in the time since its initial launch. Specifically, in March 2017, Laurent Dubois participated in a musical workshop organized at the Institute of Jamaica by Matthew Smith, chair of the Department of History and Archaeology at University of the West Indies--Jamaica, and Herbie Miller, curator of the Jamaica Music Museum. The workshop was organized around the material presented on {\em Musical Passage}. Selections from the site were played for a group of musicians led by Earl \quotation{Chinna} Smith. The musicians responded with their own interpretations of the songs, connecting them to other songs and forms over the course of several hours. In preparation for the {\em archipelagos} commons event, we edited and posted \useURL[url5][https://www.youtube.com/playlist?list=PLwcqVNt0EPDlGgT6ueKV_wjvPT1-tD_Y_][][videos]\from[url5] of these performances from the workshop. We invited participants to watch the videos and engage with them as they thought about and commented on the site.

The hypothes.is annotation software provided participants with a range of possibilities for responding to the material. \quotation{Page notes} allowed them to post general comments about the project as a whole, so we invited comments there about the overall design and future directions for the sites. We began the conversation in part by posting \useURL[url6][https://hyp.is/aB8tsjT9Eeey5C9oIr2voQ/www.musicalpassage.org/][][a general welcome]\from[url6] as the first \quotation{page note.} \quotation{Annotations,} meanwhile, could be tied to particular bits of text within the site, which were highlighted whenever one was created. Tags could be created as part of these annotations, providing an additional way to network comments around a particular topic.

The form of the annotation software offered multiple ways of engaging with the knowledge generated on the site itself. When these came in the form of annotations, they were usefully cataloged under the author's name, as with this collection of \useURL[url7][https://hypothes.is/users/rcrath][][Richard Rath's annotations]\from[url7]. Replies to such annotations, meanwhile---where, for example, Kenneth Bilby's extensive comments appeared---were not cataloged by author and so had to be located in other ways, something that might be improved in updates of the annotation software. In general, the format raised interesting questions about authorship, as the digital often does. Contributors appeared with the name of their hypothes.is account. For some, this was simply their name, but others used some kind of alias. (A full list of participants and their screen names is \useURL[url8][https://hyp.is/pBZX-kPTEemegFO8l_kIYg/www.musicalpassage.org/][][on the site]\from[url8]). For those involved in the discussion, the identities were relatively clear, but once the new object created by the collective experiment became its own digital artifact, these identities were less evident. In effect, the object itself probably needs to exist in relation to a commentary such as the one presented here in order to be fully legible in the digital space.

Once it began, the commons event rapidly became a delightful spiral of conversation. A question about the possible biographical stories we can tell about \quotation{Mr.~Baptiste,} the figure who produced the musical notation on the page, generated a \useURL[url9][https://hyp.is/72DokjUbEeeO-Kv7CZ8IeQ/www.musicalpassage.org/][][marvelous series of thirty-three responses]\from[url9], including comments from Gregory Pierrot, Hall, Rebecca Geoffroy-Schwinden, and Jessica Krug, with multiple threads and subthreads of dialogue. By the end of the week, this conversation had led the three of us site directors both to rethink the interpretation we present on the site and to pursue further research in order to expand our empirical foundations around this central question. Inspired in part by these lively debates about Baptiste's identity, Mary Caton Lingold went on to carry out research in Jamaica that ultimately helped to support our theories about the composer's identity and expertise. This new research, combined with the hypothes.is event of 2017, will form the basis for a revision of the site.

Another line of discussion opened up when the TIP high school students working with David Garner asked a \useURL[url10][https://hyp.is/XF2mqldzEeeOH0sSRdtUwg/www.musicalpassage.org/][][sharp question]\from[url10] that generated a richly detailed response from Kenneth Bilby, essentially a mini-essay on musical notation and the complexities of capturing the nuances of Jamaica music. This in turn generated \useURL[url11][https://via.hypothes.is/http://www.musicalpassage.org/\#annotations:XF2mqldzEeeOH0sSRdtUwg][][a response]\from[url11] from Richard Rath that also explored the varied ways ethnomusicologists and historians might try to understand what music sounded like in the past.

The opportunity to engage in an open forum with Bilby and Rath was particularly valuable. Rath's work was the major foundation for our own work on the site and Bilby's work on Jamaica and on Afro-Caribbean music more broadly had provided us intellectual inspiration and guidance throughout the project. In a traditional peer review process, Rath and Bilby would have been the most suitable expert reviewers for the project, but their comments would have been written for a closed circuit of authors and editors. In this commons experiment, we received expert responses from them, as well as from other participants in the forum, but in a way that allowed others to read and access the expertise they provided---and our responses to it---in an open forum. The fact that these insights were shared in such a way that something of a new publication emerged was one of the great benefits of the experiment.

The open-access nature of the online discussions was also of great value to us. These were the kinds of dialogues that normally might take place following a paper presentation at a conference and, as such, would only involve those in the room at the time and would be limited to a few moments during Q&A. While such exchanges are of course incredibly valuable for the development of one's research, they are generally accessible only to those with access to that particular academic space. Hypothes.is affords an opportunity, under the right conditions, to have much more open and inclusive conversations and thus to yield more open and inclusive scholarship.

The fact that this conversation took place in an online, multimedia format also created opportunities for a kind of dialogue that used music itself as a form of intellectual engagement. Bilby offered new insights into the interpretations of the seventeenth-century music during the workshop in Jamaica, \useURL[url12][https://hyp.is/WEVXAjWaEee3yrP-Zj8Brw/www.musicalpassage.org/][][explaining]\from[url12] that the musicians had moved from the song \quotation{Angola} into a particular song by Ras Michael and meditating on what it meant that they had done so. He \useURL[url13][https://hyp.is/j3N8zDWaEeebgDtAtrs3uQ/www.musicalpassage.org/][][led us]\from[url13] into the remarkable layers of meaning behind the interpretation of the song \quotation{Papa,} and the ways Earl \quotation{Chinna} Smith had transformed the original song into a traditional Rastafarian song, played in Nyabinghi ceremonies, called \quotation{Africa We Want to Go} or \quotation{Zion Land.} Bibly also explained that the song itself was derived from the 1950s Cuban song \quotation{Tabu.} This in turn prompted Patricia Van Leeuwarde Moonsammy to \useURL[url14][https://hyp.is/j3N8zDWaEeebgDtAtrs3uQ/www.musicalpassage.org/][][push us]\from[url14] to think more about precisely what kind of emotional and spiritual relationship underpinned the playing of these songs. She suggested that we offer more documentation about the process of engagement that unfolded around the workshop. And Jeffrey Menzies, a Jamaica-based instrument maker who was present at the workshop, \useURL[url15][https://hyp.is/HXUdMFZFEeeyOItnayd2ow/www.musicalpassage.org/][][joined in to tell us]\from[url15] that he had followed up by doing recordings of himself playing on various banjos he was crafting in the older style. In this way, the generative nature of the original workshop continued in the digital space, giving us new ideas for other potential musical encounters and workshops in Jamaica and elsewhere.

Bilby's comments during this project represented the first sustained analysis we had read of the workshop in Jamaica, effectively producing a kind of paper---in multimedia and conversation form---about how this event offered new insights into the Sloane material. As \useURL[url16][https://hyp.is/dh2lklUoEeeiZesVMvM8HA/www.musicalpassage.org/][][Bilby noted]\from[url16], the act of interpretation by these musicians was a demonstration of \quotation{how complex and multiply meaningful this web of diasporic musical interconnections} is in the Caribbean and beyond. As he pointed out, the musicians had moved from playing a version of \quotation{Koromanti 1} into the song \quotation{St.~Thomas,} best known through a Sonny Rollins interpretation. However, as Bilby then pointed out, the Rollins version itself has roots in an older Jamaican tune. These comments illuminated what had not been visible to us on first encounter with the workshop and the videos, namely, the layers and layers of commentary and understanding the musicians were offering through their performance.

The use of hypothes.is thus provided the appropriate format for a scholarly engagement with the workshop we had organized. The space was multimodal and allowed for the videos and the music posted by Bilby to be presented together. The acts of reading, listening, and historical and cultural interpretation were all intertwined, offering a new form of interpretive practice. This was also the case in a \useURL[url17][https://hyp.is/DW5SSlXWEeeEEB83jLWSsg/www.musicalpassage.org/][][fascinating thread of discussion]\from[url17] involving Gregory Pierrot, Kim Hall, and David Garner around the question of how we might include the call of \quotation{Alla, alla} in future recordings.

One of the central questions raised by the materials on the site has to do with how to think about the terms used to describe African groups or affiliations in the context of Atlantic slavery. Our hope---our sense---is that in offering a set of musical pieces linked to particular terms, such as {\em Angola}, {\em Papa}, and {\em Koromanti}, these documents might provide new insight into what such categories meant to people of African descent themselves. In one annotation, we asked, \quotation{{[}Can{]} these songs . . . actually give us a better understanding of what \quote{Koromanti} meant as a term in seventeenth century Jamaica{[}?{]} How might we read back from the music? What does the fact that three songs that are so different are all called \quote{Koromanti} signify about the term and its meaning at the time?} The responses to this question were deeply insightful, amounting to a sort of \useURL[url18][https://hyp.is/LgtBnjWbEee6YnsGxVM5tQ/www.musicalpassage.org/][][a mini-symposium]\from[url18]. Such a conversation could have happened in a conference setting but, thanks to the hypothes.is format, is available here as its own kind of public forum. Moreover, this conversation animated other discussions, inspiring vital insights from \useURL[url19][https://hypothes.is/users/Jessica_Krug][][Jessica Krug]\from[url19] about the term {\em Koromanti}. This in turn generated a \useURL[url20][https://hyp.is/vf5BZFddEeeQMyfszBXX9g/www.musicalpassage.org/][][new thread]\from[url20] of discussion. These webs of conversation became so intricate that we began posting, in the replies, links to other threads within the page itself. And Krug pushed us to think in new ways about the material when she argued that the \quotation{plurality of musical forms} that appears under this term in the Hans Sloane biography currently on the site offers us an opportunity to think in complex and deeper ways about precisely what this term meant \quotation{for those using it the in the seventeenth century.}

At the end of the week, Patricia Van Leeuwaarde Moonsammy raised a set of \useURL[url21][https://hyp.is/_57FhlhCEeekfyd-QZq8bA/www.musicalpassage.org/][][key questions]\from[url21] about how to improve the process for future experiments. Her weaving together of the threads helped us lay out a set of \useURL[url22][https://hyp.is/yGW5XFh6EeerIWt4TdmnVA/www.musicalpassage.org/][][proposed responses]\from[url22] (including writing this piece) to what we read during the course of the week. Other participants offered a number of specific suggestions as well. These included some \useURL[url23][https://hyp.is/sThQlDWaEeegNiPXRncLMw/www.musicalpassage.org/][][design ideas]\from[url23] from Marc Fields; Marlene Daut's push for us to think about other possible examples of such transcribed music; and \useURL[url24][https://hypothes.is/users/aee54][][Anne Eller]\from[url24]'s suggestion that we offer \quotation{collective biographies} of the enslaved who produced the music, so as to balance out and sit alongside the longer Sloane document.

Our active engagement, in real time, with the comments that were being posted enabled us to pose new questions opened up by the responses we got, multiplying conversations and their impact over the course of the work. And the collective experiment ended with a commitment on our part to absorb and react to the comments we received. This was in a sense comparable to the kind of response letter we might submit as part of a closed peer-review process but declarative in a different way, since it was issued to a now-constituted community. We were mindful of the fact that all the contributors had spent significant time and effort participating in this process. And while grateful, on the one hand, that the efforts they made are visible and can be read on their own terms by those who visit the space, we also, on the other hand, did leave with a sense of responsibility to respond, in time, with revisions to the original site. The depth of engagement and of critique itself has made this a longer-term project, but it is also a commitment that has encouraged us to keep working on and thinking about a project that we might otherwise have considered finished.

The open-ended nature of the discussion, which sparkled with possibilities, was itself a particularly inspiring way to think about critique and collaboration around the digital. That is partly because of the multiple avenues it opened up. The experiment pushed us to concretize what until then was a somewhat abstract hope, that is, to edit and post the films of the workshop in Jamaica. Having a community of readers we knew would engage with these materials was critical to getting that done. This was a lesson in itself, because we had in some sense been stalled by a set of design questions---Should we incorporate the videos directly into the site somehow? Would that alter and undermine our original design concept? These concerns were superseded by this conversation with colleagues that had to take place in the moment. The hypothes.is even led us to create a set of videos that have now lived on and produced other effects on their own terms. The discussions that took place, in turn, enabled us to envision other possible forms of design integration that would make use of the existing videos as well as other similar visual and sonic materials that might emerge from future iterations of the project.

From the very inception of {\em Musical Passage}, we had willed it not to be a site of permanence, understanding that this is a quixotic mission in the digital space, but rather to be something that could generate concrete conversation and new forms of knowing. This collective experiment in discussion and critique did just that. Ultimately what was produced over the course of the week was \useURL[url25][https://via.hypothes.is/http://www.musicalpassage.org/][][a new collective object]\from[url25], coauthored and multivocal, a layering of knowledge and interpretation that exists on its own terms as a new digital offering. Further, and perhaps more important, this week-long experiment in live collaboration generated what we are confident will be a dynamic, ongoing community.

\page
\subsection{Laurent Dubois}

Laurent Dubois is a professor of Romance studies and history and the founder and faculty director of the Forum for Scholars and Publics at Duke University. He is the author of seven books, including {\em The Banjo: America's African Instrument} (Harvard University Press, 2016) and the forthcoming {\em Freedom Roots: Histories from the Caribbean} (University of North Carolina Press), co-authored with Richard Turits.

\subsection{David Kirkland Garner}

David Kirkland Garner is an assistant professor of composition and theory at the University of South Carolina. He seeks to make time and history audible, particularly through an exploration of archival recordings documenting the musical traditions of the US South. Garner's first album,~{\em Dark Holler}, was released to critical acclaim in 2017 on New Focus Recordings.

\subsection{Mary Caton Lingold}

Mary Caton Lingold is an assistant professor of English at Virginia Commonwealth University, where she she is writing a book about the literary history of African music in the Atlantic world (1630--1830). Her 2017 essay on early Caribbean music was recognized as cowinner of the Richard Beale Davis prize for best essay in {\em Early American Literature}. She coedited {\em Digital Sound Studies} (Duke University Press, 2018) and \useURL[url1][https://soundcloud.com/c19podcast/tena-too-sings-america-listening-to-an-enslaved-womans-musical-memories-of-africa][][produced a podcast]\from[url1] on the transatlantic journey of a song by an enslaved woman named Tena.

\stopchapter
\stoptext