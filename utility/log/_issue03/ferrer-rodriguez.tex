\setvariables[article][shortauthor={Rodríguez, Ferrer}, date={July 9 2019}, issue={3}, DOI={10.7916/archipelagos-mq9x-dd28}]

\setupinteraction[title={Collaborating with Aponte: Digital Humanities, Art, and the Archive},author={Linda M. Rodriguez, Ada Ferrer}, date={July 9 2019}, subtitle={Collaborating with Aponte}]
\environment env_journal


\starttext


\startchapter[title={Collaborating with Aponte: Digital Humanities, Art, and the Archive}
, marking={Collaborating with Aponte}
, bookmark={Collaborating with Aponte: Digital Humanities, Art, and the Archive}]


\startlines
{\bf
Linda M. Rodriguez
Ada Ferrer
}
\stoplines


{\startnarrower\it In the last fifteen years or so, the story of José Antonio Aponte---the free black carpenter and artist who attempted to lead a major antislavery revolution in early-nineteenth-century Havana---has drawn the animated attention of scholars in history, art history, literary criticism, and anthropology, among other fields. Most of that scholarship has utilized time-worn techniques of scholarly practice: rigorous and wide-ranging archival research, perceptive close readings of judicial testimony, forays into theoretical works as interpretive aids to analysis. This essay revisits the story of Aponte, relying on the digital humanities website {\em Digital Aponte} that we---Linda Rodríguez, in collaboration with Ada Ferrer---curated. {\em Digital Aponte} invites an interdisciplinary collaboration that can be said to echo the collaborative approaches Aponte himself pursued in his creative, intellectual, and political practices. In the present, such approaches advance our understandings of Aponte and the histories of black antislavery in the Atlantic world, just as they foreground the profound link between artistic and revolutionary work. \stopnarrower}

\blank[2*line]
\blackrule[width=\textwidth,height=.01pt]
\blank[2*line]

In Havana on the night of 14 March 1812, three free men of color left the capital city for the countryside. One of the men wore a uniform, blue with gold buttons with eagles on them. He spoke French and went by the name Jean-François, general of the Haitian Revolution, even though the real Jean-François had died in Spain a few years earlier. It was harvest season, and the men were heading to sugar plantations to begin a revolution against slavery. When they arrived at the estate called Peñas Altas, they gathered the slaves. Jean-François read in French from papers he insisted were his king's orders for the liberation of Cuba's slaves. (In fact, the papers included an advertisement for William Young Birch, publisher and stationer in Philadelphia.) He told them that he was there on orders of his king to lead them in revolution. Together they burned buildings and sugar cane and killed a few whites. But at the third plantation, the rebels were defeated and captured.

As authorities questioned these three men and others, they learned the outlines of an ambitious plot. In the capital, the rebels had planned to launch separate attacks on the city's fortresses and armories, seizing weapons with which to arm hundreds of recruits they said were awaiting the signal to rise up. One of the leaders had dictated a public proclamation and nailed it to a side door of the governor's palace. They had flags and standards ready to post at the camps they planned to establish. Their networks stretched from plantations to the heart of the capital city and, according to some contemporary accounts, as far east as Santiago. And all this labor and planning was directed to one clear goal: freedom for the enslaved.

In all the investigation into the conspiracy, one name appeared over and over: José Antonio Aponte. Aponte was a free black man, born in Havana sometime around 1760. He was a veteran of Spain's free black militia in Havana, where his grandfather and father had also served. His grandfather had defended the city against the British siege of 1762. Aponte himself participated in the capture of The Bahamas from the British during the American Revolution. Aponte was also a carpenter, with a workshop at the Plaza del Cristo, just inside the city wall. He made cabinets and carved fine religious statues for the city's churches. His carving of Jesús Peregrino (Jesus Pilgrim), which he placed over the door to the house where he lived with his wife and children, was so impressive that his street came to be known by the same name.

Aponte's home and workshop were both popular places for black militia members to visit. Sometimes they came by to borrow books or to lend them. Aponte liked to draw. When he heard of an image arriving in Havana---for example, of Haitian revolutionary hero Jean François---he would head to the docks to purchase or borrow it. Then he would copy the image and lend his copy to others. Sometimes people would come visit just to see his drawings of such figures. Aponte also collected images. He owned one, for example, of George Washington; he had others related to the French Revolution and the Haitian Revolution and to Spanish kings and saints. In his house, he had a library of more than a dozen books: the third volume of {\em Don Quijote}, guides to Havana and Rome, histories of Ethiopia published in seventeenth-century Spain, grammar handbooks, art manuals, compendiums of the history of the world.

And hidden deep inside a trunk full of clothes was a pine box with a sliding top, and inside this smaller box was another book, one created by Aponte himself---his so-called {\em libro de pinturas}, or book of paintings. When authorities discovered the book during their investigation into the conspiracy, they likely figured they had their man. One of the first images they would have found in the book, located at pages 6 and 7, seemed to portray a battle scene between a black army and a white one. What is more, the black army appeared to be defeating the white one. In one corner of the picture were two black soldiers on horseback, each brandishing a white head covered in blood. Immediately convinced of Aponte's subversion, authorities confiscated the book and then arrested its creator.

As the investigation continued, authorities pored over Aponte's book. Like the other items confiscated from his house, the book's contents were a confounding mix of materials and images---hand-drawn pictures and maps, scenes or words cut from fans and prints and pasted onto the pages. Represented was a diverse array of subject matter: Greek goddesses and black saints; popes and kings in Rome, Ethiopia, and Spain; distant and local armies; Havana and the heavens. Authorities also soon learned that Aponte had shown the book to his coconspirators, explaining some of its images as a way to help prepare for the coming revolution. He had shown one companion the fascinating and complex \quotation{self-portrait} on pages 24 and 25, supposedly to help picture himself as leader of a new order without slavery; he had shown them pictures of the camps of the black militias that had protected Havana against the British in 1762 to illustrate how their own rebel camps should be organized. Though Aponte had most likely started working on his book around 1806--7, by 1812, when he shared it with others plotting revolution, he drew sometimes explicit links between his artistic creation and his political movement. He showed them his battle scenes to illustrate how they should organize their own fight against slavery in Havana. He pointed out pictures of powerful black men and of himself as would-be king as a way of illustrating that other worlds were possible. To explain and imagine revolution and its object, the book was instrumental.

For Spanish and slaveholding authorities, the book was instrumental in another way: it helped seal the fate of its creator. Just a few weeks after the discovery of the book, Aponte was condemned to death. On 9 April 1812, he was hanged in public, his head severed from his body and placed on a pike, which was then placed inside a cage strategically erected at an important crossroads in the city, very close to Aponte's own home. Then, sometime after the execution, the book disappeared. So for this revolution that never quite was, perhaps the most important source is a book that no modern scholar has ever seen.

Aponte scholars, of course, have looked for it.\footnote{See, for example, José Luciano Franco,~{\em Las conspiraciones de 1810 y 1812~}(Havana: Editorial de Ciencias Sociales, 1977); Stephan Palmié,~{\em Wizards and Scientists: Explorations in Afro-Cuban Modernity and Tradition} (Durham, NC: Duke University Press, 2002); Sibylle Fischer,~{\em Modernity Disavowed: Haiti and the Cultures of Slavery in the Age of Revolution}~(Durham, NC: Duke University Press, 2004); Matt Childs, {\em The 1812 Aponte Rebellion in Cuba and the Struggle Against Atlantic Slavery} (Chapel Hill: University of North Carolina Press, 2006); Jorge Pavez Ojeda, \quotation{Expediente contra José Antonio Aponte~y~el sentido de las pinturas que se hayan en el Libro~que se le aprehendió en su casa:~1812,}~{\em Anales de Desclasificación} 1, no. 2 (2006): 717--68; Jorge Pavez Ojeda, \quotation{Lecturas de un códice afrocubano: Naturalismo, etiopismo y universalismo en el libro de José Antonio Aponte (La Habana, circa 1760--1812),} {\em Historia crítica} 45 (2011): 56--85; Jorge Pavez Ojeda, \quotation{Painting of Black History: The Afro-Cuban Codex of José Antonio Aponte (Havana, Cuba, 1812),} in~Adrien Delmas and Nigel Penn, eds., {\em Written Culture in Colonial Context: Africa and the Americas, 1500--1900} (Leiden: Brill, 2012), 271--303; Ada Ferrer,~{\em Freedom's Mirror: Cuba and Haiti in the Age of Revolution} (Cambridge: Cambridge University Press, 2014); and Juan Antonio Hernández, {\em Hacia una historia de lo imposible: La revolución haitiana~y el Libro de pinturas de José Antonio Aponte} (Caracas: Fundación Editorial el Perro y la Rana, 2015).} Historian Matt Childs, for example, uncovered the fact that the Spanish governor of Cuba, who had ordered Aponte's execution, asked to personally view some of the material from the trial on the evening before his return to Spain (just days after Aponte's execution).\footnote{See Childs, {\em The 1812 Aponte Rebellion in Cuba}.} Following that lead, Ada Ferrer sought the book in the papers of that governor's family, held in the \useURL[url1][https://www.mecd.gob.es/ca/cultura/areas/archivos/mc/archivos/nhn/portada.html][][Archivo Histórico de la Nobleza]\from[url1] in Toledo, Spain. Taking seriously Aponte's claim to have created the book as a gift for the king of Spain, Ferrer searched in a section of the \useURL[url2][https://www.patrimonionacional.es/colecciones-reales/archivo-general-de-palacio][][Royal Palace Archives]\from[url2] called Locos y Anónimos, which houses letters and gifts to the Crown made by unknown or \quotation{insignificant} people.

For the most part, these were old-fashioned analog searches that led to many interesting things but never to the book. Yet the book's absence has not deterred scholars. Instead, they (we) have reveled in analyzing this important missing artifact, learning to work around and even to love its absence.

\subsection[reference={apontes-archive},
bookmark={Aponte's Archive},
title={Aponte's Archive}]

Without the book, scholars have relied on another fascinating historical document: the transcript of the part of Aponte's trial devoted to his book of paintings. Aponte's descriptions of his work come to us through this complex, layered archival document. The layers of the archival record register the agency of the colonial state and Aponte himself. A colonial notary recorded Aponte's words. As Kathryn Burns reminds us, notaries \quotation{gave records their words and final form}--- making myriad decisions about how to edit the archival record.\footnote{Kathryn Burns, {\em Into the Archive: Writing and Power in Colonial Peru} (Durham, NC: Duke University Press, 2010), 39.} In addition to the decisions of the notary, we encounter the decisions of the judicial officials questioning Aponte as to the contents of his book of paintings. They viewed the book with suspicion, and they asked Aponte more questions about certain images than others. As a result, the archival document records more extensive descriptions of a page such as {\em lámina} 37, with its many important black religious officials in Rome, than of later láminas with their very brief descriptions. Finally, Aponte also shaped the archival record. He decided how much information to share, and he did so while on trial for his life. In fact, Aponte may have created some images to be deliberately ambiguous, and he described the images, at times, in ways that evidence this ambiguity.\footnote{Ferrer,~{\em Freedom's Mirror}, 315.}

The fragments of Aponte's testimony, read through these layers, often appear as frustrating glimpses into the book, never quite complete. Yet as the powerful work of scholars of slavery demonstrates, grappling with the \quotation{impossibility of recovery} necessitates the improvisation of new historical methods.\footnote{Laura Helton, Justin Leroy, Max A. Mishler, Samantha Seeley, and Shauna Sweeney, \quotation{The Question of Recovery: An Introduction,} {\em Social Text} 33, no. 4 (2015): 1.} Engaging with fragments and silences forces us to see the archive as \quotation{home to the counter-narrative, or at least to its possibility.}\footnote{Jennifer L. Morgan, \quotation{Archives and Histories of Racial Capitalism: An Afterforward,} {\em Social Text} 33, no. 4 (2015): 154.}

For instance, Ferrer's {\em Freedom's Mirror} pursues one method of working within the archive to probe meanings in the book's absent presence. Using the court testimony, Ferrer compares the way Aponte described particular image to authorities and the way he described those same images to his coconspirators. Through that kind of comparison we learn, for instance, that picture 37, which Aponte described to officials as depicting important black religious men in Rome, may also (or instead) have represented Henri Christophe (which is what Aponte had told one of his coconspirators). Using other archival sources, Ferrer explores what some of the elements that appeared in the book might have meant to Aponte and to his companions. The enigmatic picture numbered 8--9, which Aponte referred to as an allegory about greed and commerce, depicted the planet Mercury of Gemini in a carriage pulled by two large rapacious birds. There was a green star and a {\em caduceo}, a staff signifying the progress of commerce. Also pictured was a guard attempting to stop contraband, who encountered death. A launch from the ship approached; avarice jumped on the dock and also encountered death. In passing, Aponte mentions the name of the ship, the {\em San Lorenzo}. Mining other sources, we learn that the {\em San Lorenzo} had transported black generals from the Haitian Revolution to Havana in 1795, had participated in the victorious Spanish siege of Bayajá, and was the ship on which at least two of Aponte's coconspirators had served as militiamen. Here, then, may have been a trace of an alternative allegory. Aponte's testimony to authorities rendered the ship element of the drawing faint and elusive, buried, like so many of his images, not just by the subsequent loss of the book but also by Aponte himself, by his strategic evasions and deflections, by the disavowals he himself was forced to effect in the courtroom.

This particular portrait is but one example of many, for other pictures as well undoubtedly contained elements that would have been obscure to authorities but that evoked different histories, memories, and resources for the conspirators. Thus the glorious black history pictured in Aponte's book and so visible to the eyes of all who saw it---of black kings and cardinals and generals and emperors of Ethiopia and Egypt---was accompanied by another black history, more subterranean yet equally subversive.

The process of identifying and tracking all such potential connections, however, seems almost interminable. Some potential leads become dead ends; others lead to dozens of other possible paths. The book---or rather the description of the book in the trial testimony---is so dense and so wide ranging in its references that it is impossible for any one person to pick up its many layers of references and allusions. It would take someone deeply familiar with Cuban history, and Spanish; with Greek and Roman history and mythology; with astronomy; with Afro-Cuban religion and freemasonry and the Bible; with early modern writing on Egypt and Ethiopia; with conventions of artistic practice in Aponte's time, and so on. Indeed, the person capable of following---much less understanding---the many intricate, contradictory references in Aponte's testimony likely does not exist.

Aponte and his story, put simply, are too big for one scholar or one method. Instead, Aponte invites a vast, new kind of collaboration in which many different people across many different fields and specialties can read Aponte's description of the book together, each potentially building on the other, each potentially taking another's interpretation in a new direction. All this without forgetting the concrete social world in which Aponte lived and created.

Here, the digital humanities---by fostering both critical methodological innovation and interdisciplinary collaboration---are especially promising. They provide frameworks for opening up the archive and engaging with the imperfect record available to us, reframing \quotation{the archive itself as a site of action rather than as a record of fixity of loss.}\footnote{Lauren F. Klein, \quotation{The Image of Absence: Archival Silence, Data Visualization, and James Hemings,} {\em American Literature} 85, no. 4 (2013): 665.} For curating our \useURL[url3][http://aponte.hosting.nyu.edu/][][{\em Digital Aponte}]\from[url3] project (see fig. 1), transforming Aponte's descriptions into a site of action meant first making that archival record as accessible as possible.\footnote{See {\em Digital Aponte}, aponte.hosting.nyu.edu. We would like to acknowledge the contributions of New York University graduate students Kris Minhae Choe and Eric Anderson to improving and enhancing {\em Digital Aponte}.}

\placefigure[here]{Screenshot of {\em Digital Aponte}.}{\externalfigure[issue03/ferrer-1.png]}


{\em Digital Aponte} aspires to foreground Aponte's words, through the archival document, as they have arrived to us. Our effort may be \quotation{predicated on impossibility,}\footnote{Saidiya Hartman, \quotation{Venus in Two Acts,} {\em Small Axe}, no. 26 (June 2008): 2--3.} but it aims to provide a platform for the kind of innovative methodology and collaboration that Aponte and his story require and deserve. First, under the tab labeled \quotation{Book of Paintings,} we introduce the archival document (\quotation{\useURL[url4][http://aponte.hosting.nyu.edu/transcript/][][Trial Transcript]\from[url4]}) by explaining the circumstances of its production before presenting Aponte's actual descriptions of his book (\quotation{\useURL[url5][http://aponte.hosting.nyu.edu/book-of-paintings/][][Láminas]\from[url5]}).

Opening up the archive and attending to Aponte's world also meant putting the archival record in conversation with Aponte as an intellectual and artist. We wanted to engage with the possible influences that contributed to his development as a thinker and creator, ultimately leading to the production of the book of paintings. In this way, the archive inspired us to design the website so it also becomes a tool to communicate our \quotation{sense of history's possibilities.}\footnote{Vincent Brown, \quotation{Mapping a Slave Revolt: Visualizing Spatial History through the Archives of Slavery,} {\em Social Text} 33, no. 4 (2015): 134.} The digital allows for this capaciousness. We can present to the site viewer these aspects of Aponte all at once, individually, or sequentially. The historian and digital humanist Jessica Marie Johnson observes that this is the \quotation{nature of a form that is chimeric, elusive, and eerily transparent at the same time---digital and social media cannot be read forward or back as if it were chapters in a book.}\footnote{Jessica Marie Johnson and Kismet Nuñez, \quotation{Alter Egos and Infinite Literacies, Part III: How to Build a Real Gyrl in 3 Easy Steps,} {\em The Black Scholar} 45, no. 4 (2015): 47.} The design of the website presents the viewer with the ability to choose the order in which they engage with Aponte's work of art and his ostensible formative influences. The viewer may choose to contemplate a recreation of Aponte's library (\quotation{\useURL[url6][http://aponte.hosting.nyu.edu/apontes-library/][][Aponte's Library]\from[url6]}), which consisted of a dozen volumes across a wide range of topics. Sometimes, when interrogators asked Aponte how he knew about a story he depicted in his book, he responded by citing a volume in his collection. By making Aponte's library accessible on the website, viewers can consider the written (and possibly illustrated) works that influenced Aponte's intellectual formation and contributed to the construction of the book of paintings. Site users may also consider the visual culture of Aponte's time, through an image gallery (\quotation{\useURL[url7][http://aponte.hosting.nyu.edu/havana-visual-culture/][][Image Gallery]\from[url7]}), and the possible sources of inspiration for both the content and form of his illustrations in the book of paintings. Lastly, a map of Havana (\quotation{\useURL[url8][http://aponte.hosting.nyu.edu/apontes-havana/][][Aponte's Havana]\from[url8]}) plots sites relevant to Aponte's life and also locations he represented in his book, providing the basis for the contemplation of Aponte's own experience of Havana. {\em Digital Aponte} presents different {\em caminos} (paths) to knowing Aponte and his work of art, echoing the different caminos that Cuban {\em orichas} (deities) follow upon consecration.

\subsection[reference={the-book-the-trial-and-collaborative-interpretation},
bookmark={The Book, the Trial, and Collaborative Interpretation},
title={The Book, the Trial, and Collaborative Interpretation}]

The \quotation{\useURL[url9][http://aponte.hosting.nyu.edu/book-of-paintings/][][Book of Paintings]\from[url9]} page introduces the contents of Aponte's book through a table that provides a link to each numbered image, describes the contents of that image, and indicates whether annotations are available. The table allows for the visitor either to read the contents of each page before deciding where to begin reading the book or to simply go directly to a page and start exploring. We wanted to create micro and macro ways to engage, so to speak, with the book. When the viewer clicks on a particular page of the book, she encounters the original Spanish text of the archival document on the left and, if applicable, an annotation sidebar (\quotation{Commentary}) on the right. Longer annotations can be expanded via a \quotation{Read More} link and collapsed via a \quotation{Read Less} link.

The collection of annotations of Aponte's trial record is the part of the website that, we hope, will be most dynamic. Presently, the annotations included are from already existing scholarship on Aponte. Our goal is to replace those with annotations written specifically for the website by those same scholars. Annotations written for the project could be shorter and more streamlined than the ones excerpted from published works. Also, scholars could respond to each other's annotations---perhaps suggesting differing interpretations or adding new elements---in ways more dynamic than the current, more static excerpts. Ultimately, however, our goal is to open up the annotation process beyond the relatively small circle of Aponte scholars whose work is already represented on the site. We hope to invite contributions from scholars working in fields that Aponte's art engages---classicists, or students of early modern Ethiopia, for example. There is significant speculation about Aponte's possible involvement with freemasonry and about his possible priestly role in Santería. Scholars (or leading practitioners) of either might provide insightful and novel commentary on Aponte's use of symbols and stories. For now, however, this ambitious, collaborative process of annotations remains in very early planning stages.\footnote{As the annotations grow, the site will have to experiment with ways to present the annotations visually on the page, perhaps moving them to the bottom (rather than the side) of the screen or making only the first line of the annotation visible before the \quotation{Read More} link, for instance.}

The idea of making {\em Digital Aponte}\quote{s annotated and annotatable version of Aponte's trial testimony an exercise in modest crowdsourcing mirrors the ways Aponte's book of paintings was itself the product of collaboration and exchange. Aponte famously used the artistic practice of collage, fully a hundred years before the date art historians generally identify as its origin. Collage might itself be understood as a method that allows the creator literally to use the work of others. Aponte testified that \quotation{not being a painter, he bought different prints and paintings to take from them, or from used fans, that which fulfilled his idea.} As Linda Rodríguez has noted, Aponte repurposed others} images, using them to create his own meaning.\footnote{Linda Rodríguez, \quotation{\quote{No siendo pintor}: José Antonio Aponte and the Possibilities of Art and Social Change in Colonial Havana,} in Zuleica Romay and Ada Ferrer, eds., {\em Los mundos de Aponte} (Havana: Instituto Juan Marinello, forthcoming).} Similarly, because he had less confidence in his drawing and painting skills, he sometimes had the apprentices in his carpentry workshop paint parts of some láminas---for example, 37 and 46---to incorporate into his book of paintings. Even some of the ideas for the images arose from social interactions---for example, informative conversations with other Havana denizens about the existence of black priests in Rome, or readings about San Antonio Abad in books shared among local free people of color. Aponte's book was his, but it was also a product of complex intellectual, social, political, and cultural interactions and collaborations. {\em Digital Aponte}'s collaborative annotation of the trial transcripts promises to mirror that very practice.

Not only was the subject matter in Aponte's book wide ranging, even individual láminas were. Each often combined elements from different repertoires in unexpected and original ways. So, too, must analysis of the book reflect that expansiveness and, for lack of a better word, syncretism. The annotation tool on the site has the potential to incorporate the insights of people in disparate fields, scholars with expertise in areas that can help shed light on Aponte's art and worldview. Because the book is full of classical references, for instance, classicists might have important things to contribute to an analysis of the book: insights into the meaning of classical symbols, say, or an ability to recognize when Aponte's use of those symbols diverges from a more customary invocation. Yet a classicist may be less familiar with how classical references might have operated in nineteenth-century Caribbean slave societies. Like Aponte cutting out images for his own purpose, we can reinterpret or reuse elements from other fields and genres to elucidate Aponte's world.

To illustrate how we might develop and apply this method, we turn to Aponte's description of his lámina 26. Aponte there depicts the Greek philosopher Diogenes inside a barrel on a desolate beach, enjoying the protection of the Egyptian goddess Isis. The Spanish Visigothic ruler King Rodrigo orders Diogenes out of the barrel. Then, taking two fistfuls of dirt from the barrel, the philosopher fashions the Spanish coat of arms and the flag of Spain. That kind of magic was not in the philosopher's purview, but he was able to do it, Aponte said, because of the guidance and protection of Isis.

A classicist interpreting that image might help us understand some of the possible meanings of Isis, the Egyptian goddess who was incorporated into Greco-Roman pantheons. A classicist might elaborate on Isis's customary association with the protection of the dead and with magical powers (which Aponte himself refers to in his testimony). A classicist would know that Isis appears in art usually as a column-like woman and could provide visual clues about what Aponte might have drawn. But that same classicist may not know about the association of Isis with Yemayá, one of the seven orichas in Afro-Cuban religion.

At the same time, a classicist familiar with the range of representations of Isis might immediately note that the pairing of Greek philosopher Diogenes with Isis seems idiosyncratic, outside the customary repertoire of classical invocations. The observation draws our attention to the link Aponte establishes between the two unlikely allies---the magic of Isis allows Diogenes to do magical things. What might that link signify? Here, Diogenes becomes more than Cynic philosopher. Diogenes, importantly, was a man formerly enslaved; in fact, early radical abolitionists, such as Benjamin Lay, used his figure to elicit and promote antislavery meanings.\footnote{See Marcus Rediker, {\em The Fearless Benjamin Lay: The Quaker Dwarf Who Became the First Revolutionary Abolitionist} (Boston: Beacon, 2017).} Aponte visualizes Diogenes as receiving the assistance of the powerful Isis, putting divine power on the side of liberation. The result of this kind of analysis is a black intellectual history that is profoundly collaborative and that engages subfields of intellectual history often artificially distant from the study of the black Atlantic.\footnote{See Laurent Dubois, \quotation{An Enslaved Enlightenment: Rethinking the History of the French Atlantic,} Social History 31, no. 1 (2006): 1--14; and Susan Buck-Morss, {\em Hegel, Haiti, and Universal History} (Pittsburgh: University of Pittsburgh Press, 2009).}

\subsection[reference={apontes-social-worlds-in-digital-form},
bookmark={Aponte's Social Worlds in Digital Form},
title={Aponte's Social Worlds in Digital Form}]

As Aponte's testimony necessarily forces us out of our comfort zone and into Greek philosophy or Egyptian religion or medieval Spain, it also repeatedly brings us back to the port city of Havana in the Age of Revolution. How, we ask, did Aponte learn about the things he depicted in his book of paintings?

{\em Digital Aponte} provides part of the answer in the section on Aponte's library. The recreation of the library foregrounds Aponte's possible intellectual influences. A brief introduction outlines possible connections between his library books and themes in the book of paintings. On the website, images of the library books sit on a bookshelf to reinforce the notion that Aponte perhaps kept his books in a similar fashion and to call to mind the physical space of his home and workshop. Clicking on a book's spine takes the viewer to a page dedicated to that particular book. Each of these pages contains commentary from scholars about how they identified the particular version of each book as the one Aponte likely owned, along with an image of the cover page and a link to a Google Books copy, if available. The recreation of Aponte's library also encourages contending with how communities of free people of color in the Atlantic world accessed and shared books.

But understanding Aponte's conceptual world requires that we go beyond his library. Aponte's book of paintings was a visual artifact, one we cannot see. {\em Digital Aponte} recreates the visual worlds with which Aponte engaged. The image gallery includes an overview of visual culture of turn-of-the-nineteenth-century Havana along with a sampling of images. The image gallery proceeds from the premise of a serious consideration of Aponte as artist. What were his possible visual influences in Havana? How did he respond to the architecture and art that surrounded him? Some of the images represent buildings he depicted in his book, for instance, helping us to visualize aspects of a work inaccessible to us. Other images depict art or architecture he may have seen in his daily life. Each thumbnail in the image gallery links to a separate page with contextual information. Positing Aponte's visual influences contributes to a fuller understanding of black artistic production. During Aponte's lifetime, white elites sought to reclaim the visual arts from the hands of black artists who dominated the field. As such, black artists from the colonial period are often underanalyzed in art histories about this era. The image gallery, in this regard, suggests we ask, What if we wrote an art history of colonial Cuba from the perspective of one of these artists?

The map section of {\em Digital Aponte} allows us to make Aponte's world more concrete. The communication---the transfer of knowledge, ideas, aspirations, feelings---that underwrote Aponte's movement was embedded in the physical city. By pointing out the places that were part of his own routines---his workshop, his home, the barracks where he had trained, and so on---and by pointing out the places that appeared in his book of paintings, the user gets a sense of the physical distance between sites important to Aponte and between those sites and critical spaces of exchange and dialogue (markets and docks, for instance). We glimpse, in other words, the literal paths Aponte walked. Ultimately, this section of the website might serve as a template for a walking tour of Aponte's Havana. Such a walking tour would help not only to preserve Aponte's memory but to open spaces (and paths) for more critical historical narratives and even perhaps for other kinds of tourism, as is being pioneered in Brazil with the \useURL[url10][http://passadospresentes.com.br/site/Site/index.php][][{\em Pasts Present}]\from[url10] and \useURL[url11][https://apublica.org/2017/07/museum-of-yesterday/][][{\em Museum of Yesterday}]\from[url11] projects.\footnote{See {\em Passados Presentes}, \useURL[url12][http://passadospresentes.com.br/site/Site/index.php]\from[url12], and {\em Museum of Yesterday}, \useURL[url13][https://apublica.org/2017/07/museum-of-yesterday/]\from[url13].}

While the map section of {\em Digital Aponte} currently focuses on Havana itself, an exciting addition might be a map that includes Peñas Altas, the plantation where Aponte's coconspirators tried to launch their slave rebellion, as well as other plantations they attacked or marched through on the night of 14 March 1812. Including such a map would be a way to remind users both of the significance of the conspiracy for Aponte and his companions and of the proximity of sugar and plantation society to the vibrant capital city of Havana. A model for a map of the areas targeted by the rebels might be Vincent Brown's \useURL[url14][http://revolt.axismaps.com/][][{\em Slave Revolt in Jamaica, 1760--1761: A Cartographic Narrative}]\from[url14].\footnote{See {\em Slave Revolt in Jamaica, 1760--1761: A Cartographic Narrative}, \useURL[url15][http://revolt.axismaps.com/]\from[url15].} Finally, the fact that the Cuban government has recently authorized the construction of a monument to Aponte (by renown sculptor Alberto Lescay) near Peñas Altas makes it important to have that space reflected on the website's map. It also suggests a generative synergy between the physical monument and {\em Digital Aponte}, itself a kind of dynamic, interactive monument to the memory of Aponte and his political and creative work.

Just as the website incorporates Aponte's Havana, it must also, of course, highlight Aponte the man. Currently the biography section of the website is minimal (\quotation{\useURL[url16][http://aponte.hosting.nyu.edu/jose-antonio-aponte/][][José Antonio Aponte]\from[url16]}). One of the obvious areas in which further research on Aponte is needed is the genealogical. We know, for example, that his grandfather, Joaquín Aponte, was a captain in the Batallón de Morenos of Havana, Spain's colonial black militia in the city. Aponte's father (like Aponte himself) also served in that military force. But we know little about Aponte's mother, or why, for instance, Aponte's second last name (Ulabarra) is different from that of the woman (Mariana Poveda) usually identified as his mother.\footnote{See María del Carmen Barcia, {\em Los ilustres apellidos: Negros en la Habana colonial} (Havana: Editorial de Ciencias Sociales, 2009); and Carlos Venegas, \quotation{La conspiración de Aponte: Lugares de su memoria,} in Romay and Ferrer, {\em Los Mundos de Aponte} (forthcoming).} It may have been from that side of the family that Aponte received the title to his carpentry workshop, according to inconclusive documents in the Cuban National Archives. Information about Aponte's descendants is even murkier. For example, the names of his children given on the official \useURL[url17][https://www.ecured.cu/José_Antonio_Aponte][][EcuRed page]\from[url17] (Cuba's answer to Wikipedia) are different from those that appear in passing in the judicial testimony.\footnote{\quotation{José Antonio Aponte,} EcuRed, \useURL[url18][https://www.ecured.cu/José_Antonio_Aponte]\from[url18].} Fleshing out that genealogy lends itself to the kinds of collaboration facilitated by digital platforms. Here, one important resource is the \useURL[url19][https://www.slavesocieties.org/][][Slave Societies Digital Archive]\from[url19] directed by historian Jane Landers at Vanderbilt.\footnote{See Slave Societies Digital Archive, \useURL[url20][https://www.slavesocieties.org/]\from[url20].} The website houses digital copies of ecclesiastical records (such as baptisms, deaths, marriages) in Cuba, Brazil, and Colombia. Included among the Cuban records are those of churches that Aponte, his family, and his coconspirators may have attended, including Jesús, María y José (near his home) and Santo Cristo del Buen Viaje (near his first workshop). Indeed, the records of the Archbishopric Archive available on the website include a \useURL[url21][https://essss.library.vanderbilt.edu/islandora/object/essss\%3A3824][][petition]\from[url21] to create a new carpenters' brotherhood. The petition was signed by Aponte and at least one of his coconspirators.\footnote{\quotation{Legajo 4, Expte. 21. Diligencias para establecer una cofradía de San José el Gremio de carpinteros, 1800,} Slave Societies Digital Archive, \useURL[url22][https://essss.library.vanderbilt.edu/islandora/object/essss\%3A3824][][https://essss.library.vanderbilt.edu/islandora/object/essss\letterpercent{}3A3824]\from[url22]. See Jane Landers, \quotation{Catholic Conspirators? Religious Rebels in Nineteenth-Century Cuba,} {\em Slavery and Abolition} 36, no. 3 (2015): 495--520.}

How, we wonder, did Aponte's heirs experience and remember his attempt at revolution and his brutal execution at the hands of Spanish authorities? Did they participate in subsequent revolutionary movements, such as the 1839 conspiracy in Havana that implicated some other participants from the 1812 movement or the major conspiracies of 1843--44, known soon after as La Escalera?\footnote{On the 1839 conspiracy, see Pavez Ojeda, \quotation{Expediente contra José Antonio Aponte}; on La Escalera, see Aisha Finch, {\em Rethinking Slave Rebellion in Cuba: La Escalera and the Insurgencies of 1841--1844} (Chapel Hill: University of North Carolina Press, 2015).} Did they participate in the wars of independence against Spain in the late nineteenth century or the protests of the Independent Party of Color in 1912?\footnote{See Alina Helg, {\em Our Rightful Share: The Afro-Cuban Struggle for Equality, 1886--1912} (Chapel Hill: University of North Carolina Press, 1995).} In 1940, when members of the Federation of the Cuban Societies of Color wrote to the Constitutional Convention calling for a constitutional article against discrimination, one such letter writer signed his name José Antonio Aponte.\footnote{See Alejandra Bronfman, {\em Measures of Equality: Social Science, Citizenship, and Race in Cuba, 1902--1940} (Chapel Hill: University of North Carolina Press, 2004), 177.} A descendant perhaps? Tracing Aponte's memory and the connections between that memory and subsequent political movements is work that yet remains to be done.

\subsection[reference={collaboration-as-creation-a-new-book-of-paintings},
bookmark={Collaboration as Creation: A New Book of Paintings},
title={Collaboration as Creation: A New Book of Paintings}]

In the meantime, one unintended but exciting byproduct of the website is the creation in a different medium, register, and time of Aponte's book of paintings. We refer here to the contemporary art exhibit {\em Visionary Aponte: Art and Black Freedom}, which opened at the Little Haiti Cultural Center in December 2017 and traveled to New York University and Duke University in February and September 2018, respectively. Subsequent showings are currently being planned at other locations across the United States as well as in Haiti, Cuba, and Guadeloupe. In the absence of Aponte's book, fifteen artists have used the material on {\em Digital Aponte} to help them reimagine Aponte's book for our time.

{\em Visionary Aponte} represents an exciting encounter and collaboration between art and scholarship. It is also a collaboration between artists and scholars, mediated in part by {\em Digital Aponte}. Artists were charged with engaging Aponte's testimony about the book of paintings and using his words about his vision as a point of departure for their own artistic creation. As scholars, we wondered what it would be like for the artists themselves to work on something so deeply historical and to begin their project from the trial record's often awkward, stilted words about pictures that no longer exist.

Only a handful of artists selected one lámina to reimagine or recreate. The majority read the testimony and did not (or could not?) limit themselves to just one image; rather, they selected themes or elements that appeared throughout and worked them into complex pieces. Some found it impossible to limit themselves to creating a single painting or drawing. Édouard Duval Carrié completed four paintings that drew on multiple láminas; José Bedia created a large, monumental piece that contains elements from perhaps a dozen láminas (see fig. 2). Marielle Plaisir came up with more than fifty pieces, and Renée Stout with some nine (plus a gun).

\placefigure[here]{José Bedia, {\em Júbilo de Aponte}, 2017; mixed media on mixed papers, 106 x 143 in. (courtesy of the artist).}{\externalfigure[issue03/ferrer-2.jpg]}


In many cases, the production of the art was preceded by sustained engagement with {\em Digital Aponte} or with the site's creators. Here it might be instructive to briefly summarize two such conversations. One was with artist Teresita Fernández, who was particularly interested in Aponte's vision of (and recourse to) the cosmos. She spoke of an Aponte drawn to the cosmos as a space of comfort, as a space that was inherently democratic. Why? Because it was a space that could never be colonized, she said. A place, we might add, where slavery had no place. It was also a space that served as a kind of respite from Aponte's burdens as a leader of a weighty, treacherous struggle for black freedom. Fernández's interpretation is novel, and it derives, in part, from her reading {\em as an artist} the words of Aponte {\em the artist} (see fig. 3). We might even say it derives from a synergy between her own artistic and political practice and Aponte's.

\placefigure[here]{Teresita Fernández, {\em Aponte}, 2017; pyrite, oil, and graphite on wood panel, 21.5 x 36 x 2 in. overall (courtesy of the artist and Lehmann Maupin, New York; photograph by Yolanda Navas).}{\externalfigure[issue03/ferrer-3.jpg]}


Another such conversation occurred with Renée Stout, who was particularly interested in what Aponte might have tried to hide or disguise in his book---both figuratively, in the way he used and created images and meaning, and more literally, by hiding or perhaps even destroying his work so that authorities would not find it. And somehow in the conversation's back and forth, something that should have been self-evident became crystal clear.

From other testimony, we know that Aponte had destroyed other images and documents in his house before the police arrived---images of Haitian leaders, for example, and one related to the French Revolution. He knew they were incriminating, and he testified that he destroyed them sometime before his arrest. What might he have thought about his book in this context? Did he consider it incriminating? That he did hide the book---he put it in a box and then hid that box at the bottom of a trunk filled with clothes---suggests that he knew the book would likely be seen as subversive. Did he consider destroying it, as the investigators' dragnet drew nearer? No. He hid it, but he did not destroy it. And as Renée and Ferrer pondered this, something simple and obvious became vividly apparent---Aponte loved his book. He could not destroy it; he protected it.

Thinking about Aponte's love for his book is a fitting way to introduce folks to both {\em Digital Aponte} and {\em Visionary Aponte}. Both are, we think, labors of love. They are a living monument to Aponte---not just to Aponte as \quotation{a fighter for freedom} but rather, or also, to the Aponte who loved and cherished his own work as an artist, creator, historian, theorist. {\em Visionary Aponte} and {\em Digital Aponte} both help educate a broader public about the book and its creator, and they help expand an emerging transnational canon of black antislavery intellectual and cultural production so as to make it less anglophone, say, and more visual. And as the exhibit and website generate greater interest in Aponte, they help expand the pool of people who understand the significance of the book of paintings and the stakes of recovering it. A future collaboration might even entail a collective search for the missing book, documented on the website, though that is not yet planned. Ultimately, then, the website can serve as more than a space for the diffusion of Aponte's story or for the advancement of novel interpretations of his intellectual, artistic, and political labor. It might also serve as a stimulus and guide for the ultimate act of recovery: somewhere finding Aponte's long-lost and beloved book.

Finding the book, however, is ultimately an improbable outcome. There is no guarantee that the book has survived in one piece. The fact that the book may not have had identifying information, such as Aponte's name or a title (\quotation{Book of Paintings} is what authorities called it; we do not know if Aponte gave it a name), means that, even if it survived, it may not be identifiable by whoever happens to come across it. Even, for the moment, imagining that the book can be found, we would still be unable to answer many of our questions. Indeed, the artifact itself is likelier to raise as many questions as it answers. However much we would all love to locate the book, the ultimate act of recovery is perhaps less about Aponte's book itself than about what the book---and our own collective endeavor to imagine it and its creator---represent: a profound fusion of the artistic, intellectual, and political in which no one of those reigns over the others. {\em Digital Aponte} tries to foreground that quality in Aponte's work and to model it in our own, cognizant always that while definitive answers (like the book of paintings itself) remain elusive, the collaborative process of seeking them out is itself part of the project of envisioning freedom.

\subsection[reference={acknowledgments},
bookmark={Acknowledgments},
title={Acknowledgments}]

We wish to thank the Polonsky Foundation-NYU Digital Humanities Internship Program, NYU's FAS Office of Educational Technology, and Digital Scholarship Services at Bobst Library. In particular, we would like to acknowledge Jennifer Vinopal, Zachary Coble, Andrew Battista, and Armanda Lewis for their institutional support, and we thank Kris Minhae Choe and Eric Anderson for their extensive and valuable work on the website.

\thinrule

\page
\subsection{Linda M. Rodriguez}

Linda M. Rodriguez (1978--2018) earned her PhD in art history at Harvard University, where she wrote a dissertation on free black artists in colonial Havana, among them José Antonio Aponte. Rodríguez was Post-Doctoral Fellow in Art History and, later, Visiting Scholar at the Center for Latin American and Caribbean Studies at New York University. In addition to curating {\em Digital Aponte}, she collaborated in {\em Separados/Torn Apart}, a project that visualizes the geography and finances of the “zero-tolerance immigration policy developed by ICE, and in {\em Hablemos de La Habana} (Let's Talk about Havana), a collaborative forum about the city's future organized by Friends of Havana and the World Monuments Fund. Dr.~Rodriguez passed away on 1 October 2018, just as the authors were starting to revise this article.

\subsection{Ada Ferrer}

Ada Ferrer is Julius Silver Professor of History and Latin American and Caribbean Studies, New York University. She is author of the award-winning books {\em Insurgent Cuba: Race, Nation, and Revolution, 1868--1898} (University of North Carolina Press, 1999) and {\em Freedom's Mirror: Cuba and Haiti in the Age of Revolution} (Cambridge University Press, 2014). She is currently the Andrew W. Mellon Foundation Fellow at the Cullman Center for Scholars and Writers at the New York Public Library, as well as a Guggenheim Foundation Fellow.

\stopchapter
\stoptext