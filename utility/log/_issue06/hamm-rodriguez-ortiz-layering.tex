\setvariables[article][shortauthor={, }, date={May 2022}, issue={6}, DOI={https://doi.org/10.7916/archipelagos-7srx-rp29}]

\setupinteraction[title={Layering Caribbean Texts and Modalities: Relational Pedagogies for Secondary Language Arts Classrooms},author={Molly Hamm-Rodríguez, Lisa Ortiz}, date={May 2022}, subtitle={Layering Caribbean Texts and Modalities}, state=start, color=black, style=\tf]
\environment env_journal


\starttext


\startchapter[title={Layering Caribbean Texts and Modalities: Relational Pedagogies for Secondary Language Arts Classrooms}
, marking={Layering Caribbean Texts and Modalities}
, bookmark={Layering Caribbean Texts and Modalities: Relational Pedagogies for Secondary Language Arts Classrooms}]


\startlines
{\bf
Molly Hamm-Rodríguez
Lisa Ortiz
}
\stoplines


{\startnarrower\it Caribbean studies paradigms grounded in relationality and interconnectedness have a rich history and contemporary presence within literary and historical scholarship. However, these paradigms remain marginalized within K--12 contexts in the United States, resulting in few opportunities for students in formal educational spaces to draw from and reimagine the foundational role of Caribbean diasporic communities across a range of contexts and time periods. This essay illustrates how digital pedagogical approaches may both disrupt the exclusions of Caribbean multimodal texts within secondary language arts classrooms and reframe reading and writing through relational, Black diasporic, and translingual perspectives. Drawing from three young adult novels (Elizabeth Acevedo's {\em Clap When You Land}, Ibi Zoboi's {\em American Street}, and Lilliam Rivera's {\em Never Look Back}), we outline digital projects that emphasize the relational entanglements of Morningside Heights, Detroit, Tampa, and the Bronx with the Dominican Republic, Puerto Rico, and Haiti. Relying on archival materials from the Digital Library of the Caribbean (dLOC) and related collections, we encourage educators to layer texts, modalities, histories, languages, and sounds to engage with themes explored across the novels, including US imperialism, Caribbean migrations, African American history, (un)natural disasters, and diasporic placemaking practices. This essay serves as a pedagogical resource for educators seeking to deepen their classroom engagement with Caribbean texts, histories, and contemporary experiences and to use more expansive literacy practices in the process.

 \stopnarrower}

\blank[2*line]
\blackrule[width=\textwidth,height=.01pt]
\blank[2*line]

\startblockquote
He knew the Caribbean. And so he knew the world.
\stopblockquote 

\startalignment[flushright]
\tfx{Tiphanie Yanique, Land of Love and Drowning.}
\stopalignment
\blank[2*line]


What does it mean to \quotation{know the world} and to what extent is that possibility foreclosed in public schools in the United States? Despite claims that students must \quotation{read widely and deeply,}\footnote{See Common Core State Standards Initiative, \quotation{Note on Range and Content of Student Reading,} section of \quotation{English Language Arts Standards: \quote{Anchor Standards}---College and Career Readiness Anchor Standards for Reading,} accessed 15 January 2022, \useURL[url1][https://learning.ccsso.org/common-core-state-standards-initiative]\from[url1].} secondary language arts classrooms in the United States often reproduce larger geopolitical, linguistic, and racial projects of exclusion both through text selection and pedagogical practices. When social issues are addressed by the language arts curriculum, they are frequently covered from a US-centric perspective that centers the nation-state and ignores broader patterns of interconnectedness that are necessary to comprehend both local and global articulations of power and resistance. Many classrooms are structured around a mythical \quotation{literary canon} that obscures the particularity---rather than self-proclaimed universality---of Eurocentric narratives and representations of the world. In addition, compelling young adult novels that grapple with complex topics of adolescence from different perspectives remain outside of many classrooms, thus narrowing the view of what constitutes literature and what it means to be a reader and a writer. As a result, multiple ways of being, using language, and expressing the self are often excluded from the pages of selected texts themselves as well as from the ways that students are expected to engage with these texts in a classroom environment.

Popular movements such as \#DisruptTexts interrupt these practices.\footnote{See \#DisruptTexts, \quotation{What Is \#Disrupt Texts?,} accessed 15 January 2022, \useURL[url2][https://disrupttexts.org/lets-get-to-work/]\from[url2].} Such efforts do not advocate for a surface-level, representational approach to diversifying classroom libraries but rather seek to disrupt how the education system misrepresents, obscures, and silences the histories of the majority of students in US classrooms. While carefully selecting texts for the secondary language arts classroom is a starting point for transforming instruction, teachers must be prepared to use a nuanced pedagogical approach that engages students with texts through a critical lens that is historically, socioculturally, and linguistically dynamic. For educators, this process requires interrogating one's own positionality in relation to the texts under study, understanding students' relationships to the texts (in ways that might differ from your own), and recognizing the substantial research and identity work required of the teacher before incorporating certain texts in the classroom. Ideally, teachers would undergo collaborative pedagogical efforts as they acknowledge their limitations, recognizing that pedagogical practices are deeply embedded in one's own embodied experiences navigating the world as well as one's political commitments.\footnote{For critical, interdisciplinary, and educational scholarship that addresses positionality regarding research, teaching, and how one's own pedagogical trajectories are navigated in light of structural inequities, see, for example, the work of Kakali Bhattacharya, Leigh Patel, Eve Tuck, and Mirelsie Velazquez.} The labor within such practices is not trivial, and the outcomes are not immediate and finite but continuous, challenging, and crucial.

We suggest such critical reflections of praxis because they are ones that guide our own processes in this collaborative endeavor, which began with our joint participation in the NEH-funded Caribbean Studies and Digital Humanities Institute on Migration, Mobility, and Sustainability.\footnote{Migration, Mobility, Sustainability, \quotation{Caribbean Studies + Digital Humanities,} accessed 23 January 2022, https://nehcaribbean.domains.uflib.ufl.edu/.} Hamm-Rodríguez is trained as a six-to-twelfth-grade English language arts teacher and currently supports both pre-service and in-service teachers who work with bilingual students in K--12 schools. With a research trajectory that spans education, anthropology, and linguistics and intersects with Dominican, Haitian, and Puerto Rican studies, her political commitments to Caribbean studies extend from more than ten years of personal and professional experience in the Dominican Republic and from her transnational work in public schools supporting both students and educators. Hamm-Rodríguez is a white female scholar committed to pedagogical practices that dismantle and decenter whiteness in education spaces while recognizing that her own work---including this essay---can only deconstruct but not be detached from that racial positionality. Ortiz, as a Puerto Rican woman, acknowledges the shared experiences of Black, Caribbean, diasporic, and immigrant and migrant individuals in all their intersecting identities. She also understands the urgency to avoid thinking of such shared experiences as monolithic and all-encompassing. As an interdisciplinary scholar grounded in Puerto Rican, Latinx/a/o and education studies, she believes expanding disciplines and fields by acknowledging Black and Afro-Caribbean perspectives and histories---in their own right---is necessary for liberatory knowledge traditions. In her critical reflections of privileges and practices, she continuously strives toward genuine solidarity in action, knowing the ongoing self-work it takes to be and do better individually, collectively, and structurally. In this essay, our citational politics draw attention to ideas that have transformed our thinking, that privilege multiple ways of knowing, and that exemplify the type of relationality we envision and strive to enact pedagogically.

\subsection[title={Foregrounding Relationality},reference={foregrounding-relationality}]

Caribbean studies paradigms offer an entry point for pedagogical practices in secondary schools to foreground {\em relationality} in the learning process, rather than the \quotation{college and career readiness} frameworks of K--12 standards that rely on the construction of reading practices as linear and transparent. In theorizing relationality as praxis in secondary classrooms, we draw from Édouard Glissant's notion of relations as grounded in both encounter and \quotation{a new and original dimension allowing each person to be there and elsewhere, rooted and open.}\footnote{É. Glissant, {\em Poetics of Relation, trans}. Betsy Wing (Ann Arbor: University of Michigan Press, 1997), 34.} As Glissant argues, \quotation{Relation functions both in the internal relationship (that of each culture to its components) and, at the same time, in an external relationship (that of this culture to others that affect it).}\footnote{Glissant, {\em Poetics of Relation, 169.}} This generative perspective imagines culture or identity formations not as bounded entities but rather as produced and reimagined through an endless process of relation that is necessarily rhizomatic rather than hierarchical---a critical departure from how reading, writing, and engaging with a range of texts and with each other is typically imagined in K--12 schools.

Inspiring our own approach to text selection for this project, Yomaira Figueroa-Vásquez expands Glissant's notion of poetics of relation to propose a \quotation{radical relational remapping} as \quotation{critical cartographic practice.}\footnote{Y. C. Figueroa-Vásquez, {\em Decolonizing Diasporas: Radical Mappings of Afro-Atlantic Literature} (Evanston, IL: Northwestern University Press, 2020).} Grounded in women of color feminism, Figueroa-Vásquez's approach puts Afro-hispanophone diasporic texts from the Caribbean and Equatorial Guinea in relation by acknowledging the incommensurability of different forms of racialization and how they shape subjective experiences while also pointing toward a range of liberation practices forged in diaspora and exile. In addition, we draw inspiration from the relational frameworks advanced by Myriam J. A. Chancy and Vanessa K. Valdés as they explore the histories and writings of the hispanophone Caribbean alongside the histories and writings of Haiti,\footnote{M. J. A. Chancy, {\em From Sugar to Revolution: Women's Visions of Haiti, Cuba, and the Dominican Republic} (Waterloo, ON: Wilfrid Laurier University Press, 2013); V. K. Valdés, {\em Racialized Visions: Haiti and the Hispanic Caribbean (}Albany: SUNY Press, 2020).} which we attempt to do here for young adult (YA) literature by focusing on the Dominican Republic, Puerto Rico, Haiti, and their diasporas. The foundational work of Chancy and Valdés fills historical silences and contests dominant paradigms---whether related to race or language---that separate rather than put in relation the intellectual and political thought and embodied experiences of places and peoples often imagined as disparate rather than interconnected.

\placefigure[here]{slide preview}{\externalfigure[issue06/valence_1_mhr_lo_figure-1.jpg]}


Thus, in this proposal for teaching Caribbean diasporic YA novels through digital projects, we not only advocate for spaces where students can use dynamic reading and writing practices but also caution against pedagogical approaches that produce singular narratives about the Caribbean by erasing the complexities and contradictions of Caribbean diasporic histories.\footnote{T. L. Goffe, \quotation{Unmapping the Caribbean: Toward a Digital Praxis of Archipelagic Sounding,} {\em archipelagos} 5 (December 2020), \useURL[url3][https://doi.org/10.7916/archipelagos-72th-0z19][][{[}https://doi.org/10.7916/archipelagos-72th-0z19{]}]\from[url3]. In particular, see Goffe's arguments about the right to opacity and the refusal to reveal fugitive perspectives and practices that are unmappable and unknowable to the outsider and the colonial enterprise (including research and pedagogy).} In this essay, we draw from three YA novels: Lilliam Rivera's {\em Never Look Back}, Ibi Zoboi's {\em American Street}, and Elizabeth Acevedo's {\em Clap When You Land}.\footnote{L. Rivera, {\em Never Look Back} (New York: Bloomsbury YA, 2020); I. Zoboi, {\em American Street} (New York: Balzer & Bray, 2017); E. Acevedo, {\em Clap When You Land} (New York: Quill Tree, 2020).} Weaving together the geographic spaces of Morningside Heights, Detroit, Tampa, the Bronx, the Dominican Republic, Puerto Rico, and Haiti, all three novels feature Afro-Caribbean protagonists who navigate adolescent life and the intersections of race, ethnicity, gender, social class, and sexuality across a backdrop of migration produced by coloniality and imperialism.

In {\em Never Look Back}, the Greek myth of Orpheus and Eurydice is reimagined through a love story between Eury, who leaves Puerto Rico after the 2017 landfall of Hurricane María, and Pheus, a Bronx-born Dominican who aspires to be a bachatero. As Eury and Pheus connect through their love of music, they embark on a journey against depression, gentrification, and colonial and imperial histories that defy geographic boundaries. In {\em American Street}, Fabiola moves from Port-au-Prince to Detroit to be with extended family. When her mother is detained by US Immigration and Customs Enforcement after the first leg of the flight, Fabiola trusts in Haitian {\em lwa} to overcome the separation while traversing unfamiliar family, friend, and romantic relationships. In {\em Clap When You Land}, a novel in verse, sisters Yahaira (New York City) and Camino (Sosúa, Dominican Republic) learn of each other's existence after their father dies in a plane crash mirroring the loss of all aboard American Airlines flight 587 (New York to Santo Domingo) in 2001. By illustrating how each protagonist navigates love, loss, kinship, and dreams of the future, the authors develop complex trajectories of Caribbean diasporic experiences that should be read and understood in relation to one another. This approach exemplifies a long line of Caribbean literary thought---from Antonio Benítez-Rojo's repeating island to Edward Kamau Brathwaite's submarine unity to Vèvè Clark's {\em marasa} consciousness---that centers the interconnectedness of Caribbean texts, histories, and experiences without erasing difference.\footnote{A. Benítez-Rojo, {\em The Repeating Island: The Caribbean and the Postmodern Perspective}, trans. James E. Maraniss, 2nd ed.~(Durham, NC: Duke University Press, 1996); E. K. Brathwaite, {\em Caribbean Man in Space and Time: A Bibliographical and Conceptual Approach}, pamphlet (Kingston: Savacou, 1975); V. A. Clark, \quotation{Developing Diaspora Literacy and {\em Marasa} Consciousness,} {\em Theatre Survey} 50, no. 1 (May 2009): 9--18.}

To teach these books in secondary language arts classrooms in US schools, we propose digital projects that use archival materials from the Digital Library of the Caribbean (dLOC) and related collections\footnote{We thank the librarians across institutions who met with us during early stages of this work and who pointed us to a plethora of useful resources: Gayle Williams (Florida International University), Kathia Ibacache (University of Colorado Boulder), Lisa Gardinier (University of Iowa), and Antonio Sotomayor (University of Illinois at Urbana-Champaign).} to explore how the multidimensional layering of texts, modalities, histories, languages, sounds, and visualities can reveal and represent the narrative depth and diasporic trajectories of the novels' protagonists beyond the written word. As a dispersed and collaborative archive, dLOC presents an opportunity for educators to seek a range of multilingual resources that have been curated and shared from a variety of perspectives. By engaging with the dLOC archives as dynamic rather than static, educators can support students in understanding texts and histories as relational and intertwined, marked by both presences (what can be found in the archives) and absences (what is left unsaid or excluded). We draw inspiration from education research within the framework of Black girls' and Black immigrant literacies as well as diaspora literacy to center the Afro-Caribbean experiences of the novels' protagonists and to highlight how the authors narrate the intersections of Caribbean migrations and African American history. We also discuss how translingual and transliteracies approaches to language arts can frame these digital projects as a way for students to consider and express how ideas, people, objects, meanings, contexts, and memories circulate in relation to experiences of (im)mobility from historical and sociocultural perspectives.

\subsection[title={Black Diaspora Literacies in K--12 Classrooms},reference={black-diaspora-literacies-in-k12-classrooms}]

\startblockquote
I am at a crossroads again.
\stopblockquote 

\startalignment[flushright]
\tfx{Ibi Zoboi, American Street.}
\stopalignment
\blank[2*line]


\placefigure[here]{American Street}{\externalfigure[issue06/valence_1_mhr_lo_figure-2.jpg]}


The crossroads play an important role in the storyline of Ibi Zoboi's novel. At the intersection of American Street and Joy Road sits Fabiola's new home in Detroit after moving from Haiti. Zoboi outlines the corner's long history of Black residents: from those who migrated north from Alabama, Georgia, and South Carolina to seek employment in the auto industry to a Haitian immigrant killed outside the Chrysler plant where he worked, leaving behind his family: Fabiola's aunt and cousins. Here, Fabiola finds {\em Papa Legba}, an intermediary between humanity and the spirit world in Haitian Vodou, and seeks his guidance about what she should do next as she faces challenging decisions and uncertainties. The crossroads has resonance for Caribbean digital projects in a number of ways. Marlene Daut uses the crossroads metaphor to illustrate how multimodal approaches \quotation{can allow for alternative ways of (humanely) archiving black sovereignty.}\footnote{M. L. Daut, \quotation{Haiti @ the Digital Crossroads: Archiving Black Sovereignty,} {\em archipelagos} 3 (July 2019), \useURL[url4][http://doi.org/10.7916/archipelagos-53xt-3v66][][{[}https://doi.org/10.7916/archipelagos-53xt-3v66{]}]\from[url4].} Jessica Marie Johnson proposes {\em xroads praxis} as a Black diasporic technology to center Black humanity \quotation{on the other side of the trick/tragedy} in ways that are both durable and transient.\footnote{J. M. Johnson, \quotation{Xroads Praxis: Black Diasporic Technologies for Remaking the New World,} {\em archipelagos} 3 (July 2019), \useURL[url5][https://doi.org/10.7916/archipelagos-4fjd-k774][][{[}https://doi.org/10.7916/archipelagos-4fjd-k774{]}]\from[url5].} In this essay, we consider the crossroads as a space for Black diasporic and Caribbean youth in US classrooms to read and write their histories (making sense of the past, present, and future) using a range of multimodal texts.

Ebony Elizabeth Thomas reminds us that YA fiction in the United States is often written from a perspective that assumes adolescence as a universal stage of human development. By focusing on Black YA fiction from a range of global contexts, Thomas highlights the specificities of Black youth and young adulthood as expressed through Black storytelling and storying traditions.\footnote{E. E. Thomas, \quotation{Young Adult Literature for Black Lives: Critical and Storytelling Traditions from the African Diaspora,} {\em International Journal of Young Adult Literature} 1, no. 1 (2020), \useURL[url6][https://doi.org/10.7916/archipelagos-72th-0z19][][{[}https://doi.org/10.24877/ijyal.27{]}]\from[url6].} This perspective is critical for understanding the types of reading and writing practices that can be paired with the selected texts, especially when taught by and for Caribbean diasporic communities. For this reason, we draw on different paradigms in literacy studies that have shaped conversations about language and literacy practices for Black youth in the United States. In rejecting singular narratives about adolescence as represented in texts and among discussions about students in US classrooms, Gholdy Muhammad and Marcelle Haddix developed a Black girls' literacies framework to highlight the multiplicity and complexity of Black girlhood and womanhood. This framework seeks to understand how Black girls engage with literacy practices in ways that are multiple, tied to their identities, historical, collaborative, intellectual, and political or critical. Muhammad and Haddix emphasize the important role of text selection in making space for these literacies. Their vision for texts is multimodal---imagery, sound, video, and performance are layered to engage with students' own processes of self-definition, cultural discourse styles, communal spaces for growth and healing, and social transformation and critique.\footnote{G. Muhammad and M. Haddix, \quotation{Centering Black Girls' Literacies: A Review of Literature on the Multiple Ways of Knowing of Black Girls,} {\em English Education} 48, no. 4 (2016), \useURL[url7][https://library.ncte.org/journals/ee/issues/v48-4/28670][][{[}https://library.ncte.org/journals/ee/issues/v48-4/28670{]}]\from[url7].} As one example of this layering, Esther Ohito outlines fugitive literacy practices that destroy (rather than simply disrupt) both whiteness and anti-Blackness. In describing a multimodal essay composition project, Ohito details how textual, aural, linguistic, spatial, and visual modes of communication were used by Black students to express Blackness as \quotation{relational,} \quotation{as diasporic---that is, unrestrained by nation-state boundaries,} and as variable.\footnote{E. O. Ohito and the Fugitive Literacies Collective, \quotation{\quote{The Creative Aspect Woke Me Up}: Awakening to Multimodal Essay Composition as a Fugitive Literacy Practice,} {\em English Education} 52, no. 3 (2020), \useURL[url8][https://library.ncte.org/journals/EE/issues/v52-3/30596][][{[}https://library.ncte.org/journals/EE/issues/v52-3/30596{]}]\from[url8].}

Several authors have used a variety of frameworks---diaspora literacy, transnational literacy, racial literacy---to center Blackness more explicitly in conversations about the literacy and language practices of youth in US schools. Much of this work focuses on Black immigrant and transnational experiences within the context of the United States, a central theme to the three novels selected for this project. Chris Busey, for example, has described how the histories of Afro-Latinx communities have been largely repressed in US K--12 classrooms and points to diaspora literacy approaches that can support the development of critical sociohistorical knowledge in ways that traverse disciplines in schools (e.g., language arts and social studies).\footnote{C. L. Busey, \quotation{Diaspora Literacy and Afro-Latin Humanity: A Critical Studyin' Case Study of a World History Teacher's Critical Sociohistorical Knowledge Development,} {\em Race Ethnicity and Education} 23, no. 6 (2018), \useURL[url9][https://doi.org/10.7916/archipelagos-72th-0z19][][{[}https://doi.org/10.1080/13613324.2018.1511531{]}]\from[url9].} Patriann Smith has developed a Black immigrant literacies framework to emphasize the divergent and convergent---but always interconnected---experiences of African American and Black immigrant youth in the United States.\footnote{P. Smith, \quotation{Silencing Invisibility: Towards a Framework for \quote{Black Immigrant Literacies,}} {\em Teachers College Record} 122, no. 13 (2020).} Focusing on English-speaking Black immigrant youth, Smith points to the need for relational language and literacy practices that can illuminate distinctive historical experiences, combat “divide-and-conquer racial politics'' across contexts and experiences, and generate dialogue about the relationships between racialized encounters in the United States and other geographic spaces. Similarly, Allison Skerrett and Lakeya Omorogun found that the language and literacy practices of Black immigrant and transnational youth contributed to the deconstruction of Blackness as a purportedly monolithic racial category, emphasizing the multiple ethnoracial identities and cultural practices that populate Blackness and Black experiences. Notably, they discuss how Black immigrant youth in US schools were not exclusively oriented to one nation-state but had attachments to multiple communities, contexts, and experiences. They found that Black immigrant youth were especially motivated by their own goals and desires to use language and literacy to connect with and explore these diverse connections.\footnote{A. Skerrett and L. Omogun, \quotation{When Racial, Transnational, and Immigrant Identities, Literacies, and Languages Meet: Black Youth of Caribbean Origin Speak,} {\em Teachers College Record} 122, no. 13 (2020).}

To be sure, these perspectives are not new in Caribbean studies and Black studies, having long been addressed by the ongoing intergenerational work of Black scholars who have fundamentally transformed the field of education, teaching, and learning in the United States.\footnote{See the foundational work of Ruth Nicole Brown, Geneva Smitherman, Carmen Kynard, Gloria Ladson-Billings, Detra Price-Dennis, Yolanda Sealey-Ruiz, and Venus Evans-Winters, among other scholars whose intellectual contributions have transformed conversations about Black girlhood, languages, and literacies in the United States, in education, and other spaces. For a recent account of the historical contributions of Black women teachers, see A. James-Gallaway and T. Harris, \quotation{We Been Relevant: Culturally Relevant Pedagogy and Black Women Teachers in Segregated Schools,} {\em Educational Studies} 57, no. 2 (2021), \useURL[url10][https://doi.org/10.1080/00131946.2021.1878179][][{[}https://doi.org/10.1080/00131946.2021.1878179{]}]\from[url10].} Yet they continue to be marginalized in K--12 classrooms. In particular, the language and literacy experiences of multilingual Black students have been underrepresented---though not entirely absent---in larger discussions about transformative pedagogies in secondary language arts classrooms. This project takes as a starting point the need for Caribbean YA fiction to be layered with multimodal texts, such as dLOC archives, in order to engage students in expansive literacy practices from a Black diasporic perspective that is related to their own histories and storying traditions.

\subsection[title={Transliteracies and Translingualism in K--12 Classrooms},reference={transliteracies-and-translingualism-in-k12-classrooms}]

\startblockquote
You can find the island stamped all over me, but what would the island find if I was there?
\stopblockquote 

\startalignment[flushright]
\tfx{Elizabeth Acevedo, Clap When You Land.}
\stopalignment
\blank[2*line]


\placefigure[here]{Clap When You Land}{\externalfigure[issue06/valence_1_mhr_lo_figure-3.jpg]}


Elizabeth Acevedo's novel is marked by reflections on mobility and immobility. Yahaira, born in the United States to Dominican parents, notes the cultural and linguistic practices that locate her as Dominican. Yet she also acknowledges her complicated relationship with the Dominican Republic itself and wonders what it would mean to \quotation{claim a home that does not know {\em you,} much less claim you as its own?} After her father's death, Yahaira discovers that his annual Dominican Republic trips were motivated by a desire to see his other daughter, Camino, and not by the business matters he long claimed as the reason for travel. Determined to connect with Camino, Yahaira defies her mother's wishes that she not return to the country and secretly takes a flight to the north coast. Both sisters come to terms with the (im)mobilities that have marked their lives in distinct ways. Yahaira is faced with the reality that her mother knew about Camino but would not agree to sponsor her immigration paperwork. Camino realizes that losing her father's remittances may prevent her from completing education at the local international school and fulfilling her dreams of attending Columbia University. The relative affordances and constraints of holding certain passports becomes a point of contention and then reunification for the sisters. The plot of this novel is deeply shaped by notions of what it means to come and go, and what it means to be together and to be apart. As the novel alternates between the perspectives of each sister across chapters, we see how space, (im)mobility, and relationships are shaped by different contexts while also being brought into dialogue.

Literacy researchers have extensively theorized the role of mobility in meaning-making processes among children and youth. While much of this work centers transnational experiences---often inadvertently romanticizing what movement can entail---Amy Stornaiuolo, Anna Smith, and Nathan Phillips propose a transliteracies framework that theorizes mobility alongside immobility. In doing so, the authors consider \quotation{how people make meaning across interactions among people, things, texts, contexts, modes, and media} while \quotation{foreground{[}ing{]} how people and things are mobilized and paralyzed, facilitated and restricted, in different measure and in relation to institutions and systems with long histories.}\footnote{A. Stornaiuolo, A. Smith, and N. C. Phillips, \quotation{Developing a Transliteracies Framework for a Connected World,} {\em Journal of Literacies Research} 49, no. 1 (2017): 68--91.} Though not explicit in this theoretical perspective, it intersects with African diasporic frameworks that emphasize (im)mobility and interconnectedness as not new but rather as grounded in the violent emergence of colonialism and capitalism in the Caribbean.\footnote{See J. S. Allen and R. C. Jobson, \quotation{The Decolonizing Generation: (Race and) Theory in Anthropology since the Eighties,} {\em Current Anthropology} 57, no. 2 (2016): 129--48, \useURL[url11][https://anthropology.fas.harvard.edu/files/anthrodept/files/jobson_decolonizing_generation.pdf][][{[}https://anthropology.fas.harvard.edu/files/anthrodept/files/jobson_decolonizing_generation.pdf{]}]\from[url11].} The transliteracies framework is grounded in social practice theories of literacy, which consider literacy as something not possessed by individuals but rather enacted in human activity and embedded in social relations and material realities. In developing a pedagogical perspective around the transliteracies framework, Smith, Stornaiuolo, and Phillips (2018) urge teachers to engage with and interrogate (im)mobility in their instruction so that students can explore the multiple and dynamic relationships among people, ideas, practices, texts, and media.\footnote{A. Smith, A. Stornaiuolo, and N. C. Phillips, \quotation{Multiplicities in Motion: A Turn to Transliteracies,} {\em Theory into Practice} 57, no. 1 (2018): 20--28.}

For Caribbean writers, translingualism is a natural feature of texts, communication, and storying practices. In describing translingual writing and composition, Suresh Canagarajah notes that translingualism \quotation{moves us beyond a consideration of individual or monolithic languages to life between and across languages.}\footnote{S. Canagarajah, {\em Translingual Practice: Global Englishes and Cosmopolitan Relations} (London: Routledge, 2013).} Translingual practices are embodied in the three novels selected for this project, illustrating how the authors use language creatively to produce new meanings that resonate across social and material contexts. These novels do not envision English, Spanish, and/or Kreyòl as separate and distinct but rather as fully integrated throughout the authors' storying processes and in how the protagonists make meaning about their worlds. Because mobility encourages the use of different language practices, a translingual perspective centers how language and literacy are not bound by place but instead emerge through social encounters and relationality. We anticipate that these digital projects will provide creative spaces for teachers' and students' translingual practices as they use the multilingual resources of dLOC to collaboratively express through digital projects how they engage with Caribbean diasporic texts and histories.

\subsection[title={Using dLOC and Other Archives to Layer Caribbean Texts in a Digital Project},reference={using-dloc-and-other-archives-to-layer-caribbean-texts-in-a-digital-project}]

\startblockquote
You are either a step away from your home or a step toward it.
\stopblockquote 

\startalignment[flushright]
\tfx{Lilliam Rivera, Never Look Back.}
\stopalignment
\blank[2*line]


\placefigure[here]{Never Look Back}{\externalfigure[issue06/valence_1_mhr_lo_figure-4.jpg]}


In Lilliam Rivera's {\em Never Look Back}, Eury moves from Puerto Rico to Tampa to the Bronx in the wake of Hurricane María. Eury's encounters with the New York City landscape and her love interest, Pheus, are marked by fears that Ato---a spirit from the Underworld who has followed her since childhood---will reappear at any turn. Metaphorically, Ato represents the weight of depression that continues to deepen for Eury. When Ato takes Eury to the Underworld---luring her to stay with idyllic scenes from the Puerto Rican {\em campo}---Pheus gathers strength from his ancestors and syncretic spirits to attempt to bring her back. Throughout the novel, readers see how the lingering effects of (un)natural disasters are carried in the moving body and how different ancestral and cultural sources of strength are drawn upon to face new challenges as they arise.\footnote{See Y. Bonilla and M. LeBrón, {\em Aftershocks of Disaster: Puerto Rico before and after the Storm} (Chicago: Haymarket, 2019).} Ultimately, this is the heart of the digital project. We propose a two-part digital project using dLOC and other archival resources for students to read these three novels relationally in order to understand complex migrations at the crossroads of Caribbean Black diasporic (im)mobility as well as the placemaking practices of young Afro-Caribbean protagonists as they imbue daily life and visions of the future with multiple meanings. In doing so, we offer educators a model that highlights pedagogical suggestions for prereading, reading, and postreading activities while encouraging teachers to modify activities and instructional foci based on the context of their classrooms, students, schools, communities, and/or regions.\footnote{You can \useURL[url12][\%7B\%7Bsite.baseurl\%7D\%7D/assets/issue06/hamm-ortiz-layering-slides.pdf][][download the slide decks for your own use]\from[url12]. If shared with other educators, please provide attribution, and remember to link to the full essay so that the slide deck does not become decontextualized.}

\placefigure[here]{digital project outline}{\externalfigure[issue06/valence_1_mhr_lo_figure-5.jpg]}


\subsection[title={Digital Project Part I: US Intervention, Caribbean Migrations, and Black Diasporic (Im)mobilities},reference={digital-project-part-i-us-intervention-caribbean-migrations-and-black-diasporic-immobilities}]

The first part of the digital project (which we envision as prereading activities) encourages students to explore the intersections of US imperialism, Caribbean migrations, and African American history using archival resources (from dLOC and other sources), which students can return to once they begin reading the novels. These lessons facilitate a Black diasporic lens on history that is not bounded by the nation-state but instead studies long patterns of forced movement and migration wrought by colonialism and imperialism. Rather than keeping Caribbean and African American history separate, as happens frequently in US schools, students will challenge prior understandings of history to map new visions of the transnational communities and diasporic experiences represented in the novels while not erasing particularities.

As background, we recommend that teachers consider introducing students to some of the following topics using dLOC and other archival materials. While not an exhaustive list of possible topics, these examples are derived from the content of the novels and can facilitate students' application of their learning through digital projects that layer multiple modalities to represent emerging understandings and interpretations of these relational histories.

To begin, throughout the duration of part 1, students may work in small groups and report back to the larger group about the long histories of US intervention in the Caribbean with which they become familiar. Examples of topics to study include US military occupation in Haiti, the Dominican Republic, and Puerto Rico as well as the US government's ongoing colonial relationship with Puerto Rico and neocolonial relationships with Haiti and the Dominican Republic.

\subsubsection[title={US Intervention in the Caribbean},reference={us-intervention-in-the-caribbean}]

\startitemize[packed]
\item
  \quotation{\useURL[url13][http://ufdc.ufl.edu/AA00001149/00001][][The American Intervention in Haiti and the Dominican Republic]\from[url13]}
\item
  \quotation{\useURL[url14][https://www.dloc.com/AA00001156/00001][][Haiti under American Control, 1915--1930]\from[url14]}
\item
  \quotation{\useURL[url15][https://dloc.com/AA00067799/00001/1x?search=us+\%3dpuerto+\%3drico][][Goff's Historical Map of the Spanish-American War in the West Indies, 1898]\from[url15]}
\item
  \quotation{\useURL[url16][https://dloc.com/AA00076641/00001?search=us+=puerto+=rico][][Laws of Cuba, Puerto Rico, and the Philippines]\from[url16]}
\item
  \quotation{\useURL[url17][https://dloc.com/AA00078642/00001?search=us+=puerto+=rico][][140 Protest U.S. Military in Vieques]\from[url17]}
\item
  \quotation{\useURL[url18][https://loc.gov/rr/hispanic/1898/index.html][][The World of 1898: The Spanish-American War]\from[url18]}
\item
  \quotation{\useURL[url19][https://salalm.org/digital-primary-resources][][Latin American & Caribbean Digital Primary Resources]\from[url19]}
\item
  \quotation{\useURL[url20][https://www.familysearch.org/wiki/en/Puerto_Rico_Archives_and_Libraries\#Archives][][Puerto Rico Archives and Libraries]\from[url20]}
\item
  \quotation{\useURL[url21][https://ufdc.ufl.edu/UF00098500/00001][][The American Colonial Handbook]\from[url21]}
\stopitemize

Students can then begin to explore a broad range of experiences of (im)mobility that traverse Black diasporic experiences. By studying the intimacies of Caribbean and African American movement, students will understand relational histories as intertwined and mutually constitutive of key moments in time across geographic space. Topics to explore include sea migration (including critical perspectives on detainment), the Great Migration, general and specific patterns of Caribbean migration to the United States, African American migration to the Caribbean, and the connections between liberation movements in the United States and those across the Caribbean.

\subsubsection[title={Black Diaspora (Im)mobilities},reference={black-diaspora-immobilities}]

\startitemize[packed]
\item
  \quotation{\useURL[url22][http://library.duke.edu/digitalcollections/caribbeansea/][][Caribbean Sea Migration]\from[url22]}
\item
  \quotation{\useURL[url23][https://www.zinnedproject.org/news/1776-counter-revolution-us-origins/][][The Counter-revolution of 1776: Origins of the United States of America]\from[url23]}
\item
  \quotation{\useURL[url24][https://nmaahc.si.edu/making-african-america][][Making African America]\from[url24]}
\item
  \quotation{\useURL[url25][http://www.inmotionaame.org][][The African-American Migration Experience]\from[url25]}
\item
  \quotation{\useURL[url26][https://web.archive.org/web/20211205224254/http://www.inmotionaame.org/migrations/landing.cfm@migration=10.html][][Caribbean Immigration]\from[url26]}
\item
  \quotation{\useURL[url27][https://www.dloc.com/AA00054410/00001][][Black Perspectives]\from[url27]}
\item
  \quotation{\useURL[url28][https://exhibits.uflib.ufl.edu/HaitianAmericanDream/?fbclid=IwAR1VqspacYxyPWm8fwuTxRDc2Fq2Tl7yNc53pv8YaR3CEUQq--YEDSglDF0][][The Haitian American Dream Timeline]\from[url28]}
\item
  \quotation{\useURL[url29][https://www.dloc.com/AA00010444/00001?search=african+=american][][Correspondence Relative to the Emigration to Hayti of the Free People of Colour in the United States]\from[url29]}
\item
  \quotation{\useURL[url30][http://islandluminous.fiu.edu/learn.html][][Haiti: An Island Luminous]\from[url30]}
\stopitemize

Finally, students can synthesize their understandings of imperialism, migration, and freedom through a case study of the Samaná Bay in the Dominican Republic. By assigning individual reading to all students, this case study provides them with an opportunity to view the Samaná Bay territory as a contested site of imperial or colonial power as well as an emancipatory site of migration for African Americans from the United States.\footnote{For further reading, see I. K. Nwankwo, {\em Black Cosmopolitanism: Racial Consciousness and Transnational Identity in the Nineteenth-Century Americas} (Philadelphia: University of Pennsylvania Press, 2005); B. R. Byrd, {\em The Black Republic: African Americans and the Fate of Haiti} (Philadelphia: University of Pennsylvania Press, 2019); and L. M. Alexander, "Black Utopia: Haiti and Black Transnational Consciousness in the Early Nineteenth Century," {\em William and Mary Quarterly} 78, no. 2 (2021). We thank the anonymous reviewer for these suggestions.}

\subsubsection[title={African American Migration to the Caribbean: Case Study of Samaná Bay},reference={african-american-migration-to-the-caribbean-case-study-of-samaná-bay}]

\startitemize[packed]
\item
  \quotation{\useURL[url31][https://www.dloc.com/UF00053407/00001?search=samana][][West Indies, Hispaniola---North Coast, Dominican Republic, Bahia de Samaná]\from[url31]}
\item
  \quotation{\useURL[url32][https://www.penn.museum/documents/publications/expedition/PDFs/47-1/Weeks.pdf][][The Samaná Americans]\from[url32]}
\item
  \quotation{\useURL[url33][https://www.nytimes.com/2018/11/30/travel/preserving-black-american-history-through-song-in-the-dominican-republic.html][][Preserving Black American History through Song in the Dominican Republic]\from[url33]}
\item
  \quotation{\useURL[url34][https://nkaa.uky.edu/nkaa/items/show/1898][][Freemen Community on Samaná Bay (Dominican Republic)]\from[url34]}
\item
  \quotation{\useURL[url35][https://haitidoi.com/2013/10/09/the-samana-affair-2/][][The Samaná Affair]\from[url35]}
\item
  \quotation{\useURL[url36][https://www.dloc.com/AA00001310/00001/2j][][Samana et ses projets de cession, 1844--91]\from[url36]}
\stopitemize

In the next stage of the project, students will be regrouped as they begin reading the novels. Each group will read one novel and then, both throughout and after reading, we encourage regular opportunities for one student from each group to convene in a new group that has all novels represented. Having students work in newly formed small groups ensures that each collective has background knowledge from prior work completed in part 1 as well as distinctive viewpoints from deep reading of the novels to bring together for this stage. Thus students not only will focus on engaging with one novel but will put content from the three novels into conversation with the primary and secondary resources studied using dLOC and other sources. After the novels have been completed, students in these mixed groups will be encouraged to use a slideshow format (PowerPoint, Google Slides, Canva Presentation, StoryMaps) to develop a multimodal presentation of the relational histories represented by their background research and the diasporic trajectories of the protagonists in the novels they read. Students might respond to the following prompts as they develop their group projects: How do the novels reveal relationships between Caribbean migrations and US imperialism? How do the novels represent the interconnectedness of Caribbean and African American histories and communities in the United States? How do the authors draw on historical research to illustrate the everyday life of the novels' Afro-Caribbean protagonists? How do relational histories help you make sense of the diasporic trajectories described in each novel?

Here we give brief examples of content from the novels that students might draw on and visually, textually, aurally, and spatially represent in their digital projects. Ibi Zoboi's {\em American Street} references experiences of Haitian migration to the United States, the detention of Haitian migrants, boats on the shores of Cité Soleil departing for Miami, dictatorships upheld by US intervention, MINUSTAH troops in Haiti, and Haitian independence, while also addressing police brutality in Detroit, the legacy of Motown, Black labor and the automobile industry, white flight in the city, and African American migration from the US South to northern states. Elizabeth Acevedo's {\em Clap When You Land} highlights the importance of migrant remittances, family separation and return visits to the Caribbean, the presence of \quotation{Dominican, Puerto Ricans, Haitians, Black Americans, and{]} Riverside white folk} in Morningside Heights, and her queer love interest from a Black southern military family, as well as the complexity of Haitian-Dominican relationships and their effect on access to health care, sex tourism, and unequal power relations of imperialism that affect the quality of life in the Dominican Republic. Lilliam Rivera's {\em Never Look Back} is marked by references to the Puerto Rican independence movement, the Schomburg Center for Research in Black Culture, the Young Lords, gentrification in the Bronx and San Juan, African American and Puerto Rican military service, Seneca Village, medical experiments on Puerto Rican women and unequal access to health care in Black communities, neoliberal and capitalist practices that benefit outside investors, and the US government's failure to support Puerto Rico after Hurricane María.

\subsection[title={Digital Project Part II: Placemaking as Diasporic Practice},reference={digital-project-part-ii-placemaking-as-diasporic-practice}]

Part 2 of the digital project encourages students to explore the placemaking practices of young Afro-Caribbean protagonists as they navigate the lingering effects of (un)natural disasters across time and space, drawing on different ancestral and cultural sources of strength to build new visions for the future and enact new meanings across daily life. This section of the project requires students to embark on more independent archival research (using dLOC and other sources) guided by their selection of key citations from the three YA novels.

To begin this part of the project, students will research three social disasters that have marked the collective memories of Caribbean and Caribbean diasporic communities and that influence the ongoing placemaking practices of the novels' female protagonists: American Airlines Flight 587 crash in Queens (2001), the 7.0 magnitude earthquake near Port-au-Prince (2010), and the Category 5 Hurricane María that made landfall in Puerto Rico (2017). This activity can take place during or after reading, but it should occur in the mixed groups so that each group discusses all three events, with each individual student researching the event associated with the novel that they read. This research will deepen students' prior understandings of migration patterns, particularly as these disasters\footnote{Here we wish to emphasize the importance of avoiding disaster as the primary analytical framework through which to understand the Caribbean. For this reason, we have paired an understanding of disaster with an exploration of placemaking practices. Though disasters do play a role in each of the novels, as they mark livelihoods and geographies in particular ways, Acevedo, Rivera, and Zoboi each imagined otherwise and expansively when writing disasters into the lives of their protagonists. Educators and students must reject singular narratives that exceptionalize, a point made by Gina Athena Ulysse in {\em Why Haiti Needs New Narratives: A Post-quake Chronicle (Middletown, CT: Wesleyan University Press,} {\em 2015)}. This element of teaching remains critical as disasters continue to unfold and their inclusion in curricula must be approached with care and humanizing practices that center Caribbean priorities, insights, and social action. As we finalize this essay, we grieve for the lives lost in southern Haiti after a category 7.2 earthquake on 14 August 2021.} relate to transnational social imaginaries and how they reveal and exacerbate long-standing inequities often rooted in colonial and imperial histories.\footnote{We thank the anonymous reviewer who provided a poignant example: the permanence of an internally displaced persons camp on the Google Map of Pétion-Ville (outside Port-au-Prince, Haiti) when the camp no longer remains; see \quotation{Pétion-Ville} in \useURL[url37][https://www.google.com/maps/place/P\%C3\%A9tion-Ville,+Ha\%C3\%AFti/@18.5272123,-72.2970543,14z/data=!4m5!3m4!1s0x8eb9e862f231e3df:0x829b82408f38c086!8m2!3d18.5138139!4d-72.2881907][][Google maps]\from[url37].}

\subsubsection[title={(Un)natural disasters},reference={unnatural-disasters}]

\subsubsubsection[title={Crash of Flight 587 (2001)},reference={crash-of-flight-587-2001}]

\startitemize[packed]
\item
  \quotation{\useURL[url38][https://www.theatlantic.com/national/archive/2011/11/remembering-americas-second-deadliest-plane-crash/248313/][][Remembering America's Second-Deadliest Plane Crash]\from[url38]}
\item
  \quotation{\useURL[url39][https://espaillat.house.gov/media/in-the-news/uptown\%E2\%80\%99s-adriano-espaillat-honors-memory-victims-flight-587][][Uptown's Adriano Espaillat Honors the Memory of the Victims of Flight 587]\from[url39]}
\item
  \quotation{\useURL[url40][https://www.washingtonpost.com/archive/politics/2001/11/17/airline-will-pay-crash-victims-families/8441b3f5-2745-4012-bf60-2d8092169558/][][Airline Will Pay Crash Victims' Families]\from[url40]}
\stopitemize

\subsubsubsection[title={Earthquake in Port-au-Prince, Haiti (2010)},reference={earthquake-in-port-au-prince-haiti-2010}]

\startitemize[packed]
\item
  \quotation{\useURL[url41][https://www.npr.org/templates/story/story.php?storyId=122567412][][Amid Rubble and Ruin, Our Duty to Haiti Remains]\from[url41]}
\item
  {\em \useURL[url42][https://www.powells.com/book/eight-days-a-story-of-haiti-9780545278492?partnerid=33733][][Eight Days: A Story of Haiti]\from[url42]}
\item
  \quotation{\useURL[url43][https://www.nytimes.com/2010/04/12/opinion/12trouillot.html][][From Disaster, Emerging Life]\from[url43]}
\item
  \quotation{\useURL[url44][https://africasacountry.com/2015/10/digital-archive-no-20-the-haiti-memory-project][][The Memory Box]\from[url44]}
\stopitemize

\subsubsubsection[title={Hurricane María (2017)},reference={hurricane-maría-2017}]

\startitemize[packed]
\item
  \quotation{\useURL[url45][https://dloc.com/AA00079427/00001?search=hurricane+=maria][][Archivo del Huracán María]\from[url45]}
\item
  \quotation{\useURL[url46][https://dloc.com/AA00077916/00001?search=hurricane+=maria][][The Puerto Rico Disaster Archive: Preserving the Cultural Legacy of Puerto Rico]\from[url46]}
\item
  \quotation{\useURL[url47][https://dloc.com/AA00062610/00001?search=hurricane+=maria][][La ruta de María]\from[url47]}
\item
  \quotation{\useURL[url48][https://listeningtopuertorico.org][][Listening to Puerto Rico]\from[url48]}
\item
  \quotation{\useURL[url49][https://puertoricosyllabus.com/syllabus/hurricane-maria/][][Hurricane María]\from[url49]}
\item
  \quotation{\useURL[url50][http://xpmethod.columbia.edu/events/2017-09-29-puerto-rico-mapathon.html][][Puerto Rico Mapathon for Hurricane Relief]\from[url50]}
\stopitemize

As students read, they should take notes of key passages that resonate with them. In particular, we recommend that students focus on excerpts that shed light on the placemaking practices that mediate experiences of migration, collective memory of (un)natural disasters, and complex notions of home. Students should be encouraged to group quotes thematically as they read their novel, staying attuned to questions that these quotes raise or the topics they feel compelled to explore in more detail. Students can regularly share and discuss these quotes in the small groups formed with others reading the same novel. Then, in the mixed groups, the quotes will be the basis for the digital research using dLOC and other resources. In these second groups, students will add new slides to the presentations developed for part I of their digital projects, this time complementing textual representations of placemaking (i.e., quotes from the text) with multimodal resources that illustrate more clearly how meaning is negotiated and how people, ideas, concepts, practices, and things interact and travel across time and space. The examples that follow illustrate how a set of quotes might guide the types of resources that students review and incorporate into their digital projects.

\subsubsection[title={Music and Popular Culture in Never Look Back},reference={music-and-popular-culture-in-never-look-back}]

In {\em Never Look Back}, Eury and Pheus connect over music. While Eury tunes out the world around her and listens to Prince on repeat, Pheus is widely known in the Bronx for his talents as a bachatero. In allusions to the Greek legend of Orpheus and Eurydice, those musical talents prove to be a possible downfall as Pheus must overcome his self-absorption to rescue Eury from Ato in the Underworld. The portal to the Underworld opens at a nightclub, where Pheus performs for a large crowd and largely ignores Eury's concerns about the owner who keeps following her. The quotes below illustrate the central role that music plays in Eury's memories of Puerto Rico, and in her relationship with Pheus.

\startblockquote
\quotation{You must miss Puerto Rico,} I say. \quotation{Yes.} Her sadness returns and overshadows the good. \quotation{I miss it so much.} \quotation{The hurricane?} I ask. She nods. I don't know what it must be like to be forced to leave your home. I don't wish it on anyone. There is a silence, and although silence can make others feel uncomfortable, I accept it. It's okay to let the person find the right words to communicate. It is the same with music. (38)

Penelope smells like the ocean. The familiar scent permeates the bedroom, and thoughts of home overwhelm me. It doesn't matter how long I've been away or what new city I land in, the island is never far from my thoughts. A scent, a familiar phrase uttered by someone, a song. Signs are presented to me on a daily or sometimes hourly basis, beckoning back. (147)

I look over Pheus's application for the music program on a borrowed iPad. I'm reading his essay, which he titled \quotation{The History of My Lamentations.} He writes about bachata music and its connection to the Dominican Republic. How this music has traveled from an island to this city. (303)
\stopblockquote

Based on these quotes, students might choose to explore placemaking practices through sources that discuss Prince and the history of bachata in the United States.

\subsubsubsection[title={Prince and bachata},reference={prince-and-bachata}]

\startitemize[packed]
\item
  \quotation{\useURL[url51][https://www.nypl.org/blog/2021/06/01/sounds-black-music-many-kings-one-prince?aff=nyplwebsite][][The Sounds of Black Music: There Are Many Kings, but Only One Prince]\from[url51]}
\item
  \quotation{\useURL[url52][https://www.latinxproject.nyu.edu/intervenxions/a-history-of-dominican-music-in-the-united-states][][A History of Dominican Music in the United States]\from[url52]}
\stopitemize

As students explore the scene of the nightclub and the entrance to the Underworld, they may also choose to focus on descriptions of Carnival characters who, in the book, inspire fear and deter Pheus from his mission of finding Eury.

\startblockquote
All around the room, the tables are occupied with lechones, piglet devils with long, curved snouts and tall horns covered in tiny spikes. During Carnaval de Santiago in Santo Domingo, people dress as lechones in silk clothes adorned in sequins with their faces concealed under papier-maché masks. But these lechones are not wearing masks. They are real. (233)

The congueros bang with their eyes closed in ecstasy. Blood covers the skin of the congas. Down on the dance floor, lechones huddle close to women. They twirl and twirl their patterns to the rhythm of the demented bachata. (233)
\stopblockquote

Through studying these quotes, students can explore Carnival masks across different geographical sites in the Caribbean, comparing and contrasting Carnival traditions within and between countries and how these practices travel across diasporic communities.

\subsubsubsection[title={Carnival (figures represented in the Underworld)},reference={carnival-figures-represented-in-the-underworld}]

\startitemize[packed]
\item
  \useURL[url53][https://dloc.com/AA00068075/00005?search=carnaval+=dominican][][Images of {\em vejigantes} and {\em lechones} from Carnaval (Part One)]\from[url53]
\item
  \useURL[url54][https://dloc.com/AA00068071/00033?search=vejigante][][Images of {\em vejigantes} and {\em lechones} from Carnaval (Part Two)]\from[url54]
\item
  \useURL[url55][https://dloc.com/AA00068071/00040?search=vejigante][][Images of {\em vejigantes} and {\em lechones} from Carnaval (Part Three)]\from[url55]
\item
  \quotation{\useURL[url56][https://dloc.com/diasporaproject][][UPR: Caribbean Diaspora DH Center]\from[url56]}
\item
  \quotation{\useURL[url57][https://dloc.com/fietmasks][][UPR: Lowell Fiet Masks]\from[url57]}
\item
  \quotation{\useURL[url58][https://www.npr.org/series/975403901/kanaval][][Kanaval: Haitian Rhythms and the Music of New Orleans]\from[url58]}
\stopitemize

\subsubsection[title={Spirituality in American Street},reference={spirituality-in-american-street}]

In {\em American Street}, Fabiola finds strength in Haitian {\em lwa} who provide her guidance in navigating new situations. As the plot of the story unfolds, Fabiola begins to see her friends, family, and circumstances in relation to the {\em lwa}, shaping her perceptions of her own transnational placemaking practices and how the {\em lwa} intervene. The following quotes provide examples of how Fabiola talks about the {\em lwa} in relation to her everyday life.

\startblockquote
So that night, a rage builds up inside me. I am hot red. I am burning coals. I am a sharp dagger and Scotch bonnet peppers in rum---Ezili Danto's favorite things. But this is only a wish because my mother---the powerful mambo---is not here with her songs and prayers and drums and offerings to make it so. (248)

Whole cities can seek vengeance, too. And even the very earth we stand on can turn on us. I remember the rumbling sound of falling walls, of angry earth. And maybe the dead rose up out of the ground the day my country split in half, and the zombies, with their guardian, Baron Samedi, leading them, forced their way out of cemeteries in search of their murderers. (307)

When the car pulls away from the curb of the house on American Street and drives down Joy Road, I turn to see Papa Legba leaning against the lamppost with a cigar in his hand and his cane by his side. He turns to me with his white glistening eyes and tips his hat. I smile and nod and mouth mesi. Thank you. He has brought my mother to the other side. (324)
\stopblockquote

Students may seek to learn more about the {\em lwa} mentioned in the book and their role in Haitian Vodou spiritual practices. Here, we emphasize the importance of deconstructing myths about Haitian Vodou, centering Haitian perspectives on Vodou practices, and ensuring that students respect the sacred nature of the {\em lwa}. Students might also seek images of the Bwa Kayiman ceremony and make connections between Vodou practices, the Haitian Revolution, and Fabiola's preparation of soup {\em joumou} on Thanksgiving with the ingredients her aunt left out for her.

\subsubsubsection[title={Vodou},reference={vodou}]

\startitemize[packed]
\item
  \quotation{\useURL[url59][https://dloc.com/vodou][][Vodou Archive]\from[url59]}
\item
  \quotation{\useURL[url60][https://dloc.com/AA00068687/00001?search=religion+=philosophy+=religion+=philosophy][][Demystifying Haitian Vodou and Its Cultural Role in the Education of Haitian Americans: Dr.~Bayyinah Bello Interview]\from[url60]}
\item
  \quotation{\useURL[url61][https://dloc.com/AA00056966/00001?search=religion+=philosophy+=religion+=philosophy][][Dr.~Celucien Joseph Lecture: \quote{What Are They Saying about Vodou?}(four parts)]\from[url61]}
\item
  \quotation{\useURL[url62][https://haitianartsociety.org/petwo-ceremony-commemorating-bwa-kayiman-1950][][Petwo Ceremony Commemorating Bwa Kayiman, 1950]\from[url62]}
\stopitemize

\subsubsection[title={Transnational Imaginaries in Clap When You Land},reference={transnational-imaginaries-in-clap-when-you-land}]

{\em Clap When You Land} is the only novel included for this project that takes place in more than one physical geographic setting. Though Yahaira is located in New York City and her sister Camino in the Dominican Republic, both locations feature prominently in the imaginaries, memories, and future plans of each protagonist. In fact, Yahaira and Camino not only think about but experience both locations at some point throughout the novel. Students might be interested in exploring transnational placemaking practices, as discussed by both sisters as they consider what it means to be from here, from there, or {\em ni de aquí, ni de allá.}

\startblockquote
It's like he bridged himself across the Atlantic. Never fully here nor there. One toe in each country. Ni aquí ni allá. (360)

I've heard him tell of New York so often you'd think I was born to that skyline. Sometimes it feels like I have memories of his billiards, Tío's colmado, Yankee Stadium, as if they are places I grew up at, & not just the tall tales he's been sharing since I was a chamaquita on his knee.” (14, Camino)

I was raised so damn Dominican. Spanish my first language, bachata a reminder of the power of my body, plátano & salami for years before I ever tasted peanut butter & jelly sandwiches. If you ask me what I was, & you meant in terms of culture, I'd say Dominican. No hesitation, no question about it. Can you be from a place you have never been? (97, Yahaira)
\stopblockquote

Students might consider exploring sociocultural placemaking practices in iconic Dominican communities in New York City, such as Washington Heights, as well as in the Dominican Republic. Though the novel focuses on Sosúa, Dominican Republic, for the purposes of this project students could widen their search to include elements mentioned in the book that span location. We also include resources that challenge singular conceptions of {\em Dominicanidad} transnationally, particularly Black narratives and experiences erased by the national ideologies of cultural identity and history that often appear in museums and other institutions. Throughout this research, students should be encouraged to question absences and silences in conversations about identity and reject both essentialism and erasure.

\startitemize[packed]
\item
  \quotation{\useURL[url63][http://www.dominicanhistoricneighborhoods.com/][][An Interactive Map Showcasing the Geographic Boundaries of a Proposed Dominican Historic District in Washington Heights]\from[url63]}
\item
  \useURL[url64][https://www.dominicanwriters.com/product-page/ni-de-aqu\%C3\%AD-ni-de-all\%C3\%A1-an-anthology][][Ni de aquí, ni de allá: A Multi-perspective Account of the Dominican Diasporic Experience]\from[url64]
\item
  \quotation{\useURL[url65][https://centroleon.org.do/signos-de-identidad-2/][][Signos de identidad]\from[url65]}
\item
  \quotation{\useURL[url66][https://www.latinxproject.nyu.edu/sp2021/ciguapaunbound?fbclid=IwAR2z9hlDwJYjf3Va0BIKJHgQmUB_d83yEYDICzgdyI3DNWoNLDaJtte3Tek][][Ciguapa Unbound: Blackness, Gender & Transnational Geographies of Marronage]\from[url66]}
\item
  \quotation{\useURL[url67][https://soundcloud.com/yaleuniversity/ritm-conversations-episode-2-dixa-ramirez][][RITM Conversations: Episode \#2 Dixa Ramirez---Conversations in Race, Indigeneity, and Transnational Migration]\from[url67]}
\item
  \quotation{\useURL[url68][https://www.youtube.com/watch?v=zF_czSEbFbc][][Racexile and the Poetics of Dominicanidad in Diaspora]\from[url68]}
\stopitemize

\subsection[title={Conclusion},reference={conclusion}]

Students in US K--12 schools must have more opportunities to read texts authored by Black, Afro-Latinx, and/or Caribbean diasporic writers. As described throughout this essay, the YA novels by Lilliam Rivera, Ibi Zoboi, and Elizabeth Acevedo contribute to relational understandings of Caribbean and Black diasporic migrations, (im)mobilities, and placemaking practices that occur in transnational discursive spaces. By teaching and learning about the world through the storying practices of the Caribbean and Caribbean diasporas, secondary language arts teachers can disrupt narrow conceptions of literacy and facilitate educational experiences in which students truly read and write more \quotation{widely and deeply.} In drawing from the Digital Library of the Caribbean and other sources, the digital projects and pedagogical practices proposed in this essay not only highlight for teachers and students the significance of archives for understanding migrations, (im)mobilities, and diasporic placemaking but also encourage students to interrogate existing archives and participate in the creation of materials that have been historically excluded or that document alternative perspectives. As we write, the K--12 education system in the United States faces ongoing pushback against teaching accurate accounts of US history that highlight the roles of white supremacy, racism, colonial and imperial intervention, capitalism, and other ideologies and systems in producing and maintaining inequalities. It is thus paramount to remain steadfast in pedagogically supporting teachers as they transform their curriculum and pedagogical practices to reveal the interconnectedness of the United States with other geographic locations while centering the stories and histories of Black diasporic communities. This essay is one contribution toward these long-standing and continuous efforts.

\thinrule

\page
\subsection{Molly Hamm-Rodríguez}

Molly Hamm-Rodríguez is a PhD candidate in equity, bilingualism, and biliteracy (with graduate certificates in culture, language, and social practice from the Department of Linguistics and Comparative Ethnic Studies from the Department of Ethnic Studies) at the University of Colorado Boulder. She received a Fulbright Doctoral Dissertation Research Abroad award to complete her dissertation research in the Dominican Republic using a raciolinguistic perspective to explore how language and literacy shape youth futures through education and employment interventions in communities affected by tourism. She is currently co-coordinator of the Transnational Hispaniola Working Group of the Caribbean Studies Association.

\subsection{Lisa Ortiz}

Lisa Ortiz is an assistant professor in the Department of Teaching, Learning, and Leading at the University of Pittsburgh. She has taught courses in education, English education, gender, women's, and sexuality studies, Latina/o/x studies, and writing. Her book project \quotation{Saberes Boricuas: 21st Century Migrant Placemaking at Work, Church, and School} focuses on knowledge production among intergenerational migrants moving between rural Puerto Rico and the rural Midwest from an intersectional, decolonial, and contemporary perspective.

\stopchapter
\stoptext