\setvariables[article][shortauthor={Jean-Charles}, date={May 2022}, issue={6}, DOI={Upcoming}]

\setupinteraction[title={Island Pedagogies: Discovery, Engagement, and Connection},author={Régine Michelle Jean-Charles}, date={May 2022}, subtitle={Island Pedagogies}, state=start, color=black, style=\tf]
\environment env_journal


\starttext


\startchapter[title={Island Pedagogies: Discovery, Engagement, and Connection}
, marking={Island Pedagogies}
, bookmark={Island Pedagogies: Discovery, Engagement, and Connection}]


\startlines
{\bf
Régine Michelle Jean-Charles
}
\stoplines


I recently made a professional move from one university to another, and thus had to execute the simultaneously overwhelming and rewarding task of sorting through over a decade of paperwork. The process of sorting, discarding, and packing prompted me to think of my own self-archiving practices in relation to Brent Hayes Edwards insightful observation: \quotation{As anyone who has spent any time in an archive knows, the vast majority of what one finds there is mundane and monotonous: the record of the inexorable production of the ordinary. Even when one happens across something unusual, it is most often recalcitrant, even gnomic, a lump that cannot be placed into anything approaching historical intelligibility.}\footnote{Edwards, Brent Hayes. \quotation{The Taste of the Archive.}~Callaloo~35, no. 4 (2012): 944-972.~\useURL[url1][http://doi.org/10.1353/cal.2013.0002][][doi:10.1353/cal.2013.0002]\from[url1].} Indeed, the vast majority of what I found was mundane and monotonous. It was also sometimes curious. Sifting through my various and sundry papers, notes, and articles at times made me wonder why on earth did I keep this or that. These encounters offered insight into what was most important to me at different points of my life as a thinker from the time I was in undergrad, then graduate school, as an assistant professor and then associate professor. There were notes from the first time I studied abroad in Paris and wrote a {\em mémoire} about the origins of the Négritude movement. I had kept notes form interviews conducted with Haitian authors living in Dakar from a summer research project I conducted as a graduate student. I came across numerous reading notes, print outs of articles, essays from graduate school, notes from talks I attended in college, and other texts that I was sometimes unsure why I had kept. I was experiencing my own form of self-archiving---noticing the quirks, gaps, and random preoccupations in my praxis as an archivist.

Among the pages I sorted through during those weeks, I located a number of printouts from {\em Île en île}, which served as an invaluable resource to me while I was in graduate school. There was the page of Cléante Valcin written by Nadève Ménard, another featuring Marie-Thérèse Colimon-Hall by Paulette Poujol Oriol, and several entries by Joëlle Vitiello that included feminist authors like Marie Vieux Chauvet and Paulette Poujol Oriol. Sorting through the pages, I noted that I had printed the pages of almost every Haitian woman writer in the database. Tucked away in folders and file devoted to the preparation of my PhD general exams, I also found Jacques Roumain, Aimé Césaire, and Jacques Stephen Alexis.

But it was the consistent presence of the women writers that I found most striking. It was as though my desire for more scholarship about Haitian women authors was temporarily satiated by what {\em Île en île} had to offer. At the time {\em Île en île} filled a void. My printed pages included authors that I have written extensively about as well as some whose work I have yet to delve into. Among these pages I found women like Marlène Rigaud Appollon and Margaret Papillon whose writings I was aware of but had never explored beyond my printed pages. Pouring over the creation of my own archive re-affirmed the significant and formative role that {\em Île en île} played in my career trajectory and intellectual life--- {\em Île en île} was foundational to me as a student, teacher, and researcher because it provided a resource and offered a point of departure for the subjects that would become my intellectual preoccupations.

Reflecting on {\em Île en île}'s impact on my formative thinking also relates to the role it has played in my entire career as a scholar and a teacher. As a pedagogical tool, {\em Île en île} helped to expand and enhance how we teach francophone literature and culture. In over a decade of teaching I have become absolutely convinced that the liberal arts classroom can be a space that exposes students to a range of materials, ignites intellectual curiosity, takes them outside of their comfort zones, challenges their perspectives, sharpens critical thinking, and proposes multiple ways of seeing the world. In developing my pedagogical praxis, I have experienced the classroom as a space of discovery, engagement, and connectivity. Using {\em Île en île} over the years has demonstrated its utility as a tool that ignites discovery, enables engagement, and fosters connection.

Since 1998 {\em Île en île} has been an unparalleled resource for teaching francophone studies, without which many of us would not have had a starting point for our burning questions. Especially for those of us teaching and researching from the United States, {\em Île en île} has offered access to documents, photographs, interviews, biographies, and literary texts in French for students to learn from and work with. I will speak in the specific context of Haitian literature, which is my primary area of research as I go into detail about how {\em Île en île} animates discovery, engagement, and connection for learners at every level. Before considering each of these areas, it is worth noting that {\em Île en île} was a digital humanities project that preceded the advent of the digital humanities. In this way it groundbreakingly set the tone for the possibilities of how francophone studies might live in the digital sphere. As a digital publication, {\em Île en île} opened the door to imagining more expansive possibilities for how literature and culture specifically, and the humanities in general circulated in the specific context of world literatures in French. Widely accessible and universally usable, {\em Île en île} was prescient in both its approach and scope.

\subsection[title={I. Discovery},reference={i.-discovery}]

As a digital archive, {\em Île en île} invites/ignites discovery. It is a perfect \quotation{first stop} for the early stages of research and allows students to discover the array of sources and connections. One of my time-tested activities for having students engage {\em Île en île} is to ask them to peruse the site on their own and report back on something new that they have learned or observed. This has resulted in fascinating classroom conversations about what makes up the \quotation{francophone world,} how different islands can be grouped together, and how to think about geographies in relation to literature and culture. I have found this practice to be a productive way of stimulating dynamic intellectual exchanges because it allows the students to explore what they have discovered together and make observations about why certain information was lacking in their knowledge base. Although one of the most extensive areas that of {\em Île en île} is in the domain of literature, it is a thoroughly inter-disciplinary and multi-genre resource that includes culinary art, visual art, film, dance, language, literature and music. As such it offers students a more complete view of Haitian culture that extends beyond literary texts and literary history to invite them to make connections across medium.~ Links to venerated Haitian institutions like Centre d'Art allow students to discover how art and culture are being approached in the context of the country they are studying just as they offer insightful glimpses of the global French speaking world. The discovery that {\em Île en île} offers is, importantly, a point of entry that will lead the curious students to more sustained and generative encounters.

\subsection[title={II. Engagement/Engaged Pedagogy},reference={ii.-engagementengaged-pedagogy}]

Over the years I have witnessed how {\em Île en île} facilitates students' active participation in learning. This feature aligns well with the goals of \quotation{engaged pedagogy} as defined by bell hooks, which requires students to take responsibility for their learning, seeking not only to empower them, but also to take on the process of self-awareness, illuminating the self. Many of us who have had the pleasure of teaching in a different language have witnessed how discovery and engagement enhance learning. Language learning classrooms are known atmospheres of excitement and energy. I teach literature through a wide range of activities (small group exercises, oral presentations, discussion questions) and with different materials (film, visual art, documentaries, and music) and incorporating {\em Île en île} has allowed me to engage students in a different kind of activity-based learning that is also self-directed. The music, photography, written texts, and video clips present on {\em Île en île} reinforces that literature should not be examined in isolation from other forms of cultural production or academic inquiry. More specifically, because I recognize that my students are part of a larger epistemological universe that privileges the study of France as exclusively hexagonal, {\em Île en île} has been particularly helpful in providing a sense of how to study French in a global frame. By inviting my students to think about geography, race, and language I aim to challenge the dominance of France in their imaginations and in their learning. In some of these introductory classes, I have students conduct group presentations on the different geographic locations represented on the syllabus. This allowed them to collaborate and to learn more about the cultures and countries independently. That many of them took the time to watch a film outside of class, learn a dance, or even in one case prepare a dish that they shared with the class served to create community and for them to be innovative in how they connected these activities to the literary text.

\subsection[title={III. Connection},reference={iii.-connection}]

Finally, {\em Île en île} can help students to connect to a world beyond themselves and to each other. It has been well established that one of the benefits of the digital humanities is that it \quotation{places the world at your fingertips.} By diminishing the distance between students and their subject matter, {\em Île en île} functions as a form of encounter for students that allows them to have access to a rich archive of material. As a pedagogical tool, {\em Île en île} invites students into a relationship with the context of francophone literatures and cultures in multiple ways. It is students' relationships to the texts they encounter; the relationship of those texts to other texts and to their contexts; students' relationships to thinking, and to one another each make up an important part of the learning process. These relationships and thoughtful contemplation of them help to energize learning.~ In {\em Pedagogies of Crossing}, Jacqui Alexander writes: \quotation{I came to understand pedagogies in multiple ways: as something given, as in handed, revealed, as in breaking through, transgressing, disrupting, displacing, inverting inherited concepts and practices\ldots{}}\footnote{Alexander,~M. Jacqui. {\em Pedagogies of Crossing: Meditations on Feminism, Sexual Politics, Memory, and the Sacred.} Durham: Duke University Press, 2005.} For me, {\em Île en île} has given to me, revealed to me, and prodded me to push my students to challenge the concepts and practices that defined their study of French; it has been formative and transformative in my own understanding of pedagogies by encouraging discovery, connection, and re-imagining in the field of francophone studies.

\thinrule

\page
\subsection{Régine Michelle Jean-Charles}

Régine Michelle Jean-Charles is the Dean's Professor of Culture and Social Justice as well as Director of Africana Studies at Northeastern University. She is a Black feminist scholar who works at the intersections of race, gender and justice. Jean-Charles is the author of {\em Conflict Bodies: The Politics of Rape Representation in the Francophone Imaginary} (Columbus: Ohio State University Press, 2014), {\em The Trumpet of Conscience Today} (New York: Orbis Press, 2021) and the forthcoming {\em Looking for Other Worlds: Black Feminism and Haitian Fiction} (UVA Press).

\stopchapter
\stoptext