\setvariables[article][shortauthor={Collins, St.~Hubert}, date={May 2022}, issue={6}, DOI={https://doi.org/10.7916/archipelagos-v5bc-3957}]

\setupinteraction[title={The Digital Library of the Caribbean: At the Crossroads of Caribbean Memory and Sustainability},author={Perry Collins, Hadassah St.~Hubert}, date={May 2022}, subtitle={The Digital Library of the Caribbean}, state=start, color=black, style=\tf]
\environment env_journal


\starttext


\startchapter[title={The Digital Library of the Caribbean: At the Crossroads of Caribbean Memory and Sustainability}
, marking={The Digital Library of the Caribbean}
, bookmark={The Digital Library of the Caribbean: At the Crossroads of Caribbean Memory and Sustainability}]


\startlines
{\bf
Perry Collins
Hadassah St.~Hubert
}
\stoplines


The idea of a Caribbean library digitization project took root at the Association of Caribbean University, Research and Institutional Libraries (ACURIL) annual meetings in 2002. Judith Rogers, known as the \quotation{founding mother} of the Digital Library of the Caribbean, envisioned a platform that could provide access to cultural, historical, and research materials held in archives, libraries, and private collections throughout the Caribbean. Together with Erich Kesse at the University of Florida (UF) and Catherine Marsicek at Florida International University (FIU), Rogers established a working group that presented findings at ACURIL in 2004. A successful proposal to the U.S. Department of Education's Technological Innovation and Cooperation for Foreign Information Access supported creation of dLOC, a centralized repository of Caribbean information resources relying on decentralized digitization. According to dLOC's website, five Caribbean and four US-based libraries officially established dLOC on 17 July 2004 at an ACURIL planning meeting in San Juan, Puerto Rico.

dLOC began as a grant-funded program; today, it relies on both grants and in-kind support from library and archive professionals, scholars, and educators. It is currently administered by UF and FIU, hubs for hosting and facilitating shared governance, technological infrastructure, professional development, and outreach. The original network of nine founding institutions has grown to \useURL[url1][https://dloc.domains.uflib.ufl.edu/partners/][][nearly eighty partners]\from[url1] in the United States, the Caribbean, Canada, Central and South America, and Europe. dLOC has emphasized people, partnerships, collaboration, and building trust in regions and communities that have witnessed the theft and extraction of their cultural patrimony. Long-term work on preservation and digital access requires sustained commitment to build trust with partners who are cautious of engagement, especially with U.S.-based academic institutions.

In this special issue of {\em archipelagos}, contributors delve into dLOC's collections and engage with the ways scholars and students have created, contributed, and utilized content across disciplines and geographies. Nearly twenty years after dLOC was established, the preservation of and access to Caribbean patrimonial materials remains critical and has only grown in importance with the impacts of climate change, political instability, and lack of funding, and the very real threat to sustainability of physical collections. The issue's contributors acknowledge and reflect on their own positionality as scholars and teachers working with or in U.S. institutions while trying to provide access to cultural knowledge and history across diasporas. They draw from their experiences across the Caribbean in safeguarding cultural heritage through collaborations and partnerships that can be at times fraught with tension and complex power dynamics. Their goals often include pathways to decolonize knowledge on the ground and to disseminate knowledge beyond academic audiences.

\placefigure[here]{dLOC logo}{\externalfigure[issue06/dloc-logo.jpg]}


\subsection[title={Navigating and Negotiating Digitization},reference={navigating-and-negotiating-digitization}]

The creation of digital collections, including scoping, scanning, and describing materials, is a major focus of the issue and reflects dLOC's original emphasis on broadly expanding access. Natália Marques da Silva, Jean Mozart Féron, and Mireille Fombrun Mallebranche discuss sustainability and capacity to steward digitized objects. Through partnership and collaboration with dLOC, Musée Ogier-Fombrun has begun to catalog and digitize its collection and archives to promote accessibility while safeguarding its materials as Haiti endures the COVID-19 pandemic and persistent crises.

One-sided \quotation{partnerships} by institutions in the Global North have led many archives in the Caribbean to distrust universities, scholars, and other actors. Claire Payton, Anne Eller, and Lewis Ampidu Clorméus discuss these tensions during their project, funded by the Endangered Archives Program, to digitize nineteenth-century Haitian newspapers at the Bibliothèque Haïtienne des Frères de l'Instruction Chrétienne with dLOC support. While important to scholars and other researchers, support for archivists, librarians, and cultural heritage has not been at the forefront of government investment in the Caribbean. As archives struggle financially, facilitating access to archival materials can be seen as one pathway to support operational costs. The authors examine assumptions around open-access models and their potential to undermine equity as well as promote it; they call for centering local institutions in ways that \quotation{provide concrete and economic support.}

\subsection[title={Language, Sound, and Memory},reference={language-sound-and-memory}]

Several contributors highlight preservation of audiovisual, sonic, and oral history materials and the potential for access to such materials to \quotation{combat silences,} decolonize the archives, and engage local communities. The Archivo de Respuestas Emergencias de Puerto Rico (AREPR) project, co-led by authors Ricia Anne Chansky and Christina Boyles, refers to a digital repository of Puerto Rican emergency response artifacts pertaining to Hurricanes Irma and María (2017), earthquakes (2019--present), and COVID-19. AREPR and dLOC are working together to make these materials accessible while preserving the experiences of Puerto Ricans throughout climate crisis. Their work reinforces the personal voices of Caribbean community members, on the frontlines of an ongoing \quotation{climate emergency.}

Nicté Fuller Medina reflects on digital repatriation and access, considering ways to make linguistic and cultural knowledge accessible to source communities. Fuller Medina delves into a case study of a collection recorded in the late 1970s focused on sociolinguistic analysis of the Spanish spoken in Belize. This essay should resonate with readers immersed in theoretical, ethical, and practical approaches to stewarding aural materials created half a century or more ago. Like Chansky and Boyles, Fuller Medina considers ways of stewarding and sharing collections in ways that center community perspectives and patrimony.

\subsection[title={Pedagogy and Student Engagement},reference={pedagogy-and-student-engagement}]

dLOC's significance lies not only in its role as a preservation repository but also as a platform that exposes students to primary sources that prompt more textured understandings of the Caribbean and the Caribbean diaspora. Ivelisse Rivera Bonilla and Nadjah Ríos Villarini discuss the importance of authentically connecting students with communities outside the university, looking specifically to the Archivo Histórico de Vieques, a group of community archives stewarding physical and digital archival materials relating the struggle against the military presence by the U.S. Navy. Through their course, students engaged with histories of colonialism in Puerto Rico and developed digital stories around items in the collection.

Similarly examining pedagogical frameworks that probe \quotation{local and global articulations of power and resistance,} Molly Hamm-Rodríguez and Lisa Ortiz examine how the layering of text and archival materials from dLOC can support secondary students' understanding of literature. The authors recommend relevant thematic materials in dLOC that align with and contextualize three texts tracing the experience of Afro-Caribbean protagonists in adolescence. Encouraging teachers to prioritize a nuanced, critical lens, the authors \quotation{caution against pedagogical approaches that produce singular narratives about the Caribbean.}

\subsection[title={Interpreting and Contextualizing},reference={interpreting-and-contextualizing}]

Other contributors have modeled dLOC's significance as a vast research collection while reflecting on gaps and potential for further growth; while far from comprehensive, its 4 million pages of content provide entry points that interconnect across geographies. René J. Kooiker's piece about Edward Kamau Brathwaite at Carifesta 1972 attests to how researchers have been able to use dLOC to reconstruct anticolonial movements and historical events in the Caribbean using newspapers and ephemera provided by dLOC partners. In her essay on the 1970s feminist periodical {\em El Tacón de la Chancleta,} Elizabeth Crespo Kebler describes an effort to make coverage of abortion, women's health, childcare, and so on more visible in Puerto Rico while highlighting the magazine's role in spurring international conversation and activism across Latin America. Laura Vargas Zuleta describes the collaborative process of developing public digital scholarship that interprets a fictional text through the lens of contemporary events, forging interpretive connections between René Depestre's {\em Hadriana dans tous mes rêves} and events documented in the Haitian newspaper {\em Le Nouvelliste.}

These three essays all draw on a major strength of dLOC: its large collections of Caribbean newspapers from the eighteenth to the twenty-first centuries. Newspapers make up about 25 percent of digital content in the collection, and facilitating access is one area that has seen considerable investment from hub institutions, partners, and funders. While not heavily featured in this issue, increasingly researchers are seeking out new ways to analyze these titles at scale and to recover or reconstitute histories. The March 2020 issue of {\em archipelagos} featured work by Amalia Levi and Tara Inniss to piece together experiences of enslaved people reported in the {\em Barbados Mercury Gazette,} a white colonialist newspaper shared in dLOC.\footnote{Amalia S. Levi and Tara A. Inniss,\quotation{Decolonizing the Archival Record about the Enslaved: Digitizing the Barbados Mercury Gazette,} archipelagos 4 (March 2020), https://archipelagosjournal.org/issue04/levi-inniss-decolonizing.html.} In 2020--21, we co-led (with Miguel Asencio and Jamie Rogers at FIU) dLOC as Data, an initiative to share and contextualize underlying newspaper data.\footnote{dLOC as Data, \quotation{A Thematic Approach to Caribbean Newspapers,} accessed 23 January 2022, https://dlocasdata.domains.uflib.ufl.edu/.} We expect to see interest in newspapers, as well as experimentation with a range of methodological approaches, continue to develop across the dLOC and wider Caribbean studies network.

Our perspectives as coeditors are complicated and enriched by our own work within dLOC and peripheral support for some of the projects and collections featured here. As workshop conveners and facilitators, advisors on data curation and rights issues, and well-wishers for the next funding opportunity or collection launch, we have each supported dLOC's individual partners and underlying mission. We approached each essay here through the lenses of our own lived and professional experiences, seeking to shape an issue that reflects dLOC as a wide-ranging source of digital content as well as a point of departure for fruitful and difficult conversations around the nature and purpose of that content. Learning from the issue's contributors, we can better consider the ethics of collection development, where growth has occurred unevenly or inequitably, and how scholars and educators are incorporating materials into their own work, often juxtaposed alongside other collections and data.

dLOC is at a crossroads administratively, seeking to steward collections while growing more robust support for partners' priorities. As T-Kay Sangwand notes, U.S.-based \quotation{archives must continually ask their partners and themselves: Does everyone have the responsibility to contribute labor and participate in social decisions?}\footnote{T-Kay Sangwand, \quotation{Preservation Is Political: Enacting Contributive Justice and Decolonizing Transnational Archival Collaborations,} KULA: Knowledge Creation, Dissemination, and Preservation Studies 2, no. 1 (2018): 10, https://doi.org/10.5334/kula.36.} To fully invest in post-custodial archival initiatives, US-based institutions should provide the labor and resources necessary for equitable partnerships. This is a challenge, requiring a firm commitment to long-term investment by universities to strengthen a community of practice and trust built over time. dLOC has assisted in building capacity for Caribbean institutions through trainings and collaborative grant projects, but dLOC's own administrative capacity has been stretched thin, without sufficient support for staff, especially term staff, as they work to engage a growing number of projects and partners. Current priorities include reinvigorating governance structures and hiring and retaining sufficient staff, including those of Caribbean and/or diasporic heritage with relevant language and technical skills, to sustain relationships with partners and to grow creation and use of collections. As the essays in this issue demonstrate, prioritizing multilingual access to source communities in the Caribbean and working alongside partners to address digital divides are key aspects to increasing equity.

\subsection[title={Acknowledgments},reference={acknowledgments}]

We dedicate this special issue to those who have been committed to building dLOC's infrastructure and social capacity, including, but not limited to, Judith Rogers, Erich Kesse, Catherine Marsicek, Laurie Taylor, Brooke Wooldridge, Vicki Silvera, Rose Nicholson, Rhia Rae, Annia González,~Liesl Picard, Chantalle Verna, Miguel Asencio, Laura Perry, Chelsea Dinsmore, Leah Rosenberg, Mark Sullivan, Bess de Farber, Margarita Vargas Betancourt, Melissa Jerome,~Gayle Williams, Judith Russell, Laura Probst, Jean Wilfrid Bertrand, Marie-France Guillaume, Lusiola Castillo, Margo Groenewoud, Sandra Barker, Astrid Britten, Barry Baker, Matthew Smith, Adam Silvia, Kimberly Green, Mireille Charles, and many others. Thanks to all past and present Scholarly Advisory Board and Executive Board members, and representatives from the host institutions.

We also thank Kaiama Glover, Alex Gil, and Kelly Baker Josephs, who continue to provide us with spaces to dream new futures together.

\thinrule

\page
\subsection{Perry Collins}

Perry Collins is the Copyright & Open Educational Resources Librarian at the University of Florida Libraries, where she manages initiatives promoting open access in education, copyright literacy, and ethical approaches to digital scholarship. In this role, she acts as a liaison to dLOC, including collaborative engagement with partners, researchers, and students seeking to share and reuse collections. Before joining UF in 2018, Collins held a similar position at the Ball State University Libraries in Muncie, Indiana, and worked for six years as a program officer in the Office of Digital Humanities at the National Endowment for the Humanities. Collins holds an MLIS from the University of Illinois at Urbana-Champaign and an MA in American Studies from the University of Kansas.

\subsection{Hadassah St.~Hubert}

Hadassah St.~Hubert, Ph.D.~is a Historian, Independent Scholar, and Senior Program Officer. She is a former Council on Library Information Resources (CLIR) Postdoctoral Fellow in Data Curation for Latin American and Caribbean Studies with dLOC at Florida International University. She received a PhD in History from the University of Miami specializing in Caribbean, Latin American, and African Diasporic history. Her dissertation, \quotation{Visions of a Modern Nation: Haiti at the World's Fairs,} focuses on Haiti's participation in World's Fairs and Expositions in the twentieth century. Dr.~St.~Hubert served as the Assistant Editor for Haiti: An Island Luminous, a digital humanities site dedicated entirely to Haitian history and Haitian studies. During her tenure as the CLIR Postdoctoral Fellow with dLOC, she led programming and digitization efforts in collaboration with dLOC's partners, such as the Diaspora Vibe Cultural Arts Incubator and L'Institut de Sauvegarde du Patrimoine National in Haiti.

\stopchapter
\stoptext