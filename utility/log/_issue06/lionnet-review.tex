\setvariables[article][shortauthor={Lionnet}, date={May 2022}, issue={6}, DOI={Upcoming}]

\setupinteraction[title={An Archipelagic Antidote to Critical Clichés},author={Françoise Lionnet}, date={May 2022}, subtitle={An Archipelagic Antidote}, state=start, color=black, style=\tf]
\environment env_journal


\starttext


\startchapter[title={An Archipelagic Antidote to Critical Clichés}
, marking={An Archipelagic Antidote}
, bookmark={An Archipelagic Antidote to Critical Clichés}]


\startlines
{\bf
Françoise Lionnet
}
\stoplines


When I first started to teach francophone literature as a \quotation{field} in the late 1980s, it was a challenge to demarcate \quotation{island studies} as a legitimate area that would make sense, both pedagogically and intellectually, to my students as much as to my colleagues. It was hard enough to get \quotation{francophone studies} accepted in French departments, but a focus on islands? \quotation{Too limited, too superficial, and of no intellectual interest}---so went the unstated general opinion. The idea of islands was (and still is for many) associated with the lure of tourism, with dreams of escape, leisure, and natural beauty. That islands could have distinct cultural histories, that they were an essential part of the process of globalization launched by European colonial and imperial expansion, and that they continue to play a role in defining continental as well as insular identities were not considered facts of any educational significance. Hadn't Napoléon referred to the \quotation{vieilles colonies} as mere \quotation{confettis d'Empire}? Hadn't De Gaulle dismissed the Antilles as \quotation{des poussières d'îles} during his infamous 1960 trip to the French Caribbean? No self-respecting professor of literature took the insular tropics to be a serious topic of study beyond the themes of exoticism and utopia.

For the inhabitants of island colonies, the sentiment is a familiar one. Islands do not have the same status imaginatively or politically as the continents of Europe, Africa, and Asia. Islands do not bestow on the traveler the same aura of acquired knowledge or esoteric wisdom; they are mythical, seem unreal, and tend to be viewed as places of escape and rest, hideaways onto which an infinite number of dreams and desires can be projected. They do not appear to have cultural integrity as do older civilizations. They are the residues of Europe's dream of empire, each a {\em tabula rasa} on which to write narratives of rebirth and renewal, not unique landmasses weighed down with histories of violent conquest, settlement, and the exploitation of natural resources.

It is thus to Thomas C. Spear's credit that he had the vision to conceive of a project such as {\em Île en île} that could begin to shift the academic narrative and provide an invaluable database for students in francophone studies. From history to visual arts, literature to nature, geography to cuisine, the site has provided deeper and richer contexts for students deciphering novels and poems about insular worlds whose existence they never suspected---or previously thought to be as shallow as the turquoise lagoons that typically surround tropical islands.

When audio and video files of interviews with writers began to be added, insights into the cultural and linguistic complexity of life in multiethnic Creole environments suddenly began to resonate powerfully with my UCLA students. Their own experiences as first-generation commuter students, growing up in immigrant families in a global megalopolis with its own creolized neighborhoods, provided them with a point of view that highlighted affinities between Los Angeles and Mauritius, for example. They were surprised to discover that daily life on an island on the other side of the world could mirror some of their own preoccupations. Far from being exotic destinations, Mauritius or Martinique came alive as the home of multilingual writers who write eloquently about cultural, familial, and personal journeys featuring stunning but familiar landscapes (sea, sun, swaying palm trees, imposing mountains, natural disasters . . . ), as well as the dark backdrops of colonial abuse, slavery, migration, and poverty.

The contradictions and disjunctures voiced by the citizens of these islands allowed the students to reflect on their own situation, on the role of history in the problems of the present, and I would submit, to become more enlightened citizens of their own polis, better able to understand what's at stake in the educational and cultural politics of the present. Could they have come to such conclusions without {\em Île en île}? Maybe. But what the site did and continues to do is provide a critical mass of research materials that simply and effectively contradicts all conservative platitudes about the so-called shallowness and lack of theoretical sophistication of francophone island studies. In addition, to view the world, however briefly, from the perspective of these small places is to return to French hexagonal literature with a renewed understanding of how clichés and stereotypes can construct or deform reality, and also how interpretive readings can never be set in stone, since new information will always provide fresh perspectives, even on canonical texts already mined by many brilliant critics. For example, I use {\em Île en île} as a resource when I teach Baudelaire's Indian Ocean poems, George Sand's {\em Indiana}, and André Breton's surrealist poetry. Students can then understand how literary representation works, how critics can misread the original texts, and how interpretive mistakes can become part of an archive of received knowledge that has little bearing on cultural truths on the ground, thus perpetuating clichés about insularity, exoticism, and tropical idleness.

In {\em A Small Place}, Jamaica Kincaid brilliantly evokes seeing oneself through foreign eyes and the deadening effect of tourism on small islands as she proceeds to interrupt that gaze by articulating the analogies between global tourism and the superficial reading practices of uninformed critics. {\em Île en île} is the antidote to tourist guidebooks and pedantic critics, it gives pride of place to the voices of writers who articulate with precision and style the agency and subjectivity of those molded by insular exiguity. These are writers who also understand the historic role of islands as crucial nodes in global networks of trade and culture.

Having worked with Spear in one context or another since the late 1980s, it is an honor to take this opportunity to thank him for all he has done to center the study of francophone islands. I am particularly appreciative of the way he has helped bring more attention to the Indian Ocean, including the conference we coorganized in Mauritius in 2009, with papers subsequently published in a special issue of the {\em International Journal of Francophone Studies} in 2010--11. It is a challenge to bring that region into focus for North American readers; but the work of the past few decades has, fortunately, been successful in countering misinformation and carving an academic space for the study of a region rich in award-winning writers.

This is all the more important today since the US military presence in Diego Garcia shows no sign of weakening despite continued efforts on the part of displaced Chagossians and their descendants to reclaim their native islands. France, Great Britain, and the United States maintain a military and political presence in the area, which allows them to claim, directly and indirectly, a seat at the table when Indian Ocean rim countries meet. Shenaz Patel's {\em Le silence des Chagos} is to date the most eloquent piece of writing on the topic, from the perspective of those most affected by the injustice meted out by the United Kingdom and the United States. This book should be required reading for citizens of those nations who, for the most part, have no idea what impact their country may be having on such distant places. The imperial foothold of Europe and the United States in the Indian Ocean today is but a contemporary version of how Early Modern powers appropriated the islands that became their strategic possessions on the way to the riches of Asia. As the 1665 map by Pierre du Val, \quotation{Géographe Ordinaire du Roy,} shows, the islands scattered across the Mer des Indes formed part of the Indes Orientales, and the mapmakers took great care to represent them on a much larger comparative scale than the nearby continents, thereby indicating their importance to navigators and colonial powers (figure 1).

\placefigure[here]{Pierre du Val, \quote{Carte des Indes Orientales} (1665)}{\externalfigure[issue06/v3_lionnet_fig1.png]}


That region remains, for the citizens of the Americas, one of the least understood geopolitical and cultural formations, one that would qualify, in Édouard Glissant's words in {\em Le discours antillais}, as \quotation{la face cachée de la terre.}\footnote{Édouard Glissant, {\em Le discours antillais} (Paris: Seuil, 1981), 141.} Thanks to {\em Île en île}, there is now much more accessible information on the unique dynamics of the former \quotation{Mer des Indes.} The site has helped redraw maps of knowledge while opening new and ever-expanding areas of investigation.

\thinrule

\page
\subsection{Françoise Lionnet}

A past-President of the American Comparative Literature Association, Françoise Lionnet is Professor of Romance Languages and Literatures, Comparative Literature, African and African American Studies, and Women, Gender and Sexuality at Harvard. In Fall 2015, she held the Mary Cornille Distinguished Visiting Professorship at the Newhouse Humanities Center, Wellesley College. She remains a Distinguished Research Professor at UCLA, where she taught from 1998-2015, serving first as Chair of French and Francophone Studies, then as Director of the James Coleman African Studies Center. Her current research focuses primarily on Indian Ocean literary, cultural, and historical studies, in relation to Atlantic and Caribbean Studies. She is interested in the longue durée of colonialism in those regions. Her volume on the 18th century Creole abolitionist poet Evariste Parny appeared in the~MLA Texts and Translations series in 2018.~Her previous books include {\em Autobiographical Voices: Race, Gender, Self-Portraiture} and {\em Postcolonial Representations: Women, Literature, Identity} (both from Cornell UP), {\em Minor Transnationalism} (Duke UP), {\em Writing Women and Critical Dialogues: Subjectivity, Gender and Irony} and {\em The Known and the Uncertain: Creole Cosmopolitics of the Indian Ocean} (both published in Mauritius by l'Atelier d'écriture).

\stopchapter
\stoptext