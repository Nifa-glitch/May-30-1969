\setvariables[article][shortauthor={Payton, Eller, Clorméus}, date={May 2022}, issue={6}, DOI={https://doi.org/10.7916/archipelagos-y9ev-cd60}]

\setupinteraction[title={Many Lifetimes of Knowledge: The History of the Bibliothèque Haïtienne des Frères de l'Instruction Chrétienne's Newspaper Collection and Its Digital Future},author={Claire Antone Payton, Anne Eller, Lewis Ampidu Clorméus}, date={May 2022}, subtitle={Many Lifetimes of Knowledge}, state=start, color=black, style=\tf]
\environment env_journal


\starttext


\startchapter[title={Many Lifetimes of Knowledge: The History of the Bibliothèque Haïtienne des Frères de l'Instruction Chrétienne's Newspaper Collection and Its Digital Future}
, marking={Many Lifetimes of Knowledge}
, bookmark={Many Lifetimes of Knowledge: The History of the Bibliothèque Haïtienne des Frères de l'Instruction Chrétienne's Newspaper Collection and Its Digital Future}]


\startlines
{\bf
Claire Antone Payton
Anne Eller
Lewis Ampidu Clorméus
}
\stoplines


{\startnarrower\it This article details the course of a British National Library Endangered Archives Programme Grant to digitize a portion of the collections that have been maintained for generations at the Bibliothèque Haïtienne des Frères de l'Instruction Chrétienne (BHFIC) in Port-au-Prince. It offers a short history of the BHFIC, the collection and preservation efforts of collectors, students, alumni, and staff, some highlights from the nineteenth-century newspapers, and comments from its current Executive Director. Implementation of the grant was profoundly collaborative; dLOC offered additional critical support and BHFIC staff completed a significant digitization. This project, like others, grapples with the quandary that efforts to facilitate access to Haitian historical sources are constrained by the same inequalities that they seek to challenge. The close of the article seeks to reflect how such digitization initiatives might better serve cultural and historical institutions, including through the restriction of access to digital materials via a paywall.

 \stopnarrower}

\blank[2*line]
\blackrule[width=\textwidth,height=.01pt]
\blank[2*line]

\subsection[title={Digital Preservation Efforts, Archives, and Power},reference={digital-preservation-efforts-archives-and-power}]

In recent years, Haiti has inspired a profound reconceptualization of epistemologies of knowledge and power. This challenge includes critical evaluation of the dynamics of ontology and history as well as the materials, networks, and institutions that make that knowledge production possible.\footnote{Michel-Rolph Trouillot, {\em Silencing The Past: Power and the Production of History} (Boston: Beacon, 1995).} As consensus about Haiti's centrality grows, librarians, archivists, and scholars have mobilized to help preserve Haitian archival and printed material and increase access through digitization. At the forefront of this movement is the Digital Library of the Caribbean (dLOC), a cooperative digital library maintained by the University of Florida and Florida International University. Founded in 2004 with nine partners, today dLOC sustains collaborations with sixty cultural institutions. Another major supporter of preservation and access initiatives is the British Library's Endangered Archive Programme (EAP). Since 2005, the EAP has used private philanthropic donations to support the digitization of at-risk materials around the world. In the Caribbean, the EAP has supported projects in Haiti, Cuba, Jamaica, Montserrat, Turks and Caicos, Anguilla, Nevis, Antigua and Barbuda, Barbados, Grenada, St.~Vincent, and Colombia.\footnote{British Library, \quotation{Welcome to the Endangered Archives Programme,} accessed 18 June 2021, \useURL[url1][https://eap.bl.uk/welcome-endangered-archives-programme]\from[url1].}

The practices of these institutions reflect the transnational nature of intellectual networks that have sustained knowledge production about the Caribbean. They also facilitate access to disparate collections of cultural and political heritage for a putatively global audience. As one of the philanthropists behind the EAP has written, \quotation{Simply put: if this is the memory of the world, the world needs to be able to access it.}\footnote{Lisa Rausing and Peter Baldwin, introduction to Maja Kominko, ed., {\em From Dust to Digital: Ten Years of the Endangered Archives Programme} (Cambridge, UK: Open Book, 2015), xxxviii.} But technology's power to shape dislocated international audiences runs the risk of erasing complex, deeply rooted intellectual and institutional histories that underpin every individual collection and the libraries that protect them. While efforts to preserve, digitize, and expand access to newspapers, ephemera, and state and private archives sometimes attempt to redress the violent colonial dynamics of the past, the invocation of a global audience that \quotation{needs to be able to access} materials from around the world runs the risk of reinforcing, through a claim to universalism, the \quotation{epistemic injustices} that shaped such profound inequalities in the first place.\footnote{Denisse Albornoz, Angela Okune, and Leslie Chan, \quotation{Can Open Scholarly Practices Redress Epistemic Injustice?,} in Martin Paul Eve and Jonathan Gray, eds., {\em Reassembling Scholarly Communications: Histories, Infrastructures, and Global Politics of Open Access} (Boston: MIT Press, 2020). See also Marlene L. Daut, \quotation{Haiti @ the Digital Crossroads: Archiving Black Sovereignty,} {\em archipelagos} 3 (July 2019), https://archipelagosjournal.org/issue03/daut.html; Chelsea Stieber, Laura Wrubel, and Watson Denis, \quotation{Cross-Boundary Digital Collaboration as Scholarly and Institutional Experimentation: Amplifying the Impact of Caribbean Periodicals,} {\em archipelagos} 4 (March 2020), https://archipelagosjournal.org/issue04/stieber-experimentation.html.} In this article, we analyze these dynamics by detailing a 2017--20 pilot grant project we organized to digitize nineteenth-century Haitian newspapers at the Bibliothèque Haïtienne des Frères de l'Instruction Chrétienne (BHFIC), funded by the EAP and implemented with support from dLOC.\footnote{The library is also colloquially known as the \quotation{Bibliothèque de Saint-Louis} given that it is the library maintained by alumni of the Institution Saint-Louis de Gonzague, a primary and secondary school discussed at length in this article. Sometimes others refer to it in English as the \quotation{Gonzague Library.}}

\placefigure[here]{The Bibliothèque Haïtienne des Frères de l'Instruction Chrétienne newspaper collection}{\externalfigure[issue06/valence_1_cp_photo-bhfic-new.jpg]}


\placefigure[here]{reading room}{\externalfigure[issue06/valence_1_cp_dsc02545.jpg]}


Sweeping digital repositories foster an ease of access and a sense of abundance. In the process, the origins and distinct contexts of materials can be lost, and with them the labor and insight of those who for decades or centuries have safeguarded them.\footnote{On these and other pitfalls, see Lara Putnam, \quotation{The Transnational and the Text-Searchable: Digitized Sources and the Shadows They Cast,} {\em American Historical Review} 121, no. 20 (2016): 377--402.} Before US- and UK-based programs like dLOC and the EAP made Caribbean historical materials available worldwide, local collectors, librarians, historians, and archivists assembled, maintained, and protected Caribbean historical-cultural materials while sustaining a number of transnational collaborations. This article aims to offset any sense of watershed emergence that may accompany digitization by showing how these initiatives represent but another chapter in the BHFIC's long institutional history. We emphasize the transnational nature of the intellectual production and preservation efforts both past and present. Finally, we conclude by reflecting on what our experience can illuminate about the fraught politics of knowledge production and the tensions between an open-access ethos rooted in ideas of universalism and the particular preservation efforts, vision, and material needs of the BHFIC. This intellectual collaboration unfolds within a context of stark material disparity that shaped the digitization process and throws into relief critical questions about digital access: specifically, the viability of open access or subscription models.

Throughout this article, we underscore the enormous importance of the collections in question: twenty-seven different nineteenth-century Haitian newspapers, encompassing almost ten thousand pages of printed materials. These papers respond to Gina Ulysse's famous call for \quotation{new narratives} of Haiti by making it possible for readers to explore what Haiti and Haitian society meant to the men and women who lived it and excavate complex political, social, and intellectual worlds. One of the most critical implications is their potential to deepen understanding not only of the elite politics that often dominate their pages but also all of the popular politics that also emerge. The materials facilitate narratives beyond what North Atlantic scholars \quotation{desire} Haiti to be (to borrow from Chelsea Stieber), shifting analysis away from the symbolic importance of the country elsewhere to a better understanding of the dynamics of power as they were lived.\footnote{Chelsea Stieber, {\em Haiti's Paper War: Post-independence Writing, Civil War, and the Making of the Republic, 1804--1954} (New York: NYU Press, 2020), 1, 258.} These texts present new vantage points onto questions of Haitian politics, culture, society, and history as seen and represented by Haitians themselves in the nineteenth century. They make it possible to see how dynamics from the revolutionary period and early independence years transformed or continued in subsequent decades. They also highlight the longer genealogies of power, politics, imperialism, urban contests, which extend into the twentieth century and beyond.

\subsection[title={The Newspapers, the Capital, and the BHFIC},reference={the-newspapers-the-capital-and-the-bhfic}]

The Bibliothèque Haïtienne des Frères de l'Instruction Chrétienne, a modest private library on the campus of the prestigious Institution Saint-Louis de Gonzague on the rue de Centre, is in the heart of downtown Port-au-Prince. The BHFIC collections represent one of the most important repositories of Haitian history in the world. The library houses approximately ten thousand books on the country's history and literature, along with a collection of more than two hundred newspapers. It hosts a unique collection of material ephemera, including an important set of stamps and coins. The library's primary public is Haitian secondary and college students, as well as professors, writers, and researchers. Individuals who need to consult legal records also frequent the library to find documentation of a legal proceeding or decision. \quotation{It is a magnificent library,} the executive director, Marie-France Guillaume, said in a recent interview. \quotation{For researchers, students, for anyone interested in Haiti whether they are in Haiti or abroad, it is a mine of knowledge. A lifetime is not long enough to discover everything the library holds.}\footnote{Interview with Marie-France Guillaume, 15 June 2021.}

Founded in 1912, the BHFIC collections are the product of French religious missionary activity in Haiti, which has a long and fraught history. After the Haitian state signed a concordat with the Holy See in 1860, French clergy, especially from the region of Brittany, arrived with the purported mission of civilizing and moralizing the population.\footnote{Lewis A. Clorméus, \quotation{Aspects des relations entre l'Église et l'État en Haïti: Le concordat du 28 mars 1860,} {\em Revue de droit canonique} 64, no. 2 (2014): 277--309; Julia Gaffield, \quotation{The Racialization of International Law in the Aftermath of the Haitian Revolution: The Holy See and National Sovereignty,} {\em American Historical Review} 125, no. 3 (2020): 841--68.} Indeed, the episcopate focused primarily on the \quotation{salvation} of Haitians by actively waging campaigns against freemasonry and Vodou.\footnote{Lewis A. Clorméus, \quotation{A propos de la première campagne anti-superstitieuse (1896--1900),} {\em Chemins Critiques} 6, no. 2 (August 2019): 89--114; Clorméus, \quotation{Les stratégies de lutte contre la \quote{superstition} en Haïti au XIXe siècle,} {\em Journal of Haitian Studies} 20, no. 2 (2014): 66--87; Clorméus, \quotation{La soutane contre le tablier: Au cœur des tensions entre le clergé breton et la franc-maçonnerie haïtienne au XIXe siècle (1867--1900),} {\em Histoire Monde et Cultures Religieuses}, no. 29 (March 2014): 33--56; Clorméus, \quotation{Haïti et le conflit des deux \quote{France,}} {\em Chrétiens et Sociétés}, no. 20 (2013): 63--84. See also Kate Ramsey, {\em The Spirits and the Law: Vodou and Power in Haiti} (Chicago: University of Chicago Press, 2011).} The clergy readily embraced education as one of the most effective ways to appeal to the Catholic and Francophile elements of the local elite. In this climate, new parochial schools opened in the capital. In 1865, the Spiritans founded the Petit Séminaire Collège Saint-Martial, with the goal of setting up a school to rival those in Paris.\footnote{Georges Corvington, {\em Port-au-Prince au cours des ans: La métropole haïtienne du XIXième siècle, 1804--1888}, 3rd ed.~(Port-au-Prince: Henri Deschamps, 1977), 127.} They also established a fire brigade, a small museum, and a library. At the end of the nineteenth century, two other congregational schools competed with the best public and private establishments in the capital: Sainte-Rose de Lima and the Institution Saint-Louis de Gonzague, the future home of the BHFIC. These Catholic schools offered intellectual training to many who would ascend to the highest ranks in the army, commerce, and public administration. Their training at these elite institutions contributed to a large extent to the renewal of Francophile discourse through the 1930s and 1940s.\footnote{Lewis A. Clorméus, {\em Le vodou, le prêtre et l'ethnologue: Retour sur la polémique Joseph Foisset / Jacques Roumain (Haïti, 1942)} (Paris: Maisonneuve & Larose / Hémisphères, 2020).}

The content of the newspapers held in the BHFIC collections reflects the easy intermingling of these elite capital-city residents and missionaries at the close of the nineteenth century and into the twentieth. The collections offer a treasure trove of information; in this article, we gesture only to small representative examples. Their collections are replete with articles and commentary offering partial and often tendentious records that nevertheless shed a bright light on both elite and popular struggles, religious campaigns, cultural dynamics, and ordinary life and dreams in Haiti during these decades. Writers documented everyday concerns, social happenings, and local and international politics. They shared poetry, theater reviews, gossip, feuds, and humor. While journalists were often keenly attuned to the contests of the political elite, their writings also offer an imperfect window into the popular politics of the capital and other cities and towns, providing scholars and readers invaluable opportunities to build on the scholarship of Georges Corvington and others.\footnote{Of his multiple books written about the city, see especially Corvington, {\em Port-au-Prince au cours des ans: La métropole haïtienne du XIXième siècle, 1804--1888}, and {\em Port-au-Prince au cours des ans: La métropole haïtienne du XIXième siècle, 1888--1915} (Port-au-Prince: Imprimerie Henri Deschamps, 1976).} The depth of these documents' content likewise presents opportunities to counteract US-centric representations of the nineteenth century.\footnote{Daut, \quotation{Haiti @ the Digital Crossroads.}}

The journals focus on describing proper comportment and proscribing behavior viewed as unseemly among the bourgeoisie, as well as on disciplining the Haitian peasantry. In one such instance, a column from {\em Le Patriote} in February 1894 denounces an \quotation{extraordinary disorder.} The author explains how men in the countryside often left their wives and went to live with other women. He denounces the practice---\quotation{{\em se placer}, as it is vulgarly known}---and describes how parents brought children from these relationships to town centers and cities to be baptized. The \quotation{unfaithful and perjurious} husband would go before a civil officer to request {\em actes de reconnaissance} (birth certificates). Officers would grant the documents, the journalist wrote furiously, \quotation{without even bothering to investigate the true legitimacy of these children.} All the while, regulations about birth certificates were enforced with \quotation{rectitude and severity} when it came to residents of the capital, the writer complains. The author appears oblivious to the gulf between the rights and resources of the peasantry and those of the urban-based elite, but the bitterness of these lines hint how, at the same time that the bourgeoisie wanted to police the intimate lives of peasants, nineteenth-century elites felt that their own intimate lives were being policed by the state. \quotation{It is completely abnormal that our institutions and our laws are not equally practiced and observed by all,} the columnist gripes.\footnote{\quotation{Un fait anormal dans l'observance de nos institutions civiles,} {\em Le Patriote}, 8 February 1894. On popular efforts to seek inclusion through notaries, see also Winter Rae Schneider, \quotation{Between Sovereignty and Belonging: Women's Legal Testimonies in Nineteenth-Century Haiti,} {\em Journal of Caribbean History} 52, no. 2 (2018): 117--34.}

Given the stakes of the presidential and municipal political contests detailed in the newspapers, the BHFIC's collections offer many scenes of popular activism in the heart of the capital as well. In one such instance, a journalist writing for the {\em Revue des Tribunaux} reports a dramatic scene unfolding in the streets on 1 May 1843, just a few months after President Jean-Pierre Boyer hastily departed Haiti for Jamaica. He begins by describing two women loudly and forcefully protesting their detainment, as soldiers usher them to jail, on the orders of the {\em commandant de la place}. Whether it was their neighbors, family, or simple observers of the injustice, onlookers quickly gathered and insisted that the women be afforded a hearing by a justice of the peace or be freed. A fight nearly broke out when the soldiers continued detaining the women, one of whom was bleeding severely from a head wound. In response to increasing public protest, an officer offered to appeal to his superiors to request the women's release. \quotation{{\em Point de juge, point de condamnation} {[}no judge, no conviction{]},} members of the crowd shouted. \quotation{Release them! In the name of the sovereign people!} As the writer in the {\em Revue des Tribunaux} attests, hope (and democratic efforts) remained strong in the 1840s in the face of repression and uncertainty. \quotation{If there had been a judge,} the writer insists, \quotation{the people would have said nothing.}\footnote{\quotation{Un arrêt du peuple souverain,} {\em Revue des Tribunaux}, 1 May 1843, 146. For a larger gendered discussion of some of these nineteenth-century contests reported in these collections and in newspapers held at the Archive Nationale d'Haïti, see Anne Eller, \quotation{Skirts Rolled Up: The Gendered Terrain of Political Protest in Nineteenth-Century Port-au-Prince,} {\em Small Axe}, no. 64 (March 2021): 61--83.}

Critically, the newspapers detail the centrality and recurrence of political repression across the decades. In the 1880s and 1890s, the dual issues of judicial and penitentiary reform surfaced at the heart of public concern. This era is well known as the \quotation{age of the coco-macaque.}\footnote{Lewis Ampidu Clorméus, \quotation{Le militarisme au fondement de l'acceptabilité de l'arbitraire en Haïti au XIXe siècle,} in Bérard Cénatus, Stéphane Douailler, Michèle Pierre-Louis, and Étienne Tassin, eds., {\em Haïti: De la dictature à la démocratie?} (Montreal: Mémoire d'Encrier, 2016), 105--20.} Repression and militarism appear constantly in the digitized records, which clearly demonstrate the banalization of state violence and authorities' abuse of ordinary citizens. In one such example, a journalist writes about a general who ordered the imprisonment of a citizen simply because he was smoking as the general passed by.\footnote{\quotation{Assez de la prison,} {\em Le Réveil} (Cap-Haïtien), 18 February 1893, 3.} Other articles detail government forces' pursuit, imprisonment, and torture of opposition figures, particularly during elections. Candidates and their family members were terrorized; others withdrew from their races. Meanwhile, various articles describe the difficult conditions of jails and the mistreatment of prisoners.\footnote{Other articles detail a need to build more jails, a sentiment authorities echoed. Justice of the Peace Azaël Nicolas reported to the secretary of state for justice that \quotation{in the absence of a prison in my neighborhood, justice is unable to {[}repress{]} the disorders} at the Organized Mountains. Letter of 6 May 1894, ANH-Justice, File 13024, p.~4.} While the journalists issued various calls for reform, none was forthcoming in these years.

Critique and satire also emerge in the newspapers' pages. They portray uneasy class divides but also considerable amusements amid sometimes biting political critiques. Editors and journalists at {\em L'Echo d'Haïti}, for example, conceptualized their audience as coming from the elite strata of Port-au-Prince. This orientation comes across not only through the language and medium (usually written French) but also in topics its writers covered, the issues they worried about, and the people they mocked. Through its columns, the paper communicated important markers of elite respectability and material comforts, such as access to water. In August 1894, a journalist groused that recently constructed hydraulic services---part of President Florvil Hyppolite's modernizing reforms---were failing to meet expectations. \quotation{Bathing in these extreme temperatures is hardly considered a luxury in the civilized countries that surround our \quote{Pearl of the Antilles,} but here it is!} the writer laments. \quotation{Certain quarters are so deprived of water that the satisfaction of taking a bath is available only to the rich.} Urban services, and the elite expectation of access, made hygiene and the body a metonym for class and nation. The author seems to take the challenges (and inequities) in stride, however, concluding sardonically, \quotation{{\em Vive le tafia pur}!}\footnote{For a larger discussion of class, urban planning, and the production of an equal access to water, see Claire Antone Payton, \quotation{City of Water Port-au-Prince, Inequality, and the Social Meaning of Rain,} {\em Journal of Urban History}, 23 February 2021, 1--22, \useURL[url2][https://doi.org/10.1177\%2F0096144221992030]\from[url2].}

The above vignettes offer a window on to the wide range of the social, political, economic, and environmental themes visible across the pages of the BHFIC's now digitized collection of nineteenth-century newspapers. Yet these windows might have been forever closed to today's students and researchers if not for the attention and labor of Haitian collectors and librarians who protected the documents from destruction against formidable odds. In the two centuries since Haitians fought for and won their liberation, profound international and domestic challenges to the stability of the state have complicated archival efforts. Civil wars and repeated fires, both accidental and politically motivated, have created a major hole in Haitian documentary heritage. These conflagrations, too, emerge in the newspaper collections.\footnote{\quotation{Incendie à Port-au-Prince,} {\em La Justice}, 18 February 1890.} The capital was particularly vulnerable to such turmoil because it was the seat of political power and thus a site of conquest for competing candidates. The streets surrounding the library have served as the heart of many battles over the presidency, Congress, and foreign interference, as newspapers in the BHFIC's collection Illuminate.

These newspapers were first collected and safeguarded by private hands. The history of these collections has been recorded and retold by the directing clergy of the BHFIC; we reproduce parts of the history here.\footnote{Brother Ernest Even, who directed the BHFIC from the 1980s until after the 2010 earthquake, communicated the history of the library's collections to Lewis Ampidu Clorméus in 2010; the BHFIC maintains internal documents recording these contributions.} Prior to the foundation of the Bibliothèque Nationale d'Haïti (1940), private collectors---including Edmond Mangonès, François-Denys Légitime, and Horace Pauléus Sannon---preserved important collections of books, newspapers, and ephemera. Students, intellectuals, and the general public wishing to engage with the various aspects of these decades of Haiti's social, cultural, political, and economic life faced significant challenges. In this context, at Saint-Louis de Gonzague, Brother Ernest-Louis Dion gathered some of the first collections that would become the holdings of the BHFIC. Initially, he and other missionaries acquired many titles in France on the banks of the Seine. Quite quickly, the missionaries began to collect collections of rare newspapers, archival items, and old books. French scholarly and journalistic production on Haiti was quite important in the nineteenth century.\footnote{Watson R. Denis, \quotation{Origins and Manifestations of Haitian Francophilia: Nationalism and Foreign Policy in Haiti (1980--1915),} {\em Secuencia}, no. 67 (January--April 2007): 93--139.}

The BHFIC's collections have helped students of the Institution Saint-Louis de Gonzague better understand the history and geography of Haiti. The library was part of an expanding intellectual network and reading public throughout the country. In the crucible of pressure of the US military occupation, intellectuals founded the Société d'Histoire et de Géographie in 1923. The organization brought together collectors and intellectuals, including history and geography teachers. In this heady ferment, Frères de l'Instruction Chrétienne decided to open their collections to a wider public. Two bibliophiles, Ulrick Duvivier and Antoine Michel, lent their invaluable support by providing the library with cases of books and sizable newspaper collections. In 1927, Michel proposed the creation of a more suitable room for public reading and the installation of metal cabinets to better preserve the holdings.\footnote{Clorméus, personal communication with Even, 2010.} Upon becoming principal director, Brother Hyppolite-Victor Géreux, former archivist of the congregation in Ploërmel, Brittany, initiated construction of this vision. All the while, he continued acquiring new holdings, drawing funds from the institution's budget to purchase Haitian publications. During his own travels to London and Paris, he avidly collected from booksellers.

Many of the newspapers digitized through our EAP grant, as well as other documents, benefited from the expertise, time, and care of Asrel Laforest {\em fils}, who, like his father, worked as a notary. His father was a signatory of many public acts at the beginning of the twentieth century and himself known to be a great collector. In 1930, Laforest offered a fine collection of newspapers from the nineteenth and early twentieth centuries to the BHFIC. Galvanized by this donation, the Brothers began collecting {\em Le Moniteur} and the country's main newspapers in earnest. When Brother Chrysostome Jouquand made the first inventory of the BHFIC's holdings in 1936, its collection boasted four thousand titles.\footnote{Clorméus, personal communication with Even, 2010.}

In 1958, under the leadership of Brother Lucien Jean Legendre, library scholars produced a comprehensive catalog of BHFIC holdings that spans 533 mimeographed pages.\footnote{Lucien-Jean Legendre, “{\em The Catalogue of the Haitian Library of the Brothers of Christian Instruction, Port-au-Prince, Haiti W.I.,“} master's thesis, Graduate School of Saint Michael's College, 1958.} In 1970, the Haitian state recognized the Bibliothèque Haïtienne des Frères de l'Instruction Chrétienne as {\em d'utilité publique}, a status conferring certain rights and support.\footnote{Article 2 of the law of 8 July 1921 regarding the recognition of public utility (la reconnaissance d'utilité publique) specifically recognizes private organizations that do work for the public good with a specific civic personality as dictated by the constitution and other laws (\quotation{une personnalité civile et tous les droits qui en découlent dans la mesure stricte où leur but déclaré le réclame et dans les limites fixées par la Constitution et les Lois.}). \quotation{Comment créer une ONG et une fondation au regard de la loi?,} {\em Le Nouvelliste}, 24 September 2015, https://lenouvelliste.com/article/150298/comment-creer-une-ong-et-une-fondation-au-regard-de-la-loi.} Since then, with financial support from the Association des Anciens {[}Alumni Association{]} de Saint-Louis de Gonzague and donations from individuals, publishers, and cultural institutions, the library's catalog has grown and now contains approximately twenty thousand titles as well as one of the most important collections of Haitian newspapers from the nineteenth and twentieth centuries.

In recent decades, many scholars and writers from the Caribbean, the United States, France, and Haiti have visited the BHFIC for research. Some researchers have donated specialized books; historian Gabriel Debien and Haitian writer Léon Laleau are two notable examples. In the early 1980s, Léon-François Hoffmann, professor at Princeton University, undertook the first efforts to preserve fragile documents through a Latin American Materials Project (LAMP) Initiative, funded by the Ford Foundation. The project created microfilm collection of 330 Haitian newspapers and journals selected specifically because they were titles \quotation{not held by other libraries in Haiti or elsewhere.}\footnote{Center for Research Libraries, \quotation{Guide to LAMP Collections,} accessed 18 June 2021, \useURL[url3][https://www.crl.edu/book/export/html/1215]\from[url3].} This impressive collection is held at the Center for Research Libraries.\footnote{Center for Research Libraries, \quotation{Haitian Periodicals in the Saint Louis de Gonzague Collection,} accessed 18 June 2021, \useURL[url4][https://www.crl.edu/sites/default/files/d6/attachments/pages/Haitian\%20periodicals.pdf][][https://www.crl.edu/sites/default/files/d6/attachments/pages/Haitian\letterpercent{}20periodicals.pdf]\from[url4].}

\subsection[title={The BHFIC and the Digital Turn},reference={the-bhfic-and-the-digital-turn}]

In the 2000s, advances in digital technologies opened up new avenues of collaboration around issues of archives and preservation. dLOC began coordinating its consortium of international partners in 2004; several participating institutions were located in Haiti, including the National Archives. This network proved critical in January 2010, when Port-au-Prince was devastated by a major earthquake. As always following disasters, the first responders were members of the local community. Library staffs acted quickly to recover books and collections from the debris. The Haitian International Council of Museums organized a task force, known as {\em SOS Patrimoine en Danger}, to coordinate and support endangered libraries and collections.

Thankfully, the BHFIC building was not compromised structurally by the earthquake. The holdings were damaged when shelves fell over and dumped papers and volumes onto the ground, but they were not crushed. The following month, Brooke Wooldridge, dLOC director, and Matthew Smith, professor at the University of the West Indies, Mona, went on a mission to Port-au-Prince to meet with local partners and discuss how dLOC could assist with efforts to protect and preserve Haitian collections while helping local institutions strengthen their institutional capacity. The initiative was called Preserving Haitian Patrimony. \quotation{Even before the earthquake, the needs of these institutions were great,} dLOC's newsletter observed. \quotation{Chronic underfunding of these institutions makes intervention in the aftermath of the earthquake even more necessary. At the same time, it is more difficult to separate pre- and post- earthquake needs. It is impossible to only focus on earthquake-related needs.}\footnote{Brooke Wooldridge, \quotation{Protecting Haitian Patrimony Initiative: Initial Assessment and Recommendations} (Florida International University, 26 February 2010), \useURL[url5][https://dloc.com/UF00098694/00001]\from[url5].}

This moment represented a further pivot toward digitization. dLOC and its Haitian partners decided that \quotation{intensive digitization training would best contribute to a significant long-term impact on . . . preservation and access to Haitian patrimony.}\footnote{Brooke Wooldridge, \quotation{Protecting Haitian Patrimony Initiative: Update} (Florida International University, June 2011), \useURL[url6][https://dloc.com/UF00098694/00010/1j]\from[url6].} In July 2011, dLOC brought five library technical staff from several Haitian institutions, including the BHFIC, to Florida International University for a ten-day workshop. In 2011--12, the BHFIC formally affiliated with the dLOC network. In subsequent years, dLOC brought materials to install a state-of-the-art digitization station in the basement of the library.

In 2016, a four-person team of Haitian and US scholars assembled to apply for funding from the British Library's Endangered Archive Programme to help further support digitization at the BHFIC. Our team consisted of three US historians of Haiti, Anne Eller, Claire Antone Payton, and Erin Zavitz, and a Haitian sociologist, Lewis Ampidu Clorméus. Each of us had conducted research at the library and recognized the unique value of its collections. We also had learned that certain holdings were in such delicate condition that they were protected from consultation. After learning about the EAP from British writer Paul Clammer on Twitter, we initiated the application to secure support for the BHFIC to digitize these fragile and deteriorating newspapers. Working across time zones, we developed our grant in a collective Google Docs draft, working independently on different sections and meeting regularly over conference calls to discuss progress.

In 2017, the EAP accepted our proposal, \quotation{Beyond the Revolution: Bibliothèque Haïtienne des Frères de l'Instruction Chrétienne Collections: Bringing Nineteenth-Century Haitian History to the World.} It was important that our project complement and expand earlier BHFIC-dLOC collaborations and dovetail with any digitization infrastructure already in place. We knew that success would depend on working closely with both the library's executive director, Marie-France Guillaume, and the director of dLOC, Miguel Asencio. Based on Asensio's recommendations, our grant proposed to buy a Canon EOS 6D, wired remote, a Tiffin color check, fluorescent light kit, light bulbs, battery backup, an iMac with additional RAM, an external hard drive, shock- and drop-resistant portable hard drives, memory cards, and a memory card reader. With these materials, the library would be outfitted with two state-of-the-art digitization stations. Our grant also funded the salaries of two technicians for over a year, which doubled the size of the staff dedicated to digitization. It also covered the cost of high-speed wireless internet for the library. These investments, especially the equipment and the funding to train and employ technicians, reflect the EAP's commitment to building up the institutional capacity of local communities.\footnote{Anthea Case, \quotation{From Dust to Digital: Ten Years of the Endangered Archives Programme,} in Kominko, {\em From Dust to Digital}, \useURL[url7][https://doi.org/10.11647/OBP.0052][][xlv, https://doi.org/10.11647/OBP.0052]\from[url7].}

\placefigure[here]{Miguel Asencio demonstration}{\externalfigure[issue06/valence_1_cp_dsc00793.jpg]}


\placefigure[here]{digitization workstation}{\externalfigure[issue06/valence_1_cp_dsc02548.jpg]}


After the funding was in place, the project launched slowly. One unforeseen obstacle was the destruction wrought across the region by Hurricane Maria. While Haiti was spared the brunt of that storm, many other Caribbean islands were not. dLOC sprang into action, mobilizing to help its partners get the resources and support they needed to recover. A planned BHFIC digitization training was delayed while dLOC attended to other needs. In November 2017, Miguel Asencio and Claire Antone Payton traveled to Port-au-Prince to transport equipment and host the workshop for technicians at the library and several other local cultural institutions. They worked with Guillaume and digitization specialist Wandred Pierre to set up the stations, install software, and review grant terms and conditions. Then the library team created an inventory of the nineteenth-century newspapers in the holdings and evaluated which could be digitized and which were too damaged.

By the time the grant ended in 2019, Pierre and his colleague Angerlo Mondésir had digitized 1,155 separate issues from twenty-seven different publications, amounting to nearly ten thousand pages of printed materials. The issues range from the 1840s to the turn of the twentieth century. While many of the newspapers were published in Port-au-Prince, a significant number of others, such as {\em Le Vœu Populaire}, were published in Cap-Haïtien and elsewhere, providing insight into Haiti's complex nineteenth-century political, cultural, and economic geographies. As a result of the labor of Pierre and Mondésir, Guillaume and Asencio, and our team, this remarkable collection is now freely available to researchers around the world on the dLOC and British Library websites.

\subsection[title={Material and Digital Futures},reference={material-and-digital-futures}]

Our project contributes to a larger wave of critical scholarship-activism that mobilizes the digital to center Haitian perspectives, collaborates on the preservation of critical Caribbean historical-cultural material, and challenges the exclusion of Caribbean knowledge production from North Atlantic scholarship. The goal of this work is to dismantle colonial silos of knowledge production and create more inclusive discourses and historical imaginaries that thicken the connections between Haitian intellectual production and writing about Haiti produced beyond its shores. As Chelsea Stieber, Laura Wrubel, and Watson Denis have argued, \quotation{It is precisely because Caribbean sites of knowledge production exist outside of---and in critique of---North Atlantic infrastructures of information that it becomes acceptable, common even, for scholarship produced outside Haiti to leave them uncited and unread.}\footnote{Stieber, Wrubel, and Denis, \quotation{Cross-Boundary Digital Collaboration.}} Stieber's effort to build a digital index of Haiti's leading social science journal, the {\em Revue de la Société Haïtienne d'Histoire, de Géographie et de Géologie} ({\em RSHHGG}){\em ,} makes a significant collection of Haitian scholarship visible and accessible to scholars in and outside of Haiti. Similarly, Marlene Daut's painstaking work collecting digital copies and transcribing issues of the two official newspapers from the administration of Henry Christophe, the {\em Gazette Officielle} and {\em Gazette Royale d'Hayti,} facilitates new and complex engagements of the history of Haitian sovereignty.\footnote{Daut, \quotation{Haiti @ the Digital Crossroads.}} Other recent projects include a grant from the Omohundro Institute of Early American History and Culture to fund a project led by Jennifer Palmer, Julia Gaffield, and Patrick Tardieu that will digitize and upload some collections from the Bibliothèque Haïtienne de Spiritains to an affiliate site of Gallica.fr.\footnote{Prior to its reopening in 2019, this library was known as Bibliothèque Haïtienne des Pères de Saint-Esprit. Due to the devastation of the January 2010 earthquake, the library had been closed to full public access for the preceding nine years.}

This critical approach extends to documents from the twentieth and twenty-first centuries. Under the brilliant stewardship of Duke University archivist Laura Wagner, the records and recordings of Jean Dominique's Radio Haiti are now available online.\footnote{Laura Wagner, \quotation{{\em Nou toujou la!} The Digital (After-)Life of Radio Haïti-Inter,} {\em archipelagos} 2 (September 2017), \useURL[url8][http://smallaxe.net/sxarchipelagos/issue02/nou-toujou-la.html]\from[url8].} This project is particularly important because, unlike many newspapers, Radio Haiti was an integrated part of Haitian popular culture. As Wagner and historian Jennifer Garçon have argued, radio in Haiti is more democratic than print culture because it is relatively inexpensive and stations like Radio Haiti broadcast in Haitian Krèyol.\footnote{Jennifer Garçon, \quotation{Haiti's Resistant Press in the Age of Jean-Claude Duvalier, 1971--1986} (PhD diss., University of Miami, 2018).}

Each of these projects, in their own way, has grappled with the reality that efforts to democratize Haitian historical sources are constrained by the same unequal power dynamics that they seek to challenge. Daut observes that these endeavors represent \quotation{a crossroads, where institutions, scholars, the public, and the spirits of Haiti must meet on uneven, and sometimes conflictual, terrain.}\footnote{Daut, \quotation{Haiti @ the Digital Crossroads.}} For example, Stieber's work resulted in a digital index of titles and topics in the {\em RSHHGG} rather than direct access to digitized articles because the society's leadership opposed \quotation{any digitization {[}given{]} valid concerns that digitization alone would not ensure North Atlantic scholars' engagement with Haiti and its institutions.}\footnote{Stieber, Wrubel, and Denis, \quotation{Cross-Boundary Digital Collaboration.}} In addition to breaking linguistic barriers of access by advocating for funding of a completely trilingual Krèyol-French-English Radio Haiti, Wagner works to overcome disparities in internet access by distributing the digitized recordings materials on thumb drives to libraries, radio stations, and cultural institutions throughout Haiti.

Our own project was no different. Evaluating the importance and the limitations of the EAP grant requires reflection, honesty, and dialogue. We are responding to an invitation to reflect on our subject position in order to mediate \quotation{epistemic harms in the open projects we promote, facilitate, and design.}\footnote{Albornoz, Okune, and Chan, \quotation{Can Open Scholarly Practices Redress Epistemic Injustice?,} 72.} Despite our best efforts, there were moments when we were confronted with structural inequalities that transcended yet channeled through us. For instance, the choice to focus on digitizing nineteenth-century papers rather than other collections was delimited by our grant application. And once digitization began, our collaborators at the library, keenly aware of long histories of theft and dispossession, expressed skepticism about the project's predatory potential. They were perhaps more savvy about the power dynamics underpinning open access digitization than we initially were. As the project was getting off the ground, they paused the work to ask whether the terms of the grant stipulated that {\em everything} the library digitized would subsequently belong to the British Library. After clarifying that only the nineteenth-century newspapers we had agreed upon while drafting the EAP proposal would be shared, the project continued smoothly.

Yet this exchange highlighted how the social locations within our collaboration created divergent perspectives on the stakes of open access. Our different positions put us, the EAP, dLOC, and the BHFIC on \quotation{uneven, sometimes even conflictual, terrain} with respect to sharing materials. While library leadership appreciated the infusion of much-needed resources that our grant represented, they questioned the concomitant obligation to \quotation{open up} access to their holdings. Even as we tried to design our project to maximize its potential to strengthen the library's institutional capacity, it nevertheless invoked colonial dynamics of extraction. This negotiation underscored how the fate of the newspapers should not be separated from the fate of the library that has maintained them for all these years.

This larger tenor of uneven exchange can sometimes be rendered invisible by enthusiastic discourse originating in powerful institutions in the global North around digitization's felicitous potential for democratization (and preservation, of a sort). Based on our experience, by permitting its funds to be used for employee salaries---a form of direct institutional support---rather than limiting them to in-kind donations, the EAP is more responsive to this reality than many other international funding organizations.

But direct resource transfers do not ultimately rectify the power differentials that separate the BHFIC from institutions like the British Library or the University of Florida. Any prestige associated with the holdings passes from the Haitian library to dLOC and the British Library. Any revenue directly or indirectly associated with internet traffic is conferred to the digital repositories rather than the original institution. While the documents may gain greater prominence and visibility, the library itself is rendered invisible or marginalized---a detail in the metadata of the online files.

The digital life of the newspapers also changes their relationship to local and larger audiences. In the hands of dLOC and the British Library, the primary audience for the papers is now \quotation{global,} whereas the original collectors who donated them did so with the hope that they would help educate generations of Haitian students about their country's history. Digitization creates new research methodologies that sometimes fully supplant in-person scholarly exchanges, sojourns, conversations, and projects. As Lara Putnam observes, digital research that is scaffolded by simple term searching, \quotation{side-glancing,} and \quotation{drive-by transnationalism,} naive to local historical or current debates or relevance, clearly represents a pitfall of digitization, even if the casualization of the academy and productivity pressures also present challenges to deep engagement.\footnote{Putnam, \quotation{The Transnational and the Text-Searchable,} 382, 397.}

The issue of hosting these materials online opens up important questions regarding sovereignty and control. As Stieber, Wrubel, and Denis observe, the infrastructures of information maintenance are themselves \quotation{often invisible and institutionally siloed.}\footnote{Stieber, Wrubel, and Denis, \quotation{Cross-Boundary Digital Collaboration.}} Transnational scholarly collaboration and digital preservation work needs to firmly address inequalities within the material world of knowledge production. Projects should honor the local institutions. We have attempted to model this here by situating our EAP grant as a recent chapter in the long history of the BHFIC and the collection. But is there any potential for this recognition to go beyond the symbolic and arrive at the material? Put otherwise, are additional local-global digital projects possible that will provide concrete and economic support to sustain these institutions into the future? Can we support online access models that are equitable rather than equal for the institutions involved?

There are some imperfect models for how to approach these questions. Our project represents one possibility; for it to become sustainable, there would have to be a succession of grants from funders like the EAP offering opportunities for direct institutional support. Another possibility is investment in the material restoration of historical documents. Unfortunately, physical restoration is prohibitively expensive. According to the estimates of Marie-France Guillaume and Miguel Asencio, the cost of repairing the newspapers to the point that they could be circulated would cost approximately \$10,000 per title. A third model is that of the {\em RSHHGG} indexing project. In that case, the titles and topics of an important intellectual resource have been shared online, but the content itself remains offline. Stieber and the {\em RSHHGG} came up with this approach to mobilize already existing university infrastructures (such as Interlibrary Loan) to provide digitized access to international scholars. This does not directly translate into revenue for Haiti's leading intellectual society, but it creates transnational exchange without foreclosing the possibility of a paywall model that turns interest in Haitian intellectual production into material support. Describing the {\em RSHHGG} Lab, Daut writes, \quotation{Thus we might say that {[}it{]} has provided a model for a digital project that has the capacity to ensure that Haitian scholars are neither silenced nor exploited.}\footnote{Daut, \quotation{Haiti @ the Digital Crossroads.}}

After implementing our EAP open access digitization grant, we have come to support this fourth model: paywalls. This approach deviates from the open access ethos that underpins many digitization projects, including our own. Open access emerged as a powerful tool for challenging the control of publishing corporations such as Elsevier, who have created monopolies in the knowledge economy and used exclusion and control to rake in enormous profits. The academic publishing giant had a reported 2019 revenue of 2.64 billion pounds sterling.\footnote{RELX, {\em Annual Report and Financial Statements}, 2019, \useURL[url9][https://www.relx.com/investors/annual-reports/2019]\from[url9].} Even the best-funded institutions are burdened by the huge subscription fees charged by such organizations to maintain access to important scholarship. Confronting this system is laudable and necessary.

The BHFIC is not Elsevier. In 2014, its operating budget was about \$20,000.\footnote{Marie-France Guillaume, \quotation{Brothers of Christian Instruction Library, Haiti: Presentation Slides for dLOC Partner Meeting at ACURIL 2014} (Digital Library of the Caribbean, 2014), \useURL[url10][https://www.dloc.com/AA00023816/00001]\from[url10].} When the EAP grant concluded, it couldn't afford to maintain two digitization specialists. One of them was let go. We need to envision and support methods of international collaboration that facilitate the material transfer of resources from wealthy institutions and learning communities to their underresourced counterparts rather than papering over the very substantial material differences between them in the name of a \quotation{global} audience. As Denisse Albornoz, Angela Okune, and Leslie Chan have warned, \quotation{When openness is simply grafted atop existing technology and power structures, the powerful are further empowered, and the dominant epistemologies are further reproduced.}\footnote{Albornoz, Okune, and Chan, \quotation{Can Open Scholarly Practices Redress Epistemic Injustice?,} 73.}

In 2020, Marie-France Guillaume and the BHFIC leadership launched a plan for the construction of an independent digital repository where the library can share digitized materials behind a paywall. The current plan proposes a two-tiered membership access: one for students under the age of eighteen, another for adults and professionals. We suggest that the library further distinguish between professional researchers in Haiti and abroad, with the highest fees falling on fully employed researchers in well-resourced institutions in the global North. A tiered fee system is not a new idea; the Haitian Studies Association has already implemented one with regard to conference and membership fees. It is ironic that the growing recognition of Haiti's importance has occurred simultaneously with technological advancements that make it possible for foreigners to study Haiti without actually traveling there and interacting with the institutions like the BHFIC, which, for a century, has been the steward of some of the country's most important collections of historical archival heritage.

These questions resonate with Daut's observation, drawing on Édouard Glissant, that those who engage in activist-archivist work \quotation{must remain unafraid of opacity, those moments when the archive refuses to speak to us, to make sense to us, or even to let us in.}\footnote{Daut, \quotation{Haiti @ the Digital Crossroads.}} The skepticism we encountered while implementing the EAP grant, and the BHFIC's dream of an independent online repository, could be considered forms of opacity, an intrinsic critique of the potentially exploitative value of transparency represented in open access. Daut continues, \quotation{Most importantly, we must respect the fact that we do not necessarily have a right to this knowledge.} Supporting the creation of a digital repository protected by a paywall is one way to concretize Daut's imperative and respond to the stark inequalities that permeate efforts to democratize knowledge.

\subsection[title={Conclusion},reference={conclusion}]

Questions about sovereignty of knowledge and access can sometimes seem far removed from the day-to-day life of the library. In 2021, its staff and leadership are focused on the logistical challenges of tending to their intellectual community amid the COVID pandemic and related political violence that have made downtown Port-au-Prince dangerous for many. Nonetheless, on a regular day, the library receives about a dozen visitors: students from the local high schools and universities as well as professors, writers, and researchers. At present, the low reliability of Haiti's internet infrastructure and chronic electricity shortages mean that local researchers are best served by visiting the library itself. There is also a larger congregation of BHFIC faithful who frequent the building for workshops, talks, and events. Its librarians are adamant about maintaining the institution's public-facing mission of cultural and historical enrichment. Even their motivation to build an independent digital repository stems from the desire to make sure that the students who need the collections for their studies can access it even when visiting in person is not possible.

Despite the different approaches, the library's work of cultivating Port-au-Prince's intellectual networks and the work of implementing our EAP digitization grant share the same foundation: the BHFIC's unique collection of cultural and political heritage is critical to deepening understanding of Haiti's extraordinary history. It is essential to publicize the existence of these digitized newspapers and their availability online. This strategy will generate new audiences and even new avenues for research. Their content is essential for a wider understanding of the social, cultural, political, and economic life of the country during the nineteenth century and beyond. Furthermore, the library's future engagement with digitization might serve as a model for other collectors and institutions to make their archival holdings more accessible by sharing them online on their own terms. The BHFIC's dream of an online subscription site, Guillaume hopes, will call attention to the thousands of pages of Haitian history that it holds, and generate material support for years to come.

\thinrule

\page
\subsection{Claire Antone Payton}

Claire Antone Payton is a historian of the Caribbean whose work analyzes Haiti's twentieth-century urban history. Her research has been supported by a Fulbright-Hays DDRA, the Woodson Institute for African American and African Studies, and Kluge Center at the Library of Congress. She is also the creator of the Haiti Memory Project, an oral history initiative that documented first person testimonies of the deadly Port-au-Prince earthquake and life in its aftermath. Currently, she works at the Memory Project at UVA's Democracy Initiative, where she promotes democratic values and racial equity through public-facing programs that shape inclusive historical memory.

\subsection{Anne Eller}

Anne Eller is an Associate Professor of History and an Affiliate of African American Studies and Spanish and Portuguese at Yale University. Her first book, We Dream Together (Duke, 2016), crafts a deep history of emancipation on Dominican soil as well as the resistance movement generated by the Spanish reoccupation of the Dominican (1861-5). The Editorial Universitaria Bonó recently published the Spanish translation, Soñemos juntos, in Fall 2021. Her current research considers the relationship of working people with the state in the Caribbean in the 1890s, including in an article in March 2021 Small Axe.

\subsection{Lewis Ampidu Clorméus}

Lewis A. Clorméus holds a doctorate in Sociology and is professor at the Université d'État d'Haïti. He also serves as the Secretary for the Haitian Society of History, Geography, and Geology (Société Haïtienne d'Histoire, de Géographie et de Géologie). His research interests particularly focus on the management of cultural heritage, the intellectual history of Haiti, and ties between religion and the state. Among his notable publications are Le vodou haïtien (Riveneuve, 2015), Duverneau Trouillot (CIDIHCA, 2016), and Le vodou, le prêtre et l'ethnologue (2020).

\stopchapter
\stoptext