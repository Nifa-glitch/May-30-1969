\setvariables[article][shortauthor={Dize}, date={May 2022}, issue={6}, DOI={Upcoming}]

\setupinteraction[title={Following in *Île en île*'s Footsteps},author={Nathan H. Dize}, date={May 2022}, subtitle={Following *Île en île*}, state=start, color=black, style=\tf]
\environment env_journal


\starttext


\startchapter[title={Following in {\em Île en île}'s Footsteps}
, marking={Following {\em Île en île}}
, bookmark={Following in *Île en île*'s Footsteps}]


\startlines
{\bf
Nathan H. Dize
}
\stoplines


\subsection[title={Beginnings},reference={beginnings}]

I remember when I was first asked to contribute to the online archive {\em Île en île}. In the summer of 2017, I was taking part in the Haitian Creole Summer Institute at Florida International University when I received an email from Thomas C. Spear. He asked me if I would be interested in composing the author page for a young Haitian novelist named Néhémy Pierre-Dahomey, whose novel {\em Rapatriés} had just been released in France in January. I later found out that a friend, knowing that I was reviewing {\em Rapatriés} (Repatriated) for the Journal of Haitian Studies, had given Thomas my contact information.\footnote{Nathan H. Dize, review of {\em Avant que les ombres s'effacent}, by Louis-Philippe Dalembert, and {\em Rapatriés}, by Néhémy Pierre-Dahomey, {\em Journal of Haitian Studies} 24, no. 1 (2018): 169--72.} I gladly accepted Thomas's offer, thinking at the time that it would be delightful to be in contact with Néhémy. This was new to me. I had never been in touch with an author before; I always found it daunting for some reason or another, like my job as a critic was to hide behind the books themselves. Working with {\em Île en île} would be my chance to grow, to build a bridge, and to ask Néhémy questions about his life, his burgeoning career, and his powerful debut. This was the {\em Île en île} process, after all---to ask authors about their lives and to compose a living biography, for living authors, at least.

At the outset, I could not see how a webpage would lead to a friendship that would reorient my scholarship. As I talked with Néhémy over email, WhatsApp, and the phone, I created the brief sketch of his life and aesthetics now preserved on {\em Île en île}. Our conversations, the discussions we had about {\em Rapatriés}, and the work we shared with one another would alter my approach to literary studies and translation. Though I only ever produced one page for {\em Île en île}, I hope that recounting my experience working with the site and Néhémy Pierre-Dahomey will reveal how impactful human connection can be for the study of literature and how essential it is to our digital humanity.

\subsection[title={Process},reference={process}]

The process of creating a page on {\em Île en île} is, by necessity, subjective. Archived in its pages are the lives of living authors, deceased authors, authors who have changed careers or who have lived multiple professional and aesthetic lives as well as established and emerging authors. In my case, I was asked to create a page for an emerging author with only a few publications to his name, including the short story \quotation{Je n'ai pas tué Amandine} (\quotation{I Didn't Kill Amandine}) and his novel {\em Rapatriés}. This meant that not only was there a dearth of critical sources available on Néhémy Pierre-Dahomey, but there was also little biographical information about him on the web from which I could compose his author profile. I knew I would have to reach out to Néhémy directly to learn the information necessary to incorporate his literary career into the robust series of pages {\em Île en île} had already created for so many archipelagic authors.

Île-en-Île has always relied on personal connection to create and curate the presence of archipelagic literature on the web. Users can experience the impact of the personal directly in their series of videos called \quotation{5 Questions pour {\em Île en île}} as well as indirectly in the individual author pages. Even though I did not produce a \quotation{5 Questions} segment with Néhémy, my conversations with him were invaluable for creating his page. To be sure, I could rely on my training as a researcher to scour the web for bibliographic information, but only by speaking with Néhémy directly could I learn about his literary influences, learn about his personal biography, and ask him about the poetic intentions and ambitions of his literature.

The {\em Île en île} process taught me a new way of working; it taught me that we have much to learn from one another by just sitting down with one another and talking, telling one another our stories. {\em Île en île} reminds us that talking, wherever it may occur---in a New York apartment with Marie-Cécile Corvington-Charlier, in the Port-au-Prince gardens of Emmelie Prophète or Kettly Mars, or over an internet connection between a Nashville café and a Parisian apartment---is an essential element of understanding literature and the impact that it has on our lives.\footnote{\quotation{Marie Vieux Chauvet, the Person and the Writer; Témoignages de Lilian Vieux Corvington et de Marie-Cécile Corvington-Charlier} (video), {\em Île en île}, updated 25 April 2021, \useURL[url1][http://ile-en-ile.org/marie-vieux-chauvet-temoignages/]\from[url1]; Emmelie Prophète, \quotation{5 Questions pour Île en Île,} {\em Île en île}, updated 26 October 2020, \useURL[url2][http://ile-en-ile.org/emmelie-prophete-5-questions-pour-ile-en-ile/]\from[url2]; Kettly Mars, \quotation{5 Questions pour Île en Île,} {\em Île en île}, updated 26 October 2020, \hyphenatedurl{http://ile-en-ile.org/kettly-p-mars-5-questions-pour-ile-en-ile/.}}

\subsection[title={Continuations},reference={continuations}]

Just after I presented Thomas C. Spear with the content that would become Néhémy Pierre-Dahomey's page on Île-en--Île, I asked Néhémy if he would allow me to translate his short story \quotation{Je n'ai pas tué Amandine.} I thought that translating the short story would be something of a second act in contributing to Néhémy's web presence, granting more readers access to his evocative prose. I had already translated a few poems by Haitian authors and was on my way to completing a sample translation of {\em The Immortals} by Makenzy Orcel, one of Néhémy's literary peers, so I felt motivated for the task.\footnote{The first poems I translated were Charles Moravia's \quotation{La vision de Président Wilson} (\quotation{President Wilson's Vision}) and Evelyne Trouillot's \quotation{Secousses} (\quotation{Tremors}). See Nathan H. Dize, \quotation{Translating Global Citizenship: Haiti, Charles Moravia, and Woodrow Wilson,} {\em sx salon} 26 (October 2017), \hyphenatedurl{http://smallaxe.net/sxsalon/discussions/translating-global-citizenship;} Evelyne Trouillot, \quotation{Secousses/Tremors,} {\em Meridians} 18, no. 2 (2019): 480--81.} As we discussed the possibility of translation, however, together we decided that it was more beneficial to Néhémy if I worked on his novel {\em Rapatriés}, instead of his short story.

There are many reasons why translation can be beneficial to a debut novelist; it can build the confidence of their publisher, increase the author's visibility abroad, and, especially with English translations, spur the interest of publishers in other languages in purchasing the rights to commission their own translations. In Néhémy's case, the case for translation was identical to the reasons for creating his {\em Île en île} page, to ensure that his writing would be available and accessible to readers, scholars, and students of Haitian literature. {\em Île en île} also presents an exhaustive list of works in translation by the authors in its archive, making it a crucial bibliography for translators and publishers assessing the potential market for a book. Though our translation of {\em Rapatriés} has yet to be acquired by a press for English publication (it has been translated into German as {\em Die Zurück gekehrten} by Lena Müller for Edition Nautilus), we continue to collaborate, building on the working relationship and friendship we kindled while compiling Néhémy's {\em Île en île} page.

\subsection[title={Futures},reference={futures}]

In November 2020, COVID-19 turned one and the world had been in various stages of lockdown for going on eleven months. Out of this separation and isolation came the desire to create a forum for Caribbean authors to present their work, especially since the ongoing pandemic precluded in-person signings, book releases, and readings. Along with my colleagues, I helped launch Kwazman Vwa: New Paths in Caribbean Literature, where we aimed to host digital book salons with Caribbean authors about books published or to be published during the pandemic.\footnote{Kwazman Vwa, home page, https://kwazmanvwa.com/, accessed 22 January 2022. Kwazman Vwa is an ongoing project and archive of conversations about Caribbean literature and culture. The most recent episodes can be found on the \quotation{previous episodes.} The group's founding members are Jennifer Boum Make, Nathan H. Dize, Corine Labridy, Erika Serrato, Jocelyn Sutton Franklin, Lucy Swanson, and Charly Verstraet.} The first author to join our program was Néhémy Pierre-Dahomey in February 2021 to discuss {\em Rapatriés} as well as his second novel, {\em Combats}, which was released in March 2021.\footnote{\quotation{Interview with Nehemy Pierre-Dahomey} (in French), Kwazman Vwa, accessed 22 January 2022, https://kwazmanvwa.com/featured-guests/.}

Initially, Kwazman Vwa (Crossing Paths) was conceived as a platform for emerging Haitian writers. Yet, as soon as we solidified our group of collaborators, we realized that our ambitions were much larger than just Haitian literature. We hoped to open a path for any Caribbean writer to discuss their work in a public, online forum held by video conference and to transcribe and translate the interviews to broaden the audience of each writer.

As I think back on it now, it is no mistake that Néhémy was the first author we invited to participate in Kwazman Vwa, nor was it by accident that we intended to transcribe, translate, and archive these conversations for future readers, listeners, and viewers.\footnote{Our conversation with Néhémy Pierre-Dahomey and our second event, featuring Jessica Oublié (author of {\em Tropiques toxiques}), are available online at https://kwazmanvwa.com/featured-guests/.} After all, thanks to {\em Île en île}, we were following a model with which we were familiar. We have dreams and aspirations of interviewing authors in their Caribbean gardens and yards. We yearn to hear the sounds of the noisy street, the honking of tap taps, and the voices of the {\em ti machann} selling their produce as we discuss Caribbean aesthetics. We wait for the day when we'll be able to return to book events and salons the world over. But until then, we'll be online, following the path that {\em Île en île} traced for us.

\thinrule

\page
\subsection{Nathan H. Dize}

Nathan H. Dize is Visiting Assistant Professor of French at Oberlin College. His work is situated at the crossroads of literary and intellectual history, cultural studies, translation studies, and the digital humanities. Nathan is particularly interested in how literature enables Haitians to practice intimate and collective rites of mourning across generations and beyond national borders. He is the translator of three Haitian novels: {\em The Immortals by Makenzy Orcel} (SUNY Press, 2020), {\em I Am Alive} by Kettly Mars (UVA Press, 2022), and {\em Antoine des Gommiers} by Lyonel Trouillot (Schaffner Press, 2023). Nathan has written for publications such as {\em archipelagos, Caribbean Quarterly, Francosphères, the Journal of Haitian Studies, sx salon,} and {\em Transition}; he also serves on the editorial board of the online magazine, ReadinginTranslation.com.

\stopchapter
\stoptext