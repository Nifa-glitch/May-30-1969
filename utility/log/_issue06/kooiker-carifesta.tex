\setvariables[article][shortauthor={Kooiker}, date={May 2022}, issue={6}, DOI={https://doi.org/10.7916/archipelagos-rw7y-p710}]

\setupinteraction[title={Edward Kamau Brathwaite at Carifesta '72: The Occasion for Caribbean Criticism},author={René Johannes Kooiker}, date={May 2022}, subtitle={Archiving Carifesta}, state=start, color=black, style=\tf]
\environment env_journal


\starttext


\startchapter[title={Edward Kamau Brathwaite at Carifesta '72: The Occasion for Caribbean Criticism}
, marking={Archiving Carifesta}
, bookmark={Edward Kamau Brathwaite at Carifesta '72: The Occasion for Caribbean Criticism}]


\startlines
{\bf
René Johannes Kooiker
}
\stoplines


{\startnarrower\it Carifesta took place for the first time in 1972 with the sponsorship of Forbes Burnham's People's National Congress in newly independent Guyana, and the festival recurs around the Caribbean to this day. Relying on print resources and digitized archives from the Digital Library of the Caribbean (dLOC), this essay recovers Edward Kamau Brathwaite's participation in the inaugural edition, which he praised as the \quotation{first ever meeting of the Caribbean.} In addition to reading from and lecturing on his own poems, he published a series of eight articles covering the events in the Barbados daily the {\em Advocate-News} in an attempt to involve readers in the ephemeral Pan-Caribbean public sphere the festival created. On the one hand, the Carifesta pieces offer a new perspective on Brathwaite's essays from the 1970s, in which he developed his notions of creolization and Caribbean aesthetics; on the other, Brathwaite's reflections posit Carifesta as epitomizing both the great promise and the disappointments of the Caribbean 1970s. The essay recommends an approach to the intellectual history of Caribbean thought as it was lived, performed, and institutionalized in public at events, congresses, and festivals such as Carifesta. In closing, the essay offers reflections on the Carifesta holdings in dLOC.

 \stopnarrower}

\blank[2*line]
\blackrule[width=\textwidth,height=.01pt]
\blank[2*line]

\quotation{Carifesta was Emancipation Day come true; collective Declaration of Independence; first ever meeting of the Caribbean.}\footnote{Edward Kamau Brathwaite, \quotation{Carifesta '72: Festival of Creative Arts of the Caribbean,} {\em Advocate-News}, 16 October 1972. For most citations of Brathwaite's work, I used the bibliography by Kelly Baker Josephs and Teanu Reid, \quotation{The Kamau Brathwaite Bibliography: A Collaboration in Progress,} {\em sx salon}, no. 27 (February 2018), https://caribbean.commons.gc.cuny.edu/wp-content/blogs.dir/441/files/2018/03/kb-bibliography.pdf.} So opened Edward Kamau Brathwaite's first article about Carifesta for the leading Barbadian daily, the {\em Advocate-News}. The first Caribbean Festival of the Arts took place in Georgetown, Guyana, from 25 August to 15 September 1972. This essay considers the significance of these articles for Brathwaite's inaugural work in what we now call Caribbean cultural studies.\footnote{Aaron Kamugisha, \quotation{On the Idea of a Caribbean Cultural Studies,} {\em Small Axe}, no. 42 (July 2013): 43--57.} Although this aspect of his thought is better known from other writings of the 1970s, such as \quotation{The Love Axe} and {\em Contradictory Omens}, these owe much to his earlier reflections on Carifesta. Yet why did Brathwaite compare Carifesta to the watersheds of independence and emancipation, despite the festival's overtones of touristic indulgence and statist spectacle? This essay positions Carifesta as part of what Honor Ford-Smith has called \quotation{the radical opening of the 1970s,} a time when the Caribbean world signified to anticolonial intellectuals like Brathwaite the promise of joining their activism to critical scholarship and thereby strengthening the region's cultural and socioeconomic autonomy.\footnote{Honor Ford-Smith, \quotation{The Body and Performance in 1970s Jamaica: Toward a Decolonial Cultural Method,} {\em Small Axe}, no. 58 (March 2019): 167.} Besides giving an account of Brathwaite's thought, this article also begins to tell the story of Carifesta '72 itself in the context of third worldist, Pan-Africanist, and Tricontinental congresses and cultural events. This is essential since, as Aaron Kamugisha rightly observes, \quotation{no extended discussion} of Carifesta exists in \quotation{Caribbean cultural history.}\footnote{Aaron Kamugisha, \quotation{The Promise of Caribbean Intellectual History,} {\em Small Axe}, no. 64 (March 2021): 49.} To fill this lack, I rely on newspaper accounts, photographs, meeting notes, printed ephemera, and other materials, most of which can be found in the Digital Library of the Caribbean (dLOC).

Visualize, for a moment, the sheer spectacle of the opening ceremony of this festival, which cost an estimated 2.5 million Guyanese dollars (GY\$).\footnote{Suzanne Burke, \quotation{The Evolution of the Cultural Policy Regime in the Anglophone Caribbean,} {\em International Journal of Cultural Policy} 13, no. 2 (May 2007): 178.} Forbes Burnham's new government had drawn on public and private sources to transform the fifty-seven acres of the National Park, the largest open space in Georgetown, into a performance venue with bleachers, a stage, and a twenty-foot-high Aztec temple in the backdrop.\footnote{A promotional pamphlet for the festival gives details on the funds for the construction projects: the Guyanese \quotation{popular businessman and sports promoter Bernard \quote{Bunny} Fernandes} was \quotation{Co-ordinating Manager} for the festival and carried \quotation{overall responsibility for the transportation and accommodation of visiting CARIFESTA artistes {[}{\em sic}{]} and for all developmental and other infrastructural work at Festival City and other centres used during CARIFESTA.} He had taken over the John Fernandes Limited shipping company---one of the few large businesses that had avoided nationalization in 1966---from his father. See {\em Carifesta} {\em '72: August 25--September 15, Guyana, South America} (pamphlet), 6, accessed 29 January 2022, \hyphenatedurl{http://ufdc.ufl.edu/UF00099655/00001.}} Around thirty thousand people gathered to watch representatives from the twenty-five participating nations and territories parade by the grand stage. To the sounds of the marching band of the Guyanese police, each of the twenty-five flags was raised and from behind the stage surged an enormous sculpture of a hand reaching for the sky, the festival symbol that graced every promotional newsletter, poster, brochure, letterhead, street sign, souvenir, flag, banner, and press release (figure 1).\footnote{See the photographs and eyewitness account in Nati González Freire, \quotation{Festival del Caribe, 1972,} {\em Bohemia}, 29 September 1972, \hyphenatedurl{http://ufdc.ufl.edu/UF00029010/00527.} For the number of participants, see the United Press International cablegram ({\em síntesis cablegráfica}) from 28 August on page 25 in the Brill primary sources collection on Carifesta from Cuba. For the plan of the opening ceremony, see as well the Carifesta newsletter in this collection, {\em Carifesta} {\em '72 Will Be Heartbeat of the Caribbean}, in {\em Cuban Culture and Cultural Relations, 1959--}, part 1, {\em Casa y cultura}, Casa de las Américas Library (Leiden, the Netherlands: Brill, 2017), https://primarysources.brillonline.com/browse/cuban-culture-and-cultural-relations.} Around this sign of creative capability the people of the Caribbean congregated. An eyewitness account in the Cuban magazine {\em Bohemia} includes a photo taken at the end of the opening ceremony, when \quotation{all participating delegations spontaneously expressed their joy about the great party of the independent and decolonized Caribbean,} with some revelers pictured holding up their country placards, reading \quotation{Cuba} and \quotation{Venezuela} (figure 2).\footnote{\quotation{Improvisadamente todas las delegaciones participantes expresaban al final del espectáculo de apertura en el National Park su júbilo por la gran fiesta del Caribe independiente y descolonizado}; González Freire, \quotation{Festival del Caribe, 1972}; my translation.}

\placefigure[here]{opening ceremony Carifesta '72}{\externalfigure[issue06/valence_1_rk_bohemia-opening1.jpg]}


\placefigure[here]{end of opening ceremony Carifesta '72}{\externalfigure[issue06/valence_1_rk_bohemia-opening2.jpg]}


For three weeks, the sounds of Pan-Caribbean music, dance, and theater rang out from the National Park and the National Culture Centre, a covered venue, constructed for the festival at a cost of GY\$750,000, that could accommodate over two thousand people.\footnote{For the budget, see {\em Carifesta} {\em '72 Will Be Heartbeat of the Caribbean}, January 1972, in {\em Cuban Culture and Cultural Relations}.} Smaller shows took place at St.~Rose's High School, Bishop's High School, Queen's College, and other locations.\footnote{González Freire, \quotation{Festival del Caribe.}} Meanwhile, the National Library hosted poetry readings two mornings a week as well as a book fair, opened by Minister of Culture Shirley Field Ridley, where Caribbean writers \quotation{might see their works displayed together,} as {\em Bohemia}'s reporter wrote.\footnote{González Freire.} The University of Guyana was the scene of a two-day symposium, \quotation{The Role of the Artist in Third World Society,} with contributions from intellectuals such as Gordon Rohlehr and Kenneth Ramchand.\footnote{For the conference proceedings, regrettably omitting Ramchand's contribution, see Celeste Dolphin, ed., \quotation{The Literary Vision of Carifesta '72: The Role of the Artist in Third World Society,} {\em Kaie}, no. 11 (August 1973): 100--115.} Many of the twelve hundred participating artists were housed in Festival City, a quickly built gated community of \quotation{250 greenheart houses on stilts,} located on the outskirts of Georgetown, complete with a \quotation{security guard (you needed a pass or Carifesta ID card to get in).}\footnote{Brathwaite, \quotation{Carifesta '72: Festival of Creative Arts of the Caribbean.}} The artists received return airfare, a living stipend of GY\$7 per day, and a chauffeured car to take them around the city (Brathwaite's fellow Barbadian Austin Clarke called them GUY-cars, after their official Guyana government plates).\footnote{Brathwaite; Austin Clarke, {\em A Passage Back Home: A Personal Reminiscence of Samuel Selvon} (Toronto: Exile, 1994), 89.}

Mobilizing the apparatuses of state, private business, and international diplomacy, Carifesta drew on the radical currents sweeping Caribbean cultural and political thought in the wake of independence movements to position Guyana as a leader of an emerging public sphere in the Americas. With its geographical position on the Latin American mainland, on the one hand, and its cultural and historical affinities with the Caribbean, on the other, Guyana could serve as nucleus in a network of decolonized states being built up throughout the Caribbean and beyond. Although ostensibly a Caribbean gathering, many of the participating countries were not actually in the Caribbean region itself. Participants from the American mainland included not just Suriname and French Guyana, but also Mexico, Venezuela, Colombia, Panama, Belize, Peru, Chile, and Brazil (figure 3).\footnote{Since these countries were absent from the initial list of potential invitees, the festival's expanded regional scope must have been a later policy development, but to date I have not found any historical evidence as to what precipitated the change. For this initial list, see Salkey's summary of the 1970 planning sessions in Andrew Salkey, {\em Georgetown Journal: A Caribbean Writer's Journey from London via Port of Spain to Georgetown, Guyana, 1970} (London: New Beacon, 1972), 276--78.} This expansion took Guyana beyond its relatively peripheral position within the Caribbean region, as defined by the legacy of plantation slavery, and it shows Burnham using cultural diplomacy to cool simmering border conflicts with Venezuela and Suriname. Carifesta '72 was the fruit and conduit of Guyana's Global South political orientation, inscribing the nation within a legacy of third worldist and Tricontinental political projects. At the same time, the newly independent country continued to receive foreign aid from the United States, positioned itself as part of the Non-aligned Movement, and joined economic associations within the Caribbean, such as the Caribbean Free-Trade Association and the Caribbean Community (CARICOM), which had developed after the breakdown of Federation.\footnote{By 1972, Guyana was in diplomatic relations with the Soviet Union, the East Germany, and the People's Republic of China, and it was about to establish the same with Cuba and North Korea in 1973. See Vibert C. Cambridge, {\em Musical Life in Guyana: History and Politics of Controlling Creativity} (Oxford: University Press of Mississippi, 2015), 201. These friendly relations with communist states would prompt the United States to drastically cut its foreign aid to Guyana in the mid-1970s.} The event thus invites comparison with 1968's radical congresses in Havana and Montreal, as well as the state-sponsored Pan-African festivals of the 1960s and 1970s, while also expressing the Burnham government's shrewd political pragmatism necessary for survival.\footnote{The bibliography on these topics is too extensive to rehearse here, but some recent work that has informed my thinking includes Rossen Djagalov, {\em From Internationalism to Postcolonialism: Literature and Cinema between the Second and the Third Worlds} (Montreal: McGill-Queen's University Press, 2020); Monica Popescu, {\em At Penpoint: African Literatures, Postcolonial Studies, and the Cold War} (Durham, NC: Duke University Press, 2020); Christopher J. Lee, {\em Making a World after Empire: The Bandung Moment and Its Political Afterlives} (Athens: Ohio University Press, 2010); Anne Garland Mahler, {\em From the Tricontinental to the Global South: Race, Radicalism, and Transnational Solidarity} (Durham, NC: Duke University Press, 2018); David Austin, {\em Moving against the System: The 1968 Congress of Black Writers and the Making of Global Consciousness} (London: Pluto, 2018); Katerina Gonzalez Seligmann, \quotation{Caliban, Why? The 1968 Cultural Congress of Havana, C. L. R. James, and the Role of the Caribbean Intellectual,} {\em Global South} 13, no. 1 (2019): 59--80; Cedric Tolliver, \quotation{Alternative Solidarities,} {\em Journal of Postcolonial Writing} 50, no. 4 (July 2014): 379--83.}

\placefigure[here]{festival country map}{\externalfigure[issue06/valence_1_rk_brochure-map.jpg]}


The festival's spectacle of grand construction projects and the largesse shown to artists and writers was meant to show off the modernity and rising prosperity of independent Guyana.\footnote{On the ways states stage their own rapid development and modernity as a spectacle, see, for instance, Andrew H. Apter, {\em The Pan-African Nation: Oil and the Spectacle of Culture in Nigeria} (Chicago: University of Chicago Press, 2005).} But if chauffeured cars and fancy catered dinners at the prime minister's house are the stuff of international diplomacy, at Carifesta those diplomats were also artists, writers, and intellectuals who saw themselves as having a cultural mission. The achievement of economic autonomy required a corresponding cultural (inter)nationalism. In the issue of {\em Kaie} dedicated to \quotation{The Vision of Carifesta,} we read in a summary of a speech given by Elvin McDavid, \quotation{Minister of Information & Culture,} that \quotation{with the establishment of the Cooperative Republic and the emphasis being placed upon self reliance {[}{\em sic}{]} in our economic life, it was important that the cultural movement should be co-ordinated.}\footnote{Elvin McDavid, \quotation{Govt. Policy on Cultural Development in the New Guyana,} {\em Kaie}, no. 8 (December 1971): 2.} As Ramaesh Bhagirat-Rivera affirms, the determining idea behind Carifesta was that the struggle for \quotation{self-determination that was occurring simultaneously across the postcolonial Global South} could not be accomplished without a corresponding \quotation{Cultural Revolution.}\footnote{Ramaesh Joseph Bhagirat-Rivera, \quotation{Between Pan-Africanism and a Multiracial Nation: Race, Regionalism, and Guyanese Nation-Building through the Caribbean Festival of Creative Arts (Carifesta), 1972,} {\em Interventions} 20, no. 7 (October 2018): 1025.} The idea of this coordination went back to 1966, when Burnham gained independence with his People's National Congress (PNC). It was on that occasion that he formulated his dream of the festival, inviting a group of \quotation{artists and writers from across the English-speaking Caribbean} to celebrate the new nation.\footnote{Bhagirat-Rivera. Invitees included such luminaries as C. L. R. James, George Lamming, Jan Carew, O. R. Dathorne, and Sam Selvon. See \quotation{Report on the Caribbean Writers & Artists Conference,} {\em Kaie}, no. 3 (1966): 3--6.}

Under Burnham, Guyanese cultural policy of independent nation building followed the logic of international development that dominated geopolitical thought at the time, albeit from the perspective of a third world country with a vanguard of Marxist intellectuals (figures 4 and 5). The buildup of national cultural infrastructure relied on Guyana's \quotation{relatively buoyant economy during the first half of the 1970s.}\footnote{Cambridge, {\em Musical Life in Guyana}, 227.} The National History and Arts Council, which had been upgraded from \quotation{committee} status in 1965, took on the task of cultural development. It was headed by Lynette Dolphin, a future Carifesta director who distinguished herself as an anthologizer and editor of Guyanese folk music.\footnote{Cambridge, 185.} Other cultural nation-building efforts of this time, many of them spurred by the shortcomings in arts infrastructure revealed by Carifesta, include the short-lived Guyfesta (Guyana Festival of the Arts, 1975--77) and the establishment of the \quotation{National School of Dance, the National Dance Company, and the Burrowes School of Art,} as well as plans for a \quotation{School of Drama and a School of Music{[}, which{]} were expected to fall under the umbrella of the Institute of Creative Arts.}\footnote{Cambridge, 200.} In a telling coincidence, the UN Educational, Scientific, and Cultural Organization (UNESCO) declared 1972 to be International Book Year, epitomizing the developmentalist teleology according to which literacy and the consolidation of autonomous cultural spheres would, with accompanying with socioeconomic development, engender a more equal and just world after decolonization.\footnote{Joseph R. Slaughter, {\em Human Rights, Inc: The World Novel, Narrative Form, and International Law} (The Bronx, NY: Fordham University Press, 2007); Sarah Brouillette, {\em UNESCO and the Fate of the Literary}, Post*45 (Stanford, CA: Stanford University Press, 2019). On the tensions inherent to the assumed link between economic and cultural development, particularly in the case of Jamaican policy, see Deborah A. Thomas, \quotation{Development, \quote{Culture,} and the Promise of Modern Progress,} {\em Social and Economic Studies} 54, no. 3 (2005): 97--125.} UNESCO not only informed the cultural policy behind the festival but also contributed materially to Carifesta (around GY\$20,000) and was \quotation{prepared to help ensure that lack of finance will be no barrier to participation in Carifesta '72 by less developed countries in the region.}\footnote{{\em Cuban Culture and Cultural Relations}, 45.} The UN Development Program (UNDP) committed to provide technical and financial assistance.\footnote{Cambridge, {\em Musical Life in Guyana}, 195.} Arthur Seymour, member of the Carifesta Secretariat and the Planning Committee, would go on to write a brief for UNESCO on Guyanese cultural policy.\footnote{A. J. Seymour, {\em Cultural Policy in Guyana} (Paris: UNESCO, 1977), https://unesdoc.unesco.org/ark:/48223/pf0000024652.} The fundamental belief shared by UNESCO and cultural policy makers like Seymour was that \quotation{throughout the early postcolonial period . . . there could be no distinction between cultural and economic development.}\footnote{Brouillette, {\em UNESCO and the Fate of the Literary}, 85.} Beyond UNESCO, Carifesta organizers drew on a much longer legacy of coordinating cultural, artistic, and political development to achieve decolonization. Seymour quoted from Aimé Césaire's 1956 speech at the Congrès International des Écrivains et Artistes Noirs and argued that \quotation{the creative artists of the Caribbean have a very special cultural burden to bear, and a very demanding role to play in the total social and political development in the Area.}\footnote{Salkey, {\em Georgetown Journal}, 273. On Césaire's speech, see Brent Hayes Edwards, \quotation{Introduction: Césaire in 1956,} {\em Social Text}, no. 103 (June 2010): 115--25.} What Césaire had called for in 1956, Carifesta '72 was meant to put into practice.

\placefigure[here]{Carifesta '72 organizing committee}{\externalfigure[issue06/valence_1_rk_committee.jpg]}


\placefigure[here]{Burnham speech}{\externalfigure[issue06/valence_1_rk_burnham-speech.jpg]}


Yet if Carifesta was Guyana's attempt to make \quotation{material reality} out of the \quotation{imagined ideal} of a Pan-Caribbean, or regional, public sphere, the festival's most immediate predecessor was the Caribbean Artists Movement (CAM).\footnote{Raphael Dalleo, {\em Caribbean Literature and the Public Sphere: From the Plantation to the Postcolonial} (Charlottesville: University of Virginia Press, 2011), 2.} The multidisciplinary group of artists and writers was founded in 1966 in London by Brathwaite, Andrew Salkey, and the Trinidadian John La Rose.\footnote{Anne Walmsley, {\em The Caribbean Artists Movement, 1966--1972: A Literary & Cultural History} (London: New Beacon, 1992), 35--62.} These West Indian intellectuals in diaspora held frequent private discussion groups as well as public readings and symposia, forums in which Brathwaite would first articulate his developing theories of the unifying substrates of Pan-Caribbean culture. Through this short-lived but influential organization Brathwaite shaped the course of the festival. He had been a driving force behind the initial meeting in Georgetown where the recommendations for Carifesta---many of which carried his watermark---were first formulated. A group of artists and writers, including CAM cofounders La Rose and Salkey, who covered the events in detail in his {\em Georgetown Journal}, had been invited to witness Forbes Burnham's declaration of Guyana as a Cooperative Republic on 23 February 1970. Burnham came to share many of CAM's motivating questions, if not ideals, as becomes clear in the speech that Salkey records in his journal.\footnote{Salkey, {\em Georgetown Journal}, 376--82.} These include CAM's \quotation{recognition of the whole Caribbean area as a meaningful historical and cultural entity} and the valuation of the culture of the Afro-Caribbean \quotation{folk,} which offered \quotation{the possibility of a radical subaltern resolution to the problem of social-cultural division.}\footnote{Bridget Jones, \quotation{\quote{The Unity Is Submarine}: Aspects of a Pan-Caribbean Consciousness in the Work of Kamau Brathwaite,} in Stewart Brown, ed., {\em The Art of Kamau Brathwaite} (Bridgend, UK: Seren, 1995), 88; David Scott, \quotation{On the Question of Caribbean Studies,} {\em Small Axe}, no. 41 (July 2013): 5--6.} Intellectuals like Brathwaite had to find their place in a new public sphere as anticolonial nationalist movements rejected the high-cultural positioning of literature, associated with colonial-era cultural domination from Britain. There was, as Raphael Dalleo observes, a \quotation{suspicion of literature as part of the desire for indigenous cultural forms.} As a way for intellectuals to preserve their position within that public sphere, they turned to (re)valuing the culture of \quotation{the people} or \quotation{the folk,} notwithstanding the middle- or even upper-class position intellectuals might occupy in a strictly material sense.\footnote{Dalleo, {\em Caribbean Literature and the Public Sphere}, 202.} Popular culture, as the culture of the masses, then, became the \quotation{voice of both the public and an oppositional counterpublic.}\footnote{Dalleo.} For Brathwaite, seeing his own thinking swept up in the historical currents of decolonization and third world state building, and his ideas implemented at the level of institutions and in the material transformation of Georgetown itself (Carifesta Avenue still exists today), must have been exhilarating. But he found himself ambivalently positioned between the state bureaucrats, political leaders, and cultural policy makers who would offer him the means of putting his ideas into practice, and the students, intellectuals, and masses whose more radical energy and disaffection with postcolonial governance had inspired those ideas. We see that ambivalence at work in two sets of writings that invoke Carifesta: Brathwaite's essays and Savacou pamphlets from the 1970s position Carifesta as the most important event of the decade, while his reporting in the {\em Advocate-News} and his statements to interviewers during the festival vacillate between great excitement and an ultimate disappointment in the festival's inability to fulfill its grand promises.

In these writings, Brathwaite understood Carifesta immediately in the context of the \quotation{radical 1970s,} not just in the Caribbean but in the third world as a whole.\footnote{Laurie R. Lambert, \quotation{Postcolonial Stirrings: The Crisis of Nationalism,} in Curdella Forbes and Raphael Dalleo, eds., {\em Caribbean Literature in Transition, 1920--1970}, vol.~2 (Cambridge University Press, 2021), 177.} Even by 1972, when Carifesta happened, the decade had already proved eventful in the Caribbean. Cultural flourished amid a broader sense of disquiet, unrest, and disillusionment with the course decolonization had taken. For David Scott, the 1974 photograph, taken by Horace Ové, of Samuel Selvon, La Rose, and Salkey on the back cover of {\em Georgetown Journal} speaks of their \quotation{generation's longing for release from false emancipations.}\footnote{David Scott, \quotation{Preface,} {\em Small Axe}, no. 15 (March 2004): v.} In 1970, the Black Power movement swept through the United States and inspired months of political unrest in Trinidad, just as the planning for Carifesta got started at the Caribbean Writers and Artists Convention that February. The 1960s had ended in 1968 with the Cultural Congress in Havana and the Montreal Congress of Black Writers. With the maturation of the first generation of scholars on the University of the West Indies campuses, which had been established as part of the effort toward West Indian federation, a more radical generation of intellectuals consolidated itself around the Walter Rodney affair.\footnote{On the establishment of cultural studies on the UWI campuses, see Carolyn Cooper, \quotation{Race and the Cultural Politics of Self-Representation: A View from the University of the West Indies,} {\em Research in African Literatures} 27, no. 4 (1996): 97--105. See also the recollections of Edward Baugh; Eddie Baugh, \quotation{Confessions of a Critic: (Keynote Address, 23rd Annual Conference on West Indian Literature, Grenada, March 8--11, 2004),} {\em Journal of West Indian Literature} 15, no. 1/2 (2006): 15--28.} As David Scott points out, the banishment of the Guyanese scholar and public intellectual from Jamaica, in 1968, turned out to be a \quotation{catalytic} moment that made it \quotation{imperative to think \quote{culture} as a knowledge-domain of power and struggle.}\footnote{Scott, \quotation{On the Question of Caribbean Studies,} 5--6.} Brathwaite had come back to London in 1968 as well, and he spoke to CAM about \quotation{an artistic alternative, a social alternative, to our so-far accepted Eurocentric kind of culture.} Having witnessed a \quotation{revolution of consciousness,} he called for \quotation{new aesthetic tools.}\footnote{Anne Walmsley, \quotation{A Sense of Community: Kamau Brathwaite and the Caribbean Artists Movement,} in Stewart Brown, ed., {\em The Art of Kamau Brathwaite} (Bridgend, UK: Seren, 1995), 108.}

In the essays of the 1970s, Brathwaite would not only refer to Carifesta as a lens to understand these developments in retrospect, he also used the festival itself as an occasion for thinking and speaking. The oft-quoted essay \quotation{Caribbean Man in Space and Time} (1975) would be included as part of the anthology published on the occasion of Carifesta '76 in Jamaica, but it dates back to a 1973 conference in Barbados---shortly after Carifesta.\footnote{Edward Kamau Brathwaite, \quotation{Caribbean Man in Space and Time,} in John Hearne, ed., {\em Carifesta Forum: An Anthology of 20 Caribbean Voices} (Kingston: Institute of Jamaica and Jamaica Journal, 1976), 199--208. Scott, \quotation{On the Question of Caribbean Studies,} 4.} First delivered as a talk at the University of Pennsylvania in 1975, \quotation{The Love Axe (1): Developing a Caribbean Aesthetic, 1962--1974} lists Carifesta as one of the \quotation{several events and movements, at one time invisible, unrecognized, ignored {[}which{]} are slowly revealing their significance, connections and continuity with what we can now call the folk or alternative tradition.} In the essay, Brathwaite lamented a certain exhaustion of creative energies, which he attributed to the fact that \quotation{all of our major novelists continue to live, as they have done since the 1950s and early 1960s, abroad; and so they are increasingly cut off from the metaphorical and stylistic explosions that are even now taking place at home.} \quotation{The New Movement,} he would call it later on in the piece. This naming is deliberately inchoate and imprecise, capturing the specificity of the feeling of change and coalition but not its content, direction, or basis.\footnote{Edward Brathwaite, \quotation{The Love Axe (1): Developing a Caribbean Aesthetic, 1962--1974,} in {\em Reading Black: Essays in the Criticism of African, Caribbean, and Black American Literature} (Ithaca, NY: Cornell University Press, 1976), 20--21.} Throughout these writings, he would draw up lists and inventories with a strong sense that he was living through great change and that he was outlining a certain zeitgeist: \quotation{We must sort out, categorize, store, define and then begin the process of education, re-definition, discovery and re-assessment.}\footnote{Brathwaite, 26. See also Jones, \quotation{\quote{The Unity Is Submarine,}} 96--97, who perceives \quotation{a fundamental analogy . . . between these {[}CAM{]} gatherings and a favoured structure in Brathwaite's imagination: aggregation.}} \quotation{The Love Axe} vacillates between a sense of rupture---\quotation{explosions,} a \quotation{volcanic eruption,} \quotation{sudden sparks}---and slow revelation: \quotation{awakenings,} \quotation{on the verge of articulat{[}ion{]},} a \quotation{yeast-like influence.}\footnote{Brathwaite, \quotation{The Love Axe,} 20--23.} {\em Contradictory Omens} echoes this observation, dubbing Carifesta a \quotation{volcanic eruption.} The common thread is that the great changes are taking place \quotation{at home,} that they are \quotation{local,} \quotation{in our part of the world.} If this new movement was an \quotation{explosion,} Brathwaite framed it explicitly in the context of the \quotation{break-up of the Federation of the West Indies,} which created a sense of \quotation{post-colonial blues} and an \quotation{implosion of people and thought.} The postcolonial omens were contradictory. Building on the religious terms, he judged this time of crisis a \quotation{re-formation of values.} And it \quotation{culminat{[}ed{]},} he wrote, \quotation{in the miracle of Carifesta in Guyana.}\footnote{Edward Brathwaite, {\em Contradictory Omens: Cultural Diversity and Integration in the Caribbean} (Mona, Jamaica: Savacou, 1974), 22, 32, 25, 30.}

These are the layers of cultural criticism and historical thinking encapsulated in Brathwaite's bold statement I quoted at the beginning of this essay: \quotation{Carifesta was Emancipation Day come true; collective Declaration of Independence; first ever meeting of the Caribbean.}\footnote{Brathwaite, \quotation{Carifesta '72: Festival of Creative Arts of the Caribbean.}} Following Brathwaite's thinking through the eight retrospective articles that appeared in the {\em Advocate-News} between October and December 1972, I lay out, first, what made Brathwaite proclaim Carifesta a \quotation{miracle} and, second, where he found it coming up short. These disappointments show us where his true commitments lay, and along the way it becomes clear that the article series is part of Brathwaite's attempt not just to record his impressions for posterity but also to make good on the promise it represented to him. Specifically, I focus on Brathwaite's ambivalence regarding the following issues: the appropriation of mass and folk culture for commercial and touristic ends; the relative marginalization of literature and intellectuals at the actual festival; the ideology of multiracial harmony and nationalism professed by Carifesta against the background of racial conflict in Guyana; and the festival's failure to spur the growth of a movement like CAM in the Caribbean.

In what sense was Carifesta a rupture? In his first article, Brathwaite frames it in terms of the history of slavery: \quotation{The slave masters were absent,} he begins. \quotation{There were no whips at Carifesta.} Quickly he moves to framing Carifesta as a break, too, with ongoing, neocolonial domination after emancipation and independence. \quotation{No foreign magistrates of taste or art. No missionaries or sergeant-majors.} Its values were not dictated by the public sphere of the metropole. As a result, he implies that the festival took place outside the gaze of the Global North. \quotation{There were no Euro-American camera crews, no dotty anthropologists taking notes.} For him, Samuel Selvon summed it up when saying, \quotation{at a literary meeting: \quote{Look man, it down matter how we do things here; is our thing and we goin} do it our way, understan?'}\footnote{Brathwaite, \quotation{Carifesta '72: Festival of Creative Arts of the Caribbean.}} The second article describes a second aspect of the revolution in values: Carifesta as the first successful attempt to unify the Caribbean people as people, not just as nations: the \quotation{Amerindian, Euro-Caribbean, Afro-Caribbean, Indo-Caribbean, Sino-Caribbean, albino, mulatto, mestizo . . . were not one people} because of Western \quotation{spectacles,} the \quotation{cracked mirror of recognition,} inherited \quotation{from the tale of the slave master} and \quotation{the forked tongue of the colonial magistrate.} This, for Brathwaite, had been the flaw of the West Indian Federation (1958--62): the attempted \quotation{Kingdom of the Federal Idea} was modeled on the \quotation{metropolitan style of control} and had it backward. Waving a \quotation{magic wand over our shattered islands} and enthroning \quotation{our own kings and queens (disguised as presidents and premiers)} did not work, Brathwaite contends, because \quotation{there was no Kingdom of the Word.}\footnote{Edward Kamau Brathwaite, \quotation{Carifesta '72: The Caribbean Cultural Revolution,} {\em Advocate-News}, 22 October 1972.} The trauma of the Federation's collapse still felt significant to Brathwaite---as it had been for so many intellectuals of the preceding generation---and even if West Indian writers had played a large part in conceiving of and embodying the promise of federation, its failure convinced him that \quotation{consciousness} of a shared cultural \quotation{root and trunk} had to come first.\footnote{Edward Kamau Brathwaite, \quotation{Carifesta '72: Government and People,} {\em Advocate-News}, 29 October 1972. On the \quotation{political responsibility accorded to literature in the context of West Indian Federation,} see Alison Donnell, \quotation{West Indian Literature and Federation: Imaginative Accord and Uneven Realities,} {\em Small Axe}, no. 61 (March 2020): 78--86.} Brathwaite insists on the word \quotation{root,} here; in the third piece, he complains about a \quotation{printer's devil} who had misprinted or cut out this word from the quote in the second article. But \quotation{root and trunk and offspring} together represent \quotation{cultural wholeness: past, present, future.} Bringing these cultures together is something that government can facilitate but never legislate; it can \quotation{legislate against class barriers,} \quotation{against colour/ethnic discrimination,} and \quotation{help the sense of being local,} but the \quotation{Alpha and Omega of the new legislation will be consciousness,} the mutual recognition of shared culture.\footnote{Brathwaite, \quotation{Carifesta '72: Government and People.}}

In articles that followed (weeks 4 and 5), he laid out his current understanding of \quotation{plural societies,} the mechanisms of \quotation{culture---contact,} the place of \quotation{Africa in the Caribbean,} and the question of how to achieve \quotation{multicultural synthesis} in the politics of the Caribbean.\footnote{Edward Kamau Brathwaite, \quotation{Carifesta '72: Ancestral Consciousness,} {\em Advocate-News}, 5 November 1972.} Brathwaite looked for authenticity as opposed to expressions that he saw as touristic or as colonial mimicry---stale adaptations of Western classical forms in music, dance, and literature that only reminded Brathwaite of West Indian elites who were alienated from their origins. He found this authenticity most powerfully expressed in Afro-Caribbean folk arts. Still derided by political elites and subject to censorship---according to Brathwaite---practices and arts like Haitian Vodou rituals, Jamaican reggae and Rastafarian music, and the dances of the Surinamese Djuka maroons revived the people's \quotation{ancestral consciousness} of African elements in the Caribbean. To Brathwaite, this heritage was the truly Pan-Caribbean one around which the people could come together. In this respect, his views were perfectly congruent with what the Burnham government meant to promote with the festival.

Brathwaite barely mentions, however, the tensions between the Afro-Guyanese and East Indian Guyanese communities that had led to numerous violent clashes over the course of the 1960s. Prominent Indo-Guyanese organizations boycotted Carifesta because the organizers appropriated money \quotation{from the Indian Immigration Fund,} established for the return of indentured laborers to India, in order to pay for construction projects. Rivera estimates that \quotation{at the time of independence, the fund} contained between \quotation{GY\$380,000 and GY\$440,000,} a substantial portion of Carifesta's overall expenses of GY\$2.5 million.\footnote{Bhagirat-Rivera, \quotation{Between Pan-Africanism and a Multiracial Nation,} 1031.} In response, PNC supporters and party leaders accused the Indo-Guyanese community of continuing to divide the nation, condemning their supposedly India-oriented separatist sympathies. Although the People's Progressive Party ultimately did not boycott the festival, it was still haunted by the trauma of the breakup of the 1950s \quotation{multiracial coalition} of nationalists under Burnham and Cheddi Jagan and the UK- and US-backed efforts to ensure that Burnham would prevail as the more \quotation{moderate} leader.\footnote{See the detailed account of these tensions and debates in Bhagirat-Rivera.} Brathwaite himself only remarked on the \quotation{Indian unwillingness to join in} and \quotation{really make all-a-we-is one} a reality. Ultimately, Brathwaite betrays his bias when he implies that only Afro-Caribbean cultural forms are flexible and dynamic enough to create both a national and a Pan-Caribbean culture: \quotation{The East Indians, who came after the structure had established itself---plot and plantation---came bearing their own bound, inflexible ikons, and adjusted slowly, tentatively; unwilling allies of merchant and planter; but beholding; and fearing contact with the black cultural outcastes. They developed a tension, therefore, between body and spirit, object and inject.}\footnote{Brathwaite, \quotation{Carifesta '72: The Presence of Africa in the Caribbean.}} Brathwaite's usually astute sense for politics seems to falter here, as he blames an ahistorical cultural essence for what was fundamentally a conflict about the democratic process in the new nation, a long history of labor struggles, and the spatial division of rural East Indian agricultural workers versus the largely urban Afro-Guyanese population.

This misunderstanding also expresses the overdetermined meanings of \quotation{culture} at work in Brathwaite's thinking here. He uses what Deborah Thomas has called a \quotation{generically anthropological definition} of culture as \quotation{\quote{what people do} in all spheres of life}: \quotation{We must know: the way we talk; verb, trombone slide, smooth pebble of syllable, incantation. How we walk, cook, dance, make love, worship; how we interpret our dreams, prepare out plots for planting; how we bury our dead.}\footnote{Brathwaite, \quotation{Carifesta '72: Government and People.}} It is, \quotation{simply, the way of life of a people.}\footnote{Brathwaite, \quotation{Carifesta '72: Ancestral Consciousness.}} No cultural form is inherently higher or more pure than any other; walking, talking, dreaming, and mourning are part of a continuum that includes music and poetry as well. His \quotation{we} is broadly defined as Caribbean. Anticipating his style in later essays from the 1970s, Brathwaite uses a list of concrete metonymies, quick-paced in its omission of articles, that cuts culture up into constituent fragments that indicate medium or use---music becomes a \quotation{trombone slide,} poetry a \quotation{smooth pebble of syllable} or \quotation{incantation}; talk is just a \quotation{verb.}\footnote{Note how this pebble recalls Brathwaite's much-quoted image of the Caribbean in \quotation{Calypso}: \quotation{The stone had skidded, arc'd, and bloomed into islands}; Edward Kamau Brathwaite, {\em The Arrivants: A New World Trilogy} (Oxford: Oxford University Press, 1973), 48. See also Pamela Mordecai, \quotation{The Image of the Pebble in Brathwaite's {\em Arrivants},} {\em Carib}, no. 5 (1989): 60--78.} Avoiding the finality of the period, Brathwaite connects these fragments fluidly by way of colons, a usage of that punctuation mark that is characteristic of and unique to his style in print. The lists furnish common ground for Pan-Caribbean culture; yet the anaphoric buildup and the rhythms internal to each element create an overarching rhythm that can hold the whole, enacting Brathwaite's thesis that there is an underlying wholeness beneath the fragments. Two years after Carifesta, he would ask \quotation{how to study the fragments / whole,} but in these pieces he already performs this study not just at the macro-level of historiography but also at the micro-level of the sentence.\footnote{Brathwaite, \quotation{Caribbean Man in Space and Time,} 199.}

This view of culture lent itself well to Brathwaite's lifelong project of revaluing the Afro-Caribbean folk and his use and theorization of \quotation{nation language.} Yet when he objects to a supposed Indo-Guyanese inflexibility, he betrays a bias toward seeing culture on an \quotation{instrumental} level, as something that state and cultural policy makers can bend toward other ends, such as modernization, nation building, and intraregional diplomacy. As instrument, then, \quotation{culture} becomes \quotation{a problem to be solved, and at the same time, the basis for solutions,} as well as the basis for mapping \quotation{cultural practices onto particular groups of people.}\footnote{Thomas, \quotation{Development, \quote{Culture,} and the Promise of Modern Progress,} 108.} In effect, Brathwaite accuses the Indo-Guyanese of being stuck in cultural tradition---an anachronism and inflexibility best understood in opposition to the ways Afro-Caribbean cultures seemed amenable to various other ends. In his description of the Afro-Guyanese as \quotation{black cultural outcastes,} his idiosyncratic spelling deliberately evokes the Indian caste system. He seems to imply that the East Indian community sees itself as higher caste than the Afro-Guyanese population---no matter what their ancestors' position might have been in India---and that they, harboring anti-Black sentiments, are to some extent responsible for the latter group's continuing marginalization. Calling the Afro-Guyanese \quotation{outcastes} also evokes the caste system's particular prohibition on cultural contact across hierarchically separated groups, and specifically aligns the Afro-Guyanese with the Dalit, or \quotation{untouchable,} caste. Brathwaite's view seems to be that Indo-Guyanese importation of this system---insofar as the caste system would prohibit intercultural contact---thwarted the creolization process, which thus failed to produce a pluralistic national and regional sentiment in Guyana.

Yet the very idea of (re)staging popular culture for the people could be further critiqued as part of a bourgeois campaign of nationalist appropriation. For all his praise of popular and folk culture, Brathwaite laments the way they were staged at Carifesta as spectacle: \quotation{There was too much preoccupation by the organisers . . . with the immediate and emotional impression of the people, rather than a concern for a more lasting and meaningful impression.}\footnote{Thomas.} Whereas Brathwaite found Rex Nettleford's National Dance Theatre Company a revelation and the festival's highlight, Sylvia Wynter criticized the group in no uncertain terms in her wide-ranging review of the anthology {\em One Love} (1971), edited by Audvil King, Althea Helps, Pam Wint, and Frank Hasfal. She sees Nettleford's effort ultimately as a symptom of an emerging neocolonial era after independence (\quotation{new industrial corporate imperialism}) that was masked over by an ideology of \quotation{literary \quote{blackism,}} which she defined as the \quotation{revindication of the black mystique} by the emerging national bourgeoisie while \quotation{obscuring the basic relations of the denial of black humanity} and separating culture \quotation{from its political and economic base.}\footnote{Sylvia Wynter, \quotation{One Love--Rhetoric or Reality?--Aspects of Afro-Jamaicanism,} {\em Caribbean Studies} 12, no. 3 (October 1972): 70--77. On cultural development policy and the institutionalization of \quotation{folk} or \quotation{African} blackness in 1960s independent Jamaica, see Deborah A. Thomas, {\em Modern Blackness: Nationalism, Globalization, and the Politics of Culture in Jamaica} (Durham, NC: Duke University Press, 2004), 65.} In a characteristically withering analysis, liberally citing Fanon's \quotation{On National Culture,} Wynter derides the \quotation{middle class folkism} of writers such as Andrew Salkey, poet-performers such as Louise Bennett, and Nettleford's dance theater as no better than the trivializing reification of national culture at the hands of \quotation{the admen, the television boys, {[}and{]} the gossip columnists.}\footnote{Wynter, \quotation{One Love,} 83.} Had she written about Carifesta, which was indeed a product of Guyana's \quotation{cultural establishment} and which featured many of these artists, she might well have described it as one of the characteristic postindependence \quotation{pitfalls of national consciousness.}\footnote{Walmsley, 271. To put it in the terms of Fanon's classic third chapter in {\em The Wretched of the Earth}, trans. Constance Farrington (New York: Grove, 1963).}

The divergence between Wynter's point of view in this essay and Brathwaite's effusive praise of Carifesta is all the more surprising given the numerous commonalities in their work in the early 1970s. Both critics observe the upsurging of a new kind of black popular culture in a \quotation{frontier zone} but diverge about the risks and realities of its co-optation by their own intellectual class.\footnote{Wynter, \quotation{One Love,} 66. See also the interview with David Scott in which Wynter revisited this time and these questions; David Scott, \quotation{The Re-enchantment of Humanism: An Interview with Sylvia Wynter,} {\em Small Axe}, no. 8 (September 2000): 119--207.} Brathwaite shows his awareness of these matters but limits his criticism of \quotation{censorship} to commercialism, not direct political cooptation. He deemed certain performances mere pandering to tourism. For instance, he summarily dismissed Barbados's contributions in the visual arts, quoting from \quotation{Art} by Bajan oral poet Bruce St.~John: \quotation{painter paint beach beach beach / painter paint tree tree tree / painter doan paint studyation / painter doan paint worryation / painter doan paint de nation.}\footnote{Edward Kamau Brathwaite, \quotation{Carifesta '72: Carry on Big Inglan,} {\em Advocate-News}, 26 November 1972.} And he shuddered at the prospect that, if the festival took place in Jamaica \quotation{next time,} it would be, \quotation{God forbid, . . . at Montego Bay: a boost of dollar bills?}\footnote{Edward Kamau Brathwaite, \quotation{Carifesta '72: The Presence of Africa in the Caribbean,} {\em Advocate-News}, 12 November 1972. For the published poem, see Bruce St.~John, \quotation{Art,} {\em Savacou}, no. 3/4 (December 1970/March 1971): 82.}

The ways Carifesta created \quotation{material reality} out of an \quotation{imagined ideal} of a Pan-Caribbean, or regional, public sphere did not quite produce the counterpublic of the folk that Brathwaite would describe in \quotation{The Love Axe.} Brathwaite spoke of Carifesta as the \quotation{most important event ever to take place in the Caribbean,} yet he was disappointed to find that the honors Burnham had initially placed on literature and publishing---as he had dreamed his dreams largely alongside writers---did not pay out in actual attention during the festival. Austin Clarke, his fellow Bajan writer, found that \quotation{writers . . . had been insulted. It was they, he felt, who had conceived of Carifesta: Lamming, Carter, Seymour James.}\footnote{Edward Kamau Brathwaite, \quotation{Carifesta '72: The Beguine---and without the Whips,} {\em Advocate-News}, 3 December 1972.} The writers may have been guests of honor, but the festival organizers allocated their largest venues to dance and theater performances where \quotation{the people} would could witness them directly, without the mediation of intellectuals. Even those venues, large as they were, proved inadequate to the \quotation{massive demand of people interested in attending.}\footnote{Bhagirat-Rivera, \quotation{Between Pan-Africanism and a Multiracial Nation,} 1029.} Bhagirat-Rivera quotes several accounts of attendees, such as \quotation{O. R. Dathorne, a US-based Guyanese intellectual and writer} and \quotation{Jamaican artist Karl Parboosingh,} who doubted whether \quotation{everyday people were able to attend.}\footnote{Bhagirat-Rivera.} Yet the smaller venues did not satisfy either. Moving from the opening ceremony at the National Park to the National Library, Brathwaite realized that all literary events and creative writers had been, as he put it to Rickey Singh of the {\em Guyana Graphic}, \quotation{relegated to small public appearances,} which impressed on him that writers were not given \quotation{equal status} at the festival.\footnote{Rickey Singh, \quotation{Poet Praises Carifesta but Feels Too Much Emphasis on Entertainment,} {\em Guyana Graphic}, 2 September 1972, https://dloc.com/CA00199904/00001.} Equal, that is, to the more spectacular performers of music, dance, and theater.

The festival would have been one occasion to spur reader demand and enable a true meeting between writers and their readers from across the Caribbean. The book fair that opened Carifesta comprised \quotation{works from Dutch, English, French, Portuguese and Spanish speaking countries} and was explicitly framed in terms of \quotation{regional writers.}\footnote{\quotation{We Need to Know Who We Are,} {\em Guyana Graphic}, 26 August 1972, \hyphenatedurl{http://ufdc.ufl.edu/CA00199794/00001.}} The Carifesta book fair embodied a wish voiced by many Caribbean writers for a regional book publishing infrastructure. The organizers of Carifesta not only hoped the festival might kick-start cultural development in the region in general but specifically addressed the topic of publishing.\footnote{See Rex Nettleford's discussion of the \quotation{idea of regional festivals of arts . . . as means to cultural co-operation {[}and{]} as a device to forge the solidarity of the Caribbean collective consciousness} in {\em Cultural Action and Social Change: The Case of Jamaica: An Essay in Caribbean Cultural Identity} (Ottawa: International Development Research Centre, 1979), 152--56.} On 30 May 1966, during the \quotation{Caribbean Writers & Artists} conference at Queen's College where the \quotation{dream} of Carifesta was first formulated, Guyana's prime minister, Forbes Burnham, had \quotation{asked for ideas on the practicability of a publishing company for the region} and hoped this might create \quotation{the atmosphere which would encourage artists to cease being emigres and having to go to London for recognition.}\footnote{McDavid, \quotation{Govt. Policy on Cultural Development in the New Guyana,} 5--6. Many of the invited writers were such \quotation{émigrés.}} And in late February 1970, during the second Caribbean Writers and Artists Convention, which coincided with the inauguration of Guyana as a Cooperative Republic, Burnham again acknowledged the need for \quotation{a publishing house} and framed it explicitly as a solution to the problem of the exiled writer. Having this regional infrastructure would take away the \quotation{physical limitations} that compelled writers to seek residence and publication in London while also solving the \quotation{psychological difficulties} the writer faced while \quotation{singing his song in a foreign land.} Burnham saw only two options, one authentic, the other inauthentic: \quotation{Are we writing to portray what we are in the Caribbean, what it is like to be a West Indian, what has been our political struggle and what is still our economic struggle to come? . . . Or are we writing to please the vulgar crowd in the metropolitan cities of western Europe and North America?}\footnote{Salkey, {\em Georgetown Journal}, 378--79.}

The day after the opening ceremony, the {\em Guyana Graphic} published a photograph of Brathwaite at the library meeting Arthur Seymour, who was there as member of the Carifesta Secretariat and the Planning Committee, editor of the anthology \quotation{New Writing in the Caribbean,} and deputy chairman of the National History and Arts Council (figure 6).\footnote{Salkey.} In animated discussion, Brathwaite and Seymour stand in front of what the photo's caption calls a \quotation{montage of some of Brathwaite's poetry.}\footnote{\quotation{We Need to Know Who We Are,} {\em Guyana Graphic}, 26 August 1972, https://dloc.com/CA00199794/00001.} Around a central portrait of Brathwaite, covers of his books are taped to a notice board. In the top right corner, we see the covers of his collections {\em Masks} (1968) and {\em Islands} (1969), with {\em Rights of Passage} (1967) presumably mounted just outside of the frame.\footnote{I have not been able to identify any of the other covers.} A similar photo, published in the same paper that day, shows a notice board filled with Wilson Harris's works (figure 7).\footnote{\quotation{Poet M. R. Moonar (Image),} {\em Guyana Graphic}, 26 August 1972, https://dloc.com/CA00199839/00001.} In my account, the photograph of Seymour and Brathwaite meeting at the library symbolizes Carifesta as an attempt to make \quotation{material reality} out of the \quotation{imagined ideal} of a Pan-Caribbean, or regional, public sphere---to use Raphael Dalleo's terms again.\footnote{Dalleo, {\em Caribbean Literature and the Public Sphere}, 2.} On the one hand, it recalls Seymour's anthology {\em New Writing in the Caribbean}---convoking regional writers of all major regional languages together in print---as well as the longer history of local literary publishing and consecration that he was able to establish through his editorship of the magazine {\em Kyk-over-al}.\footnote{Arthur J. Seymour, ed., {\em New Writing in the Caribbean} (Georgetown: Guyana Lithographic, 1972).}~Recognizing the importance of translation in \quotation{learning about the creative work that is going on in the various parts of the Caribbean,} Seymour even called for \quotation{a modest Bureau of Caribbean Translation} for translation between all the languages, where \quotation{UNESCO could be of very great assistance in the selection and translation} of works.\footnote{Arthur James Seymour, \quotation{We Must Hear Our Brothers Speak,} in {\em The Literary Vision of Carifesta} {\em '72: The Role of the Artist in Third World Society} (Georgetown: National History and Arts Council of Guyana, 1973), 5--8.} On the other hand, the photo reveals Brathwaite's poetry montage as an image of what it would have been like to publish his breakthrough collections at home. And yet one can't help notice that many if not most of the exhibited works had been published in the metropolitan centers: Brathwaite's by Oxford University Press, Harris's by Faber and Faber. Katerina Gonzalez Seligmann points out that George Lamming and V. S. Naipaul \quotation{theorize their positions as writers in literary exile} in terms that nuance Pascale Casanova's influential model in {\em The World Republic of Letters}. Generally, now-canonical Caribbean writers published their first works in the metropole directly---it's where they went to \quotation{materially make their books}---and this situation is evidence not so much of an absence of literary resources in the peripheries as of impoverished \quotation{means of literary production.}\footnote{Katerina Gonzalez Seligmann, \quotation{The Void, the Distance, Elsewhere: Literary Infrastructure and Empire in the Caribbean,} {\em Small Axe}, no. 62 (July 2020): 1--16, \useURL[url1][https://doi.org/10.1215/07990537-8604442]\from[url1]. For a detailed history of this process, and Brathwaite's trajectory at OUP in particular, see Gail Ching-Liang Low, {\em Publishing the Postcolonial: Anglophone West African and Caribbean Writing in the UK, 1948--1968} (New York: Routledge, 2011). For how this situation still largely applies in the contemporary period, see Kimberley Anne Robinson-Walcott, \quotation{Publishing Anglo-Caribbean Fiction: The Return to the Metropole,} {\em International Journal of the Book} 3, no. 3 (2006): 55--59.}

\placefigure[here]{Seymour and Brathwaite}{\externalfigure[issue06/valence_1_rk_brathwaite-seymour.jpg]}


\placefigure[here]{M.R. Moonar}{\externalfigure[issue06/valence_1_rk_wilson-harris.jpg]}


The first great disappointment of Carifesta for Brathwaite, then, was that it only provided an ephemeral regional public and publishing opportunity, without leading to any improvement in regional publishing infrastructures. Indeed, where were the books themselves? Unlike later events of this kind, such as the International Book Fairs of Radical Black and Third World Books held in the United Kingdom, this fair did not feature any actual bookselling.\footnote{See Walmsley, 303; and Sarah White, Roxy Harris, and Sharmilla Beezmohun, eds., {\em A Meeting of the Continents: The International Book Fair of Radical Black and Third World Books---Revisited: History, Memories, Organisation and Programmes, 1982--1995} (London: New Beacon and George Padmore Institute, 2005).} The Cuban delegation had donated a selection of two hundred books---including the works of José Martí, Che Guevara, and Nicolás Guillén---to the Guyanese library as a diplomatic gesture, and they may have been exhibited there as part of the five hundred or so books presented.\footnote{Omar Vázquez, \quotation{Actuarán \quote{La Aragón} y el Conjunto Folklórico para los trabajadores de Linden, Guyana,} {\em Granma}, 31 August 1972, in {\em Cuban Culture and Cultural Relations}. Walmsley reports the figure of 500 books; Walmsley, {\em The Caribbean Artists Movement}, 273.} It's hard to tell from the newspaper accounts whether the \quotation{montage} of Brathwaite's poetry included an exhibition of physical books or merely displayed covers and photographs. In the interview with Rickey Singh, Brathwaite reiterated the call for a \quotation{local publishing house} and expressed his disappointment \quotation{that the organisers of Carifesta did not see the necessity or value in having a book shop specially set up for the sale of works by Caribbean writers rather than just mounting a book exhibition, and limited at that.}\footnote{Singh, \quotation{Poet Praises Carifesta.}} In his chronicle of the 1970 Caribbean Writers and Artists Convention, Salkey found himself \quotation{profoundly disturbed} by the paucity of \quotation{recent Caribbean fiction and non-fiction} on the shelves of bookstores in Georgetown, which \quotation{drowned in a sea of Caribbean travel books and histories by British authors,} while bookstore managers speculated on the possible but ephemeral boost in sales that the visit of the delegation of distinguished writers might cause.\footnote{Salkey, {\em Georgetown Journal}, 285--87.}

Second, Brathwaite suggested that he \quotation{would have liked to see for example, a major poetry reading session at the National Culture Centre with the active participation of the public.} One of the problems Burnham and other writers had highlighted during the conception of Carifesta---the presumed lack of a regional reading public, and the exiled writer's distance from such a public, being familiar to the people only \quotation{on a bookshelf}---remained unaddressed. Not without reason, Brathwaite feared that \quotation{the authors, poets, painters, sculptors, musicians and critics} would end up \quotation{talking among themselves in a grand cerebral exercise} instead of \quotation{the public {[}being{]} made to benefit from this meeting.}\footnote{Singh, \quotation{Poet Praises Carifesta.}} The organizers indeed shared this fear---Minister of Information and Culture Elvin McDavid urged the National History and Arts Council \quotation{to ensure that people do not come to view {[}the council's members{]} as a group of well educated people enjoying themselves}---but Carifesta's literary and intellectual sessions may not have prevented this perception.\footnote{McDavid, \quotation{Govt. Policy on Cultural Development in the New Guyana,} 3.} Brathwaite pointed out that the poetry sessions, already small, were inaccessible \quotation{for many working people,} because they took place \quotation{at inconvenient times,} even as lately, Brathwaite felt, \quotation{the writers of the Caribbean are truly speaking the language and expressing the feelings of the region's people.} It was a \quotation{glorious opportunity} whose promise was ultimately not fulfilled.\footnote{Singh, \quotation{Poet Praises Carifesta.}}

Another promise Brathwaite saw in Carifesta was the chance to bring CAM home to the Caribbean, just as the group fell apart in England. Although Burnham had called on Brathwaite as an intellectual during the 1970 Caribbean Writers and Artists Convention, the festival had little room for intellectuals now. Brathwaite did not feel that his poetic performances allowed him to also be a public intellectual with an educative role. This was especially disappointing given that Carifesta finally presented him with the opportunity to address a transnational, Pan-Caribbean public. This aspect of his ideals especially reflected \quotation{his CAM experience, his CAM aspirations.}\footnote{Walmsley, {\em The Caribbean Artists Movement}, 278.} In her brief account of Carifesta '72, Anne Walmsley underscores the resonance between the festival and \quotation{the cultural endeavour of Caribbean people which was nurtured within CAM.}\footnote{Walmsley, 279. Walmsley's whole account comprises pages 271--79.} She asked Brathwaite about the \quotation{golden opportunity} for a \quotation{possible re-formation of CAM in the Caribbean} that Carifesta presented, and was told that his efforts met with the crushing indifference of the festival organizers:

\startblockquote
When I got there I was distressed to find that the organizers did not know anything about CAM, or didn't seem to care. . . . So eventually I persuaded Arthur Seymour to set up a CAM meeting, which we did have. . . . We set up this thing, and it was well attended, people were very keen. But then what happened is what I expected to happen. The Carifesta Secretariat felt that they would be the best people to deal with it, and it was therefore agreed that CAM would be run by {[}Frank{]} Pilgrim from the Carifesta desk at CARICOM and that was the end of it. We haven't heard of it since. And this is just what would happen, once you become involved with governments.\footnote{Walmsley, 278.}
\stopblockquote

The festival thus marked Brathwaite's \quotation{last attempt to re-form CAM in the Caribbean.}\footnote{Walmsley, 282.} As part of this work, he even contemplated writing \quotation{a book on the Caribbean Festival of Creative Arts} and was reportedly \quotation{busily engaged in doing research while being involved in the official Carifesta programme.}\footnote{\quotation{To Write a Book on Carifesta,} {\em Guyana Graphic}, 2 September 1972.} And although he may have enjoyed the symposium organized at the University of Guyana---the proceedings can be found in {\em Kaie}---he did \quotation{not feel that writers conferences . . . can achieve much since they are not rigorous enough to be concerned with ideas.}\footnote{Singh, \quotation{Poet Praises Carifesta}; Dolphin, \quotation{The Literary Vision of Carifesta '72.}} For Brathwaite, a festival like this should not just be entertaining, it should also educate the public. He wanted the writers to engage with ideas and educate their listeners. As he would tell an interviewer, \quotation{if the festival is to be educational and not just entertaining, some attempt must be made by the host country to present analyses and comparisons of the cultures of the participating countries.}\footnote{\quotation{To Write a Book on Carifesta.}}

Brathwaite's work on the festival in print, then, was his effort to make good on the promises that Carifesta extended to him as a public intellectual. This \quotation{festival public} as created through the newspaper articles continued and broadened the \quotation{counterpublic} he had been a part of with CAM. The mix of genres Brathwaite uses in each of the articles reflects the various purposes he thought the festival should have had and embodies the \quotation{festival public} he addresses in the newspaper. He recalls the conception of the festival, his journey and stay there, in a mode of straightforward reporting or travel writing. At the same time, he educates his readers on Carifesta's role in Caribbean history and cultural studies, using vivid examples from the festival itself. Along the way, he lays out his emerging theories of creolization and nation language. Finally, he interprets and reviews specific performances and art pieces in exhaustive depth. Brathwaite barely mentions his own contribution to the festival---he read {\em Rights of Passage} in full, as he had done at the inaugural session of CAM, as well as the poem \quotation{The Visibility Trigger,} along with providing a lecture on the biographical and historical arc of his poetic trilogy of {\em Rights}, {\em Masks}, and {\em Islands}, which was to be published a year later as {\em The Arrivants}.\footnote{\quotation{Session in Poetry Reading,} {\em Guyana Graphic}, 31 August 1972, https://dloc.com/CA00199881/00001; Marilyn Nichols-Agard, \quotation{Brathwaite's Lecture--Poetry All the Way,} {\em Sunday Chronicle}, 17 September 1972.}

This \quotation{festival public} remediated the faults of the real Carifesta and extended it beyond the immediate context of the events to give it a longer legacy; the way to get from one to the other was by way of print. It is significant, then, that Brathwaite chose to publish his chronicle over the course of eight weeks in the Barbadian daily newspaper, the {\em Advocate-News}. In the 1970s, Brathwaite did not publish much in local newspapers---if he did, it was poetry appearing in the Jamaican {\em Sunday Gleaner}. He preferred to publish critical prose in journals such as {\em Bim}, {\em Jamaica Journal}, {\em Caribbean Quarterly}, and {\em Savacou}. Most of his newspaper writing dates back to the early 1960s in the {\em Sunday Advocate} (the {\em Advocate} would merge with the {\em Daily News} in 1968 to become the {\em Advocate-News}) and the {\em Voice of St.~Lucia}, as he lived briefly on that island as a teacher.\footnote{See Josephs and Reid, \quotation{The Kamau Brathwaite Bibliography.}} Why might Brathwaite have chosen to write for the {\em Advocate-News}? He may have been sent there on assignment, ensuring his travel and pay over the three weeks, although he was already guaranteed those things as an honored participant in the festival. Perhaps Brathwaite felt that he was there as a representative of Barbados and therefore that he owed an account to his fellow islanders. This seems most likely, as he includes an entire article---\quotation{Carry on Big Inglan}---devoted to reviews of the Barbadian contributions.\footnote{Brathwaite, \quotation{Carifesta '72: Carry on Big Inglan.}} Whether Brathwaite was relatively famous among the newspaper readers of Barbados is hard to assess, but given the many months of reporting leading up to the festival---some employing the phrase {\em Carifesta time} to evoke the intense anticipation people feel about Carnival time---habitual readers may have turned to Brathwaite's columns with high interest.

The book plan was meant to address this, too, and would help \quotation{the people to have some rich and lasting impressions,} as opposed to the more ephemeral medium of newspaper print.\footnote{Brathwaite.} The fact that Brathwaite did produce a book on the occasion of Carifesta 1976 shows that he took the matter of archiving the festival and providing an educational component very seriously.\footnote{Edward Kamau Brathwaite, {\em Our Ancestral Heritage: A Bibliography} (1976; repr., Kingston: Savacou, 1977).} Yet the book on Carifesta '72 never came. Perhaps he had written lengthier manuscripts at some point, which may still be out there or which may have been destroyed or lost, but I am inclined to believe that the article series for the {\em Advocate-News} was the ultimate result of Brathwaite's book plan. To present these articles in the most widely read paper on his home island made his ideas more accessible than in book form. A book might have been produced either locally by his own Savacou press and circulated only in small print runs, or by a larger metropolitan publishing house, both of which would have made the work inaccessible to the very people he wanted to reach. If he could not achieve this \quotation{lasting impression} for a wide group of people, the compromise was to use the newspaper and to mitigate the medium's ephemerality by spreading the articles out over eight Sunday editions. In the articles, then, Brathwaite addresses a composite \quotation{festival public,} extending Carifesta time by several months and addressing readers in Barbados who had attended Carifesta and those who had not, while reminding them of the festival's wider transnational Caribbean audience.

The example of Brathwaite at Carifesta '72 shows us vividly how Caribbean criticism is forged in relation. What some have called \quotation{Caribbean method} often originates in events and occasions.\footnote{Kamugisha, \quotation{The Promise of Caribbean Intellectual History,} 53.} At Carifesta, Brathwaite confronted the \quotation{problem space} opened up by the Caribbean 1970s with his developing theories of \quotation{creolization,} finding in the festival an example of what it would mean to find---as Scott puts it, quoting Brathwaite---\quotation{a conceptual framework of Caribbean studies that combine{[}s{]} the \quote{social arts} with the \quote{social sciences.}}\footnote{Scott, \quotation{On the Question of Caribbean Studies,} 5.} His search for a methodological middle ground parallels his search for a mediating role as a radical public intellectual. On the ground during Carifesta, moving from the intellectuals and politicians at Burnham's 1970 Caribbean Writers and Artists Convention to the thousands of attendees in the Georgetown streets, Brathwaite began to formulate a culture concept that connected the ways the people of the Caribbean and the Americas might be \quotation{culturally interconnected from below}---in a submarine unity---while also explaining how they had been \quotation{socially and politically separated from above.}\footnote{Scott, 6.} Carifesta has been the understudied counterpart of other Pan-Caribbean moments in the nineteenth and twentieth centuries: not only the attempts at political unification of Ramón Emeterio Betances's Confederación Antillana and the West Indian Federation, for instance, but also the radical cultural and intellectual gatherings such as Havana '68 and Puerto Rico's 1952 Caribbean Festival of the Arts. Carifesta is unique in that it represents both varieties of the Pan-Caribbean impulse.\footnote{On the Pan-Caribbean impulse at work in mid-twentieth-century print publications, see Katerina Gonzalez Seligmann, {\em Writing the Caribbean in Magazine Time} (New Brunswick, NJ: Rutgers University Press, 2021).} By studying the institutions, publications, conferences, policies, and events---like the ones involved in the production of Carifesta---we can track the emergence of Caribbean literature and thought in the \quotation{middle-zone of cultural space} between minute \quotation{forms of close reading} and \quotation{general theories of cultural production} and history on a national or even global level.\footnote{James F. English, {\em The Economy of Prestige: Prizes, Awards, and the Circulation of Cultural Value} (Cambridge, MA: Harvard University Press, 2009), 12.} We might follow Brathwaite's own example in the articles I study here in our attempt to formulate at once a materialist and a poetic sense of culture. What would it look like to attempt an intellectual history of Caribbean thought as lived, performed, and institutionalized in public? What happens when we follow those thinkers and writers we so often bring together in our books, articles, and syllabi to the places where they gathered and exchanged both in the flesh and in print? This would not mean merely to reconstruct historical facts and restore descriptive thickness to such events; it would also require us to theorize occasional, public forms---such as public addresses, manifestos, occasional anthologies and translations, and solidarity statements---while following these texts as they circulate in different media and situations, as Brathwaite's works of the 1970s so often did.\footnote{See, for instance, Julie-Françoise Kruidenier, \quotation{Francophone Manifestos: On Solidarity in the French-Speaking World,} {\em International Journal of Francophone Studies} 12, no. 2--3 (December 2009): 271--87.}

In closing, I wish to offer some reflections on the kinds of problems any critic and literary historian writing about Carifesta might face: archival, ethical, and scalar. It is important to note that Carifesta happens to this day; its history is not over. The fifteenth edition is set to take place in 2022 in Antigua and Barbuda. With the growing popularity of regional literary festivals in the Caribbean, the history of Carifesta should have wide interest for contemporary writers and readers alike.\footnote{Ifeona Fulani, \quotation{Developing and Sustaining Literary Publics: Prizes, Festivals and New Writing,} in Donnell and Ronald Cummings, eds., {\em Caribbean Literature in Transition, 1970--2020}, vol.~3 (Cambridge: Cambridge University Press, 2021), 235--50. The website Caribbean Literary Heritage (https://www.caribbeanliteraryheritage.com/) lists twenty-five regional literary festivals and book fairs.} That the history of Carifesta '72 is yet to be written does not mean that its history does not exist. It exists in the memories and dreams of every participant---artists, writers, cultural policy bureaucrats, GUY-car drivers, translators, construction and service workers, journalists foreign and Caribbean, and many, many more people present Georgetown during those three weeks. Some of these have made it into the print and media record, but most have not. This is not to say that the archival records are not, by themselves, overwhelmingly voluminous; they are. dLOC's collaborations with libraries and archives around the Caribbean and worldwide are an invaluable accomplishment without which very little of this essay could have been written.

Yet even the heaps of newspaper articles, photographs, and print ephemera that dLOC has made available reveal gaps. Some of these gaps present obstacles to the very Pan-Caribbean, transnational aspirations that Carifesta has come to represent. I echo Roopika Risam's conclusion in {\em New Digital Worlds} that \quotation{the exclusions and biases that have characterized print culture}---so vividly portrayed by Brathwaite and so visibly at work in the events and conception of Carifesta---also \quotation{have been reproduced in the digital cultural record.}\footnote{Roopika Risam, {\em New Digital Worlds: Postcolonial Digital Humanities in Theory, Praxis, and Pedagogy} (Evanston, IL: Northwestern University Press, 2018), 139.} Even dLOC, with its \quotation{more than three million pages of content, over seventy institutional partners, and more than three million views each month,} bears traces of these inequities in its Carifesta collections.\footnote{Hélène Huet, Suzan Alteri, and Laurie N. Taylor, \quotation{Radical Collaboration to Improve Library Collections,} in Roopika Risam and Kelly Baker Josephs, eds., {\em The Digital Black Atlantic} (Minneapolis: University of Minnesota Press, 2021), 97.} The festival's roving nature means that no single institutional archive is associated with it---its organization has left traces all over the Caribbean, while participants and observers came not just from the region but also from North America, Latin America, Europe, and Africa---so the only way forward are collaborative collection-building and translation efforts. The only single centralized body that has had oversight over Carifesta is CARICOM (the Caribbean Community, a unified market and supranational body for coordinating economic and development policy), which has contributed most materials on the festivals to dLOC by working with the national libraries of governments that have hosted the festivals. Yet many perspectives and stories remain silenced as a result. Almost all of the documents in the Carifesta collection are in English, reflecting the predominantly anglophone membership of CARICOM, and most of the material associated with any particular festival comes from (print) media in the edition's host country. But Cuba, which hosted in 1979, has never been a member of CARICOM, and as a result there are only eight entries on the 1979 edition in dLOC, all from anglophone print publications. Not to mention the countless reports and stories that must have appeared in the media of participating countries from the entire circum-Caribbean---which, for Carifesta's purposes, includes diasporic communities around the world, much of Latin America, and the United States. The physical archive's privileging of print media has so far shaped the Carifesta collections as well. dLOC does not contain the various multimedia records (including photography, film, radio, and television) that were employed extensively during the lead-up to and as part of all Carifestas, with the most recent editions being widely documented on social media. Finally, the institutional archives on dLOC skew heavily toward the early, arguably peak years of the festival in the 1970s (including Barbados 1981), with only slim holdings from the 1990s revival of the festival to the editions of twenty-first century. Further collaboration and translation, along with research into physical archives around the world, may be one but certainly not the only way to preserve and honor the legacy of Carifesta.

\quotation{{[}W{]}e have to open our minds to partially knowing, and perhaps to recognizing our inability to know everything. Most importantly, we must respect the fact that we do not necessarily have a right to this knowledge. . . . Part of digitally archiving from the perspective of decolonization, therefore, is not only making our work available and accessible to others but also letting go of control---perhaps through collaboration---and letting go of perfection by allowing work to go out into the world in medias res.} Crowdsourcing, hosting multimedia archives, wider accessibility, collaborative criticism: these are the promises of the digital for a possible future archive of Carifesta. Yet any attempt to tell the story of Carifesta faces the obstacles encountered by Caribbean digital archivists generally: the scattered state of personal papers (in particular for diasporic artists); the lack of infrastructure and funding for libraries, universities, and archives in the region; the questions of translation and anglophone dominance; the difficulties with intraregional collaboration; and the (neo)colonial asymmetries and inequalities resulting in the Global North's overwhelming ownership of many of the most important Caribbean collections, which are often inaccessible in the region itself. Compound these with the fragmentariness and partiality that challenge attempts to recount any performance, any event of this scale, and the prospect becomes daunting. This challenge requires another approach, then, following Marlene L. Daut's recommendations in \quotation{Haiti @ the Digital Crossroads: Archiving Black Sovereignty.}\footnote{Marlene L. Daut, \quotation{Haiti @ the Digital Crossroads: Archiving Black Sovereignty,} {\em archipelagos}, no. 3 (July 2019), https://archipelagosjournal.org/issue03/daut.html.}

The refrain among the written recollections I've read (Salkey's, Clarke's, and Brathwaite's---there are doubtless others) is that this first edition of Carifesta was one to remember. Yet I am also reminded of something said of another epochal event, Festac '77: The Second World Black and African Festival of Arts and Culture in Lagos. According to Ntone Edjabe, who has interviewed participants for his experimental book on Festac, most \quotation{participants speak of it as a paradigm shift, one of the most important events they've attended,} while \quotation{it seldom appears as a full story: . . . the people who experienced Festac seemed unwilling to write it, as if bound by an unspoken nondisclosure agreement.}\footnote{Chimurenga, \quotation{Reproducing Festac '77: A Secret among a Family of Millions,} {\em Chimurenga}, accessed 29 January 2022, https://chimurengachronic.co.za/reproducing-festac-77-a-secret-among-a-family-of-millions/. See Chimurenga, {\em FESTAC} {\em '77: 2nd World Black and African Festival of Arts and Culture} (London: Afterall, 2019).} My hope is that we can respect the terms of such an agreement, while also recognizing that the history of Caribbean literature and culture in the twentieth century would doubtless be a richer, more accurate one for including Carifesta.

\thinrule

\page
\subsection{René Johannes Kooiker}

René Johannes Kooiker is a PhD student in the Department of Comparative Literature at Yale University. His main interests are twentieth-century Caribbean literature in English, Dutch, French, and Spanish. He is currently researching the Pan-Caribbean cultural and literary festival Carifesta along with its Pan-African counterparts, such as FESTAC '77 and PANAF 1969. His book reviews have been published in {\em Modern Language Quarterly} and the {\em Dutch Review of Books}.

\stopchapter
\stoptext