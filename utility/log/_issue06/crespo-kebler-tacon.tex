\setvariables[article][shortauthor={Crespo Kebler}, date={May 2022}, issue={6}, DOI={https://doi.org/10.7916/archipelagos-0208-s925}]

\setupinteraction[title={Colección digital El Tacón de la Chancleta, una mirada a los feminismos de la década de 1970 en Puerto Rico},author={Elizabeth Crespo Kebler}, date={May 2022}, subtitle={El Tacón de la Chancleta}, state=start, color=black, style=\tf]
\environment env_journal


\starttext


\startchapter[title={Colección digital El Tacón de la Chancleta, una mirada a los feminismos de la década de 1970 en Puerto Rico}
, marking={El Tacón de la Chancleta}
, bookmark={Colección digital El Tacón de la Chancleta, una mirada a los feminismos de la década de 1970 en Puerto Rico}]


\startlines
{\bf
Elizabeth Crespo Kebler
}
\stoplines


{\startnarrower\it \quotation{La chancleta está al mismo nivel del suelo y su uso está limitado a no salir de los confines de la casa. Es además un objeto de poco valor. Creemos que a esa chancleta le ha salido un tacón. {[}\ldots{}{]} Con el tacón se puede salir a la calle y se está un poco por encima del nivel del suelo. Pero su naturaleza sigue siendo la misma: una chancleta.} Con este adagio se lanza la publicación feminista {\em El Tacón de la Chancleta} para describir el momento histórico y la transformación social que desde sus páginas se propuso presentar. Surge en Puerto Rico en el 1974 con un número preliminar publicado en la revista {\em Avance}. En el 1975 se publican los cinco números de este periódico feminista que se encuentran en la Biblioteca Digital del Caribe (dLOC) junto a otros documentos que abren una ventana al contexto histórico y al alcance de la publicación más allá de Puerto Rico. Un análisis de los temas que se incluyen y los debates que se ventilan en sus páginas permite una mirada al devenir de los feminismos para ver las continuidades, rupturas y resignificaciones de los asuntos que le conciernen. Se destacan temas sobre el Año Internacional de la Mujer, la Primera Conferencia Mundial sobre la Mujer celebrada en México en 1975 y el feminismo como movimiento político, el aborto y el derecho al control de las mujeres sobre su cuerpo, la orientación sexual, el sexismo en los medios de comunicación, el currículo escolar no sexista, la violencia contra las mujeres y el arte feminista. Forma parte de la Colección Documentos del Feminismo en Puerto Rico.

 \stopnarrower}

\blank[2*line]
\blackrule[width=\textwidth,height=.01pt]
\blank[2*line]

{\em El Tacón de la Chancleta} forma parte de la Colección Documentos del Feminismo Rivera Lassén--Crespo Kebler, que contiene sobre cuatro mil documentos de más de 60 organizaciones feministas del periodo de 1970 a 2010. Trabajamos para incorporarla en su totalidad a dLOC para el uso de estudiantes, personas estudiosas de los feminismos, interesadas en elaborar políticas públicas así como para el público general. No existe otra colección como ésta. Son documentos que las autoras de la colección hemos recopilado como resultado de nuestro activismo y donaciones de otras activistas feministas. dLOC es el espacio idóneo para esta colección por la visibilidad y permanencia que ofrece como repositorio digital. Es importante mencionar además que hay otras colecciones documentales de la Universidad de Puerto Rico sobre periodos previos al 1970 que se encuentran en dLOC y son importantes referencias para las historias de los feminismos en Puerto Rico. Destacamos la colección del periódico {\em El Mundo}\footnote{Uno de los temas que los volúmenes del periódico {\em El Mundo} de 1928 a 1939 contribuyen a documentar es el acceso a los métodos contraceptivos así como la oposición de la Iglesia Católica y el Partido Nacionalista.}, y la Biblioteca Digital Puertorriqueña. Esta última contiene un archivo único en su clase de documentos sobre la sufragista Ana Roqué de Duprey, figura que, como veremos, inspiró a las editoras del {\em Tacón de la Chancleta}\footnote{Contiene una colección única de documentos sobre Ana Roqué de Duprey figura que inspiró a las editoras del {\em Tacón de la Chancleta} por ser una de las protagonistas del movimiento feminista sufragista que las precedieron. Además de la lucha por el derecho al voto, otros temas que distinguieron a este movimiento fueron la lucha por mejores condiciones de trabajo y la educación como derecho para las mujeres. Ana Roqué fue autora de novelas y cuentos y escribió no solamente sobre los derechos de las mujeres, sino también sobre astronomía, filosofía, geografía y botánica.}.

El periódico feminista {\em El Tacón de la Chancleta} comenzó a circular con un número preliminar que se publicó en la revista de noticias semanal {\em Avance} del 30 de septiembre de 1974\footnote{{\em Avance} fue una revista de discusión política que se publicó del 1972 al 1975 en San Juan de Puerto Rico. Cubrió en sus páginas varios artículos sobre las organizaciones feministas y otros movimientos, de los que se destacan los siguientes. El dosier sobre la homosexualidad tiene entrevistas que incluyen a integrantes de La Comunidad de Orgullo Gay, organización creada para luchar por los derechos de las personas homosexuales y lesbianas, y para derogar el artículo del Código Penal de Puerto Rico que castigaba el \quotation{Infame crimen contra natura} con uno a diez años de cárcel. {\em Avance}, 2 de septiembre de 1974, 10--17. Dossier sobre concursos de belleza por Ada Nivea Guerra, \quotation{Concursos de Belleza: Miss, Sí; Miss, No...}, {\em Avance}, 20--26 de julio de 1972, 25--30; \quotation{En la silla de los acusados: La mujer se defiende de los cargos que le formulan}, {\em Avance}, 3 de marzo de 1975, 36--39.}. Además del número preliminar, se publicaron cinco números más, todos en el año 1975. Con la mirada del presente podemos afirmar que esta publicación fue emblemática de lo que se ha llamado la segunda ola de los feminismos en América Latina y el Caribe. También fue pionera en lo que se llamaría periodismo feminista alternativo.

\subsection[title={Panorama político y contexto histórico del {\em Tacón de la Chancleta}},reference={panorama-político-y-contexto-histórico-del-tacón-de-la-chancleta}]

Para comprender las aportaciones del {\em Tacón}, comienzo con una descripción del panorama político de Puerto Rico y lo que se ha llamado la segunda ola del feminismo que tomó forma en muchas partes del mundo en las décadas de 1960 al 1980. Surge a nivel mundial como oleada de reclamos luego de las conquistas del derecho al voto que ocurrieron desde comienzos del siglo XX: 1920 en Estados Unidos, 1929 en Puerto Rico y Ecuador, 1932 en Brasil y República Dominicana, 1947 en Argentina y Venezuela, y 1964 en Bahamas, para mencionar algunos lugares en nuestro hemisferio geográfico\footnote{Ana María Bigegain, \quotation{La obtención del sufragio femenino en los estados latinoamericanos, avances y ambigüedades (1917--1961)}, en {\em Mujer, nación, identidad y ciudadanía: siglos XIX y XX (IX Cátedra Anual de Historia Ernesto Restrepo Tirado, 28 al 30 de octubre de 2004)} (Bogotá: Ministro de Cultura, 2005).}. Así como el movimiento sufragista fue uno global, los movimientos feministas de la segunda ola también lo fueron\footnote{Robin Morgan, ed., {\em Sisterhood Is Global} (New York: Feminist Press at the City University of New York, 1996); Caroline Daley, {\em Suffrage and Beyond: International Feminist Perspectives} (Auckland, New Zealand: Auckland University Press, 1994).}. La segunda ola del feminismo estuvo acompañada de cambios significativos en las condiciones de vida de las mujeres como el aumento en la participación en la fuerza laboral, cambios en la estructura de las familias y en los modelos de feminidad\footnote{Yasmine Ergas, \quotation{El sujeto mujer: el feminismo de los años sesenta-ochenta}, en Georges Duby y Michelle Perrot, eds., {\em Historia de las mujeres: el siglo XX, la nueva mujer} (Barcelona: Taurus, 1993), 154--81.}. El proceso del movimiento feminista de Puerto Rico se adelantó a la región de América Latina y el Caribe por lo que muchas de sus conquistas no fueron una realidad en el resto de la región hasta fines de la década de 1980. Por ejemplo, el derecho al aborto, la creación de entidades gubernamentales como la Comisión para el Mejoramiento de los Derechos de la Mujer (1973), el Centro de Ayuda a Víctimas de Violación (1977), así como entidades no gubernamentales para ofrecer servicios para las mujeres y albergues para mujeres víctimas de violencia doméstica\footnote{Magaly Pineda, introducción a Ana Irma Rivera Lassén y Elizabeth Crespo Kebler, eds., {\em Documentos del feminismo en Puerto Rico: facsímiles de la historia}, vol.~1, {\em 1970--1979} (San Juan: Editorial de la Universidad de Puerto Rico, 2001), xiv; Magaly Pineda, \quotation{The Spanish-Speaking Caribbean: We Women Aren't Sheep}, en Morgan, {\em Sisterhood Is Global}, 131--34; Marjorie Agosín, \quotation{Chile: Women of Smoke}, en Morgan, {\em Sisterhood Is Global}, 138--41.}. Sin embargo, la actividad feminista en Puerto Rico de esas primeras décadas se desconoce o no se ha reconocido en la región de América Latina y el Caribe.

En Puerto Rico, el carácter global y transnacional de los feminismos\footnote{Anneeth Kaur Hundle, Ioana Szeman y Joanna Pares Hoare, \quotation{What Is the Transnational in Transnational Feminist Research?}, {\em Feminist Review} 121, no. 1 (marzo 2019): 3--8.} estuvo matizado por varios factores: el estatus legal de Puerto Rico como país no independiente o territorio no-incorporado de los Estados Unidos, el movimiento constante de personas entre Estados Unidos y Puerto Rico y los ámbitos de acción política entre distintos espacios geopolíticos de Norteamérica, América Latina, el Caribe y Europa. El carácter transnacional obliga a pensar los feminismos en Puerto Rico en su relación con el colonialismo, el nacionalismo y el patriarcado así como sus efectos sobre las mujeres y los asuntos relacionados al género y la sexualidad.

El estatus legal de Puerto Rico como un territorio no-incorporado de los Estados Unidos es complejo creando tanto aislamiento como puntos de encuentro con otras partes del mundo. Puerto Rico no tiene un asiento en las Naciones Unidas y tampoco tiene acceso igual a recursos económicos internacionales. Por razones históricas, geográficas y lingüísticas, la historia de los movimientos de mujeres en Puerto Rico no es considerada en los Estados Unidos como parte del feminismo nacional. Se añade a las fuerzas de aislamiento el hecho de que contrario al conocimiento común, las decisiones judiciales de los Estados Unidos no aplican de manera automática a Puerto Rico. Ejemplos de esto son el derecho al voto y el derecho al aborto, que tienen una historia distinta a la de Estados Unidos\footnote{Ana Irma Rivera Lassén, \quotation{Del dicho al derecho hay un gran trecho o el derecho a tener derechos: decisiones del Tribunal Supremo de Puerto Rico ante los derechos de las mujeres y de las comunidades LGBTTI}, {\em Revista Jurídica UIPR} 44, no. 1 (agosto--mayo de 2009--10): 39--68.}. Por otro lado, el estatus territorial de Puerto Rico vincula los feminismos puertorriqueños con los Estados Unidos, con América Latina y el Caribe de otras maneras porque para promover sus demandas deben hacerlo dentro del contexto de los mecanismos de participación regionales tales como las conferencias mundiales promovidas por las Naciones Unidas, los foros regionales de la Organización de Estados Americanos, la Comisión Económica para la América Latina y la Comisión Interamericana de Derechos Humanos, entre otros. La participación desde los espacios de sociedad civil en dichas instancias internacionales ha ayudado a que sea en la región de América Latina y el Caribe donde más cercanía y presencia ha tenido Puerto Rico, como explicamos a continuación.

Los vínculos transnacionales de los feminismos ocurren también a través del movimiento constante de personas entre Estados Unidos y Puerto Rico y el activismo junto a otros grupos migrantes que también se mueven a y desde los Estados Unidos y entre países de la región\footnote{Elizabeth Crespo Kebler, \quotation{Las Buenas Amigas}, en \quotation{Revisiting Puerto Rican Queer Sexualities}, número especial del {\em Centro Journal of the Center for Puerto Rican Studies} 30, no. 2 (verano de 2018): 378--405.}. Ya que la historia, idioma e identidad nacional de Puerto Rico es Latinoamericana y Caribeña, las organizaciones feministas y de mujeres han mirado a esta región. Las organizaciones feministas no solamente se proyectaron dentro del país, sino que fundaron y formaron parte de redes feministas y de derechos humanos, encuentros feministas, eventos feministas y gubernamentales internacionales, organismos regionales y globales. A través de este activismo, las feministas puertorriqueñas se insertaron en el mundo superando los limitados ámbitos de participación del gobierno de Puerto Rico. Esta participación internacional arroja luz a la compleja relación política entre Puerto Rico y los Estados Unidos, la geopolítica de América Latina y el Caribe y las luchas por los derechos humanos de las mujeres.

El movimiento feminista que surge en América Latina y el Caribe a partir de 1970 se constituyó también como un movimiento diverso. Unas se insertaron en espacios políticos tradicionales como el Estado, los sindicatos, los partidos políticos, y las organizaciones internacionales. Cabe destacar que muchos países de la región tenían y enfrentaban situaciones políticas dictatoriales, por lo que la relación con los Estados de los grupos de mujeres fue muy desigual en las décadas anteriores a los años 1990. Algunas activistas feministas en estos países habían sido antes perseguidas políticas, e incluso exiliadas en algún momento. Como en otras partes de América Latina y el Caribe, en Puerto Rico también surgen organizaciones autónomas formadas fuera de esos espacios tradicionales con el fin de incidir sobre los cambios culturales, sociales, políticos y económicos que se requerían para promover la equidad y los derechos de las mujeres. {\em El Tacón de la Chancleta} y la organización Mujer Intégrate Ahora (MIA) fueron parte de este movimiento feminista autónomo\footnote{Rivera Lassén y Crespo Kebler, {\em Documentos del feminismo en Puerto Rico}; Norma Mogrovejo, {\em Un amor que se atrevió a decir su nombre} (México, DF: Plaza y Valdés, 2000; Magdalena León de Leal, ed., {\em Mujeres y participación política: avances y desafíos en América Latina} (Bogotá: TM, 1994).}.

\subsection[title={Mujer Intégrate Ahora y {\em El Tacón de la Chancleta}},reference={mujer-intégrate-ahora-y-el-tacón-de-la-chancleta}]

Mujer Intégrate Ahora fue la primera organización feminista autónoma creada en Puerto Rico a partir la década de 1970. Las mujeres que formaron MIA se reunieron primero bajo el nombre de Comité de Mujeres Puertorriqueñas. Respondían a una convocatoria que hizo Nilda Aponte entre las asistentes a las vistas públicas celebradas en Puerto Rico en el 1971 donde se discutía un informe de la Comisión de Derechos Civiles de Puerto Rico que documentaba el discrimen contra las mujeres en todos los ámbitos de la sociedad puertorriqueña. Este era un evento muy significativo, pues apenas un año antes se había publicado otro informe de un comité nombrado por el gobernador de Puerto Rico donde se afirmaba que no había discrimen contra las mujeres. Esta falta de reconocimiento del discrimen era característica del momento político de principios de la década de 1970. Los espacios para las mujeres en el ámbito tradicional de la política eran muy limitados. Pocas mujeres eran electas a puestos públicos y la jerarquía de los partidos, los sindicatos y organizaciones políticas estaba dominada por los hombres. Esta situación de invisibilidad se aprecia en voz de una de las fundadoras de MIA: \quotation{Para esa época yo leía todo lo que encontraba del tema de las mujeres y como no habían prácticamente historias escritas de las luchas en Puerto Rico de las sufragistas y de las feministas de principios de siglo, me dediqué a ir en mi tiempo libre a leer en la Biblioteca de la Universidad de Puerto Rico y la Biblioteca del Ateneo}\footnote{Ana Irma Rivera Lassén, \quotation{La Organización de las Mujeres y las organizaciones feministas en Puerto Rico: Mujer Intégrate Ahora y otras historias de la década}, en Rivera Lassén y Crespo Kebler, {\em Documentos del feminismo en Puerto Rico}, 106.}. Nos narra que su pasión encontró eco y compañía en otras mujeres, tres puertorriqueñas y dos norteamericanas que tenían el mismo interés y querían impulsar el movimiento feminista en Puerto Rico. Constituyeron formalmente a Mujer Intégrate Ahora en enero de 1972 con un documento que describía sus propósitos y objetivos y un reglamento para la membresía.

El editorial del número preliminar del {\em Tacón de la Chancleta} explica el origen de la publicación. \quotation{Un grupo de mujeres dentro de la organización Mujer Intégrate Ahora (MIA), al revisar la historia del feminismo en Puerto Rico, encontró una tradición histórica donde se publicaron varias revistas y periódicos dirigidos a promover los derechos de la mujer}\footnote{{\em El Tacón de la Chancleta}, suplemento especial, {\em Avance}, 30 de septiembre de 1974, 29--41.}. Acorde con el propósito de visibilizar la historia feminista, en las páginas del {\em Tacón} se reseñaron las publicaciones locales e internacionales de las sufragistas puertorriqueñas, en especial las de Ana Roqué de Duprey y sus colaboradoras: {\em Euterpe} (1888), {\em La Mujer} (1894), {\em La Evolución} (1902), {\em El Álbum Puertorriqueño} (1918), {\em La Mujer del Siglo XX} (1917) y {\em El Heraldo de la Mujer} (1919)\footnote{Ana I. Rivera Lassén, \quotation{Doña Ana Roqué de Duprey precursora del movimiento sufragista en Puerto Rico}, {\em El Tacón de la Chancleta}, julio--agosto de 1975, 8--9. La investigación sobre esta figura presentada en las páginas del {\em Tacón de la Chancleta} les dirigirá a otras figuras y organizaciones, algunas de los que forman parte de la Biblioteca Digital Puertorriqueña y otras colecciones en dLOC. Entre ellas, la historia de la Liga Femínea Puertorriqueña promotora del derecho al voto para las mujeres, el National Woman´s Party y su sección en Puerto Rico, la Liga Social Sufragista asociada al International Alliance of Women, la Asociación de Mujeres Sufragistas y la Asociación Insular de Mujeres Votantes. Asimismo, dirigirá la atención a otras destacadas figuras sufragistas puertorriqueñas: Ricarda López de Ramos, Dra. Marta Robert de Romeu, Muna Lee, Pilar Barbosa, Ricarda Ramos Casellas, Milagros Benet Newton, Beatriz Lassalle e Isabel Andreu de Aguilar (primera mujer de la Junta de Gobierno de la UPR y presidenta de la Liga de Mujeres Votantes).}.

\placefigure[here]{Editoras {\em Tacón de la Chancleta}}{\externalfigure[issue06/valence_1_eck_editoras-tacon.jpg]}


Según exponen en ese primer editorial, iniciaron el periódico feminista con la idea de que pudiera venderse y mercadearse junto a las publicaciones comerciales. Querían marcar una diferencia con las publicaciones femeninas de la época que reforzaban los patrones de sumisión e inferioridad de las mujeres\footnote{Rivera Lassén y Crespo Kebler, {\em Documentos del feminismo en Puerto Rico}, 122--23; \quotation{¿Por qué?}, {\em El Tacón de la Chancleta}, suplemento especial, {\em Avance}, 30 de septiembre de 1974, 2.}. Gloria Steinem, figura muy visible en los medios de comunicación en los Estados Unidos, feminista renombrada internacionalmente y editora de la revista {\em Ms.}, visitó Puerto Rico en junio de 1974 e hizo un donativo para ayudar a financiar la publicación del {\em Tacón de la Chancleta}.

\startblockquote
La chancleta está al mismo nivel del suelo y su uso está limitado a no salir de los confines de la casa. Es además un objeto de poco valor. Creemos que a esa chancleta le ha salido un tacón. {[}. . .{]} Con el tacón se puede salir a la calle y se está un poco por encima del nivel del suelo. Pero su naturaleza sigue siendo la misma: una chancleta. Esta publicación surge de la necesidad de concientizar a las mujeres y sacar a relucir las situaciones por las cuales injustamente se nos ha llamado y se nos trata como chancletas.\footnote{\quotation{¿Por qué {\em El} {\em Tacón de la Chancleta}?}, {\em El Tacón de la Chancleta}, enero de 1975, \useURL[url1][https://ufdc.ufl.edu/AA00070290/00001]\from[url1].}
\stopblockquote

Con este adagio se lanza {\em El} {\em Tacón de la Chancleta} para describir el momento histórico y la transformación social que desde sus páginas se propuso presentar.

Según afirman las fundadoras del {\em Tacón}, aunque {\em El Tacón de la Chancleta} pretendió ser una publicación independiente, sus editoriales y muchos de sus artículos reflejaban las posiciones de MIA. Esto se corrobora al examinar la publicación oficial de la organización, titulada {\em MIA Informa}\footnote{Rivera Lassén y Crespo Kebler, {\em Documentos del feminismo en Puerto Rico}, 218--45.}. Una mirada a la prensa y revistas de Puerto Rico así como a los documentos de la organización reflejan el impacto que tuvieron las actividades de la organización y las posiciones tomadas sobre temas como el aborto, la imagen de la mujer en los medios de comunicación, la igualdad en el matrimonio, la igualdad en el crédito y su participación en la economía del país. También trajeron a la palestra pública la falta de textos escolares que mostraran a las mujeres en el quehacer público y privado del país, las licencias por maternidad, los centros de cuidado infantil, la prostitución, los concursos de belleza, la homosexualidad y el lesbianismo. Produjeron análisis de las plataformas de los partidos políticos, sus propuestas y promesas para las mujeres para así incidir en las decisiones de políticas públicas.

Un análisis de los temas que se discutieron en las páginas del {\em Tacón} y los debates que allí se ventilaron es una ventana al pasado que contribuye a analizar el devenir de los feminismos mirando las continuidades, rupturas y resignificaciones de los asuntos que allí se presentaron.

\placefigure[here]{{\em El Tacón de la Chancleta}}{\externalfigure[issue06/valence_1_eck_foto2.jpg]}


\subsection[title={El feminismo como movimiento político},reference={el-feminismo-como-movimiento-político}]

La relación entre el feminismo y las organizaciones socialistas e independentistas provocó importantes debates en los feminismos de América Latina y el Caribe. Al igual que en otros países, en Puerto Rico la influencia de los esquemas teóricos de las izquierdas establecía una jerarquía de prioridades donde la lucha en contra del capitalismo tenía prioridad sobre la lucha en contra de la opresión de las mujeres\footnote{Discuto la relación entre las izquierdas y los feminismos en \quotation{Liberación de la Mujer: los feminismos, la justicia social, la nación y la autonomía en las organizaciones feministas de la década de 1970 en Puerto Rico}, en Rivera Lassén y Crespo Kebler, {\em Documentos del feminismo en Puerto Rico}, 39--95. Véase también \quotation{Entrevistas a Norma Valle Ferrer y Flavia Rivera Montero}, en Rivera Lassén y Crespo Kebler, {\em Documentos del feminismo en Puerto Rico}, 151--78.}.

Mujer Intégrate Ahora estuvo al centro de este debate porque desde sus inicios rechazó identificarse con los partidos políticos y no abogó por ningún estatus político para Puerto Rico (ni la independencia, ni la estadidad, ni la libre asociación). Estuvo a favor de temas controversiales como el aborto, el amor libre y una educación no sexista. La justicia social y los derechos humanos, particularmente los de la mujer, era su norte\footnote{Rivera Lassén y Crespo Kebler, {\em Documentos del feminismo en Puerto Rico}, 218--19.}.

Desde la fundación de MIA en 1972, estas posturas provocaron reacciones de muchos sectores. Por un lado, los medios de comunicación masivos acusaban a Mujer Intégrate Ahora de ir contra la familia y la cultura puertorriqueña. Por otro lado, sectores independentistas y socialistas las tildaban de burguesas y víctimas de ideas extranjeras que venían de Estados Unidos. En el periódico {\em Claridad}, vocero del Partido Socialista Puertorriqueño, Lolita Aulet, una destacada líder de ese partido, escribió una columna titulada \quotation{Reformismo versus revolución}. Argumentaba que \quotation{las feministas, influenciadas por la deformación y falta de información, dirigen sus campañas proclamando a los hombres sus enemigos y pretenden contrarrestar la opresión de que son víctimas con una persistente negativa a tener hijos y otras formas de protesta vanas.} \quotation{Si bien la demanda de igualdad para la mujer es democrática, la exigencia de su liberación total no puede ser sino socialista.}\footnote{Lolita Aulet, \quotation{Reformismo versus revolución}, {\em Claridad}, 17 de abril de 1973, 11.}

En respuesta a la columna de Aulet, Ana Irma Rivera Lassén, líder de MIA, publicó un artículo, también en el periódico {\em Claridad}, donde argumentó que las ideas de liberación femenina no solo chocan contra la muralla levantada por el sistema imperialista y el subdesarrollo, sino también contra la muralla del machismo. Denunció que se ve la liberación de la mujer como la asimilación a patrones culturales ajenos a la cultura puertorriqueña. Para Rivera Lassén, el problema de la mujer era un problema de la humanidad y parte esencial de la liberación del ser humano que involucraba tanto a hombres como mujeres. Para Rivera Lassén, la complicidad con el machismo dentro de los partidos socialistas se manifestaba en la falta de participación igual de las mujeres, motivo de denuncia por las propias mujeres que militaban en ese partido\footnote{Ana Irma Rivera Lassén, \quotation{Un debate: La liberación femenina}, {\em Claridad}, 3 de junio de 1973, 14.}.

La fundación en 1975 de la organización feminista Federación de Mujeres Puertorriqueñas (FMP), provocó un debate sobre la autonomía de las organizaciones feministas y su independencia organizativa e ideológica de los movimientos de izquierda. Desde las páginas del {\em Tacón de la Chancleta} se presentó una fuerte crítica al planteamiento de la FMP de que la liberación de la mujer se daría solo después de cambiar las estructuras económicas y políticas capitalistas y coloniales. También llamó la atención al miedo y resistencia de la organización a usar la palabra {\em feminismo}\footnote{\quotation{Editorial}, {\em El Tacón de la Chancleta}, febrero de 1975, 2, https://ufdc.ufl.edu/AA00070290/00002.}.

El editorial del tercer número del {\em Tacón}, publicado en marzo--abril de 1975, estuvo dedicado a articular su visión del feminismo como movimiento político. El siguiente fue uno de sus planteamientos claves: \quotation{Los movimientos por los derechos de la raza negra, grupos étnicos, clase trabajadora, orientación y preferencia sexual, grupos estudiantiles, liberación nacional, juntos tienen la fuerza suficiente para lograr cambios en las sociedades. Pero para lograr un verdadero cambio hay que terminar no solo con la opresión y el poder de unos sobre otros sino con la idea misma de la opresión y el poder. La conciencia feminista es imprescindible si verdaderamente se quiere lograr un cambio real en la sociedad porque la opresión de la mujer es algo común en todos ellos.}\footnote{\quotation{Editorial}, {\em El Tacón de la Chancleta}, marzo--abril de 1975, 2, https://ufdc.ufl.edu/AA00070290/00003.}

Esta visión se amplía en el escrito de Nilda Aponte Raffaele publicado en el {\em Tacón}, \quotation{Liberación humana liberación femenina}, donde argumentó que \quotation{la liberación femenina es esencial a la verdadera liberación humana}\footnote{Nilda Aponte Raffaele, \quotation{Liberación humana liberación femenina}, {\em El Tacón de la Chancleta}, julio--agosto de 1975, 10, https://ufdc.ufl.edu/AA00070290/00005.}. Decía que algunos insisten en que la liberación de la mujer debe quedar subordinada a otras luchas y debe quedar pendiente hasta que se logre la justicia social, independencia o un mundo mejor. Argumentó que las luchas se entrelazan y no se autoexcluyen, se complementan, tiene cada una su razón de ser aun cuando algunas veces no pueden existir unidas. \quotation{Cuando como individuos nos sentimos obligadas a reconocer más de una lucha podemos hacerlo sin tener que decidir si es la una compatible con la otra. Lo que tenemos que decidir es a cuál lucha nos hemos de entregar, no por ser ésta la más importante necesariamente, sino porque en estos momentos esa lucha responde mejor a nuestras circunstancias, nuestras necesidades, habilidades y mayor urgencia personal.}\footnote{Aponte Raffaele, \quotation{Liberación humana liberación femenina}, 10.} Aponte Raffaele y Rivera Lassén, ambas mujeres negras, articularon una visión que iba más allá del binario reformismo o revolución a un feminismo situado en sus experiencias y las formas en que sus vidas estaban atravesadas por varias formas de opresión de manera simultánea. Esta perspectiva la articuló desde otro espacio geográfico el Combahee River Collective en su ensayo \quotation{A Black Feminist Statement} en 1977\footnote{Combahee River Collective, \quotation{A Black Feminist Statement} (1977), in Linda Nicholson, ed., {\em The Second Wave: A Reader in Feminist Theory} (New York: Routledge, 1997), 63--70.}.

Una mirada histórica permite apreciar que adelantaron también una visión de lo que en la década de 1980 y 1990 se llamó interseccionalidad, que más allá de sumar formas de opresión propone hacer visibles las particularidades de las desigualdades de género cuando interactúan con raza, etnia, clase social, orientación sexual y otras categorías de exclusión\footnote{Kimberlé Crenshaw, \quotation{La intersección de raza y género}, en Celina Romany, ed., {\em Raza, etnicidad, género y derechos humanos en las Américas: Un nuevo paradigma para el activismo} (San Juan: Publicaciones REG, 2004), 127--39; Celina Romany, \quotation{Tema de conversación sobre raza y género en el derecho internacional en materia de derechos humanos}, en Romany, {\em Raza, etnicidad, género y derechos humanos en las Américas}, 121--26.}.

\subsection[title={El Año Internacional de la Mujer y la Primera Conferencia Internacional de la Mujer},reference={el-año-internacional-de-la-mujer-y-la-primera-conferencia-internacional-de-la-mujer}]

La primera celebración del día 8 de marzo, Día Internacional de la Mujer, llevada a cabo por una organización autónoma feminista en Puerto Rico la hizo MIA en el 1974. Esto aún antes de que la Organización de las Naciones Unidas (ONU) reconociera la fecha como tal. MIA conmemoró el día con gran acogida en el centro comercial Plaza las Américas. Luego continuó celebrándolo y así lo publica en el primer número del {\em Tacón de la Chancleta} al inicio de 1975, Año Internacional de la Mujer proclamado por la ONU. El Año Internacional de la Mujer y la convocatoria a la Primera Conferencia Internacional de la Mujer, celebrada en México en el 1975 bajo los auspicios de la ONU, proyectó a MIA y {\em El Tacón de la Chancleta} fuera de Puerto Rico. También trajo a Puerto Rico la atención al feminismo como fenómeno global. El debate sobre el feminismo revolucionario y el reformista, el feminismo burgués o el proletario, se daba también entre grupos feministas a nivel global. Esto se recoge en las páginas del {\em Tacón de la Chancleta} con relación a la Primera Conferencia Internacional de la Mujer. El número del {\em Tacón de la Chancleta} de julio--agosto de 1975 se dedica a cubrir esta conferencia.

El evento según organizado por la ONU tenía por un lado una Conferencia Gubernamental y otra reunión paralela para las organizaciones no gubernamentales denominada la Tribuna del Año Internacional de la Mujer. En la Conferencia Gubernamental los Estados miembros se comprometerían con el plan de acción mundial para la consecución de los objetivos del Año Internacional de la Mujer, que contenían directrices para los avances de las mujeres hasta 1985. En la Conferencia Gubernamental se reconoció el papel subordinado de las mujeres en la vida social, económica y política de todos los países. No obstante, Rivera Lassén toma nota de que se eligió a un hombre como presidente de la Conferencia Gubernamental y muchas de las participantes eran esposas de algún hombre importante. \quotation{Así transcurrió y concluyó en el país que más se conoce por su machismo, y bajo la presidencia de un hombre, la Conferencia del Año Internacional de la Mujer.}\footnote{Ana Irma Rivera Lassén, \quotation{Conferencia Mundial de la Mujer, el feminismo se quedó en la aduana}, {\em El Tacón de la Chancleta}, julio--agosto de 1975, 4--5, https://ufdc.ufl.edu/AA00070290/00005.}

La conferencia paralela, la Tribuna, estaba destinada a recoger las preocupaciones de las feministas de los diferentes países que acudían en su carácter individual o representando a una organización no gubernamental. Aquí estaban representadas mujeres conservadoras que proponían incorporar más mujeres en puestos de liderato, más mujeres policías, más mujeres en el ejército y en estructuras que promovían el estatus quo. Estaban también las mujeres que decían que los problemas de las mujeres eran los mismos que los de los hombres porque sus únicos enemigos eran el imperialismo, el capitalismo y el colonialismo. Para algunas, el movimiento de liberación femenina dividía a los hombres y a las mujeres. Otras argumentaban que los problemas de las mujeres debían discutirse sin entrar en la política. La división entre las mujeres latinoamericanas por un lado, y por otro lado, las norteamericanas, europeas y africanas se dejó sentir por el poco espacio que se les otorgó a las latinoamericanas en el programa oficial y la cobertura desmedida de la prensa a las norteamericanas. Delegaciones de mujeres indígenas mexicanas fueron excluidas de la Tribuna mientras sus espacios fueron ocupados por las miembras del partido de gobierno mexicano Partido Revolucionario Institucional.

En su columna del {\em Tacón de la Chancleta}, Rivera Lassén resume de la siguiente manera el estado de situación frente a este panorama. Siguiendo el paradigma de la época donde los grupos proponían identificar un feminismo \quotation{verdadero}, dice lo siguiente: \quotation{El feminismo se quedó en la aduana. . . Lo que pasó allí fue que precisamente no hubo feminismo. Si hubo algún triunfo, fue el de los que se dedican a desprestigiar el movimiento feminista. No existen varios tipos de feminismo, solo es uno, que bien entendido es el que lucha por los derechos de la mujer y por una sociedad más justa.}\footnote{Rivera Lassén, \quotation{Conferencia Mundial de la Mujer}, 5.} En las décadas posteriores, el debate no sería por identificar el feminismo verdadero, sino reconocer la pluralidad de los feminismos.

\subsection[title={El aborto y derecho al control de las mujeres sobre su sexualidad y su cuerpo},reference={el-aborto-y-derecho-al-control-de-las-mujeres-sobre-su-sexualidad-y-su-cuerpo}]

La visión del feminismo propuesta en las páginas del {\em Tacón de la Chancleta} en torno a la salud y los cuerpos de las mujeres generaba críticas por considerar que eran temas reformistas o secundarios a los temas obreros, sindicales o anti-imperialistas. En palabras de Aulet reseñadas arriba, el feminismo que llamaba reformista proclamaba a los hombres sus enemigos y se caracterizaban por \quotation{una persistente negativa a tener hijos y otras formas de protesta vanas}.

Uno de estos temas catalogados como reformista era el aborto, que desde el 1903 ha sido legal en Puerto Rico para proteger la vida o la salud de la madre, lo que se conoce como el aborto terapéutico. Antes de la decisión del Tribunal Supremo de Estados Unidos en {\em Roe v. Wade} (1973), cuando en los Estados Unidos el aborto era ilegal, Puerto Rico era uno de los lugares donde las mujeres acudían para obtener abortos legales y seguros. Desde su fundación en el 1972, Mujer Intégrate Ahora abogó por el derecho al aborto a petición de la mujer que ampliaba el derecho consignado en el aborto terapéutico. {\em Roe v. Wade} otorgó el aborto a petición como parte del derecho a la intimidad. Bajo este concepto, la decisión era de las mujeres. En el 1974, la Corte de Distrito Federal de los Estados Unidos en Puerto Rico reconoció la aplicabilidad de {\em Roe v. Wade} y también el derecho de las mujeres a decidir sobre el aborto bajo la protección de la Constitución de Puerto Rico\footnote{{\em Acevedo Montalvo vs Hernández Colón}, 377 Federal Sup. 1332, 1974.}.

El ensayo en el número preliminar del {\em Tacón} titulado \quotation{¿Y los aborteros dónde están?} expone la disyuntiva presentada en la decisión de la Corte Federal del año 1974\footnote{Ronnie Lovler, \quotation{¿Y los aborteros dónde están?}, {\em El Tacón de la Chancleta}, ejemplar preliminar, {\em Avance}, 30 de septiembre de 1974, 12.}. Aquí se dictaminó por un lado que los hospitales públicos debían abrir sus facilidades a las mujeres que solicitaran abortos, y por otro lado que no se podía obligar a ningún doctor o doctora a practicar un aborto si sus conciencias así se lo dictaran. Así las cosas, los hospitales públicos se ampararon en la libertad de conciencia para obstaculizar los intentos de las mujeres a obtener un aborto en una facilidad pública. La pregunta que hace el título del artículo apunta al afán de lucro y las presiones desde el gobierno para imponer políticas que violentaban el derecho de las mujeres a decidir sobre sus cuerpos y sus vidas.

El derecho al aborto fue visto por la izquierda política como una imposición colonial. Grupos de abogados, dirigentes de partidos independentistas y el presidente del Ateneo Puertorriqueño protestaron ante la ONU por lo que describían como una imposición ajena a los valores de la cultura puertorriqueña\footnote{\quotation{Protestan ante ONU imposición aborto}, {\em Claridad}, 13 de febrero de 1973, 6; Raúl González Cruz, \quotation{Una imposición colonial}, {\em Claridad}, 3 de febrero de 1973, 11; \quotation{Rechaza apliquen ley permite los abortos}, {\em Claridad}, 4 de febrero de 11973, 5.}. En la década de 1970, el derecho de las mujeres a controlar su cuerpo continuó en pugna con discursos nacionalistas tanto dentro del feminismo como en las izquierdas socialistas e independentistas sobre los métodos contraceptivos y la esterilización como políticas genocidas\footnote{Ver Elizabeth Crespo Kebler, \quotation{Ciudadanía y nación: debates sobre los derechos reproductivos en Puerto Rico}, {\em Revista de Ciencias Sociales: Nueva Época} 10 (2001): 57--84; Laura Briggs, \quotation{Discourses of \quote{Forced Sterilization} in Puerto Rico}, {\em Differences} 10 (verano de 1998): 30--66; Rosa E. Marchand-Arias, \quotation{Clandestinaje legal: el aborto en Puerto Rico de 1937 a 1970}, {\em Puerto Rico Health Sciences Journal} 17 no. 1 (marzo de 1998): 15--26; Yamila Azize-Vargas y Luis A. Avilés, \quotation{Abortion in Puerto Rico: The Limits of Colonial Legality}, {\em Reproductive Health Matters} 5, no. 9 (1997): 56--65. \hyphenatedurl{http://www.jstor.org/stable/3775136.}}.

Por el contrario, desde las páginas del {\em Tacón}, se abogó por el derecho de las mujeres sobre sus cuerpos, a favor de los anticonceptivos gratuitos, en contra de la esterilización forzosa y a favor del aborto accesible y provisto en hospitales públicos para las víctimas de violación. Los ensayos \quotation{Los contraceptivos: La ruleta del sexo} en el número preliminar, \quotation{Conoce tu cuerpo: infecciones vaginales} en el primer número, \quotation{Conoce tu cuerpo: el raquet de las duchas vaginales} en el segundo número y \quotation{El parto sentada} en el tercer número son muestra de la importancia que le adjudicó la publicación a este tema\footnote{Véase también el escrito de Nilda Aponte Raffaele, una de las editoras del {\em Tacón de la Chancleta}, publicado en la revista {\em Avance}: \quotation{Abortos: la mujer es la que decide}, 16 de abril de 1973, 19--21; {\em Avance}, \quotation{Las mujeres \quote{liberacionistas} y el aborto}, 16 de abril de 1973, 16--18.}. El tema de la salud de las mujeres y el derecho al control sobre su sexualidad y su cuerpo fue tema que desarrollaría la organización Taller Salud unos años después como asunto prioritario de su activismo feminista desde su fundación a finales de 1979.

\subsection[title={Otros temas},reference={otros-temas}]

La necesidad de un currículo escolar que presentara a las niñas como protagonistas y rompiera con los roles estereotipados asignados a los sexos fue tema importante en la década de 1970 y continúa siéndolo hoy en día. El número preliminar del {\em Tacón de la Chancleta} presentó en su portada el título \quotation{La puertorriqueña dócil} y un artículo de fondo que reseñó los hallazgos del estudio de Haydeé Yordán Molini sobre las imágenes usadas en los libros de texto del Departamento de Instrucción Pública. Mientras los niños se presentaban como protagonistas con mucha más frecuencia que las niñas, como seres activos y fuertes, con gran iniciativa y creatividad, las niñas presentaban actitudes pasivas y sumisas con un exagerado énfasis en la obediencia y docilidad\footnote{\quotation{La puertorriqueña dócil}, {\em El Tacón de la Chancleta}, ejemplar preliminar, {\em Avance}, 30 de septiembre de 1974, 3--5.}.

El tema de la violencia contra las mujeres ocupa la portada del primer número del {\em Tacón} \quotation{En busca de nuestra identidad}. El editorial expresa inconformidad con la idea del orden natural de las cosas que justifica la violencia sexual contra las mujeres como parte de la identificación de la mujer como sujeto pasivo y el hombre como agresor a la vez que se enfatiza en que la felicitad de la mujer es tener un hombre a su lado. Uno de los artículos de este número relata la historia de la estadunidense de origen puertorriqueña y cubana Inez García, quien mató a uno de los dos hombres que la ultrajaron. Su juicio y convicción por asesinato generó grandes protestas y apoyo del movimiento de liberación femenina en Estados Unidos y a través del mundo\footnote{{\em El Tacón de la Chancleta}, enero de 1975, 1, 2, 5, https://ufdc.ufl.edu/AA00070290/00001.}. La atención a la violencia contra las mujeres fue un reclamo que surgió en muchos países en América Latina y el Caribe y se concretó en el Primer Encuentro Feminista Latinoamericano y del Caribe en Colombia en el año 1981 cuando se decidió conmemorar el 25 de noviembre de cada año como el Día Internacional de No Más Violencia contra la Mujer. Al escoger a Minerva, Patria y María Teresa Mirabal, hermanas asesinadas por la dictadura de Trujillo en República Dominicana, se trajo atención a la violencia política contra las mujeres y también otras violencias como la doméstica, la violación, el acoso sexual y la tortura. En el 1999, la Asamblea General de las Naciones Unidas, acogiendo lo que ya era un día feminista de lucha, designó el 25 de noviembre como el Día Internacional de la Eliminación de la Violencia contra la Mujer.

A través de las publicaciones y todo el material gráfico que se encuentra en las páginas del {\em Tacón}, también se puede hacer una apreciación crítica del arte feminista, cómo redefine la sexualidad femenina y las imágenes de las niñas y las mujeres. La imagen de la mujer taína parturienta del número preliminar del {\em Tacón de la Chancleta} es de la renombrada artista gráfica Consuelo Gotay, entonces conocida como Consuelo Claudio. La entrevista a la artista gráfica Myrna Báez hecha por Ana Irma Rivera Lassén, la expone como protagonista y exponente destacada del arte en una profesión dominada por hombres y en una sociedad donde no hay espacios vitales para el desarrollo de mujeres artistas. Este ensayo es citado ampliamente por la crítica del arte hecho por mujeres\footnote{Ana Irma Rivera Lassén, \quotation{Myrna Báez: liberación a través del arte}, {\em El Tacón de la Chancleta}, ejemplar preliminar, {\em Avance}, 30 de septiembre de 1974, 8--9.}. Proliferan a través de todos los números del {\em Tacón de la Chancleta} las ilustraciones de la artista gráfica Ivonne Torres. Ensayos fotográficos como el de Peggy Ann Bliss, \quotation{El rostro de la mujer puertorriqueña} en el primer número (enero de 1975), añaden a este acervo. El arte feminista aporta significados a los textos y nuevas dimensiones al activismo que motivó esta publicación.

El recorrido a través del {\em Tacón de la Chancleta} nos convida a reflexionar sobre la historia de los feminismos latinoamericanos y caribeños. Vemos que los temas de la década de 1970 continúan aún como parte de los retos del presente señalando la importancia de los archivos históricos para conocer los caminos recorridos e imaginar futuros de inclusión y equidad. Invitamos a nuestras lectoras y lectores a conocer {\em El Tacón de la Chancleta} y dLOC, un espacio privilegiado para hacer accesibles, preservar y conectar con otras colecciones de documentos sobre el Caribe.

\thinrule

\page
\subsection{Elizabeth Crespo Kebler}

Elizabeth Crespo Kebler tiene un doctorado en sociología y es estudiosa de los temas de género, sexualidad, raza, etnicidad y feminismos en Puerto Rico, el Caribe y América Latina. Es catedrática y directora del Departamento de Ciencias Sociales en la Universidad de Puerto Rico en Bayamón.

\stopchapter
\stoptext