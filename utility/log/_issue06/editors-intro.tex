\setvariables[article][shortauthor={Glover, Gil}, date={May 2022}, issue={6}, DOI={Upcoming}]

\setupinteraction[title={},author={Kaiama L. Glover, Alex Gil}, date={May 2022}, subtitle={}, state=start, color=black, style=\tf]
\environment env_journal


\starttext


\startchapter[title={}
, marking={}
, bookmark={}]


\startlines
{\bf
Kaiama L. Glover
Alex Gil
}
\stoplines


This special issue is, ultimately, a love letter to librarians and archivists. Seven years into our {\em archipelagos} project, we pay tribute to those whose mandate is to shape and steward the historical and cultural record, and by extension, to create the possibilities for and contours of knowledge production in the historical and interpretive disciplines. Librarians and archivists are the first line of defense against silence and often their cause, transmuting epistemology into tangible, material forms, the \quotation{sources} and \quotation{archives} of Michel-Rolph Trouillot. After the digital turn, our record and its silences are a hybrid construct---part analog, part digital---and is likely to remain so for the foreseeable future. As a result, librarians and archivists have had to take on many more roles: engineers, designers, project managers, photographers, and well-positioned architects of the new macroscopes for our history and culture. Issue (6) of {\em archipelagos} is meant to make this work and these phenomena explicit---to take a sweeping look at the \quotation{back-end}---by centering the work of our largest regional digital library: The Digital Library of the Caribbean (dLOC).

We were incredibly fortunate to find the ideal guest editors in Perry Collins and Hadassah St.~Hubert, who have, in both their editorial work and their professional trajectories, modeled the very \quotation{best practices} brought to the fore by the essays in this issue. The two of them have opened up a space of debate and knowledge-sharing that echoes the diversity underlying the Digital Library of the Caribbean itself---differing levels of seniority, professional roles, institutional affiliations, national origins, and else enrich the essays in this issue. Given this range, our special issue engages a panoply of actors and relationships both within and beyond the academy, that correspond with and speak to the breadth of audiences served by open access digital collections. The work featured here is relational, adamantly decolonial, committed to social justice, and reflective of values shared across the borders of nation and language.

At the heart of this issue is the gift to all of us that is the Digital Library of the Caribbean, one of the largest and most stable such open access regional digital libraries on the planet. Its collections range from historical newspapers, to literary magazines, to contemporary academic production. In recent years, dLOC has begun to explore collections as data, planting the seeds for our future exercises in computational analysis of our cultural and historical record. Their partnerships span across the different languages and geographies of the Caribbean; dozens of institutions have long benefited from their non-extractive, redistributive model. The relationships they have nurtured consistently translate into research opportunities across and beyond the region. At the moment of this writing, the networks engaged by dLOC are easily as expansive as those that participate in major Caribbeanist academic conferences. In addition, dLOC's librarians have increasingly entered the classroom as facilitators of pedagogical materials and lessons and as teachers in their own right.

It would be impossible to do justice to this vast and beneficent ecosystem in a single special issue, let alone in an introduction. Nonetheless, we hope that this issue will encourage our readers to learn more, and to become more involved in the maintenance and use of this invaluable resource. We hope, too, that this issue resonates beyond a tribute to dLOC to praise the work of all the other, smaller digital libraries, archives, and exhibits curated and produced in and for the Caribbean. We stand firm in our conviction that the first step in combating historical silence is to make sure all the voices of our Caribbean Digital community become celebrated contributors to the lasting record.

Kaiama & Alex

\page
\subsection{Kaiama L. Glover}

\useURL[url1][https://barnard.edu/profiles/kaiama-l-glover][][Kaiama L. Glover]\from[url1] is Associate Professor of French and Africana Studies at Barnard College, Columbia University. She is the author of \useURL[url2][http://liverpooluniversitypress.co.uk/products/61903][][Haiti Unbound: A Spiralist Challenge to the Postcolonial Canon]\from[url2] (Liverpool UP 2010), first editor of \useURL[url3][http://yalebooks.com/book/9780300214192/yale-french-studies-number-128][][Marie Vieux Chauvet: Paradoxes of the Postcolonial Feminine]\from[url3] (Yale French Studies 2016), and translator of Frankétienne's Ready to Burst (Archipelago Books 2014). She has received awards and fellowships from the National Endowment for the Humanities, the Mellon Foundation, and the Fulbright Foundation. Current projects include forthcoming translations of Marie Vieux Chauvet's {\em Dance on the Volcano} (Archipelago Books) and René Depestre's {\em Hadriana in All My Dreams} (Akashic Books), and the multimedia platform {\em In the Same Boats: Toward an Afro-Atlantic Visual Cartography}.

\subsection{Alex Gil}

\useURL[url4][http://www.elotroalex.com/][][Alex Gil]\from[url4] is the Digital Scholarship Librarian at Columbia University Libraries. His research and practice focuses on digital humanities, epistemic design, minimal computing, and Caribbean literature. He is co-founder and moderator of \useURL[url5][http://xpmethod.plaintext.in/][][Columbia's Group for Experimental Methods in Humanistic Research]\from[url5], and coordinator of the Butler Studio at Columbia University Libraries.

\stopchapter
\stoptext