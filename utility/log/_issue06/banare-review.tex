\setvariables[article][shortauthor={Banaré}, date={May 2022}, issue={6}, DOI={Upcoming}]

\setupinteraction[title={*Île en île*~: comme un nouvel instrument de navigation},author={Eddy Banaré}, date={May 2022}, subtitle={Un nouvel instrument de navigation}, state=start, color=black, style=\tf]
\environment env_journal


\starttext


\startchapter[title={{\em Île en île}~: comme un nouvel instrument de navigation}
, marking={Un nouvel instrument de navigation}
, bookmark={*Île en île*~: comme un nouvel instrument de navigation}]


\startlines
{\bf
Eddy Banaré
}
\stoplines


«~La lecture est une amitié~»~; la phrase de Proust me vient quand il m'est demandé de décrire la contribution d'{\em Île en île} à mon parcours tant professionnel que personnel. Je ne peux que raconter les moments décisifs auxquels ce site a participé, les univers qu'il a pu dévoiler. Dès sa création en 1998, l'archive en ligne {\em Île en île} a progressivement ouvert des horizons au lycéen que j'étais alors, pris dans la préparation du baccalauréat. Ce site était le seul lieu où l'on pouvait s'informer sur un patrimoine littéraire qui commençait à éveiller ma curiosité. En dépit de l'implication de mes professeurs, Molière, Balzac ou Baudelaire semblaient éloignés de ma réalité immédiate et irrémédiablement attachés à la chose scolaire. Ces auteurs ne me rejoignaient qu'à travers des signes comme la «~tabatière pleine de macouba~» du Père Goriot ou la présence de Jeanne Duval auprès d'un Baudelaire.

Je commençais, dans un joyeux désordre, à découvrir Aimé Césaire, Patrick Chamoiseau, Raphaël Confiant, Joseph Zobel et Frantz Fanon, mais aussi Jorge Amado et James Baldwin, qui m'aidaient à tracer d'autres chemins littéraires et imaginaires. Ils me procuraient la sensation de mieux saisir ma réalité créole américaine. La période 1996--98 a été particulièrement riche pour la littérature antillaise de langue française, avec notamment le roman {\em L'esclave vieil homme et le molosse} et l'essai {\em Écrire en pays dominé} (1997) pour Chamoiseau et les essais {\em Le traité du Tout-Monde} et {\em Faulkner, Mississippi} pour Édouard Glissant. {\em Le bassin des ouragans} (1994) et {\em La baignoire de Joséphine} (1997)---récits graves et truculents de Raphaël Confiant---ont également joué un rôle important dans cette nouvelle curiosité que je développais alors pour la littérature.

J'ai ensuite tenté d'aborder des œuvres plus «~canoniques~»~avec plus ou moins de succès : presque tout Césaire~; {\em Peau noire, masques blancs} de Fanon~; {\em La Lézarde} et {\em Le discours antillais} de Glissant mais surtout les grandes voix haïtiennes comme Jacques-Stephen Alexis, Jacques Roumain et René Depestre. J'ai été frappé par l'opacité, la force et la violence des images, mais surtout par la pertinence des concepts et des analyses~qui exigent du lecteur d'être armé de solides connaissances historiques. L'année 1998, avec sa commémoration des 150 ans de l'abolition de l'esclavage, participait également à une atmosphère littéraire particulièrement chargée. J'assistais aux représentations de {\em La tragédie du Roi Christophe} et de {\em Et les chiens se taisaient} de Césaire. Au moment de la recherche de soi et des questionnements politiques et identitaires, la Caraïbe m'apparaissait de plus en plus comme un écosystème historique et politique dense que la littérature pouvait aider à déchiffrer. Une littérature que j'ai donc décidé d'explorer dès mon entrée à l'Université des Antilles-Guyane (devenue depuis Université des Antilles). C'est ici que le rôle d'{\em Île en île} a été particulièrement déterminant~; la navigation dans cette bibliothèque ne pouvait plus totalement se faire à vue.

\subsection[title={Naviguer dans l'archipel},reference={naviguer-dans-larchipel}]

L'engagement dans les études universitaires devait créer un nouveau commerce intellectuel, plusieurs romans, essais, pièces de théâtre et recueils de poésie se sont imposés, soit par l'intermédiaire d'enseignants comme Raphaël Confiant ou Jean Bernabé, soit par les échanges avec d'autres camarades étudiants. Je me retrouvais donc avec des œuvres à découvrir cette fois-ci de manière disciplinée et surtout selon un rythme compatible avec l'obtention d'un diplôme. Ce besoin de découverte obéissait à une urgence intérieure que je ne me suis jamais expliquée. Nous étions encore aux débuts de l'informatisation des études~; les anthologies et autres dictionnaires constituaient l'essentiel des outils de recherche. La publication de l'{\em Éloge de la créolité} en 1989 et le prix Goncourt obtenu par Chamoiseau en 1992 avaient déjà signalé de manière paradoxale le manque de visibilité de l'expression littéraire francophone dans les Amériques. La lecture de l'{\em Éloge} provoquait une épiphanie en me révélant des fraternités esthétiques et historiques à travers des figures tutélaires---Stephen Alexis, William Faulkner, Gabriel García Márquez, Césaire, Glissant---et posaient véritablement un nouveau programme de lecture qu'il me fallait évidemment enrichir.

Le site crée par Thomas C. Spear en 1998 s'est définitivement imposé comme outil de travail en 2001. Je devais mener un travail de recherche sur {\em L'esclave vieil homme et le molosse} de Patrick Chamoiseau, mis au programme d'un cours sur les littératures francophones de la Caraïbe. Un second travail portait sur l'historique des premières transcriptions de la langue créole et me révélait le nom de Gilbert Gratiant. Avec beaucoup d'enthousiasme, je découvrais que l'attention portée à ces expressions littéraires venait d'une prestigieuse institution universitaire aux États-Unis où Glissant officiait depuis quatre ans. Cette révélation à distance de nos trésors littéraires avait une dimension incongrue (et paradoxale) que je ne parvenais pas totalement à m'expliquer~; ces œuvres étaient pour ainsi dire ignorées, voire évitées, sur leur terre natale mais célébrées aux États-Unis. Par l'organisation même de ses menus déroulants, {\em Île en île} affirmait et confirmait des relations géographiques et historiques. Surtout, les onglets Atlantique, Caraïbes, Méditerranée, Océan Indien et Pacifique me disaient tous les liens à tisser, que les îles et les archipels avaient une parole singulière.

\subsection[title={Les géographies comparées~: relations océaniennes},reference={les-géographies-comparées-relations-océaniennes}]

L'entrée à l'université, c'est également la découverte des théories littéraires et de la littérature comparée---champs disciplinaire dont je me revendique encore aujourd'hui. L'enseignement de Georges Voisset, spécialiste du genre pantoun, m'a appris à penser la littérature en motifs convergents et surtout en rencontres d'imaginaires. On pouvait donc former des corpus~; une étude sur le motif de l'île pouvait donc rassembler un récit de voyage du XVIIe, un poème de {\em Moi, laminaire . . .} d'Aimé Césaire, un extrait de {\em Foe}, roman du Sud-africain J.-M. Coetzee et {\em L'île des rêves écrasés} de la tahitienne Chantal Spitz. Cette approche à la fois rigoureuse et indisciplinée de la littérature a été une véritable révélation et me confortait dans la fréquentation d'{\em Île en île} qui, depuis, participait à mon agenda de lectures.

La découverte d'Edward Said a également été décisive. Dans {\em Culture et impérialisme}, traduit en 2000, le théoricien palestinien expliquait le rôle de celui «~qui réfléchit à la culture~» qui est «~donc de ne pas prendre pour naturelle la politique identitaire, mais de montrer comment toutes ces représentations sont construites, à quelles fins, par qui et à partir de quoi~». Il ajoutait plus loin que «~toute société ou tradition officielle repousse les interférences dans ses grands récits. Ceux-ci acquièrent avec le temps un statut presque théologique~: les héros fondateurs, idées et valeurs vénérées, allégories nationales, ont un impact inestimable dans la vie culturelle et politique~»\footnote{Edward Said, {\em Culture et impérialisme} (Paris~: Fayard, 2000), 435--36.}. Ces lignes, à la teneur littéralement explosive, devaient sceller mes engagements futurs. De plus, la Martinique commençait à me paraitre étroite. J'ai lu {\em Saint-John Perse et le conteur} (1971) d'Emile Yoyo, {\em L'intention poétique} de Glissant et {\em Écrire en pays dominé} de Chamoiseau. Le premier essai mettait en lumière une rencontre décisive et instituait la plantation esclavagiste en lieu fondateur d'une esthétique. {\em L'intention poétique} me dévoilait Victor Segalen~; soit une trajectoire entre la Bretagne, Tahiti et les Amériques.

L'influence de Segalen sur Glissant m'a immédiatement fasciné parce qu'elle a participé à forger les concepts de «~Relation~» et de «~Créolisation~». Glissant explique également le choc poétique qu'a constitué ici le Divers défini par Segalen en ce que «~le Divers n'est donné à chacun que comme une relation, non comme un absolu pouvoir ni une unique possession. Le divers renaît quand les hommes se diversifient concrètement dans leurs libertés différentes. Alors il n'exige plus que l'on renonce à soi. L'Autre est en moi, parce que je suis moi~»\footnote{Édouard Glissant, {\em L'intention poétique (Poétique II)} (Paris~: Gallimard, 1969), 101.}. Cette analyse a donc achevé de me révéler les pouvoirs de l'expression littéraire. Je me suis donc plongé avec passion dans l'œuvre de Segalen, notamment son roman {\em Les immémoriaux}. De même que nos littératures créoles américaines racontent la remontée du gouffre de la traite, Segalen raconte le cataclysme de la colonisation. Plus encore que des récits de trauma, il s'agissait de réinventions de soi. Toujours selon Glissant, «~La pensée décisive de Segalen est que la rencontre de l'Autre surractive l'imaginaire et la connaissance poétique~»\footnote{Édouard Glissant, {\em Poétique de la} {\em Relation (Poétique III)} (Paris~: Gallimard, 1990), 42.}. Il m'apparaissait donc plus clairement que ce qui reliait les littératures réunies dans {\em île-en-île} étaient les diverses expériences d'effondrements historiques de cosmogonies, de sacrés et d'imaginaires. La littérature prenait alors définitivement un caractère anthropologique~: elle témoignait de ce que font les humanités pour survivre et mieux vivre~; c'est ainsi que je comprenais cette idée de «~connaissance poétique~».

C'est donc naturellement qu'après des travaux sur Chinua Achebe, Driss Chraïbi et Segalen, j'ai saisi l'occasion de me rendre en Nouvelle-Calédonie pour étudier les représentations littéraires de l'activité minière. La mine est une véritable métaphore de la colonisation en ce qu'elle anéantit les espaces, fait reculer ou efface les populations mais peut aussi créer un nouveau collectif. En littérature, la mine représente à la fois les masses ouvrières d'Émile Zola, les hécatombes maya des conquistadors ou encore les intrigues politiques du {\em Nostromo} de Joseph Conrad. Arrivé dans cet archipel du Pacifique, j'ai découvert que l'activité minière recouvrait d'effarantes complexités et horreurs coloniales.~Elle signifiait, dans un premier temps, l'accaparement, la destruction des terres, le traumatisme du peuple kanak, mais aussi l'exploitation des engagés venus d'Indonésie, du Tonkin ou du Japon. Mais le nickel racontait la naissance d'un peuple et d'un langage nouveaux, ainsi que le rêve d'émancipation. Les expressions littéraires s'étant constituées autour du métal racontent ce chaos.

Engagé dans l'écriture de ma thèse, j'ai réalisé que la section Pacifique (celle de la Nouvelle-Calédonie) du site Ile-en-Ile était peu animée. C'est la raison pour laquelle je me suis décidé à contacter Thomas C. Spear pour lui proposer de rédiger les notices de Georges Baudoux et Jean Mariotti, deux auteurs qui ont su restituer les tourments liés à la colonisation singulière de l'archipel. La réponse de C. Spear a été aussi spontanée que bienveillante et j'ai ainsi pu contribuer à l'outil qui avait participé à ma découverte de ces littératures archipéliques. J'ai pu rencontrer C. Spear en Nouvelle-Calédonie, en Martinique et au Québec~; j'ai vu une sensibilité de chercheur à l'œuvre~; toujours soucieuse de capter de nouvelles voix. De là sont nés une amitié et un respect indéfectibles.

\thinrule

\page
\subsection{Eddy Banaré}

Après un master en littérature comparée à l'université des Antilles-Guyane, Banaré a obtenu en 2010 un doctorat sur les représentations littéraires de l'industrie minière en Nouvelle-Calédonie. Une version remaniée de sa thèse a été publiée en 2012 sous le titre {\em Les récits du nickel en Nouvelle-Calédonie (1853-1960)} (Honoré Champion, Collection Francophonies). Il a enseigné le français à l'Alliance Française de Suva, ainsi qu'à la University of the South Pacific. Spécialisé dans les expressions littéraires coloniales et postcoloniales, Banaré mène des recherches sur les relations entre la presse et la littérature. Membre associé de l'équipe de recherche TROCA (Trajectoires d'Océanie) de l'université de la Nouvelle-Calédonie, il participe également aux enseignements de littérature comparée.~

\stopchapter
\stoptext