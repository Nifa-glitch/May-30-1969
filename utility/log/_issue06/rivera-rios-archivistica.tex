\setvariables[article][shortauthor={Rivera Bonilla, Ríos Villarini}, date={May 2022}, issue={6}, DOI={https://doi.org/10.7916/archipelagos-3xs6-tz02}]

\setupinteraction[title={Archivística y sonoridad: un ejercicio práctico para conocer mejor el patrimonio documental del Archivo Histórico de Vieques en la Biblioteca Digital del Caribe},author={Ivelisse Rivera Bonilla, Nadjah Ríos Villarini}, date={May 2022}, subtitle={Archivística y sonoridad}, state=start, color=black, style=\tf]
\environment env_journal


\starttext


\startchapter[title={Archivística y sonoridad: un ejercicio práctico para conocer mejor el patrimonio documental del Archivo Histórico de Vieques en la Biblioteca Digital del Caribe}
, marking={Archivística y sonoridad}
, bookmark={Archivística y sonoridad: un ejercicio práctico para conocer mejor el patrimonio documental del Archivo Histórico de Vieques en la Biblioteca Digital del Caribe}]


\startlines
{\bf
Ivelisse Rivera Bonilla
Nadjah Ríos Villarini
}
\stoplines


{\startnarrower\it El artículo describe y reflexiona sobre la experiencia de un grupo de estudiantes universitarios que exploraron parte de la colección digital del Archivo Histórico de Vieques (AHV) ubicada en la Biblioteca Digital del Caribe (dLOC) en el curso {\em Sociedad y Cultura de Puerto Rico} en la Universidad de Puerto Rico en Humacao de agosto a diciembre de 2020. En la actividad \quotation{Historias de lucha y resistencia en Vieques} el estudiantado desarrolla competencias de información para visibilizar y crear conciencia sobre los problemas sociales, ambientales y políticos que ha enfrentado la comunidad de Vieques en su lucha en contra de la Marina de Guerra de los Estados Unidos. Como parte de la didáctica crítica, el estudiantado descubre las fuentes primarias disponibles en el AHV y elabora una \quotation{reflexión sonora} en la cual describe un ítem del AHV y reflexiona sobre su relevancia y vigencia. Las autoras de este artículo argumentan que el ejercicio de buscar fuentes primarias confiables en un archivo comunitario, combinado con el ejercicio creativo de generar un archivo sonoro, provee nuevas posibilidades pedagógicas en el desarrollo de competencias de información, análisis crítico y concientización en el estudiantado.

 \stopnarrower}

\blank[2*line]
\blackrule[width=\textwidth,height=.01pt]
\blank[2*line]

\subsection[title={El Archivo Histórico de Vieques},reference={el-archivo-histórico-de-vieques}]

El Archivo Histórico de Vieques (AHV) es parte de un grupo de archivos comunitarios en Puerto Rico que tiene como norte la preservación de los acervos documentales en su formato físico y digital. La visión de un archivo con un enfoque comunitario y participativo se enfoca en que los miembros de la comunidad aporten materiales de valor histórico al Archivo, se adiestren en la preservación, digitalización y catalogación de los mismos y los utilicen para profundizar en su conocimiento sobre la historia y las luchas del pueblo viequense. En la actualidad, el Archivo está ubicado en las facilidades del Fuerte Conde de Mirasol en el Barrio Mambiche de la isla municipio de Vieques (foto 1). El depósito de materiales incluye un área en la cual se ubican los materiales que requieren protección especial de la humedad (por ejemplo, documentos antiguos, periódicos, grabaciones y videos) (foto 2). Los documentos que han sido digitalizados y catalogados se encuentran en la página web de la Biblioteca Digital del Caribe (dLOC, por sus siglas en inglés)\footnote{Digital Library of the Caribbean, \quotation{Archivo Histórico de Vieques}, fecha de acceso: 14 de enero de 2022, \useURL[url1][https://dloc.com/vieques]\from[url1].}.

\placefigure[here]{Fuerte Conde Mirasol}{\externalfigure[issue06/valence_1_irb_nrv_foto1.png]}


En la isla municipio de Vieques, al igual que en otros pueblos de la Puerto Rico, no hay bibliotecas públicas ni archivos municipales que garanticen a los ciudadanos el acceso a fuentes primarias y secundarias sobre la historia de su pueblo. El AHV es un laboratorio vivo que permite que sectores diversos de la población tengan acceso a la información sobre la historia viequense. De otro modo, los residentes de la isla municipio tendrían que ir al Archivo General de Puerto Rico o a la Colección Puertorriqueña de la Universidad de Puerto Rico, ambos en San Juan, para obtener información relacionada a la historia local de sus comunidades. El acceso que tiene la comunidad al AHV viabiliza que públicos diversos desarrollen interés por la investigación histórica.

El carácter comunitario de este archivo promueve modelos ágiles, participativos y democráticos de preservación y conservación de documentos físicos, grabaciones sonoras y colecciones de imagen y movimiento. A esas tareas se incorporan estudiantes, vecinos y voluntarios locales y de fuera de Puerto Rico, quienes desarrollan destrezas de conservación, digitalización y catalogación. Al igual que otros archivos comunitarios en Puerto Rico, el AHV tienen un carácter multiuso ya que el espacio donde está ubicado se comparte con una radio emisora comunitaria, un museo y espacios de reunión. Es un lugar que convoca a diferentes poblaciones y promueve que se amplíe el alcance comunitario del archivo y el interés por los recursos que alberga (fotos 3--6).

\placefigure[here]{Participantes en el taller sobre Preservación Audiovisual trabajan con la Colección Andrés Nieves}{\externalfigure[issue06/valence_1_irb_nrv_foto2.jpg]}


\placefigure[here]{Miembros de la Junta Asesora del Archivo Histórico de Vieques}{\externalfigure[issue06/valence_1_irb_nrv_foto3.jpg]}


La agenda de preservación y conservación de los recursos del Archivo Histórico de Vieques ha recibido el apoyo e impulso de las alianzas que se han creado con organizaciones y universidades en Puerto Rico y Estados Unidos. El AHV cuenta con el respaldo del Instituto de Cultura Puertorriqueña (ICP). Desde la plataforma digital Coloqueo del ICP\footnote{Instituto de Cultura Puertorriqueña, canal de YouTube, fecha de acceso: 14 de enero de 2022, \useURL[url2][https://www.youtube.com/channel/UCLQ95MO9mB_bjbTV6WVsMcQ]\from[url2].}, se transmite una vez al mes una conferencia sobre los documentos que el público general puede acceder a través del portal digital del AHV. Además, desde el 2019, se han organizado tres sesiones de trabajo con el grupo {\em Audiovisual Preservation Exchange} (APEX) de la Universidad de Nueva York\footnote{NYU Tisch, \quotation{APEX Puerto Rico 2019}, \useURL[url3][https://tisch.nyu.edu/cinema-studies/miap/research-outreach/apex/apex-puerto-rico-2019]\from[url3].} con el propósito de desarrollar colecciones audiovisuales que documentan más de treinta años de luchas ambientales en la isla municipio. Otra colaboración meritoria es el apoyo que recibe el AHV de la Biblioteca Digital del Caribe (dLOC) de la Universidad de Florida\footnote{Digital Library of the Caribbean (dLOC), página de inicio, fecha de acceso: 14 de enero de 2022, \useURL[url4][https://www.dloc.com/]\from[url4].}. Con esta alianza, se mantiene la página electrónica del AHV, al mismo tiempo que dLOC apoya en el resguardo de todos los materiales digitalizados hasta el momento.

El AHV ha servido de laboratorio para que diversos grupos de estudiantes tengan experiencias de investigación y de internado. Estudiantes de la escuela superior de Vieques, así como estudiantes universitarios de todos los niveles, desde bachillerato hasta doctorado, han utilizado los recursos del AHV para sus proyectos de investigación a lo largo de los años. Los proyectos de digitalización recientes y el acceso que provee dLOC a algunas de las colecciones del AHV ha facilitado la investigación académica, especialmente durante la pandemia por el COVID-19. Por otro lado, el AHV es también un espacio educativo para la formación de futuros profesionales de archivística y bibliotecología. Este es el caso de la Escuela de Ciencias y Tecnologías de la Investigación de la Universidad de Puerto Rico de Río Piedras. La variedad de recursos disponibles hace del AHV un lugar idóneo para el desarrollo de estudiantes universitarios en varias disciplinas afines a la preservación y difusión digital, así como la investigación histórica.

El AHV cuenta con un director, posición que ocupa el Sr.~Robert Rabin, y con una junta asesora constituida por miembros de la comunidad viequense, así como académicos de instituciones locales y fuera de Puerto Rico, artistas y miembros de la diáspora que se mantienen vinculados a las luchas y reclamos viequenses por la justicia social. Es este cuerpo directivo quien se encarga de discutir, planificar y desarrollar las colecciones y actividades del Archivo Histórico de Vieques\footnote{La junta asesora del AHV está compuesta por Robert Rabin, Diana Ramos Gutiérrez, Sofía Gallisá, Juan Carlos Rodríguez, Lowell Fiet, María Cristina Rodríguez, Ivelisse Rivera Bonilla y Nadjah Ríos Villarini.}.

Entre las colecciones del AHV, figuran registros eclesiásticos y gubernamentales importantes del siglo XIX como son los documentos del Ayuntamiento y los Registros Bautismales de la Iglesia Católica. Además, el AHV cuenta con una amplia colección de fotografías que datan del 1920 hasta el presente y que documentan las costumbres y tradiciones de los viequenses, así como la presencia militar de la Marina de Guerra de los Estados Unidos. Por otro lado, el AHV tiene una vasta colección de imagen y movimiento, audio y material efímero sobre las luchas civiles y ambientales del Comité Pro-rescate y Desarrollo de Vieques. El AHV alberga, además, documentación, mapas y planes sobre desarrollo económico, descontaminación y mitigación ambiental.

\subsection[title={¿Por qué este archivo es importante?},reference={por-qué-este-archivo-es-importante}]

Es importante destacar el valor histórico, político y geográfico de la isla municipio de Vieques para darles el mérito justo a las colecciones que el AHV resguarda y cuida. La isla municipio de Vieques cuenta con una población aproximada de 9,300 personas según los informes del Censo del 2020. A principios del siglo XX, los Estados Unidos identificaron las islas municipio de Vieques y Culebra como bastiones importantes para la defensa del Caribe. Este hecho, junto a la adquisición de las Islas Vírgenes Americanas en 1917, hizo que se diera un flujo migratorio entre las islas municipios y los territorios recientemente adquiridos. Es por ello que estudiosos del tema hayan descrito ambos municipios como islas puentes con el Caribe\footnote{M. González and N. Ríos-Villarini, \quotation{Floating Migration, Education, and Globalization in the US Caribbean}, {\em International Journal of Qualitative Studies in Education} 25, no. 4 (2012): 471--86.}.

En 1941 la Marina de Guerra de los Estados Unidos ocupó dos terceras partes de la isla de Vieques con el propósito de establecer una base militar. Como consecuencia, miles de viequenses fueron desplazados fuera de sus comunidades. Muchos de ellos se vieron obligados a migrar a las Islas Vírgenes Americanas, así como a la isla de Puerto Rico. En la base militar se realizaron, por décadas, prácticas militares con explosivos químicos tales como agente naranja y napalm, afectando la calidad de vida de los residentes de Vieques. No es hasta el 2003 que se declara un cese y desista de estos ejercicios militares y el cierre total de la base militar como resultado de la denuncia tenaz y las actividades de desobediencia civil de cientos de ciudadanos de Vieques, Puerto Rico, la diáspora puertorriqueña y diversos países del mundo.

Las manifestaciones en contra de la Marina de Guerra de los Estados Unidos, así como el justo reclamo de la limpieza y entrega de los terrenos contaminados, constituye una de las luchas políticas y ambientales más importantes en la historia de Puerto Rico. El Archivo Histórico de Vieques preserva y cuida toda la documentación asociada a esta lucha. Ejemplo de estos recursos son los boletines de la Cruzada Pro-rescate de Vieques. Además, se han digitalizado seis periódicos locales producidos entre el 1950 y el 2003: el semanario {\em Vieques Times} (1987--2003), {\em Isla Nena} (1985--93), {\em La Página de Cheo} (1990--96), {\em El Nuevo Vieques} (2000--2003), {\em El Navegante} (1992--2000) y el {\em Vieques Breeze} (1943). Esta colección de periódicos locales producidos en Vieques constituye un acervo documental importante que no está disponible en ningún otro lugar y que aporta a la lucha inconclusa del pueblo viequense. Por último, se destacan las colecciones fotográficas Catanzaro y Gerry Carr. Estas colecciones, recientemente digitalizadas y catalogadas, fueron donadas por ambas familias al AHV. La importancia de ambas colecciones radica en que en conjunto ofrecen una mirada cotidiana a la convivencia entre viequenses y militares.

Sin duda, el AHV resguarda un acervo documental importante para la historia de la isla municipio y del país en general. Sus fuentes de información abren posibilidades para establecer conexiones entre Puerto Rico y el Caribe anglófono. Desde la óptica ambiental, el AHV ofrece un capítulo importante en la historia de siglo XX. La variedad de los recursos hace de este archivo un espacio único que visibiliza a una comunidad que ha sido históricamente excluida.

\subsection[title={El Archivo Histórico de Vieques como recurso educativo},reference={el-archivo-histórico-de-vieques-como-recurso-educativo}]

La Universidad de Puerto Rico en Humacao (UPRH) es una de las universidades que mantiene lazos estrechos de colaboración con el Archivo Histórico de Vieques. En cursos del Departamento de Ciencias Sociales de la UPRH se realizan, desde 2014, actividades en las que estudiantes se familiarizan con el AHV, colaboran en la digitalización y la catalogación de documentos, e investigan sobre la historia de Vieques y sus luchas en las colecciones que están disponibles en la plataforma dLOC. Algunas de esas actividades se han desarrollado en conjunto con estudiantes del {\em Puerto Rico International Education Program} de la Universidad Estatal de California en Fullerton, con el cual la UPRH tiene un acuerdo de colaboración.

En el curso {\em Sociedad y Cultura de Puerto Rico} que facilita una de las autoras en el Departamento de Ciencias Sociales de la UPRH se realiza un ejercicio en el cual el estudiantado conoce el patrimonio documental del Archivo Histórico de Vieques. Uno de los objetivos del curso es que el estudiantado explore la formación de la sociedad puertorriqueña y los discursos de identidad nacional desde una perspectiva histórica. Esa exploración se hace a través de lecturas y actividades que visibilizan los silencios en la historia oficial sobre aspectos constitutivos de nuestra identidad cultural como sujetos colonizados, caribeños, afrodescendientes y como sujetos políticos con una larga historia de articular nuestra voz propia para cuestionar, exigir y proponer estrategias para el desarrollo social, político y económico que visualizamos para el país. La experiencia de explorar temas silenciados en las narrativas de la historia oficial de Puerto Rico\footnote{Entre los temas que no forman parte, de manera intencionada ni consistente, de los currículos a nivel escolar y universitario en Puerto Rico se encuentran: la historia e implicaciones colectivas y personales de la sujeción colonial, el impacto del militarismo en la vida cotidiana puertorriqueña, la persecución política de los grupos e individuos que cuestionan el régimen político y económico vigente, el impacto del patriarcado y del racismo (estructural, inter- e intrapersonal) en las subjetividades, la desigualdad social y económica que empobrece a la mayor parte de la población, las múltiples formas en que comunidades diversas del país y la diáspora ejercen y practican el poder ciudadano para proteger y exigir que se respeten sus derechos, los saberes sobre cómo se vive y se siente en puertorriqueño que se generan desde el arte, y las migraciones como elemento intrínseco de la identidad nacional.} genera entre los estudiantes universitarios sorpresa, admiración, indignación, coraje e inspiración. Entre esos temas se encuentra el impacto de la militarización en la vida cotidiana y en la salud del pueblo viequense y las luchas por recuperar los terrenos ocupados por la Marina de Guerra de los Estados Unidos. Periódicamente, los medios noticiosos de Puerto Rico reportan sobre las dificultades que enfrentan los viequenses al carecer de servicios médicos básicos y de un sistema de transportación marítima confiable. No obstante, en los currículos académicos, por lo general no se estudia formalmente o con profundidad las raíces históricas de los problemas sociales y económicos que enfrenta la isla municipio. Tampoco se explora cómo la historia y la situación actual de Vieques aporta al estudio y comprensión del colonialismo, las implicaciones de la economía extractivista y los procesos de {\em gentrification} en Puerto Rico, por mencionar sólo algunos ejemplos de temas que merecen mayor atención en el estudio de la historia del archipiélago puertorriqueño. Estos son temas que el estudiantado puede investigar en las colecciones del Archivo Histórico de Vieques.

En el semestre de agosto a diciembre de 2020, se incorporó a la unidad \quotation{Visiones sobre lo puertorriqueño: cultura oficial, silencios y contra propuestas} del curso {\em Sociedad y Cultura de Puerto Rico} la actividad \quotation{Historias de lucha y resistencia en Vieques}. La misma está organizada en dos partes.

En la primera parte el estudiantado aprende, a través de videos y conferencias, sobre la historia del colonialismo en Puerto Rico y su efecto en la vida cotidiana y la psiquis puertorriqueña. También aprende sobre el impacto del colonialismo en Vieques, especialmente a través de la militarización de su territorio. En las discusiones sobre estos temas el estudiantado se expone a la propuesta de pensar Puerto Rico como un {\em conjunto de islas, islotes y cayos}, entre los cuales están habitadas las islas de Vieques, Culebra y la Isla Grande (término que usan los habitantes de las otras dos islas para referirse a Puerto Rico). Pensar el país como archipiélago tiene el efecto de visibilizar dos poblaciones históricamente marginadas por el gobierno central y, simultáneamente, {\em exotizadas} en el imaginario de los puertorriqueños de la Isla Grande y los turistas extranjeros.

La segunda parte de la actividad \quotation{Historias de lucha y resistencia en Vieques} es el proyecto de fin de curso. Para este proyecto cada estudiante aprende a utilizar la colección digital del AHV y realiza una reflexión sobre los acervos documentales del archivo, la cual graba y comparte a través de WhatsApp. Este audio es lo que llamamos en este artículo una \quotation{reflexión sonora}. La producción de la reflexión sonora es un trabajo que cada estudiante realiza de manera independiente y que requiere explorar los materiales del AHV disponibles en dLOC (la investigación), analizar críticamente el material investigado para producir una reflexión escrita y una reflexión sonora (el análisis), y compartir la reflexión sonora con sus compañeras de curso y el AHV (la divulgación).

\placefigure[here]{\goto{Videos de estudiantes (https://www.youtube.com/embed/zkAWh3gmj6E)}[url(https://www.youtube.com/embed/zkAWh3gmj6E)]}{\externalfigure[issue06/rivera-video.jpg]}


Los pasos que sigue cada estudiante para completar el proyecto \quotation{Historias de lucha y resistencia en Vieques} están descritos en la “{\em Guía para el proyecto} {\em final”}. De manera resumida, los pasos para realizar el proyecto son los siguientes:

\startitemize[n][stopper=.]
\item
  Aprender a navegar los materiales y las colecciones del AHV disponibles en dLOC\footnote{Digital Library of the Caribbean, \quotation{Archivo Histórico de Vieques}.} usando como referencia dos videos disponibles en YouTube, uno que describe el tipo de documentación que contiene el AHV y otro que explica cómo navegar el archivo\footnote{R. Rabin, \quotation{Archivo Histórico de Vieques: recursos para la investigación sobre Vieques con un enfoque en las luchas sociales a través de los pasados siglos}, Coloqueo, 2020, \useURL[url5][https://www.youtube.com/watch?v=PUcSmJWjtpQ]\from[url5]; R. Rabin y V. Fernández, \quotation{Documentos del Archivo Histórico de Vieques ubicados en la Biblioteca Digital del Caribe}, Coloqueo, 2020, \useURL[url6][https://www.youtube.com/watch?v=v1ajBheLUC0&t=165s]\from[url6].}.
\item
  Explorar los materiales del AHV disponibles en la plataforma dLOC, según sus intereses personales, profesionales o en relación a algún tema discutido en el curso. Para ello, el estudiantado puede hacer búsquedas de temas específicos usando palabras claves o puede explorar una de las colecciones del AHV\footnote{Las colecciones del AHV están organizadas en tres categorías: según el medio de publicación, según quien crea o dona la colección y por temáticas. Las colecciones disponibles en dLOC a la fecha de la publicación de este artículo son

    {\em el} {\em periódico} {\em Claridad} (https://dloc.com/AA00062838/00001);

    el periódico {\em Isla Nena} (https://dloc.com/AA00062839/00001);

    el periódico {\em Vieques} {\em Times} (https://dloc.com/AA00062475/00001);

    la colección de máscaras de Lowell Fiet (https://dloc.com/fietmasks);

    la colección de fotos Catanzaro 1942 (https://dloc.com/diasporaproject/contains/?t=\letterpercent{}22August\letterpercent{}20Catanzaro\letterpercent{}22&f=AU);

    la colección de fotos del Carnaval (https://dloc.com/diasporaproject/contains/?t=\letterpercent{}22Camilo\letterpercent{}20Carrion\letterpercent{}22&f=AU); y

    las Fiestas Patronales de Vieques (https://dloc.com/AA00062845/00001).}.
\item
  Seleccionar un ítem del AHV que le genere curiosidad intelectual.
\item
  Redactar una reflexión en la que describe el material seleccionado y su importancia. La reflexión se puede enfocar en la relación que guarda el ítem seleccionado con los objetivos o los conceptos discutidos en el curso, el valor de que ese material se haya preservado, la importancia que tiene ese material para conocer la historia de Vieques, el valor que pueda tener ese ítem para otras personas que hacen investigación sobre Vieques o para una población particular (estudiantes, jóvenes, mujeres, maestros, viejos, personas de Vieques, Puerto Rico, del Caribe, de la diáspora boricua, etc.), o la reflexión se puede enfocar en lo que significa para el o la estudiante \quotation{descubrir} y estudiar ese material.
\item
  Grabar un audio de dos minutos en el que describe el ítem seleccionado y explica los aspectos sobresalientes de su reflexión (esto es lo que denominamos la reflexión sonora).
\item
  Compartir su reflexión sonora con la profesora y los otros estudiantes del curso a través del chat de WhatsApp de la clase.
\stopitemize

La \quotation{{\em Guía para el proyecto} {\em final”} incluye la descripción y los objetivos del proyecto; los enlaces a la plataforma dLOC, las colecciones del AHV y los recursos de referencia; las instrucciones de cómo realizar y compartir la reflexión sonora; y el consentimiento para divulgarla. En el proceso de diseñar esta actividad se tomó en cuenta el acceso limitado de los estudiantes a los recursos bibliotecarios en el contexto de la pandemia por el COVID-19. El diseño también responde al interés de la profesora en fomentar la participación de los estudiantes en el curso y el intercambio entre elles, dados los retos de transformar un curso presencial a uno asistido por la tecnología de manera precipitada durante la pandemia\footnote{Durante el semestre de agosto a diciembre de 2020 el curso {\em Sociedad y Cultura de Puerto Rico} se ofreció por primera vez de manera remota. Esta fue una experiencia de nuevos retos y posibilidades para la profesora y el estudiantado, quienes estaban aprendiendo juntos cómo facilitar y participar de un curso asistido por la tecnología. En el contexto de la pandemia por el COVID-19, las actividades del curso se fueron revisando y adaptando a la educación asistida por tecnología según transcurría el semestre. Las clases se llevaron a cabo en dos formatos: de manera sincrónica (reuniones en vivo de la profesora y estudiantes a través de la plataforma Google Meet) y de manera asincrónica (actividades de estudio independiente del estudiantado que se anunciaban a través del correo electrónico, WhatsApp y la plataforma Google Classroom). Se utilizó el chat de WhatsApp para que el estudiantado compartiera sus presentaciones y reflexiones, de manera que sus compañeros las escucharan y reaccionaran. La profesora exploró usar este medio como estrategia para contrarrestar la poca participación activa de los estudiantes en los encuentros sincrónicos y para estimular las interacciones entre estudiantes.}. Aunque la guía didáctica fue creada con estudiantes universitarios en mente, la misma puede adaptarse para ser usada por estudiantes de otros niveles académicos. La}{\em Guía para el proyecto} {\em final}” está disponible en el Apéndice de este artículo.

Las dieciséis reflexiones sonoras creadas por los estudiantes universitarios del curso {\em Sociedad y Cultura de Puerto Rico} son tan variadas como elles y como los materiales disponibles en el AHV. Una muestra de las reflexiones sonoras está disponible en el canal de YouTube {\em Sociedad y cultura puertorriqueña AHV}\footnote{La creación del canal de YouTube (\useURL[url5][https://www.youtube.com/channel/UCczoUXQQXtXtbBksyHSp2gA/featured?app=desktop]\from[url5]) estuvo a cargo de Edelmaris Figueroa, quien trabaja con Nadjah Ríos Villarini en el Archivo Histórico de Casa Pueblo.}{\em .} En cada uno de los siguientes enlaces a continuación se pueden escuchar las reflexiones sonoras creadas por el estudiantado. Los audios disponibles en el canal YouTube abordan los siguientes temas:

\startitemize
\item
  la vida cotidiana en Vieques durante el proceso de militarización de la isla en la década de 1940\footnote{\quotation{Reflexión Yarib Sánchez Nieves}, curso Sociedad y Cultura Puertorriqueña, profesora Ivelisse Rivera Bonilla, fecha de acceso: 14 de enero de 2022, \useURL[url6][https://www.youtube.com/watch?v=P00XVFnyPzI]\from[url6].}
\item
  la lucha del pueblo viequense contra la Marina de Guerra de los Estados Unidos\footnote{\quotation{Reflexión Yeira López}, curso Sociedad y Cultura Puertorriqueña, profesora Ivelisse Rivera Bonilla, fecha de acceso: 14 de enero de 2022, \useURL[url7][https://www.youtube.com/watch?v=GQXPUgpghB4]\from[url7].}
\item
  prácticas culturales como el carnaval y las fiestas patronales\footnote{\quotation{Reflexión Ingrid Cruz}, curso Sociedad y Cultura Puertorriqueña, profesora Ivelisse Rivera Bonilla, fecha de acceso: 14 de enero de 2022, \useURL[url8][https://www.youtube.com/watch?v=zkAWh3gmj6E]\from[url8].}
\item
  problemas sociales, ambientales y políticos que ha enfrentado la comunidad viequense\footnote{\quotation{Reflexión Alejandra Matos Vázquez}, curso Sociedad y Cultura Puertorriqueña, profesora Ivelisse Rivera Bonilla, fecha de acceso: 14 de enero de 2022, \useURL[url9][https://www.youtube.com/watch?v=-QrQ58qaTrc]\from[url9].}
\item
  la expropiación y rescate de terrenos\footnote{\quotation{Reflexión Lilliam M. Fonseca Rivera}, curso Sociedad y Cultura Puertorriqueña, profesora Ivelisse Rivera Bonilla, fecha de acceso: 14 de enero de 2022, \useURL[url10][https://www.youtube.com/watch?v=YjGplFZEVoM]\from[url10].}
\item
  arqueología en Vieques\footnote{\quotation{Reflexión Elvin D. Fontanéz Ubiles}, curso Sociedad y Cultura Puertorriqueña, profesora Ivelisse Rivera Bonilla, fecha de acceso: 14 de enero de 2022, \useURL[url11][https://www.youtube.com/watch?v=n67wZ0Ma7dk]\from[url11].}
\item
  las migraciones entre Vieques y otras islas del Caribe\footnote{\quotation{Reflexión José Manuel Sánchez Dávila}, curso Sociedad y Cultura Puertorriqueña, profesora Ivelisse Rivera Bonilla, fecha de acceso: 14 de enero de 2022, \useURL[url12][https://www.youtube.com/watch?v=0XMgZnqlNxQ]\from[url12].}
\stopitemize

La reflexión sonora sobre la vida cotidiana en Vieques durante el proceso de militarización de la isla en la década de 1940\footnote{\quotation{Reflexión Yarib Sánchez Nieves}.} describe la colección de más de 700 fotografías que miembros de la familia Catanzaro tomaron en Vieques entre 1941 y 1942. La colección está disponible en la plataforma dLOC\footnote{Digital Library of the Caribbean, \quotation{UPR: Caribbean Diaspora DH Center}, fecha de acceso: 14 de enero de 2022, \useURL[url13][https://dloc.com/diasporaproject/contains/?t=\%22August\%20Catanzaro\%22&f=AU][][https://dloc.com/diasporaproject/contains/?t=\letterpercent{}22August\letterpercent{}20Catanzaro\letterpercent{}22&f=AU]\from[url13].}. En la reflexión sonora del estudiante de comunicaciones Yarib Sánchez se describe que la colección incluye fotos de las condiciones de vida de los viequenses y fotos de la base militar y los militares. Estas fotos fueron confiscadas cuando la familia Catanzaro regresó a Estados Unidos. El estudiante que creó esta reflexión sonora se pregunta, ¿Por qué Estados Unidos no quería que la \quotation{historia gráfica} que documentaron los Catanzaro sobre el proceso de militarización se conociera? También reflexiona sobre cómo, al preservarlas, estas fotos \quotation{personales} de una \quotation{colección familiar} se convierten en registros de cómo se vivía en una época y sobre el valor de donar materiales de valor histórico al AHV.

\placefigure[here]{Desfile}{\externalfigure[issue06/valence_1_irb_nrv_foto4.png]}


La reflexión sonora sobre las migraciones entre las islas de Vieques y Santa Cruz describe los vínculos que existen entre ambas islas a nivel político, económico y sociocultural\footnote{\quotation{Reflexión José Manuel Sánchez Dávila}.}. También describe el rol de Vieques como puente de Puerto Rico con las Antillas Menores. La reflexión sonora del estudiante de ciencias sociales José M. Sánchez tiene como punto de partida un artículo escrito por Robert Rabin, actual director del Museo del Fuerte Conde de Mirasol y director del Archivo Histórico de Vieques. Sánchez describe que a principios del siglo XX inmigrantes de diferentes Antillas Menores migraron a Vieques a trabajar en los campos de cultivo de caña y que, luego de la crisis financiera de la década de 1930, el patrón de migración cambió de Vieques hacia las Islas Vírgenes Americanas:

\startblockquote
{\em El hecho de que los Estados Unidos {[}\ldots{}{]} otorgara la ciudadanía estadounidense a los puertorriqueños mediante el Acta Jones facilitó el que los desempleados emigraran sin mayores restricciones a las islas vecinas. Según relatos de personas de la época, en 1944, ya una vez culminó la construcción de {[}la base militar de la{]} Marina de Guerra de los Estados Unidos en Vieques, la situación empeoró. La emigración de familias enteras a la isla de Santa Cruz incrementó y muchos se fueron permanentemente.}
\stopblockquote

El estudiante reflexiona sobre el valor de visibilizar los nexos de Puerto Rico con las Antillas Menores, otro tema con el que los estudiantes puertorriqueños están poco familiarizados, y el valor de contar con documentación sobre esta en el Archivo Histórico de Vieques.

Aunque las reflexiones sonoras tienen unos componentes similares en cuanto a su formato, la voz de cada estudiante aporta un matiz particular a cada una. Hay audios que son persuasivos en su invitación al público a conocer los materiales del AHV, hay otros que transmiten la emoción de conocer más sobre la historia de Vieques, que es también la historia propia, y hay audios que transmiten la indignación del estudiantado ante los atropellos que ha sufrido la comunidad viequense. La dimensión sonora de las reflexiones tiene el potencial de acercar nuevos usuarios al AHV.

El proyecto \quotation{Historias de lucha y resistencia en Vieques} cumplió tres objetivos principales. El primero fue familiarizar a los estudiantes con fuentes primarias confiables e incentivar su curiosidad por la investigación de las mismas. Esto se dio en un contexto en el que el acceso del estudiantado del área este de Puerto Rico a los recursos bibliotecarios de la Universidad de Puerto Rico Humacao se vio limitado desde el que el huracán María azotó el archipiélago en 2017. El segundo objetivo que se cumplió estaba dirigido a estimular la creatividad del estudiantado, en un contexto de agotamiento y, en muchos casos, frustración con la educación asistida por la tecnología a nueve meses de iniciada la pandemia. El tercer objetivo del proyecto que se cumplió fue promover la empatía y la \quotation{solidaridad desde el conocimiento}\footnote{Rabin, \quotation{Archivo Histórico de Vieques}.} con la comunidad viequense. La experiencia de conocer sobre la historia de Vieques, sobre las luchas de sus habitantes en contra del militarismo y el abandono institucional y sobre los esfuerzos comunitarios por preservar su historia, especialmente aquella que es documentada por la propia comunidad son algunas de esas temáticas silenciadas en la historia oficial de Puerto Rico que se enseña en las escuelas y universidades. La exploración de estos temas invariablemente da pie a la pregunta, {\em ¿Por qué no había estudiado esto antes?} Es un despertar, a veces incómodo, pero siempre transformador. En palabras de los propios estudiantes:

\startblockquote
{\em Para mi estudiar sobre la historia de Vieques siempre ha sido algo que me llena de frustraciones, por las injusticias, pero más de admiración hacia el pueblo viequense, con alma de luchador incansable, décadas tras décadas.} (Alejandra Mattos, estudiante de comunicaciones)

{\em Es verdad, que en las escuelas hablan de esta parte de la historia puertorriqueña, pero sigue siendo desde la perspectiva de los opresores. {[}\ldots{}{]} Es importante visitar estas lecturas para conectar con ese lado de nuestra historia y no continuar creyendo que somos parte de los colonizadores.} (Luis Rivera, estudiante de ciencias sociales)
\stopblockquote

Como resultado de esas experiencias transformadoras, los estudiantes dieron su consentimiento para contribuir, con sus reflexiones sonoras, a divulgar la existencia del AHV, sus recursos y su valor para la preservación del patrimonio cultural viequense, puertorriqueño y caribeño. Con ello hay, por un lado, una valoración del esfuerzo de la comunidad viequense por preservar los fragmentos dispersos de su historia, pero también una apuesta a un futuro en el que las y los puertorriqueños conozcamos mejor nuestra propia historia.

\subsection[title={Posibilidades pedagógicas},reference={posibilidades-pedagógicas}]

A partir de la experiencia de realizar un proyecto de exploración y divulgación de los materiales del Archivo Histórico de Vieques disponibles en la Biblioteca Digital del Caribe (dLOC) en el contexto de la pandemia por el COVID-19 concluimos lo siguiente:

\startitemize
\item
  Nombrar y estudiar los temas silenciados en la historia oficial puertorriqueña aporta al desarrollo de una consciencia crítica del estudiantado universitario sobre la compleja trama cultural y política de la que somos parte.
\item
  Visibilizar los recursos del AHV promueve la curiosidad del estudiantado y, como resultado, el desarrollo de competencias de investigación (búsqueda de información, análisis, redacción, divulgación). Estas son competencias que promueven el compromiso de los estudiantes con su educación y aumentan las probabilidades de que persistan en sus estudios subgraduados y se gradúen de la universidad\footnote{J. Caraballo-Cueto, I. Godreau, R. Tremblay, \quotation{From Undergraduate Research to Graduation: Measuring the Robustness of the Pathway at a Hispanic-Serving Institution}, Journal of Hispanic Higher Education, (January 2022), doi:10.1177/15381927221074026}.
\item
  Explorar fuentes de información no tradicionales como los archivos comunitarios aporta a que el estudiantado aprenda a valorar otros saberes. Entre ellos los que se desprenden de la historia oral, una de las colecciones importantes del AHV.
\item
  En el contexto de la pandemia, el uso de la plataforma de mensajería WhatsApp para divulgar las reflexiones sonoras es una alternativa efectiva para sustituir las presentaciones orales en la sala de clases. A través de WhatsApp, todos los estudiantes del curso tienen acceso a las presentaciones de sus compañeros en cualquier momento y en cualquier lugar\footnote{Sobre el uso de WhatsApp en contextos educativos, ver L. Centikaya, \quotation{The Impact of WhatsApp Use on Success in Education Process}, {\em International Journal of Research in Open and Distributed Learning} 18, no. 7 (2017), \useURL[url14][https://www.researchgate.net/publication/321381328_The_Impact_of_Whatsapp_Use_on_Success_in_Education_Process]\from[url14].}.
\item
  El desarrollo de reflexiones sonoras por parte de los estudiantes y su publicación en YouTube expande la dimensión comunitaria del archivo. Al crear un segundo nivel de información sobre los materiales del AHV, los estudiantes se convierten en partícipes del desarrollo del archivo, a la vez que contribuyen a ampliar su acceso.
\item
  Las reflexiones sonoras sobre ítems del AHV facilitan que poblaciones diversas tengan acceso a los recursos del archivo. Los textos y audios disponibles en YouTube añaden una nueva capa de información que revitaliza la fuente original. Además, facilitan que personas con diversidad funcional tengan acceso al archivo.
\item
  Conectar recursos digitales existentes como las colecciones del AHV en dLOC y las conferencias \quotation{Coloqueo} del Instituto de Cultura Puertorriqueña con proyectos emergentes como las reflexiones sonoras de les estudiantes potencia cada uno de estos proyectos, ampliando su alcance y divulgación.
\stopitemize

En su reflexión sobre la importancia de visibilizar los archivos de los afectos que nos forman como seres senti-pensantes, Díaz Quiñones nos alerta que \quotation{la memoria no es siempre espontánea, requiere deseo y trabajo}\footnote{A. Díaz Quiñones, \quotation{El arca de Noé}, Lección Magistral, 2015, Universidad de Puerto Rico en Bayamón, \useURL[url15][https://arcadiodiazquinones.com/portfolio/el-arca-de-noe-2015/]\from[url15].}. Esa intención está vigente en el esfuerzo titánico que se gesta desde el Archivo Histórico de Vieques para coleccionar los fragmentos de la historia que heredamos, preservarlos, catalogarlos, digitalizarlos y facilitar su acceso a través de dLOC. Ese compromiso comunitario con el coleccionismo nos invita a mirar el pasado para entender mejor el presente. Desde nuestra gestión como docentes, al conectar al estudiantado con la memoria histórica y la reflexión crítica sobre el presente vamos, también, dándole forma al futuro.

\thinrule

\page
\subsection{Ivelisse Rivera Bonilla}

Ivelisse Rivera Bonilla es catedrática del Departamento de Ciencias Sociales de la Universidad de Puerto Rico en Humacao. Obtuvo su grado doctoral en el Departamento de Antropología de la Universidad de California, Santa Cruz. Coordina el Programa de Investigación de Acción Social. Sus investigaciones están basadas en los principios de la investigación-acción participativa. Ha realizado investigaciones con estudiantes y comunidades en Vieques, Humacao y con el grupo de teatro Agua, Sol y Sereno en temas de justicia social a través de la educación y las artes, participación ciudadana y la educación popular.

\subsection{Nadjah Ríos Villarini}

Nadjah Ríos Villarini es catedrática del Departamento de Inglés de la Facultad de Estudios Generales en la Universidad de Puerto Rico, recinto de Río Piedras. Obtuvo su grado doctoral en el Departamento de Antropología en la Universidad de Texas en Austin. Co-dirige el Proyecto Diáspora y el Centro de la Humanidades Digitales de la Universidad de Puerto Rico junto a la Dra. Mirerza González. Actualmente desarrolla varios archivos comunitarios, entre ellos el Archivo Histórico de Vieques, el Archivo de la Fundación Culebra y más recientemente el Archivo Histórico de Casa Pueblo.

\stopchapter
\stoptext