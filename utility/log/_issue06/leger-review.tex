\setvariables[article][shortauthor={Léger}, date={May 2022}, issue={6}, DOI={Upcoming}]

\setupinteraction[title={Rétrospective sur le site *Île en île*},author={Frenand Léger}, date={May 2022}, subtitle={Rétrospective}, state=start, color=black, style=\tf]
\environment env_journal


\starttext


\startchapter[title={Rétrospective sur le site {\em Île en île}}
, marking={Rétrospective}
, bookmark={Rétrospective sur le site *Île en île*}]


\startlines
{\bf
Frenand Léger
}
\stoplines


Je fais partie de la génération dont la naissance dans les années 1970 coïncide avec le développement des nouvelles technologies de l'information et de la communication. Cette génération de lecteurs, qui se situe à la charnière entre l'époque non encore révolue du texte imprimé et de celle du texte numérique, a l'avantage d'avoir pu se familiariser avec les différents supports textuels, qu'ils soient traditionnels ou modernes. S'il est vrai que la technologie numérique offre des possibilités éditoriales inédites, le livre papier présente des caractéristiques particulières qui en font un objet palpable plein de charme et d'attrait. A côté des multiples propriétés intéressantes de la technologie numérique et de ses diverses applications multimédias, exploitées à fond sur le site Internet {\em Île en île}, le maniement du livre imprimé a quelque chose de spécifique. Il induit un type de jouissance d'une dimension que seule la nature humaine peut appréhender.

En plus des sensations et des émotions que suscite d'habitude l'information véhiculée par le texte, on peut également prendre plaisir à manipuler le livre, à contempler sa reliure, à sentir l'odeur de l'encre et du papier, à feuilleter et annoter les pages, et surtout à posséder un objet concret auquel on s'y attache et qu'on peut conserver dans sa bibliothèque. Tant que le texte numérique demeure incapable de satisfaire les besoins que comble le livre imprimé, les deux formats de textualité semblent être appelés à coexister en harmonie, soit à se compléter plutôt qu'à se supplanter. En ce sens, on pourrait alors soutenir que le site Internet {\em Île en île} s'inscrit habilement dans cette vision de complémentarité du texte imprimé et du texte numérique pour assurer une plus grande visibilité aux littératures des îles francophones.

{\em Île en île}, c'est quoi exactement ? Pourquoi avoir construit un tel site web ? A quoi sert-il ? Que représente-t-il pour celles et ceux qui s'intéressent sérieusement aux patrimoines littéraires francophones insulaires~du Sud ? Pour trouver des réponses très éloquentes à ces questions, je vous invite à aller directement à la source, c'est-à-dire à visiter le site {\em Île en île}, particulièrement la section «~A propos~» où il y a un dossier-bilan contenant des propos de Thomas C. Spear recueillis par Stève Puig à l'occasion du dixième anniversaire du site. Cette exploration vous permettra non seulement de vous informer sur les tenants et aboutissants de ce beau projet, mais aussi de découvrir par vous-mêmes la richesse de cette mine de documents écrits et audiovisuels accessibles gratuitement en ligne. Cela dit, je propose de me pencher quand même sur la dernière question, car elle me permet de parler un peu de mon expérience personnelle avec le site. J'ai découvert {\em Île en île} au début des années 2010. Je venais à peine de commencer ma scolarité dans le programme de doctorat en études françaises et francophones à l'Université de Toronto. Comme la plupart de mes camarades qui ont fait leurs études au moment de la transformation numérique des universités, j'étais très actif sur l'Internet. C'est ainsi que je suis tombé par hasard sur le site {\em Île en île}. Lorsque je rédigeais les chapitres de ma thèse et que j'avais rapidement besoin de certaines informations factuelles sur un auteur ou sur une œuvre francophone, {\em Île en île} était le site web que je consultais en premier et j'étais le plus souvent satisfait.

Ma relation avec {\em Île en île} était dans un premier temps celle d'un simple utilisateur du site. L'étudiant avide de connaissance que j'étais a bien profité de toutes les données fiables et utiles fournies à travers des dizaines de dossiers. Ces dossiers d'auteurs qui regorgent d'informations biographiques et bibliographiques, mais aussi analytiques, m'ont beaucoup aidé à progresser dans mes recherches. A mesure que j'avançais dans mes études et que j'approfondissais mes connaissances dans le domaine, j'éprouvais le besoin de me rendre utile à mon tour. Je ressentais en fait le besoin d'offrir quelque chose en retour pour exprimer ma gratitude envers {\em Île en île}. C'est ainsi que j'ai pris la décision d'écrire à Thomas pour lui proposer de collaborer avec lui et son équipe. Je me rappelle lui avoir offert une contribution ciblée, non pas sur les auteurs contemporains, puisqu'ils étaient déjà légion sur le site, mais sur des auteurs haïtiens du XIXe siècle.

Dans l'un des chapitres de ma thèse, j'examine le rôle que la critique (qu'elle soit française ou haïtienne) a joué dans la dévalorisation des productions littéraires haïtiennes du XIXe siècle. Les écrivains haïtiens du XIXe siècle sont pour la plupart victimes d'une critique guidée par une lecture euro-centriste empreinte de préjugés, ce qui a servi à l'oblitération~dans la mémoire collective~de l'intérêt fondamental dont ils devraient être l'objet. Ce sont des publicistes français qui ont d'abord véhiculé l'idée selon laquelle le corpus des lettres haïtiennes du XIXe siècle est en grande partie «~un pâle reflet de la littérature française\footnote{Voir Alexandre Bonneau, « Les Noirs, les Jaunes et la littérature française en Haïti », {\em La Revue contemporaine}, Paris, 1er décembre 1856, p.~108 et Jean Price Mars, {\em De Saint Domingue à Haïti : Essai sur la culture, les arts et la littérature}, Paris, Présence Africaine, 1959, p.~103.} ». Ma recherche vise à réhabiliter un certain nombre de ces œuvres fondatrices extrêmement importantes pour Haïti et pour le monde francophone en général. Je dois dire que ce sont les dossiers d'Yves Chemla sur Anténor Firmin et Louis Joseph Janvier qui m'ont servi de modèles pour préparer entre autres ceux sur Ignace Nau et Thomas Madiou. J'en profite pour remercier Yves pour la facture de ses dossiers sur {\em Île en île} et pour la grande qualité de ses travaux sur la littérature haïtienne.

Voilà les circonstances dans lesquelles j'ai écrit à Thomas pour lui proposer de participer à la production de cette masse d'informations méthodiques basées sur des recherches académiques menées par des spécialistes de haut calibre. Si la rigueur intellectuelle est de mise sur le site {\em Île en île}, l'accessibilité n'en reste pas moins le maitre-mot. Des ressources accessibles en ligne librement et en tout temps~y sont en effet présentées sans langue de bois dans un discours de type informatif agrémenté d'éléments graphiques et audiovisuels rendu possible grâce à l'environnement numérique. L'entrée en matière au début de ma rétrospective a précisément servi à situer le projet {\em Île en île} dans le contexte du développement des nouvelles technologies sans perdre de vue sa principale mission, notamment celle de promouvoir la production littéraire des îles francophones ou les livres écrits par les intellectuels de ces pays moins nantis.

Pour être plus précis et en même temps plus englobant, disons que, depuis le lancement du site en octobre 1998, Thomas et ses collaborateurs n'ont jamais cessé de l'enrichir au point qu'il est devenu aujourd'hui, avec ses 400 dossiers, son index littéraire et son vaste catalogue recensant les coordonnées des principales institutions, organisations et agences culturelles des îles francophones, un espace documentaire numérique essentiel pour la promotion et la sauvegarde des œuvres littéraires d'expression française d'auteur·e·s issu·e·s des îles anciennement colonisées par la France. Nous ne sommes pas sans savoir que la grande majorité de ces auteurs d'une littérature dite de la « périphérie~du Sud » ne bénéficient pas des dispositifs de soutien technique et financier disponibles dans le monde éditorial de langue française des pays du Nord\footnote{Voir l'article \quotation{Le nécessaire rééquilibre du marché du livre francophone} de Caroline Montpetit, dans {\em Le Devoir}, Montréal, juin 2021.}. Comme le dit si bien Thomas, «~Il fallait donner vie à {\em Île en île}, lui donner corps, l'assurer. The City University of New York (CUNY) offre l'espace internet, un lieu sans mur ni clôture où tous les "petizarts"~et les "islomaniaques" peuvent découvrir et faire découvrir les patrimoines insulaires.\footnote{Extrait de la section \quotation{A propos} du site web {\em Île en île}.} » C'est justement son caractère inclusif et démocratique qui m'a le plus attiré et qui m'a incité à prendre contact avec Thomas pour lui offrir ma collaboration.

Je voudrais finir en exprimant ma gratitude, car je suis encore heureux d'avoir pu profiter des ressources d'{\em Île en île} lorsque j'étais étudiant et je suis également très fier d'avoir pu apporter mon humble contribution pour faire du site ce qu'il est aujourd'hui, c'est-à-dire une référence incontournable pour les amateurs de littératures insulaires d'expression française. Souhaitons longue vie à {\em Île en île} en espérant un support technique et financier plus important de la part des bailleurs de fonds francophones et francophiles. C'est ce qui manque le plus actuellement à l'équipe afin d'assurer la relève et de garder vivante cette belle et noble initiative culturelle.

\thinrule

\page
\subsection{Frenand Léger}

Before joining Carleton University in 2016, Léger taught French and Haitian-Creole at Indiana University while completing a M.A.~program in French Instruction (applied linguistics). He received his PhD in French/Francophone literature at the University of Toronto, where he also taught a number of French language and literature courses. His research in Francophone literature examines interactions between Creole orality and French writing in Haitian novels and short stories. It concentrates on a substantial body of fictional literary texts written by major Haitian novelists and short story writers from the nineteenth century to the present day. In his future research, he plans to explore the great influence of Haitian literature on Francophone writers of the global South, particularly on Caribbean and African writers. In the field of French didactics as a Foreign/Second Language, he is working with other colleagues on the creation of textbooks based on the principles of the Common European Framework of Reference for Languages (CEFRL), that take into consideration the great diversity of the French language and cultures throughout Canada and the world.~

\stopchapter
\stoptext