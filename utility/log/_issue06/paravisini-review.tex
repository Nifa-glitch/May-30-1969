\setvariables[article][shortauthor={Paravisini-Gebert}, date={May 2022}, issue={6}, DOI={Upcoming}]

\setupinteraction[title={*Île en île*: A Digital Voyage of Discovery},author={Lizabeth Paravisini-Gebert}, date={May 2022}, subtitle={A Digital Voyage of Discovery}, state=start, color=black, style=\tf]
\environment env_journal


\starttext


\startchapter[title={{\em Île en île}: A Digital Voyage of Discovery}
, marking={A Digital Voyage of Discovery}
, bookmark={*Île en île*: A Digital Voyage of Discovery}]


\startlines
{\bf
Lizabeth Paravisini-Gebert
}
\stoplines


Long before the launch of {\em Île en île} in 1998, I remember Thomas as someone focused on the new technologies surfacing around us (personal computers, email, the internet!) as potential solutions to the conditions in which we conducted research on what were then the \quotation{marginal} fields of Caribbean, francophone, and post-colonial literatures. We were then---in the mid 1980s---young colleagues at Lehman College (CUNY) who, intrigued by the availability of personal computers, gravitated to the computer center in search of insights into what they could do to bridge over the gap between our fields and the unavailability of primary materials for study. (As faculty members in the humanities, we were indeed very {\em rarae aves} in Lehman's computer center.) But at that time bookstores rarely carried titles relevant to our research, and never those written in languages other than English; translations were scarce, and library holdings had scant resources on Caribbean and postcolonial literatures. Research required traveling to special collections and long hours spent photocopying or, when photocopying was not allowed, hours that stretched into days and weeks copying texts by hand. Literary histories needed to be traced, copies of rare editions of books needed to be located, texts printed in ephemeral magazines had to be unearthed---all treasures to be discovered to set the foundations for emerging fields. These forays far and wide in search of materials were not without adventures. I recall in particular a morning we spent together browsing in the bookstores of Paris' Boulevard Saint-Michel looking for francophone novels---after which Thomas had to rescue me from a threatened arrest. But that is a story best left for another place and time.

The pressure to secure materials for teaching and scholarship was only matched by the dearth of translations. The Caribbean---the area on which my own research has been based---was then an even more linguistically fragmented space where writers across the archipelago rarely knew one another's work. This was a serious area of concern to Thomas, whose vision of francophone literature was both archipelagic and global. The late 1980s and early 1990s were marked by the attention to translation as a necessary step in developing more holistic, multidisciplinary, from-the-ground-up theories and methodologies to make sense of post-colonial literary traditions. And here Thomas also had his finger on the pulse of the problem. A gifted translator himself, Thomas kindly contributed translations when I was working with our mutual colleague Carmen Esteves on bringing together the short stories for {\em Green Cane and Juicy Flotsam} (1991), but also generously used his (even then) considerable contacts to help us locate and correspond with authors to obtain the necessary translation rights and permission for publication. His help was invaluable, and in this help we could already glimpse the talents, interests, contacts, generosity, persistence, and drive that would make possible {\em Île en île} possible a few years later. We were not, by any stretch of the imagination, the only people he helped in this way. This potential for collaboration and generosity were, I think, basic qualities behind the launching of {\em Île en île}.

I feel a deep affection for Thomas, as we have been friends for at least three and a half decades. And at the foundation of that friendship has been my admiration for the dedication and work ethic behind everything he does---and of which {\em Île en île} is the most public manifestation. We bonded over obsessions with texts that could not be easily found---Marie Chauvet's works, among them---and critical articles that seemed mythical in their elusiveness. We had met at the height of the AIDS epidemic, when the losses among our friends and colleagues seemed unbearable, and looking for books, poems, stories sometimes felt like a quest for the ghosts of those lost. An effort at recovering what should not have been ephemeral felt somewhat like wrenching poetic justice out of the grips of the established canon. I always think back to those days as planting the seed for the commitment Thomas brought to {\em Île en île}, for the indefatigable work of recovering, archiving, and preserving voices, images, texts, stories that the canonical archive would never have thought worth the effort to preserve.

I have watched {\em Île en île} develop through the years with a sense of awe, perhaps because I have had privileged glimpses into the very hard work through which the site grew from website to vital archive with so very little support except from friends and colleagues. The polished and elegant site belies the intense labor, frustration, and exasperation behind its production. The lack of institutional and grant support has been lamentable---a product perhaps of an indifference on the part of grant agencies and academic institutions to what were considered \quotation{marginal} literatures in the early days of the construction of the archive, coupled with its development during a time when academia had not yet figured out the value behind what is now celebrated as \quotation{the digital humanities.} It is ironic that one of the most valuable digital humanities sites in our field had to be created and supported with little institutional support.

The greater reason, then, to celebrate Thomas' very impressive achievement in the creation of {\em Île en île} and its growth and maintenance over more than twenty years. The site was unique in its creation at a time when developing such a site required more technical knowledge and effort than it would need today. But it was in imagining at such an early date that the form most suited to the purpose of bringing together literary and creative communities separated by geographies and languages was through a digital route, that the future would indeed be digital, that {\em Île en île} was so ahead of its time. One should remember that {\em Île en île} was launched in the same year as the founding of Google, with the internet perhaps not in its infancy, but certainly still not in its prime. The technical work behind the site was onerous and time consuming at the time and the site only grew in technical complexity as new features were added. Yet the technological challenges allowed for a creation of an archive of great importance and practical utility---an invaluable tool for research and teaching whose added value comes from the fact that it is not merely built on collecting materials but went the huge extra step to create an archive of new materials to capture what would otherwise would have been ephemeral and lost to future generations.

Thomas' work ethic, his perseverance and satisfaction in discovering new materials, securing yet another interview, adding a new writer to his archive, is why the site he created is of such central value to present and future scholars. Throughout the last year, as Thomas was preparing the site to be archived---all done during a time when access to collections was severely limited because of Covid---I have watched the resolve with which Thomas has worked to assure that any remaining questions on the site were answered, that every quotation was correct and verified, that any materials that needed to be consulted were accessed. In short, I saw him make sure that the materials archived were as complete and accurate as they could be. For me, there is in this commitment to exactitude and quality a measure of Thomas' true dedication to his work as a scholar, translator, interviewer, and archivist. These were qualities already there when we met some thirty-five years ago. And the very same qualities for which I have always been so glad to count him as a friend.

\page
\subsection{Lizabeth Paravisini-Gebert}

Lisa Paravisini-Gebert works in the fields of literature and cultural studies, specializing in the multidisciplinary, comparative study of the Caribbean. Growing up in her native Puerto Rico, she became fascinated by the many cultural connections between Caribbean peoples despite our different histories and languages and has made that the subject of her research and teaching. She is based in the Hispanic Studies Department at Vassar College, where she holds the Randolph Distinguished Professor Chair. She is also a participating faculty member in the Programs in Environmental Studies, Latin American Studies, International Studies, and Women's Studies at Vassar. She is the author of a number of books, among them {\em Phyllis Shand Allfrey: A Caribbean Life} (1996), {\em Jamaica Kincaid: A Critical Companion} (1999), {\em Creole Religions of the Caribbean} (2003, with Margarite Fernández Olmos), and most recently, {\em Literatures of the Caribbean} (2008).~

\stopchapter
\stoptext