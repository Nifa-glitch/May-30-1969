\setvariables[article][shortauthor={Levi, Inniss}, date={January, 2020}, issue={4}, DOI={https://doi.org/10.7916/archipelagos-5k90-bd50}]

\setupinteraction[title={Decolonizing the Archival Record about the Enslaved: Digitizing the *Barbados Mercury Gazette*},author={Amalia S. Levi, Tara A. Inniss}, date={January, 2020}, subtitle={Decolonizing the Archival Record}]
\environment env_journal


\starttext


\startchapter[title={Decolonizing the Archival Record about the Enslaved: Digitizing the {\em Barbados Mercury Gazette}}
, marking={Decolonizing the Archival Record}
, bookmark={Decolonizing the Archival Record about the Enslaved: Digitizing the *Barbados Mercury Gazette*}]


\startlines
{\bf
Amalia S. Levi
Tara A. Inniss
}
\stoplines


{\startnarrower\it This article discusses the digitization of the historic newspaper~the {\em Barbados Mercury Gazette} (1783--1848) and its potential for historical research and teaching. The newspaper sheds light on every aspect of life in a dystopian colonial reality. The digitization of the gazette is a beginning, rather than a "finished product." The aim of the project is not simply to facilitate reading the gazette in a digital format; if so, then digitization would risk reproducing and re-inscribing inequalities, silences, and exclusions. The aim is to engage users, particularly Barbadians, in researching their enslaved ancestors. At the same time, the article reflects on the meaning of digitization in postcolonial settings in the global South and the challenges arising at the intersection of access and conservation. While digitization is promoted as digital preservation, it cannot save what is not accounted for, what is not described properly, and what has not been documented. \stopnarrower}

\blank[2*line]
\blackrule[width=\textwidth,height=.01pt]
\blank[2*line]

The {\em Barbados Mercury and Bridgetown Gazette} (henceforth, the {\em Mercury}) was digitized in 2018 at the Barbados Archives.\footnote{The title of the newspaper changed throughout the years, beginning in 1783 as the {\em Barbados Mercury} and changing to the {\em Barbados Mercury and Bridgetown Gazette} circa 1805. For more contextual information about the history of the newspaper, please see "The {\em Barbados Mercury and Bridgetown Gazette} Newspaper Collection," British Library Endangered Archives Programme, \useURL[url1][https://eap.bl.uk/collection/EAP1086-1]\from[url1].} Digitization was funded through an Endangered Archives Programme (EAP) grant~by the British Library.\footnote{See "Welcome to the Endangered Archives Programme," \useURL[url2][https://eap.bl.uk]\from[url2]. Established in 2004, the EAP awards grants for identifying and digitizing culturally important "material that is at risk of loss or decay, and is located in countries where resources and opportunities to preserve such material are lacking or limited" ("About the Programme," \useURL[url3][https://eap.bl.uk/about]\from[url3]). The EAP makes available digitized material through local partners and its website.} The award, the first of its kind for Barbados, was the result of an international collaboration between scholars and practitioners in Barbados, the United States, and Canada. Three people came together to make this project happen: Lissa Paul, Brock University, who during her own research trips to Barbados recognized the importance of the newspaper as a source of information about the enslaved and was alarmed at the rate at which the microfilmed copies she was using were deteriorating;\footnote{See Heather Junke, "Brock Prof Part of International Mission to Save Caribbean History," {\em The Brock News, a News Source for Brock University}, 25 October 2017, \useURL[url4][https://brocku.ca/brock-news/2017/10/brock-prof-part-of-international-mission-to-save-caribbean-history]\from[url4] (accessed 15 December 2019).} Ingrid Thompson, director of the Barbados Archives, who provided permission for digitization as well as space and staff time as in-kind support; and Amalia S. Levi, who, as an archivist conducting archival processing and digitization projects in Barbados, had been involved with the Barbados Archives. The catalyst for this serendipitous collaboration was Laurie Taylor, the digital scholarship director of the~\useURL[url5][https://dloc.com/][][Digital Library of the Caribbean]\from[url5] (dLOC), who knew each person individually and recognized the opportunity for digital preservation of this important primary source.\footnote{To learn more about the collaboration, see "Collaboration across Disciplines to Make a Path Where None Existed" (panel presented at ACURIL 2017: Multidisciplinary Research in the Caribbean, 4--8 June 2017, San Juan, Puerto Rico), \useURL[url6][https://dloc.com/AA00054636/00001/citation][][dloc.com/AA00054636/00001/citation]\from[url6]. ACURIL is the Association of Caribbean University, Research, and Institutional Libraries.}

As a primary source covering the period from 1783 to 1848, the {\em Mercury} offers a sweeping view of every aspect of life in a British colony in the Caribbean during the eighteenth and nineteenth centuries. These were formative decades for the island, leading up to and following the 1816 slave rebellion led by the African-born slave Bussa. Although the rebellion was quickly put down, it was part of a series of rebellions in the West Indies during the nineteenth century that led to the abolition of slavery~in British colonies in 1834.

\placefigure[here]{A pristine issue of the {\em Barbados Mercury Gazette} August 1815}{\externalfigure[issue04/levi-inness/LeviFig1.png]}


Scholars have mined the newspaper for information covering many different arenas, including maritime history, economic history, military history, social and cultural history, material culture, and geography. The newspaper has been especially useful in reconstructing this period in the island's slave society and economy, since sources that otherwise would provide insight---such as a full catalogue of plantation records or parish or local government records---are not readily available.\footnote{See Jerome S. Handler, {\em A Guide to Source Materials for the Study of Barbados History, 1627--1834} (Carbondale: Southern Illinois University Press, 1971).}

Among other information contained in the newspaper's pages, of particular importance are the "runaway slave" ads, which appear regularly. We usually find precious little descriptive information about the enslaved in archives because their lives were~rarely recorded; when the enslaved do appear, they were enumerated or appraised as commodities. These ads thus offer a relatively rich trove of information about individual people, including his or her name, age, physical appearance, skin color, clothing, accent, distinguishing features (such as body modifications from his or her country of origin or signs of violence that had resulted in bodily harm), friends, relatives, and skills.

\subsection[title={Overview: The Digitization of the {\em Mercury}},reference={overview-the-digitization-of-the-mercury}]

The digitization project entailed roughly three phases: (a) the planning phase, during which we worked through project logistics and also organized a workshop to introduce the project and to involve local heritage professionals in contextualizing the {\em Mercury}; (b) the training phase, during which local personnel were provided with professional development under the guidance of a University of Florida technical team, part of the grant's stipulation for local capacity building; and (c) the execution phase, during which we digitized the {\em Mercury} in accordance with the EAP guidelines and organized an event to formally launch the digitized project.\footnote{To learn more about each phase, see Amalia S. Levi, "Digitisation of the {\em Barbados Mercury Gazette}," {\em Endangered Archives Blog}, 3 August 2018, \useURL[url7][https://blogs.bl.uk/endangeredarchives/2018/08/digitisation-of-the-barbados-mercury-gazette.html]\from[url7].}

In total, twenty volumes---covering twenty-eight years between 1783 and 1848 and including 2,333 issues with four pages per issue---were digitized at three hundred dots per inch. This work resulted in nine thousand digital images at approximately 103 megabytes each, for a total of 1 terabyte of data. We produced an Excel spreadsheet with nearly three thousand lines of metadata describing the newspaper at the collection level, as a whole, and at the series level, considering each year as a series.

The EAP follows the post-custodial archival model, meaning that physical custody of the {\em Mercury} remained with the Barbados Archives.\footnote{See Maja Kominko, "Crumb Trails, Threads, and Traces: Endangered Archives and History," in {\em From Dust to Digital: Ten Years of the Endangered Archives Programme}, ed.~Maja Kominko (Cambridge: Open Book, 2016), xlix--lxviii, \useURL[url8][https://books.openedition.org/obp/2216]\from[url8]. To read more about the post-custodial model, see Christian Kelleher, "Archives without Archives: (Re)Locating and (Re)Defining the Archive through Post-Custodial Praxis," {\em Journal of Critical Library and Information Studies} 1, no. 2 (2017), \useURL[url9][http://doi.org/10.24242/jclis.v1i2.29][][doi:10.24242/jclis.v1i2.29]\from[url9].} After digitization, three sets of archival masters were produced in three external hard drives: one drive is stored at the Barbados Archives, one was sent to the EAP at the British Library, and one set was sent to the Digital Library of the Caribbean.\footnote{The {\em Mercury} online can be accessed either through the EAP portal at \useURL[url10][https://eap.bl.uk/project/EAP1086]\from[url10] or through the dLOC portal at \useURL[url11][https://www.ufdc.ufl.edu/AA00047511/02334/allvolumes]\from[url11].}

\subsection[title={Endangered Archives in Postcolonial Settings},reference={endangered-archives-in-postcolonial-settings}]

Recent natural disasters, such as the 2010 earthquake in Haiti, and catastrophic hurricanes, such as María in 2017, make clear that the severe effects of climate change have become the new reality. Furthermore, manmade disasters, such as the destruction of the National Museum of Brazil on 2 September 2018, demonstrated how easy it is for a nation to lose its history in flames in a matter of minutes.\footnote{Hugh Eakin, "We're All in Danger of Watching Our History Go Up in Flames," {\em Washington Post}, 5 September 2018, \useURL[url12][https://www.washingtonpost.com/opinions/our-art-has-burned-for-centuries-it-will-go-up-in-smoke-again/2018/09/05/53702182-b11b-11e8-a20b-5f4f84429666_story.html]\from[url12].} Before that, on 25 March 2018, a fire had destroyed nearby St.~Lucia's Folk Research Center. Within the structure, according to local press, was "the most extensive collection of the island's cultural artefacts and history," proving that institutions, particularly in the Caribbean, are highly vulnerable to such disasters and the value of their collections is only realized once they have been lost.\footnote{"Fire Destroys Building Housing St.~Lucia's Most Extensive Collection of Cultural and Historical Artefacts," {\em Pride News} (blog), 27 March 2018, \useURL[url13][https://pridenews.ca/2018/03/27/fire-destroys-building-housing-st-lucias-extensive-collection-cultural-historical-artefacts/]\from[url13].}

In many instances, digitization projects are conducted in adverse or dangerous conditions on the ground (ranging from political unrest to lack of basic necessities) created by natural or manmade disasters. Such projects can seem like a race against time. For example, many of the records digitally preserved by the \useURL[url14][https://www.slavesocieties.org/][][Slave Societies Digital Archive]\from[url14] (formerly the Archive of the Ecclesiastical and Secular Sources for Slave Societies) have been damaged considerably, while others have disappeared, having been either misplaced or stolen.\footnote{See \quotation{About the Archive,} {\em Slave Societies Digital Archive}, \useURL[url15][https://www.slavesocieties.org/node/23]\from[url15] (accessed 14 December 2019).} Often, because of emergency conditions or funding availability, such rapid \quotation{guerilla digitization} necessarily prioritizes access and digital capture over conservation.\footnote{Angela Sutton, "The Digital Overhaul of the Archive of Ecclesiastical and Secular Sources for Slave Societies (ESSSS)," {\em sx archipelagos}, no. 2 (July 2017), \useURL[url16][https://smallaxe.net/sxarchipelagos/issue02/essss.html]\from[url16].}

Digitization has been endorsed as an accepted preservation reformatting option by various professional organizations for a range of materials, including newspapers.\footnote{See Kathleen Arthur et al., "Recognizing Digitization as a Preservation Reformatting Method," {\em Microform and Imaging Review} 33, no. 4 (2004): 171--80.} Many, though, have lamented the fact that reformatting of historical newspapers through microfilming or digitization frequently has been seen as an opportunity to destroy the hard copies afterward.\footnote{See Nicholson Baker, "Discards," {\em New Yorker}, 4 April 1994, \useURL[url17][https://www.newyorker.com/magazine/1994/04/04/discards]\from[url17].} Others have criticized this tendency to romanticize physical newspapers that disregards the necessity of selectivity owing to tangible reasons, such as funding, preservation, and space.\footnote{Richard J. Cox, "The Great Newspaper Caper: Backlash in the Digital Age," {\em First Monday}, 4 December 2000, \useURL[url18][https://doi.org/10.5210/fm.v5i12.822][][doi:10.5210/fm.v5i12.822]\from[url18].} However, it has been proven that original newspapers still retain their importance as master copies for generating better reproductions as technology and scholarly expectations change; for regenerating copies in case of corrupted digital or microfilm copies; for correcting faulty copies; or for supporting historical research regarding the materiality of the newspaper, its usage, and its production techniques.\footnote{See Randy Silverman, "Retaining Hardcopy Papers Still Important in Digital Age," {\em Newspaper Research Journal} 36, no. 3 (2015): 363--72, \useURL[url19][https://doi.org/10.1177/0739532915600749][][doi:10.1177/0739532915600749]\from[url19].}

As important as such digital resources are for access, they must be managed within a conservation plan for often-decaying records. Digitization itself cannot save what is not protected. When researchers access the {\em Mercury}'s pages on either DLoC or the British Library website, researchers are opening seemingly pristine and enhanced images. Stripped from their materiality during the digitization process, they are now a far cry from the original documents, which are extremely fragile and vulnerable to adverse environmental conditions. Moreover, in small countries with still developing economies like Barbados (which is currently undergoing a prolonged economic recession), there is not enough funding for the staffing of cultural institutions like the archives, let alone for costly conservation measures, which usually involve expensive and slow conservation work done locally or the equally expensive strategies of either bringing in expert conservators or sending documents for conservation overseas.

\placefigure{Dust and pieces of paper}{\externalfigure[images/levi-inness/LeviFig2.png]}
Many regional archives do not have any digitization policies at all, let alone digitization policies that include conservation. Digitization helps to provide wider access to these documents if available on websites or digital platforms, which helps archivists to prevent overuse of fragile materials. But digitization is not conservation. And unfortunately, some policymakers (not trained archivists) believe that digital records can replace original records, hence that digitization represents a cost savings to archives and libraries in terms of storage and conservation.\footnote{Recent government initiatives in Barbados suggest that there is a move to digitize public records on a large scale with the help of Inter-American Development Bank (IDB) funding. However, it is not clear how this policy will dovetail with a larger conservation policy. See Emmanuel Joseph, "PM: I'm Sorry," {\em Barbados Today}, 23 October 2018, \useURL[url20][https://barbadostoday.bb/2018/10/23/pm-im-sorry]\from[url20].}

Funding for digitization that focuses on digital reformatting of originals and online access without funding for the conservation needs of the physical copies presents an additional question to regional recipients of funding---though the archive is now accessible, is it truly no longer endangered?

Archives in postcolonial contexts are endangered because there has not been enough investment in policy, human resources, or conservation strategies to maintain these records on the part of governments and memory institutions. Furthermore, by funding short-term, temporary work in return for shared access to digital copies, post-custodial digitization grants made available by institutions in the global North end up "replicating or enabling the neoliberal structures of transnational inequality," thus "reifying a neoliberal North-South power imbalance."\footnote{Hannah Alpert-Abrams, David A. Bliss, and Itza Carbajal, "Post-Custodialism for the Collective Good: Examining Neoliberalism in US-Latin American Archival Partnerships," in "Evidences, Implications, and Critical Interrogations of Neoliberalism in Information Studies," ed.~Marika Cifor and Jamie A. Lee, special issue, {\em Journal of Critical Library and Information Studies} 2, no. 1 (2019), \useURL[url21][https://doi.org/10.24242/jclis.v2i1.87][][doi:10.24242/jclis.v2i1.87]\from[url21].}

Therefore, in addition to funding digital projects, policies to support conservation alongside digitization should also be a part of these grants. Such support should also be extended to digitized files. Digitization itself is not a panacea if robust digital preservation protocols and processes are not in place. Digitized collections can be effective for disaster preparedness and mitigation only if the long-term preservation of files and associated metadata can be ensured through infrastructures, such as digital archives, digital libraries, and digital repositories embedded in robust communities.\footnote{Brooke Wooldridge and Laurie Taylor, "The Role of Digital Libraries in Disaster Preparedness and Mitigation" (paper presented at ACURIL 2011: The Role of Libraries and Archives in Disaster Preparedness, Response, and Research, 30 May--3 June 2011, University of South Florida, Tampa), \useURL[url22][https://scholarcommons.usf.edu/cgi/viewcontent.cgi?article=1015&context=acuril_2011]\from[url22].} Such communities can come together in case of disasters to support each other by sharing resources and aiding to retrieve lost physical materials.

The advocacy for these policies and movement for investment in archives is also a space for which we see graduates of the Department of History and Philosophy at the University of the West Indies (UWI), Cave Hill Campus, playing a role, since a number of them will go on to work in policymaking branches of government as educators or archivists, as information managers, and perhaps even as conservators.

\subsection[title={Reflections on the Digitization Process},reference={reflections-on-the-digitization-process}]

Digitization of the {\em Mercury} allowed us to come into close contact with the archival material, its informational content, and the materiality of the medium. Although we had to keep to a specific timeline with respect to the deliverables of the project, it was impossible for our eyes not to linger on news and announcements that reflected the dystopian reality of the slave society from which the materials emerged. It was also impossible not to interact in a very physical way with the vulnerability of these records: their fragile pages, the pieces that flew through the air or accumulated on the floor, often disintegrating despite our efforts to handle them properly.\footnote{For more on the process through the thoughts of the digitization team, see Brian Inniss and Lenora Williams, "The {\em Barbados Mercury}: Thoughts from the Digitisation Team," {\em Endangered Archives Blog}, 19 March 2019, \useURL[url23][https://blogs.bl.uk/endangeredarchives/2019/03/digitising-the-barbados-mercury-.html]\from[url23].}

Beyond the technical scope of the project, then, digitization, as a process, allowed us to reflect on its many facets.

\subsubsection[title={Digitization as collaboration},reference={digitization-as-collaboration}]

As an international project, our work on the {\em Mercury} gave us the opportunity to learn a number of crucial lessons. Collaboration is a balancing act that requires all parties to be clear about, define, and adhere to expectations and to navigate different perspectives---both cultural and professional. The process was guided by a series of small workshops with local historians, policymakers and archivists, librarians, and museum practitioners. The aim of these workshops was to share details about the project but also to begin to shape ideas for future projects and to discuss the role digitization can play in the development of national archives.\footnote{For more information about the workshops, see "The {\em Barbados Mercury Gazette} Digitization Launch Workshop---December 12, 2017," \useURL[url24][https://dloc.com/AA00061880/00001/citation]\from[url24]; "The {\em Barbados Mercury Gazette} Digitization Training Workshop," \useURL[url25][https://dloc.com/AA00063750/00001/citation]\from[url25]; and "The {\em Barbados Mercury Gazette}: Event to Celebrate the Launch of the Newspaper Online," \useURL[url26][https://dloc.com/AA00066602/00001/citation]\from[url26].}

These workshops provided fertile ground for historians, some of whom had never previously had unobstructed access to the newspaper because of its fragile condition and access restrictions; they immediately established the importance of the project within a historical and scholarly context. Policymakers started to understand the value of these records not only as tools for research into the formation of Barbadian identity and social history but also as a way of bolstering academic, educational, and genealogical tourism, which have become a major focus of niche-tourism market development in the island. Academic and genealogical visitors tend to stay longer in the island, increasing visitor expenditure, and tend to become repeat visitors. The workshops also gave the archivists and information managers an opportunity to share the challenges they faced in a resource-scarce environment and to establish the centrality of their own roles in the development of policies and programs around the island's documentary heritage.

On the whole, all participants were able to explore the documents together and to recognize the importance of increased accessibility to these documents through digitization as a means of enhancing their visibility and creating research projects that increase public awareness vis-à-vis the national significance of archives and libraries.

An additional academic-oriented workshop featured a representative from the Digital Library of the Caribbean, one of the final repositories for the digitized {\em Mercury} files. Its goal was to introduce students, teachers/academics, and the community to other potential digital tools and thereby to enhance student learning via exposure to digital skills and media literacy.

\placefigure[here]{Workshop participants engaging with the print copies of the {\em Mercury}, 12 December 2017}{\externalfigure[issue04/levi-inness/LeviFig3.png]}


\subsubsection[title={Digitization as preservation},reference={digitization-as-preservation}]

Adverse environmental conditions and decades of improper storage prior to its arrival at the Barbados Archives in the early 1970s had left the {\em Mercury} in a state of tremendous decay, despite the current air-conditioned storage conditions. Our aim in digitizing the paper was to accomplish the critical work of storing its informational content (print) as well as to preserve a snapshot of how the {\em Mercury} looked at a particular moment in time. During digitization, the {\em Mercury} volumes were unbound to facilitate work without placing stress on the pages. The grant itself does not provide for extensive conservation work; therefore, at the end of the digitization process, pages were not rebound but were carefully wrapped by Conservation Department staff with the aim of storing them properly.

\placefigure[here]{Disbinding the {\em Mercury} in the Conservation Department, Barbados Archives}{\externalfigure[issue04/levi-inness/LeviFig4.png]}


Given the newspaper's fragile state, since making the images freely available online the Barbados Archives has restricted access to the physical copies to prevent further damage.

\subsubsection[title={Digitization as destruction and reconstruction},reference={digitization-as-destruction-and-reconstruction}]

The digitization process inadvertently involved destruction. While some volumes were in good condition, others proved particularly challenging: some pages were in pieces and were either brought together by the conservators or had to be reconstructed on the copy stand before being digitized. We had to work very carefully with the material, since some would disintegrate on touch. The Conservation Department staff demonstrated safe handling of the material to the digitization team, and every effort was made to proceed without damaging the material. At once frustrated and mesmerized by the specks of dust and paper collecting on the floor around us, we were well aware that we were the last people who would touch the physical copies of the {\em Mercury}.\footnote{Amalia Skarlatou Levi (@amaliasl), "We digitize and we destroy. We put together a puzzle that will never exist again. The pieces and the dust as embodiment of the necessity and inevitability of it all. // Some more spoils of \#digitization, volume 1812 of the \#BarbadosMercury @bl_eap," Twitter, 15 November 2018, 11:3 am, \useURL[url27][https://twitter.com/amaliasl/status/1063152517075353600]\from[url27].}

\placefigure{A puzzle of disintegrating pages.}{\externalfigure[images/levi-inness/LeviFig5.png]}
\subsubsection[title={Digitization as communication},reference={digitization-as-communication}]

Through regular postings on Twitter, we found a community of people with whom we shared surprising findings, sad stories, or the frustration of working with the fragile material. We posted, and interacted with people, about bugs that we found desiccated among the pages of the newspaper; intricate trails left by the pests who had eaten away at the pages; and unexpected findings, such as a random piece of ribbon resting over an advertisement for a shop importing ribbons, fabrics, and hats.\footnote{Read project posts on Twitter through the hashtag \#BarbadosMercury.} Such tweets resonated with scholars working with Caribbean archives and reflected their experiences handling such documents. As an example, on {\em Digital Cultural Heritage Cluster} Nathan Dize wrote about the precarity of the archival material and the poetics of the trails of burrowing pests that transform documents over time, "creating new layers of meaning and sensation."\footnote{Nathan H. Dize, "Aphids and Digital Archives: Thinking through Digital Preservation with Dionne Brand's {\em The Blue Clerk}," {\em Digital Cultural Heritage Cluster}, Vanderbilt University, 13 October 2018, \useURL[url28][https://my.vanderbilt.edu/digitalculturalheritage/2018/10/13/aphids-and-digital-archives-thinking-through-digital-preservation-with-dionne-brands-the-blue-clerk]\from[url28].} However, the tweets with the greatest impact were those we posted of runaway-slave ads, which contained stories of violence, of course, but also of courage and liberation.

\placefigure{Unexpected marginalia: a bug}{\externalfigure[images/levi-inness/LeviFig6.jpg]}
\placefigure{Trails made by hungry, wandering bookworms}{\externalfigure[images/levi-inness/LeviFig7.jpg]}
\subsection[title={Beyond Digitization},reference={beyond-digitization}]

Handling and interacting closely with the physical copies of the {\em Mercury} underlined the importance of care for collections. Digitization is useful for promoting access and limiting handling of material; however, it is in many ways secondary to conservation. Digitization cannot save what is not accounted for, what is not described properly, and what has not been documented. For archivists and librarians, the primary responsibility must be the safeguarding of our collections.

Safeguarding collections has two aspects: (a) the physical preservation of these documents, which involves establishing correct environmental conditions, housing material in proper conditions, protecting against fire, safeguarding against natural disasters, and physically controlling items in the collection; and (b) the process of inventorying, which entails providing detailed and accurate descriptions that facilitate access to and retrieval of collections.

Having good descriptions in place allows us to have intellectual control over our collections, promote access to our material, and increase contextualization of the records by providing historical insight about their value and links to other items in our institution or other institutions. More importantly, in the case of disasters, it helps us have good documentation and representation of what was there and offers the possibility to reconstruct it, even if partially.\footnote{See, for example, {\em Beyond 2022: Ireland's Virtual Records Treasury}, which aims to recreate the Public Record Office of Ireland that was destroyed by fire in 1922, \useURL[url29][https://beyond2022.ie]\from[url29].}

All digitization is necessarily selective, and some items are prioritized over others, based on funding or institutional goals. Digitization is usually seen by funding agencies as the "sexy" aspect of archival projects and is much easier to fund, while other vital aspects of archival work---like creating descriptions and inventories, providing proper storage, and doing the labor-intensive and expensive work of conservation---are neglected.

Digitization inevitably results in loss of information. The process of scanning materials produces digital images that obscure the materiality of the items and risks conflating the multitude of voices within them. Metadata created during digitization usually favor the "creator" of a collection, and with limited to no possibility to search the images, marginalized populations remain obscured and difficult to find.

Considering that all digitization involves selection, collections that are "most used" or belong to "important people" may end up prioritized for digitization. Such collections usually represent dominant voices---in the case of colonial societies, the voices of the white class of planters, merchants, and colonial authorities. Digitization that does not take this into consideration risks elevating and promoting access to such voices and thus contributing to the reproduction and amplification of the very colonial logics and worldviews we mean to avoid.

\subsection[title={Decolonizing the Record: The Case Study of the {\em Barbados Mercury Gazette}},reference={decolonizing-the-record-the-case-study-of-the-barbados-mercury-gazette}]

To mitigate the challenges of digitization, we aim to contribute to decolonizing the archival record about the enslaved in Barbados, using the {\em Mercury} as a starting point for several initiatives. We are doing so in a purposeful, conscious way that strives to overcome the gaps, silences, and omissions either inherent in the colonial epistemologies that the {\em Mercury} represents or created by the limitations of digitization technology. Recent scholarship about the enslaved has shown that it is possible to challenge the erasure of colonial archival sources and seek out interdisciplinary methodologies to help tease out information about the enslaved that might seem absent.\footnote{See, for example, Marisa J. Fuentes, {\em Dispossessed Lives: Enslaved Women, Violence, and the Archive} (Philadelphia: University of Pennsylvania Press, 2018).}

\subsubsection[title={Extracting information},reference={extracting-information}]

Despite the proliferation of digitization efforts, "cultural heritage institutions have rarely built digital collections or designed access with the aim to support computational use."\footnote{Always Already Computational: Collections as Data project team, "The Santa Barbara Statement on Collections as Data," version 2, \useURL[url30][https://collectionsasdata.github.io/statement/]\from[url30] (accessed 6 July 2019).} Developing and providing access to the digitized {\em Mercury} images as data is crucial for retrospectively creating a body of information about the enslaved that can potentially be computationally malleable.

\subsubsubsection[title={Optical character recognition (OCR)},reference={optical-character-recognition-ocr}]

One way to extract information from digitized images is through OCR, as recent work with digitized newspapers has shown.\footnote{One such collection provided as data is the {\em Gaceta de la Habana} at the University of Miami Libraries; see \useURL[url31][https://github.com/UMiamiLibraries/collections-as-data/tree/master/LaGaceta]\from[url31].} However, we had bad results when we OCR'ed sample pages from the {\em Mercury} because of the acidity of the paper and the unevenness of the color.\footnote{OCR was conducted with Prime Recognition.} We plan to try handwritten recognition engines (e.g., Transkribus), which have worked well in a similar case, in the hope that this might produce better results.\footnote{See Dario Kampkaspar, "Digital Edition of the \quote{Wienerisches Diarium,}" November 2017, \useURL[url32][https://read.transkribus.eu/wp-content/uploads/2017/07/Kampkaspar_Wienerisches_Diarium.pdf]\from[url32].}

\subsubsubsection[title={The Barbados Runaway Slaves Digital Collection project},reference={the-barbados-runaway-slaves-digital-collection-project}]

In March 2019, a new collaboration between the Barbados Archives and the \useURL[url33][file:///C:\Users\willo\Downloads\ecda.northeastern.edu][][Early Caribbean Digital Archive]\from[url33] resulted in initiating the Barbados Runaway Slaves Digital Collection project.\footnote{The {\em Early Caribbean Digital Archive} is an open-access digital archive, an online repository of documents, travel narratives, natural histories, novels, poetry, and maps from the sixteenth- to nineteenth-century Caribbean, which offers dynamic interactions with early Caribbean texts, a robust research community, and pedagogical and scholarly resources; see \useURL[url34][https://ecda.northeastern.edu]\from[url34]. At the time of the writing of this article, the website of the Barbados Runaway Slaves Digital Collection project has not yet been finalized. Furthermore, the title of the digital collection is still tentative.} The collaboration was officially launched through a series of workshops and events in May 2019 that served to introduce the project and engage the public.\footnote{Paula Harper-Grant, "Barbados Has Digital Runaway Slaves Collection," {\em GIS} (blog), 13 March 2019, \useURL[url35][https://gisbarbados.gov.bb/blog/barbados-has-digital-runaway-slaves-collection/]\from[url35].} These initial workshops were followed by monthly workshops during fall and winter 2019.\footnote{To learn more about the workshops conducted during fall 2019, please see Amalia S. Levi, "Beyond Digitisation: Engaging the Community around the {\em Barbados Mercury}," {\em Endangered Archives Blog}, 2 January 2020, \useURL[url36][https://blogs.bl.uk/endangeredarchives/2020/01/beyond-digitisation-the-barbados-mercury.html]\from[url36].}

The aim of this initiative is to extract runaway-slave ads, transcribe the text, and enrich the human stories in each ad with additional contextual information. At a later stage, we hope to be able to input the information from the ads into a database.

At the outset, we were committed to engaging the public in this work through regular monthly meetings. We felt that if we did not do so, Barbadians would not be aware of this resource and the digitization of the {\em Mercury} would end up being an extractive project. As was made clear during the launch events, involving the public creates community and incentivizes people to work together to research their history. This is especially important for this project, given that many Barbadians are not conscious of their enslaved past or of the ways their ancestors contested and resisted slavery, as the very fact of these runaway-slave ads affirms. Currently, there is very little formal or informal discussion about the role of slavery in shaping the country's development. History is not a mandatory subject through primary and secondary school, and slave studies only comprise a relatively small part of the overall curriculum.

It was, then, especially moving to see people interact with the archival material during the launch events and with the digitized copies during the subsequent workshops, and to witness their realization that these materials could be a source of rich information for understanding the Barbadian past. In a sense, we cannot understand the ads independently from the people because in the postcolony, the community {\em is} the archives.

\placefigure{Runaway-slave ad, 3 January 1789}{\externalfigure[images/levi-inness/LeviFig8.png]}
\placefigure{Runaway-slave ad, 27 January 1807.}{\externalfigure[images/levi-inness/LeviFig9.png]}
\placefigure{Runaway-slave ad, 14 April 1807.}{\externalfigure[images/levi-inness/LeviFig10.png]}
\placefigure{Runaway-slave ad, 14 April 1807.}{\externalfigure[images/levi-inness/LeviFig11.png]}
\placefigure{Runaway-slave ad, 30 August 1783.}{\externalfigure[images/levi-inness/LeviFig12.png]}
\placefigure{Runaway-slave ad, 14 April 1807.}{\externalfigure[images/levi-inness/LeviFig13.png]}
Participants in the workshops saw vivid, human stories that provided descriptions of and gave a voice to voiceless ancestors. By engaging with these ads, participants could discuss past traumas and even acknowledge the triumphs of enslaved runaways in their quest to challenge the authority of enslavers. This kind of resistance is not fully appreciated by younger Barbadians who have often been told by older generations, that is, by Barbadians who did not have the benefit of a postcolonial education, that their enslaved ancestors were less rebellious than in other slave societies.\footnote{For resistance strategies used among the enslaved in Barbados, the following sources provide more information: Hilary Beckles, "An Economic Life of Their Own: Slaves as Commodity Producers and Distributors in Barbados," {\em Slavery and Abolition} 12, no. 1 (1991): 31--47, \useURL[url37][https://doi.org/10.1080/01440399108575021][][doi:10.1080/01440399108575021]\from[url37]; Hilary Beckles, "Caribbean Anti-slavery: The Self-Liberation Ethos of Enslaved Blacks," in {\em Caribbean Slave Society and Economy}, ed.~Hilary Beckles and Verene Shepherd (Kingston: Ian Randle, 1991); and Hilary Beckles, {\em Natural Rebels: A Social History of Enslaved Women in Barbados} (New Brunswick, NJ: Rutgers University Press, 1989).} Sharing these stories creates dialogue and allows people to speak about a past that is often shunned or avoided in the present.

An important part of this work will be to fill the gaps in the records. There are aspects of slave society that the ads reveal to us, and there are other aspects that they imply or that we can infer from the context. But much remains incomplete or unknown. We may, however, be able to speculate and creatively try to understand what is not there. This is critical during the process of extracting data from the information in the {\em Mercury}. Lest we think that data are an objective, accurate representation of reality, we must remember, as Jessica Marie Johnson forcefully contends,

\startblockquote
There is no bloodless data in slavery's archive. Data is the evidence of terror, and the idea of data as fundamental & objective information\ldots{}obscures rather than reveals the scene of the crime.\footnote{Jessica Marie Johnson, "Markup Bodies: Black Life Studies and Slavery Death Studies at the Digital Crossroads," {\em Social Text}, no. 137 (December 2018): 57--79, \useURL[url38][http://doi.org/10.1215/01642472-7145658][][doi:10.1215/01642472-7145658]\from[url38].}
\stopblockquote

The Barbados Runaway Slaves Digital Collection remains a relatively new venture, and there is still a great deal of work to be done, particularly because the Barbados Archives holds a wealth of information (plantation records, wills, slave registers, etc.) that can be used to contextualize the ad information and disambiguate individuals. The collection can also be usefully cross-referenced with other digitized archival sources globally, such as the \useURL[url39][http://www.slavevoyages.org][][{\em Slave Voyages}]\from[url39] database or the University College of London's \useURL[url40][http://www.ucl.ac.uk/lbs][][{\em Legacies of British Slave Ownership}]\from[url40] database. Eventually it might be possible to become part of larger initiatives like the \useURL[url41][http://enslaved.org/][][{\em Enslaved: The People of the Historic Slave Trade}]\from[url41] project, which aims to bring together data about the people who took part in the slave trade across disparate projects and institutions, and to allow people to search across multiple databases.\footnote{See {\em Slave Voyages}, \useURL[url42][https://www.slavevoyages.org]\from[url42]; Legacies of British Slave-Ownership, \useURL[url43][http://www.ucl.ac.uk/lbs]\from[url43]; and {\em Enslaved: The People of the Historic Slave Trade}, \useURL[url44][https://www.enslaved.org]\from[url44].}

\subsubsection[title={Promote use in digital humanities projects},reference={promote-use-in-digital-humanities-projects}]

We hope that extracting information and data from the {\em Mercury} will promote its use for digital humanities projects. The information contained in the {\em Mercury} can be of use to many different disciplines. It can also form the basis for student projects. Below are some ways scholars and researchers might interact with the {\em Mercury} using digital tools:

As the \useURL[url45][https://jesuitplantationproject.org/s/jpp/page/welcome][][{\em Jesuit Plantation Project}]\from[url45] shows, it is possible to infer relationships and extract networks of people even when explicit information is lacking, or when the individuals do not have last names, particularly when enriching this information with geographical linked data.\footnote{See {\em Life and Labor under Slavery: The Jesuit Plantation Project}, \useURL[url46][https://jesuitplantationproject.org/s/jpp/page/welcome]\from[url46].} The {\em Mercury} offers a veritable treasure trove of human characters and stories, particularly through the runaway-slave ads, and we hope that scholars will be able to work to bring these lives to light, particularly when other available archival material might be scattered or inaccessible, as is the case with colonial records.

The {\em Mercury} offers great opportunities for creating spatial humanities projects. It is full of movement---of people, goods, ideas, and transportation methods. Considering that the newspaper was a knowledge infrastructure and informational tool for the planter and merchant class, it contains a wealth of information about ships arriving or departing; sales of goods on the island; and news from Europe or other colonies (in the Caribbean or North America). The {\em Mercury} was also an implacable tool for the relentless pursue of lost property, that is, the enslaved human beings who managed to escape from bondage. These ads offer incredibly rich geographical information, since their aim was to find and apprehend these individuals and to point to all possible locations where they might be in hiding. Projects like \useURL[url47][http://revolt.axismaps.com/][][{\em Slave Revolt in Jamaica}]\from[url47] or \useURL[url48][http://mapping-marronage.rll.lsa.umich.edu/welcome][][{\em Mapping Marronage}]\from[url48] prove that seeing geographical information in a holistic way can reveal patterns of movement, escape, resistance, and organization.\footnote{See {\em Slave Revolt in Jamaica, 1760--1761}, \useURL[url49][https://revolt.axismaps.com]\from[url49]; and {\em Mapping Marronage}, \useURL[url50][https://mapping-marronage.rll.lsa.umich.edu/welcome]\from[url50].}

The {\em Mercury} also provides a vivid picture of Bridgetown; its shops, workplaces, houses, and port; and the people inhabiting and working in these places. As an example, ads about shops and the goods they imported and sold can form the basis of an augmented reality application that would overlay present-day locations with their historical antecedents. Geolocated ads can also help tell stories using layers of the island's past, as with the example of \useURL[url51][http://www.layersoflondon.org][][{\em Layers of London}]\from[url51].\footnote{See {\em Layers of London: Recording the Layers of London's Rich Heritage}, \useURL[url52][http://www.layersoflondon.org]\from[url52].}

Considering both changes in the island's urban landscape after the nineteenth century, and especially after Barbados gained its independence in 1966, and the continuous use of old pathways and locations, the {\em Mercury} can be a great source of geographical information. Such information pertains to locations on the island and to "waterways" that underpinned transportation in the Atlantic world. Not only does the gazette contain location information about shops or houses, it also faithfully notes the goings and comings of all ships departing from or arriving to the island. Working with and enriching such geographical information with Linked Open Data can form the basis for a Barbados-specific gazetteer that in turn can become a pilot project for a gazetteer for the English-speaking Caribbean. This gazetteer could complement the {\em LatAm Gazetteer of Colonial Latin America} being developed under the aegis of the Pelagios project.\footnote{Pelagios, "Final Report on LatAm: A Historical Gazetteer of Colonial Latin America and the Caribbean," {\em Medium} (blog), 14 June 2019, \useURL[url53][https://medium.com/pelagios/final-report-on-latam-a-historical-gazetteer-of-colonial-latin-america-and-the-caribbean-4772c7eae9e2]\from[url53].}

\subsubsection[title={The {\em Mercury} as a teaching tool},reference={the-mercury-as-a-teaching-tool}]

The Department of History and Philosophy at the University of the West Indies (UWI), Cave Hill Campus has increasingly been looking at digital humanities scholarship and methodologies as a way of engaging students in primary source research, while also improving their digital skills and literacy.

A recent Caribbean Examinations Council Task Force on History report examined the decline in Caribbean Secondary Examination Certificate--level and Caribbean Advanced Examination Certificate--level history registrants over the last decade or more.\footnote{Caribbean Examinations Council, {\em Caribbean Examinations Council (CXC) Task Force on History Report} (Bridgetown, Barbados: Caribbean Examinations Council, 2018).} The task force investigated a number of factors for the decline and made recommendations, but the factors largely revolved around competition with other new, seemingly career-ready subjects (e.g., business studies, information technology, animation, and gaming and tourism) and a lack of knowledge by students and parents concerning the transferable and employable skills gained in a subject like history.

Some of the recommendations made included using history to teach other subjects, especially those requiring digital literacy skills (including the perceived career-ready subjects mentioned above), as well as to demonstrate how history could be represented using technology to make it more relevant and accessible to students across the region. University lecturers have been cognizant of the decline for a number of years, so opportunities to get involved in digital history projects, including the {\em Mercury} newspaper and associated projects through the Digital Library of the Caribbean and Northeastern University's {\em Early Caribbean Digital Archive} through the Slave Runaway Project, have allowed students and graduates of the UWI History Department to be a part of the digital application of their historical skills. Students are not only using the repositories in their courses but also helping to build associated databases as projects for their assessments. There are also opportunities for students at both the undergraduate and postgraduate levels to use digital resources as part of their thesis projects.

Some of the student participants involved in the most recent workshops immediately recognized how their historical skills can be applied to "reading between the lines" of the runaway ads that were transcribed in the workshop. The sessions help to reinforce students' confidence in the historical skills they are acquiring, while also making these skills relevant to practical project work.\footnote{One undergraduate student reflected, "You can see how this session promotes the idea of critical analysis and would engage and promote the desire to study history from the secondary school level. I believe that some effort should be made to promote digital humanities beyond the university"; Tevin Maynard, e-mail to authors, 20 May 2019. Another student mentioned how the exercise helped her to understand more about the lived experience of enslaved people in Barbados and the risks they took to challenge the system of slavery by running away and absenting themselves from slaveholders.} Moreover, recent graduates of the undergraduate and postgraduate degree programs in history and heritage studies are being employed to assist archivists with the completion of digitization projects, thereby demonstrating (to themselves and to their parents) the importance of digital history skills in the workplace.

\subsection[title={Challenging Narratives, Reclaiming Ancestral Voices},reference={challenging-narratives-reclaiming-ancestral-voices}]

The digitization of the {\em Barbados Mercury Gazette} has great potential for research projects that can aid students and other researchers to better understand their history. The examples that stand out are the numerous runaway ads in almost every edition of the paper until the end of slavery. These ads in particular help to contest the popular narrative of Barbadian acquiescence to slavery; they help to reconstruct the family and community networks that were important to enslaved Barbadians; and they help to give enslaved and free ancestors a voice, whereas the intent of these records was to leave them voiceless. That said, the potential inherent in a now readily accessible and (hopefully) free archive will depend on public and community engagement and on the creation of participatory projects among students and educators, particularly at the secondary and tertiary levels in the region. The more sources that become available, the more promise there is to move this information and its analysis out of the realm of academic scholarship and into the classroom for students and communities. This kind of development would allow archives to break free of the stratified colonial silences and privilege-based voids in access. It would allow archives to become spaces for community development, capacity building, social consciousness, and social justice---but only if we invest in their preservation in the first place. If archival documents are merely digitized and uploaded online with no commitment to conservation, programming, and engagement, then the colonized legacy of archives remains locked in the patterns and hierarchies of the past---invisible in HTML (hypertext markup language) code on websites unexplored by the people who need them the most.

\thinrule

\page
\subsection{Amalia S. Levi}

Amalia S. Levi is an archivist and cultural heritage professional currently residing in Barbados. She is the founding director of the HeritEdge Connection, a nonprofit organization dedicated to forging collaborations and connecting people, resources, and institutions through cultural heritage projects. She has conducted various documentation and digitization projects funded through grants by the Endangered Archives Programme and the Modern Endangered Archives Program. Amalia holds an MLS, with a concentration in archives, and an MA in history, with a concentration in Jewish studies, both from the University of Maryland; an MA in museum studies from Yildiz University in Istanbul, Turkey; and a BA in archaeology and history of art from the University of Athens, Greece. Amalia is interested in augmenting and facilitating historical scholarship on diasporas and marginalized groups by linking and enriching dispersed collections with digital tools.

\subsection{Tara A. Inniss}

Tara A. Inniss is a lecturer in the Department of History and Philosophy at, the University of the West Indies (UWI), Cave Hill Campus. The areas of focus for her teaching and research include history of medicine, history of social policy, and heritage and social development. In 2002--3, she received a Split-Site Commonwealth PhD Scholarship to study at the UWI/University of Manchester. Her thesis was on children's health during slavery and the apprenticeship periods. In 2007, she completed a master's in international social development at the University of New South Wales in Sydney, Australia. She received her PhD in history from the University of the West Indies, Cave Hill Campus in 2008. Tara currently sits on several committees for the Barbados World Heritage Committee, the Barbados Museum and Historical Society, and the Association of Caribbean Historians. She is the director of the UWI/OAS Caribbean Heritage Network and the coordinator of Heritage studies.

\stopchapter
\stoptext