\setvariables[article][shortauthor={Jégousso, O'Dell}, date={January, 2020}, issue={4}, DOI={https://doi.org/10.7916/archipelagos-6fz4-kb03}]

\setupinteraction[title={Thinking Digital: Archives and Glissant Studies in the Twenty-First Century},author={Jeanne Jégousso, Emily O'Dell}, date={January, 2020}, subtitle={Thinking Digital}]
\environment env_journal


\starttext


\startchapter[title={Thinking Digital: Archives and Glissant Studies in the Twenty-First Century}
, marking={Thinking Digital}
, bookmark={Thinking Digital: Archives and Glissant Studies in the Twenty-First Century}]


\startlines
{\bf
Jeanne Jégousso
Emily O'Dell
}
\stoplines


{\startnarrower\it The {\em Library of Glissant Studies}, an open-access website, centralizes information by and on the work of Caribbean writer Édouard Glissant (Martinique, 1928--France, 2011). Endorsed by literary executors Sylvie and Mathieu Glissant, this multilingual, multimedia website provides bibliographic information, reproduces rare sources, connects numerous institutions, and includes the works of both renowned and emerging scholars in order to stimulate further research. The library chronicles all forms of academic work devoted to Glissant's writings, such as doctoral dissertations, MA theses, articles, books, and conference presentations. In this article, Jeanne Jégousso and Emily O'Dell address the origin, the evolution, and the challenges of the digital platform that was launched in March 2018. \stopnarrower}

\blank[2*line]
\blackrule[width=\textwidth,height=.01pt]
\blank[2*line]

Martinican author Édouard Glissant (1928--2011) revolutionized contemporary literary thought while providing new ways to theorize and understand transnational exchanges thanks to his key notions of Relation, developed in his 1990 {\em Poétique de la Relation} ({\em Poetics of Relation}), and the {\em Tout-monde} (Whole-World), which was also the title of his 1993 novel. Drawing from the diverse philosophical, poetic, and oral traditions of Europe, Africa, and the Americas, Glissant questioned and reconsidered the very notion of national boundaries and cultural legacies in order to create a dialogue among civilizations across time and geographic borders.

Toward the end of his life, Glissant became particularly preoccupied with the continuation of the work that the Canadian scholar Alain Baudot had completed in his {\em Bibliographie annotée d'Édouard Glissant} ({\em Annotated Bibliography of Édouard Glissant}), which contains more than thirteen hundred references and sixty illustrations pertaining to Glissant's work and its criticism. However, because Baudot's bibliography had been out of print since 1993, Glissant expressed the need for a new project that would \quotation{poursuivre le travail d'Alain Baudot} (\quotation{continue Alain Baudot's work}) and include the books, articles, conference talks, interviews, newspaper clippings, and poems that Glissant had written in the 1990s and 2000s.\footnote{Édouard Glissant, discussion with Raphaël Lauro, December 2010, Paris. Unless otherwise noted, all translations are ours.} The resulting project is the {\em Library of Glissant Studies}, a digital platform created by Raphaël Lauro and Jeanne Jégousso dedicated to inspiring new research and facilitating collaboration between scholars. In this article, we address the inspiration behind and purpose of the {\em Library of Glissant Studies}, as well as the hermeneutic questions that arose during the first stages of this project and the research philosophy that led to the design and functionality of the online platform.

The \useURL[url1][https://www.glissantstudies.com][][{\em Library of Glissant Studies}]\from[url1] ({\em LoGS}) was designed as a digital bibliography dedicated to making textual references by and about Édouard Glissant accessible and widely available to readers, students, and scholars on every continent.\footnote{See the {\em Library of Glissant Studies}, \useURL[url2][https://www.glissantstudies.com]\from[url2].} The objective of the project is to preserve and share information on Glissant's work and thought in an open-access website and also to facilitate scholarly collaboration by, on the one hand, including materials in numerous languages from geographically disparate areas and by, on the other, creating research groups in several institutions. It is important to emphasize that although the majority of Glissant's work was produced in French, the {\em Library of Glissant Studies} was not designed or developed exclusively for French speakers and French materials.

Instead, the team pays special attention to including all texts pertaining to Glissant or his work in any language, which demonstrates Glissant's theory formulated in his fifth essay of poetics, {\em La Cohée du Lamentin}, that there is a \quotation{multirelation où toutes les langues du monde . . . trament ensemble des chemins qui sont autant d'échos} (\quotation{a multirelation where all the languages of the world . . . hatch together paths that are as many echoes}).\footnote{Édouard Glissant, {\em La Cohée du Lamentin} (Paris: Gallimard, 2005), 137.} In other words, the implementation of this philosophy in the construction of the digital bibliography supposes that the aggregation of languages will facilitate the emergence of new research trends that will eventually resonate and influence one another. In addition, Glissant writes in his 1981 essay {\em Le discours antillais} that plurilingualism is one of the paths to Relation, which consists of ongoing exchanges between various elements as well as cultural {\em metissage}, and that \quotation{multilinguisme \quote{disperse} le texte écrit dans une diversité concrète dont il faut dès maintenant explorer les accès inconnus} (\quotation{plurilingualism \quote{diffuses} the written text in a concrete diversity of which we must immediately explore the unknown paths}).\footnote{Édouard Glissant, {\em Le discours antillais} (Paris: Gallimard, 1997), 616.}

Therefore, in keeping with Glissant's conception of plurilingualism, the {\em Library of Glissant Studies} strives to facilitate exchanges by including multiple languages from numerous countries and to encourage students, scholars, and readers to explore \quotation{unknown paths} in Glissant's literary production while developing an aesthetic and structure reflecting his notions of Relation and \quotation{archipelic-thinking.} In the 2005 essay {\em La Cohée du Lamentin}, Glissant distinguished the \quotation{pensée continentale, qui dévoile en diasporas les splendeurs absolues de l'Un{[}, et la{]} pensée archipélique, où se concentre l'infinie variation de la Diversité} (\quotation{continental-thinking, which reveals in diasporas the absolute splendors of the One{[}, and the{]} archipelic-thinking, where the infinite variation of Diversity is concentrated}).\footnote{Glissant, {\em La Cohée du Lamentin}, 231.} By placing these two types of thinking in opposition, Glissant favors the diverse multiplicity (archipelic-thinking) over the repetition of the same (continental-thinking). By bringing together languages, materials, and scholars from various disciplines and geographical areas, we designed our project to be faithful to this emphasis on plurality and diversity.

Although Glissant was influenced by his Caribbean heritage, his thought and writings are not anchored to one particular place but rather illustrate a worldly and intercontinental way of thinking. He left his native island of Martinique at the age of eighteen to study in Paris, he lived in Louisiana and in New York City, and he traveled extensively to several islands in the Caribbean and all over the world. In fact, Glissant repeatedly rejects the notion of {\em identité-racine} (rooted-identity) to favor an {\em identité-rhizome} (rhizomatic-identity), which embraces a vision of the world that focuses on diversity and plurality instead of unicity and purity. Thus the {\em Library of Glissant Studies} was designed with an esthetics of the Whole-World in mind, and chronology seemed to be a more essential element to emphasize than geography because it gives a clearer picture of the evolution of Glissant's key notions and of their reception at particular moments in time. Therefore, the project will soon include an evolutive map that showcases the dispersion of the author's ideas throughout time by showing dots on a map that correspond to the number of works by and about the author created in a particular year.

Glissant's role in redefining Caribbean studies and Atlantic studies and their roles as models for global studies is central to many recent studies, such as, for example, Kristin Van Haesendonck and Theo D'Haen's 2014 edited volume {\em Caribbeing: Comparing Caribbean Literatures and Cultures} and John E. Drabinski and Marisa Parham's 2015 volume of essays {\em Theorizing Glissant: Sites and Citations}. Therefore, this project also responds to the growing interest surrounding Glissant's literary production. For instance, according to the French database {\em Fichier central des thèses}, the number of completed doctoral theses in France mentioning Glissant's theories and works has increased from only 11 in 2009 to 25 in 2017. Similarly, ProQuest.com lists 681 theses devoted to Glissant produced in North America from 2008 to 2017. It is clear that Glissant's texts and notions are continuously and increasingly being studied and debated. Another indicator of the growing interest in Glissant studies is that these works are no longer relegated to French and francophone literary studies and are now the subject of study in other academic fields and disciplines (e.g., comparative literature, postcolonial theory, cultural criticism, linguistics, anthropology, philosophy, history, etc.), as demonstrated by Van Haesendonck and D'Haen's and Drabinski and Parham's volumes mentioned above.\footnote{Kristin Van Haesendonck and Theo D'Haen, {\em Caribbeing: Comparing Caribbean Literatures and Cultures} (Amsterdam: Rodopi, 2014); and John E. Drabinski and Marisa Parham, {\em Theorizing Glissant: Sites and Citations} (London: Rowman and Littlefield, 2015). Other examples include Celia Britton, {\em Édouard Glissant and Postcolonial Theory: Strategies of Language and Resistance} (Charlottesville: University Press of Virginia, 1999); Christina Kullberg, \quotation{Crossroads Poetics: Glissant and Ethnography,} {\em Callaloo} 36, no. 4 (2013): 968--82; Alexandre Leupin, {\em Édouard Glissant, philosophe: Héraclite et Hegel dans le Tout-monde} (Paris: Hermann, 2016); and Michael Wiedorn, {\em Think like an Archipelago: Paradox in the Work of Édouard Glissant} (New York: State University of New York Press, 2018).}

As perspectives on Glissant's work evolve, it remains difficult to access and stay current with the numerous publications, events, and documents created around the globe. This is also true of Glissant's original works, which are inaccessible to the majority of researchers because many of his writings have never been republished since their initial appearance in currently unavailable literary journals, newspapers, and catalogues of art exhibits. It was essential to gather these materials to make them available to research and to further our understanding of Glissant's work. {\em LoGS} now offers unique documents, such as poems published in the early 1950s, rare interviews published in various newspapers, and exclusive pictures and manuscripts of the author sent in by various contributors. Thanks to this collection, readers and scholars are able to have a better understanding of the movement and evolution of Glissant writing process. For the project's directors, the accumulation of documents creates an {\em archéologie relationnelle} (relational archaeology) of Glissant's writing.\footnote{This notion and its consequences were presented in a symposium at the University of Cambridge. See Raphaël Lauro, \quotation{Édouard Glissant, archiviste de lui-même: L'exemple du {\em Discours antillais}} (paper presented at Édouard Glissant: Le cri et la parole, Cambridge, June 2019).}

In fact, it allows the user to compare various texts from the same time period, which reveals the ways Glissant developed one theme throughout various genres and from different perspectives. For example, during the 1970s Glissant focused his literary production on the creation of a written language resulting from an assemblage of both French and Creole. This theory appears in 1975 in three forms: a conference presentation at the University of Milwaukee--Madison in April (later modified and included in Glissant's 1981 essay {\em The Caribbean Discourse}), a radio interview on 10 July, and a novel titled {\em Malemort}. Before the {\em Library of Glissant Studies}, these documents were scattered across different physical and digital archives. In addition, some of them, such as Glissant's presentation, were entirely unknown to the general public and a majority of scholars. This illustrates how our digital project very concretely lends itself to the creation of a relational archaeology, which places texts from different genres and time periods in relation. As a result, the 1975 conference presentation can be used to clarify sections of the {\em Caribbean Discourse} published in 1981, therefore opening new textual interpretations and permitting a better understanding of Glissant's texts and evolutions of thought.

In addition, the digital archives allow us to perceive a \quotation{technique de l'entassement, de la reprise, de la répétition} (\quotation{technique of stacking, reiteration, repetition}),\footnote{Ibid.} which results in the formulation of new research questions, including, Why did the author stop writing poetry for nineteen years? How did a particular notion evolve between the first draft of an article or conference talk and the final publication? It is a significant advancement in the field of Glissant studies to be able to \quotation{see} the {\em ressassement} at work by placing literary journals from the 1950s and essays from the late 1960s in conversation. {\em Ressassement}, which means repetition with minor changes and is often symbolized by a spiral, plays an integral role in Glissant's writing as he explained to Alexandre Leupin in {\em Les entretiens de Baton Rouge}: \quotation{La répétition et le ressassement m'aident ainsi à fouiller} (\quotation{Repetition and {\em ressassement} help me to search}).\footnote{Alexandre Leupin, {\em Les entretiens de Baton Rouge} (Paris: Gallimard, 2008), 58.} For instance, a poem included in Glissant's 1969 essay {\em Poetic Intention} was actually published nine years previously in {\em La Voix des Poètes}, a literary journal directed by Simone Chevalier.\footnote{Édouard Glissant, {\em L'intention poétique} (Paris: Gallimard, 1997), 26.} Therefore, a digital bibliography allows us to create a complete overview of Glissant's work, to formulate new questions, and to examine new dimensions of the author's thought.

Another objective of the project was to solve the problems created when scholars and students in the United States are unfamiliar with or do not have access to the developments in Glissant studies in Japan and when scholars in France are not aware of the works being produced in Canada, and so on. Although platforms such as JSTOR, WorldCat, Project MUSE, Érudit, and EBSCOHost make it possible to access some of these critical and literary productions, they are incomplete and often limited to particular languages or geographical regions. They also do not offer the interactive features necessary to allow Glissant scholars to collaborate with one another. By gathering all of Glissant's criticism in one place, it is possible to see which issues have been discussed at length, such as the notion of Relation, creolization, and {\em antillanité}, and which ones require further study. For example, the journal {\em Acoma} created by Glissant and divided into five issues has been analyzed only in pieces in two scholarly articles.\footnote{See {\em Acoma: Revue de littérature, de sciences humaines et politiques trimestrielle, 1--5, 1971--1973} (Perpignan: Presses Universitaires de Perpignan, 2005).} In keeping the bibliography current, we hope to see new topics of research emerge in the future and to encourage Glissantian criticism not to repeat itself by keeping people apprised of what types of work are being produced, and to keep the field of Glissant studies lively and innovative.

From a practical standpoint, the {\em Library of Glissant Studies} consists of an interface organized into two primary bibliographic categories: texts and works written by Édouard Glissant and texts and works written about Édouard Glissant. The items in these sections are organized in chronological order to emphasize how particular essays and novels were shaped by previous publications in literary journals and prefaces and also to easily compare the author's work, its reception, and its criticisms. In addition, each user will be able to search by language, year, type of document, or keyword, and each of these subcategories includes the language and country of publication. In the near future, when looking for a specific topic, researchers will be able to discover additional materials through network visualization. Under each of the categories, bibliographical references will be indicated using the eighth edition of the {\em Modern Language Association Handbook} and accompanied by a PDF document or digital scan of materials in the public domain or by a link to the text in question if it is still under copyright (via Cairn, JSTOR, or other reference sites). The website also has a section in which users can submit documents to be considered for the project's platform. All the submitted documents are reviewed by our executive board to ensure the accuracy and authenticity of the information before posting them on our website.

One of the most common challenges faced in academic endeavors is finding monetary support, and, because of a lack of funding, {\em LoGS} relies on a free web platform and the voluntary work of dozens of dedicated individuals. During the first year of the project, Raphaël Lauro, who had archived Glissant's manuscripts for the National Library of France, and Jeanne Jégousso, who recovered and archived documentation for the Center for French and Francophone Studies at Louisiana State University, worked together with the support of Édouard Glissant's widow, Sylvie Glissant, to archive the author's lesser-known works. During this period, Lauro and Jégousso indexed most of the scholarly work produced in France and in the United States, designed and built the website, and contacted established researchers and graduate students to create a network of Glissant scholars. This interdisciplinary team of students, scholars, and independent researchers is responsible for fostering greater accessibility to past and present work and to creating a discussion with a wide and diverse audience.

Today, {\em LoGS} is structured around seven research and editorial groups, divided into geographical areas and led by one or several scholars. Each group is in charge of relaying the new publications, events, archives, and other type of materials to the project's e-mail address (glissantstudies{[}at{]}gmail.com). The information is then processed and added to the website. Each team leader has involved his or her institution to support the project and to contribute to the efforts of securing grants. The teams are located in Japan (Takayuki Nakamura, Waseda University), France (Sylvie Glissant and Loïc Céry, Institut du Tout-monde), Québec (Raphaël Lauro, Université de Montréal), the United States (Jeanne Jégousso, Hollins University, and Charly Verstraet, University of Albama at Birmingham), Martinique (Axel Arthéron and Dominique Aurélia, Université des Antilles), Cuba (Camila Valdés, Casa de las Americas), and Italy (Elena Pessini, University of Parma, and Guiseppe Sofo, Università Ca' Foscari Venezia).

In addition to the editorial board, the {\em Library of Glissant Studies} relies on the involvement of an advisory board and a team of contributors. The advisory board consists of renowned scholars in the field of Glissant studies, and their endorsement is a testament to the rigor of the project. The contributors are students, scholars, and critical thinkers who have sent at least five new references to the project. Their efforts are recognized by adding their picture and biography to our website. This is to say that the {\em Library of Glissant Studies} is first and foremost a collective project, where people share their references, meet one another during colloquia focusing on Glissant's work, and engage with each other on our social media platforms. Over the past few months, our team has provided documentation and support to curators for exhibits in Paris and Miami, created awareness of upcoming publications and translations (e.g., an upcoming Japanese version of the {\em Caribbean Discourse}, the first Spanish translation of Glissant's essay {\em The Philosophy of Relation}, and an updated English translation of Glissant's poems), and provided pedagogical support and materials to help professors find the ideal documents for their students.

During this process, we had to think about how to build a digital bibliography and how to examine our digital practices in order to create and design a unique tool that is faithful not only to Glissant's work and philosophy but also to our wish to further research and collaboration while being as inclusive as possible. The {\em Library of Glissant Studies} was created to facilitate \quotation{la consultation des sources sans véritablement se prononcer, en apparence, sur les méthodes d'analyse qui doivent être employees} (\quotation{the consultation of sources without determining the methods of analysis that must be used}), to quote Pierre Mounier in his 2018 essay {\em Les humanités numériques: Une histoire critique}.\footnote{Pierre Mounier, {\em Les humanités numériques: Une histoire critique} (Paris: Éditions de la Maison des sciences de l'homme, 2018), 48.} Following this principle, the {\em Library of Glissant Studies} was designed as a collection without any interpretation to guide or influence the user, thereby leaving the interpretation and analysis of the documents and archives to the users.

Concerning the project's sustainability, in the future we plan to consolidate our digital platform by gaining more team members and universities and by adding a digital initiatives and metadata librarian to our team. This will ensure that the project is built to last, thus preserving all the materials and exclusive documents we are currently providing to the public. We have also recently undergone the process of copyrighting exclusive documents (inscribed books, archives, personalized poems written by Glissant, etc.) and watermarking photos in order to convince people to securely share their personal archives. This also serves to remind users of the source of the material and to pay tribute to contributors who took the time to send their personal documents to our team.

\placefigure{Works by Édouard Glissant: Inscribed Books}{\externalfigure[images/odell-jegousso/LibraryofGlissantStudies_Image1.png]}
Since doing so, we have seen an increase in the number of contributions and hope to continue to see these numbers rise. We are currently working to create individual archival funds, which will collect in one place the documents related to Glissant that belong to a single individual. This new initiative has been successful and has encouraged several scholars and several of Glissant's former colleagues to share their materials with the project.

\placefigure{Works by Édouard Glissant: Manuscripts}{\externalfigure[images/odell-jegousso/LibraryofGlissantStudies_Image2.png]}
The {\em Library of Glissant Studies} has received an increasing amount of support over the past few months. It has been added to the {\em Caribbean Literary Heritage} digital collection, it has been included in several university libraries in Florida and Louisiana, and numerous newspaper articles about the project have been published in France, Japan, and the United States. However, there are still several ongoing challenges that we strive to address. For example, our team has thus far not been successful in accessing data in Eastern Europe, Eastern Asia, or several African countries, so we are working to establish and develop new partnerships with individuals in these localities in order to index more materials, which would make these works of literary criticism available to a greater number of users. We also hope that by presenting the project in as many forums as possible we will be able to receive more feedback to improve our search options, further develop our interface, and add forums for people to ask questions, exchanges ideas, and discuss pedagogical strategies.

\thinrule

\page
\subsection{Jeanne Jégousso}

Jeanne Jégousso is the cofounder and codirector of the {\em Library of Glissant Studies}. She is an assistant professor of French at Hollins University (VA), where she teaches francophone literatures and culture of the Caribbean and the Indian Ocean. She has published articles in the journal {\em Nouvelles Études Francophones} and in the books {\em La Louisiane et les Antilles, une nouvelle région du monde} (Presses Universitaires des Antilles, 2019) and {\em Édouard Glissant: L'éclat et l'obscur} (Presses Universitaires des Antilles, 2019). She is also a coeditor of the collection {\em Teaching, Reading, and Theorizing Caribbean Texts} (Lexington, forthcoming). She is a member of the board of the Centre international d'études Édouard Glissant, and she is currently working on her first book titled \quotation{La poétique du dépassement dans les littératures contemporaines des Antilles et de l'Océan Indien.}

\subsection{Emily O'Dell}

Emily O'Dell received her PhD from Louisiana State University and is currently an English lecturer at Georgia College. Her articles have been featured in {\em Postcolonial Interventions}, the {\em Louisiana Folklife Journal}, the {\em Louisiana Folklore Miscellany}, and {\em Atlantic Studies}, as well as in the collected volumes {\em La Louisiane et les Antilles, une nouvelle région du monde} (Presses Universitaires des Antilles, 2019) and {\em Utopia and Dystopia in the Age of Trump: Images from Literature and Visual Arts} (Rowman and Littlefield, 2019){\em .} She is also a coeditor of the collection {\em Teaching, Reading, and Theorizing Caribbean Texts} (Lexington, forthcoming).

\stopchapter
\stoptext