\setvariables[article][shortauthor={Glover, Gil}, date={February 2020}, issue={4}, DOI={https://doi.org/10.7916/archipelagos-39cd-mn05}]

\setupinteraction[title={},author={Kaiama L. Glover, Alex Gil}, date={February 2020}, subtitle={}, state=start, color=black, style=\tf]
\environment env_journal


\starttext


\startchapter[title={}
, marking={}
, bookmark={}]


\startlines
{\bf
Kaiama L. Glover
Alex Gil
}
\stoplines


It has been something of a particular challenge to reach the finish line with this fourth issue of our journal, and this for two reasons---both of them, in the end, exhilarating. First, there is what feels like a momentous transition: hatched four years ago and graciously incubated in the luxurious intellectual nest that is the Small Axe Project, we now set off on our very first solo flight. From {\em sx archipelagos} we have evolved into the newly-named {\em archipelagos journal}---a critical space that is at once very much of its origins and original unto itself, both devoted to SXP's foundational critical intentions and committed to forms of experimentation and networks of expansion that are unique to our digital Caribbean context. Second, we are making a joyful and ambitious move toward plurilingual representation. Our Caribbean speaks in many tongues, and honoring that reality through translation is our urgent priority. We are thrilled to launch this issue with its trilingual back-end and opening to the possibility of publication here in Spanish and French as well as English---an initial, modest effort toward building a more inclusive and relational intellectual space. There is more to do, we know, and we aim to do it in time, with the help of and in service to the very community that sustains us.

Questions of community, relation, and service run through every one of the contributions to this issue. In our first section, Elena Machado calls on us to think carefully about the ethics of mobilizing community via the fraught context of social media. She looks closely at Lin-Manuel Miranda's crafty stagings of intimacy and \quotation{emotional debt} with his Twitter followers, comparing his use of the platform to that of Latinx creative actors who operate in a decidedly \quotation{more minor key.} Machado's essay also reflects on and models, with bold transparency, an ethical practice of scholarly engagement with social media authors. Detailing the successes and failures of the methodology she deployed to constitute the archive powering her research, Machado both shows and tells her hard-won best archival practices.~

Kimberly Takahata, Jeanne Jegousso and Emily O'Dell, and Chelsea Stieber similarly offer thoughtful reflections on Caribbean archives, and do so from a variety of perspectives---all of which involve questions of collaboration, translation, accessibility, and experimentation. In an essay that is in fruitful dialogue with much of the work we have published in earlier issues of {\em archipelagos}, Takahata outlines the counter-plantational digital praxis that undergirds the deeply collaborative \quotation{Digital Grainger} project. Takahata lays out the strategies and queries that determined the project's purposeful re-presentation of a colonial text. Her essay highlights the exemplary digital design practices that can allow contestatory readings of white supremacist narratives. In their essay devoted to the {\em Library of Glissant Studies} site, Jeanne Jegousso and Emily O'Dell make a case for the unique affordances of the digital in building scholarly community across the borders of nation and language that are the legacies of empire in our academic lives.~

The section's final two essays evoke the particular \quotation{care and maintenance work} required to embark on digitization projects for Caribbean-sited archives. Chelsea Stieber, Watson Denis, and Laura Wrubel relate the theoretical and practical dimensions of a partnership between a longstanding but under-resourced Haitian intellectual institution and a network of scholars and institutions in the United States. Their essay generously reveals the technical and interpersonal dynamic behind the open-access, multilingual platform they've developed as a US-Haitian team. Amalia Levi and Tara Inniss are equally thorough in their account of the specific processes involved in the twinned practices of digitization and preservation behind their decolonizing engagement with a particular Bajan periodical. Their essay beautifully lays out the political and intellectual stakes of intervening in an economically imperiled and environmentally vulnerable Caribbean archive.

Our project peer review section features the first stage in an ongoing exchange between our blind reviewers and Schuyler Esprit, motive force behind the digital environmentalist platform Carisealand. Esprit's project recognizes the multi-vectored man-and-nature-made perils facing the Americas, and so proposes a multi-vectored approach to confronting them: practices of pedagogy, archiving, and storytelling are but a few of the registers of her project. In this exchange, Esprit responds to our review by laying out the challenges presented by the unique circumstances of the project's creation and outlines future plans for its development.

Bringing us full circle to a final engagement with social media and {\em Latinidad}, we wrap up this issue with Élika Ortega's public review of @protestitas and @TinyProtests. Ortega looks at this pair of bots of conviction---one in Spanish, the other in English---created by Puerto Rican scholar of e-poetry Leonardo Flores, and asks to what extent transnational solidarity can be facilitated by the parallel mobilization of linguo-cultural contexts that both speak to and diverge from one another.

{\em Et voilà}. Welcome to the fledgling issue of {\em archipelagos \letterbar{} a journal of Caribbean digital praxis}. We embark on this next phase of the adventure humbly, and with gratitude to David Scott and the whole of the Small Axe Project team for so well preparing us to soar.

Onward,

Kaiama & Alex

Editors

\page
\subsection{Kaiama L. Glover}

\useURL[url1][https://barnard.edu/profiles/kaiama-l-glover][][Kaiama L. Glover]\from[url1] is Associate Professor of French and Africana Studies at Barnard College, Columbia University. She is the author of \useURL[url2][http://liverpooluniversitypress.co.uk/products/61903][][Haiti Unbound: A Spiralist Challenge to the Postcolonial Canon]\from[url2] (Liverpool UP 2010), first editor of \useURL[url3][http://yalebooks.com/book/9780300214192/yale-french-studies-number-128][][Marie Vieux Chauvet: Paradoxes of the Postcolonial Feminine]\from[url3] (Yale French Studies 2016), and translator of Frankétienne's Ready to Burst (Archipelago Books 2014). She has received awards and fellowships from the National Endowment for the Humanities, the Mellon Foundation, and the Fulbright Foundation. Current projects include forthcoming translations of Marie Vieux Chauvet's {\em Dance on the Volcano} (Archipelago Books) and René Depestre's {\em Hadriana in All My Dreams} (Akashic Books), and the multimedia platform {\em In the Same Boats: Toward an Afro-Atlantic Visual Cartography}.

\subsection{Alex Gil}

\useURL[url4][http://www.elotroalex.com/][][Alex Gil]\from[url4] is the Digital Scholarship Librarian at Columbia University Libraries. His research and practice focuses on digital humanities, epistemic design, minimal computing, and Caribbean literature. He is co-founder and moderator of \useURL[url5][http://xpmethod.plaintext.in/][][Columbia's Group for Experimental Methods in Humanistic Research]\from[url5], and the Studio@Butler at Columbia University Libraries.

\stopchapter
\stoptext