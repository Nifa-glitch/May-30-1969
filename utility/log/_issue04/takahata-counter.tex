\setvariables[article][shortauthor={Takahata}, date={February 2020}, issue={4}, DOI={https://doi.org/10.7916/archipelagos-2c5j-mc45}]

\setupinteraction[title={A Caribbean Counter-Edition: *Digital Grainger* and the Breaking of James Grainger's *The Sugar-Cane*},author={Kimberly Takahata}, date={February 2020}, subtitle={A Caribbean Counter-Edition}, state=start, color=black, style=\tf]
\environment env_journal


\starttext


\startchapter[title={A Caribbean Counter-Edition: {\em Digital Grainger} and the Breaking of James Grainger's {\em The Sugar-Cane}}
, marking={A Caribbean Counter-Edition}
, bookmark={A Caribbean Counter-Edition: *Digital Grainger* and the Breaking of James Grainger's *The Sugar-Cane*}]


\startlines
{\bf
Kimberly Takahata
}
\stoplines


{\startnarrower\it An approximately twenty-five-hundred-line poem, James Grainger's {\em The Sugar-Cane} (1764) details the process of sugar production on eighteenth-century Caribbean plantations and celebrates the commercial might of the British Empire. This article discusses {\em Digital Grainger: An Online Edition of \quotation{The Sugar-Cane} (1764)} and how it challenges Grainger's poetic imperialism by presenting readers with a counter-edition, versus an edition, of Grainger's poem. Furthermore, it examines the central connection between the digital nature of the counter-edition and its ability to reflect editing practices inspired by Caribbean traditions of postcolonial scholarship. In particular, the {\em Digital Grainger} team built an openly accessible edition of the poem that can be loaded on lower internet capacities and a variety of devices, attending to the embodied and material conditions of the digital archive. The site provides more than seven hundred footnotes to contextualize and challenge Grainger's authority and proslavery argument, building networks around the text to offer readers alternative understandings of the poem. The primary intervention, however, is \quotation{The Counter-Plantation,} comprised of excerpts that collect and highlight themes of Afro-Caribbean and indigenous expertise, survival, and resistance. As such, {\em Digital Grainger} stages this intervention by \quotation{breaking apart} the poetic lines and footnotes of {\em The Sugar-Cane} to highlight how possibilities for life were structural elements of settler colonialism and plantation slavery. These methods emphasize long histories of Caribbean exchange and encounter, ultimately working to read a colonial text in a manner that does not replicate its mission and violence. At the same time, {\em Digital Grainger} reveals that further work must be done to truly decolonize Caribbean digital archives, confronting the legacy of colonial silencing.

 \stopnarrower}

\blank[2*line]
\blackrule[width=\textwidth,height=.01pt]
\blank[2*line]

Over the past few decades, the colonial Caribbean has served as a site of robust digital work. Most well known is \useURL[url1][https://www.slavevoyages.org/][][{\em Voyages: The Trans-Atlantic Slave Trade Database}]\from[url1], which contains information about thirty-six thousand slaving voyages from the sixteenth through the nineteenth centuries. There are numerous others, however, including \useURL[url2][https://ecda.northeastern.edu/][][{\em The Early Caribbean Digital Archive}]\from[url2], the goal of which is to assemble a similarly wide-ranging collection of texts about the colonial Caribbean. Other projects focus on more localized historical events and spaces, including \useURL[url3][http://revolt.axismaps.com/][][{\em Slave Revolt in Jamaica, 1760--1761}]\from[url3] and \useURL[url4][https://historydesignstudio.com/projects/two-plantations-enslaved-families-in-virginia-and-jamaica][][{\em Two Plantations}]\from[url4]. Finally, projects like \useURL[url5][http://www.musicalpassage.org/][][{\em Musical Passage: A Voyage to 1688 Jamaica}]\from[url5] excavate the deep histories of single pages within texts in an attempt to recover information about the lives of enslaved and other marginalized subjects.\footnote{See {\em Voyages: The Trans-Atlantic Slave Trade Database}, \useURL[url6][https://www.slavevoyages.org/voyage/database]\from[url6]; {\em The Early Caribbean Digital Archive}, \useURL[url7][https://ecda.northeastern.edu/]\from[url7]; {\em Slave Revolt in Jamaica, 1760--1761}, \useURL[url8][http://http://revolt.axismaps.com/project.html][][http://revolt.axismaps.com/project.html]\from[url8]; {\em Two Plantations}, \useURL[url9][http://twoplantations.com/]\from[url9]; and {\em Musical Passage: A Voyage to 1688 Jamaica}, \useURL[url10][http://www.musicalpassage.org/]\from[url10].} As diverse as these projects have been in form and content, they are united by their desire to use digital technologies to question, challenge, and mediate the archives of empire and slavery that serve as their sites of analysis.

\useURL[url11][https://digital-grainger.github.io/grainger/][][{\em Digital Grainger: An Online Edition of \quotation{The Sugar-Cane} (1764)}]\from[url11] is one of the latest projects in this line of critical digital humanities work on the colonial Caribbean. {\em Digital Grainger} takes as its primary object James Grainger's 1764 {\em The Sugar-Cane}, an approximately twenty-five-hundred-line poem that describes the workings of the eighteenth-century Caribbean sugar plantation. Created by a team of scholars from Columbia and Fordham Universities, including Julie Chun Kim, Cristobal Silva, Alex Gil, Kimberly Takahata, Ami Yoon, Steve Fragano, Lina Jiang, and Elizabeth Cornell, {\em Digital Grainger} provides an open and accessible version of the poem for undergraduate and graduate teaching. As a digital edition of a text increasingly taught in courses about the early Caribbean and the early Americas, the project prompts students and other readers to grapple with how to read a text that explicitly advocates for the continuation of the plantation system. In particular, the edition seeks to guide readers both to realize this text's complicity in colonialism and slavery and to ask the questions, What can {\em The Sugar-Cane} reveal to us not only about colonial ideologies but also about the histories and lives of the enslaved and other marginalized subjects who sustained the eighteenth-century economy of sugar? Are there ways to read a colonial text so that one does not simply accept or replicate its biases or even our own expectations as we develop interpretations and analyses?

Critical work over approximately the last fifteen years has laid the ground for asking such questions about {\em The Sugar-Cane}. Read by Jim Egan in 2003 as a testament to colonial anxieties about sustaining British identity in a distant, tropical environment, the poem also has been interpreted as expressing Grainger's doubts about his ability to maintain control over the tropical landscapes and sugar plantations he describes as well as over the enslaved persons who labored on them.\footnote{Jim Egan, \quotation{The \quote{Long'd-for Aera} of an \quote{Other Race}: Climate, Identity, and James Grainger's {\em The Sugar-Cane},} {\em Early American Literature} 38, no. 2 (2003): 189--212.} Moreover, while some studies have shown how Grainger sought to shore up his authority through scientific and medical footnotes, others have begun to challenge that authority by identifying the enslaved and indigenous sources of information and expertise that would have provided Grainger with much of his understanding of plantation ecologies.\footnote{For a full list of sources consulted and cited during the project, see the bibliography for {\em Digital Grainger}, \useURL[url12][https://digital-grainger.github.io/grainger/bibliography.html]\from[url12].}

Building on these conversations, {\em Digital Grainger} uses the space of the digital to open editorial possibilities not afforded by print media to draw out these networks and practices. As a result, this project serves as one manifestation of what Roopika Risam terms \quotation{postcolonial digital humanities,} which seeks to use new tools and methods to \quotation{intervene in colonial and neocolonial dimensions of the digital cultural record} with the goal of \quotation{{[}creating{]} space for underrepresented communities to populate the digital cultural record with their own stories.}\footnote{Roopika Risam, {\em New Digital Worlds: Postcolonial Digital Humanities in Theory, Praxis, and Pedagogy} (Evanston, IL: Northwestern University Press, 2018), 9.} Our primary strategy involves creating what we call a \quotation{counter-edition} containing multiple versions of the poem. Rather than simply replicating Grainger's poem and thereby allowing his voice to remain unchallenged, we instead seek to surround and undermine Grainger's status as a singular author by calling attention to the other authorities that he relied on to construct his poem. We also present readers with multiple reading experiences, including excerpts that constitute what we call \quotation{The Counter-Plantation} section of our edition. By excerpting {\em The Sugar-Cane}, we encourage our readers to see the poem as a work that can be manipulated and even broken apart to reveal stories other than the ones Grainger wanted to tell. Ultimately, as we conclude, these interventions reveal the indigenous and Afro-Caribbean networks of knowledge that undergird {\em The Sugar-Cane}, and our edition therefore works to challenge Grainger's goal of imposing order on the plantation and his own authority over readers. Nevertheless, even with these interventions, the complexities of {\em The Sugar-Cane} and other colonial texts, which actively worked to elide and suppress the contributions of marginalized agents and individuals, means that further creative digital work is needed to recover their lives and experiences.

\subsection[title={{\em The Sugar-Cane}'s Multiple Authorities},reference={the-sugar-canes-multiple-authorities}]

James Grainger, the author of {\em The Sugar-Cane}, was born in the mid-1720s in Duns, Scotland. Shortly after he was orphaned at the age of ten, he was apprenticed to an Edinburgh surgeon and eventually enrolled at the University of Edinburgh to study medicine. He served in the Jacobite Rising of 1745 and the War of Austrian Succession (1740--48) as a regimental surgeon and afterward moved to London to practice medicine. Even though Grainger was a doctor, his true passion, arguably, was literature, and he had ambitions of becoming a well-known and respected poet. His move to London therefore coincided with his formation of connections with such prominent authors as Samuel Johnson, Bishop Thomas Percy, Oliver Goldsmith, William Shenstone, and Charlotte Lennox. During this period, Grainger succeeded in publishing several short poems, reviews, and translations of Tibullus's elegies.

Nevertheless, Grainger could not support himself financially with his writing and traveled to the Caribbean in 1759 under the patronage of former pupil John Bourryau. Grainger embarked on a sea voyage to St. Kitts, where Bourryau had recently inherited a sugar estate. Grainger soon found himself with an even more direct connection to sugar plantations, however: during his Atlantic voyage, Grainger treated a woman infected with smallpox. He also met her daughter, Daniel Matthew Burt, whom he married soon after arriving in St.~Kitts. Although Burt came from a family of wealthy planters, Grainger himself did not have the means to immediately establish his own plantation. Still, he concluded his patronage and became a plantation physician with the hopes of eventually becoming a planter himself.

Even as his situation changed, Grainger maintained the literary ambitions he entertained while living in London. As a result, he wrote an early draft of {\em The Sugar-Cane} while in St.~Kitts, sending a manuscript back to England in 1762 for Percy's review. The first edition of {\em The Sugar-Cane} was then published in London in 1764, with two more editions published in London and Dublin in 1766. There were also excerpts of {\em The Sugar-Cane} published in numerous magazines and anthologies. Grainger's poem is a self-proclaimed \quotation{West-India georgic}; that is, it is a georgic poem set in and written about the West Indies.\footnote{James Grainger, preface to {\em The Sugar-Cane: A Poem} (London: R. and J. Dodsley, 1764), vii; hereafter cited in the text. Poem lyrics are cited by book, page, and line numbers.} Modeled after Virgil's {\em Georgics}, eighteenth-century neogeorgic poems like Grainger's followed traditional georgics in their use of four books to instruct readers in the arts of agriculture and to praise the effects of agricultural labor on society. Grainger's decision to characterize his work this way makes a decisive statement about his ambitions: he wanted to establish himself as a writer fit to follow in the footsteps of Virgil, and he sought to demonstrate the integral role the Caribbean colonies and their agricultural economy played in sustaining the British Empire.\footnote{Between 1764 and 1765, Grainger published {\em The Sugar-Cane}, {\em An Essay on the More Common West-India Diseases}, and \quotation{Bryan and Pereene, a West-Indian Ballad,} which was included in Percy's {\em Reliques of Ancient English Poetry}. Grainger died in St.~Kitts in December 1766. See John Gilmore, {\em Poetics of Empire: A Study of James Grainger's The Sugar Cane} (London: Athlone, 2000); Thomas W. Krise, ed., {\em Caribbeana: An Anthology of English Literature of the West Indies, 1657--1777} (Chicago: University of Chicago Press, 2009); and Anna Foy, \quotation{Getting to Know James Grainger,} {\em OUPblog}, 31 March 2017, \useURL[url13][https://blog.oup.com/2017/03/getting-know-james-grainger/]\from[url13].} Yet in order to enact such goals, Grainger had to balance the aesthetic demands of the genre with the physical reality of the plantation system. In other words, by looking to refashion the Caribbean in a European image, he sought to transplant the georgic onto a locale and political and economic system for which it was not originally designed. This tension is visible throughout the poem, such as in a moment when he describes an aspect of sugar harvesting as comparable to the shearing of sheep:

\startblockquote
SOME bending, of their sapless burden ease\\ The yellow jointed canes, (whose height exceeds\\ A mounted trooper, and whose clammy round\\ Measures two inches full;) and near the root\\ Lop the stem off, which quivers in their hand\\ With fond impatience\ldots{}\\\strut ~~~~~~~~~~~~~~~~~~~\ldots{}What of the Cane\\ Remains, and much the largest part remains,\\ Cut into junks a yard in length, and tied\\ In small light bundles; load the broad-wheel'd wane,\\ The mules crook-harnest, and the sturdier crew,\\ With sweet abundance. As on Lincoln-plains,\\ (Ye plains of Lincoln sound your Dyer's praise!)\\ When the lav'd snow-white flocks are numerous penn'd;\\ The senior swains, with sharpen'd shears, cut off\\ The fleecy vestment; others stir the tar;\\ And some impress, upon their captives sides,\\ Their master's cypher; while the infant throng\\ Strive by the horns to hold the struggling ram,\\ Proud of their prowess. (3.93--94.111--16, 125--38)
\stopblockquote

Here, Grainger presents the cane as practically harvesting itself, omitting the subject of the verbs \quotation{ease} and \quotation{lops.} Sugarcane \quotation{quivers in their hand with fond impatience,} presenting harvesters as merely collecting the plants that naturally and spontaneously arise from an ever-fruitful landscape. And while Grainger briefly mentions the specific labor that harvesting cane requires, specifying that workers had to cut the cane into \quotation{junks a yard in length} and tie them into \quotation{small light bundles,} he refers to the enslaved who would have been doing this work as \quotation{swains,} obscuring the conditions under which they labored. He also evokes a more traditionally pastoral subject by turning from the harvesting of cane to the shearing of sheep. Nevertheless, in these lines, the description of the sheep as \quotation{captives} harkens back to the status of the enslaved themselves, and Grainger even encodes the common practice of branding enslaved laborers by describing the branding of sheep with their \quotation{master's cypher.} In this sense, the poem fails to use aestheticized tropes and language to suppress the mismatch between plantation agriculture and idealized, Virgilian cultivation.\footnote{For a fuller discussion of these excerpts, see Lina Jiang's introduction to \quotation{Sugar Work,} {\em Digital Grainger}, \useURL[url14][https://digital-grainger.github.io/grainger/excerpts/sugar-work.html]\from[url14].}

Such moments when Grainger's awareness of the realities of enslavement cannot be mediated through poetic language appear throughout the poem. In particular, he has considerable difficulty obscuring the harsh and brutal realities of living and working on a sugar plantation. For example, as Grainger narrates the process of feeding cane stalks into the mill to extract their sugary syrup,

\startblockquote
AND now thy mills dance eager in the gale;\\ Feed well their eagerness: but O beware;\\ Nor trust, between the steel-cas'd cylinders,\\ The hand incautious: off the member snapt\\ Thou'lt ever rue; sad spectacle of woe! (3.95.165--69).
\stopblockquote

The footnote that follows reads, \quotation{This accident will sometimes happen, especially in the night: and the unfortunate wretch must fall a victim to his imprudence or sleepiness, if a hatchet do not immediately strike off the entangled member; or the mill be not instantly put out of wind,} by which Grainger means that the sugar mill, powered by wind, would stop turning (95). Personifying the mills as he previously did the sugarcane to avoid commenting on the harsh conditions of the plantation, Grainger is nevertheless still forced to confront the dangers that the enslaved faced every day. Even so, he seeks to compartmentalize blame, characterizing the hand of the enslaved as \quotation{incautious} and the laborer as imprudent or sleepy, while omitting the fact that enslavers during the height of sugarcane harvesting forced laborers to work around the clock with no rest.

This rhetorical strategy serves an underlying goal: Grainger primarily used these instances to promote amelioration, arguing that if plantation owners made conditions more bearable and engaged in \quotation{humane} treatment of the enslaved, they would experience higher profits. Put another way, by treating the enslaved as kindly as actual swains, Grainger believed plantation yields (including his own profits) would increase.\footnote{For an analysis of how Grainger's use of medical rhetoric structures contradictory readings of the poem as both rationalizing and critiquing slavery, see Steven W. Thomas, \quotation{Doctoring Ideology: James Grainger's {\em The Sugar Cane} and the Bodies of Empire,} {\em Early American Studies} 4, no. 1 (2006): 78--111. For a discussion of Grainger's negotiation of enslaved labor in the georgic, see Markman Ellis, \quotation{\quote{Incessant Labour}: Georgic Poetry and the Problem of Slavery,}in {\em Discourses of Slavery and Abolition}, ed.~Brycchan Carey, Markman Ellis, and Sara Salih (New York: Palgrave Macmillan, 2004), 45--62; and Cristobal Silva, \quotation{Georgic Fantasies: James Grainger and the Poetry of Colonial Dislocation,} {\em ELH} 83, no. 1 (2016): 127--56.} At the same time, emphasizing the \quotation{eagerness} of the mill and its resulting goods depicts ongoing plantation production as practically inevitable, a logic necessary in a poem that demonstrates the West Indies' importance to the British Empire. By using poetic language to navigate these discussions, {\em The Sugar-Cane} presents improved conditions and sustained production as related goals, allowing those in power in the Caribbean and Britain to continue profiting from plantation production.

The above passage about the feeding of cane into the mill also reveals how Grainger utilized both the body of the poem and its footnotes to establish his narrative authority. While the poem as a whole functions as a georgic, it is in the footnotes that Grainger could include specific and concrete examples of his expertise in plantation life. Grainger used this space of the poem to comment on the day-to-day operations of a St.~Kitts plantation as well as to include information about the medicinal uses of various plants and to make extensive references to other scientific and agricultural texts about the Caribbean islands more broadly. It is thus in the footnotes that the \quotation{West-India} modifier of the poem is most visible, as Grainger offers Latinate plant names and histories of their applications, at times noting their geographic spread and trade distribution. With such explanations, he sought to include himself in the prestigious circles of natural historians whose works were highly valued and widely circulated amongst British audiences in the long eighteenth century.\footnote{See Christopher P. Iannini, {\em Fatal Revolutions: Natural History, West Indian Slavery, and the Routes of American Literature} (Chapel Hill: University of North Carolina Press, 2012).}

These footnotes, however, often specify Afro-Caribbean and indigenous use of plants, and, although Grainger does not explicitly identify his sources or instructors, the glimpses they afford of marginalized knowledge indicate the extent to which his expertise derived from that of others. Beth Fowkes Tobin has pointed out, for example, that even though Grainger attempts to categorize enslaved Africans themselves as plants, animals, and other natural objects, carefully reading his footnotes underscores direct links between the produce grown by the enslaved on provision grounds or gardens and Grainger's lengthy descriptions of tropical plants.\footnote{See Beth Fowkes Tobin, {\em Colonizing Nature: The Tropics in British Arts and Letters, 1760--1820} (Philadelphia: University of Pennsylvania Press, 2004), 46, 51.} Indeed, on the mountainous island of St.~Kitts, plantations were typically located along the coasts, with provision grounds located in the less rich soil of the rockier terrains of the mountains, a wide swath of land in regard to the island as a whole.\footnote{For a history of the relations between indigenous persons, including a general overview of the specific geography of each island and its role in colonial development, see Philip P. Boucher, {\em Cannibal Encounters: Europeans and Island Caribs, 1492--1763} (Baltimore: Johns Hopkins University Press, 1992). See also Anthony Ravell and Thomas Jefferys, cartographers, {\em Isle St.~Christophe ou St.~Kitts}, 1779, 45.5 x 59.5 cm, Bibliothèque nationale de France, \useURL[url15][https://catalogue.bnf.fr/ark:/12148/cb43835771j]\from[url15]; Anthony Ravell, cartographer, and Thomas Jefferys, engraver, {\em St Christophers surveyed}, {[}17..?{]}, 47 x 61 cm, Bibliothèque nationale de France, \useURL[url16][https://catalogue.bnf.fr/ark:/12148/cb40717619n]\from[url16]; and John Lodge and John Bew, authors, {\em An accurate map of the island of St.~Christophers from an actual survey}, 1782, {[}360 x 280?{]}, Bibliothèque nationale de France, \useURL[url17][https://catalogue.bnf.fr/ark:/12148/cb40666170n]\from[url17].} Building on Tobin's analysis, Britt Rusert argues that the proliferation of footnotes in Grainger's poem signals Grainger's fears about the \quotation{disordering} potential of \quotation{tropical diseases, wild animals, and maroon communities} and reveals the fragility of plantation order that he was trying to construct via his poem.\footnote{Britt Rusert, \quotation{Plantation Ecologies: The Experimental Plantation in and against James Grainger's {\em The Sugar-Cane},} {\em Early American Studies} 13, no. 2 (2015): 345--46. Rusert's analysis builds on the work of Jill H. Casid, who emphasizes the hybrid nature of colonial landscapes. See Casid, {\em Sowing Empire: Landscape and Colonization} (Minneapolis: University of Minnesota Press, 2005). For a discussion of the diaspora of plants as well as people, including an examination of Maroon subsistence strategies, see Judith A. Carney and Richard Nicholas Rosomoff, {\em In the Shadow of Slavery: Africa's Botanical Legacy in the Atlantic World} (Berkeley: University of California Press, 2009).} In these studies, then, Grainger emerges as a figure who sought to order Caribbean nature based on colonial models and desires but failed to do so because of the realities of the island's geography and resistance of its inhabitants, and scholars have recast {\em The Sugar-Cane} as a work that reveals networks of expertise subtending his own, singular performance of authorship. With the footnotes that sprawl across his pages, Grainger seeks to place his own knowledge at the center of the poem and reader's experience, but, simultaneously, these footnotes attest to other forms of knowledge emerging from the peripheries.\footnote{For discussions of how the poem uses recognizably paradisiacal and idyllic language to negotiate cultural and political authority between Britain and the Caribbean, see Shaun Irlam, \quotation{\quote{Wish You Were Here}: Exporting England in James Grainger's {\em The Sugar-Cane},}{\em ELH} 68, no. 2 (2001): 377--96; and Lorna Burns, \quotation{Landscape and Genre in the Caribbean Canon: Creolizing the Poetics of Place and Paradise,}{\em Journal of West Indian Literature} 17, no. 1 (2008): 20--41.}

\subsection[title={The Design of {\em Digital Grainger}},reference={the-design-of-digital-grainger}]

How exactly to replicate and represent these footnotes, however, has been a question for those seeking to create editions of the poem. Perhaps because he was so anxious to demonstrate his knowledge of nature, Grainger often included footnotes that took up more space on his pages than his lines of verse. Editors therefore have utilized divergent strategies for dealing with Grainger's footnotes. For example, John Gilmore's {\em Poetics of Empire}, which contains the complete text of the poem and extensive annotations by Gilmore, places Grainger's footnotes at the end of the poem and therefore presents them as endnotes. Conversely, Thomas W. Krise includes the complete text of the poem in his anthology {\em Caribbeana} but tries to replicate Grainger's footnote arrangement by positioning them at the bottom of the poem's pages. Yet because Grainger's footnotes take up so much room, there is no place to put Krise's own annotations and explanations on the same page as the lines of verse, forcing Krise to separate them from the poem and include this information only as endnotes. Such a move is far from ideal, since, as the history of the poem suggests, Grainger's poem and particularly its proplantation sympathies need considerable contextualization for present-day readers to maintain the goal of challenging Grainger's colonial vision.

\placefigure[here]{Grainger's Footnotes}{\externalfigure[issue04/takahata/footnote-page.jpg]}


As such, one of the main aims of creating {\em Digital Grainger} was to use the possibilities afforded by a digital edition to circumvent the space limitations faced by editors of print editions of the poem.\footnote{Although digital space also comes with costs, we decided to host our edition on GitHub, which, for the time being, provides users with free storage space as long as they make all of their code and materials public. Given that we were ourselves using some open-source materials and hope others find our edition a useful model, we were happy to make our code and materials available.} While we wanted to reproduce the poem in its entirety for those who might want to read or teach the whole text, we also wanted to surround and, more importantly, to challenge Grainger's lengthy poem and footnotes with our own substantial commentary on Afro-Caribbean and indigenous experiences.

The {\em Digital Grainger} team therefore made the decision to produce not one but multiple versions of the poem. Two of these versions appear in a section of {\em Digital Grainger} that is called \quotation{The 1764 Edition.} Both versions include the complete text of Grainger's poem and all of Grainger's footnotes. The two versions differ, however, in how they present the poem to readers: (1) the \quotation{full text} version includes the entire poem and all of Grainger's footnotes on one webpage, and (2) the \quotation{page-by-page} version presents readers with individual pages of the poem, which they can click through. In the page-by-page version, we replicated the layout of the original 1764 edition in the sense that we numbered our pages in the same way that the 1764 edition's pages were numbered, and we included the same text on each of our pages that was on the corresponding pages of the 1764 edition. We also included in this version links to scanned images of pages from the 1764 edition.

\placefigure[here]{A Page-by-Page View}{\externalfigure[issue04/takahata/footnote-view.jpg]}


In other words, the versions included in \quotation{The 1764 Edition} section of {\em Digital Grainger} were produced in accordance with many of the standard conventions of scholarly editing. We included the entire poem, and we also tried to give readers a sense of how the poem looked when it appeared in 1764. The versions we present in \quotation{The 1764 Edition} section, in fact, could be viewed as forming a semidiplomatic edition of the poem, since we largely treated the poem as a historical document and reproduced the language from the 1764 edition without any corrections to standardize Grainger's spelling or punctuation.\footnote{See M. J. Driscoll et al., \quotation{Levels of Transcription,} in {\em Electronic Textual Editing}, ed.~Lou Barnard, Katherine O'Brien O'Keeffe, and John Unsworth (New York: Modern Language Association of America, 2006), 254--61.} We also triple-checked our transcription to ensure accuracy.

Nevertheless, even while committing to a faithful reproduction of Grainger's words, we made several design choices to introduce alternative, nonhegemonic modes of interpreting {\em The Sugar-Cane}. Our first major intervention comes with our site design, which attends to the embodied and material conditions of the digital archive.\footnote{For a delineation between decolonial and postcolonial computing, see Syed Mustafa Ali, \quotation{Towards a Decolonial Computing,}\useURL[url18][https://www.academia.edu/6524133/Towards_a_Decolonial_Computing]\from[url18].} Our transcriptions are built from Markdown files, and the site itself was created using Ed, a Jekyll theme designed under the principles of minimal computing.\footnote{For a definition of minimal computing, see \useURL[url19][http://go-dh.github.io/mincomp/about/]\from[url19].} As a result, the site is static in regard to user generation and does not require constant interaction with a server to load. This decision to minimize the amount of markup meant that our team primarily worked with Markdown, HTML (hypertext markup language), and CSS (cascading style sheets), resulting in a site that can be loaded on lower internet capacities. Most importantly, this design enables access from those living in the Caribbean, for although the world is going increasingly digital, access to these resources and preservation of material remains uneven. And because this access also impacts what devices are available to users, {\em Digital Grainger} is designed to load on a variety of screen sizes, from computer and laptop to tablet and phone. Additionally, replicating the full text of {\em The Sugar-Cane} provides readers with free, online versions of the poem, otherwise primarily available through the aforementioned printed editions or in databases behind institutional paywalls.

The second major intervention involves the contextual materials that we provide to readers, in the hope that these will enable them to see through Grainger's elisions of enslaved and indigenous authority. To achieve this goal, the {\em Digital Grainger} team added more than seven hundred footnotes explaining terms used by Grainger. And while many annotations explain Grainger's numerous classical allusions (derived from Virgil and Greco-Roman contexts), many of the notes also include botanical information about Afro-Caribbean and indigenous uses of plants. This choice reflects the project's goal of highlighting the expansive networks of expertise and agency that exist at the periphery of Grainger's original poem. Especially since space limitations were not a consideration, {\em Digital Grainger} did not have to limit the number of editorial footnotes added to Grainger's text. We were therefore able to \quotation{surround} Grainger's text with our own commentary. Moreover, because our footnotes are on the same page as Grainger's text---again, since there is no digital limit on the size of pages---readers have easy access to them. As a result, while it could be said that \quotation{The 1764 Edition} primarily stages its interventions on the edges of the poem, building frameworks underneath and around it rather than directly intervening in the original text, the site design and editorial footnotes intellectually, visually, and structurally challenge Grainger's attempts to construct himself as uniquely knowledgeable and give readers alternate contexts for understanding the poem.

\subsection[title={The Counter-Plantation},reference={the-counter-plantation}]

Our most explicit intervention comes via the section of the site called \quotation{The Counter-Plantation.} We borrow the term {\em counter-plantation} from Jean Casimir, who first used it to characterize the historical development of various economic, political, and social practices in Haiti following independence. Moreover, this emphasis enabled an epistemological rethinking of the daily life of those living under enslavement and emancipation, opening possibilities of how to conceive of their relations and communities in ways that were not solely aligned with the violence and production of the plantation.\footnote{See Jean Casimir, {\em La Cultura Oprimida} (Mexico: Nueva Imagen, 1981).} Casimir and Yvonne Acosta later used this term to describe elements of plantation life in St.~Lucia that helped the enslaved become independent cultivators after emancipation, a stage toward the development of a peasantry. These elements included provision grounds and gardens, which the enslaved used to grow food crops for their own survival. Acosta and Casimir explain that even though the enslaved living on plantations were subject to strict discipline, the plantation nevertheless allowed for forms of agency, however minor.\footnote{Yvonne Acosta and Jean Casimir, \quotation{Social Origins of the Counter-Plantation System in St Lucia,} in {\em Rural Development in the Caribbean}, ed.~P. I. Gomes (London: C. Hurst and St.~Martin's, 1985), 34--35.}

Generally, this concept of the counter-plantation specifies a broader theme in scholarship discussing plantation life: namely, the contradiction between the \quotation{severely constrained} nature of the plantation and spaces of negotiation, agency, and creativity within the institution, a distinction between what Michel-Rolph Trouillot calls the model of the \quotation{ideal plantation} and the \quotation{different types of realities} that these spaces actually contained.\footnote{Ibid., 35; Michel-Rolph Trouillot, {\em Culture on the Edges: Caribbean Creolization in Historical Context} (Durham, NC: Duke University Press, 2002), 200--201.} Sidney W. Mintz argues across multiple essays, for instance, that what is later characterized as Afro-Caribbean resistance derives from \quotation{accommodations} or \quotation{adaptations} to plantation enslavement. Whether through producing their own subsistence or selling commodities, Afro-Caribbean persons preserved their humanity and asserted their individuality through \quotation{built-in oppositions} to the plantation, doing more than merely surviving.\footnote{See Sidney W. Mintz, {\em Caribbean Transformations} (New York: Columbia University Press, 1989).} As with Casimir's and Acosta's development of \quotation{counter-plantation,} Mintz focuses on the social, political, and economic valences of land cultivation in provision grounds and gardens in order to emphasize individual expression and creativity. This tension between plantation control over subsistence and Afro-Caribbean agency over these plots allows scholars like Woodville K. Marshall to emphasize how the enslaved controlled their time and their modes of production in these spaces. Of course, provision grounds and gardens are always fraught, defined as much by attempts to limit the freedom of the enslaved as they are by the ownership of such spaces by Afro-Caribbean persons.\footnote{See Woodville K. Marshall, \quotation{Provision Ground and Plantation Labour in Four Windward Islands: Competition for Resources during Slavery,} in {\em The Slaves' Economy: Independent Production by Slaves in the Americas}, ed.~Ira Berlin and Philip D. Morgan (London: Frank Cass, 1995), 48--67. See also the introduction in the same volume.} Yet at the same time, they serve as a concrete site of broader epistemological reframing, preserving Afro-Caribbean agency within an institution thought to eliminate this possibility.

{\em Digital Grainger} uses the framework of the counter-plantation both thematically and methodologically. In its most literal interpretation, this concept works as a rubric for identifying various forms of Afro-Caribbean and indigenous botanical and agricultural knowledge discussed in the project's footnotes, like provision grounds. Additionally, by taking the counter-plantation as an analytic prompt, {\em Digital Grainger} also uses the term to signify strategies of survival and resistance more broadly. Just as there could be agency within the plantation system, there are also numerous moments of creativity and negotiation built into the rhetoric and narrative of Grainger's poem that point to dissent from the strictures of the plantation. In order to identify these moments, \quotation{The Counter-Plantation} section of {\em Digital Grainger} contains several pages highlighting themes ranging from \quotation{Indigenous Presence} and \quotation{Provision Grounds} to \quotation{Obeah} and \quotation{Fire.} Each of these pages contains excerpts from the poem related to the chosen theme, beginning with a headnote or short essay written by team members to help readers understand the significance of the theme and the selected passages.

\placefigure[here]{The Counter-Plantation}{\externalfigure[issue04/takahata/counter-plantation.jpg]}


Another way to describe what \quotation{The Counter-Plantation} section and the various excerpts are doing is via the metaphor of breaking, in the sense that we use the excerpts to break apart the poem and provide the reader with only those passages that challenge and contradict Grainger's overarching narrative of plantation success and harmony. As Risam has noted, the paradigm of breaking is inherent to the work of digital humanities as well as postcolonial studies. Bridging the two, postcolonial digital humanities combines the creative and building possibilities of \quotation{hacking} with the mission of \quotation{reading the colonial archives against the grain.}\footnote{Risam, {\em New Digital Worlds}, 57.} Although excerpting is common editorial practice for anthologies and similar works, we see our excerpts as critical interventions, in the sense that they seek to redirect the reader's attention from what Grainger considered important. In other words, there is a kind of violence being performed on the text, as is represented by the image on our homepage. This image is an original work of art by Vanessa Lee that reproduces the 1764 edition's frontispiece, which depicted a single stalk of cane. Notably, in Lee's version of the frontispiece, the cane has been broken apart by a rat, an animal that Grainger discusses in the poem as a serious threat to cane crops.

\placefigure[here]{Breaking apart {\em The Sugar-Cane}}{\externalfigure[issue04/takahata/homepage.jpg]}


Following Kim Gallon's call to scholars to practice a black digital humanities that uses digital tools to \quotation{bring forth the full humanity of marginalized peoples,} however, we are concerned not merely with breaking apart Grainger's poem and revealing its omissions.\footnote{Kim Gallon, \quotation{Making a Case for the Black Digital Humanities,}in {\em Debates in the Digital Humanities}, ed.~Matthew K. Gold and Lauren F. Klein (Minneapolis: University of Minnesota Press, 2016), \useURL[url20][http://dhdebates.gc.cuny.edu/debates/text/55]\from[url20].} Rather, our goal is to use the excerpts to highlight possibilities for life even under the highly restrictive system of slavery and thereby to enact \quotation{breaking as a form of making anew.}\footnote{Risam, {\em New Digital Worlds}, 57.} In Grainger's sprawling, lengthy poem, moments of survival, fugitivity, and resistance may seem isolated, but when collated and placed side by side, they begin to create an alternate picture of the Caribbean that was always buried in the poem and available for recovery. The digital counter-plantation thus uses the flexibility of a digital edition to demonstrate the structural nature of these modes of resistance to Grainger's poem as well as our project, literally giving each theme its own space within the site. In other words, the space of the digital translates the analytic intervention of the counter-plantation into a design feature. For example, in the excerpt \quotation{Movement,} we include a passage that alludes to the fact that enslaved individuals often traveled between plantations at night to visit family members and friends. Planters did not approve of such movement because it occurred beyond their reach and control. As Grainger warns his readers, whom he imagines in this moment to be planters,

\startblockquote
COMPEL by threats, or win by soothing arts,\\ Thy slaves to wed their fellow slaves at home;\\ So shall they not their vigorous prime destroy,\\ By distant journeys, at untimely hours,\\ When muffled midnight decks her raven-hair\\ With the white plumage of the prickly vine. (4.158.606--11)
\stopblockquote

That Grainger would disapprove of such traveling is not surprising, of course, given the overall investment of his poem in support of the plantation system. Nevertheless, Grainger's lines inadvertently reveal the multifarious and sometimes unexpected uses to which the botanical knowledge of the enslaved could be deployed. When Grainger describes the \quotation{white plumage of the prickly vine,} he is referring to the flower commonly known as night-blooming cereus ({\em Selenicereus grandiflorus}), the large, white petals of which have an opalescent quality to allow them to be seen at night by moths and other nocturnal pollinators. In this case, however, Grainger's reference to the flower in lines that discuss the travels of enslaved individuals suggests that they, too, could have been using the reflected light from its petals to navigate. As such, putting \quotation{Movement} next to other excerpts like \quotation{Provision Grounds,} which also discusses Afro-Caribbean botanical knowledge, means that readers can begin to see this knowledge as extensive enough to undergird networks of escape and autonomy versus fragmentary and scattered.\footnote{For a full discussion of this theme, see Julie Chun Kim's introduction to \quotation{Movement,} {\em Digital Grainger}, \useURL[url21][https://digital-grainger.github.io/grainger/excerpts/movement.html]\from[url21].} These moments are built into the space of the plantation and Grainger's georgic poem, allowing us to rethink how we imagine the projects of both.

Our decision to focus on these moments from the poem was inspired in large part by Nicole Aljoe's work to identify the \quotation{embedded narratives} of and by the enslaved in colonial historical texts. Notably, while Aljoe's analysis of such narratives serves as the topic of her book, {\em Creole Testimonies: Slave Narratives of the British West Indies, 1709--1838}, this strategy has been mobilized by the {\em Early Caribbean Digital Archive} as well, which identifies the speakers and subjects of these embedded narratives as creators equivalent to those who wrote and published the texts (Aljoe is one of the primary creators of the {\em ECDA}). In other words, the {\em ECDA} is also interested in finding methods for distributing agency and authority, and the creators of the {\em ECDA} similarly seek to \quotation{disrupt, review, question, and revise} colonial knowledge.\footnote{See \quotation{Decolonizing the Archive: Remix and Reassembly,} {\em Early Caribbean Digital Archive}, \useURL[url22][https://ecda.northeastern.edu/home/about/decolonizing-the-archive/]\from[url22]. See also Nicole Aljoe, {\em Creole Testimonies: Slave Narratives from the British West Indies, 1709--1838} (New York: Palgrave Macmillan, 2012).} More broadly, both Aljoe's scholarship at the level of individual texts and the {\em ECDA}'s intervention at the level of the archive echo Saidiya Hartman's call for scholars to craft \quotation{counter-histories of slavery} or counter-narratives seeking to illuminate the experiences of the enslaved by disrupting the colonial archive.\footnote{Saidiya Hartman, “Venus in Two Acts, {\em Small Axe}, no. 26 (June 2008): 4.}

In contrast to the embedded narratives highlighted by the {\em ECDA}, the passages excerpted by {\em Digital Grainger} are not openly attributed to or associated with particular enslaved or marginalized individuals and lack any markers of quotation, interaction, or speech. Yet the gestures Grainger makes to alternative forms of survival and knowledge allow us to identify other sources of expertise and resistance, even as they remain unnamed, and thereby to take another approach to the construction of counter-narratives and counter-histories. Especially because so much work has been done by historians and other scholars of colonial science and agriculture, we know that Grainger would have relied on conversation and exchange---although often coerced---with Afro-Caribbean and other marginalized individuals for his information about plants, and one of our goals is highlighting the fact of these anonymous interactions, even if we cannot identify the exact individuals with whom Grainger spoke.\footnote{For discussions of knowledge exchanges between Euro-American naturalists and Afro-Caribbean and indigenous informants, see Susan Scott Parrish, {\em American Curiosity: Cultures of Natural History in the Colonial British Atlantic World} (Chapel Hill: University of North Carolina Press, 2006); Londa L. Schiebinger, {\em Plants and Empire: Colonial Bioprospecting in the Atlantic World} (Cambridge, MA: Harvard University Press, 2004); James Delbourgo, {\em Collecting the World: Hans Sloane and the Origins of the British Museum} (Cambridge, MA: Harvard University Press, 2017); and Miles Ogborn, \quotation{Talking Plants: Botany and Speech in Eighteenth-Century Jamaica,} {\em History of Science} 51, no. 172 (2013): 251--82.}

Indeed, the long history of Caribbean exchange, encounter, and colonization is a central theme in \quotation{The Counter-Plantation} section and its excerpts. In particular, the excerpt \quotation{Indigenous Presence} reckons with the long traditions of indigenous knowledge that {\em The Sugar-Cane} draws on. For instance, Grainger's inclusion of indigenous words such as {\em Liamuiga} and {\em suirsaak} indicates a connection to the natural knowledge of Kalinago Caribs or Arawaks. Whether or not these words were directly spoken to Grainger by indigenous persons or presented as inherited knowledge by Afro-Caribbean and other individuals, these gestures denote genealogies of expertise that manifest themselves in {\em The Sugar-Cane} despite Grainger's attempts to establish and stabilize his own narrative authority. In fact, tracing such references highlights how he relegates most mentions of indigenous communities and knowledge to the footnotes. By collecting these footnoted moments, \quotation{Indigenous Presence} brings these peripheral annotations to the foreground to demonstrate the depth of Amerindian botanical and ecological knowledge. With this strategy, we thus expand the ways scholars can recognize authorship and authority, even when explicit citations and quotations of named contributors are absent.\footnote{For further discussion, see the headnote for \quotation{Indigenous Presence,} {\em Digital Grainger}, \useURL[url23][https://digital-grainger.github.io/grainger/excerpts/indigenous-presence.html]\from[url23].}

Nevertheless, this anonymity is itself a possible limitation of our project. As Acosta and Casimir explain, the counter-plantation was a \quotation{system in opposition to, but dependent upon, the plantation.}\footnote{Acosta and Casimir, \quotation{Social Origins of the Counter-Plantation System in St Lucia,} 35.} While the excerpts in \quotation{The Counter-Plantation} section demonstrate the possibilities of life and agency that oppose the order of the plantation while existing within it, our recognition of these instances continues to remain dependent on the plantation system and the literature it produced. In her theorization of counter-histories, Hartman grapples with the simultaneous \quotation{necessity} and \quotation{inevitable failure} to represent \quotation{the lives of the subaltern, the dispossessed, and the enslaved.} This failure arises from the contours of the colonial archive, which is fundamentally shaped by histories of violence against marginalized subjects.\footnote{Hartman, \quotation{Venus in Two Acts,} 12.} As a result, our analysis at times must rely on general discussions of Afro-Caribbean and indigenous resistance and agency across the West Indies rather than St.~Kitts specifically, or to merely note the absences and gaps in Grainger's narration that open possibilities for other interpretations. Indeed, many of our choices, including our decision to produce an edition that challenges {\em The Sugar-Cane}, harken back to what Édouard Glissant terms \quotation{counterpoetics,} which always establishes itself in opposition to, and thus in relation to, the order of colonization. Because we must constantly refer back to a colonial text in order to recognize alternate networks of agency and resistance, {\em The Sugar-Cane} maintains a central place in our analysis, even as we seek to decentralize Grainger's authority, enacting what Glissant notes is a \quotation{forced poetics} that struggles to break free of the constraints of colonial discourse.\footnote{Édouard Glissant, {\em Caribbean Discourse: Selected Essays}, trans. J. Michael Dash (Charlottesville: University Press of Virginia, 1989), 120--21.}

This limitation, however, can be viewed as productive, for it reminds us that the digital does not resolve the structural and historical problems that we have inherited in the form of the colonial archive. As Leah Rosenberg prompts us to recognize, \quotation{The digital age places significant responsibility on scholars to redress the marginalisation of Caribbean literature and to ensure its future.}\footnote{Leah Rosenberg, \quotation{Refashioning Caribbean Literary Pedagogy in the Digital Age,} {\em Caribbean Quarterly} 62, nos. 3--4 (2016): 423.} These strategies of redress can and should take multiple forms, and the limits of {\em Digital Grainger} provide an opportunity to reflect on the numerous possibilities a turn to the digital can authorize, particularly in regard to a counter-edition. Because {\em Digital Grainger} both maintains an accessible version of an eighteenth-century text and contextualizes and decenters Grainger's authority through local networks of exchange, expertise, and resistance, its approach to the epistemological creativity of the counter-plantation assumes a form defined by breaking the words and voice of the poem apart. The creativity inherent to the counter-plantation, though, provides deep ground for further development of alternatives. As with our own counter-edition, these alternatives can focus thematically---on the sonic or geographic experience of St.~Kitts and Afro-Caribbean life---or through experiential designs, including spaces for underrepresented communities to contribute or develop their own narratives or paths of movement across the landscape.\footnote{For an example of a printed analysis of a focused analysis of a specific plant, see J. S. Handler, \quotation{The History of Arrowroot and the Origin of Peasantries in the British West Indies,} {\em Journal of Caribbean History}, no. 2 (May 1971): 46--93. As Marlene L. Daut argues, decolonial forms of digital archiving can also examine its means of production, either presenting work \quotation{in medias res} or through collaboration. See Daut, \quotation{Haiti @ the Digital Crossroads: Archiving Black Sovereignty,} {\em sx archipelagos} 3 (July 2019), \useURL[url24][http://archipelagosjournal.org/issue03/daut.html]\from[url24].} Ultimately, as further editions and projects are built, they can in turn reference and respond to one another, building conversations and interpretations that further deauthorize colonial works like {\em The Sugar-Cane}.

Most of all, as we strive to meet this responsibility, we must resist the urge to assume that we can speak for those silenced in the archive or present narrative forms from the colonial period completely independent of the plantation system and the literatures that it produced. Decolonization requires more than breaking apart colonial narratives. While {\em Digital Grainger} provides readers with strategies to nuance, complicate, and resist Grainger's authority, it and similar projects must continually confront and actively work against the possibility that digitizing a colonial text simply replicates long-standing forms of silence, bringing the same careful reading and analysis to our own work as we do to works of the past.

\subsection[title={Acknowledgments},reference={acknowledgments}]

I would like to thank the anonymous readers for their generous and constructive feedback. Thank you as well to Julie Kim for her thoughtful and insightful reading of earlier forms of this essay. Lastly, thanks to the {\em Digital Grainger} team: Julie Kim, Cristobal Silva, Alex Gil, Ami Yoon, Steve Fragano, Lina Jiang, and Elizabeth Cornell.

\thinrule

\page
\subsection{Kimberly Takahata}

Kimberly Takahata is a PhD candidate at Columbia University in the Department of English and Comparative Literature, where she specializes in early American literature. She is a team member of {\em Digital Grainger}.

\stopchapter
\stoptext