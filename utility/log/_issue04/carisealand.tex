\setvariables[article][shortauthor={Esprit}, date={February, 2020}, issue={4}, DOI={Upcoming}]

\setupinteraction[title={Carisealand: Sustainable Caribbean Futures},author={Schuyler Esprit}, date={February, 2020}, subtitle={Carisealand}]
\environment env_journal


\starttext


\startchapter[title={Carisealand: Sustainable Caribbean Futures}
, marking={Carisealand}
, bookmark={Carisealand: Sustainable Caribbean Futures}]


\startlines
{\bf
Schuyler Esprit
}
\stoplines


\subsection[reference={archipelagos-presents-carisealand},
bookmark={*archipelagos* Presents *Carisealand*},
title={*archipelagos* Presents *Carisealand*}]

The {\em Carisealand} (Caribbean Sea and Land) digital research project, launched in 2015 by scholar, teacher, and activist Schuyler K Esprit, provides an interactive framework within which to imagine a more sustainable way of life for Caribbean people, both in the present day and in the future. The project aims to create awareness around the impact of climate change in the Caribbean region; to initiate and sustain collaborations with scholars and artists in an effort to find solutions to social injustices resulting from climate change activity; and to present models for alternate Caribbean futures. The project is interdisciplinary; it is undergirded by thoughtful humanistic considerations of and scientific, anthropological, and sociological inquiries into the ways and extent to which the diverse linguistic, ethnic, social, and other spaces of the Caribbean endure the \quotation{natural} effects of empire. Through sustained engagement with an ever-widening community of regional cultural and political actors, {\em Carisealand} offers a platform on which to showcase visions of survival through the perils of a changing natural environment, including recent massive hurricanes.

\subsection[reference={archipelagos-review},
bookmark={*archipelagos* review},
title={*archipelagos* review}]

{\em Carisealand} is an ambitious exploration of an alternative, collective response to climate change. Set against the backdrop of the Caribbean's increasing geographic and sociopolitical vulnerability, the work presented in this project is both timely and urgent. However, there is evidence that the project is still in its very early stages of conceptual, methodological, and material design; as a result, there are a number of areas that can be addressed.

While the authors want to engage with the issue of climate change as it affects the Caribbean region, at this point, the majority of the project focuses primarily on Dominica (with the exception of the Map artifact). While this initial focus is not problematic, the focus should be explained. Additionally, how the authors anticipate and plan for the systematic coverage of the wider Eastern Caribbean region is not entirely clear. While the key questions of the project (on the home page) indicate a specific focus on the Eastern Caribbean (EC), this focus isn't clearly articulated in the project description. Additionally, resources from other islands not traditionally identified as part of the EC---for example, Guyana and Trinidad and Tobago---are showcased. If the project seeks to focus on the EC, are there a peculiar set of vulnerabilities that make this part of the region a necessary focus for the authors? Aside from these general concerns, I recommend that the authors consider the following revisions.

\subsubsection[contribution]{Contribution}

The target audience should be made clearer in the project overview on the home page. For example, who do the authors expect to come to this platform and why? What type of engagement do they expect and solicit from this audience? Answers to these questions must not only be made explicit in some written description but also should be made implicit in the design (site nomenclature, for example) of the platform as well.

Perhaps my biggest concern at this stage involves the methodological rigor of the project. For example, how are projects chosen? What is the rationale for choosing the nine project topics (e.g., agriculture, water, and food security)? How do these topics explicitly connect to the overarching concern of climate change and to the other project topics listed? What theories inform these decisions and artifacts? Which Caribbean theorists are being invited into conversation through this project? What projects currently exist that do this work? What is the unique contribution that this digital project makes in this space? Aside from the insightful list of questions provided with each of the subtopics, the authors should consider adding a reading list that engages with the topics' issues (I am thinking, for example, of work by Tanya Chung Tiam Fook and Christopher Corbin).\footnote{Tanya Chung Tiam Fook, \quotation{A \quote{Win-Win} Strategy for All? Guyana's Climate Change Strategies and Implications for Indigenous Communities,}{\em Caribbean Journal of International Relations and Diplomacy} 1, no. 1 (2013): 3--38; Christopher Corbin, \quotation{\quote{There Is No Green without Blue}: An Analysis of the Importance of Coastal and Marine Resources to the Development of Green Economies by Caribbean SIDS,} {\em Caribbean Journal of International Relations and Diplomacy} 1, no. 3 (2013): 47--59.}

\subsubsection[credit]{Credit}

While work on certain artifacts indicate authorship, often one must view the artifact (e.g., video) to access this information. The project needs to develop some style guide for acknowledging and referencing sources for the artifacts. Bios of contributors would be useful and would provide a welcomed transparency to the project.

\subsubsection[design]{Design}

Subsection project pages need some written overview. The questions that currently appear, while useful, do not provide the reader with an orientation to the content; for example, the Map initially provides a description of the resources on environmental conservation and sustainability in Dominica. However, it is not obvious until the user begins scrolling that resources on other islands are also described. Therefore, some overview to the Map feature needs to be provided. What can the reader do here? How is this information relevant to the project's overarching purpose?

Because the project is still in its early stage of development, accessibility should be addressed now. For example, are there plans to provide transcripts of and closed captioning for audio/video recordings? Minimally, ALT tags (alternative text) must be included for all images.

The usability of the site needs to be addressed, particularly concerning navigation, to ensure that the user knows where they are in the site's architecture and how to move around the site. For example, when a link takes the reader off the site to an external source, this should be indicated. Additionally, on some pages, when the user is in the lower levels of the site, it is not always clear how to return to the home page (see Map content, for example). Additionally, I encourage a rethinking of the information architecture/nomenclature of the site's menu, since some labels (The Lab, Network, for example) are not intuitive.

If the authors have not already discussed this, there should be some user guidelines/agreement for this work. How is the project engaging with the issue of intellectual property? How do the authors want artifacts cited? What can be downloaded? How is fair use defined? Finally, about the authors should consider audience's citation needs and ensure that all information is available for complete citations (e.g., dates of page updates, authors' names, proper titles of articles and pages, etc.).

\subsubsection[preservation]{Preservation}

I am uncertain if preservation of the site's content has been addressed. Practically, how will existing content be maintained, archived, or updated? Is there consideration of a mark-up language to support the reconstitution of content? Is Wordpress as a content management system a feasible long-term option for the administration of the project's needs?

\subsection[reference={response-from-schuyler-esprit-project-lead-of-carisealand},
bookmark={Response from Schuyler Esprit, project lead of *Carisealand*},
title={Response from Schuyler Esprit, project lead of *Carisealand*}]

In 2015, Caribbean writer and environmental activist Oonya Kempadoo approached the \useURL[url1][http://createcaribbean.org/create/][][Create Caribbean Research Institute]\from[url1] collective with a number of ideas about how to collaborate in generating more conversation around environmentalism and how the arts and humanities might influence the visibility and change required in a Caribbean space faced with the realities of a vulnerable planet.\footnote{The Create Caribbean Research Institute, located at Dominica State College, is the first digital humanities center in the Caribbean. Its goal is to bridge research, technology, and community outreach with digital scholarship as a way to help students imagine and build a better Caribbean. Create Caribbean's mission is to use digital humanities for investigating and sharing stories about Caribbean culture, to bridge the digital divide in the region through technology education programs, and to use its digital resources to support community efforts, particularly through environmental activism and literacy advocacy. See \useURL[url2][http://createcaribbean.org/create/]\from[url2].} A few weeks after this initial conversation, Tropical Storm Erika hit Dominica, bringing more than twenty inches of rain in fewer than twelve hours, causing island-wide flooding and leading to the deaths of more than thirty people. The urgency was clear. Our conversations about the lack of a cohesive regional message on environmental activism informed the digital research project Kempadoo named Carisealand. Emphasizing the Caribbean archipelago's geographical, cultural, and linguistic connections through land and sea, the original goal of the project was to create a shared resource for Caribbean scholars and activists concerned about environmental sustainability.

{\em Carisealand} began with one question: What can we do to compile and organize as many resources as possible on environmentalist action and climate activism in the Caribbean region? We imagined a database that would reach from Florida in the north to Venezuela in the south, inclusive of Central American territories. This geographical expansiveness was designed to place climate-change workers across the region in conversation with one another---a starting point for collaboration and dialogue. Over time, the project expanded to include both intellectual inquiry and narrative representation of the histories and possibilities of the Caribbean's struggle with modernity through the discourses of nature and environment.

At its core, the {\em Carisealand} project has sought to bring together scholars, artists, and activists to think through a number of challenging questions. {\em Carisealand} is, above all, a pedagogical space within which undergraduate students in both digital humanities and Caribbean studies might explore issues at the intersection of research, technology, and civic engagement, as outlined in \useURL[url3][http://createcaribbean.org/create/the-create-mission/][][the mission of Create Caribbean]\from[url3]. Students working on {\em Carisealand} are interns in the \useURL[url4][http://createcaribbean.org/create/internship/][][Research and Service Learning Internship Program of Create Caribbean]\from[url4], and they are required to take courses in \useURL[url5][http://createcaribbean.org/his115/][][digital humanities research]\from[url5] during their first semester in the program and to enroll in a one-credit-per-semester Research Colloquium during their tenure at Dominica State College. Students from the past three cohorts (2016 through 2018) have collaborated on {\em Carisealand} as their graded course requirement. Content under \useURL[url6][http://carisealand.org/acf-docuseries/][][ACF Series]\from[url6], \useURL[url7][http://carisealand.org/class-projects/][][Class Projects]\from[url7], \useURL[url8][http://createcaribbean.org/create/carisealand/][][Blog]\from[url8], and Map all reflects work produced for students' digital humanities research course, continued during internship hours, and supervised under their research colloquium enrollment.\footnote{For Create Caribbean's mission, see \quotation{The Create Mission,}\useURL[url9][http://createcaribbean.org/create/the-create-mission/]\from[url9], and on the Research and Service Learning Internship Program of Create Caribbean, see \quotation{Internship,}\useURL[url10][http://createcaribbean.org/create/internship/]\from[url10]. For information on the required research course, see \quotation{Digital Humanities Research (HIS 115); Dominica State College---Instructor, Schuyler K Esprit, PhD,}\useURL[url11][http://createcaribbean.org/his115/]\from[url11].}

Like every other digital humanities project supported by Create Caribbean, the target audience for {\em Carisealand} is primarily high school and college students but extends to the wider public; it is not necessarily centered on academic spaces. For example, the projects published by Create Caribbean are frequently used as teaching aids in classrooms in Dominica and used as research resources for patrons in the national public library service. As such, the content on the site avoids academic jargon, heavy theoretical explication, or deep engagement with theory in the public interface. That being said, students of Create Caribbean engage in critical readings of Sylvia Wynter, Kamau Brathwaite, Emily Raboteau, Elizabeth DeLoughrey, and Elizabeth Thomas-Hope, among others, as they create their projects. Additionally, their course calls on literary influences such as Wilson Harris, Derek Walcott, Octavia Butler, and Nalo Hopkinson---creative writers who imagine alternative futures for black people amid environmental strife and injustice. The content produced and published on the site thus far reflects students' interpretations and responses to their course readings.

Expansion of the project goals, in particular the decision to focus more deliberately on Eastern Caribbean concerns, came in the aftermath of Hurricanes Irma and María, which decimated a number of smaller nations in September 2017. The trauma and uncertainty of surviving a category-5 hurricane resulted in a sense of responsibility to move the conversations and actions forward to document the humanistic questions and answers that emerge from climate disaster. At that time, the project authors worked with students of Create Caribbean to reflect on the broad areas they viewed as most relevant to the conversation about how climate activity affected their day-to-day lives. The project topics presented on the site were proposed by students in the 2017 cohort as the most impactful areas of public discourse following that year's hurricane season.

In support of the intention to direct content to students, the authors plan to include one of the theoretical exercises of the digital humanities research course in the final layout of the site---the Carisealand Syllabus, which includes a bibliography of all the work informing the project design and the resources cited within projects and used in the classroom, as well as resources curated by students during their individual research exercises.

Navigation has been the greatest challenge to the public-facing platform of {\em Carisealand}. With so much content that is so extensive across Caribbean and climate studies, the project architects have struggled with refining the Home page to reflect the full depth and breadth of the project. To respond to some of these concerns, we will redesign the interface in three main categories: \useURL[url12][http://carisealand.org/mapping/neatline/fullscreen/carismapping][][the Map]\from[url12], the Carisealand Syllabus, and \useURL[url13][http://carisealand.org/the-lab/][][the Lab]\from[url13]. The Map will feature a completed visualization of the original database. We intend to migrate the map feature to ArcGIS instead of Omeka to accommodate some logistic and design features that would be difficult in its current format. The Carisleand Syllabus space will capture most of the primary and secondary content featured on the site---class projects, a formal bibliography, project topic content, and the Alternative Caribbean Futures Video Series. The Lab, which is the most incomplete part of the site to date, is a digital humanities experimental space where students will present their finished 3-D modeling project of a redesigned sustainable Caribbean community, representing in the first instance the village of Mahaut, Dominica. What currently appears on the page labeled \quotation{The Lab} is a visualization of the historical context of the community, organized according to the students' selected project topics.

The intellectual and pedagogical approach to this project and all others at Create Caribbean is an exercise in narrative. Storytelling, which includes listening to our elders and ancestors in the ananse tradition of \quotation{krik! krak!} call-and-response, is our humanities praxis. Listening to and recounting the stories of others, and then telling our own stories, shifts the paradigm from defeated victims of natural and manmade disasters to defenders of our right to exist no matter how small or how far away.

In our previous digital research projects, primarily focused on documenting Dominican and Caribbean history and literature, the intellectual and personal motive has been to resist and reject the way the physical space of Dominica and the Caribbean have been defined as too small, too poor, or too underdeveloped for intellectual and economic progress. In this case, with {\em Carisealand}, and with the Lab more particularly, much more is at stake: survival and viability. This is a call to resist the concept of resilience, of acceptance and response to defeat, a call to instead choose to fight with everything we have in the here and now to save our homes, our friends, our own lives.

The vision for {\em Carisealand} has transformed and expanded since its launch in 2015. Through the three pillars of the project---the resource map, the Carisealand Syllabus, and the Lab---{\em Carisealand} aims to reflect constantly on one idea: What does it mean to be a courageous Caribbean citizen in the age of climate change? Part of that courage requires telling and retelling stories of the history and future of the Caribbean as we see it, despite the inevitability of environmental and sociopolitical upheaval.

A significant challenge for the site has been to engage the scholars working on the project in guiding first- and second-year undergraduate students toward the critical thinking-through of some of the more complex theoretical questions and methodologies required for rigorous inquiry and digital humanities project design. Through various collaborations, the authors have identified faculty and graduate student mentors in the Caribbean and North America to work with students on the development of their individual contributions to the site. The project has since received some of this support through a collaboration with Duke University's \useURL[url14][https://fsp.trinity.duke.edu/about-forum-scholars-and-publics][][Forum for Scholars and Publics]\from[url14], and the authors now estimate a publication date of June 2020 for a revised {\em Carisealand}.\footnote{On Duke University's forum, see \quotation{About Forum for Scholars and Publics,}\useURL[url15][https://fsp.trinity.duke.edu/about-forum-scholars-and-publics]\from[url15].} Some of the more obvious issues with a credits page, streamlined navigation, and a site map are underway for a formal relaunch of the completed version; the final version of the Lab will also be launched on a linked platform.

\thinrule

\page
\subsection{Schuyler Esprit}

Schuyler Esprit is a scholar of Caribbean literary and cultural studies and digital humanities. Her current research explores the intersections of environmental humanities, futurism, and radical Caribbean literary and political discourse. She is the founding director of the Create Caribbean Research Institute, the first digital humanities center in the Caribbean, based in Dominica. She is currently a program officer at the University of the West Indies, Open Campus.

\stopchapter
\stoptext