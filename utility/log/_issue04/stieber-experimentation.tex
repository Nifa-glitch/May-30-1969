\setvariables[article][shortauthor={Stieber, Wrubel, Denis}, date={January, 2020}, issue={4}, DOI={https://doi.org/10.7916/archipelagos-vdts-xg77}]

\setupinteraction[title={Cross-Boundary Digital Collaboration as Scholarly and Institutional Experimentation: Amplifying the Impact of Caribbean Periodicals},author={Chelsea Stieber, Laura Wrubel, Watson Denis}, date={January, 2020}, subtitle={Cross-Boundary Digital Collaboration}, state=start, color=black, style=\tf]
\environment env_journal


\starttext


\startchapter[title={Cross-Boundary Digital Collaboration as Scholarly and Institutional Experimentation: Amplifying the Impact of Caribbean Periodicals}
, marking={Cross-Boundary Digital Collaboration}
, bookmark={Cross-Boundary Digital Collaboration as Scholarly and Institutional Experimentation: Amplifying the Impact of Caribbean Periodicals}]


\startlines
{\bf
Chelsea Stieber
Laura Wrubel
Watson Denis
}
\stoplines


{\startnarrower\it This essay analyzes the role that experimental work played in the multi-institutional, cross-boundary collaborative construction of the RSHHGG Lab, an interactive online index of Haiti's preeminent social science journal, the {\em Revue de la Société haïtienne d'histoire, de géographie et de géologie}. Here, the authors understand experimentation to be a mindset, an attitude of \quotation{try it and see what happens} through iterative small risks and an openness to involving others and changing direction. This approach draws on the spirit of collaboration, experimentation, and sharing that informs the LC Labs initiative at the Library of Congress, and the {\em revue} (magazine-journal) as a print space of exchange, confrontation, and collective creation. Following recent reflections on information maintenance and care ethics, the authors agree that infrastructures of information maintenance are often invisible and institutionally siloed and tend to reproduce structures of oppression and silencing. They aim here to highlight the institutional and infrastructural experimentation and labor necessary to create what is, essentially, a project of information management: making a key repository of knowledge usable, discoverable and accessible. The authors contend that experimental work was central to successfully working within---and also challenging---extant infrastructures of information maintenance in order to achieve an independent, sustainable project with room to grow and transform. In attending to the various processes of experimentation that underwrote the RSHHGG Lab, the authors hope to demonstrate the fruitfulness of cross-boundary collaborative projects and, ultimately, to advocate for more infrastructures of experimentation in many forms: research leave, labs, residencies, and further creative approaches to blurring the boundaries of labor.

 \stopnarrower}

\blank[2*line]
\blackrule[width=\textwidth,height=.01pt]
\blank[2*line]

\startblockquote
The fact that outsiders were not interested in what was happening in Haiti does not permit us to conclude that nothing had happened there.
\stopblockquote 

\startalignment[flushright]
\tfx{Michel-Philippe Lerebours, {\em Haïti et ses peintres}}
\stopalignment
\blank[2*line]


Over the last ninety years, Haitian scholars and intellectuals have published more than 260 issues and more than 1,300 articles in the pages of the {\em Revue de la Société haïtienne d'histoire, de géographie et de géologie} ({\em RSHHGG}). The publication is the greatest repository of historical research produced on Haiti, from Haiti. We contend that Haitian scholarly periodicals, and Caribbean periodicals more broadly, are key sites of knowledge production that engage in a crucial critique of North Atlantic episteme and infrastructures of information. Yet the {\em RSHHGG} and other Caribbean periodicals like it are rarely cited by scholars outside Haiti. Indeed, we argue that it is precisely {\em because} Caribbean sites of knowledge production exist outside of---and in critique of---North Atlantic infrastructures of information that it becomes acceptable, common even, for scholarship produced outside Haiti to leave them uncited and unread.\footnote{This is not to say that the {\em RSHHGG} and its scholarly production have not traveled outside the country; specialists of Haitian studies, such as David Geggus, Carolyn Fick, Laurent Dubois, Sibylle Fischer, Jacques Cauna, and Matthew Smith, have long recognized the importance of the resource and used it in their own research. They utilized the journal's scholarly production despite its relative scholarly \quotation{obscurity}---precisely because they had a specialist's knowledge of its importance. Epigraph: \quotation{De plus, le fait que l'étranger ne se fût pas intéressé à ce qui se passait en Haïti, n'autorise pas à conclure qu'il ne s'y fut rien passé}; Michel-Philippe Lerebours, {\em Haïti et ses peintres: De 1804 à 1980; souffrances and espoirs d'un peuple}, 2 vols. (Port-au-Prince, Haïti: Imprimeur II, 1989), 1:24.} Such a blind spot is even more remarkable today, as North Atlantic scholars express ever more interest in Haiti's role in the Age of Revolution and studies of the Atlantic world. Put otherwise: If the Haitian Revolution is central to the study of the Atlantic world, why not Haitian histories and Haitian historians?\footnote{On the problem of centering in comparative studies, see Carol Boyce Davies, \quotation{Beyond Uni-centricity: Transcultural Black Presences,} {\em Research in African Literatures} 30, no. 2 (1999): 96--109; Micol Siegel, \quotation{Beyond Compare: Comparative Method after the Transnational Turn,} {\em Radical History Review}, no. 91 (2005): 62--90; Marlene Daut, \quotation{Daring to Be Free / Dying to Be Free: Toward a Dialogic Haitian--US Studies,} {\em American Quarterly} 63, no. 2 (2011): 375--89; and Chelsea Stieber, \quotation{Beyond Mentions: New Approaches to Comparative Studies of Haiti,} {\em Early American Literature} 53, no. 3 (2018): 961--75.}

The gap between the vital importance of Haiti's premier social science journal and its relative lack of impact in North Atlantic scholarship led to the creation of the \useURL[url3][http://rshhgglab.com/][][RSHHGG Lab]\from[url3], an interactive online index of the nearly century's worth of knowledge produced in the {\em RSHHGG}.\footnote{See RSHHGG Lab, \useURL[url4][https://rshhgglab.com/]\from[url4]. The site is also available in French (\useURL[url5][https://rshhgglab.com/francais/]\from[url5]) and Kreyol (\useURL[url6][https://rshhgglab.com/kreyol/]\from[url6]).} Developed at the intersection of disciplinary fields and in cooperation with various institutional partners, including the Société haïtienne d'histoire, de géographie et de géologie (SHHGG), the Library of Congress's Labs and John W. Kluge Center, the Catholic University of America, and George Washington University, this cross-boundary collaborative digital project was designed to expand the impact of this crucial repository of Haitian social science research. The project developed along the same lines of what Marlene Daut has recently theorized as \quotation{a Haitian-informed digital praxis---centered on access, content, context, and collaboration.}\footnote{Marlene L. Daut, \quotation{Haiti @ the Digital Crossroads: Archiving Black Sovereignty,} {\em sx archipelagos}, no 3 (2019), \useURL[url7][https://smallaxe.net/sxarchipelagos/issue03/daut.html]\from[url7].} It is an open-access, multilingual website (available in \useURL[url8][http://rshhgglab.com/][][English]\from[url8], \useURL[url9][http://rshhgglab.com/francais/][][French]\from[url9], and \useURL[url10][http://rshhgglab.com/kreyol/][][Kreyol]\from[url10]) that features a \useURL[url11][http://rshhgglab.com/search/][][searchable index]\from[url11] of every article published in the {\em RSHHGG}, a growing collection of \useURL[url12][http://rshhgglab.com/annotations/][][article annotations]\from[url12] from Haitian and non-Haitian scholars and researchers, and a space for \useURL[url13][http://rshhgglab.com/experiments/][][experimentation]\from[url13].

This essay analyzes the role that experimental work played in the multi-institutional, cross-boundary collaborative construction of this DH project. Here, we understand experimentation to be a mindset, an attitude of \quotation{try it and see what happens} through iterative small risks and an openness to involving others and changing direction. This approach draws on the spirit of collaboration, experimentation, and sharing that informs the \useURL[url14][https://labs.loc.gov/][][LC Labs]\from[url14] initiative at the Library of Congress, and the {\em revue} (magazine-journal) as a print space of exchange, confrontation, and collective creation.\footnote{On the LC Labs project, see \useURL[url15][https://labs.loc.gov/]\from[url15]. On the {\em revue} as experimentation, see Olivier Corpet, \quotation{Que vivent des revues,} {\em Bulletin des bibliothèques de France} 33, no. 4 (1988): 282--90.} Following recent reflections on information maintenance and care ethics, we agree that infrastructures of information maintenance are often invisible and institutionally siloed and often reproduce structures of oppression and silencing.\footnote{The Information Maintainers: Devon Olson, Jessica Meyerson, Mark Parsons, Juliana Castro, Monique Lassere, Dawn Wright, Amelia Acker, et al., {\em Information Maintenance as a Practice of Care} (2019), \useURL[url16][https://doi.org/10.5281/zenodo.3236409][][doi:10.5281/zenodo.3236409]\from[url16].} We aim here to highlight the institutional and infrastructural experimentation and labor necessary to create what is, essentially, a project of information management: making a key repository of knowledge usable, discoverable, and accessible. The authors contend that experimental work was central to successfully working within---and also challenging---extant infrastructures of information maintenance in order to achieve an independent, sustainable project with room to grow and transform.

In what follows, we consider the process of knowledge-creation in the history of the SHHGG and the nearly one-hundred-year existence of its {\em Revue}, the process of indexing this original source material into database form, and transforming this database into a workable digital product. In each of these stages, experimentation and cross-boundary collaboration were key---between the humanities and library science disciplines, between expectations and realities, and between multiple languages and registers: Kreyol, French, English, and code. In attending to the various processes of experimentation that underwrote the RSHHGG Lab, the authors hope to demonstrate the fruitfulness of cross-boundary collaborative projects and, ultimately, to advocate for more infrastructures of experimentation in many forms: research leave, labs, residencies, and further creative approaches to blurring the boundaries of labor.

\subsection[title={The Haitian State, Politics, and the Evolution of the Société haïtienne d'histoire, de géographie et de géologie and its {\em Revue}},reference={the-haitian-state-politics-and-the-evolution-of-the-société-haïtienne-dhistoire-de-géographie-et-de-géologie-and-its-revue}]

In 2019, the Société haïtienne d'histoire, de géographie et de géologie celebrated its ninety-sixth anniversary.\footnote{The Société d'histoire et de géographie d'Haïti evolved with time and with the broader transformations in Haitian society. The name changed as well. In April 1947, as he stepped down as his term as president, Jean Price-Mars asked the members to consider adding \quotation{haïtienne} to the name in order to specify its country of origin. In 1950, the executive committee added \quotation{géologie.} Henceforth, it was known as the Société haïtienne d'histoire, de géographie et de géologie. While there has been relatively little published on geology since that era, it goes without saying that after the terrible earthquake on 12 January 2010, geology has reasserted itself in Haitian society as an indispensable field of inquiry.} Its survival is no small feat: Haitian society has long experienced the kind of tumult that takes down those institutions in civil society committed to critical reflection, knowledge, and scientific research. The SHHGG is considered the eldest among Haiti's scientific, educational institutions, perhaps the oldest learned society in the Caribbean region still in existence today. Its press organ, the {\em Revue de la Société haïtienne d'histoire, de géographie et de géologie}, is a standard bearer and scientific publication of reference in Haiti, given that so few scientific periodicals are currently published in Haiti, especially with regularity. In many respects, the {\em Revue} attests to the level of critical reflection, studies, and research in social sciences undertaken in Haiti, by Haitians. These background paragraphs are designed to give an idea of the evolution of the SHHGG and the {\em Revue} from 1923 to today and the importance of these institutions in the intellectual sphere and the world of scientific research in Haiti. By bearing witness to the century of knowledge produced within the pages of the {\em Revue}, we hope to shed light on the stakes of information infrastructures that do not sufficiently render this knowledge accessible.

The society was founded officially on 8 December 1923 during the political and military occupation of Haiti by the United States (1915--34).\footnote{The SHHGG has historical antecedents in Haiti, such as the Société des sciences et de géographie d'Haïti founded in Port-au-Prince in 1882.} A group of notable historians, professors, intellectuals, and politicians came together with the goal of producing historical works of scientific value on Haiti's history in order to loosen the grip of the US occupation.\footnote{Sannon later clarified that the society was formed in response to \quotation{a general ignorance of our History} (\quotation{une ignorance à peu près générale de notre histoire}) and with the goal of \quotation{guiding minds toward the study of history so that the country might one day see a writer, or several, take up the original works of {[}Thomas{]} Madiou, {[}Joseph{]} Saint-Rémy and {[}Beaubrun{]} Ardouin to update and complete them} (\quotation{diriger les esprits vers l'étude de l'histoire, afin de donner au pays une chance de voir quelque jour un ou des écrivains reprendre, pour les renouveler et les achever, les travaux de Madiou, de Saint-Rémy et de Ardouin}). Gaétan Mentor, \quotation{Esquisse historique d'une société savante: La SHHGG ou 90 années au service de l'intelligentsia haïtienne,} {\em Revue de la Société haïtienne d'histoire, de géographie et de géologie}, nos. 251--52 (2013): 47. Unless otherwise indicated, all translations are ours.} The society's first president, Horacius Pauléus Sannon, echoed these efforts in one of his early speeches, excerpted here: \quotation{In times of great crisis, all people instinctively look back to their history to find lessons of collective patriotism, new rules of conduct either to better defend their threatened existence, or to lift themselves back up after they fall. . . . Each people's history is the unique source from which flows or renews, depending on the circumstances, its legitimate and indestructible national aspirations.}\footnote{\quotation{Aux heures de grandes crises, tous les peuples se sont d'instinct reportés en arrière pour chercher dans leur histoire des leçons de patriotisme collectif, de nouvelles règles de conduite, soit pour pouvoir mieux défendre leur existence menacé, soit pour se relever plus rapidement de leurs chutes. . . . L'histoire de chaque peuple est la source unique d'où découlent et ou se renouvellent, selon les circonstances, ses légitimes et indestructibles aspirations nationales}; Horacius Pauléus Sannon, \quotation{Discours de Me. H. Pauléus Sannon,} {\em Revue de la Société d'histoire et de géographie d'Haïti} 5, no. 13 (1934): 4.} To achieve these political goals, the society defined a set of concrete objectives: to create a list of the country's private archives that related to national history and geography; to establish a complete, accurate index of scholarly works dedicated to Haitian history and geography by Haitian or foreign authors; to encourage the publication of works aimed at making Haiti better known from a historical and geographical point of view; and to facilitate the diffusion of historical and geographical knowledge among the Haitian people.

The first issue of the {\em Revue} appeared on 1 May 1925 under the title {\em Bulletin de la Société} {\em d'histoire et de géographie d'Haïti} and quickly filled a gap in Haitian society.\footnote{In 1932, the {\em Revue} had 134 subscribers, including scholars and institutions abroad. The number of subscribers increased and decreased depending on the handling and quality of the publication, but the management committee has nevertheless always remained committed to publishing the {\em Revue}.} Because of a lack of funds, the second issue was not published until December 1926. In this second issue, the board of directors replaced the name {\em bulletin} with {\em revue}, which has been maintained to the present day, and the journal name later reflected the changes to the society's name (see note 7). Though it was born out of the political circumstances of the US occupation, the SHHGG as an institution exists independently of the Haitian state and works to maintain its existence outside the government or a political regime. Put otherwise: the SHHGG is not an institution formed by the political authority of the state, but in its own sphere of work it contributes to the development of the Haitian state.\footnote{In 1930 and by presidential decree, the society was recognized for its \quotation{public utility.} As such, it was to enjoy a certain number of rights and privileges accorded by the Haitian state, though in practice these were granted only irregularly. Nevertheless, the SHHGG has managed to survive financially thanks to membership dues and subscriptions to the {\em Revue}, to the consideration of a few important figures, and to the sale of issues of the {\em Revue}.} The society's collaboration with government bodies (notably with the Ministry of National Education, the Ministry of Foreign Affairs, and more recently with the Ministry of Culture) can be fruitful or fraught, depending on political circumstances. The well-known saying is that the society does not get involved with government politics. For the most part, the members of the management committee have always tried to maintain the autonomy of the organization's operations, its independence as an association, and its status as a learned society in the face of power. Nevertheless, an individual member can be involved in a particular government as she sees fit. Indeed, the society has often been led by key figures who have occupied high-ranking positions in the state and the university, while several Haitian presidents were close to the SHHGG. By commitment to maintain a division between the actions of individual members and the autonomy of the organization, the society has weathered the convulsions and commotions so favored in the country's political arena. The existence of the SHHGG first and foremost!

This is not to say that political turmoil has not had an impact on the society and the {\em Revue}. On the contrary, Haiti's political transformations have had repercussions, directly or indirectly, on the evolution of the SHHGG and the {\em Revue} either in terms of financial resources or the integration of new historians, researchers, and young professionals in its activities. Today, the current 2018--19 political crisis---known as {\em peyi lòk} (country lock)---has negatively affected the production of the journal, the society's main office, the financial resources needed to publish the {\em Revue}, and the general ability to realize the SHHGG's programming.

\placefigure[here]{Data Visualization}{\externalfigure[issue04/stieber/stieber-image_1.png]}


With the benefit of data visualization from the RSHHGG Lab index, the impact of the Duvalier dictatorship on the {\em Revue}'s output is clear, though it warrants some additional contextualization here.\footnote{Information in this segment relies primarily on Mentor, \quotation{Esquisse historique d'une société savant.}} When François Duvalier came to power in 1957, Jean Price-Mars was president of the society. In declining health, Price-Mars had planned to hand over leadership of the organization to a joint group of young historians: Leslie F. Manigat (future president of the Republic), Gérard Mentor Laurent, and Hénock Trouillot. But under Duvalier's increasingly repressive regime, Manigat and Laurent left the country, one after the other---albeit for different reasons and by different means. Only the intrepid Trouillot, a faithful disciple of the social theorist Price-Mars, remained to take the reins.

Many could assume that the society and the {\em Revue} during Hénock Trouillot's tenure were simply a tool for Duvalierist propaganda. After all, Duvalier and Trouillot knew each other well: Duvalier had previously cowritten articles for the {\em Revue} with Lorimer Denis and had even participated in some of the society's activities.\footnote{See François Duvalier and Lorimer Denis, \quotation{Les civilisations de l'Afrique noire et le problème haïtien,} {\em Revue de la Société d'histoire et de géographie d'Haïti} 7, nos. 21--22 (1936): 19--41; François Duvalier and Lorimer Denis, \quotation{La civilisation haïtienne: Notre mentalité est-elle africaine ou gallo-latin,} {\em Revue de la Société d'histoire et de géographie d'Haïti} 7, no. 23 (1936): 1--31.} Nevertheless, Trouillot chose not to align the SHHGG behind Duvalier's regime and did not turn the {\em Revue} into an outlet for the political discourse of Duvalierism. As such, Trouillot assumed sole responsibility for the society and for the {\em Revue} for eighteen years at a time when many intellectuals had left Haiti under the dictatorship. During the 1960s and 1970s Trouillot did his utmost, sometimes at his own expense and through his sole intellectual labor, to publish the {\em Revue}. In comparison to other decades of the {\em Revue}, his was a modest output but one that allowed the society and its project to survive.

Without institutional or managerial support, the weight of these professional and intellectual obligations took a toll on Trouillot's health. At the initiative of a group of intellectuals, including the renowned historians Jean Fouchard and Gérard Mentor Laurent (back in Haiti after a diplomatic mission to Africa), the Ministry of Education asked Trouillot to retire as the head of the society. Mentor Laurent presided over a new executive committee, which had been appointed to restructure the society. The 1980s marked a new era for the society and the {\em Revue}. Under the leadership of Alain Turnier, the society created its history prize and welcomed the first woman into the organization's leadership, the specialist of library science Françoise Thybulle, who was then director of the Bibliothèque nationale d'Haïti.\footnote{The society's history prize was inaugurated in 1982. The first prize was awarded in 1984 to Marcel and Claude B. Auguste for their work {\em L'expédition Leclerc, 1801--1802} and the second prize to Emmanuel Justin Castra for his work entitled, {\em Aux origines de la presse haïtienne, 1764--1850}. Between 1986 and 1994, Marc Péan, Suzy Castor, Michel Hector, Gusti-Kara Gaillard, Charles Tardieu-Dehoux, Léon Denius Pamphile, Guy Pierre, and Claude B. Augsute (again) won the prize. In 2015, after a suspension for a variety of reasons, the history prize resumed in partnership with the Fondation Roger Gaillard. Since the award's return, historians Délide Jospeh and Jean Alix René won in 2015 and 2017, respectively.} After the fall of Jean-Claude Duvalier and the end of the Duvalier dictatorship in February 1986, a new group of scholars joined the management committee, including Michel Hector, Suzy Castor, Gusti-Klara Gaillard, and Pierre Buteau. Hector served for fifteen years as president (2000--2015), and it is with great sadness that the authors of this article acknowledge his passing on 5 July 2019.\footnote{See Watson Denis, \quotation{Haïti / Hommage au mapou Michel Hector (1932--2019): Professeur, historien et militant politique,} {\em AlterPresse}, 13 July 2019, \useURL[url17][https://www.alterpresse.org/spip.php?article24537]\from[url17].}

Today, Pierre Buteau serves as president of the society, and the management committee is comprised of Guerdy Lissade, vice-president; Watson Denis, general secretary; Ampidu Lewis Clorméus, secretary; and advisory board members Gusti- Klara Gaillard-Pourchet, Itazienne Eugène, Marc-Antoine Archin, Michel Acacia, Gladys Berrouet, and Marc Désir. Membership in the SHHGG is work of charity, goodwill, and patriotism; no one receives a salary. In the spirit of reconstruction and renewal that followed the terrible earthquake of 12 January 2010, the society has taken steps to contribute to this common project of rebuilding Haiti. Since 2013, the SHHGG has leased a space where it can organize academic activities and cultural programming (conferences, book launches, history workshops, seminars, debates, and meetings of all kinds). The {\em Revue} has followed suit. Since 2011, the management committee has adopted norms and protocols for the presentation and submission of articles and essays. These have greatly facilitated the editorial work of the {\em Revue}, most notably regarding the uniformization of articles and the quality and level of the publication.

\placefigure[here]{Cover Art}{\externalfigure[issue04/stieber/stieber-image_2.jpg]}


\subsection[title={Decolonizing Infrastructures of Information Maintenance},reference={decolonizing-infrastructures-of-information-maintenance}]

Multiple levels of experimentation and cross-boundary collaboration shaped the conception, design, and realization of the RSHHGG Lab digital index project. The project itself developed during Chelsea Stieber's research fellowship at the John W. Kluge Center at the Library of Congress for the academic year 2016--17. It is important to note, however, that she had not applied specifically for a DH residence; she was there to complete research on her book project on civil war and factionalism in postindependence Haitian writing. Nevertheless, because of the cross-boundary work encouraged within the Kluge Center and the LC Labs initiative specifically, she was given the time, space, and resources (Kluge interns, \useURL[url18][https://labs.loc.gov/][][LC Labs fellows]\from[url18], technology, and expertise) to pursue the project.\footnote{Stieber worked closely with junior fellows Samantha Herron and Luke Menzies.} It would be useful, then, to first understand the process by which the need for the index became clear.

If, as we maintain in the introduction to this essay, Caribbean periodicals are crucial sites of knowledge production, why does scholarship produced {\em outside} the Caribbean leave them uncited and unread? Or, to return to our opening gambit: If the Haitian Revolution is central to the study of the Atlantic World, why not Haitian histories and Haitian historians? Here, let us make the case more clearly regarding the silencing of twentieth-century Caribbean periodicals within North Atlantic infrastructures of information maintenance via Stieber's experience with the periodical in the Library of Congress.

Stieber's residence at the Kluge Center was focused on the Library of Congress's vast collection of locally produced Haitian scholarship: monographs, newspapers, and journals, mostly from the twentieth century. Her emphasis on locally produced print materials, especially periodicals, is part of a larger research imperative that she has called attention to in her other work.\footnote{See Chelsea Stieber, \quotation{The Haitian Literary Magazine in Francophone Postcolonial Literary and Cultural Production,} in {\em Beyond Tradition: French Cultural Studies (1800--2014)}, ed.~Anne O'Neil-Henry, Kathryn Kleppinger, and Masha Belenky (Newark: University of Delaware Press, 2017).} She had dedicated a large portion of her time to working her way through the nearly full run of the {\em RSHHGG} in the library's holdings, one of the only such collections that exists in the United States, observing research trends, identifying notable contributors, and flagging articles to read in full. She quickly realized that it would not be so easy. A catalog search of the periodical revealed only the most basic information, \quotation{History---Periodicals,} and no indication of the contents of individual issues. This periodical, one of the Caribbean's most important social science journals, was virtually opaque to normal scholarly searching practices. More onerous still, the early issues through 1950 were only available on microfilm.

\placefigure[here]{Library of Congress Catalog Entry}{\externalfigure[issue04/stieber/stieber-image_3.png]}


Stieber's initial exploratory research into these questions identified two main reasons for this lack of engagement with Haitian scholarly and intellectual production: language and accessibility. The primary language of Haitian scholarship is French and, more recently, Haitian Kreyol---barriers to access for most anglophone scholars. The question of accessibility is not only linguistic, however, nor is it unique to Haitian periodicals. Rather, it has to do with infrastructures of information maintenance: the indexing and library cataloguing practices for scholarly periodicals.

The Library of Congress's depth of reference staff and subject librarians allowed Stieber to connect quickly by phone with a reference staff in the Newspaper and Current Periodical Reading Room. She learned that when libraries process incoming periodicals, they create a catalog entry that includes the basic information about the publication: title, publisher, frequency, and a basic subject heading. While libraries will routinely update the catalog record with the most recent holdings, adding catalog notes on editors, reprints, and other information, these catalog records do not index the {\em contents} of the periodical itself: the different article titles, author names, and subject headings for each issue. Which is to say, there was nothing specific to the Caribbeanness of the {\em RSHHGG} that rendered it opaque within the library catalog. All periodicals, from the {\em American Historical Review} ({\em AHR}) to the modernist magazine {\em The Dial} are more or less illegible in the library catalog. Indeed, this is precisely how the catalog is designed to work, as the reference staff explained. While in the predigital past, researchers would use the printed volume index of the specific periodical that interested them, today, virtually all North Atlantic scholarly periodicals have searchable indexes online via JSTOR, ProQuest, or Project MUSE, for example. Crucially, these periodicals have also merged their archives onto these digital platforms: the {\em AHR} is searchable and browsable from volume 1 (1895) to present.

But it is the comparison between Caribbean scholarly periodicals and modernist magazines (rather than North Atlantic scholarly periodicals) that is, we think, a more fruitful one. Modernist magazines were long thought of as the excluded or silenced site of knowledge creation and artistic experimentation in twentieth-century infrastructures of information maintenance. Their ephemeral, countercultural, and avant-garde status meant that they did not assimilate easily into the catalog or the archive, which was often their purpose in the first place: to challenge epistemological frames, systems of value, and notions of art. But they have long since been attended to, and we would venture, never really silenced: they have been chronicled in scholarly monographs and articles, in anthologies, and, in the last twenty years, in massive, sustained digital archiving projects, such as the \useURL[url19][http://modjourn.org/][][Modernist Journals Project]\from[url19].\footnote{On the Modernist Journals Project, see \useURL[url20][https://modjourn.org]\from[url20]. See also Suzanne Churchill's Index of Modernist Magazines, \useURL[url21][https://modernistmagazines.org/]\from[url21]; and the University of Wisconsin-Madison's Little Magazines Collection, \useURL[url22][https://uwlittlemags.tumblr.com/]\from[url22]. Recently, the Institut national d'histoire de l'art embarked on a much-needed revision to the West-centric modernist magazine movement with their Seigmography of Struggle project, a global history of critical and cultural journals and are also constructing a new database of global modernist periodicals; see \useURL[url23][https://gap.inha.fr/fr/projets/art-global-et-periodiques-culturels/exposition.html]\from[url23] and https://gap.inha.fr/fr/ressources/bibliographies-et-recensements/bases-de-donnees.html.} Yet for the {\em RSHHGG}, and for most other foreign-language, non-US-based scholarly periodicals from earlier in the twentieth century, there is no such searchable index, no such online platform for their archives.

The absence of care or maintenance work for Caribbean scholarly periodicals is in no small part a result of their marginal status within the Western episteme and their distance from the dominant discourses and arbiters of North Atlantic scholarly production. While modernist magazines were avant-garde and enjoyed a provocative, outsider status to the dominant norms of late Victorian or nineteenth-century art, literature, and culture, they nevertheless made this critique from within Western institutions and infrastructures. We cannot say the same for Haitian scholarly periodicals, which produced their knowledge outside the dominant Western discourse, in their own centers of knowledge and episteme. Non-Western or non-North Atlantic scholarly periodicals are inscrutable in the catalog and, as a result, are silent---and silenced---in North Atlantic scholarship, even, perversely, scholarship that purports to be on or about these regions. We therefore believe it is absolutely necessary for North Atlantic scholars to engage Haitian-based scholarly and intellectual production when they are writing and thinking about Haiti. In short: we need new structures of information maintenance that render visible, legible, and searchable the scholarly and intellectual production that exists outside the dominant (Western) infrastructure---precisely because this work challenges the logic and episteme of these dominant frames.

The complex interplay of silence, inequity, and illegibility (re)produced between infrastructures of information maintenance and North Atlantic scholarship led Stieber to a series of provocations, elaborated in dialogue with the \useURL[url24][http://smallaxe.net/sxarchipelagos/index.html][][{\em Small Axe Archipelagos}]\from[url24] project, that informed the shape of the project to come: How can we unsilence locally produced Caribbean knowledge within North Atlantic infrastructures of information maintenance and render if more accessible and legible for a public that should be reading and citing it? In the process of doing this, how can we ensure that we are doing this work in a way that actually strengthens the infrastructures and institutions in the Caribbean that generated this work? Finally, might we be able to decolonize North Atlantic research and citational practices on and about the Caribbean by indexing and making more accessible and legible key resources in twentieth-century Caribbean scholarly and intellectual production?\footnote{On decolonizing knowledge and the archive, see Achille Mbembe, {\em Decolonizing Knowledge and the Question of the Archive} (Africa Is a Country Ebook, 2015), \useURL[url25][https://africaisacountry.atavist.com/decolonizing-knowledge-and-the-question-of-the-archive.][][https://africaisacountry.atavist.com/decolonizing-knowledge-and-the-question-of-the-archive]\from[url25]; Roopika Risam, {\em New Digital Worlds: Postcolonial Digital Humanities in Theory, Praxis, and Pedagogy} (Evanston, IL: Northwestern University Press, 2019); and Jeannette A. Bastian, Stanley H. Griffin, and John A. Aarons, {\em Decolonizing the Caribbean Record: An Archives Reader} (Sacramento, CA: Litwin, 2019).}

\placefigure[here]{Collections as Data}{\externalfigure[issue04/stieber/stieber-image_4.png]}


\subsection[title={Cross-Boundary Collaboration as Experimentation},reference={cross-boundary-collaboration-as-experimentation}]

The creation of the searchable index was possible only through multi-institutional, cross-boundary collaborative experimentation. Here, we will narrate collaborative work that was central to successfully working within---and also challenging---extant infrastructures of information maintenance in order to make scholarly production from Haiti legible to non-Haitian scholars who are writing and thinking about Haiti. First and foremost was to secure support from the SHHGG, which Stieber considered the primary stakeholder in the project. If the idea was indeed to make Haitian scholarly production accessible, it was also to create an information infrastructure that directly involved the institution, organizational structures, and people involved with the SHHGG. The society expressed their support for the indexing project but reiterated their opposition to any digitization because of valid concerns that digitization alone would not ensure North Atlantic scholars' engagement with Haiti and its institutions (what Daut rightly calls \quotation{the \quote{Columbusing} of Haitian-produced scholarship}).\footnote{Daut, \quotation{Haiti @ the Digital Crossroads.}} By engaging the SHHGG as a stakeholder in the project, Stieber connected with the organization's general secretary, Watson Denis, who became an invaluable liaison to the society and a central collaborator for the indexing project.

In a stroke of coincidence and true luck, Adam Silvia, a seasoned Caribbeanist and digital humanist, had just arrived at the Library of Congress to start a new job. A historian by training, Silvia had long believed that the {\em RSHHGG} was a precious resource that deserved wider use among scholars. He offered crucial advice in this early stage of the project, setting up a workflow to index the contents of the journal in order to make them accessible, legible, and discoverable. It is this first stage of infrastructural information maintenance that is both the most laborious and often the most invisible.\footnote{The Information Maintainers, {\em Information Maintenance as a Practice of Care}.} It would not do to give it short shrift here. Over the course of a calendar year, Silvia (off the clock, in his personal time); a Kluge Center intern, Chadwick Dunefsky; and a Catholic University of America PhD candidate and graduate research assistant, Maria Jose Gutierrez, assisted Stieber in indexing the contents of each issue of the periodical into a database that recorded author, title, issue and volume numbers, notes, and library location (1,381 items in all). The Catholic University of America Interlibrary Loan Staff provided invaluable assistance tracking down newer or missing issues. Without the care and expertise of these contributors, the project would not have been possible.

As the indexing phase neared completion, the LC Labs and visiting librarian Laura Wrubel provided Stieber invaluable guidance and consultation on how to turn this dataset into an index on a digital platform. The \quotation{guiding principles} of accessibility and sustainability that Stieber brought to this next stage of the project were essential to narrow down all the possibilities and stick to the guiding ethos of the project. The accessibility piece was inspired by the Small Axe Archipelagos project, which emphasizes critical thinking in digital humanities scholarship in order to \quotation{encourage collaboration with, increase accessibility for, and otherwise work to narrow the gap between Caribbeanist researchers, especially those in the North Atlantic academy, and the communities we are committed to serving.}\footnote{\quotation{About Us,} {\em sx archipelagos}, \useURL[url26][http://smallaxe.net/sxarchipelagos/about.html]\from[url26].} The sustainability piece came from the principles of minimal computing utilized by Alex Gil and others, which advocates finding a balance between \quotation{high-performance} technology and a sustainable, data \quotation{light} site (ideal for Caribbean users who access on their mobile phones using 3G data) that is easy to maintain in the long-term.\footnote{See \quotation{About,} Minimal Computing: A Working Group of GO::DH, \useURL[url27][https://go-dh.github.io/mincomp/about/]\from[url27].}

Wrubel, a librarian and software developer at George Washington University, was at that time commencing three months of research leave, pursuing professional development goals related to digital scholarship and computational analysis of library collections.\footnote{Meghan Ferriter, \quotation{Welcoming Laura Wrubel and Exploring Digital Scholarship at the Library of Congress,} {\em The Signal} (blog), 20 November 2017, \useURL[url28][https://blogs.loc.gov/thesignal/2017/11/welcoming-laura-wrubel-and-exploring-digital-scholarship-at-the-library-of-congress/]\from[url28].} Librarians at George Washington University are granted the opportunity---relatively unique in academic librarianship---to propose research leave projects of up to six months in support of goals related to research, teaching, and professional development. Wrubel's goals included developing her expertise with software tools for analysis and exploration of library digital collections; contributing to the efforts of LC Labs to implement services for digital scholarship at Library of Congress; and experimenting with the Library of Congress resources, such as their APIs, to create demonstration projects and documentation.

As background, in 2016, the Library of Congress commissioned a study of library digital scholarship services and conducted a pilot in order to provide recommendations about how the library could more robustly support digital methods and computational analysis using their collections.\footnote{Michelle Gallinger and Daniel Chudnov, for the Library of Congress Lab, {\em Library of Congress Digital Scholars Lab Pilot Project Report}, 21 December 2016, \useURL[url29][https://labs.loc.gov/]\from[url29].} The report recommended development of a lab that would foster experimentation and provide expertise for researchers, including Kluge scholars, in use of digital collections and related workflows and tools. It recommended that a Library of Congress Lab be accessible to all and provide instruction and access to data scholarship methods for research with digital collections. As a visiting librarian in LC Labs, Wrubel sought to contribute to the LC Labs team's efforts to gain experience with the needs of digital scholarship projects and foster experiments that would inspire others. The team had already been facilitating the work of Kluge scholars studying digital collections, and it was through their bridge-building that Wrubel was connected with Stieber.

At the start of their collaboration, Wrubel took on the role of a facilitator of experimentation. As a librarian working with faculty and students to apply technological and often exploratory approaches to their research, this was a role with which she had experience. She first sought to understand the story of the RSHHGG Lab project to that date, including Stieber's immediate goals for the project and her dreams for its future, as well as the agreed-upon guidelines set forth by the SHHGG's executive committee.\footnote{The committee granted us access to all their materials and deputized two of their advisory board members to assist us in whatever way they could. The committee did, however, as previously mentioned, request that we not engage in a digitization project of full issues of the {\em Revue}.} This conversation was similar in some ways to the \quotation{whiteboard dream containment technique} in which the interpreter (Wrubel) and her colleagues in LC Labs would take the preliminary ideas about the shape of the project as envisioned by the dreamer (Stieber), draw out requirements, and \quotation{reduce the scope of the project to a minimally viable interpretation in which the dreamer still recognizes her initial desires.}\footnote{Kaiama Glover and Alex Gil, \quotation{The Desires of the Humanist: On the Interpretation of Digital Caribbean Dreams} (forthcoming).} Wrubel would work with Stieber to develop an achievable path for the project that would meet her short-term goal to make the {\em RSHHGG}'s contents discoverable, on a platform she could take forward beyond the project's origins at Library of Congress, all while respecting her guiding principles of accessibility and sustainability. Through these conversations, Wrubel and Stieber developed the technical requirements for this online version of the project. It needed to

\startitemize[packed]
\item
  display citations from the journal by year, tag, author, and possibly other ways;
\item
  support searching citations by keyword in any field including abstract;
\item
  be searchable or browseable by tag;
\item
  load quickly on mobile devices with limited bandwidth;
\item
  show the work as it's completed (annotations and citations were to be added over time); and
\item
  be easy to customize and maintain, as well as be portable and independent of any organizational home.
\stopitemize

The last item was of particular concern, since both Stieber and Wrubel were working on this project while on a fellowship and research leave, respectively. Although Stieber was pursuing funding from other sources, RSHHGG Lab needed to survive independent of any particular institution's support and to facilitate the future integration of other partners. In addition to supporting the goals of the project as a whole, the platform must require minimal technical effort and funding to update, migrate, and keep current; allow others to contribute to the technical and data-related work of the project; and provide an exit strategy for the data, to enable future migrations to other platforms. These criteria relate directly to accessibility, sustainability and maintenance considerations.

At the initial stage of the work, Stieber and Wrubel examined the article data Stieber and her team had created in their work with the {\em RSHHGG}'s print volumes. The approximately eight hundred citations they created were in a Google Sheet, with columns for each citation field. Google Sheets had proven helpful in supporting collaboration on data entry but did not enforce consistent metadata practices. As a result of the many participants in data entry, inconsistencies in formatting and gaps crept into the citations. For example, date formats varied, start and end pages were in the same column, and varying placeholder text was used for articles with no authors. Wrubel identified clean-up tasks that would make the citation data more amenable to transformation into other formats and usable in digital platforms. This work surfaced the value of collaborating with a librarian throughout the process but especially in the initial stages to plan how data will be gathered and managed in a digital project. Early consideration of data structure and formats can enable easier integration and transformation later.

While this data clean-up was continuing, Wrubel investigated platforms that met the requirements of the project, prioritizing those that would enable Stieber to control the data and its presentation, and would be appropriate for her and her collaborators' technical skills and resources. Zotero had emerged as a key part of the technical approach, since it provides easy management of large numbers of citations and data portability. Wrubel wrote a \useURL[url30][https://gist.github.com/lwrubel/e6e8acfa6318771fa4383cabec05f798][][{\em Python script}]\from[url30] for a one-time transformation of a CSV (comma-separated values) export of the data in the Google Sheet into the RIS format (citation format developed by Research Information Systems)that is easily imported into Zotero. She walked Stieber through the Python code, ensuring that Stieber had Python set up on her own laptop so she would be empowered to work with the data and tools herself or with other collaborators.

To understand the affordances of possible platforms, Wrubel created several prototypes using similar citation data. Prototyping helped Wrubel understand and communicate each approach's features, setup and maintenance needs, and trade-offs. Prototyping allowed Stieber to envision how each approach might manifest and to make decisions about the project as an informed collaborator. To more speedily set up prototypes, Wrubel used Docker, containerization software for deploying applications and their dependencies, which draws on a public repository of images for these containers. WordPress and Omeka both were available as Docker images that Wrubel could download and deploy with minimal setup and configuration.

WordPress with the \useURL[url31][https://wordpress.org/plugins/zotpress/][][{\em Zotpress plug-in}]\from[url31] offered the easiest setup and maintenance and flexibility in how the site looks and works, and it took advantage of Zotero's ease of managing large numbers of citations. Hosting for WordPress is widely available, including from Reclaim Hosting, and it has simple themes available for supporting fast load times. No server administrator is required, and a GUI (graphical user interface) admin site is available for managing the site. The Zotpress plug-in displays citations from Zotero, retrieved in real-time via Zotero's API (application programming interface) and has various filters available. All components were actively being updated and maintained. Setting up the prototype helped demonstrate the limitations of options for displaying items, such as the lack of API support for a query by volume number, a common need for journal data. Also, display is based on a live API call, so results can take a second or two to load, no matter what the user's bandwidth.

Wrubel also prototyped an Omeka site that used the Zotero Import plug-in. This approach requires more manual effort to configure and load data. The Omeka web-publishing software for exhibits and digital collection allows the user to import citations from Zotero as items to display in an exhibit. Because data is usually loaded only once, not changed over time, it would work better with a finished project with a complete set of citations and annotations.

Wrubel also explored the static site generator Jekyll but did not recommend it to Stieber. While Jekyll is lightweight, widely used, and customizable with plug-ins, this project had too many citations to manually maintain within the Jekyll pages and posts framework. Adding searchability through a plug-in such as lunr.js would be cumbersome, considering the number of records (citations). It is against the grain of Jekyll usage to integrate a database, nor was there at the time a Zotero plug-in for Jekyll. Wrubel also looked at Open Journal Systems, an open-source journal hosting platform that provides built-in navigation for a journal. While its website structure and layout matches the shape of the project, it requires either a system administrator or developer to set it up and maintain it or third-party hosting, which requires funding on an annual subscription basis.

Based on the review of options, Stieber selected the WordPress and Zotpress platform. In addition to the advantages already described, she knew that at her home institution there were graduate students and librarians knowledgeable about WordPress and Zotero and that therefore those applications would more easily enable future collaboration. We must acknowledge the contributors, maintainers, and communities behind the software and platforms we explored and used in this project.\footnote{See Python, \useURL[url32][https://www.python.org/community/]\from[url32]; Zotero, \useURL[url33][https://www.zotero.org/support/credits_and_acknowledgments]\from[url33]; Jekyll, \useURL[url34][https://github.com/jekyll/jekyll\#sponsors]\from[url34]; and Omeka, \useURL[url35][https://omeka.org/about/staff/]\from[url35].} These open-source projects are critical infrastructure for digital projects such as RSHHGG Lab. And while we ended up not selecting Jekyll or Omeka, using them in our prototyping helped us clarify our needs and contributed to our understanding of how the project might develop in the future. The process of experimentally configuring projects with a small set of citation data revealed the affordances of each tool, highlighting the consequences of any set of design choices and feeding our imaginations of what a fully developed lab site could offer.

\subsection[title={Creating a Searchable Online Index},reference={creating-a-searchable-online-index}]

Wrubel's Python script, while simple, was the keystone to being able to move the database from CSV to a useable, searchable, accessible online platform. This transition to the final WordPress phase came at a good time: Wrubel had since returned to George Washington University and assumed her regular duties. Stieber, now back at Catholic University herself, would (in theory) be able to manage the WordPress setup and ZotPress integration on her own. In another stroke of luck, Anne Nguyen, a brilliant high school senior at Thomas Jefferson High School for Science and Technology was eager to experiment with her knowledge of French and computer science. She spent a few weeks of her summer break working on tagging schemes and cleaning up the Zotero bibliography. Christian James, a web application librarian at Catholic University Libraries, also proved indispensable in the site-building phase, helping Stieber to work out kinks in the ZotPress code and consulting as needed.\footnote{See Christian James's faculty profile, \useURL[url36][https://lis.catholic.edu/faculty-and-research/faculty-profiles/james-christian/index.html]\from[url36].} The end result: a relatively data-light site that queried our original (and easily updatable) Zotero bibliography through a searchable index by item, by tag, and with a \quotation{browse by} option. Because of our emphasis on legibility and accessibility, we provided the main site information, including directions for searching, browsing, and about the lab, in English, French, and Kreyol. Paul Belony, PhD, provided Kreyol translations for the site, while Karine Belizar, a PhD candidate at Louisiana State University, provided French translations.

Because the project was designed to be more than just an index, Stieber sought ideas for how to make it into a dynamic space of collaboration and exchange that would bridge the North Atlantic/Haitian divide. Here, Adam Silvia's previous experience curating the project \useURL[url37][http://islandluminous.fiu.edu/][][{\em An Island Luminous}]\from[url37] was invaluable.\footnote{See {\em An Island Luminous}, \useURL[url38][http://islandluminous.fiu.edu/]\from[url38].} Silvia suggested a model of scholarly annotations similar to that project: we would send out a call for collaborators to researchers of all levels throughout Haiti and the North Atlantic (primarily the United States, Canada, and France) in \useURL[url39][http://rshhgglab.com/call-for-collaborators-rshhgg-lab/][][English]\from[url39], \useURL[url40][http://rshhgglab.com/appel-aux-collaborateurs-le-labo-rshhgg/][][French]\from[url40], and \useURL[url41][http://rshhgglab.com/apel-pou-kolaborasyon/][][Kreyol]\from[url41].\footnote{See \quotation{Call for Collaborators: RSHHGG Lab,} \useURL[url42][https://rshhgglab.com/call-for-collaborators-rshhgg-lab/]\from[url42]; also in French (\quotation{Appel à collaboration: Le Labo RSHHGG,} \useURL[url43][https://rshhgglab.com/appel-aux-collaborateurs-le-labo-rshhgg/]\from[url43]) and Kreyol (\quotation{Apèl pou Kolaborasyon: RSHHGG Lab,} \useURL[url44][https://rshhgglab.com/apel-pou-kolaborasyon/]\from[url44]).} Because the articles themselves would not be accessible, the idea was to create an online, collaborative space for scholars inside and outside Haiti to write short summaries of articles they had used from the journal for their own research. The annotations are a space especially of research collaboration and cross-boundary collaboration: they indicate if and in what publication the original journal article has been cited and are beginning to link to related digital scholarly projects (such as the \useURL[url45][https://repository.duke.edu/dc/radiohaiti][][Radio Haiti Archive]\from[url45]).\footnote{See Duke University Libraries' Radio Haiti Archive, \useURL[url46][https://repository.duke.edu/dc/radiohaiti]\from[url46].} During this process, Nathan Dize, a PhD candidate at Vanderbilt University, was instrumental. He outlined a social media and communications strategy and worked with \useURL[url47][http://rshhgglab.com/collaborators/][][contributing annotators]\from[url47].\footnote{See \quotation{Collaborators,} RSHHGG Lab, \useURL[url48][https://rshhgglab.com/collaborators/]\from[url48].} Many of these main project partners have themselves contributed article annotations to the site, all of which you can find \useURL[url49][http://rshhgglab.com/annotations/][][in the site's annotations section]\from[url49]. The site also seeks to be useful to its other stakeholders---the researchers who use and contribute to it---as academic \quotation{capital} on a curriculum vitae or in a promotion dossier. To that end, we have maintained a scholarly editing process: Matthew Casey, a historian at the University of Southern Mississippi, serves as the English-language annotations editor, and Vanessa Mongey, a historian at Newcastle University (UK), is the French-language annotations editor.

So far, the site has achieved its goal of increasing the searchability and accessibility of the journal: in its first year, it logged nearly twenty thousand visits and received positive feedback from Haitian and North Atlantic scholars alike. Marlene Daut has recently called the site a \quotation{model for a digital project that has the capacity to ensure that Haitian scholars are neither silenced nor exploited in this moment of resurgent interest in Haitian history.}\footnote{Daut, \quotation{Haiti @ the Digital Crossroads.}} Still, there remain processes to be improved and further outreach to be done. The flexibility for experimentation provided by a non--institutionally bound project also created a spotty, ad-hoc workflow that we need to improve. Moreover, while there has been fruitful collaboration with Haitian scholars, we can do more to create an egalitarian, collaborative scholarly exchange, primarily through groundwork and outreach in Haiti.

Stieber participated in one such outreach event when she visited the headquarters of the SHHGG in November 2018. She presented the project and its successes in making Haitian scholarly production accessible while also creating a system that directly involved the institution, infrastructure, and people involved with the SHHGG. Having laid the framework for a collaborative digital space that serves North Atlantic scholars {\em and} Haitian researchers, we hope a next step can be to work with the SHHGG to create a digital archive of the {\em RSHHGG.} The idea would be to use a more traditional journal storage site using the \useURL[url50][https://pkp.sfu.ca/ojs/][][Public Knowledge Project's Open Journals System]\from[url50] or some other open-source software CMS (content management system), which would be geared primarily to North Atlantic scholars.\footnote{See Public Knowledge Project, \quotation{Open Journals System,} \useURL[url51][https://pkp.sfu.ca/ojs/]\from[url51].} In light of the constraints on data usage and insufficient WiFi in Haiti, we would also create a physical archive on external hard drives, thumb drives, or, in an ideal situation, a Raspberry Pi minicomputer that would be geared toward a local, Haitian scholarly community.\footnote{We could model the idea on an open-source device called the KoomBook, researched and developed by Libraries without Borders / Bibliothèques sans frontières, \useURL[url52][https://www.librarieswithoutborders.org/koombook/]\from[url52]. WiFi is wireless networking technology commonly used for local area networking of devices and internet access.} Because the device does not need to be connected to the internet and can emit its own WiFi signal, users (patrons of the SHHGG reading room or perhaps another library or community center) could connect to the archive with phones, tablets, or computers and download directly without using data. Such a setup would allow us to create a more vibrant, dynamic community space centered on the exchange and creation of knowledge.

\subsection[title={Conclusion},reference={conclusion}]

The case of the {\em RSHHGG} and its silence in North Atlantic scholarship is far from unique. Scholarly and cultural knowledge from the Caribbean---and the global South more broadly---sits in the pages of unindexed periodicals. The opacity or unsearchability of locally produced, nonanglophone scholarly periodicals like the {\em RSHHGG} (from the Caribbean, Latin America, and elsewhere in the global South) is especially important to recognize {\em and rectify} because these are the scholarly sites---and centers---of local scholarly and intellectual production. In the spirit of experimentation and the \quotation{Lab,} we have just launched our first \useURL[url53][http://rshhgglab.com/experiments/][][experiment]\from[url53] on the site: a crowd-sourced list of locally produced Caribbean periodicals.\footnote{See \quotation{Locally produced Caribbean Periodicals: A Crowd-Sourced List,} at \quotation{Experiments,} RSHHGG Lab, \useURL[url54][https://rshhgglab.com/experiments/]\from[url54].} It is our hope that we can further emphasize the importance of locally produced Caribbean periodicals as sites of knowledge production that challenge dominant ways of knowing and infrastructures of information and catalyze additional projects to render them accessible and legible.

There are many models for institutionally structuring the work of digital scholarship, from centers to collectives, and for fostering partnerships among scholars, librarians, archivists, and technologists.\footnote{Katie Rawson and Trevor Muñoz, \quotation{Toward Collective Models of Digital Scholarship,} {\em OSF}, 17 October 2018, \useURL[url55][https://osf.io/j7qvk/]\from[url55].} Stieber, Wrubel, and Denis's collaborative effort overlapped with several organizational support arrangements, developing fluidly as the project's context changed. This changing context allowed for experimentation less bound by institutional mandates or competing responsibilities, and yet these collaborations also provided benefits to many home organizations over time. LC Labs gained experience supporting Kluge scholars and facilitating digital projects, while also being able to showcase a \quotation{collections as data} project that could inspire others to explore Library of Congress collections. Wrubel's institution benefits from her deepened experience with digital scholarship projects, exposure to digital humanities projects, and practical understanding of tools. She has been able to draw on what she learned from this project to facilitate and advise others in her library and academic community. Stieber's institution benefitted from the visibility of her project in the world and the new insights and methodologies for scholarly practice that she brings to her teaching and research. Denis and the SHHGG received greater visibility and impact while maintaining the scholarly and intellectual capital of their work within their institutional infrastructures at home. We and our many partners are experimenters and collaborators but also information maintainers who have the responsibility and opportunity to ensure the continued existence of an entry point to this periodical. We look forward to adapting and experimenting as we continue to the next stage of the RSHHGG Lab.

\thinrule

\page
\subsection{Chelsea Stieber}

Chelsea Stieber is an assistant professor of French and francophone studies at the Catholic University of America. She is the author of {\em Haiti's Paper War: Post-independence Writing, Civil War, and the Making of the Republic, 1804--1954} (New York University Press, forthcoming). She is also a coeditor of the critical translation {\em Haiti for the Haitians} (Liverpool University Press, forthcoming).

\subsection{Laura Wrubel}

Laura Wrubel is a software development librarian at the George Washington University. She supports digital scholarship at the Libraries and Academic Innovation, including applications such as the open-source software \useURL[url1][https://go.gwu.edu/sfm][][Social Feed Manager]\from[url1] for building social media datasets. She is a \useURL[url2][https://carpentries.org/][][Carpentries]\from[url2] instructor and teaches workshops on Python, Git, and other computational topics for students, faculty, and library staff.

\subsection{Watson Denis}

Watson Denis is a professor of history, international relations, and Haitian social thought at the State University of Haiti and the general secretary of the Société haïtienne d'histoire, de géographie et de géologie. He is the author of {\em Haïti au-delà du 12 janvier 2015: Droit, constitution et politique} (C3 Éditions, 2015), {\em Haïti: Changer le cours de l'histoire} (C3 Éditions, 2016), and {\em Haïti, la CARICOM et la Caraïbe: Questions d'économie politique, d'intégration économique et de relations internationales} (C3 Éditions, 2018). His latest book, {\em La négation du droit à la nationalité: La situation d'apatridie des Dominicains et Dominicaines d'ascendance haïtienne en République dominicaine} is forthcoming in 2020.

\stopchapter
\stoptext