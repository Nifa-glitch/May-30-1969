\setvariables[article][shortauthor={Glover, Gil}, date={July 2017}, issue={2}, DOI={10.7916/D8NS1689}]

\setupinteraction[title={},author={Kaiama L. Glover, Alex Gil}, date={July 2017}, subtitle={}]
\environment env_journal


\starttext


\startchapter[title={}
, marking={}
, bookmark={}]


\startlines
{\bf
Kaiama L. Glover
Alex Gil
}
\stoplines


We're back. One year in with this experiment of a journal, and, though our baby is by no means all grown up, she has come a ways. A year ago {\em sx archipelagos} was a nascent and necessarily fragile thing. As we brought her out into the world, we were cautious and concerned in every respect---helicopter-parenting each submission, pouring our hearts and souls into the back-end development, praying that the techo-critical gamble of our minimal computing workflow design would pay off, soliciting the counsel of our closest colleagues, making sure to understand every nut and bolt of infrastructure, every nuance of narrative. \useURL[url7][\%7B\%7Bsite.baseurl\%7D\%7D/issue01.html][][{\em sxa} (1)]\from[url7] was our announcement of the Small Axe Project's investment in and fascination with our Caribbean digital community as we editors see it and believe it to be---in all its complexity and promise.

So what now? Having built it, can we be certain they will come? Have we created the right platform for our people---a space our community will want to contribute to, learn from, challenge, and explore? These are the nail-biter questions we have had to pose in putting together this, our sophomore issue of {\em sx archipelagos}. We have had to confront the successes and, let us say, the \quotation{not-quite-successes} of our first steps. We have had to solicit and attend to critique. We have had to integrate into {\em sxa} (2) the lessons learned and pledges made. This has meant fine-tuning our own understanding of what Caribbeanist scholarship of the digital can and should do and communicating this to our contributors. It has meant insisting on the distinction between the descriptive and the analytical---even as we make room for both---and pressing our authors not only to identify and marvel at computational phenomena, the creation and preservation of digital surrogates, wonders of interaction, and such, but also to attend rigorously to the specific intellectual projects that undergird and, importantly, call expressly for the digital in order to tell their story or to make their contribution. As we considered submissions for publication, and as we engaged with our authors and peer reviewers, foregrounding the balance between the how and the why of engaging with the digital in the Caribbean emerged as the critical ethos guiding our editorial perspective.

The essays in this issue all turn around the question of Haiti, history, and the digital. Laura Wagner's \quotation{{\em Nou toujou la!} The Digital (After)Life of Radio Haïti-Inter} queries the role of technology in doing memory work. As lead archivist of Duke University Library's Radio Haiti Project, Wagner asks what it means for an elite US institution to assume responsibility for the caretaking of and making available a crucial element of Haiti's social and political patrimony. Her essay brings the matter of the library, and practices of archival stewardship more broadly, to the forefront of our ruminations on digital technologies in the specific geocultural space of the Caribbean. In \quotation{Digital Saint-Domingue: Playing Haiti in Videogames,} Sarah Juliet Lauro offers insight into the relatively uncharted realm of the representation of the Caribbean in the gaming world. Lauro questions whether and how it is possible to play at revolution---to do so responsibly, that is. She looks hard at the perils and promise of entertainment as pedagogical practice. In the section's closing essay, \quotation{Intervening in French: {\em A Colony in Crisis}, the Digital Humanities, and the French Classroom,} Nathan H. Dize, Kelsey Corlett-Rivera, Abby R. Broughton, and Brittany M. de Gail are similarly concerned with pedagogy and ethical/accurate representations of the Haitian past. Their essay insists on the crucial role digital technology can and should play in the French-language classroom.

Our second section, \quotation{Digital Projects,} features the public peer review of the digital cartography site {\em Ramble Bahamas: A Project by \quotation{From Dat Time.}} Presented with feedback on both the form and content of their platform, site authors Jessica Dawson and Tracey Thompson responded generously and thoroughly to the {\em sx archipelagos} reviewer's queries and critiques, clarifying and streamlining elements of their site and providing an insightful written account of the site's logic and intentions. The second essay in this section, Angela Sutton's \quotation{The Digital Overhaul of the Archive of Ecclesiastical and Secular Sources for Slave Societies (ESSSS),} breaks new ground for us. With this submission we opened up the \quotation{Digital Projects} section to the {\em inventio} genre---often called \quotation{project narratives} or \quotation{progress narratives}---familiar to digital humanities audiences. In her essay, Sutton opens a window onto the infrastructural, labor, and intellectual concerns behind a digital archive project.

Finally, we round out the issue with two digital project reviews. J. Cameron Monroe and Peter James Hudson look at {\em The Digital Archaeological Archive of Comparative Slavery} and {\em The Caribbean Memory Project}, respectively. Both of these review essays celebrate the ways these projects convey the incisiveness and originality of their creators' vision while relying on both the sustained collaboration and punctual contributions of their intellectual communities.

The contents of this second issue of {\em sx archipelagos} reflect the distance traveled since the launch of this platform a year ago. We have met the bar we set with that first issue and even raised the stakes in ways. We have grown and engaged with a community of generous readers and collaborators. We have the support of a brilliant (and generous) newly-constituted \useURL[url8][\%7B\%7Bsite.baseurl\%7D\%7D/credits.html\#editorial-board][][editorial board]\from[url8]. We have been indexed by Google Scholar and WorldCat. We have joined our voice enthusiastically to a vast network of Caribbean conversations online. We have made partnerships to ensure deep archiving. We are already lining up exciting content for {\em sxa} (3) . . .

We are no more a fragile thing.

Onward,\crlf
Kaiama & Alex\crlf
Editors

\page
\subsection{Kaiama L. Glover}

\useURL[url1][http://web.archive.org/web/20170904034516/https://barnard.edu/profiles/kaiama-l-glover][][Kaiama L. Glover]\from[url1] is Associate Professor of French and Africana Studies at Barnard College, Columbia University. She is the author of \useURL[url2][http://liverpooluniversitypress.co.uk/products/61903][][Haiti Unbound: A Spiralist Challenge to the Postcolonial Canon]\from[url2] (Liverpool UP 2010), first editor of \useURL[url3][http://yalebooks.com/book/9780300214192/yale-french-studies-number-128][][Marie Vieux Chauvet: Paradoxes of the Postcolonial Feminine]\from[url3] (Yale French Studies 2016), and translator of Frankétienne's Ready to Burst (Archipelago Books 2014). She has received awards and fellowships from the National Endowment for the Humanities, the Mellon Foundation, and the Fulbright Foundation. Current projects include forthcoming translations of Marie Vieux Chauvet's {\em Dance on the Volcano} (Archipelago Books) and René Depestre's {\em Hadriana in All My Dreams} (Akashic Books), and the multimedia platform {\em In the Same Boats: Toward an Afro-Atlantic Visual Cartography}.

\subsection{Alex Gil}

\useURL[url4][http://www.elotroalex.com/][][Alex Gil]\from[url4] is Digital Scholarship Coordinator for the Humanities and History at Columbia University Libraries. He collaborates with faculty, students and the library on the use of technologies on humanities research, pedagogy and scholarly communications. His research is focused on textual scholarship, digital humanities and Caribbean studies. Current projects include \useURL[url5][http://web.archive.org/web/20170904034523/http://elotroalex.github.io/ed/][][Ed]\from[url5], a foundation for {\em sx archipelagos}; the Open Syllabus Project; a geo-bibliography of Aimé Césaire; the Translation Toolkit; and, In The Same Boats, a visualization of trans-Atlantic intersections of black intellectuals in the 20th century. He is co-founder and active member of the Global Outlook::Digital Humanities initiative, \useURL[url6][http://xpmethod.plaintext.in/][][Columbia's Group for Experimental Methods in the Humanities]\from[url6], and the Studio@Butler at Columbia University.

\stopchapter
\stoptext