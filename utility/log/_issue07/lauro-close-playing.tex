\setvariables[article][shortauthor={Lauro}, date={May 2023}, issue={7}, DOI={10.7916/archipelagos-0703}]

\setupinteraction[title={A Close-Playing: The Videogames of Patrick Chamoiseau},author={Sarah Juliet Lauro}, date={May 2023}, subtitle={A Close-Playing}, state=start, color=black, style=\tf]
\environment env_journal


\starttext


\startchapter[title={A Close-Playing: The Videogames of Patrick Chamoiseau}
, marking={A Close-Playing}
, bookmark={A Close-Playing: The Videogames of Patrick Chamoiseau}]


\startlines
{\bf
Sarah Juliet Lauro
}
\stoplines


{\startnarrower\it In his many works, Martinican author Patrick Chamoiseau wrestles with Caribbean identity, colonial influence, the complexity of blackness, and the history of slavery. Best known for his Prix Goncourt winning novel {\em Texaco} (1992), Chamoiseau is considered a major voice in francophone literature, contributing significantly to the literature of the African diaspora and its self-definition. His works disrupt expectations of genre and form: {\em Texaco}, for example, braids together disparate kinds of narrative (journal entries, letters) and cultural references (historic incidents and spiritual concepts). This attitude to the polyvalence of storytelling may have led to Chamoiseau's work in another narratological form, the field of videogames. He is credited on two titles both produced by Cocktel Vision: {\em Freedom: Rebels in Darkness}, an 8-bit game produced in 1988, and {\em Méwilo}, a 1987 first-person point and click adventure game, which like {\em Texaco}, dramatizes the 1902 volcano eruption that destroyed Saint Pierre. {\em Freedom: Rebels in Darkness} combines the adventure game (and even some Street Fighter-like button mashing) with the intense historical reality of slave resistance. {\em Méwilo} is a detective game where the playable character must solve a mystery about a paranormal haunting with its roots in the 1831 slave revolt. This paper offers a reading of these two games alongside several of Chamoiseau's literary offerings and the broader theory espoused in his work in order to analyze these texts as commemorations of slave resistance that make use of a strategy of opacity evident in much of Chamoiseau's work. This article comes out of a piece I was writing on \quotation{Creolizing videogames,} and I think of these two articles as sisters. Here, I offer a close-playing of the games, and my goal is to introduce these understudied texts to those who might want to play them. In this medium, I can share screencaps and videos of my gameplay.

 \stopnarrower}

\blank[2*line]
\blackrule[width=\textwidth,height=.01pt]
\blank[2*line]

\subsection[title={Part I. Introduction.},reference={part-i.-introduction.}]

A \quotation{close-playing,} for those unfamiliar with the term, is like a close-reading of a literary text, but for videogames; it is a term coined by Edmond Chang, comprising analysis of the game's \quotation{intersection of form, function, meaning, and action.}\footnote{Edmond Chang, \quotation{Close Playing: A meditation on teaching with videogames} (11.11.2010), \useURL[url1][http://www.edmondchang.com/2010/11/11/close-playing-a-meditation/]\from[url1].} Just as performing a close-reading requires paying particular attention to the way form works in tandem with the content (in a poem this may mean scrutinizing enjambment, line length, eye-rhyme, etc.) a close-{\em playing} pays special attention to the ludic operations of the game---the game mechanics--- for the part they play in enhancing the narrative. What are we going to be close-playing?

The famed author Patrick Chamoiseau had a role in scripting aspects of two videogames, {\em Méwilo} (Coktel 1987) and {\em Freedom: Les guerriers de l'ombre} (Coktel 1988), to which we will turn our attention directly. But I want to emphasize from the start that these games were produced by Muriel Tramis, a notable Black woman video game designer from Martinique. Given the accusations of misogyny with which Chamoiseau's {\em Éloge de la creolité} (co-authored with Jean Bernabé and Raphael Confiant) was met, I want to make sure that we begin by acknowledging her work, even though this essay focuses on Chamoiseau's contributions to the written scripts.\footnote{{\em Éloge de la creolité} was a manifesto whose authors (Jean Bernabé, Patrick Chamoiseau and Raphaël Confiant) argued that the predecessor movements of {\em négritude} (Aimé Césaire) and {\em antillanité} (Édouard Glissant), which sought to define Afro-Caribbean identity, did not go far enough in embracing the hybridity, plurality, and radical potential of creolization. According to the authors of {\em Éloge}, Césaire's {\em négritude} was a \quotation{violent and paradoxical therapy} (889) that, while seeking to break the stranglehold of European cultural domination by appealing to a pan-African identity, ultimately amounted to the substitution of one brand of mimeticism for another. By contrast, the authors quarrel with Glissant's {\em antillanité} was minor. They agreed with his assessment of the inherent plurality of \quotation{Caribbeanness} but argued that his writings on the subject were \quotation{more a matter of vision than a concept} (890). And in short, they hoped to set Glissant's philosophy on its feet. Their complaint with Glissant is hard to distill down to a line or two, possibly because it is unconvincing. Holly Collins writes: \quotation{One could say that the idea of creolité, based largely on Glissant's theories and writings, evolved out of Antillean self-assertion and additionally seeks to promote Creole as a language as well as a cultural idea} (70). Chamoiseau later clarified that he supported the Glissantian \quotation{appreciation of the ongoing process of creolization as opposed to an essentialist celebration of Creole culture} (Knepper, 27). A major bone of contention was its masculinist characterization of creoleness and a general carelessness with the issue of gender, which would lead to an overt critique by Maryse Condé, first in the 1995 collection published by Condé and Madeleine Cottenet-Hage called {\em Penser la créolité.} For a helpful gloss on this issue, Note 52, p.~143 in Wendy Knepper's {\em Patrick Chamoiseau: A Critical Introduction}.} Tramis's body of work includes \quotation{les jeux Adi} ({\em Accompagnement Didacticiel Intelligent}), early childhood education games well-known to French school children, and the infamous sexual adventure game {\em Emmanuelle} (Coktel 1989).\footnote{Tramis also has another game about slavery, {\em Lost in Time} (Coktel Vision, 1993) which is described as a time travel adventure game, but Chamoiseau did not work on that one, and so far I have been unable to play it.} Recent interviews that Tramis has given might suggest that she would object to sharing equal credit for these labors.\footnote{In her interview with Sepinwall, Tramis reveals that she \quotation{tapped Chamoiseau to write {[}both{]} game's dialogue and text boxes} (217). See also n 12, p.~292 which contrasts the text of the jewel case materials for the French language Atari version of {\em Freedom}, in which it is stated that the original idea was Chamoiseau's; in Tramis's recent recollection, they came up with the concept \quotation{collaboratively.}} I want to provide more than a sidelong glance at Tramis, inasmuch as her work provides a way of introducing some of the key themes discussed in the gamic narratives.

Muriel Tramis was the first female game designer to be awarded France's Legion of Honor in 2018. In the wake of Gamergate and increased attention to gender in videogame studies---and one also might suggest, in the era of Black Lives Matter, the 1619 project, and a broader moment highlighting legacies of slavery---Tramis's work has garnered some interest even in Anglophone games journalism and scholarship. In the 2020 article \quotation{Une vie bien jouée/A Life Well Played: The Cultural Legacy of Games Designer Muriel Tramis,} Alenda Chang compares Tramis to \quotation{France's own Roberta Williams.}\footnote{Alenda Chang,\quotation{Une Vie Bien Jouée/A Life Well Played: The Cultural Legacy of Game Designer Muriel Tramis,} {\em Feminist Media Histories} 6, no. 1 (2020), 151. Chang provides a good overview of Tramis's work in the preface to the interview. See also Filip Jankowski, \quotation{The Presence of French Female Game Designers in Video Game Industry, 1985-1993.} Tramis's work has also been profiled in Tristan Donovan's {\em Replay: The History of Video Games}~(2010) and Mark J.P Wolf's edited collection {\em Videogames Around the World} (2015).}

The greater attention paid to Tramis also might be a function of the retrogaming trend, wherein old games enjoy reinvigorated popularity due to the internet. Tramis had been on my radar since Maddi Chilton published an article in {\em Killscreen} titled, \quotation{A Forgotten, Decades-old Game about Slavery has returned.}\footnote{Maddi Chilton, \quotation{A Forgotten, Decades-old Game about Slavery has returned,} {\em Killscreen,} (May 31, 2016).} Chilton was right to note how \quotation{ahead of its time} the game now feels, writing that the creation of {\em Freedom}, \quotation{a game by an Afro-Caribbean woman about a violent slave revolt on the plantations of Martinique,} at the birth of mainstream videogame culture, \quotation{is frightening in its honesty.}\footnote{Chilton.} Since 2016, two books have emerged that take on the subject of videogames about slave revolt: Alyssa Sepinwall's exhaustive study, {\em Slave Revolt on Screen: The Haitian Revolution in Film and Videogames}\footnote{Alyssa Goldstein Sepinwall, {\em Slave Revolt on Screen: The Haitian Revolution in Film and Videogames} (Jackson: U of Mississippi, 2021).}, and my thin volume for the University of Minnesota's Forerunner series, {\em Kill the Overseer! The Gamification of Slave Resistance} (2020). The theme of slave rebellion is at the heart of both of these games upon which Chamoiseau worked, the aforementioned {\em Méwilo} (1987) and {\em Freedom: les guerriers de l'ombre} (1988) both produced by Coktel Vision. Part III of Sepinwall's monograph is devoted to \quotation{Video Games on Slavery and the Haitian Revolution,} with chapter 8 addressing the Assassin's Creed franchise and its depictions of Haiti and Haitian rebels and chapter 9 on Tramis's games about slavery. {\em Kill the Overseer! The Gamification of Slave Resistance} (Lauro) focuses especially on games that make the resistive slave a playable character, and it profiles Tramis's {\em Freedom: les guerriers de l'ombre} briefly, among its discussion of educational games made for use in the classroom, art games, and mainstream console games.\footnote{For those looking for a discussion of slavery and videogames more broadly, see Emil Hammar \quotation{Counter-Hegemonic Commemorative Play,} and Souvik Mukherjee's article \quotation{Video Games and Slavery.} Souvik Mukjerjee raises an important question in \quotation{Videogames and Slavery,} regarding whether it is \quotation{impossible to play from the position of a slave as that involves an absence of agency and selfhood} (252), but as these games are more about resistance than about slavery, this remains unanswerable. For those interested in games and empathy the discussion of \quotation{serious games} may also prove fruitful to explore, and that is discussed in Bogost's {\em Persuasive Games: The Expressive Power of Videogame}. The MIT Press, 2010.} {\em Méwilo} describes the reverberations of an event that took place during a slave revolt in Martinique, and {\em Freedom} makes a game of attempted slave rebellion.

The games I am looking at here foreground Creole culture and folklore, including mythical figures like {\em zombis} and {\em soucouyans} in {\em Méwilo}; and, at least according to the game manual, the rebels in {\em Freedom} can be aided by sorcerers and even a goddess.\footnote{The word \quotation{zombi} (or \quotation{zombie}) hardly needs a footnote, but the use of the word here (and elsewhere in Chamoiseau's oeuvre) is less like the more familiar walking corpse that came into popular understanding through the Haitian concept of the {\em kokadav}, and more like a ghost, while a soucoyan is a shape-shifting vampire. Page 2 of the Freedom manual explicitly defines the potential roles of the quimboiseurs (sorcerers) and of the Sirène Manman-Dlo, the mermaid water diety, who can add more time to the player's clock if encountered.} {\em Méwilo} is based on a folktale about a slave master's buried gold that was previously featured in Chamoiseau's novel {\em Chronique des sept misères} (1986), and which, according to Renée Gossen, \quotation{symbolizes the wealth that the {\em békés}, white French settlers and plantation owners, have wrongfully reaped from a land which is not theirs and, quite literally, on the backs of African slaves.}\footnote{Renée K. Gosson, \quotation{For What the Land Tells: An Ecocritical Approach to Patrick Chamoiseau's}Chronicle of the Seven Sorrows” {\em Callaloo}, 26, no. 1, (Winter, 2003): 232.} Tramis explicitly mentions this as a long-standing myth in her interview with Tristan Donovan in~ {\em Replay: The History of Video Games}: \quotation{The game was inspired by the Carib legend of jars of gold,} explained Tramis. \quotation{At the height of the slave revolts, plantation masters saved their gold in the worst way. They got their most faithful slave to dig a hole and then killed and buried him with the gold in order that the ghost of the unfortunate slave would keep the curious away from the treasure.}\footnote{Tristan Donovan, {\em Replay: The History of Video Games}. Sussex: Yellow Ant (2010): 127.

  For more on this trope of buried treasure, often guarded by the ghost of a murdered slave, see Coverdale, \quotation{Notes} (note 54, p.144).}

I am interested here in thinking of these games as \quotation{Creole} not only in their {\em content}---that is, by making reference to various aspects of the Afro-Diasporic-Indigenous-European cultural stew that one finds in the Caribbean and other sites of colonization--- but also in their {\em form,} as by defying expectation, instructions, or genre boundaries. The word \quotation{Creole} has a long history, signifying, first, both white persons of European descent and enslaved persons born in the colonies; and later, a population with mixed heritage; it can signify both language and culture. I understand \quotation{creole} through Glissant's antillanité and Chamoiseau, Confiant, and Bernabé's \quotation{créolité,} as that which pushes back against \quotation{false universalism, monolinguism and purity} (Éloge de la créolité). For me, Creole defies rigid compartmentalization. Elsewhere I have sought to articulate more specifically a new category of \quotation{creolizing videogames,} that reference oral culture and folklore even as they exercise a kind of Creole modality in their multisensory forms, linguistic play, defiance of genre, emphasis upon collective subjectivities, and more (in Russworm and Murray). I am especially interested in creolizing as a hermeneutic process, as in the series Creolizing the Canon put forth by the Caribbean Philosophical Association. Lastly, I want to emphasize here that Chamoiseau, Confiant, and Bernabé insist that the concept of créolité \quotation{is not limited to the American continent \ldots{} {[}but{]} refers to the brutal interaction, on either insular or landlocked territories {[}\ldots{}{]} of culturally different populations}\footnote{Jean Bernabé, Patrick Chamoiseau and Raphaël Confiant {\em Éloge de la Créolité}. (1989.) Trans. Mohamed B. Taleb Khyar, {\em Callaloo}, 13, no. 4 (Autumn, 1990): 893.} and thus invites broad application. The sense in which I use it here stresses a willful defiance of monolithic categories, as in recent efforts to creolize the canon.\footnote{I have not yet explored scholarly work written in French on the subject of slavery in games. In part, this is because work on videogames was first taken seriously in US academic circles, and the field remains more deeply engaged by scholars writing in English. However, Julien Bazile has written a thesis on the broader topic of the Assassin's Creed games for Université de Lorraine; Université de Sherbrooke (Québec, Canada); he also delivered a conference paper specifically on Muriel Tramis's games about slavery called \quotation{L'Histoire à traverser. Regards vidéoludiques sur l'histoire de la Martinique dans Mewilo et Freedom de Muriel Tramis.}} At the nexus of form and content, we might consider the ways that digital games can remap written archives (as by referencing oral narratives).

One goal of this essay is to emphasize that the archive of a people's oral history, which we find in song, literature, and folklore, may also sometimes turn up in unexpected media, like videogames. Tramis stated this intention in an interview with game blogger Ilmari:

\startblockquote
{[}T{]}he dictatorship of sugar has caused an imbalance in the environment and human relations of {[}the Caribbean{]} region, and the Creole society still bears the aftermath. I felt a great emptiness on the side of our myths and founding legends, on the side of our history, unknown or suffocated under a bundle of shame. I wanted to extend the Creolity {[}sic{]} movement to video games too.\footnote{Ilmari. \quotation{Interview with Muriel Tramis,} {\em The Adventure Gamer} 19 (Mar 2018).

  <\useURL[url2][https://advgamer.blogspot.com/2018/03/interview-with-muriel-tramis.html]\from[url2]\underbar{>}}
\stopblockquote

It is an important intervention to emphasize the créolité of videogames, and most of the articles on Tramis's games about slavery speak of it as {\em testimony}.\footnote{Sepinwall's interview with Tramis reveals that she sought to address the absences in the historical record, \quotation{to integrate into games the history of her enslaved ancestors, which was discussed privately but not taught in school} (211). This is a central theme, too, in the writings of Chamoiseau, particularly evident in School Days.} \quotation{Fugitive slaves, my ancestors, were true warriors that I had to pay tribute to\ldots{} It was {\em my duty to remember}.}\footnote{Donovan, {\em Replay: The History of Video Games}, 127.} A videogame can, if not {\em overcome}, at least draw attention to, the elisions of the historical archive. Sepinwall reports in her interviews with Tramis that this was one of the designer's explicit goals: \quotation{Tramis sought to overcome {\em archival silences}; she wanted to bring enslaved people to life by drawing from community memory rather than documents left by colonists}\footnote{Sepinwall, 212.} (italics added). It is for this reason that her games turn to folklore for the political experiences, traumas, and victories of a people without ready access to ink and vellum, exploring oral histories as an alternative archive.\footnote{For two examples of the ways that folklore inscribes the history of slavery, see my book {\em The Transatlantic Zombie: Slavery, Rebellion, and Living Death} and Bryan Wagner, {\em Tar Baby: A Global History}.}

The insufficiency of the historical archive is a major point of discussion in slavery studies. Many scholars have grappled with this lack and how to fill it.\footnote{Here I am explicitly referring to Achille Mbembe's \quotation{The Power of the Archive and its Limits} and Saidiya Hartman's \quotation{Venus in Two Acts.} Those interested in archive trouble should see the special issue of the journal History of the Present, edited by Brian Connolly and Marisa Fuentes. My book Kill the Overseer and article \quotation{Digital Commemorations of Slave Revolt} also discuss the redress of the historical archive through the digital form. Readers specifically interested in the archive and Martinique should see Christine Chivallon's ethnography of accounts of the descendants of slaves, as a counter-archive.} The voices of those humans who were considered property are rarely found among the holdings of libraries. Just as {\em créolité,} as articulated by Bernabé, Chamoiseau, and Confiant in their 1989 manifesto, rejected the colonial imposition of the French and experimented with linguistic play, so do these games celebrate Caribbean blackness and culture by looking to folklore and local histories of resistance for inspiration. Further, this essay seeks to articulate the way these games use a particular device that we associate with Chamoiseau's literary œuvre, opacity. In brief, I argue that the narrative complexity of these ludic texts raises a question about whether or not the history of slavery (and resistance) ought to be made {\em playable}. Extrapolating from Edouard Glissant's invocation of opacity in theorizing identity, we can consider {\em opacity} as a literary and artistic device that has broad application for establishing limits to encroachment, cultural consumption, or the prurient interest of outsiders.\footnote{In Glissant's {\em Poetics of Relation}, opacity is discussed in terms of identity and acceptance of others. The key idea is that we should \quotation{give up this old obsession with discovering what lies at the bottom of natures} (190). The idea is that one should not have to walk a mile in another's shoes in order to be respectful.} In terms of identity, opacity signifies “the survival of diversity, of irreducible otherness \ldots{} an irreducible particularity, the stubborn otherness of the Other.\footnote{Linda Coverdale, \quotation{Afterword,} in Patrick Chamoiseau, {\em Chronique des sept misères.} (1986) Trans. Linda Coverdale. (Lincoln: University of Nebraska Press, 1999), 128.} In artistic practice, opacity might be a bulwark against exoticization, fetishization, othering. It can be a tool of resistance. It may look like a narrative barricade through which only certain readers can pass to a deeper level of understanding, when, as with the use of an untranslated Creole expression in the midst of a francophone passage, the text {\em selects} its reader. It's a mode we see often employed across Chamoiseau's body of work. I theorize that some uses of Glissantian opacity in the games productively block this history from complete commercialization.

In the \quotation{Afterword} to her translation of {\em Chronique des sept misères}, Linda Coverdale writes, \quotation{Chamoiseau is not fond of glossaries and believes that the \quote{Creole dimension} of his work should be safeguarded against the alienating ideal of \quote{transparency,}}\footnote{Coverdale, 216.} but rather than just using Creole, he is apt to create his own words, what one scholar called his \quotation{Chamoisifying} of language.\footnote{See Knepper, 96.} As he wrote in the introduction to {\em Creole Tales}, \quotation{the storyteller must take care to use language that is so opaque, so devious---its significance broken up into a thousand sibylline fragments\ldots{}} \footnote{Patrick Chamoiseau and Georges Puisy. {\em Delgrès, Les Antilles Sous Bonaparte}, éditions Émile Désormeaux, Pointe à Pitre (1981), xvii.} (italics in original). We might go so far as to hazard that the collaborative process of reading is the reassembly of these fragments into a picture, and that the participatory nature of the videogame narrative only makes this more visible. With a hat tip to Edouard Glissant, Chamoiseau continues, \quotation{the Storyteller's object is almost {\em to obscure as he reveals}} (italics in original).\footnote{Chamoiseau and Puisy, xvii.}

Opacity and creolization are of a piece: both work against unilateral apprehension. Chamoiseau's œuvre is also creolized in form: it challenges notions of authorship, genre, media, and language. The author's diverse output includes screenplays, multimodal works, and collaborations. For example, Chamoiseau's 1999 fable {\em Seven Dreams of Elmira} is a collection of different versions of a folktale accompanied by photo portraits of the tellers by Jean-Luc de Laguarigue. The text is written from the perspective of an old worker at the Saint-Etienne Rum Distillery, a man who tells us that he is over one hundred years old and recounts the distillery's various mythologies. A note from the author (signed P.C.) lists his informants, \quotation{commanders of memory,} as his sources for the various ideations of the folktale included therein. Chamoiseau also penned a comic book---in Creole and in French---with Georges Puisy, {\em Delgrès, Les Antilles sous Bonaparte}, about the struggle against the reestablishment of slavery in Guadeloupe by Napoleon. One aspect of creolized texts is a disruption of authorial expectations. Inasmuch as they are inherently collaborative, videogames are a ripe medium in which to stage conversations about collectivity and identity, thus the form lends itself to being considered through the lens of creolization.

That one of Caribbean literature's most notable authors collaborated on the production of two videogames in the 1980s has been largely left unconsidered by Chamoiseau scholars. This is especially surprising given how the games echo (or foreshadow) his novels.\footnote{To name but a few: {\em Chronique des sept misères}, 1986; {\em Solibo Magnifique}, 1988; the Prix-Goncourt winning {\em Texaco}, 1992; {\em L'Esclave vieil homme et le Molosse},1997; as well as several autobiographical works, like {\em Chemin d'Ecole}, 1994} Many of the major elements of his work appear in these ludic texts: {\em Méwilo,} like his novel {\em Texaco}, evokes the tragic eruption of Mont Pelée in 1902\footnote{~Patrick Chamoiseau, {\em Texaco} (1992) Trans. Rose-Myriam Réjouis and Val Vinokurov, trans, 1997. (New York: Vintage Books, 1998). ~} and, like the novel {\em Solibo Magnifique}, features a detective trying to solve a magical mystery.\footnote{{\em Solibo Magnificent} is a police procedural about a surreal event: a man who died of an {\em égorgette}, which is translated as being \quotation{snick'd} (20), {\em killed} by a word he uttered.} Similar to {\em Seven Dreams of Elmira}, {\em Méwilo} stages some of its action in a haunted former plantation and, as with {\em The Chronicle of the Seven Sorrows,} the game contains a tale of buried treasure and the ghost (or {\em zombi)} of its enslaved guardian.\footnote{~Chamoiseau, {\em Chronicle of the Seven Sorrows}. (1986). Trans. Linda Coverdale. (Lincoln: University of Nebraska Press, 1999); As Sepinwall has noted previously (220), in {\em Freedom} there are man-eating dogs, which features centrally in {\em Slave Old Man.}} What's more, at least one scholar has referred to Chamoiseau's use of creole folktales as \quotation{ludic,} so one would think that the literal games he designed would be of interest.\footnote{~See Wendy Knepper on the \quotation{ludic function of Creole stories}: \quotation{For all of its magic, play, and opacity, the tales are engaged, politically and socially, with the situation of reader and storyteller} (101).} I second Alyssa Goldstein Sepinwall's surprise that major books on Chamoiseau fail to address these videogames, given the diversity of the author's artistic output and scholars' attention to it.\footnote{~See Sepinwall, n.~5, 291. I'm grateful to Sepinwall for her chapter on these games, and for her generosity sharing resources.} The games are rich with themes that we see across his wider œuvre ---slavery, colonialism, language and folklore---but they are also exceedingly difficult to find and to play successfully, which may account for their scholarly neglect.

I believe that frustrated interactivity (a concept in videogame studies) reverberates with potential for Creole storytelling as a mechanism of opacity.\footnote{On choice and limitation in videogames see Ian Bogost, {\em Persuasive Games: The Expressive Power of Videogames} and {\em How to Do Things With Videogames} as well as Alexander Galloway, {\em Gaming: Essays on Algorithmic Culture} and Jesper Juul, {\em Half-Real: Videogames between Real Rules and Fictional Worlds}.} What does this look like, in practice? On the one hand, it looks like games that include challenges that are not easily solved, that incorporate uses of operational weakness and in-game frustration to make a point about the subject matter. On the other, it may look like games that are explicitly unwinnable. My own gameplay, described in the pages that follow, consisted of much running into brick walls (figuratively in these games, but elsewhere, literally). The gamic obstructions concretize the mechanism of opacity, making the path forward obscure, difficult, or unfun.

{\em Freedom} and {\em Méwilo} rely significantly on productive opacity. They are difficult, if not altogether unwinnable. Their opacity evokes the near-impossibility of resistance to the plantation machine, but on a deeper level it may also protect this history from appropriation or even amusement. As Coverdale has written, opacity can be \quotation{a mode of pushback against cultural appropriation} (\quotation{Afterword} 128). The game mechanics demand that the player work to understand Caribbean history. And even then, the legacy of slave resistance may be beyond the player's control.

Part of what makes the videogame an ideal form in which to create Creolizing texts is its inherent plurality: there is no one experience of the game, just as creolizing texts emphasize collaborative storytelling above individual authors.\footnote{On this point see Edouard Glissant's \quotation{A Word Scratcher,} an introduction to the republished translation of {\em Chronique des sept misères} in Patrick Chamoiseau, {\em Chronique des sept misères}. (1986) Trans. Linda Coverdale. (Lincoln: University of Nebraska Press: 1999).} My experience of the gamic text will be radically different from that of any other player, due to all the permutations of play. As such, the experience of playing is one of collaborative meaning-making between the player and the design(er/s). My play of these games is as an outsider, non-Caribbean, a middle-aged woman not very good at gaming, and making attempts to play in the years 2019-2021, in which the technology of the original games has become obsolete. That is, because I do not have a functional Atari lying around, I can only play these games through online emulators that come themselves with a range of problems. All of this is part of {\em my} experience of the game.\footnote{There is the game as it was designed (the program), but each gamer brings to the game diverse skill levels, from hand- eye coordination to predecessor knowledge of game structures, to cultural references, historical knowledge, and language ability. This is, of course, equally true of the experience of reading a book---but the form of the game, which actively resists the player's ascension to the next level, makes this more obvious than a written text does.} Yours would no doubt be different.\footnote{~For those interested in seeing more of these games, there are some limited playthrus available on Youtube. Watching these will give one a sense of the difficulty of playing these retro games for the contemporary gamer---the mechanics are not intuitive for someone who has grown up with a different level of technology, and the fact that the manual is only available online in French presents a further difficulty for some. I have found three playthrus online (each between 6 and 8 mins). The first two made no attempt to rally the slaves, but the third, by petslimjim2 allows you to watch, in real time, the gamer figuring out that you have to not only click on a slave's hut to force a conversation with them, but then, further, to click on their forehead in the picture that is then brought up of them in order to attempt to convince them to join the cause. (It was excruciating to watch, and the manual does not specify this aspect of the mechanics). Petsasjim1's playthru provides a handy illustration of the game's temporal opacity for today's gamer. <https://www.youtube.com/watch?v=jQ_LgjDO4dQ&t=63s>} But a question that arises is: Who are these games {\em for}? The French child who had perhaps visited Martinique on holiday but knew little of its history? Or the child of Caribbean ancestry who would recognize in the games something of their own heritage? This essay will not try to answer that question definitively, but instead will highlight, through the presentation of my own gameplay, the ways that the gamic narratives change depending on the player's lived personal experience. Leaning upon my previous scholarship, I'll present here my own (perilous, difficult and often failed) journey through the games.\footnote{~See Lauro, {\em Kill the Overseer}. This article also comes out of a piece I was writing on \quotation{Creolizing videogames} as a strategy of antiracism, and I think of these two articles as linked. See Lauro, \quotation{Creolizing Video Games,} forthcoming, in Soraya Murray and TreaAndrea Russworm's edited collection {\em Anti-Racist Futures: Games, Play, and the Speculative Imagination.}}

\subsection[title={Part II. Unwinnable.},reference={part-ii.-unwinnable.}]

On the unwinnability of {\em Freedom}, game blogger Ilmari proffers several options:

\startblockquote
Chances are:

It's a bug

It's an intended feature, meant to teach you cruelly that rebellion never pays

There's something wrong with the copy I've been playing

It's a clever copy protection that will fail you at the very last minute

I've still missed some implicit winning condition\footnote{Ilmari, \quotation{Missed Classic 28.}}
\stopblockquote

It was a great relief to me to find these words, written by a more adept gamer than myself; it means that it's not me and my poor gaming skills that have resulted in me failing the game, over and over again. I have played {\em Freedom: Les guerriers de} {\em l'ombre} in two different languages and through two different platforms over a period of some years. I have played as all four playable protagonists. I have tried tweaking the settings, and (when the game system allowed me to do so) I have opted for different skill levels and side-characters, using the manual's descriptions to try to pick the weakest foes. And yet, I consistently ended up mauled by dogs, captured, whipped, burned alive, cut up, my bones broken with a hammer, with molten lead poured into my wounds.

\placefigure[here]{Screencapture of a Game Over screen}{\externalfigure[issue07/lauro1.png]}


\placefigure[here]{Screencapture of a Game Over screen}{\externalfigure[issue07/lauro2.png]}


\placefigure[here]{Screencapture of a Game Over screen}{\externalfigure[issue07/lauro3.png]}


Yes, it's a ghastly game. It should be acknowledged that the manual claims that the game is winnable: “{\em Le but à atteindre varie avec le type de maître,“}\footnote{\quotation{The goal to be attained varies with the type of master.} Unless otherwise noted, all translations are my own,} and Sepinwall verified that a code for Victory exists in the game files.\footnote{Sepinwall, 222.} The manual states that if you are playing with Pommeraie as the master, for example, it is sufficient to kill all the dogs and flee the plantation, having reduced his crops by 50\letterpercent{} and 25\letterpercent{} of his buildings. For Arnaud de Ronan, you must kill all the dogs and flee after reducing his slaves by 75\letterpercent{}, presumably by enlisting them to join you. In both cases, the manual further states, killing the owner of the plantation resets the game. He can be defeated only by progressively weakening his personnel. The master cannot be killed. Make of that what you will.

Ilmari has suggested that the unwinnability of these versions of the game---which many of us now, 30 years after the game's original release, must access via emulators---may be an ingenious copyright mechanism, presumably meaning that the game is unwinnable only because the software is not running on the original hardware. Regardless of the reason, I personally have made very little headway. I will go into more detail about my failures below, but I first want to suggest a possibility that Ilmari has overlooked, and which I suggested as a larger strategy of game design concerning \quotation{the gamification of slave resistance} in {\em Kill the Overseer!}: the difficulty of the game may be intended not to teach the player \quotation{cruelly that rebellion never pays,} but to safeguard this history from appropriation, to make sure that it eludes the grasp of the gamer, to signify the sacredness of this history that ultimately should not be reduced to an entertainment commodity. From our vantage point in 2021, as Ilmari suggests, it is very difficult to tell where the game's planned difficulty ends, and technological obsolescence begins to have an effect. The emulators are glitchy, and the manuals are hard to find online, all of which we might think of as what Stephanie Boluk and Patrick LeMieux call \quotation{metagaming,} that is, belonging to the text's totality. In the spirit of considering this totality, I have divided this discussion of my own close-playing into two phases: pre-play set-up (which remains part of the world of the gameplay) and in-game play.

{\em Pre-play set up.}

Some might say that the game {\em Freedom} begins with the title card, which reads:

\startblockquote
Night is falling over the prosperous Grand Parnasse plantation. The slaves, under their supervisor's thumb, have left the carts of cane and gone back to their huts. Meanwhile, the owner sips his rum on the veranda and the director rubs his hands; the plantation makes a considerable profit. The accountant finishes his records; there are 200 casks of sugar piled up in the buildings. The supervisor is worried. The lashes of his whip are no longer enough to keep the rhythm in the fields. The wind of rebellion is blowing over the negroes' huts! But who would dare become the leader of a rebellion? (English version via Internet Archive)
\stopblockquote

Yet, it also might be argued that the game begins much earlier. The material reality of the game is the technology that it runs on---and different versions of the game will begin differently, with slightly different loading screens, and different levels of graphics and sound.\footnote{~The game was engineered for Amstrad CPC, Amiga and Atari ST/E. It can be played online at \useURL[url3][http://dos.zone/freedom-rebels-in-the-darkness-1988/]\from[url3] or software can be downloaded, such as from myabandonware.com and run with an emulator. For the Atari software, one can use the Hatari emulator, but it takes some finagling to get it to run, I found. My experience of playing the game was a bit smoother once I ran the software on Hatari, rather than through the websites, so I describe only my play with this specific platform.} Perhaps we might think of this like different editions of a book, which may be prefaced by translator's notes or forewords, ragged covers, or pristine ones, all of which contribute to one's experience. I have used this analogy elsewhere to explain what videogames can do for narrative: imagine encountering a book that refuses to be opened. For those players struggling to find the right emulator upon which to run the program, the game might feel as if it begins when (huzzah!) the loading screen finally appears. In the various versions of the game, this loading screen depicts a nearly naked human of ambiguous gender. The person appears to be genuflecting, but then they break their chains over their head. On Atari, the image is accompanied by drums and set against a lush background featuring huts and fire, and in the lower bit version, there is a simple purple sunset. But in both high- and low-resolution versions, the invitation to gaze upon this naked prostrate body causes discomfort---there is something vaguely inappropriate about rendering this historical material {\em playable}---even if the person's gesture is ultimately one of empowerment.

Regardless of whether the game is played via an emulator or on a platform that allows the player to bypass downloading the game to play online (such as DOSzone), the first true interface screen is a copyright protection. First, the player is asked for a passcode, included with the manual to eliminate piracy: this is a large sheet of color-coded boxes to which the player gains access by telling the program the color of the given box number (such as A123) on the rainbow-colored sheet. Even without this code, the patient player can keep guessing among the provided answers until they turn up the right one; but locating the correct box is a dizzying proposition even with the code sheet. The commercial aspect of the game, the fact that it was an entertainment commodity made available for purchase, is thus the first thing the player understands. Our entry into the game may entail a sense of unease, as we begin to play a game about the commodification of human bodies from within a capitalist system dependent on the legacy of transatlantic slavery. It is only after the player has navigated to the correct color option to bypass the copyright protection that the character selection begins.

At the start, the player is given a choice as to which one of four playable enslaved characters from the Grand Parnasse plantation they want to incarnate---Solitude, Makandal, Delia, or Sechou---and whether to play them as \quotation{Defiant,} \quotation{Rebellious,} or \quotation{Fanatical,} which the manual intimates (but doesn't specify) are akin to easy, medium, or hard difficulty settings for gamers of disparate skill levels. As Sepinwall emphasizes in her discussion of the game in {\em Slave Revolt on Screen}, these characters are based on historical (or symbolic) freedom-fighters from various places in the Caribbean, thereby making Grand Parnasse representative of the whole region and its history.\footnote{~See Sepinwall's chapter on this game for a discussion of the historical allusions among these characters' names---I would add that it is not apparent that the characters are meant to incarnate their historical namesakes; for example, the manual makes no mention of special qualities that we associate with these figures. The game's Makandal, for instance, has no special poisoning skills despite being named for Ayiti's \quotation{Lord of Poison,} and instead the manual divides the character options between \quotation{salt water and freshwater,} those born in the colony versus those recently imported from Africa.}

In the easiest iteration, the game allows you to tinker with the skill settings even further.

My desk is papered with sticky notes that read:

\startblockquote
\quotation{Defiant} Solitude: 76 fire-starting ability; 8 climbing; 16 lock-picking
\stopblockquote

\startblockquote
\quotation{Rebellious} Solitude: 72 fire-starting ability; 8 climbing; 20 lock-picking
\stopblockquote

\startblockquote
\quotation{Fanatical} Solitude: 76 fire-starting ability; 8 climbing; 16 lock-picking
\stopblockquote

And so on, with variations for all four characters. Though some slight difference might be noticeable in how many enslaved people agreed to join me if I played as a character with high charisma (meaning their persuasiveness in rallying non-playable characters to revolt), like Solitude as opposed to Sechou, and I might be more likely to defeat a weak enemy like Bonnetierre, who the manual says \quotation{se bat mal}\footnote{\quotation{fights poorly}} as opposed to the overseer Nelson, described as \quotation{brutal et sevère}\footnote{\quotation{brutal and severe}}, on the whole, I still found the game unwinnable. The selection of levels, characteristics, and non-playable characters (NPCs) had little discernable effect.\footnote{Sometimes, though it did not seem consistently dependent on platform, I could get the system to allow me to make further selections, such as adjusting my chosen characters' skill levels in their other aspects, like charisma, which is necessary to recruit other enslaved persons. In the easiest setting, the game also allows one to select among options for the others on the plantation, making it seem as if the player might choose all the weakest enemies and cakewalk through the game, but alas, it is not so. I studied the manual, chose all of the weakest antagonists and the strongest possible compatriots, thinking that I would surely sail through the gameplay this time. If this had been effective, it would mean that a good part of the gameplay could consist of researching the various characters in the manual and adjusting the percentages for strengths like fire-starting, lock-picking, climbing, and their talents in fortitude and persuasion, as well as selecting personae for the gameplay.} The arbitrariness of the game mechanics heightens the player's frustration in a way that echoes the unfairness of the plantation system.

Further, it might even be said that the manual itself is untrustworthy. The manual states that named characters like Afoukal (sharing the name of the protagonist in Chamoiseau's novel {\em Chronicle of the Seven Sorrows}) or Canope bring special skills to the team, if you can manage to recruit them. Canope, for example, has the ability to poison dogs, which is immensely helpful, as the dogs that rove the plantation often send up the warning and can easily end your mission. Having played the game countless times however, I have never managed to recruit one of these named persons. In an exciting moment, I discovered an NPC in a different place from where I usually encountered him; still, he declined to join me. The manual also claims that the deity Manman Dlo may help a player, but I never spotted her. If this untrustworthiness in the manual were a purposeful orchestration on the game designers' part, it might be read as an ingenious commentary on the injustice of slavery, but it is more likely that, as Ilmari says, the game is exceptionally difficult, especially from the vantage point of contemporary gamers accessing the technology via the interface of glitchy emulators.\footnote{On this point, Tramis's answer to Ilmari is evasive: Q: While I was playing~{\em Freedom}, I couldn't complete the game. The manual mentioned that winning would require killing all guard dogs, but I always failed to kill the last of them. Now, I understand if you have forgotten the minor details about the game, but was there~some~trick to how to~deal~with the~dogs?~And~what~was~the~ending~of~{\em Freedom} like?\crlf
  A: If it~is impossible~to~win~this~is not~intentional~on~our~side.~The~game~certainly lacks fine tuning and it's probably different from one machine to another (Amstrad, Amiga or Atari) because of the difference in speed of the processors.} All of this is to say, whether this difficulty is the result of the techno-temporal opacity or is endemic to the game as designed by Tramis is a moot question: the opacity of the game is a part of the gamic text writ large, and thus is fitting for a game about an attempted slave revolt.

\subsubsection[title={In-game play},reference={in-game-play}]

The player of {\em Freedom} navigates a maze-like layout on the main screen, being careful to avoid the dogs on patrol. The game was made in 1988, so we are closer to Pac-Man level graphics than to the rich cinematics of today's games: on the navigation screen, you are a blinking light moving across the space, amongst the small boxes that represent the buildings. Smaller boxes are the slave huts and larger ones are the various storehouses (where you can choose to commit acts of sabotage, like setting casks of sugar on fire) and other buildings, including the residences of the overseers, the religious officials (one of whom is Père Labat), and the masters.\footnote{~Jean-Baptiste Labat (1663-1738) was a member of the Dominican order stationed in Martinique, who wrote extensively (but uncritically) on the lives of slaves and is credited with modernizing rum production.} When you pilot your dot to a location and select it, it brings up another scene in which you can decide to explore your surroundings, commit acts of sabotage, set fires, encounter adversaries, or try to convince other enslaved persons to join you.

There seems to be no mechanic for killing the dogs one encounters on the plantation's map screen. Presumably, Canope could take them out of the equation if you could stumble upon the algorithm that tempts him to join you, but you can also face off with the dogs by encountering them in a fight screen, which is by no means pretty to look at. Given the centrality of dogs to the game mechanics, it is worthwhile to briefly address Chamoiseau's novel {\em Esclave vieil homme et le molosse} (1997). The novel tells the tale of a very old, enslaved man who flees his plantation one day and struggles mightily to evade his master's monstrous dog. The narrative comes to a head as the dog nearly drowns in quicksand but is ultimately spared; and while the old man is injured and dies, he finds a sacred AmerIndian stone, and something appears to transpire between the man and the dog. At the end, the beast returns to his master profoundly changed into a docile creature, and the dead man's bones are seen laying by the stone, leading the author to conjure---or to channel---this tale. There is a sudden shift from third person to first person upon the discovery of the magic stone engraved by the indigenous peoples of the island, and a sort of spiritual fusion between man and beast occurs. The final chapter is narrated by the author himself, who shares the events that led him to uncover this story and to write the book.

The novel has many of the hallmarks of Chamoiseau's œuvre, including made-up words (\quotation{eleventy-thousand,} 36), genre-crashing, meta-moments (such as the shift in perspective), references to folklore (diablesses, 4; mermaids, 19; zombies, 24, 84; living dead, 32; femmes-zombies, 40); and the weaving together of the history of these \quotation{bitter lands of sugar} (3) and the meaning-making produced by fantastical storytelling about this painful history. On the surface level, {\em Freedom}'s horrible dogs (witness them in action in the embedded clips below) call to mind Chamoiseau's {\em molosse}. But on the level of aesthetics, the game's mechanics, which obstruct and frustrate, also echo the techniques that Chamoiseau espoused in his literary œuvre, particularly in his preference for texts that defy genre and cross the threshold between fiction and reality by blending the author into his characters.\footnote{~For example, see Maeve McCusker, {\em Patrick Chamoiseau: Recovering Memory}, who writes that Chamoiseau's œuvre \quotation{explicitly and consistently engages with the key debates in this highly contested field: the relationship between private and public memory, the borderline between history and fiction, the significance of repressed and traumatic memory, the processes and problematics of autobiographical memory, the relationship between the body and memory, the role of the witness and the archive in storing, transmitting and (de)forming memory, the lure of nostalgia, the commodification of memory} (15). See also Wendy Knepper, {\em Patrick Chamoiseau: A Critical Introduction}, especially for her discussion of Chamoiseau's use of \quotation{masquerade} in character/author hybrids; his \quotation{use of masks can be situated as a response to the \quote{chronological illusion} which has been shaped through and by colonial discourses} (5-6).} In addition to the specific detail of the man-eating dog, a horrifying aspect of the historical reality of the slave plantation, the videogame presents an inscrutability that recalls the opacity of Chamoiseau's literary works. In videogames more broadly, frustration becomes part of the gameplay, and in the best cases, we can read this as the game's productive opacity.\footnote{See Ian Bogost, {\em Persuasive Games} and {\em How to Do Things with Videogames} (20). For a discussion of the particular uses of frustration, see Kiel Gilleade and Alan Dix, \quotation{Using Frustration in the Design of Adaptive Videogames.}} But there are also moments when the limits embedded in this game seem unproductive.

The most problematic aspect of {\em Freedom} is the fact that, although there appear to be dozens of {\em cases nègres}\footnote{\quotation{Negro huts}} on the plantation, there are only two different graphic frames that reappear, over and over. As the player navigates to each abode and selects it, a dialogue box appears, allowing the player to try \quotation{recruiting} the person dwelling therein. Sometimes the person is a fieldhand, sometimes a craftsman, and sometimes a named character who, presumably, would bring special skills to the revolt if ever they could be convinced to become an ally, as outlined in the manual.

\placefigure[here]{Screen captures from the English version of the game showing the face of a man}{\externalfigure[issue07/lauro4.png]}


\placefigure[here]{Screen captures from the English version of the game showing the face of man}{\externalfigure[issue07/lauro5.png]}


The same two male faces are used over and over for the unnamed fieldhand and the craftsman, representing scores of anonymous enslaved persons. Because there is no narrative to the game, it is very hard to feel as if one is advancing---and this is complicated by the fact that the slaves' houses seem to repeat themselves, so that even though, looking at the map, it would seem as if there are hundreds of recruitable persons (and the manual provides numbers---\quotation{Ils sont 204 répartis en 14 cases de 10 esclaves aux champs + 6 cases de 10 ouvriers petits chef + la case du quimboiseur + la case du séancier + la case du leader + un prisonnier})\footnote{\quotation{There are 204 {[}persons{]} in 14 huts of ten field slaves + six huts of ten craftsmen + the {[}separate{]} huts of medicine man, obeah man, leader, and prisoner.}}---in the actual gameplay the same two pictures of slave huts (Fig. 4, 5) repeat infinitely. The fact that the same two images of the slaves' quarters repeat gives the sense that these persons are interchangeable. It might have been better, I humbly suggest, to use a blank avatar like a silhouette, rather than two distinct faces to represent a range of different persons. I realize that this is likely due to the limitations of the technology (and especially to the memory capacity of the 1980s gaming system), but the outcome is unfortunate: the repetition of the same places and the same faces gives one the sense of an indistinguishable horde of the enslaved, apart from the few named characters, who are rarely encountered in your mission.

\placefigure[here]{Screen captures from the English version of the game}{\externalfigure[issue07/lauro6.png]}


Detection points (represented by the butterfly icon on the main navigation screen) signify how dangerously close the player is coming to failing the mission, and these diminish with each encounter one has with other individuals, regardless of the outcome. The player's objective is to try to convince others to join their cause, and yet points are lost each time one visits the {\em cases nègres} to recruit others, whether the attempt is successful or a failure. As such, sometimes I lasted the longest (in terms of game-playing time) when I made no attempt to rally anyone, but instead merely meandered about, committing acts of damage to the plantation's various buildings. This is an important aspect of the game's construction because it speaks to the historical reality of slave resistance and to the difficulty of increasing one's chances of successful revolt by taking on numbers: risk is heightened with each person with whom the plan is shared. As such, the game avoids a potential pitfall: more efforts to recruit result in a loss to one's detection points, but this avoids the sense of running about \quotation{collecting} enslaved persons.\footnote{~See Emil Hammar's discussion of this unfortunate element of the Ubisoft game {\em Freedom Cry} in \quotation{Counter-Hegemonic Commemorative Play.}}

Because the game is not stacked in levels, the player can directly encounter the plantation master, albeit to disastrous effect. Otherwise, they can continue traveling through the landscape, avoiding the man-eating dogs, encountering the same two images of slave huts over and over, and occasionally finding a new enemy to face in hand-to-hand combat or a set of buildings to ransack and destroy. Although one's points diminish more slowly the less the player avoids committing acts of senseless damage (for example, burning down only those buildings that contain things of value, like the stores of sugar), it is arguably more fun to run around the plantation setting everything on fire.

\placefigure[here]{Gameplay from DosZone for Freedom: Rebels in the Darkness, English version.}{\externalfigure[issue07/lauro-vid1.png]}


In summation, {\em Freedom: les guerriers dans l'ombre} is essentially a slave revolt sandbox game. I am not convinced that it can be won, given my own experience with the game and having read others' critiques of it.\footnote{The game's first-generation audience also found fault with the game, see for example {\em Computer + Video Games's} review of {\em Freedom}, preserved here by Amiga Magazine Rack: \useURL[url4][http://amr.abime.net/review_24095]\from[url4] 5} Admittedly, I am a poor gamer: after countless attempts, I have yet to figure out what the climbing function is for, though it is counted as one of the chief skillsets to be valued among the four playable character options. A game designer might say that the architecture of the game is simply inferior to that of {\em Méwilo}, to which we will turn in the next section, though contemporary reviews were mixed.\footnote{~In contrast to the aforementioned review in {\em Computer + Video Games}, see the reviews by Alistair Scott and Nev Astly, both from {\em Amiga Computing}, which have more positive views of the game's graphics and sound.} For example, there is no cursor that returns the player to a specific spot on the map in {\em Freedom}, whereas navigating the hummingbird to the hibiscus flower in the bottom left corner of the screen in {\em Méwilo} allows the player to zoom back out to the map. It is impossible to determine with certainty what is due to temporal opacity (the difficulty of accessing the technology), what is the result of my own poor gaming skills, what is planned obfuscation, and even whether the game's manual is a trustworthy document.

Nonetheless, the inscrutability of {\em Freedom} is not unlike Chamoiseau's own literary inscrutability and, further, this inscrutability befits a game about slave resistance. In {\em Kill the Overseer!} I discussed how scholar and video game designer Ian Bogost's concepts, \quotation{selective interactivity} and \quotation{operationalized weakness,} are useful to addressing the history of slave rebellion.\footnote{~For a discussion of \quotation{selective interactivity} see Ian Bogost, {\em Persuasive Games}, 46; on \quotation{operationalized weakness,} see Ian Bogost, {\em How to Do Things with Videogames}, 20.} I will confess, though, that {\em Freedom} can just feel frustrating in an unproductive manner.

\placefigure[here]{Gameplay from DosZone for Freedom: Rebels in the Darkness, English version.}{\externalfigure[issue07/lauro-vid2.png]}


{\em Freedom}'s chaos and futility, at once in those aspects crafted by Chamoiseau, those designed by Tramis, and those that result from vintage gaming, are all part of the game text's totality. In their combination they not only depict the difficulty of slave resistance, but they also obstruct \quotation{enjoyment} of what is ultimately a sacred history. That being said, {\em Freedom}'s opacity feels somehow less generative than my experience of {\em Méwilo}, a game made the year before {\em Freedom}, to which we turn our attention now.\footnote{Sepinwall, 222.}

\placefigure[here]{Screen captures from Atari, French version}{\externalfigure[issue07/lauro7.png]}


\placefigure[here]{Screen captures from Atari, French version}{\externalfigure[issue07/lauro8.png]}


\subsection[title={Part III. Winnable(?)},reference={part-iii.-winnable}]

\placefigure[here]{Screen capture from {\em Méwilo}, Atari, French version.}{\externalfigure[issue07/lauro9.png]}


{\em Méwilo} serves as a perfect example of a creolized videogame: it foregrounds Martinican history, culture, and folklore; and, as a co-authored videogame, is necessarily multimodal. But not only does the game rely on text, sounds, visuals, and animation, but it also includes a variety of paratextual elements---a manual featuring a story by Patrick Chamoiseau, \quotation{Les derniers jours d'une mulâtresse,} a glossary of select terms, and a recipe. The original packaging supposedly also came with a tape by a band called Malavoi.\footnote{~Salvador, \quotation{Muriel Tramis.}} The game and its packaging constitute a kind of Creole stew, which may be intentional given that the recipe for callaloo will play an important role in the player's path through the game.

Chamoiseau's commitment to opacity is evident in the game's limited invitation to participate in the narrative. {\em Méwilo} is rigidly compartmentalized on a structural level. The player faces multiple challenges, some of which can only be solved through trial-and-error, and others that, in my experience, required seeking the help of others online. Whether the game is more opaque for the retrogamer---obliged to find workarounds in order to get it to load---or easier, thanks to the internet and access to google---is a matter for another place and time. Whichever the case, I found the game's puzzles very difficult to solve.

\placefigure[here]{Screen capture from {\em Méwilo}, Atari, French version.}{\externalfigure[issue07/lauro10.png]}


Even more so than with {\em Freedom}, {\em Méwilo} seems like it should be simple, as the structure is logically plotted from one episode to the next. Nevertheless, the game holds the player at arm's length. Its form accords integrally with the central theme of its narrative: the challenges of putting the ghosts of slavery to rest.

\subsubsection[title={Axiomatic Play},reference={axiomatic-play}]

{\em Méwilo}, like {\em Freedom}, is a videogame \quotation{about} slavery in the Caribbean, and both games use a map as the central home screen. That is where the similarities end. Unlike {\em Freedom}, which is set on a plantation, in {\em Méwilo}, the player steps into the shoes of an unnamed PC, who appears to be a detective with paranormal abilities, in order to solve the case of a ghost that haunts a former plantation.

\placefigure[here]{Gameplay from very early in {\em Méwilo}, German version, Atari system, played on Hatari.}{\externalfigure[issue07/lauro-vid3.png]}


{\em Méwilo}'s stated objective is to calm the restless spirits that literally haunt the residents of the plantation Grand Parnasse. Figuratively, the game grapples with the tensions that exist between descendants of the enslaved, slave-masters, and the cast of so-called mixed-race people. By the time gameplay has ended, the player has been asked to figure out a genealogical map of the various characters they have encountered in order to set free the {\em zombi}, the trapped spirit of a murdered slave, whose ghost guards the master's buried gold.

The opening title cards read: \quotation{Nous sommes le 7 mai 1902, après 15 jours de traversée vous arrivez en vue des cotés verdoyantes de l'île aux fleurs. Dans ce pays de magie, vos pouvoirs paranormaux sont décuplés.}\footnote{\quotation{It is the 7th of May, 1902. After 15 days of crossing you arrive in sight of the green coast of the isle of flowers. In this magic country, your paranormal powers are heightened.}} It then explains how to use the cursor to navigate the map, but even this is put in terms of shapeshifting: \quotation{Pour vous déplacer aisement, vous prenez l'apparence d'un colibri\ldots{}}\footnote{To move yourself easily, you take the form of a hummingbird\ldots{}} Aside from this instruction regarding how to zoom out to the map, little direction is provided, and part of the gameplay involves exploring the space to discover what is or is not interactive.

Many sites on the map, like the Pitt, do nothing until that dimension of the quest has been unlocked. Other sites remain insurmountable until all the other pieces are in place: traveling to Petite Savane, for example, results in the player's instant death because of the presence of a deadly snake. One may die here several times, before finally accepting that it is a dead end. The \quotation{game over} card reads: \quotation{{\em Vous etes transformé en soucougnan, condamné à errer l'éternité durant\ldots{}}}\footnote{You are transformed into a {\em soucougnan}, condemned to wander for eternity\ldots{}} Only once the gamer has obtained a mongoose (successfully bribing it with fruit obtained in a different part of the game) can this landscape be traversed. But that comes much later.

The game thus begins with the illusion of choice. The player thinks they can pilot to any of the many named locations on the map, but soon learns that the game operates by a kind of locked door mechanism: only one door will open at a time, bringing up new content to learn, or a challenge to be solved, at which point another door opens somewhere within the game. Figuring out which door has been activated is part of the gameplay. The map will look the same, but by navigating to the various locations, the player may discover a new element. Suddenly, a character may appear, or a spot that yielded nothing before will now activate when clicked upon.

Until very near the end of the game, clicking on Montagne Pelée brings up only the following warning, \quotation{Le 8 mai 1902 à 8h02, c'est l'enfer: le flanc sud-ouest de la montagne s'ouvre pour vomir une énorme fumée ardente sillonnée d'éclairs. 30 000 morts! St.~Pierre n'existe plus\ldots{}}\footnote{On the 8th of May, 1902 at 8:02 am, it's hell: the southwest flank of the mountain opens to spew forth an enormous cloud of fiery smoke ribboned with light. 30, 000 are dead. St.~Pierre no longer exists.} We might assume that it is the PC's psychic ability that allows us to access this advance knowledge. Chamoiseau's stamp on the game makes for uncommonly vivid videogame copy.

The first door of the game's narrative is opened by visiting the location on the map called Grand Parnasse. Yes, it {\em is} the same name as the plantation in {\em Freedom}, and Arnaud de Ronan, named here as the ancestral master of Grand Parnasse, was one of the potential NPC masters in {\em Freedom}, suggesting that the two games are set on the same plantation.\footnote{This contradicts the idea that {\em Freedom} was meant to be obliquely located somewhere in the Caribbean to suggest the region's shared history, whether the setting be Haiti, or Guadeloupe, rather than concretely situated in any one country, for example, explicitly set in Martinique and on the same plantation as in {\em Méwilo}. The confusion seems to arise from the fact that in interviews Muriel Tramis contradicts what is stated in the manual. Tramis appears to indicate one answer to A. Chang, (148) explicitly stating that the location of {\em Freedom} is Martinque, which Sepinwall (and, in my estimation, the game manual) disproves (Sepinwall 216-7 and manual, first page).} In {\em Méwilo}, much of the backstory concerns a slave revolt, furthering the sense that the two games are connected. At Grand Parnasse, one encounters Monsieur et Madame Hubert-Destouches, the pair who have hired the PC to fix their haunted plantation problem and who provide the important backstory about the haunting.

As one navigates the game, with each conversation leading to a clue as to how to find and attempt the next activated challenge, three things become evident. First, there is ample foreshadowing of the volcano's eruption, with characters noting strange natural phenomena. Second, the characters paint a portrait of political tensions among the various classes inhabiting St.~Pierre and the neighboring plantations, with much strife evident between the various social classes, including the white {\em béké} class, the \quotation{mixed-race} freedmen and women, and the Black enslaved population.\footnote{The tension between the castes, classes, and races (including koulis laborers) is something that I cannot address in this study: someone with a better understanding of Martinique's early 20th century political history should take this up, especially for the game's references to an upcoming election. I see a great part of my contribution here as the invitation to other scholars to investigate these works for their content and the treatment of Caribbean history therein.} My focus will be on the game's productive difficulty, and what that tells us about interpellation of this rich history.\footnote{~The experience of actually getting the games to play via emulators was fairly excruciating for me, so let me provide a kind of a recipe card here. I spent the better part of a day noodling around with different emulators and abandonware downloads. Hatari will run Atari games, for example, but there are other emulators. It took me far longer than perhaps it should have to figure out that I needed to use an emulator that was compatible with Mac OS and that, even then, I had to download a TOS to get it to run properly. After the emulator is downloaded and installed, one will have to get the software from a website repository like myabandonware.com, and then reconfigure some of the settings and image files to get it to load. (I would suggest starting with the Youtube tutorials by Gryphin Jun 12 2020, <https://www.youtube.com/watch?v=eU4cQkVssr8&t=149s> and gwEm's gwEmbassy Jun 6 2020 video <https://www.youtube.com/watch?v=bMNYd398Syk>) If this seems overwhelming, {\em Freedom} is available in a lower-resolution version to play online via DOSzone, \useURL[url5][http://dos.zone/freedom-rebels-in-the-darkness-1988/]\from[url5], and I would suggest that people interested in these games start there. There is also a full playthru of the Amstrad version of {\em Méwilo}, which does have some subtle differences from the Atari version available on Youtube, posted by Amstrad Maniaque, Jan 24, 2018 <https://www.youtube.com/watch?v=YUjuFSrbS1w&t=1s>}

\placefigure[here]{Clip 4. Gameplay from early in {\em Méwilo}, French version, Atari system, played on Hatari.}{\externalfigure[issue07/lauro-vid4.png]}


{\em Méwilo} is decidedly a better game than {\em Freedom}, with more structured scaffolding and a clear and compelling narrative that might interest many scholars of Caribbean history, as well as those familiar with Chamoiseau's work. It is a simple point and click adventure game, very much in the same vein as Roberta Williams' {\em Kings Quest} series (1980 Sierra Entertainment). For example, a {\em loupe} (magnifying glass) on a table in the salon of a plantation house can be touched, but only once the player obtains a mysterious old letter does its utility become clear.\footnote{~This is not in some versions of the game, like Amstrad.} Similarly, only upon learning the story of how the Ronan family, in escaping the slave rebellion of 1831, sought shelter with their neighbors, the Banvilles, can the player navigate to the Banville plantation, Lys Vert. Prior to obtaining this piece of information, clicking on Fond Cacao on the map led only to the player being turned away after viewing an image of a cart in a field.

Thus, the game proceeds with the discovery of various keys that open various doors, giving the game a linear trajectory. It operates on a funnel mechanism, as there is only one path forward. However, this is not to say that gameplay is always smooth, with one step leading logically to the next. There are several hurdles in the game that feel insurmountable, and it is to these we now turn. The game's true variability is the matter of how long it takes the player to solve each challenge, many of which are very difficult, explicitly embracing Chamoiseau's preferred modality, opacity.

\subsubsection[title={Opacity},reference={opacity}]

There are several hurdles in the game that take quite some time to master. Echevin, a man living on the plantation, has information you need about the strange goings-on at Grand Parnasse. But he will only reveal the necessary information if given a bottle of rum.\footnote{~As with {\em Freedom}, I played this game via the Atari version on an emulator. An alternate version of the game, made for Amstrad CPC, which I saw on a YouTube playthru, did not contain the character of Miss Clarisse, whom I encountered in the plantation kitchen and who gave me this clue as to how to loosen Echevin's tongue, so perhaps this was added later to reduce player frustration. See Amstrad Maniaque, posted Jan 24, 2018, Youtube.} To get this rum, the player must go to the distillery and correctly answer eight questions on Caribbean history and culture, including about Martinique's original peoples, its conquest by the French, and various aspects of its language and culture. These questions are hard. Miss even one, and the player has to go off and do some other things before the distiller will restart the quiz. The questions also rotate, so it can take several tries to get past this part of the game.

\placefigure[here]{Fig. 11. Screen capture from {\em Méwilo}, Atari, French version.}{\externalfigure[issue07/lauro11.png]}


The questions posed in the quiz contain colloquialisms unique to Martinique, which makes it difficult to cheat by using Google (I know, I tried.) Within their multiple-choice options, they impart a sense of the weight of the history of slavery, as slavery and resistance haunt various questions and their potential responses. For example, a possible answer for the definition of a \quotation{{\em fer de lance}} is \quotation{a weapon used by a rebel slave,} whereas snake is the correct response. A \quotation{{\em collier de chou}} might be a device used to yoke slaves to each other (which it is not) or a kind of jewelry (which it is) or a type of vegetable (which Google erroneously led me to believe.)\footnote{~I am very grateful to the anonymous reviewers for their comments on my draft, and especially for the comment that a \quotation{collier de chou} would likely be recognized not only by French Antilleans, but by anyone who had taken a tour in Martinique.} Finally, by successfully answering all eight questions given by the distiller, the PC is given a bottle of rum, which can be brought to Papa Echevin. A player from the Caribbean would have a vastly different experience of this part of the game and might sail through the test on the first try; Thereby, the game's cultural opacity functions as a sieve, allowing certain players to pass and holding up others, at least until they learn a few things. Although certain moments in the game feel pedagogical, where clicking on a particular spot brings up extra information, such as the names of common flowers in Martinique (as in the garden at Grand Parnasse) or information about the types of crops (at Fond Cacao), these moments only serve to contrast with the episodes of profound opacity, in which the game refuses to yield to a certain type of player.

Like Timolet in {\em Freedom}, Echevin was modeled as Chamoiseau's doppelganger.\footnote{Sepinwall, 215.} And like the author, the character of Echevin has a way with words. He first states, \quotation{Un rhum comme ca, c'est une clé pour mon cœur, pour ma gorge, pour ma mémoire,}\footnote{\quotation{A rum like this is a {\em key} to my heart, my throat, my memory,} emphasis added.} which admits the broader mechanics of the game that work on the level of closed doors and object or information \quotation{keys.} He then states, \quotation{Il faut haïr les chiens mais reconnaître leurs dents blanches,}\footnote{\quotation{One must hate dogs but recognize that their teeth are white.}} an idiom that, as he swiftly explains, means that despite the whites' many faults, they make good rum. Sepinwall notes from her interviews with Tramis that the games are \quotation{suffused with Creole proverbs} and cites this as an example.\footnote{Sepinwall, 293.} Thus while the game is available only in French and German, Chamoiseau has infused it with his trademark creolizing linguistic play.\footnote{~Many of the author's works explicitly discuss the tension between French and Creole in Martinique. See Glissant, \quotation{A Word Scratcher,} and Coverdale's Afterword to {\em Chronique des sept misères} (1986), in which she writes of Chamoiseau's address of the linguistic crisis of the creole writer: \quotation{the gap between the orality of Creole, a language born---of cruel necessity---in the days of slavery, and French, the language of a colonial power that proudly, ruthlessly, and efficiently discredited almost all Creole culture in the name of {\em la civilization française}} (Coverdale, 213). This theme is very evident in the novel {\em Solibo Magnifique}, in the memoir {\em Chemin d'Ecole}, and in many other places throughout the author's oeuvre. See for example, Rose-Myriam Réjouis, \quotation{Afterword: Sublime Tumble} {\em Solibo the Magnificent} (1988).}

\placefigure[here]{Fig. 12 Screen capture from {\em Méwilo}, Atari, French version.}{\externalfigure[issue07/lauro12.png]}


Echevin gives us another aphorism that is harder to decode: \quotation{Menteur est sans mémoire mais la mémoire n'est pas menteuse,}\footnote{\quotation{The liar is without memory, but memory is not a liar.}} before divulging that the angry spirit plaguing the household is a member of the ancestral family. That decoding everyone's familial relationships is central to the game is already suggested in this early scene (when the point count is 03/99) in Echevin's passing reference to his sister \quotation{Celucie,} whom the player later comes to know as Man Cécé, a woman serving callaloo at the port.

\placefigure[here]{Fig. 13. Screen capture from {\em Méwilo}, Atari, French version.}{\externalfigure[issue07/lauro13.png]}


Another major stumbling block for the player occurs with a visit to Gwanzong the {\em quimboiseur} (defined in the manual's glossary as a \quotation{sorcier craint et respecté} a feared and respected sorcerer), who lives at the top of the mountain that is already spewing ash and convulsing (as is gleaned from conversation with the various characters one encounters). The player is told by Malou at the lycée to bring the {\em quimboiseur} a gift of a black frizzled chicken. The presence of the Pitt on the map (which, as the gamer may learn from the distiller's quiz, signifies a cock-fighting ring) makes the next move obvious, and this location at last becomes active. But the keeper of the cocks is a prankster who says he will only donate a chicken if the player correctly answers one of his mysterious riddles. They are very difficult, even nonsensical, and seem to have been invented by the author. As with the distiller's multiple-choice quiz, there is a large pool from which the questions are drawn, but there is no multiple choice to help here. This time, with a blank to fill in, trial-and-error is not an option. Even if one finally guesses correctly, the cheeky character insists that the player answer a second riddle to prove that the first time was no fluke!

\placefigure[here]{Clip 5. Gameplay from late in {\em Méwilo}, French version, Atari system, played on Hatari.}{\externalfigure[issue07/lauro-vid5.png]}


To be perfectly honest, I would never have beaten the game had I not resorted to reading gamers' blogs for tricks and tips and to watching a playthru of the Amstrad version, which is very nearly the same (Amstrad Maniaque). This business with the chicken riddler was very hard, but I spied some familiar riddles in the playthru on Youtube, which was helpful. Watching someone else play the game does not solve all the problems. Although the sample complete playthru on Youtube posted by Amstrad Maniaque shows the player returning to the cathedral, obtaining a secret letter from Minerve Doussaint by clicking on the book under her arm, I could not get this to activate for me no matter what I tried. Was this a glitch? Was I doing something wrong? Perhaps the iterations of the game were different.\footnote{Ultimately there was no Clarisse to be found in the nonexistent kitchen on Grand Parnasse in the Amstrad playthru, no magnifying glass in the salon, and no theater in St.~Pierre, so the game versions are clearly slightly different. I never did get the theater to activate in the Atari version, and as this was not present in the Amstrad version of the game, I'm not sure what action was meant to happen there, though the theater is mentioned in the story that accompanies the game in the manual, written by Patrick Chamoiseau.}

With a score of 63/99, and having finished more than half of the game, I was ready to call it quits. And then I found a cheat, a blog by a gamer named Oli who gives step-by-step instructions on how to defeat the game.\footnote{A warning: if you use this right off the bat, you will not experience the frustrations that I think are part of the gameplay. It even includes answers to all of the distiller's questions and a few riddles. Best, I think, to resort to this guide only when one gets totally bogged down. <https://jeux.dokokade.net/2015/02/12/soluce-retrocompatible-mewilo-atari-st-amiga-amstrad/>} Oli calls the part of the game where I got blocked \quotation{un petit bug,} but with the help of his explicit instructions, I was able to get past it.\footnote{~In my recent forerunner, {\em Kill the Overseer: The Gamification of Slave Resistance} I discussed the glitch as one among many opportunities for productive opacities within games about slave revolt.} Relying on his instructions the rest of the way, which I followed like a recipe, I completed the game, but the truth is that I absolutely would never have been able to beat it on my own. Even after several more passes through the game, I still remain uncertain as to how the player is meant to deduce the solution to the final problem, in which one can release the trapped spirit of the faithful slave by correctly naming their oldest living descendants. To arrive at the answers, one would have to plot a family tree for each character from the start of the gameplay. Although there are subtle clues as to how the townspersons are related to each other, their lineage is rarely concretely and directly laid out; instead, there are obscure asides made about the mixed-race children of slave-masters, and clues to be decoded in their names.\footnote{Here I am referring to a bit of conversation from Maitre DuChaudé: \quotation{Savez-vous par exemple que certains maitres baptisaient leur rejêton illegitime de l'anagramme de leur nom? C'est pourquoi devant un nom d'esclave, il vous faudra toujours cherchez l'astuce\ldots{}} {[}Did you know, for example, that certain masters baptized their illegitimate children with an anagram of their names? That's why when looking at a slave's name, you should always seek the clue\ldots{}{]}} With this attention to ancestry and genealogies, {\em Méwilo} adopts the opacity of Chamoiseau's griot, who obscures as he reveals,\footnote{Chamoiseau, introduction to {\em Creole Tales,} xiii.} in this case, raising unspoken questions about the abuse of enslaved women. Although I now know the correct answers (thanks to Oli), I still fail to see how one could reasonably arrive at the correct answers. In a stroke of bad luck, even after getting the answers from this cheat online, I spelled one of the names wrong when entering it into the game system, necessitating that I start the whole thing over from the beginning. But then again, I was stumped by the distiller's quiz, so another player may have a different experience: perhaps the game isn't as punishing for all players as it was for me.

In {\em The Dark Side of Game Play: Controversial Issues in Playful Environments}, Miguel Sicart presents \quotation{abusive game design} as fostering \quotation{dark play} to create a specific kind of aesthetic experience. \quotation{{[}A{]}busive game design creates objects that resist appropriation,} he argues, \quotation{and in that resistance they create the space of possibility in which the very point of playing is questioned}.\footnote{Miguel Sicart, \quotation{Darkly Playing Others} in {\em The Dark Side of Game Play: Controversial Issues in Playful Environments}. Torrill Elvira Mortensen, Jonas Linderoth, Ashley ML Brown, eds.~(New York: Routledge, 2015), 103.} Although abusive game design's dark play and what I am describing in {\em Méwilo} both engender experiences of frustration, displeasure, and unhappiness (as opposed to what we expect from \quotation{play}), the kind of punishing game design Sicart describes differs from the modes of opacity deployed in games that depict slavery and revolt. First, abusive game design reflects the designer's effort to initiate a conversation with the player.\footnote{Qtd. in Sicart, 101.} The aesthetics of opacity I seek to map out need not have been implemented willfully, but can also include things like bugs, glitches, misunderstandings, misinformation, and temporal opacity or technological obsolescence. Secondly, abusive game design is understood to be \quotation{highly non-ideological,} apolitical and aesthetic rather than persuasive,\footnote{Sicart, 113.} whereas the use of opacity that interests me as a resistive strategy aims specifically to protect history from becoming satisfying play.

Aside from his physical likeness and the poetic dialogue, Chamoiseau's stamp on the game is evident in the manual. Its inclusion of a short story, \quotation{Les derniers jours d'une mulâtresse}\footnote{\quotation{The final days of a mulatta.}} and of a limited Creole glossary set the scene for the game; the included tale itself evokes similar narrative elements to the game, including the eruption of the volcano and a haunted atmosphere. The story may serve as an introduction for those totally unfamiliar with Martinique, to words like \quotation{{\em colibri},} \quotation{{\em béké},} and \quotation{{\em vetiver},} but it is of no help with respect to any of the game's pivotal challenges. Insofar as the manual and glossary do not actually help with answers needed at several key points in the game, they might be thought of as a clever use of misdirection. Although several of the same locations in the game are referenced in this story, the narrative has no connection to the game's plot. And the text's emphasis of the number 1344 (the number of pages in the diary of the deceased titular character) is repeated several times, in bold, suggesting to the reader/player that this has some significance to one of the game's many puzzles.\footnote{~In the short story, the line \quotation{C'est pourquoi nous gardâmes dans la tête 1344 tandis que les cahiers rejoinaient les objets de Man Kalisa et s'y perdaient dans la même poussière, la même inanité.} \quotation{That's why we keep in mind {[}the number{]} 1344 since the notebooks joined the objects of Man Kalisa {[}the late mulâtresse, victim of the volcano{]} and were lost in the same dust, the same inanity.} Although absent from the Amstrad version, in the Atari game Pélagie St.~Just's parrot squawks a four-digit code that could be activated on Minitel for cheat clues; perhaps there were clues about the recipe and other difficult aspects of the game there, but in the age of the internet, Minitel is long since defunct. The parrot's code, importantly, is not 1344.} It does not.

The manual also states that the callaloo recipe provided therein will be useful to answer a question at some point in the game, but the manual provides a recipe for the way the dish is prepared in Guadeloupe. The player is prompted to tell Man Cécé, who sells her dish at the port, which ingredient is added in Martinican versions of the dish. Here we might choose to read the game as hailing particular players, because those with certain cultural knowledge can pass to a higher level of the game than those without. Because the Guadelopean rather than the Martinican recipe is given in the manual, it is not a question of guessing ingredients from a given list. Rather, the player must seek outside knowledge to determine what should be added. How this would have been possible for players unfamiliar with Caribbean cuisine in the world before the internet is unclear: perhaps they would have sought out a recipe book or availed themselves of the network of game cheats provided in gaming magazines and on Minitel, the pre-cursor to the internet, now lost to us.\footnote{Although a few reviews appear to be noted on Amiga Magazine rack, I have been unable to access these, nor to find game hacks preserved online.} It is certainly difficult to imagine what the expectations of the contemporary player might have been. What is certain is that, as a result of playing the game, the player will have learned (or demonstrated knowledge) about Martinique's history and culture. But many other aspects of the game remain, importantly, beyond the player's grasp.

\subsection[title={Part IV. Conclusion.},reference={part-iv.-conclusion.}]

My study of {\em Méwilo} and {\em Freedom} has been restricted here to a broad analysis of the mechanisms utilized in the games insofar as these are similar to Chamoiseau's literary works. Both of these games make use of opacity in order to safeguard the history of slavery and slave revolt even in this interactive, playable form. Many have asked, should {\em gaming} be the medium for telling this story? Muriel Tramis's response to this fraught question is perhaps the most satisfying:

\startblockquote
Like cinema or comics, videogames must be used to testify or make people think. There are very harsh themes that have been dealt with in cinema such as war or disease, yet cinema is an industry of \quote{entertainment.} The interactivity that exists in videogames offers an additional dimension of involvement for users. \footnote{A. Chang, 156.}
\stopblockquote

Perhaps Tramis is thinking here of the pedagogical benefits of interactivity, but what makes the medium of videogames ideal for staging conversations about slavery and resistance is not only its interactivity, but also the designer's ability to withhold interactivity in structuring insurmountable challenges. Even further, the medium is well-suited to showcase narratives of enslavement and resistance because although the structure of the game may attempt to impose a rigid order, the system itself may defy even the designer---as by glitching, crashing, or presenting other technological issues. When the legacy of historical enslavement and slave revolt becomes fodder for gamers, it is all the more important that we attend to the ways that the form of videogames can work in tandem with the content, deploying opacity in ways that safeguard the subject matter from commercialization, appropriation, and mindless enjoyment.

\thinrule

\page
\subsection{Sarah Juliet Lauro}

Sarah Juliet Lauro is associate professor of hemispheric literature at the University of Tampa. She is the author and editor of many works that address the folkloric figure of the living dead zombie in literature and film, including the article \quotation{A Zombie Manifesto} (boundary2, 2011) and the monograph {\em The Transatlantic Zombie: Slavery, Rebellion, and Living Death} (Rutgers UP, 2015). Her next book project turns from zombies as a figuration of slavery and slave revolt, which is her central interest in the myth, to commemorations of slave rebellion in literature, art, film, and digital culture. On this topic she published {\em Kill the Overseer! The Gamification of Slave Resistance} with the University of Minnesota Press (2020), which is about slave rebellion in videogames, as well as recent articles in {\em TDR: The Drama Review} and {\em History of the Present}.

\stopchapter
\stoptext