\setvariables[article][shortauthor={Glover, Gil}, date={May 2023}, issue={7}, DOI={Upcoming}]

\setupinteraction[title={},author={Kaiama L. Glover, Alex Gil}, date={May 2023}, subtitle={}]
\environment env_journal


\starttext


\startchapter[title={}
, marking={}
, bookmark={}]


\startlines
{\bf
Kaiama L. Glover
Alex Gil
}
\stoplines


It seems as good a time as any to say---to ask---the quiet part out loud. {\em Why do this digital}? It's the question that hums steady in the background of all the work we welcome into our journal. It's the question we ask ourselves, our collaborators, and our students at all stages of the projects we take on. And it's the ground zero question for the contributors to every issue of {\em archipelagos} we publish. There's no need to rehash at too great length what anyone who has landed on this page already knows: notably, that the digital is not salvation---that it is as vulnerable to the hierarchies, bigotries, and biases of the colonial order and its afterlives as are the analog spaces that so many of us have long been working to upend. Given this---given that to engage with the digital in the contexts of our research and teaching requires us to navigate not only the landmines of modern history, but also those presented by tools and technologies that are changing at an unprecedentedly breathless pace---we must always be asking: what do we get for our digital trouble?

Humanists before all else, we would venture to say that the \quotation{get} is more questions---old ones asked and answered in new ways, new ones surfaced through the ones and zeros of our code. We take nothing for granted in our digital praxis, and our seventh issue makes that premise explicit. The work featured here enjoins us to embrace a Black Digital Practice that is, as our contributors Landels et al.~remind us, \quotation{invested in process rather than in product.} Landels et al.~lay out the specific ways they have worked to craft small stories from big data in their extraordinary cartographic project, \quotation{Remembering the Middle Passage.} Working with technology that invites the amassing and comprehending of information on the most sweeping of scales, Landels et al.~propose, almost counterintuitively, that we narrow our focus. Their work is an eloquent and compelling call to \quotation{disaggregate,} to \quotation{remix,} and to \quotation{remediate} the grand narratives produced by the increasingly massive stores of information at our disposal. Faced with the vast expanse of the data forest, we must aim, they insist, to bring not just the trees, but individual branches, into better view.

The work of making colonial tools engender different stories and chart otherwise journeys is, ultimately, a refusal of comprehension as capture. This practice of fugitivity animates Isabel Bradley's two contributions to this issue---a descriptive process essay and project peer review of the {\em Monograph of Haiti Map}. Developed in collaboration with historian Laurent Dubois and geographer Nicolas Scheffer, this cartographic project aims to refashion (to disaggregate, to remix, to remediate) aerial maps of Haiti generated by the US marines during their nearly twenty-year occupation of the country. As they repurpose imperial instruments designed for the surveillance and control of Haitian bodies and territories, Bradley, Dubois, and Scheffer work to resignify and render opaque spaces that US Empire sought, and literally fought, to make transparent. As is evidenced in her careful relating of the project's conceptualization and execution, their digital remapping deploys today's technologies to undermine technologies of the past, denying the map's original intentions and pointing toward fugitive pathways that never were nor could be fully captured/comprehended/apprehended by colonial power.

Neither Sussman and Landels nor Bradley and her team are kidding themselves, of course. They are well aware that attempting to redeem the proverbial master's tools or to turn those tools to disorderly use always risks backfiring. Sarah Juliet Lauro and Kris Singh address this danger in their examinations, respectively, of gaming technology and social media. Both contributors query the domain of the ludic and, as such, are compelled to consider the fundamental {\em appropriateness} of playing in and with technologies of amusement that more and less directly implicate the traumas of the Caribbean past and present. Lauro's \quotation{close-playing} of two slavery video-games is an exemplary instance of what it means to infuse technological experiences with rigorous humanist inquiry. She asks---and answers, ultimately, in the affirmative---whether \quotation{playing} enslavement can be anything other than problematic. Implicating Glissantian opacity, as is so often our Caribbeanist wont, Lauro suggests that the frustrations and obstacles of the game-play serve a mimetic purpose. The gamer is made to confront the impossibility of fully escaping bondage within a totalizing (colonial-cum-technological) system designed for domination and constraint.

Singh, for his part, is less sold on the pedagogical value of alternative media when it comes to Caribbean fugitivity. He asks---and answers, ultimately, in the negative---whether the showcasing of Caribbean speech patterns on platforms like TikTok de facto facilitates decolonial relationships and practices. He is concerned that social-cum-capital preoccupations encourage creators to trade irony for \quotation{funniness,} diminishing the legibility of Caribbean subjects' material problems---the wrong sort of opacity, as it were. Can we, he asks, even really speak of opacity or decoloniality within the consumerist, surveillance-based algorithmic economy of social media? Unconvinced of such a possibility, Singh turns to what he argues is the more grounded, analog space of Andrew Salkey's illustrated novel {\em Hurricane} as an exemplar of a constructively mediated Caribbean vernacular.

Storms are at the center of Elise Foote, Winnie E. Pérez Martínez, and Charlotte Rogers's digital exhibition {\em Coasts in Crisis: Caribbean Arts and Cultures after Hurricanes}, featured in the \quotation{Digital Projects} section of this issue. {\em Coasts in Crisis} proposes a mediation of disaster through the vernacular of visual art, poetry, and music by artists who experienced Hurricanes Irma and Maria in 2017. Explicit connections between art and social justice activism, fostered across national borders, are made possible by digital mediation: spiralic Caribbean geographies become epistemic frames; form and content are placed in provocative dialogue; art is made accessible beyond the confines of conventional museum and gallery spaces; technology is thoughtfully mobilized to create and sustain \quotation{IRL} community. Similarly, Alexandra P. Gelbard's {\em Cabildo de Regla Digital Archive} is a project immersed in the Caribbean real. Stepping into the space of cultural preservation, in close collaboration with a specific group of Cuban community organizers and practitioners of the Lucumí religion, Gelbard's project documents, curates, and amplifies localized responses to powerful forces that determine the material realities of Caribbean subjecthood. The project presents a digital archival practice of memory and knowledge transmission that has emerged from and serves a community struggling for legibility on its own terms.

So, why do these things digital? To answer that question, the people and projects featured in these pages focus expressly on the how. They proceed cautiously, never presuming an inherent decolonial praxis without ascertaining, with maximal clarity, the intention of creators, the context of creation, and the reception of consumer-users. They begin with the premise that redemption and resistance are elusive when it comes to representing our Caribbean through the prism of technology. This does not stop them from trying to understand why that is the case, or dreaming up new freedom architectures around the impasses. For our authors and teams, the digital is not beyond interpretation, because ultimately it is not disconnected from our struggles for justice and joy. Neither is the digital a conquering armada from Silicon Valley that we simply accept as if it were the force of gravity. To the contrary, the digital continues to be, for them and for us, a key strategic terrain where our historically-conscious, critically-minded, and theoretically-inflected hermeneutics have the potential to transmute the very object of our study, and with it our all-too-real fates. Our digital will be humanist or will not be at all.

Onward,\crlf
Kaiama and Alex\crlf
Editors

\page
\subsection{Kaiama L. Glover}

\goto{Kaiama L. Glover}[url(https://barnard.edu/profiles/kaiama-l-glover)] is Associate Professor of French and Africana Studies at Barnard College, Columbia University. She is the author of \goto{Haiti Unbound: A Spiralist Challenge to the Postcolonial Canon}[url(http://liverpooluniversitypress.co.uk/products/61903)] (Liverpool UP 2010), first editor of \goto{Marie Vieux Chauvet: Paradoxes of the Postcolonial Feminine}[url(http://yalebooks.com/book/9780300214192/yale-french-studies-number-128)] (Yale French Studies 2016), and translator of Frankétienne's {\em Ready to Burst} (Archipelago Books 2014). She has received awards and fellowships from the National Endowment for the Humanities, the Mellon Foundation, and the Fulbright Foundation. Current projects include forthcoming translations of Marie Vieux Chauvet's {\em Dance on the Volcano} (Archipelago Books) and René Depestre's {\em Hadriana in All My Dreams} (Akashic Books), and the multimedia platform {\em In the Same Boats: Toward an Afro-Atlantic Visual Cartography}.

\subsection{Alex Gil}

\goto{Alex Gil}[url(http://www.elotroalex.com/)] is Senior Lecturer II and Associate Research Faculty of Digital Humanities in the Department of Spanish and Portuguese at Yale University, where he teaches introductory and advanced courses in digital humanities, and runs project-based learning and collective research initiatives. His research interests include Caribbean culture and history, digital humanities and technology design for different infrastructural and socio-economic environments, and the ownership and material extent of the cultural and scholarly record. He is currently co-organizer of The Caribbean Digital annual conference, and co-principal investigator of the Caribbean Digital Scholarship Collective, funded by the Andrew W. Mellon foundation.

\stopchapter
\stoptext