\setvariables[article][shortauthor={Bradley, Dubois}, date={May 2023}, issue={7}, DOI={Upcoming}]

\setupinteraction[title={Monograph of Haiti Map},author={Isabel Bradley, Laurent Dubois}, date={May 2023}, subtitle={Monograph of Haiti Map}]
\environment env_journal


\starttext


\startchapter[title={Monograph of Haiti Map}
, marking={Monograph of Haiti Map}
, bookmark={Monograph of Haiti Map}]


\startlines
{\bf
Isabel Bradley
Laurent Dubois
}
\stoplines


\startsectionlevel[title={{\em archipelagos} presents the {\em Monograph of Haiti Map}},reference={archipelagos-presents-the-monograph-of-haiti-map}]

As a result of foreign military surveillance, wartime strategies to control national governance, and global antioccupation movements, the data created about Haiti during the US occupation period is one of the deepest resource pools for interrogating the historical meanings and afterlives of the nineteen-year occupation. Pairing the aerial images taken by the US Marine Corps in 1932 with contemporary satellite images the {\em Monograph of Haiti Map} offers a cartographic entry point for interrogating one of the recurring lines of inquiry in Haitian Studies regarding the immediate and ultimate long-term outcomes of the occupation. From the description and data presented, the project carefully and critically bridges a historical, spatial, geographic, and technological line between the occupation period and current day narratives and socio-spatial renderings of Haiti.

\stopsectionlevel

\startsectionlevel[title={{\em archipelagos} review of the {\em Monograph of Haiti Map}},reference={archipelagos-review-of-the-monograph-of-haiti-map}]

\startsectionlevel[title={Contribution:},reference={contribution}]

As the project statement explains, \quotation{There can be many possible uses for this map, including studying large scale patterns of land use, deforestation, urbanization, and other environmental trends.} The \quotation{bird's eye view} of the occupation also opens new lines of inquiry about surveillance, the ecological footprint of the occupation, and the everyday lives of Haitians during the occupation that are articulated in other older and recent disciplines and works---geography and sociology (George Anglade), literature (Nadève Ménard), and women's and gender history (Grace Sanders Johnson)--- but do not have an articulated digital orientation. Additionally, as a catalogue and curation of various forms of data, the project will also be a useful resource for those thinking about the relationship between environmental humanities, earth sciences, chemistry, and the postcolonial (Vanessa Agard-Jones).

There is an effect of surveillance and violence that lingers in this digital work. As the authors rightfully explain, this is the result of \quotation{the conditions and effects of {[}these maps'{]} original creation.} While the authors attend to the images as \quotation{a key part of strategic cataloging of information from {[}the{]} landscape as part of a violent military occupation,} the authors may also consider how the current day satellite images are constructed. What is the relationship between these respective technologies of surveillance? What are the implications of power between these images? As it stands, the implication is that the contemporary images are banal archival captures. Considering the preservation of the project and future purposes, this is a conversation worth considering. What is the relationship or discourse between these texts? And how could attention to this relationship help the authors answer their question: \quotation{How can we use this map to consider historical patterns regarding struggles for sovereignty in Haiti?} In order to guide the user through some of the trajectories of their own thinking with this question, the authors might consider other projects in the Caribbean that engage US aerial surveillance archives and the relationship between foreign intervention and sovereignty across colonial and postcolonial moments. One example is Deborah Thomas's work on Jamaica (\goto{see footage from 0:01-0:49}[url(https://vimeo.com/ondemand/fourdaysinmay)]).

While the occupation maps sync up with the contemporary satellite maps in extraordinary and haunting ways, the authors may want to contend with the near-century-long gap in the technology between the {\em Monograph of Haiti} images and the satellite images that the authors use in the project. How do the authors account for the time and other social, political, and environmental impacts on the topography and general ecology of Haiti that occurred in the decades after the military occupation and before the recent imaging? Are these other influences on the landscape of Haiti informed by the resonances of the occupation? If so, how? There are not, for example, any additional sets of images at a different time interval to establish a type of time-lapse visual between 1932 and the contemporary satellite images. I believe locating such images, if they exist, would be beyond the scope of this project, but it would be important for the authors to attend to this chronological and technological gap in their argument and project framing. Is there a relationship between these systems and technologies of retrieval and viewing that need to be named and---by extension, as the authors have invited---redressed and reimagined?

The methods used in the project are appropriate to the research questions it poses. The various choices for viewing the archival and current day landscapes are interesting and aesthetically evoke questions about authorship, refusal, and recreation that the authors communicate in their statement. I found the \quotation{Base Map Gallery} options particularly compelling. For example, the \quotation{imagery hybrid,} \quotation{topographic,} and \quotation{street view} maps each yielded different kinds of questions about utility and genre. The authors might consider crafting some project-framing questions that speak to the viewer's use of the satellite digital elements. The authors might also consider their use of metadata as a way to engage the context of the original gathering and cataloguing of the documents. Along these lines, the significance of the images being originally bound together, then unbound and digitized, and then digitally reconfigured in relation to one another could be more explicitly engaged. This project is step in the right direction in the work involved in digitizing historical documents. Foregrounding the project within this digital genealogy could also be a way to establish the project's preservation and future.

A further engagement with the genealogy of the digital aspects of this project may also be an area where the authors might consider using the written text of the manuscript to \quotation{re-map} and \quotation{re-interpret} the images. As the authors establish, in addition to the detail in these maps, \quotation{Haitians were dwelling in them, making their lives in them, through complex mobilities and collaborations with the environment.} Having worked with the {\em Monograph of Haiti}, focusing on the texts and creating maps from the prose, I would invite the authors to consider using the prose to create new ledgers. For example, the authors could create a ledger that identifies use of particular roads and thereby better accounts for Haitians' \quotation{presence and place-making.} In the maps presented in the project, a passable road is indistinguishable from a non-passable road, but the prose accompanying the images in the monograph explicitly names roads as \quotation{too narrow for vehicular traffic,} \quotation{passable for horses at all times,} or \quotation{no more than an ordinary trail.} A \quotation{passability} ledger, for example, would offer topographical and cartographic indicators of Haitians moving and inhabiting the space despite the inability to see them and often in contradiction to the desired outcomes of the occupiers' data collection. That is, \quotation{no more than an ordinary trail} is evidence of Haitians moving in spaces for their own purposes that had not, as of 1932, been invaded by occupation road or bridge projects. Read alongside the contemporary technology of the satellite maps, users would be able to see if those roads are still functioning and in what ways they were being used in \quotation{real} time. This suggestion may also attend to an additional outstanding question about the big data and metadata quality of the project. Might it be possible for the authors to intervene in \quotation{a project that sought to mobilize geography in order to make war on Haitians and their sovereignty} by rearticulating the project itself as metadata or by using the metadata as a site for redress? Considering the second half of this proposition, the authors might think about ways to intervene in the form of the \quotation{metadata} section under the viewer options.

\stopsectionlevel

\startsectionlevel[title={Design},reference={design}]

The overlay aspect of the maps is fascinating and as stated above truly communicates the authors' intent and care to communicate the overlapping cartographies of Haitian history. At the same time, a feature that would allow the viewer to see the areas side-by-side would be useful for the purposes of comparison that the authors encourage in their statement. It seems that the \quotation{swipe} feature might do this work, but the split screen that currently appears makes it unclear as to what should happen between the two screens.

The opening \quotation{mosaic-like} map is stunning and aesthetically communicates the violence of the occupation's surveillance project as well as the possibility within this same archive. Additionally, the \quotation{fragmented} components invite the user to think about the gaps and strategic choices of the archive while visually encouraging the user to see something different from the outset. The evolving silhouette quality recalls other historical renderings of Haitian geography (one of many examples being the cover of Anglade's {\em L'espace haïtien}). Along these lines, the actual presentation of the \quotation{mosaic-like} map can be strengthened with a layout that does not wash out the image with a white background. It took several zooms in and out to actually understand what the \quotation{dots} were establishing. This image should be a set size for visual continuity with the option to zoom in via embedded hyperlinks/clicks to each map rather than only via the city list on the left. The list on the left is an important navigational tool but, for a project that attends to geography, the digital space between the list, the maps, and the overlayed images is slightly disorienting. For example, if a user does not know where Anse à Galets is, clicking on the name takes you to a zoomed in location that does not offer any spatial orientation to the larger Haitian map without zooming out and \quotation{losing your place} on the 1932 map. In the user experience for this project, it is important to consider the ways the actual navigability \quotation{amplif{[}ies{]} the martial view of the landscape.} The authors might consider prioritizing relationships between the maps to trouble the singularity and surveillance quality of the images. For example, if you go to \quotation{Croix des Bouquets} when you zoom out on the map view there are no other military maps to compare. That is, you cannot then see the military aerial map of Port-au-Prince 1 or 2 at the same time. I use this as an example because these maps would be next to one another and of possible interest for comparison. Along with the list of cities, it may be useful to have the opening map be more dynamic, so that when you zoom into the \quotation{mosaic-like} map you can also click on the images and go to the \quotation{map viewer,} \quotation{open in screen viewer,} or \quotation{metadata} from there.

\stopsectionlevel

\startsectionlevel[title={Credit},reference={credit}]

Yes, the project credits its contributors openly and fairly. It is unclear who did what work with the GIS software for the project and what kind of labor commitment that entailed. Are Keener, Dubois, Bradley, and Scheffer all doing the digital formatting? If so, what is each person contributing? If not, can the authors be more explicit about their particular roles (ex. conceptual, theoretical, research, funding, coding, and design)? The work cites its (digital scholarship) precedents and data sources appropriately. The information that is included in the hyperlink about the monograph before and after digitization (including the images) could be incorporated into the project to avoid having to leave the project page to get this digital context.

\stopsectionlevel

\startsectionlevel[title={Preservation},reference={preservation}]

The questions that the authors pose in the project suggest that the authors have thought about the future of the project and are enthusiastic about its possibilities. Also see earlier suggestions about metadata.

Finally, the bibliography provided by the authors is an additional important framing that strengthens the argument of the project. The authors may also consider:

Giorgia Lupi, “\goto{Data Humanism: My Manifesto for a New Data World}[url(https://giorgialupi.com/data-humanism-my-manifesto-for-a-new-data-wold)]

Nydia A. Swaby, \quotation{Archival Experiments, Notes and (Dis)orientations,} {\em Feminist Review}, Issue 125 (2020): 4-16.

Ekman, Ulrik, Daniela Agastinho, et. al, {\em The Uncertain Image}, 2019.

\stopsectionlevel

\stopsectionlevel

\startsectionlevel[title={Response from Isabel Bradley and Laurent Dubois, site authors of~{\em Monograph of Haiti Map}},reference={response-from-isabel-bradley-and-laurent-dubois-site-authors-of-monograph-of-haiti-map}]

We are tremendously grateful for these thoughtful, rich, and inspiring responses to our project. To have someone who is clearly deeply anchored in and familiar with the historical context of early 20th century Haiti, and even specifically with the {\em Monograph of Haiti}, engage so deeply with our project is an honor. We appreciated the encouragement the reviewer provided, confirming our sense that there are exciting possibilities for this project and that it has the potential to nourish exploration and reflection about Haiti's geography and history, as well as about larger questions around mapping and surveillance. We are also extremely grateful for, and inspired by, the probing questions and suggestions for how we might expand and improve the project, both in terms of how users might interface with it and how we might enrich the theoretical questions about space, surveillance, and mapping that it dialogues with. As we read the review, the question about the kind of labor that was involved also led us to reflect on the way our process led to the particular form the project has at this point, with its possibilities and limitations.

We depended heavily on the expertise of geospatial analyst Drew Keener and more broadly on the infrastructure put in place by Duke University libraries to support mapping work with GIS software. Our project would have been very difficult, perhaps impossible, to pursue with that support, as Keener is the one who identified the software and taught us how to use it, and Duke also offered hosting for the site. As a geographer, Nick Scheffer did enter the project with greater sophistication about GIS software and mapping more broadly, and that expertise was invaluable.

Once we were trained and situated in the use of the software, however, all three of us did a great deal of georeferencing, poring over the images and finding coordinating points between the satellite imagery and the photographs. We developed and shared techniques---using church and government buildings and key roads and crossroads, for instance---as we did so. In the process, we had regular discussions about the theoretical and methodological questions being raised by the project. We feel that the combination of very focused labor around the point-to-point connecting of photographs and satellite imagery with the larger theoretical reflections made for a richer experience in both directions.

But the reviewer raises questions---particularly about the fact that our project took images that were \quotation{originally bound together, unbound and digitized, and then digitally reconfigured in relation to one another}---that would be worth dwelling on a bit more in our presentation of the site. This all also suggests that perhaps a more explicit description of our process and what we learned from it could be a useful addition to some of the paratext of the site.

We welcome the reviewer's call to more explicitly discuss satellite and surveillance technology in relation to Haitian sovereignty in the present moment. We definitely do not want the implication to be that the contemporary satellite images are \quotation{banal archival captures.} On the contrary, we ultimately hope the project will effectively foreground the continuities between these two forms of apprehending and representing space (aerial photographs and satellite images). Perhaps in the text framing the project there should be more discussion of how colonial and postcolonial strategies of imperial surveillance have evolved into, for example, \goto{toolkits for immigration enforcement}[url(https://www.bostonreview.net/articles/ryan-fontanilla-immigration-enforcement-and-afterlife-slave-ship/)]. We also will add Deborah Thomas's film, \quotation{\goto{Four Days in May}[url(https://vimeo.com/ondemand/fourdaysinmay)],} to the \quotation{Further Reading} section of the project website.

We are also taken with the reviewer's call for us to contend with the \quotation{the near-century -long gap in the technology between the {\em Monograph of Haiti} images and the satellite images that the authors use in the project.} There is, we feel, a longer-term possibility that this project might be the foundation for a fuller history of mapping of Haiti, perhaps going back to the 16th and 17th century and into a more multiple set of 20th century envisionings of the landscape. This would have the advantage of really highlighting the fact of the constructedness of all forms of geographic visualization. At the same time, the inclusion of different types of landscape imaging would also raise new questions in each case, and would require a more robust and complex interface. We still need to do further research on when the next set of aerial images of Haiti were made, and where we might locate those. Potentially, there might be a third set of images that could be placed between the 1930 images and the contemporary satellite imagery.

Throughout our work on the project, we have discussed and pondered whether there might be ways to include very different types of visualization, such as Haitian paintings of landscapes or excerpts from works of literature (for instance, the works of Marie Vieux-Chauvet) as a kind of counterpoint to the forms of surveillance imagery showcased here. Though we could not do this for all the locations, perhaps there would be a way of creating clusters of imagery around certain places---Jacmel comes to mind as one powerful potential, given the profusion of representations of the city---as a way of disrupting and questioning other aspects of the project. The idea of creating some kind of dialogue like this within the project is very exciting, if also slightly daunting for the design questions it raises.

One immediate way of adding new layers of complexity to our readings of the landscape is the brilliant proposal of creating a \quotation{passability} ledger based on the prose notations in the {\em Monograph} itself that, as the reviewer notes, name \quotation{roads as \quote{too narrow for vehicular traffic,} \quote{passable for horses at all times,} or \quote{no more than an ordinary trail}.} This aligns with the project's goal of revealing resistant, anti-occupation geographic practices because, as the reviewer writes, it \quotation{would offer topographical and cartographic indicators of Haitians moving and inhabiting the space despite the inability to see them and often in contradiction to the desired outcomes of the occupiers' data collection.} We were really taken with this suggestion, which also made us realize that once we had pulled the images out of the {\em Monograph}, we hadn't sufficiently returned to the text. Now we have a guide for doing so. A fairly simple next step could include annotating roads, trails, and passages as the reviewer suggests, or even including further metadata (for example, the location of hubs of resistance, identified as \quotation{bandit areas} in the {\em Monograph}). This would serve to articulate some of the more implicit arguments of the photos, while serving to contextualize and \quotation{engage the context of the original gathering and cataloging of the documents.}

We appreciated the questions raised about the navigability of the map, which we agree is a problem in the current version of the project. While we remain invested in the idea of presenting a \quotation{mosaic-like} map, and appreciated the positive reading of this offered by the reviewer, we definitely feel the need to then allow for a better set of ways to move through this landscape. We also will explore the suggestion that we might consider a different background that doesn't \quotation{washout} these images---perhaps gray or blue---to change the initial engagement with the mosaic. And we really like the idea of \quotation{prioritizing relationships between the maps to trouble the singularity and surveillance quality of the images,} and thinking through what kinds of comparisons might be most useful for this.

Our feeling at this stage is that, thanks to these suggestions and questions, we will be in a position to start a new round of experimentation with the site to explore different design and presentation optics and to see which of these offer a response to the reviewer's excellent analysis and critique. We will also try to balance the need for new kinds of expository and theoretical reflections with our original commitment to have the site be visually focused as an experience. In this sense, our ultimate goal will be to see if we can really embody the theoretical questionings that we might lay out in textual format in a better interactive design that invites reflection through the use of the site.

In moving to the next stages of the project, we will need to find new collaborators who can help us work through these developments of the project. We feel that the reviewer here has already become a valued collaborator through this process, co-thinking with us in a shared commitment to offering new ways of seeing and knowing Haiti.

\stopsectionlevel

\page
\subsection{Isabel Bradley}

Isabel Bradley is a PhD candidate in Romance Studies at Duke University. Her work is broadly framed by currents of decolonial thought and their intersections with environmental studies in the Francophone Caribbean. She studies the ways in which modes of being, sensory perception, and historicity emerge from embodied engagements with ecologies such as subsistence plots, plantation monocultures, mornes, and oceans. Grounded in French-language natural historical texts, visual and cartographic materials, and Caribbean literatures, her dissertation project traces the role of the manioc root in sustaining relational, counter-plantation modes of being human from the 16th century to the present. Isabel has held research fellowships at the John Carter Brown Library and the Library Company of Philadelphia and spent a visiting semester at the École normale supérieure in Paris.

\subsection{Laurent Dubois}

Laurent Dubois is John L. Nau III Bicentennial Professor of the History & Principles of Democracy and the Academic Director of the Karsh Institute of Democracy at the University of Virginia.~ From 2007 to 2020, he was Professor of Romance Studies & History at Duke University, where he co-directed the Haiti Laboratory from 2010-13 and then founded and directed the Forum for Scholars & Publics. He has written about the Age of Revolution in the Caribbean, with~{\em Avengers of the New World: The Story of the Haitian Revolution~}(2004) and~{\em A Colony of Citizens: Revolution and Slave Emancipation in the French Caribbean, 1787-1804}~(2004), winner of the Frederick Douglass Book Prize. His 2012~{\em Haiti: The Aftershocks of History}~was a New York Times Notable Book of the Year. He has also written about the politics of soccer, with~{\em Soccer Empire: The World Cup and the Future of France}~(2010) and~{\em The Language of the Game: How to Understand Soccer}~(2018). His work on the cultural history of music,~{\em The Banjo: America's African Instrument}~(2016), was supported by a Guggenheim Fellowship, a National Humanities Fellowship, and a Mellon New Directions Fellowship. His most recent book is~{\em Freedom Roots: Histories from the Caribbean~}(University of North Carolina Press, 2019), co-authored with Richard Turits. He has also translated works by Jean Casimir, Achille Mbembe and Lilian Thuram into English. His writings on music, history and sport have appeared in~{\em The Atlantic, The Nation},~{\em The New Republic},~{\em The New Yorker},~and~{\em The New York Times}. He is currently writing a history of the French Atlantic, under contract with Basic Books, tentatively titled~{\em Seven Rivers & a Sea}.

\stopchapter
\stoptext