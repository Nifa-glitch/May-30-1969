\setvariables[article][shortauthor={Landels, Bradley, Desir, {\em et. al.}}, date={May 2023}, issue={7}, DOI={10.7916/archipelagos-aszw-fy69}]

\setupinteraction[title={“Died a small boy”: Re-Centering the Human in Geospatial Data from the Middle Passage},author={Tye Landels, Isabel Bradley, Kelsey Desir, Grant Glass, Jane Harwell, Anya Lewis Meeks, Charlotte Sussman}, date={May 2023}, subtitle={“Died a small boy”}, state=start, color=black, style=\tf]
\environment env_journal


\starttext


\startchapter[title={\quotation{Died a small boy}: Re-Centering the Human in Geospatial Data from the Middle Passage}
, marking={\quotation{Died a small boy}}
, bookmark={“Died a small boy”: Re-Centering the Human in Geospatial Data from the Middle Passage}]


\startlines
{\bf
Tye Landels
Isabel Bradley
Kelsey Desir
Grant Glass
Jane Harwell
Anya Lewis Meeks
Charlotte Sussman
}
\stoplines


{\startnarrower\it \quotation{Remembering the Middle Passage} is a collective of scholars using digital humanities tools to represent, and possibly memorialize, enslaved Africans who died along the Middle Passage. Taking up Jessica Marie Johnson's call for a \quotation{Black digital practice,} or methods that \quotation{challenge reproduction of black death and commodification, countering the presumed neutrality of the digital} (Johnson, 2018), our project seeks to map discrete deaths in a way that centers personal subjectivities and the communal bonds severed by these deaths. Our early efforts, as seen on our website, used ArcGIS technology to chart the 1757 slaving voyage of the {\em Good Hope}, marking places at sea where individuals died, as deduced from the geocoordinates in the ship's logbook. In this article, we reflect on these initial attempts at mapping, critiquing our reliance on presumably neutral geospatial technologies, finding them to be rooted in persisting colonial cartographic practices. Despite our intentions to disaggregate archival data and to re-center, through our design choices, the human lives lost, mapping proved a blunt instrument for our purposes. On our website, we made use of critical fabulation, as theorized by Saidiya Hartman, to trouble the flatness of the map and to use the subjunctive mood to imagine the lives obscured by coordinates. In our article, we explore the consequences of using critical fabulation in the digital realm. Whereas previous examples of critical fabulation, including Hartman's own, have engaged with print archives such as legal records and newspaper reports, we ask in this article what it means to \quotation{digitize} the toolset of critical fabulation, applying its methods, in particular, to geospatial data. Bringing critical fabulation into the digital archive opens up new possibilities for both critical fabulation and Black digital practice, enabling us to better grapple with mundane, everyday, and otherwise unpublicized (and therefore unprinted) violence in the archive of slavery.

 \stopnarrower}

\blank[2*line]
\blackrule[width=\textwidth,height=.01pt]
\blank[2*line]

\startblockquote
Where are your monuments, your battles, martyrs?Where is your tribal memory? Sirs,in that grey vault. The sea. The seahas locked them up. The sea is History.
\stopblockquote 

\startalignment[flushright]
\tfx{Derek Walcott, "The Sea is History," ll. 1-4}
\stopalignment
\blank[2*line]


\subsection[title={1. Introduction: Measurements, Maps, and Memory at Sea},reference={introduction-measurements-maps-and-memory-at-sea}]

At approximately 3:00 AM on the morning of May 11, 1757, on an otherwise normal sailing day, a small boy died of flux while on board the slave ship the {\em Good Hope}. By the time of the boy's death, the wind had risen to the level of a \quotation{fresh breeze} and the air around the ship was \quotation{hazy.}\footnote{\quotation{\useURL[url1][http://hdl.handle.net/11134/30002:22229874][][Log book of slave traders between New London, Conn. and Africa, 1757-58 by Samuel Gould]\from[url1],} Connecticut State Library, Harford, CT.} The {\em Good Hope} was then in the middle of the Atlantic Ocean, at latitude 10.75 and longitude -32.95, two weeks into a voyage from Bunce Island, Sierra Leone to the slave markets of St.~Kitts.

The event was deplorable but not singular. The small boy who died at 10.75 by -32.95 in May 1757 was one of more than 1.8 million captive Africans who died along the Middle Passage. The unceremonious disposal of his body in the Atlantic, marked by nothing more than an entry in a logbook, is one of a vast number of such events, very few of which have ever been geolocated. For decades, scholars have added up such deaths, trying to determine the total mortality of the Middle Passage by piecing together traces gleaned from logs, registers, and bills of sale. Such cumulative efforts have been crucial to the world's understanding of the cataclysmic nature of the transatlantic slave trade, but they have left individual deaths largely unmourned.

Scholars of slavery have long acknowledged the methodological difficulty and moral paradox posed by their reliance on an archive \quotation{fundamentally marked by \ldots{} a profound, irreparable violence.}\footnote{Britt Rusert, \quotation{New World: The Impact of Digitization on the Study of Slavery,} {\em American Literary History} 29, no. 2 (2017), p.~270.} Indeed, historians' calculations of mass mortality in the Middle Passage often and unavoidably depend on statistical data generated by enslavers themselves in their attempts to control and discipline enslaved subjects.\footnote{As we define \quotation{data} as an operative term in our article, we reckon with Jennifer Morgan's concern that, in histories of the slave trade, \quotation{{[}t{]}oo often quantitative evidence {[}or data{]} of Black suffering and commodification is treated as if it is irrefutably transparent.} Furthermore, we recognize the thorniness of the term itself, as Johnson seeks to question \quotation{the stability of what has been or can be categorized as data, the uses the idea of data has been put to, and the stakes underlying data's implicit claim to stability or objectivity.} See Jennifer L. Morgan, {\em Reckoning with Slavery: Gender, Kinship, and Capitalism in the Early Black Atlantic} (Durham, NC: Duke University Press, 2021), p.~21; and Jessica Marie Johnson, \quotation{Markup Bodies: Black {[}Life{]} Studies and Slavery {[}Death{]} Studies at the Digital Crossroads,} {\em Social Text} 36, no. 4 (2018), p.~58.} Caitlin Rosenthal, for example, shows how enslavers used accounting practices, and other quantitative management methods, to brutally extract increasing sums of labor from enslaved Africans. As she argues, \quotation{precise management and violence went hand in hand \ldots{} on plantations, the soft power of quantification supplemented the driving force of the whip.}\footnote{Caitlin Rosenthal, {\em Accounting for Slavery: Masters and Management} (Cambridge, MA: Harvard University Press, 2018), p.~2.} Likewise, Jessica Marie Johnson sees \quotation{the rise of the independent and objective statistical fact as an explanatory ideal {[}as{]} party to the devastating thingification of black women, children, and men.}\footnote{Johnson, \quotation{Markup Bodies,} p.~58.} Unsettling as the similarity might be, there is a way in which historians' calculations of mass mortality in the Middle Passage, while generated to demonstrate the horrific enormity of the event, not only draw upon, but also resemble enslavers' calculations of their human cargo. Both calculations arguably rely on a process that reduces personhood to something only meaningful as part of an aggregate---in the case of the slave ship, a full complement of human cargo composed of a set quantity of {\em pieza de India}.\footnote{On enslavers' methods of quantifying human cargo at ports of departure, see Stephanie Smallwood, {\em Saltwater Slavery: A Middle Passage from Africa to America Diaspora} (Cambridge, MA: Harvard University Press, 2008), pp.~33-64. It bears noting that enslavers' use of quantitative data corresponded with, and in a certain sense preceded, the rise of statistics and demography in the late eighteenth and early nineteenth centuries. On this development, see Ian Hacking, \quotation{Biopower and the Avalanche of Printed Numbers,} {\em Humanities in Society} 5 (1982), pp.~279-95; Mary Poovey, {\em A History of the Modern Fact: Problems of Knowledge in the Sciences of Wealth and Society} (Chicago: University of Chicago Press, 1998); as well as Morgan, {\em Reckoning with Slavery}.}

Recognizing this, some digital humanists have attempted to reckon with the violence of slavery's archive and the moral ambiguity of numerical data within it. Digital humanities projects such as the \useURL[url2][https://ecda.northeastern.edu/][][Early Caribbean Digital Archive]\from[url2], \useURL[url3][http://mapping-marronage.rll.lsa.umich.edu/welcome][][Mapping Marronage]\from[url3], \useURL[url4][https://unsilencing-slavery.org/][][(Un)silencing Slavery]\from[url4], \useURL[url5][http://revolt.axismaps.com/][][Slave Revolt in Jamaica]\from[url5], and \useURL[url6][http://www.musicalpassage.org/][][Musical Passage]\from[url6], to name only a few of many, seek to subvert the violence of the archive by remediating and remixing datapoints in different ways. Ultimately, these works seek to use digital methods to illuminate both the everyday and exceptional life practices of enslaved persons and communities otherwise obscured in the analog archive of slavery---e.g., networks of exchange, strategies of revolt and resistance, Black vernacular creative work, etc. Collectively, these projects and the scholars who reflect on them form part of the emerging field of Black Digital Humanities, the radical and interdisciplinary ethos of which has provided a touchstone for our own work.\footnote{See Alanna Prince and Cara Marta Messina, \quotation{\useURL[url7][http://www.digitalhumanities.org/dhq/vol/16/3/000645/000645.html][][Black Digital Humanities for the Rising Generation]\from[url7],} {\em Digital Humanities Quarterly} 16, no. 3 (2022).}

Expanding on the above-mentioned projects, \quotation{Remembering the Middle Passage} is an attempt to begin the difficult but urgent work of mourning lives lost along the Middle Passage in digital space. A digital humanities project based out of Duke University from c.~2019-2022, \quotation{Remembering the Middle Passage} brought together undergraduate and graduate students, faculty, and librarians from five disciplines and three institutions interested in this question: how can we use digital tools to represent, and possibly memorialize, enslaved Africans who died along the Middle Passage? The major outcomes of this project are published on our website, \useURL[url8][http://rememberingthemiddlepassage.com/][][Remembering the Middle Passage]\from[url8], and include both a map of where individuals' deaths occurred along the voyage of the slave ship the {\em Good Hope} in 1757, as well as a speculative retelling of one of those individual's life story. While interest in maps has been a consistent preoccupation of Middle Passage scholarship, as seen most recently in essays by Jamila Moore Pewu and others in {\em The Digital Black Atlantic}, there exist markedly few maps of the Middle Passage---and those that do exist, e.g., those generated from the {\em Slave Voyages Database}, mostly portray broad navigational trends across aggregated slaving voyages.\footnote{Roopika Risam and Kelly Baker Josephs, eds., {\em The Digital Black Atlantic} (Minneapolis: University of Minnesota Press, 2021). In her essay in this volume, \quotation{Digital Reconnaissance: Re(Locating) Dark Spots on a Map,} pp.108-20, Pewu writes, \quotation{Can digital maps and other spatial technologies provide us with new ways of conceptualizing, documenting, redefining, or relocating the humanity/inhumanity placed on, attributed to, or celebrated by Black people? As we discover from the works compiled in {\em The Digital Black Atlantic} volume, the short answer is yes} (p.~110).} Our project is the first to map where individual deaths occurred along a single voyage and, indeed, one of the first to attempt to map Middle Passage mortality in any capacity.\footnote{Andrew Sluyter has previously attempted to map mortality on board Dutch slaving voyages. See Andrew Sluyter, \useURL[url9][https://sites.google.com/site/atlanticnetworksproject][][The Atlantic Networks Project]\from[url9], accessed 16 August 2022; and Andrew Slutyer, \quotation{Death on the Middle Passage: A Cartographic Approach to the Atlantic Slave Trade,} {\em Escalvages & Post-Escalvages / Slavery & Post-Slaveries} 3 (2020), pp.~1-19.}

In what follows, we look back on our mapping efforts over the past three years, reflecting on both the limitations and intellectual-creative possibilities of geospatial and DH methodologies in representing Middle Passage mortality. Throughout, we consider the tension between aggregation and disaggregation, collective and individual experience, and make an argument for the (hermeneutical and ethical) value of {\em disaggregation} as a way to bear witness to the human toll of the Middle Passage. Digitally mapping individual deaths along a single slave voyage, we suggest, gave us an important and new perspective on these deaths; however, it was also a perspective beset with the legacies and logics of colonial cartographic practices. Visually rendering where in the Atlantic Ocean these deaths occurred led us to speculate about the individual persons behind, and in some sense obscured by, the markers on our map. Inspired by the work of scholar Saidiya Hartman, we decided to attempt a \quotation{critical fabulation} of one of those person's life experiences. In the penultimate section, we reflect on the intellectual and creative gains afforded by combining critical fabulation and digital humanities methodologies, attending to how these methodologies can augment each other by opening up more meaningful and intentional ways of interfacing with slavery's archive.

\subsection[title={2. The Case for Disaggregation},reference={the-case-for-disaggregation}]

Beginning with the work of economic historians such as Philip Curtin, scholarship on the transatlantic slave trade has sought to tell the story of the Middle Passage through its scale. In many cases, the numerical statistics that these scholars draw from take on a narrative power that obscures the human agents who are represented by and responsible for the numbers.\footnote{Cf. Smallwood, who expresses her concern that \quotation{{[}b{]}y tallying the dead to measure the toll the voyage took on African life, we have made that body count the most potent symbolic measure of the horrors of the middle passage \ldots{} overall numbers---and our interpretation of them---correspond only loosely to the ways African captives experienced and understood shipboard mortality.} See Smallwood, {\em Saltwater Slavery,} p.~137.} To be sure, motivating much of this work is a keen desire to reckon with the massive loss of human life and human suffering occasioned by the Middle Passage. In the twenty-first century, these census-like statistics have provided the source data for a number of digital projects that implicitly argue for the historical and ongoing significance of Middle Passage atrocities through spectacular visual deployments of this mass of data. The \useURL[url10][https://www.slavevoyages.org/][][Slave Voyages Database]\from[url10], for example, has amassed digital records (in tables, timelines, maps, etc.) for over 36,000 slave voyages by which over 12 million enslaved Africans were taken from their home continent and almost 2 million perished at sea. Similarly, \useURL[url11][https://slate.com/news-and-politics/2021/09/atlantic-slave-trade-history-animated-interactive.html][][\quotation{The Atlantic Slave Trade in Two Minutes}]\from[url11] (Kahn and Bouie) animates the comings and goings of over 20,000 slave ships between 1545 and 1860 overlaid on a map of the Atlantic in a mere 120 seconds.

In its early stages, \quotation{Remembering the Middle Passage} worked towards similar goals, analyzing the tens of thousands of overlapping routes in the {\em Slave Voyages} database in an attempt to situate areas in the Atlantic with the highest density of mortality. We believed that these large datasets might provide specific GIS points where enslaved people died and thus help us in our overall purpose of site-based memorialization. We limited our search to British voyages, whose records would be in English, that were cross-listed in both the {\em Slave Voyages} and \useURL[url12][https://www.historicalclimatology.com/cliwoc.html][][CLIWOC]\from[url12] (Climatological Database for the World's Oceans), as CLIWOC provided geocoordinates. By cross-referencing these datasets, we hoped to use the advantages of mass data for the purposes of locating individuals' places of death. In this way, we sought to acknowledge the reality and validity of personal tragedy both through and against \quotation{big data}-based narratives of genocide and crimes against humanity.\footnote{From eighteenth-century abolitionism through to the development of international human rights law and the modern-day reparations movement, there is a long tradition of classifying slavery and slave trafficking as a \quotation{crime against humanity.} See, e.g., Ariela Gross, \quotation{\quote{A Crime Against Humanity}: Slavery and The Boundaries of Legality, Past and Present,} {\em Law and History Review} 35, no. 1 (2017), pp. 1-8.} Although we believed several of these slave voyages were in both databases and could therefore be mapped, our efforts were frustrated by the aggregative structure of the {\em Slave Voyages} and CLIWOC databases, which either do not contain the necessary geocoordinates ({\em Slave Voyages}) or do not distinguish between individual voyages that might have shared the same name (CLIWOC).\footnote{We initially set out to combine the {\em Slave Voyages} and CLIWOC databases in order to find an overlapping subset of voyages for which there was both information about their exact routes and records of their involvement in the slave trade. Our hope was to find an area in the Atlantic where a great number of deaths had occurred. Andrew Sluyter has done a similar project with twenty-six Dutch slaving voyages, but we hoped to produce results on a larger scale. In order to make sense of this large archive, we cross-referenced ships' names and dates of departure in the {\em Slave Voyages Database} with CLIWOC to produce a map of nine journeys.~Mapping these journeys revealed that many of the ships in CLIWOC were not slaving vessels and that the names of these ships were often reused.} Thus, like many overly hopeful digital humanities researchers, we were stumped by the incomplete nature of the archive. While the group did generate some useful visualizations of mortality in the Middle Passage, the large-scale project proved too daunting, primarily because of the absence of individuated data about mortality and ships' routes.

While faced with the pragmatic difficulty of representing mortality in the digital archive as it is currently structured, we confronted the simultaneous moral and ethical questions surrounding the use of scale as a primary method of storytelling. Indeed, while initiatives such as the {\em Slave Voyages Database} and \quotation{The Atlantic Slave Trade in Two Minutes} ably convey the awful scope of the transatlantic slave trade, they run the risk of assimilating what was in actuality 12 million individual human beings' personal tragedies into a single dataset of a single tragedy of human history. In a recent article, Johnson argues that such projects run the risk of dehumanizing Black people and thingifying their bodies in ways that are not only complicit with the historical injustices of slavery but also the present-day manifestations of that legacy.\footnote{Johnson, \quotation{Markup Bodies.}} Vincent Brown likens the statistical analysis of slave-trade mortality rates to the chalk outline of a murder victim, arguing that \quotation{{[}t{]}he data delineate scale, proportion, and distribution quite well, but they cannot represent the wrenching personal trials endured by the enslaved.}\footnote{Brown writes, \quotation{Gains that derive from elucidating general trends are offset by insensitivity to the experience of the historical subject. Considerations of scale, variation, and typicality trade the anguish and confusion of dimly discernible experiences for perceived mastery of the facts.} See Vincent Brown, {\em Reaper's Garden: Death and Power in the World of Atlantic Slavery} (Cambridge, MA: Harvard University Press, 2010), pp.~28-29.} Johnson's and Brown's arguments here coincide with a broader concern about the violence of the slave trade archive towards the subjecthood of the enslaved, as expressed by scholars and writers such as Saidiya Hartman, M. NourbeSe Philip, Britt Rusert, and others cited in our introduction.\footnote{See the previous section as well as Saidiya Hartman, \quotation{Venus in Two Acts,} {\em Small Axe} 12, no. 2 (2008), pp.~1-14; and M. NourbSe Philip, {\em Zong! As Told to the Author by Setaey Adamu Boateng} (Middletown: Wesleyan University Press, 2008), pp.~108-207.} To \quote{short-circuit} this violence, Johnson looks to what she calls \quotation{Black digital practice,} or \quotation{the many ways users, content creators, coders, and programmers have worked ethical, intentional praxis into their work in pursuit of more just and humane productions of knowledge.}\footnote{Johnson, \quotation{Markup Bodies,} p.~66.} In an attempt to honor the ethos of Black digital practice, it became our challenge as digital humanists to visually represent mortality in the Middle Passage while resisting the \quotation{double dehumanization} of quantification unaccompanied by human narratives. Instead, we looked for ways to reinsert individuality, subjectivity, and humanity into the digital archive of slavery while holding room for the impossibility of redress.\footnote{Cf. Hartman's recognition that \quotation{redressing the violence that produced numbers, ciphers, and fragments of discourse, which is as close as we come to a biography of the captive and the enslaved} is an \quotation{impossible goal.} See Hartman, \quotation{Venus in Two Acts,} p.~3.}

In moving away from the overdetermined conclusions and misleading sense of completeness conveyed by large-scale visuals and datasets, we were inspired by the methodology of critical cartography.\footnote{According to Matthew Wilson, critical cartography proceeds from the assumption \quotation{{[}t{]}hat maps stratify and create the potentialities for resistance.} See Matthew W. Wilson, {\em New Lines: Critical GIS and the Trouble of the Map} (Minneapolis: University of Minnesota Press, 2017), p.~7.} In particular, we were intrigued by the example of \useURL[url13][https://forensic-architecture.org/investigation/the-left-to-die-boat][][\quotation{The Left-to-Die Boat,}]\from[url13] a digital investigation of the journey and eventual sinking of a boat of African migrants in the Mediterranean in 2011, led by Charles Heller and Lorenzo Pezzani of the Forensic Architecture Lab at Goldsmiths University, London. This work, prosecutorial in nature, aims to repurpose surveillance technologies such as military and coast guard photography as evidence of the negligence and ultimate guilt of European governing bodies in migrant deaths. Particularly impressive is how \quotation{The Left-to-Die Boat} achieves epistemic distance from its source data; its use of satellite images in defamiliarizing ways shows not only how (digital) surveillance technologies are made complicit in rendering certain human lives expendable, but also how the human agents who wield these technologies are responsible for their choices to leave others to die. The researchers' decision to focus on the case study of a single migrant voyage helps throw into sharper relief these issues of agency and contingency, allowing the project to make visible the specific persons implicated in the sinking of a specific vessel even as these events are taken to be representative of larger trends of violence and injustice. For our own work in the context of Atlantic slavery, the \quotation{Left-to-Die Boat} prompted us to think critically about how the cumulative mortality along the Middle Passage was the consequence of specific persons' choices (those of company officials, ship captains, crew members, etc.), albeit punctuated by subversive acts of captive Africans, including suicide as form of resistance. We therefore came to regard disaggregation as a potentially powerful tool of Black digital practice. If we could uncover archival records regarding specific persons on a specific slaving voyage, and if we could map this voyage as the \quotation{Left-to-Die Boat} researchers did, might we also be able to make room for the counterfactual possibilities and contingencies contained within the awful statistics and overdetermined narratives of Atlantic slavery?

\subsection[title={3. Mapping a Single Voyage},reference={mapping-a-single-voyage}]

Both materially constrained by existing datasets (e.g., {\em Slave Voyages} and CLIWOC) and increasingly motivated on a theoretical and ethical plane towards other types of approaches, the team attempted to locate a single captain's or first mate's logbook. This crucial record would contain both (a) the latitudinal and longitudinal coordinates necessary to map a voyage, and (b) the date and time of specific deaths onboard the ship. In other words, we turned at this point from big data to what we might call \quote{small data}; from aggregated data to data as disaggregated as possible. We felt there were advantages to such an approach, that in doing so we were taking seriously Catherine D'Ignazio's and Lauren Klein's call to \quotation{embrace emotion and embodiment} in our project rather than obscuring individual bodies and their visceral experiences through large datasets and databases.\footnote{Catherine D'Ignazio and Lauren F. Klein. {\em Data Feminism} (Cambridge, MA: MIT Press, 2020), p.~73.} The work of \quotation{embracing emotion and embodiment} and re-membering bodies opens up possibilities not only to acknowledge them as once inhabited, as possessing human subjecthood, but also to begin memorializing lives (cut short) by recasting otherwise unmarked ocean spaces as their places of death.\footnote{Following Njelle Hamilton, we understand \quotation{re-membering} as signifying \quotation{the connections between remembrance---the repeated psychological reassembling of memories---and re-membering---the physical reassembling of the disarticulated or shattered members of the \ldots{} body.} See Njelle W. Hamilton, {\em Phonographic Memories: Popular Music and the Contemporary Caribbean Novel} (New Brunswick, NJ: Rutgers University Press, 2019), p.~80. Likewise, we acknowledge and share Katherine McKittrick's concerns around scholarly obsession with Black bodies: \quotation{{[}c{]}an the preoccupation with the black body---whether that involves liberating it from scientific racism or honoring it as a site of resistance---perhaps conceal a range of black knowledge formations that, while certainly embodied, are not reduced to the biologic?} See Katherine McKittrick, {\em Dear Science and Other Stories} (Durham, NC: Duke University Press, 2021), p.~49.}

Observations about the location of a ship and any onboard deaths were often---though, as we came to realize, not always---recorded daily in captains' or first mates' logbooks.\footnote{For instance, in one of the team's first research trips, team member Jane Harwell found that the logbook of the slave ship {\em Marlborough} (c.~1750) contained neither observations about location nor about mortality.} In October 2019, we were able to locate the logbook of the slave ship the {\em Good Hope} (c.~1757). The logbook was authored by Samuel Gould, a native of Killingly, Connecticut, who served as first mate aboard the {\em Good Hope} as it trafficked enslaved persons from Bunce Island off the coast of present-day Sierra Leone to a lifetime of plantation slavery in St. Kitts. Gould's logbook contains daily entries that mark the date, ship's latitude and (dead-reckoned) longitude, as well as any deaths of enslaved Africans on board the ship, making it the kind of primary source that would help us punctuate aggregated mortality data with individual stories. Select pages of the logbook, which is housed at the Connecticut State Library, are accessible on the website of the \useURL[url14][https://museumofcthistory.org/log-book-of-slave-traders/][][Museum of Connecticut History]\from[url14]. The library staff generously supplied us with scans of the rest of the logbook.

In order to decipher the logbook, we sought the help of Dr.~James Delgado, an archaeologist with expertise in the history of marine navigation.\footnote{Dr.~Delgado has recently been involved with the excavation of the {\em Clotilda}, the last known slave ship to land in the United States. See Michael Levenson, \quotation{\useURL[url15][https://www.nytimes.com/2021/12/25/us/clotilda-slaveship-africa-alabama.html][][Last Known Slave Ship Is Remarkably Well Preserved, Researchers Say]\from[url15],} {\em New York Times}, 25 December 2021, accessed 29 August 2022.} We then learned how to translate the then-common method of deductive (\quotation{dead}) reckoning into contemporary mapping geocoordinates.\footnote{In the logbook of the {\em Good Hope}, latitudinal and longitudinal coordinates are given in the format of degrees minutes seconds (DMS), with the small issue that seconds are not listed, but the crucial information of degrees and minutes is captured. This made determining the actual position of the {\em Good Hope} at any given point of time uncertain but not impossible. For example, one of the entries on Saturday, 30 April 1757, had the boat at 9 28 N, 16 1 W or 9 Degrees 28 Minutes North 16 Degrees 1 Minute West. In order to map these coordinates using modern mapping software, we needed to convert the DMS coordinates into decimal degrees. For the purposes of our project, we estimated seconds as zero.} By sifting through the multiple entries used to track the ship's course, we were able to locate the data necessary to map the voyage.\footnote{It is worth reflecting on how these changing methods of mapping the ocean and measuring movement across it inflect our understanding of those who, like the small boy, disappeared below its surface. In {\em The Social Construction of the Ocean}, Philip Steinberg notes that, \quotation{{[}b{]}y the early seventeenth century, maps featuring sea monsters representing marine nature and ships representing the social activity that transpired in the area of ocean-space began to be replaced by maps portraying a grid over an essentially featureless ocean.} This focus on the surface as the \quotation{known} aspect of the ocean had the effect of rendering the oceanic depths the complementary \quotation{unknown.} By ejecting the bodies of the enslaved from the realm of the known and the measurable, enslavers may have hoped to erase them entirely. On the contrary, as Katherine McKittrick and Clyde Woods argue, \quotation{{[}t{]}he storm-torn bodies, those thrown overboard and forgotten, and the many other narratives and experiences that are violently and/or uncomfortably situated with the geography of reason have produced what Edouard Glissant calls submarine roots: a network of branches, cultures, and relations that position black geographies and the oceanic history of diaspora as integral to and entwining with---rather than outside---what has been called coloniality's persistence.} See Philip E. Steinberg, {\em The Social Construction of the Ocean} (Cambridge: Cambridge University Press, 2001), p.~105; and Katherine McKittrick and Clyde Woods, \quotation{No One Knows the Mysteries at the Bottom of the Ocean,} in {\em Black Geographies and the Politics of Place}, ed.~Katherine McKittrick and Clyde Woods (Toronto: Between the Lines, 2007), p.~5.} For example, on the page of the logbook shown in figure 1, observations about the ship's course, including distance traveled during the previous twenty-four hours, as well as the latitude and longitude, are listed directly above the date. The three leftmost columns record the hour in two-hour increments starting from noon of the previous day and ending the following noon, as well as the ship's speed in knots and half-knots. In the \quotation{Remarks} section furthest to the right, Gould inserts an entry at 4:00 PM that states \quotation{fresh breezes + hazy.} Below is the kind of notation that we see on 14 separate occasions during this part of the voyage. It reads: \quotation{at 3 am died a small boy slave with a flux.} By converting these observations and the events they describe into mappable datapoints, we hoped to re-member or memorialize deaths such as this one---evidence of which appears in the archive as fleetingly and mundanely as the day's weather---while also shedding light on the agency of captive Africans in their deaths.

\placefigure[here]{The logbook of the {\em Good Hope}}{\externalfigure[issue07/logbook.jpg]}


Working with the Center for Data Visualization Sciences at Duke University, we were able to input this historical data into modern GIS software and, in doing so, draw a digital and visual link between the events of 1757 and the (present, enduring) units of ocean-space where these events took place, marking with modern technology and in our own time the scenes of these past atrocities. At this point, we had a map that realized our goal of accurately locating the deaths of enslaved people in the Middle Passage, albeit on quite a small scale. However, the map itself, and the methods we used to produce it, generated their own set of questions. Our chosen technology---ArcGIS---presented us with certain limitations on how we were able to represent the records of mortality we found in the logbook of the {\em Good Hope}.

In the process of translating archival data into a map, the team was highly conscious of the imperial legacies of western cartographic projects. Throughout the age of \quotation{Enlightenment,} European colonizers effectuated spatial expansion in parallel with the dehumanization of the other peoples they encountered. Ways of representing and producing space through Cartesian coordinate systems found their counterpoint in the annihilation of myriad indigenous African and American spatial and geographic knowledges. Not only did condensing places to coordinates enable transatlantic navigation, territorial conquest, genocide, and mass deportation, but the process of mapmaking also acted as a practice by which the white cartographic subject invented itself at the expense of indigenous knowledge producers. Tiffany King names this process as the \quotation{cartographic writing of the human through the spatial and orthographic negation of Native and Black people.}\footnote{Tiffany Lethabo King, {\em The Black Shoals: Offshore Formations of Black and Native Studies} (Durham, NC: Duke University Press, 2019), p.~32.} Because of this, it is uncertain whether maps created within the epistemological scope of this inheritance and using the same logics (with ArcGIS) can be made to convey oppositional spatial and geographic experiences. In other words, should we represent atrocities with the logics that made them possible in the first place?

Responding to this question with a tentative \quote{no,} geographer Matthew Wilson, a theorist of critical cartography, suggests that \quotation{attaching points, lines, and polygons to narrative may actually delimit the kinds of narrative that can be advanced \ldots{} placing history onto the map may only further distract from historical (and spatial) interpretation made most powerful in narrative.}\footnote{Wilson, {\em New Lines}, p.~37.} Indeed, in our case, the transposing of logbook entries into two-dimensional and purportedly objective geocoordinates projects a sense of narrative linearity and certainty that does not reflect the shifting reality of oceanic space. One of the ways in which we attempted to suspend this sense of narrative closure was by inverting the textures and tones on our map in order to highlight the ocean rather than the land masses from which and to which the {\em Good Hope} was bound (see figure 2). In this decision, we were particularly inspired both by \quotation{The Left-to-Die Boat} and the work of Jennifer Reed, who, in a recent article, reflects at length on the \quotation{associative and emotional} significance of color in mapping acts of sexual violence on the estate of enslaver Thomas Thistlewood.\footnote{Beate Weniger qtd. in Jennifer Reed, \quotation{Representing Sexual Violation in the Archive of Caribbean Enslavement,} {\em Studies in Eighteenth-Century Culture} 49 (2020), p.~94. Reed reflects on the connotations of different colors and the ways in which color choices can support or undermine the aims of digital humanities projects. Regarding her own work in mapping where enslaved persons were sexually violated, she writes, \quotation{{[}t{]}he use of color in such a map is particularly sensitive, given a context in which the pigmentation of skin determined vulnerability to myriad abuses} (p.~96).} However, while visually emphasizing the ocean helped us convey a sense of existential uncertainty on our map, we still struggled to account for the experiential or transformative element of the Middle Passage, or what Sowande' M. Mustakeem calls the \quotation{human manufacturing process} of the transatlantic slave trade.\footnote{Sowande' M. Mustakeem, {\em Slavery at Sea: Terror, Sex, and Sickness in the Middle Passage} (Champaign: University of Illinois Press, 2016), p.~6, {\em passim}.} That is to say, we wanted to avoid the empiricist trap that comes from the certainty of mapping while simultaneously acknowledging the atrocities that transpired on the {\em Good Hope}. In one sense, we need technologies such as GIS in order to make the ship's voyage visible, but in another, as critical humanists, it behooves us to trouble and add dimensionality to the dots and lines of the journey.

\placefigure[here]{GIS rendering of the voyage of the {\em Good Hope}}{\externalfigure[issue07/gis.png]}


\subsection[title={4. Critical Fabulation and the Digital},reference={critical-fabulation-and-the-digital}]

Our experiments with GIS technology speak to a range of debates in the digital humanities about the role of DH in the scholarly process. Resisting the tendency to define digital projects in terms of their professionally recognized and often institutionally rewarded outcomes or deliverables (e.g., the digital map, database, exhibit, etc.), the authors of \quotation{The Digital Humanities Manifesto 2.0} embrace a vision of the digital humanities as invested in process rather than in product: \quotation{{[}a{]}nything that stands in the way of the perpetual mash‐up and remix,} they write, \quotation{stands in the way of the digital revolution \ldots{} Untapped gold mines of knowledge are to be found in the realm of {\em process}.}\footnote{Peter Lunenfeld, Todd Pressner, Jeffrey Schnapp, et al., \quotation{\useURL[url16][http://www.humanitiesblast.com/manifesto/Manifesto_V2.pdf][][The Digital Humanities Manifesto 2.0]\from[url16],} 2009, accessed 17 August 2022, p.5; emphasis added.} Later, they contend that \quotation{the phrase {[}digital humanities{]} has use‐value to the degree that it can serve as an umbrella under which to group both people and projects seeking to reshape and reinvigorate contemporary arts and humanities practices, and expand their boundaries.}\footnote{Ibid., p.~13. Cf. Matthew Kirschenbaum, \quotation{Digital Humanities as/is a Tactical Term,} in {\em Debates in the Digital Humanities}, ed. Matthew K. Gould (Minneapolis: University of Minnesota Press, 2012), pp.~415-28.} What these comments suggest is that, when practiced intentionally, DH is but one step in the scholarly process or instrument in the scholarly toolkit, one which can broaden the interpretive horizon of the professional humanities, but which cannot (or at least need not) furnish ready-made interpretations and insights by itself.

This instrumental vision of the digital humanities has particular import in the field of digital work on slavery, where scholars such as Britt Rusert and Jessica Marie Johnson have expressed concern about the fetishization of the digital among their peers. Rusert warns that \quotation{digitization does not necessarily enable more access to the perspectives and stories of the enslaved \ldots{} {[}e{]}ven collections that make available runaway slave notices, plantation records, and other documents of slavery add but one more layer of mediation between the scholar/listener and slavery's archive, through the process of digitization itself.}\footnote{Rusert, \quotation{New World,} p.~272.} While digitization and digital methods cannot close the interpretive distance between the enslaved and the scholar of slavery, they can, however, remediate and remix the archive in ways that expose this distance and the silence it produces, prompting us to acknowledge the present absence of the perspectives and stories of the enslaved that are beyond access. One digital project we found particularly useful in opening up new avenues of scholarly and creative inquiry was Vincent Brown's digital project {\em Slave Revolt in Jamaica, 1760--1761: A Cartographic Narrative}. In an article about this project, Brown emphasizes that \quotation{{[}m{]}apping the revolt and its suppression illustrates something that is difficult to glean from simply reading the textual sources,} viz., the role of geography in \quotation{the tactical dynamics of slave insurrection and counterrevolt,} ultimately making possible many of the insights in his 2020 book {\em Tacky's Revolt: The Story of an Atlantic Slave War}.\footnote{Vincent Brown, \quotation{Mapping a Slave Revolt: Visualizing Spatial History through the Archive of Slavery,} {\em Social Text} 125 (2015), p.~136.}

In our case, the work of digital mapping brought us into close contact with an individual's death. In the previous section, we referred to an entry in the {\em Good Hope} logbook that describes the death of a \quotation{small boy} who \quotation{at 3 am died \ldots{} with a flux} fourteen days into the journey of the {\em Good Hope}. This entry initially captured our attention because it was one of the few recorded deaths in the logbook that had a modifier, \quotation{small,} before the standard \quotation{boy,} \quotation{man,} \quotation{girl,} descriptors. Indeed, to engage with the logbook of the {\em Good Hope} is in large part to foreground the voice of an enslaver, viz., Samuel Gould, whose perspective typifies the empirical hubris of European colonizers in their quest to capture and quantify space, time, and human beings. To generate a map from Gould's geocoordinates then is to take his epistemological points of departure as our own; mapping this entry comes close to being an act of mere transposition. In looking for ways to move beyond the impasse of simply visualizing enslavers' worldviews, we were moved by Gould's decision to characterize the small boy as \quotation{small.} Gould likely included this modifier for economic reasons: small might have denoted the boy as a {\em smaller} fraction of a commodity, marking him as less valuable. However, for us, the fact of the boy's being \quotation{small} provided an opening to glimpse the lives, experiences, and knowledges of the people fixed as commodities by Gould's record-keeping and wider epistemological violence. We wrestled with what this specificity meant. How \quotation{small} was this \quotation{small boy}? How old was he? Was he \quotation{small} simply because he was young, or for another reason? How might his \quotation{smallness} have shaped or otherwise contributed to his death?

Gould's geocoordinates enabled us to locate the site of the small boy's death, or, more specifically, his resting place at sea, and to know the apparent medical circumstances of his death. Seeing a GIS rendering of Gould's notation made the event he casually describes feel more tangible to us, putting us into closer contact with the small boy's life through the place of his death. But, as humanists, we wanted to know more: What was his name? Who was his mother? When he fell ill, did anyone care for him, physically or emotionally? Was he alone when he died in the early hours of the morning? How did he experience the suspended time and space of the slave ship? The written archive, populated by the notations of men like Gould, cannot answer these questions. Even digital mediations involving archival objects like the logbook of the {\em Good Hope} cannot put sufficient distance between enslavers' observations and the violence done to people like the small boy.

To better achieve this distance, we turned to the methodology of critical fabulation, as practiced and theorized by Saidiya Hartman. In her conceptualization of the methodology, Hartman draws from the following descriptions of fabula: \quotation{denotes the basic elements of story, the building blocks of the narrative \ldots{} {[}a{]} series of logically and chronologically related events that are caused and experienced by actors.}\footnote{Hartman, \quotation{Venus in Two Acts,} p.~11.} While not exactly fiction, critical fabulation is a way of producing knowledge that exists at the \quotation{intersection of the fictive and the historical} and, according to Stephanie Smallwood, whose 2007 book {\em Saltwater Slavery} incorporates the subjunctive mood called for by Hartman to powerful effect, illuminates \quotation{the false dichotomies of truth vs.~fiction and fact vs.~fantasy.}\footnote{Hartman, \quotation{Venus in Two Acts,} p.~12; and Stephanie Smallwood, \quotation{The Politics of the Archive and History's Accountability to the Enslaved,} {\em History of the Present} 6, no. 2 (2016), p.~127.} While critical fabulation may escape the restrictions of traditional historiography, questions---both ethical and methodological---still arise. For instance, one must take care not to use narrative to \quotation{fulfill our yearning for romance, our desire to hear the subaltern speak, or our search for the subaltern as heroic actor whose agency triumphs over the forces of oppression.}\footnote{Smallwood,\quotation{The Politics of the Archive and History's Accountability to the Enslaved,} p.~128.} To critically fabulate, or to think and work in the subjunctive, is to sit with the discomfort of trying to achieve the \quotation{impossible goal} of redress.\footnote{Hartman, \quotation{Venus in Two Acts,} p.~3.}

The small boy who died on May 11, 1757 died after three weeks at sea on the {\em Good Hope}. Our critical fabulation of the circumstances of the small boy's death is \useURL[url17][http://rememberingthemiddlepassage.com/\#small_boy][][published on our website]\from[url17] and weaves together (retro-)speculative accounts of the boy's childhood, family, illness, and burial. For us, the radical individuation of critical fabulation provided a counterhistory to the dehumanizing drive of slavery's archive, restoring the kinship and tribal connections of the dead and reminding us of the funerary principles of which he was deprived. But we found that disaggregating the death of this one small boy from the many like him who died on the Middle Passage, even the several \quotation{small boys} and girls who died on the 1757 voyage of the {\em Good Hope}, came at its own cost. Indeed, it tends to obscure the fact that complete disaggregation is impossible. Small Boy can never be just one small boy who died of flux during his Middle Passage journey. Like Hartman's Venus, he will always carry the \quotation{immeasurable weight of black lives made barely knowable by the violence of racial slavery.}\footnote{Smallwood, \quotation{The Politics of the Archive and History's Accountability to the Enslaved,} p.~119.} Our speculation of what he could have experienced on the {\em Good Hope}, his potential kinship networks, and the mourning practices of his people mostly come from collective historical information regarding enslaved West Africans. In other words, critical fabulation runs the risk of spectacularizing the historical subjects whose stories it seeks to tell, making historical individuals, e.g., the archetypes of \quotation{small boy} and \quotation{Venus,} stand in for broader collectives of enslaved Africans.

Yet, we contend that bringing critical fabulation into the digital archive and into conversation with digital humanities methodologies, specifically those informed by Black digital practice, can help alleviate this problem of spectacularization. In many ways, the methodological aims of both critical fabulation and Black digital practice are commensurate. According to Hartman, critical fabulation responds to the problem that the lives of the enslaved are \quotation{entangled with and impossible to differentiate from the terrible utterances that condemned them to death.}\footnote{Hartman, \quotation{Venus in Two Acts,} p.~3.} (In the case of the {\em Good Hope} logbook, Gould's description of the \quotation{small boy} marks its referent as \quotation{degraded matter, dishonored life.}) Like critical fabulation, Black digital practice seeks to reconfigure enslavers' language and turn it---fabulate it---into something new and more humanistic.

However, where Hartman's model of critical fabulation and our deployment of it arguably differ is in the kinds of primary sources with which they engage. In texts such as {\em Lose Your Mother} and \quotation{Venus in Two Acts,} Hartman draws upon a primarily prose-based and printed archive; as she points out, \quotation{the lives of two girls {[}whose stories she critically fabulates{]} are recorded in the {\em official} annals of British law and parliamentary debate.}\footnote{Hartman, \quotation{The Dead Book Revisited,} {\em History of the Present} 6, no. 2 (2016), p.~208; emphasis added.} (We emphasize the descriptor \quotation{official} here because it registers the public and publicized nature of Hartman's comparatively \quote{analog} archive).\footnote{This is not to suggest that Hartman is uninterested in other kinds of archival records. Indeed, after the quote cited above, she refers to a range of \quotation{fragments and scraps of the archive,} including \quotation{the ship's manifest, the legal case, the newspaper profile, the death table, the actuarial chart, the autopsy report, the tally of police killings.} See Hartman, \quotation{The Dead Book Revisited,} p.~208. Moreover, in her latest work of critical fabulation, Hartman explicitly \quotation{ma{[}kes{]} use of a vast range of archival materials \ldots{} culled from the journals of rent collectors; surveys and monographs of sociologists; trial transcripts; slum photographs; reports of vice investigators, social workers, and parole officers; interviews with psychiatrists and psychologists; and prison case files.} See Saidiya Harman, \quotation{A Note on Method,} in {\em Wayward Lives, Beautiful Experiments: Intimate Histories of Riotous Black Girls, Troublesome Women, and Queer Radicals}, (New York: Norton, 2019), pp.~2-3. That Harman broadens her archive in her latest monograph is perhaps owing to the greater availability of such materials in the post-emancipation period with which the monograph is concerned.} By contrast, in our own experiment with deploying critical fabulation in the digital realm, the kinds of records that interested us were primarily numerical---{\em digital} in the broad and original sense of the term---and handwritten. As critical fabulators, therefore, we encounter and negotiate the violence of slavery's archive differently than does Hartman. Whereas Hartman turns to critical fabulation to cut through an excess of words spoken and published by officials whose very verbosity obscures and does violence to the subjectivity of enslaved persons, we turn to critical fabulation to supplement a lack of words in the digital archive and to lend quality to the quantities we find calculated therein, moving from numbers towards narratives.

This shift from numbers to critically fabulated narrative was, in part, made possible by a growing body of rich and meticulous historical work comprising the field of \quotation{Middle Passage Studies.} Historians of the Middle Passage such as Marcus Rediker, Stephanie Smallwood, and Sowande' Mustakeem share our grim fascination with \quotation{the ledgers, bills of lading, and other instruments of accounting} that make up slavery's digital archive.\footnote{Smallwood, {\em Saltwater Slavery}, p.~4.} Rediker and Smallwood, for their part, seek to supplement these digital and handwritten records with enslavers' private correspondence in an effort to bring to the fore what they call \quotation{the high human drama of the slave trade} (Rediker) or \quotation{the human story of the slave trade} (Smallwood).\footnote{Rediker, {\em The Slave Ship: A Human History} (London: John Murray, 2007), p.~41; and Smallwood, {\em Saltwater Slavery}, p.~5.} Mustakeem, conversely, finds digital records valuable in their own right, regarding them as \quotation{more than crude and cold numbers} inasmuch as they are accompanied by handwritten notes that \quotation{enable a more textured description of slaves' bodies} and furnish insights into \quotation{the tangible effects of slavery at sea.}\footnote{Mustakeem, {\em Slavery at Sea}, p.~12.} What distinguishes our own engagement with these kinds of digital sources from the work of these historians is precisely our decision to prioritize an individual human's life story over the more general \quotation{human story of the slave trade.} The entry in Gould's logbook in which we encounter the small boy might have provided evidence for a historian like Mustakeem for a broad historical claim about, e.g., the prevalence of flux among enslaved children onboard slave ships; but we wanted to explore how focusing on the details of the small boy's death, rather than treating him as one example among many, deepens our understanding of the Middle Passage---and we turned to the methodology of critical fabulation to make this narrative maneuver possible. We acknowledge, however, that this turn would not be possible without the extraordinary scholarship of Smallwood, Mustakeem, and others.

At the root of our experiments with critical fabulation, and our efforts to lend narrative quality to the numerical records of slavery's digital archive, was the fraught mapping process discussed in the previous section. Katherine McKittrick maintains that the \quotation{{[}i{]}interplay between narrative and material worlds is especially useful in black studies, because our analytical sites, and our selfhood, are often reduced to metaphor, analogy, trope, and symbol.}\footnote{McKittrick, {\em Dear Science and Other Stories,} p.~10.} By returning to and re-presenting the material world that was once the scene of the small boy's death, we opened the way for a narrative to surface. Our practice of critical fabulation grew out of the counter-abstraction work made possible by digital methods and allowed us to depart from somewhere other than the mere seven words left by an enslaver. The act of plotting the coordinates accompanying Gould's \quotation{terrible utterance} distilled vast ocean-space into the material site of the small boy's last moments. If, as McKittrick argues, \quotation{{[}t{]}he dead spaces are inextricably linked to the dehumanizing scripts,}\footnote{Ibid., p.~11.} then reinvesting \quotation{dead}-reckoned and empty ocean-space with significance as a human resting place helps us to resist reducing a human life to a spectacular symbol or to a dehumanized datapoint.

In \quotation{Venus in Two Acts,} Hartman makes her readers bear witness to the acts that transpired on the deck of a slave ship, the {\em Recovery} (1791), reconfiguring the legal testimony given by its captain in the course of legal proceedings against him. Whereas Hartman's narrative rests on the liminal deck of the slave ship, our own digital work uses geospatial data to precisely situate a different slave ship in ocean-space, including the persons held captive in its hold. Visualizing this otherwise unmarked space, and anchoring it in specific coordinates, provided us with an basis upon which to critically fabulate the narrative of a death that, in its mundaneness, does not appear in the printed archive (e.g., in legal records, newspaper reports, etc.) as Venus's does. Unlike Venus, the small boy was not identified as a victim in a murder trial; his death, like those of countless others, was the consequence of an illness that, while endemic, was not unavoidable, and for which his enslavers bore responsibility. Mapping the site of the small boy's death reveals the site where a historical injustice occurred. Locating and visualizing this site renders more proximate the conditions that caused the small boy's premature death and therefore provides a material point of departure for creative reckonings with his life lived and life lost.

\subsection[title={5. Conclusion: Towards (Digital) Community},reference={conclusion-towards-digital-community}]

For our research collective, the combination of critical fabulation and digital methods proved a compelling way to take up the mantle of Black digital practice, \quotation{chart{[}ing{]} a path against the drive for data} and expanding on the unquantifiable.\footnote{Johnson, \quotation{Markup Bodies,} p.~70.} Imagining in a subjunctive mood the life story of the small boy referenced in Gould's logbook entry enabled an epistemic shift towards the \quotation{more just and humane productions of knowledge} that Johnson calls for.\footnote{Ibid., p.~66.} We have argued that critical fabulation is a powerful response to the absented presences held within the clean lines of digital mapping, and contended that spatializing slavery's archive with digital tools affords new ways to expand on critical fabulation, providing a numerical rather than prose-based grounding. The reciprocal interplay of these techniques in our project, we hope, works to \quotation{resist and counteract slavery's dehumanizing impulses.}\footnote{Ibid.}

However, just as it was not enough for us to simply map sites of death along a slaving voyage without further interpretation, it is arguably not enough for our digital project, including the narrative of the small boy, to exist solely as an academic digital humanities project whose users and critics are primarily scholars and students at elite institutions. Indeed, as it is currently structured and published, our project falls short of many of the ideals that Johnson envisages for Black digital practice, especially in its more radical aims---for example: broad and democratic community engagement, amplifying the voices of descendants of the enslaved; subversion of elite and narrowly academic forms of digital literacy and knowledge production; and, allyship with Black activism and revolutionary projects. Ultimately, we maintain that this project was an experiment in how digital methods can disrupt colonial productions of space and challenge the disposability of Black life; but we do not claim that our methods are the only (or even a sufficient) way to achieve this objective or that they have some claim to legitimacy over other Black digital praxes.

One of the core claims of this article has been that, in order to re-center the human amid the mass of Middle Passage mortality data, we must first disaggregate datapoints that otherwise remain obscured in aggregates of body counts, total voyage numbers, etc. That being said, we harbor concerns about how the work of disaggregation, e.g., singling out the life story of one particular individual among the mass of individuals who died along the Middle Passage, could be seen as a methodological maneuver undergirded by Western imperial and liberal values that privilege individual over communal experience. In disaggregating the small boy's death from the overwhelming data of the archive, it is important that we nonetheless hold room for the collectives of family, community, and diaspora to which he belonged, as well as the spiritual communion that he may have found in death. This includes his relations and kinsfolk among the present-day descendants of enslaved Africans to whom, we, as researchers of Atlantic slavery whose own racial positionalities vary, are first and foremost responsible.

While, as the foregoing analysis bears out, we follow Audre Lorde in our skepticism about the degree to which the master's tools can dismantle the master's house, we remain committed to and encouraged by the participatory possibilities of the digital as a platform for research and conversations about the histories and legacies of slavery. Our own work is published on an open-access website. And, in January 2021, we hosted a virtual conference inspired by our project in which talks from historian Jessica Marie Johnson, economist William A. Darity Jr., folklorist A. Kirsten Mullen, and community activist Pierce Freelon addressed an audience of nearly two hundred geographically disparate and professionally diverse stakeholders, and community members on how commemoration can lead to material and impactful change for the descendants of slavery. At its best, the digital can be a powerful force of collaboration, making possible new and more humane kinds of aggregations---alliances, collectives, communities, coalitions, etc.---and it is in this spirit that we share with the readers of {\em archipelagos} these reflections on our own digital experiments with representing Middle Passage mortality.

\thinrule

\page
\subsection{Tye Landels}

Tye Landels is a doctoral candidate in English at Duke University.

\subsection{Isabel Bradley}

Isabel Bradley is a doctoral candidate in Romance Studies at Duke University.

\subsection{Kelsey Desir}

Kelsey Desir is a doctoral candidate in English at Duke University.

\subsection{Grant Glass}

Grant Glass is a doctoral candidate in English at University of North Carolina, Chapel Hill.

\subsection{Jane Harwell}

Jane Harwell is a doctoral candidate in English at Duke University.

\subsection{Anya Lewis Meeks}

Anya Lewis Meeks is a doctoral candidate in English at Duke University.

\subsection{Charlotte Sussman}

Charlotte Sussman is Professor and Chair of English at Duke University.

\stopchapter
\stoptext