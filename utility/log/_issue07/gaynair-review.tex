\setvariables[article][shortauthor={Gaynair}, date={May 2022}, issue={7}, DOI={10.7916/archipelagos-07dpr1}]

\setupinteraction[title={A Review of “Rendering Revolution: Sartorial Approaches to Haitian History.”},author={Marlene Gaynair}, date={May 2022}, subtitle={A Review}, state=start, color=black, style=\tf]
\environment env_journal


\starttext


\startchapter[title={A Review of \quotation{Rendering Revolution: Sartorial Approaches to Haitian History.}}
, marking={A Review}
, bookmark={A Review of “Rendering Revolution: Sartorial Approaches to Haitian History.”}]


\startlines
{\bf
Marlene Gaynair
}
\stoplines


A 4x4 grid of images of portraits, artwork, and videos of the global Haitian experience covers the screen; a white background and black Gothic type font linger in the margins. In scrolling over each square, a caption or brief description appears and offers insight into the selected image. A click on \quotation{Go to original post,} opens a new window and reveals an Instagram post with additional pictures, digital content, and comments. This is the front page of \useURL[url1][https://renderingrevolution.ht/][][Rendering Revolution: Sartorial Approaches to Haitian History.]\from[url1]

According to its about page, the website is \quotation{a queer, bilingual, feminist experiment in digital interdisciplinary scholarship that uses the lens of fashion and material culture to trace the aesthetic, social, and political reverberations of the Haitian Revolution as a world-historical moment.} The link attached to \quotation{a world-historical moment,} takes you to an {\em aeon} article by Laurent Dubois, a historian of Haitian history at the University of Virginia, which argues that \quotation{Haiti, not the US or France, was where the assertion of human rights reached its defining climax in the Age of Revolution.} Using Dubois' argument for recentering the importance of this historic freedom struggle, Rendering Revolution \quotation{explores the many ways in which modern identities (and concepts such as human rights) were formed in relation to the legacy of slavery in the Americas.} The site authors do this through their curation and translation of the material artifacts and intellectual contributions of \quotation{occluded figures} in history, particularly women and members of the LGBTQI+ community. Significantly, on this about page and throughout the website, everything is published in both English and in Haitian Creole, with the latter presented in light blue font. Rendering Revolution thus pointedly centers speakers and readers of Haitian Creole and reminds us that a digital archive of Haitian history must be accessible to everyone of the Haitian diaspora, especially those who are or have been traditionally excluded from the archives.

Siobhan Meï and Jonathan Michael Square are co-founders of Rendering Revolution. It is clear, however, that the site is a broadly collaborative effort, as is evidenced by its practice of citation and acknowledgment. The user encounters the names of contributors like fashion designer and activist Stella Jean, drag king Mildred \quotation{Dred} Gerestant, and feminist performer-activist-scholar, Gina Athena Ulysse. Under the Network menu, there are expansive biographies of co-founders Meï and Square, in addition to that of Danielle M. Dorvil, a PhD candidate at Vanderbilt University; Nathan H. Dize, a teacher, scholar, and translator of Haitian and Francophone Caribbean literature; Nicole Willson, a Leverhulme Trust Early Career Fellow in the Institute for Black Atlantic Research at the University of Central Lancashire; Kai Toussaint Marcel, researcher for the curatorial department; and Mirline Pierre, an editor and instructor of Haitian Literature and Francophone studies. Noticeably, the biography of Dieulermesson Petit Frère, a poet, writer, and French language and literature professor, is published in Haitian Creole. This forces or, rather, encourages the reader to translate the words into English (or another language), if they are not able to read Creole, a reversal of the common experience that many non-English readers are obliged to face. The co-founders of Rendering Revolution fully embrace this kind of revolutionary scholarship and curatorial practice and they have worked to create spaces for others to do the same.

Rendering Revolution proposes five menu items: About, Contact, Exhibition, Network, and Syllabus. While the Contact and Network menus present Meï and Square's biographies and contact information, as well as those of the other major contributors to the digital archive, the Syllabus section reveals a truly fascinating side of the site's revolutionary intention. Described as \quotation{A collection of sources and stories that speak to the importance of dress, fashion, materiality, and adornment to visions of freedom before, during, and after the Haitian Revolution,} this syllabus focuses on the Haitian Revolution and its afterlives through the lens of fashion, in order to examine and analyze the \quotation{local and global legacies of the Haitian Revolution in novel and productive ways.} Under headings like \quotation{Archival Documents,} \quotation{Clothing, Decorative Objects, and Other Material References,} \quotation{Ethnographic Engravings,} \quotation{Portraits,} and \quotation{Travelers' Accounts,} hyperlinks take the user throughout the Haitian diaspora, past, present, and future. For instance, under the \quotation{Films} heading, a click on \quotation{Toussaint Louverture, Phillipe Niang (2012)} opens another window to the IMDb page of a television miniseries about the formally enslaved general, leader, and hero of the Haitian Revolution. When one clicks on \quotation{Daveed Baptiste} under the Rendering Revolution exclusive interview heading, a window opens to the site's official \useURL[url2][https://www.instagram.com/p/CN7VUUMFW-w/][][Instagram page]\from[url2], which features samples of the Haitian-American artist series, \quotation{Haiti to Hood.} Not only can the user see Baptiste's work outside of the museums and art galleries in which it has been featured, but the site's creators have also included a penetrating interview with the artist. The syllabus is taken beyond the classroom and made accessible to all who come to this digital archive. The Rendering Revolution syllabus continues the work of decolonizing and building archives and traditional academic knowledge production by proposing a course of study that presents a diverse array of cultural artifacts, languages, and modes of expression from within Haiti and its wide diaspora.

While the Exhibition is under construction, the entirety of Rendering Revolution's digital project is a masterclass in the possibilities of digital humanities. The website is intuitive and easy to use for most internet users. A quick click on the black and white circular logo will always bring you back home to the constantly upgraded grid of the site's Instagram feed. Users can also click on the recognizable Facebook and Twitter icons, to engage with the conversations taking place on those popular social media spaces. For the moment, this website features the curatorial posts on their Instagram account, also reproduced on their Facebook and Twitter accounts, and their curated collection of stories and sources in their syllabus. As written under the Exhibition section, \quotation{We are in the early stages of curating an exhibition inspired by the work that we do at Rendering Revolution. More to come here soon,} and as a visitor, one hopes they do not make us wait too long.

One of the best reasons to come back to this website again and again is that it is an ongoing digital compendium of the Haitian experience in a digital space. It was built to be a living accessible document, and one of its goals is to center the contributions of individuals, as Michel-Rolph Trouillot said, \quotation{who have unequal access to the means for such production.} It is unlike anywhere and anything else on the internet, and a necessary, welcomed intervention and contribution to the digital humanities field. As a collaborative, interactive, digital interdisciplinary project, with a focus on queer, bilingual, feminist expressions of fashion and material culture, Rendering Revolution provides an expansive look at the Haitian diasporic world, past, present, and futures.

\page
\subsection{Marlene Gaynair}

Dr.~Marlene Gaynair is currently a social and cultural historian of the modern Black Atlantic at Washington State University. She specializes in the histories of the United States, Canada, and Anglo Caribbean during the long twentieth century. Her research interests cover popular culture, identity, , citizenship, diasporas, public memory, immigration, transnational studies, and urban histories and spaces. She is also the architect of \quotation{Islands in the North,} an ongoing digital exhibit which (re) creates Black cultural and spatial identities in Toronto. She continues to engage in digital histories and humanities to explore other dimensions of historical scholarship and public engagement. She is currently working on her book manuscript, which is a transnational study of Jamaicans in Canada, the United States, and the Black Atlantic after Emancipation.~

\stopchapter
\stoptext